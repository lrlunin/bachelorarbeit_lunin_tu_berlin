\newacronym[user1=\emph{engl. „plasma x-ray source“}]{pxs}{PXS}{Laser getriebene Plasma-Röntgenquelle}
\newacronym[user1=\emph{engl. „refection zone plate“}]{rzp}{RZP}{Reflexionszonenplatte}
\chapter{Laser-getriebenes Instrument für resonante Streuung mit weichen Röntgenstrahlen}
\label{text:quelle_roentgen}
Typischerweise werden die Streuexperimente, wie es in der Einleitung bereits erwähnt wurde, mithilfe der Röntgenquellen mit der möglichst monochromatischen Strahlung, hohen räumlichen Kohärenz und dem hohen Photonenfluss durchgeführt. Die Strahllinie P04 am Synchrotron PETRA III kann als ein Vertreter solcher Röntgenquellen betrachtet. Die technischen Angaben stammen aus \cite{viefhaus_variable_2013}.
% \noindent
% Der Strahl an der Strahlinie ist stark kollimiert. So ist dessen Fläche von 10 bis \SI{50}{\micro\meter\squared} groß, wobei der Photonenfluss mehr als $\SI{1e12}{\photons\per\second}$ beträgt. Die Strahlung wird in Form der zeitlich periodischen Pulsen emittiert. Die typische Pulsdauer $t_\text{PETRA}$ ist \SI{44}{\pico\second} und Pulsperiode $T_\text{PETRA}$ kann zwischen den Werten \qtylist{8;16;192}{\nano\second} variiert werden \cite{petra-values-website}.

\noindent
Als die Quelle der weichen Röntgenstrahlung in dem weiteren Experiment dient eine \gls{pxs}, deren Aufbau und Charakteristiken in \cite{schick_laser-driven_2021} detailliert beschrieben sind. Ihrer Funktionsprinzip liegt die Emission von Wolfram im weichen Röntgenbereich zugrunde, die durch die hochenergetischen Laserpulsen getrieben wird \cite{mantouvalou_high_2015}. So wird die Röntgenstrahlung an der \gls{pxs} ebenso in Form von Pulsen emittiert. Im Vergleich zu PETRA III, wo die typische Pulsdauer $t_\text{PETRA}$ \SI{44}{\pico\second} und Pulsperiode $T_\text{PETRA}$ \qtylist{8;16;192}{\nano\second}  betragen, ist die Pulsdauer $t_\text{PXS}$ wesentlich kürzer und beträgt ca. \SI{10}{\pico\second}, wobei die Pulsperiode $T_\text{PXS}$ viel größer ist und im Bereich von \SI{10}{\milli\second} liegt.
\begin{figure}[H]
    \centering
    \input{images/xps/xps_spectrum_700_1400_mrad.pgf}
    \caption{Das Spektrum der benutzten \gls{pxs}. Mit der Abkürzung „BB“ ist die Bandbreite zu verstehen. Die schwarzen Hilfelinien mit den Titeln entsprechen den Photonenenrgien $h\nu_{\text{Fe, L3}} = \SI{706.97}{\eV}$ bzw. $h\nu_{\text{Gd, M5}} = \SI{1184,79}{\eV}$. Adaptiert von \cite{schick_laser-driven_2021}, mit Genehmigung von \href{https://orcid.org/0000-0001-7988-6489}{D. Schick}.}
    \label{fig:pxs_spectrum}
\end{figure}

\noindent
Im Gegensatz zur hoch kollimierten monochromatischen Strahlung an der Strahllinie P04, wo die zu emittierende Photonenenergie direkt technisch voreinstellbar ist, strahlt die \gls{pxs} gleichzeitig die Photonen des ganzen Spektrums (s. Abb. \ref{fig:pxs_spectrum}) in alle Richtungen aus.

\noindent
Nichtsdestotrotz ist es möglich, die Strahlung zu bündeln und einen bestimmten Photonenenergiebereich zu selektieren. Dafür wird die emittierte Strahlung mithilfe einer \gls{rzp} fokussiert. Zur Verfügung stehen nämlich drei \gls{rzp}, die für die Zielphotonenenergien von Fe (\SI{705}{\eV}), Gd (\SI{1189}{\eV}) und Dy (\SI{1292}{\eV}) konstruiert wurden. Der mit der \gls{rzp} fokussierte Strahl wird in Form einer Sanduhr auf der Detektorfläche (s. Abb. \ref{fig:butterfly_moench}) abgebildet, wobei die Zielphotonenenergie in der Sanduhrtaille liegt und die benachbarten Photonenenergien entlang der horizontalen Symmetrieachse des Strahlprofils um den Fokuspunkt energetisch aufgelöst werden. Die Lage des Strahls auf dem Detektor lässt sich sowohl horizontal, als auch vertikal mit den hochpräzisen Schrittmotoren feinjustieren.
\begin{figure}[H]
    \centering
    \subfloat[]{%% Creator: Matplotlib, PGF backend
%%
%% To include the figure in your LaTeX document, write
%%   \input{<filename>.pgf}
%%
%% Make sure the required packages are loaded in your preamble
%%   \usepackage{pgf}
%%
%% Also ensure that all the required font packages are loaded; for instance,
%% the lmodern package is sometimes necessary when using math font.
%%   \usepackage{lmodern}
%%
%% Figures using additional raster images can only be included by \input if
%% they are in the same directory as the main LaTeX file. For loading figures
%% from other directories you can use the `import` package
%%   \usepackage{import}
%%
%% and then include the figures with
%%   \import{<path to file>}{<filename>.pgf}
%%
%% Matplotlib used the following preamble
%%   \usepackage{amsmath} \usepackage[utf8]{inputenc} \usepackage[T1]{fontenc}\usepackage[output-decimal-marker={,}]{siunitx} \sisetup{per-mode=fraction, separate-uncertainty = true, locale = DE} \usepackage[acronym, toc, section=section, nonumberlist, nopostdot]{glossaries-extra}
%%
\begingroup%
\makeatletter%
\begin{pgfpicture}%
\pgfpathrectangle{\pgfpointorigin}{\pgfqpoint{3.020000in}{3.020000in}}%
\pgfusepath{use as bounding box, clip}%
\begin{pgfscope}%
\pgfsetbuttcap%
\pgfsetmiterjoin%
\pgfsetlinewidth{0.000000pt}%
\definecolor{currentstroke}{rgb}{1.000000,1.000000,1.000000}%
\pgfsetstrokecolor{currentstroke}%
\pgfsetstrokeopacity{0.000000}%
\pgfsetdash{}{0pt}%
\pgfpathmoveto{\pgfqpoint{0.000000in}{0.000000in}}%
\pgfpathlineto{\pgfqpoint{3.020000in}{0.000000in}}%
\pgfpathlineto{\pgfqpoint{3.020000in}{3.020000in}}%
\pgfpathlineto{\pgfqpoint{0.000000in}{3.020000in}}%
\pgfpathlineto{\pgfqpoint{0.000000in}{0.000000in}}%
\pgfpathclose%
\pgfusepath{}%
\end{pgfscope}%
\begin{pgfscope}%
\pgfpathrectangle{\pgfqpoint{0.000000in}{0.000000in}}{\pgfqpoint{3.020000in}{3.020000in}}%
\pgfusepath{clip}%
\pgfsys@transformshift{0.000000in}{0.000000in}%
\pgftext[left,bottom]{\includegraphics[interpolate=true,width=3.027778in,height=3.027778in]{butterfly_latex-img0.png}}%
\end{pgfscope}%
\begin{pgfscope}%
\pgfpathrectangle{\pgfqpoint{0.000000in}{0.000000in}}{\pgfqpoint{3.020000in}{3.020000in}}%
\pgfusepath{clip}%
\pgfsetbuttcap%
\pgfsetmiterjoin%
\pgfsetlinewidth{1.003750pt}%
\definecolor{currentstroke}{rgb}{1.000000,0.000000,0.000000}%
\pgfsetstrokecolor{currentstroke}%
\pgfsetdash{{3.700000pt}{1.600000pt}}{0.000000pt}%
\pgfpathmoveto{\pgfqpoint{1.264625in}{1.536425in}}%
\pgfpathlineto{\pgfqpoint{1.793125in}{1.536425in}}%
\pgfpathlineto{\pgfqpoint{1.793125in}{1.317475in}}%
\pgfpathlineto{\pgfqpoint{1.264625in}{1.317475in}}%
\pgfpathlineto{\pgfqpoint{1.264625in}{1.536425in}}%
\pgfpathclose%
\pgfusepath{stroke}%
\end{pgfscope}%
\begin{pgfscope}%
\pgfpathrectangle{\pgfqpoint{0.000000in}{0.000000in}}{\pgfqpoint{3.020000in}{3.020000in}}%
\pgfusepath{clip}%
\pgfsetbuttcap%
\pgfsetmiterjoin%
\pgfsetlinewidth{1.003750pt}%
\definecolor{currentstroke}{rgb}{1.000000,0.000000,0.000000}%
\pgfsetstrokecolor{currentstroke}%
\pgfsetdash{{3.700000pt}{1.600000pt}}{0.000000pt}%
\pgfpathmoveto{\pgfqpoint{1.438275in}{0.524725in}}%
\pgfpathlineto{\pgfqpoint{1.589275in}{0.524725in}}%
\pgfpathlineto{\pgfqpoint{1.589275in}{0.260475in}}%
\pgfpathlineto{\pgfqpoint{1.438275in}{0.260475in}}%
\pgfpathlineto{\pgfqpoint{1.438275in}{0.524725in}}%
\pgfpathclose%
\pgfusepath{stroke}%
\end{pgfscope}%
\begin{pgfscope}%
\definecolor{textcolor}{rgb}{1.000000,0.000000,0.000000}%
\pgfsetstrokecolor{textcolor}%
\pgfsetfillcolor{textcolor}%
\pgftext[x=1.868625in,y=1.536425in,left,base]{\color{textcolor}\rmfamily\fontsize{10.000000}{12.000000}\selectfont Gd M5}%
\end{pgfscope}%
\begin{pgfscope}%
\definecolor{textcolor}{rgb}{1.000000,0.000000,0.000000}%
\pgfsetstrokecolor{textcolor}%
\pgfsetfillcolor{textcolor}%
\pgftext[x=1.664775in, y=0.666511in, left, base]{\color{textcolor}\rmfamily\fontsize{10.000000}{12.000000}\selectfont RZP-Zielphotonen-}%
\end{pgfscope}%
\begin{pgfscope}%
\definecolor{textcolor}{rgb}{1.000000,0.000000,0.000000}%
\pgfsetstrokecolor{textcolor}%
\pgfsetfillcolor{textcolor}%
\pgftext[x=1.664775in, y=0.524725in, left, base]{\color{textcolor}\rmfamily\fontsize{10.000000}{12.000000}\selectfont energie von Gd}%
\end{pgfscope}%
\end{pgfpicture}%
\makeatother%
\endgroup%
\label{fig:butterfly_moench_sum}}
    \hfill
    \subfloat[]{\input{images/xps/butterfly_latex_single_shot.pgf}\label{fig:butterfly_moench_single_shot}}
    %\includegraphics[width=0.4\textwidth]{images/butterfly_sum_26026_frames_plt.png}
    \caption{Das direkte Strahprofil an dem Detektor, das mit der \gls{rzp} für Gd fokussierst wurde. Abgebildet sind (a) die Summe von 2000 Pulsen und (b) ein einzelnes Puls. Der hellste Bereich entspricht der \SI{1189}{\eV}, die Photonenenergie nimmt von unten nach oben ab. Der Bereich, wo die Photonenenergie $h\nu_{\text{Gd, M5}} = \SI{1184,79}{\eV}$ abgebildet wurde, ist mit dem rotem Viereck schematisch markiert.}
    \label{fig:butterfly_moench}
\end{figure}
\noindent
Es kann jedoch keine sinnvolle Aussage zum Photonenfluss der \gls{pxs} gemacht werden, da er nicht nur von der Wahl der \gls{rzp} abhängt, sondern auch von ihrer Position und dem  Photonenenergiebereich. Nichtsdestotrotz kann die Photonenfluss von \gls{pxs} innerhalb eines Pulses von oben abgeschätzt werden\footnote[2]{\label{footnote:einschub_in_auswertung} Das wurde im Rahmen dieser Arbeit bestimmt und ist im Abschnitt \ref{text:auswertung} detailliert präsentiert. Die Angaben sind jedoch detektorspezifisch, da es unter anderem die Transmissionscharakterstik des Detektors infrage kommt.}. Dieser liegt deutlich unter 1000 Photonen pro Puls. In Hinblick auf die Pulsperiode $T_\text{PXS}=\SI{10}{\milli\second}$ ergibt sich die obere Grenze von Photonenfluss in Höhe von ca. \SI{1e4}{\photons\per\second}, wobei der Photonenfluss an der Strahllinie P04 höher als $\SI{1e12}{\photons\per\second}$ ist.

\noindent
Dies führt dazu, dass der ausreichende Kontrast bei denselben oder vergleichbaren Belichtungszeiten des Detektors nicht zu erreichen ist. In dem Fall gibt es zwei Strategien fürs Aufnahmeverfahren: es kann entweder die extrem hohe Belichtungszeit gesetzt werden, wobei das Hintergrundrauschen des Detektors relativ klein gegen der Belichtungszeit steigt, oder eine Methode vom Trennen des Hintergrundrauschens vom dem Streusignal angewendet werden. Um die Wahl der Lösung und die Bedingungen, die dem Detektor gegebenenfalls auferlegt werden, wird es im nächsten Kapitel genauer diskutiert. AUFBAU SKIZZE
