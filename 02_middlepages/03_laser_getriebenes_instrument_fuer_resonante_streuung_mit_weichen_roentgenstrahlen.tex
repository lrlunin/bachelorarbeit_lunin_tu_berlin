\newacronym[user1=\emph{engl. plasma X-ray source}]{pxs}{PXS}{Laser-getriebene Plasma-Rönt\-gen\-quelle}
\newacronym[user1=\emph{engl. refection zone plate}]{rzp}{RZP}{Reflexionszonenplatte}
\chapter{Laser-getriebenes Instrument für resonante Streuung mit weichen Röntgenstrahlen}
\label{text:quelle_roentgen}
Typischerweise werden resonante Streuexperimente an Synchrotronstrahlungsquellen oder \gls{fel}s durchgeführt. Wegen der kleinen magnetischen Streuquerschnitte wird ein hoher Photonenfluss benötigt. Die Photonenenergie muss frei durchstimmbar sein und eine Kontrolle über die Polarisation ist vorteilhaft. Die Strahllinie P04 an der Synchrotronstrahlungsquelle PETRA III soll hier als Beispiel solcher Röntgenquellen dienen \cite{viefhaus_variable_2013}.
% \noindent
% Der Strahl an der Strahlinie ist stark kollimiert. So ist dessen Fläche von 10 bis \SI{50}{\micro\meter\squared} groß, wobei der Photonenfluss mehr als $\SI{1e12}{\photons\per\second}$ beträgt. Die Strahlung wird in Form der zeitlich periodischen Pulsen emittiert. Die typische Pulsdauer $t_\text{PETRA}$ ist \SI{44}{\pico\second} und Pulsperiode $T_\text{PETRA}$ kann zwischen den Werten \qtylist{8;16;192}{\nano\second} variiert werden \cite{petra-values-website}.
\begin{figure}[H]
    \centering
    \input{aufbau_skizze.pdf_tex}
    \caption{Schematische Skizze zur Versuchsgeometrie. Röntgenstrahlung (blauer Strahl) wird erzeugt, indem ein scharf fokussierter Laserstrahl (roter Strahl) auf einen Wolframzylinder gerichtet wird. Die erzeugte Röntgenstrahlung wird mit einer Reflexionszonenplatte gebündelt, wobei die Photonenenergien entlang der vertikalen Achse aufgelöst werden. So dient ein horizontaler Spalt zum Durchlass bestimmter Photonenenergien, die auf die Probe abgebildet werden. Der an der Probe gestreute Strahl wird mit einem Detektorsensor aufgenommen. Eine zu klein eingestellte Spalthöhe kann jedoch unerwünschte Beugungseffekte an den Spaltkanten erzeugen, welche sich am Detektorsensor beobachten lassen.}
    \label{fig:pxs_aufbau}
\end{figure}
%textwidht = 6.49733inch
\noindent
Als Quelle der weichen Röntgenstrahlung dient hingegen eine \gls{pxs}, deren Aufbau und Eigenschaften in \cite{schick_laser-driven_2021} detailliert beschrieben sind. Ihrem Funktionsprinzip liegt die Emission von Wolfram im weichen Röntgenbereich zugrunde, die durch die hochenergetischen Laserpulse getrieben wird \cite{mantouvalou_high_2015}. So wird auch die Röntgenstrahlung an der \gls{pxs} in Form von Pulsen emittiert. Im Vergleich zu PETRA III, wo die typische Pulsdauer \SI{100}{\pico\second} (FWHM), und Pulsperiode \SI{192}{\nano\second}  betragen, ist die Pulsdauer der \gls{pxs} mit \SI{10}{\pico\second} (FWHM) wesentlich kürzer, wobei ihre Pulsperiode mit \SI{10}{\milli\second} viel größer ist.
\begin{figure}[H]
    \centering
    \input{images/xps/xps_spectrum_700_1400_mrad.pgf}
    \caption{Das Spektrum der benutzten \gls{pxs}. Unter der Abkürzung „BB“ ist die Bandbreite zu verstehen. Die schwarzen Hilfslinien mit den Titeln entsprechen den Photonenenergien $h\nu_{\text{Fe, L3}} = \SI{706.97}{\eV}$ bzw. $h\nu_{\text{Gd, M5}} = \SI{1184,79}{\eV}$. Adaptiert von \cite{schick_laser-driven_2021}, mit Genehmigung von \href{https://orcid.org/0000-0001-7988-6489}{D. Schick}.}
    \label{fig:pxs_spectrum}
\end{figure}
\noindent
Im Gegensatz zur hoch kollimierten Strahlung an der Strahllinie P04 strahlt die \gls{pxs} die Photonen in einem breiten Spektrum (s. Abb. \ref{fig:pxs_spectrum}) und in alle Raum-Richtungen aus.

\noindent
Nichtsdestotrotz ist es möglich, die Strahlung zu bündeln und einen bestimmten Photonenenergiebereich zu selektieren. Dafür wird die emittierte Strahlung mithilfe einer \gls{rzp} fokussiert. Zur Verfügung stehen zwei \gls{rzp}, die für die Zielphotonenenergien von Fe (\SI{705}{\eV}) und Gd (\SI{1189}{\eV}) konstruiert wurden. Der mit der \gls{rzp} fokussierte Strahl wird in Form einer Sanduhr auf der Detektorfläche (s. Abb. \ref{fig:butterfly_moench}) abgebildet, wobei die Zielphotonenenergie in der Sanduhrtaille liegt und die benachbarten Photonenenergien entlang der vertikalen Achse energetisch aufgelöst werden.
\begin{figure}[H]
    \centering
    %% Creator: Matplotlib, PGF backend
%%
%% To include the figure in your LaTeX document, write
%%   \input{<filename>.pgf}
%%
%% Make sure the required packages are loaded in your preamble
%%   \usepackage{pgf}
%%
%% Also ensure that all the required font packages are loaded; for instance,
%% the lmodern package is sometimes necessary when using math font.
%%   \usepackage{lmodern}
%%
%% Figures using additional raster images can only be included by \input if
%% they are in the same directory as the main LaTeX file. For loading figures
%% from other directories you can use the `import` package
%%   \usepackage{import}
%%
%% and then include the figures with
%%   \import{<path to file>}{<filename>.pgf}
%%
%% Matplotlib used the following preamble
%%   \usepackage{amsmath} \usepackage[utf8]{inputenc} \usepackage[T1]{fontenc} \usepackage[output-decimal-marker={,},print-unity-mantissa=false]{siunitx} \sisetup{per-mode=fraction, separate-uncertainty = true, locale = DE} \usepackage[acronym, toc, section=section, nonumberlist, nopostdot]{glossaries-extra} \DeclareSIUnit\adu{\text{ADU}} \DeclareSIUnit\px{\text{px}} \DeclareSIUnit\photons{\text{Pho\-to\-nen}} \DeclareSIUnit\photon{\text{Pho\-ton}}
%%
\begingroup%
\makeatletter%
\begin{pgfpicture}%
\pgfpathrectangle{\pgfpointorigin}{\pgfqpoint{6.124092in}{3.809020in}}%
\pgfusepath{use as bounding box, clip}%
\begin{pgfscope}%
\pgfsetbuttcap%
\pgfsetmiterjoin%
\pgfsetlinewidth{0.000000pt}%
\definecolor{currentstroke}{rgb}{1.000000,1.000000,1.000000}%
\pgfsetstrokecolor{currentstroke}%
\pgfsetstrokeopacity{0.000000}%
\pgfsetdash{}{0pt}%
\pgfpathmoveto{\pgfqpoint{0.000000in}{0.000000in}}%
\pgfpathlineto{\pgfqpoint{6.124092in}{0.000000in}}%
\pgfpathlineto{\pgfqpoint{6.124092in}{3.809020in}}%
\pgfpathlineto{\pgfqpoint{0.000000in}{3.809020in}}%
\pgfpathlineto{\pgfqpoint{0.000000in}{0.000000in}}%
\pgfpathclose%
\pgfusepath{}%
\end{pgfscope}%
\begin{pgfscope}%
\pgfsetbuttcap%
\pgfsetmiterjoin%
\definecolor{currentfill}{rgb}{1.000000,1.000000,1.000000}%
\pgfsetfillcolor{currentfill}%
\pgfsetlinewidth{0.000000pt}%
\definecolor{currentstroke}{rgb}{0.000000,0.000000,0.000000}%
\pgfsetstrokecolor{currentstroke}%
\pgfsetstrokeopacity{0.000000}%
\pgfsetdash{}{0pt}%
\pgfpathmoveto{\pgfqpoint{0.277338in}{0.654164in}}%
\pgfpathlineto{\pgfqpoint{3.050715in}{0.654164in}}%
\pgfpathlineto{\pgfqpoint{3.050715in}{3.427541in}}%
\pgfpathlineto{\pgfqpoint{0.277338in}{3.427541in}}%
\pgfpathlineto{\pgfqpoint{0.277338in}{0.654164in}}%
\pgfpathclose%
\pgfusepath{fill}%
\end{pgfscope}%
\begin{pgfscope}%
\pgfsys@transformshift{0.277778in}{0.656242in}%
\pgftext[left,bottom]{\includegraphics[interpolate=true,width=2.777778in,height=2.777778in]{butterfly_two_shots-img0.png}}%
\end{pgfscope}%
\begin{pgfscope}%
\pgfpathrectangle{\pgfqpoint{0.277338in}{0.654164in}}{\pgfqpoint{2.773377in}{2.773377in}}%
\pgfusepath{clip}%
\pgfsetbuttcap%
\pgfsetmiterjoin%
\pgfsetlinewidth{1.003750pt}%
\definecolor{currentstroke}{rgb}{1.000000,0.000000,0.000000}%
\pgfsetstrokecolor{currentstroke}%
\pgfsetdash{{3.700000pt}{1.600000pt}}{0.000000pt}%
\pgfpathmoveto{\pgfqpoint{1.438689in}{2.065119in}}%
\pgfpathlineto{\pgfqpoint{1.924030in}{2.065119in}}%
\pgfpathlineto{\pgfqpoint{1.924030in}{1.864050in}}%
\pgfpathlineto{\pgfqpoint{1.438689in}{1.864050in}}%
\pgfpathlineto{\pgfqpoint{1.438689in}{2.065119in}}%
\pgfpathclose%
\pgfusepath{stroke}%
\end{pgfscope}%
\begin{pgfscope}%
\pgfpathrectangle{\pgfqpoint{0.277338in}{0.654164in}}{\pgfqpoint{2.773377in}{2.773377in}}%
\pgfusepath{clip}%
\pgfsetbuttcap%
\pgfsetmiterjoin%
\pgfsetlinewidth{1.003750pt}%
\definecolor{currentstroke}{rgb}{0.000000,1.000000,0.000000}%
\pgfsetstrokecolor{currentstroke}%
\pgfsetdash{{3.700000pt}{1.600000pt}}{0.000000pt}%
\pgfpathmoveto{\pgfqpoint{1.598159in}{1.136038in}}%
\pgfpathlineto{\pgfqpoint{1.736827in}{1.136038in}}%
\pgfpathlineto{\pgfqpoint{1.736827in}{0.893368in}}%
\pgfpathlineto{\pgfqpoint{1.598159in}{0.893368in}}%
\pgfpathlineto{\pgfqpoint{1.598159in}{1.136038in}}%
\pgfpathclose%
\pgfusepath{stroke}%
\end{pgfscope}%
\begin{pgfscope}%
\pgfsetrectcap%
\pgfsetmiterjoin%
\pgfsetlinewidth{0.803000pt}%
\definecolor{currentstroke}{rgb}{0.000000,0.000000,0.000000}%
\pgfsetstrokecolor{currentstroke}%
\pgfsetdash{}{0pt}%
\pgfpathmoveto{\pgfqpoint{0.277338in}{0.654164in}}%
\pgfpathlineto{\pgfqpoint{0.277338in}{3.427541in}}%
\pgfusepath{stroke}%
\end{pgfscope}%
\begin{pgfscope}%
\pgfsetrectcap%
\pgfsetmiterjoin%
\pgfsetlinewidth{0.803000pt}%
\definecolor{currentstroke}{rgb}{0.000000,0.000000,0.000000}%
\pgfsetstrokecolor{currentstroke}%
\pgfsetdash{}{0pt}%
\pgfpathmoveto{\pgfqpoint{3.050715in}{0.654164in}}%
\pgfpathlineto{\pgfqpoint{3.050715in}{3.427541in}}%
\pgfusepath{stroke}%
\end{pgfscope}%
\begin{pgfscope}%
\pgfsetrectcap%
\pgfsetmiterjoin%
\pgfsetlinewidth{0.803000pt}%
\definecolor{currentstroke}{rgb}{0.000000,0.000000,0.000000}%
\pgfsetstrokecolor{currentstroke}%
\pgfsetdash{}{0pt}%
\pgfpathmoveto{\pgfqpoint{0.277338in}{0.654164in}}%
\pgfpathlineto{\pgfqpoint{3.050715in}{0.654164in}}%
\pgfusepath{stroke}%
\end{pgfscope}%
\begin{pgfscope}%
\pgfsetrectcap%
\pgfsetmiterjoin%
\pgfsetlinewidth{0.803000pt}%
\definecolor{currentstroke}{rgb}{0.000000,0.000000,0.000000}%
\pgfsetstrokecolor{currentstroke}%
\pgfsetdash{}{0pt}%
\pgfpathmoveto{\pgfqpoint{0.277338in}{3.427541in}}%
\pgfpathlineto{\pgfqpoint{3.050715in}{3.427541in}}%
\pgfusepath{stroke}%
\end{pgfscope}%
\begin{pgfscope}%
\definecolor{textcolor}{rgb}{0.000000,0.000000,0.000000}%
\pgfsetstrokecolor{textcolor}%
\pgfsetfillcolor{textcolor}%
\pgftext[x=0.000000in,y=3.704879in,left,base]{\color{textcolor}\rmfamily\fontsize{10.000000}{12.000000}\selectfont (a)}%
\end{pgfscope}%
\begin{pgfscope}%
\definecolor{textcolor}{rgb}{1.000000,0.000000,0.000000}%
\pgfsetstrokecolor{textcolor}%
\pgfsetfillcolor{textcolor}%
\pgftext[x=1.993365in,y=2.065119in,left,base]{\color{textcolor}\rmfamily\fontsize{10.000000}{12.000000}\selectfont Gd M5}%
\end{pgfscope}%
\begin{pgfscope}%
\definecolor{textcolor}{rgb}{0.000000,1.000000,0.000000}%
\pgfsetstrokecolor{textcolor}%
\pgfsetfillcolor{textcolor}%
\pgftext[x=1.806162in, y=1.277824in, left, base]{\color{textcolor}\rmfamily\fontsize{10.000000}{12.000000}\selectfont RZP-Zielphotonen-}%
\end{pgfscope}%
\begin{pgfscope}%
\definecolor{textcolor}{rgb}{0.000000,1.000000,0.000000}%
\pgfsetstrokecolor{textcolor}%
\pgfsetfillcolor{textcolor}%
\pgftext[x=1.806162in, y=1.136038in, left, base]{\color{textcolor}\rmfamily\fontsize{10.000000}{12.000000}\selectfont energie von Gd}%
\end{pgfscope}%
\begin{pgfscope}%
\pgfsetbuttcap%
\pgfsetmiterjoin%
\definecolor{currentfill}{rgb}{1.000000,1.000000,1.000000}%
\pgfsetfillcolor{currentfill}%
\pgfsetlinewidth{0.000000pt}%
\definecolor{currentstroke}{rgb}{0.000000,0.000000,0.000000}%
\pgfsetstrokecolor{currentstroke}%
\pgfsetstrokeopacity{0.000000}%
\pgfsetdash{}{0pt}%
\pgfpathmoveto{\pgfqpoint{3.350715in}{0.654164in}}%
\pgfpathlineto{\pgfqpoint{6.124092in}{0.654164in}}%
\pgfpathlineto{\pgfqpoint{6.124092in}{3.427541in}}%
\pgfpathlineto{\pgfqpoint{3.350715in}{3.427541in}}%
\pgfpathlineto{\pgfqpoint{3.350715in}{0.654164in}}%
\pgfpathclose%
\pgfusepath{fill}%
\end{pgfscope}%
\begin{pgfscope}%
\pgfsys@transformshift{3.347222in}{0.656242in}%
\pgftext[left,bottom]{\includegraphics[interpolate=true,width=2.763889in,height=2.777778in]{butterfly_two_shots-img1.png}}%
\end{pgfscope}%
\begin{pgfscope}%
\pgfpathrectangle{\pgfqpoint{3.350715in}{0.654164in}}{\pgfqpoint{2.773377in}{2.773377in}}%
\pgfusepath{clip}%
\pgfsetbuttcap%
\pgfsetmiterjoin%
\pgfsetlinewidth{1.003750pt}%
\definecolor{currentstroke}{rgb}{1.000000,0.000000,0.000000}%
\pgfsetstrokecolor{currentstroke}%
\pgfsetdash{{3.700000pt}{1.600000pt}}{0.000000pt}%
\pgfpathmoveto{\pgfqpoint{4.512067in}{2.065119in}}%
\pgfpathlineto{\pgfqpoint{4.997408in}{2.065119in}}%
\pgfpathlineto{\pgfqpoint{4.997408in}{1.864050in}}%
\pgfpathlineto{\pgfqpoint{4.512067in}{1.864050in}}%
\pgfpathlineto{\pgfqpoint{4.512067in}{2.065119in}}%
\pgfpathclose%
\pgfusepath{stroke}%
\end{pgfscope}%
\begin{pgfscope}%
\pgfpathrectangle{\pgfqpoint{3.350715in}{0.654164in}}{\pgfqpoint{2.773377in}{2.773377in}}%
\pgfusepath{clip}%
\pgfsetbuttcap%
\pgfsetmiterjoin%
\pgfsetlinewidth{1.003750pt}%
\definecolor{currentstroke}{rgb}{0.000000,1.000000,0.000000}%
\pgfsetstrokecolor{currentstroke}%
\pgfsetdash{{3.700000pt}{1.600000pt}}{0.000000pt}%
\pgfpathmoveto{\pgfqpoint{4.671536in}{1.136038in}}%
\pgfpathlineto{\pgfqpoint{4.810205in}{1.136038in}}%
\pgfpathlineto{\pgfqpoint{4.810205in}{0.893368in}}%
\pgfpathlineto{\pgfqpoint{4.671536in}{0.893368in}}%
\pgfpathlineto{\pgfqpoint{4.671536in}{1.136038in}}%
\pgfpathclose%
\pgfusepath{stroke}%
\end{pgfscope}%
\begin{pgfscope}%
\pgfsetrectcap%
\pgfsetmiterjoin%
\pgfsetlinewidth{0.803000pt}%
\definecolor{currentstroke}{rgb}{0.000000,0.000000,0.000000}%
\pgfsetstrokecolor{currentstroke}%
\pgfsetdash{}{0pt}%
\pgfpathmoveto{\pgfqpoint{3.350715in}{0.654164in}}%
\pgfpathlineto{\pgfqpoint{3.350715in}{3.427541in}}%
\pgfusepath{stroke}%
\end{pgfscope}%
\begin{pgfscope}%
\pgfsetrectcap%
\pgfsetmiterjoin%
\pgfsetlinewidth{0.803000pt}%
\definecolor{currentstroke}{rgb}{0.000000,0.000000,0.000000}%
\pgfsetstrokecolor{currentstroke}%
\pgfsetdash{}{0pt}%
\pgfpathmoveto{\pgfqpoint{6.124092in}{0.654164in}}%
\pgfpathlineto{\pgfqpoint{6.124092in}{3.427541in}}%
\pgfusepath{stroke}%
\end{pgfscope}%
\begin{pgfscope}%
\pgfsetrectcap%
\pgfsetmiterjoin%
\pgfsetlinewidth{0.803000pt}%
\definecolor{currentstroke}{rgb}{0.000000,0.000000,0.000000}%
\pgfsetstrokecolor{currentstroke}%
\pgfsetdash{}{0pt}%
\pgfpathmoveto{\pgfqpoint{3.350715in}{0.654164in}}%
\pgfpathlineto{\pgfqpoint{6.124092in}{0.654164in}}%
\pgfusepath{stroke}%
\end{pgfscope}%
\begin{pgfscope}%
\pgfsetrectcap%
\pgfsetmiterjoin%
\pgfsetlinewidth{0.803000pt}%
\definecolor{currentstroke}{rgb}{0.000000,0.000000,0.000000}%
\pgfsetstrokecolor{currentstroke}%
\pgfsetdash{}{0pt}%
\pgfpathmoveto{\pgfqpoint{3.350715in}{3.427541in}}%
\pgfpathlineto{\pgfqpoint{6.124092in}{3.427541in}}%
\pgfusepath{stroke}%
\end{pgfscope}%
\begin{pgfscope}%
\definecolor{textcolor}{rgb}{0.000000,0.000000,0.000000}%
\pgfsetstrokecolor{textcolor}%
\pgfsetfillcolor{textcolor}%
\pgftext[x=3.073377in,y=3.704879in,left,base]{\color{textcolor}\rmfamily\fontsize{10.000000}{12.000000}\selectfont (b)}%
\end{pgfscope}%
\begin{pgfscope}%
\definecolor{textcolor}{rgb}{1.000000,0.000000,0.000000}%
\pgfsetstrokecolor{textcolor}%
\pgfsetfillcolor{textcolor}%
\pgftext[x=5.066742in,y=2.065119in,left,base]{\color{textcolor}\rmfamily\fontsize{10.000000}{12.000000}\selectfont Gd M5}%
\end{pgfscope}%
\begin{pgfscope}%
\definecolor{textcolor}{rgb}{0.000000,1.000000,0.000000}%
\pgfsetstrokecolor{textcolor}%
\pgfsetfillcolor{textcolor}%
\pgftext[x=4.879539in, y=1.277824in, left, base]{\color{textcolor}\rmfamily\fontsize{10.000000}{12.000000}\selectfont RZP-Zielphotonen-}%
\end{pgfscope}%
\begin{pgfscope}%
\definecolor{textcolor}{rgb}{0.000000,1.000000,0.000000}%
\pgfsetstrokecolor{textcolor}%
\pgfsetfillcolor{textcolor}%
\pgftext[x=4.879539in, y=1.136038in, left, base]{\color{textcolor}\rmfamily\fontsize{10.000000}{12.000000}\selectfont energie von Gd}%
\end{pgfscope}%
\begin{pgfscope}%
\pgfsetbuttcap%
\pgfsetmiterjoin%
\definecolor{currentfill}{rgb}{1.000000,1.000000,1.000000}%
\pgfsetfillcolor{currentfill}%
\pgfsetlinewidth{0.000000pt}%
\definecolor{currentstroke}{rgb}{0.000000,0.000000,0.000000}%
\pgfsetstrokecolor{currentstroke}%
\pgfsetstrokeopacity{0.000000}%
\pgfsetdash{}{0pt}%
\pgfpathmoveto{\pgfqpoint{0.277338in}{0.415495in}}%
\pgfpathlineto{\pgfqpoint{3.050715in}{0.415495in}}%
\pgfpathlineto{\pgfqpoint{3.050715in}{0.554164in}}%
\pgfpathlineto{\pgfqpoint{0.277338in}{0.554164in}}%
\pgfpathlineto{\pgfqpoint{0.277338in}{0.415495in}}%
\pgfpathclose%
\pgfusepath{fill}%
\end{pgfscope}%
\begin{pgfscope}%
\pgfpathrectangle{\pgfqpoint{0.277338in}{0.415495in}}{\pgfqpoint{2.773377in}{0.138669in}}%
\pgfusepath{clip}%
\pgfsetbuttcap%
\pgfsetmiterjoin%
\definecolor{currentfill}{rgb}{1.000000,1.000000,1.000000}%
\pgfsetfillcolor{currentfill}%
\pgfsetlinewidth{0.010037pt}%
\definecolor{currentstroke}{rgb}{1.000000,1.000000,1.000000}%
\pgfsetstrokecolor{currentstroke}%
\pgfsetdash{}{0pt}%
\pgfusepath{stroke,fill}%
\end{pgfscope}%
\begin{pgfscope}%
\pgfsys@transformshift{0.277778in}{0.420131in}%
\pgftext[left,bottom]{\includegraphics[interpolate=true,width=2.777778in,height=0.138889in]{butterfly_two_shots-img2.png}}%
\end{pgfscope}%
\begin{pgfscope}%
\pgfsetbuttcap%
\pgfsetroundjoin%
\definecolor{currentfill}{rgb}{0.000000,0.000000,0.000000}%
\pgfsetfillcolor{currentfill}%
\pgfsetlinewidth{0.803000pt}%
\definecolor{currentstroke}{rgb}{0.000000,0.000000,0.000000}%
\pgfsetstrokecolor{currentstroke}%
\pgfsetdash{}{0pt}%
\pgfsys@defobject{currentmarker}{\pgfqpoint{0.000000in}{-0.048611in}}{\pgfqpoint{0.000000in}{0.000000in}}{%
\pgfpathmoveto{\pgfqpoint{0.000000in}{0.000000in}}%
\pgfpathlineto{\pgfqpoint{0.000000in}{-0.048611in}}%
\pgfusepath{stroke,fill}%
}%
\begin{pgfscope}%
\pgfsys@transformshift{0.522117in}{0.415495in}%
\pgfsys@useobject{currentmarker}{}%
\end{pgfscope}%
\end{pgfscope}%
\begin{pgfscope}%
\definecolor{textcolor}{rgb}{0.000000,0.000000,0.000000}%
\pgfsetstrokecolor{textcolor}%
\pgfsetfillcolor{textcolor}%
\pgftext[x=0.522117in,y=0.318273in,,top]{\color{textcolor}\rmfamily\fontsize{10.000000}{12.000000}\selectfont 0}%
\end{pgfscope}%
\begin{pgfscope}%
\pgfsetbuttcap%
\pgfsetroundjoin%
\definecolor{currentfill}{rgb}{0.000000,0.000000,0.000000}%
\pgfsetfillcolor{currentfill}%
\pgfsetlinewidth{0.803000pt}%
\definecolor{currentstroke}{rgb}{0.000000,0.000000,0.000000}%
\pgfsetstrokecolor{currentstroke}%
\pgfsetdash{}{0pt}%
\pgfsys@defobject{currentmarker}{\pgfqpoint{0.000000in}{-0.048611in}}{\pgfqpoint{0.000000in}{0.000000in}}{%
\pgfpathmoveto{\pgfqpoint{0.000000in}{0.000000in}}%
\pgfpathlineto{\pgfqpoint{0.000000in}{-0.048611in}}%
\pgfusepath{stroke,fill}%
}%
\begin{pgfscope}%
\pgfsys@transformshift{1.281356in}{0.415495in}%
\pgfsys@useobject{currentmarker}{}%
\end{pgfscope}%
\end{pgfscope}%
\begin{pgfscope}%
\definecolor{textcolor}{rgb}{0.000000,0.000000,0.000000}%
\pgfsetstrokecolor{textcolor}%
\pgfsetfillcolor{textcolor}%
\pgftext[x=1.281356in,y=0.318273in,,top]{\color{textcolor}\rmfamily\fontsize{10.000000}{12.000000}\selectfont \num{3.7e+06}}%
\end{pgfscope}%
\begin{pgfscope}%
\pgfsetbuttcap%
\pgfsetroundjoin%
\definecolor{currentfill}{rgb}{0.000000,0.000000,0.000000}%
\pgfsetfillcolor{currentfill}%
\pgfsetlinewidth{0.803000pt}%
\definecolor{currentstroke}{rgb}{0.000000,0.000000,0.000000}%
\pgfsetstrokecolor{currentstroke}%
\pgfsetdash{}{0pt}%
\pgfsys@defobject{currentmarker}{\pgfqpoint{0.000000in}{-0.048611in}}{\pgfqpoint{0.000000in}{0.000000in}}{%
\pgfpathmoveto{\pgfqpoint{0.000000in}{0.000000in}}%
\pgfpathlineto{\pgfqpoint{0.000000in}{-0.048611in}}%
\pgfusepath{stroke,fill}%
}%
\begin{pgfscope}%
\pgfsys@transformshift{2.020075in}{0.415495in}%
\pgfsys@useobject{currentmarker}{}%
\end{pgfscope}%
\end{pgfscope}%
\begin{pgfscope}%
\definecolor{textcolor}{rgb}{0.000000,0.000000,0.000000}%
\pgfsetstrokecolor{textcolor}%
\pgfsetfillcolor{textcolor}%
\pgftext[x=2.020075in,y=0.318273in,,top]{\color{textcolor}\rmfamily\fontsize{10.000000}{12.000000}\selectfont \num{7.3e+06}}%
\end{pgfscope}%
\begin{pgfscope}%
\pgfsetbuttcap%
\pgfsetroundjoin%
\definecolor{currentfill}{rgb}{0.000000,0.000000,0.000000}%
\pgfsetfillcolor{currentfill}%
\pgfsetlinewidth{0.803000pt}%
\definecolor{currentstroke}{rgb}{0.000000,0.000000,0.000000}%
\pgfsetstrokecolor{currentstroke}%
\pgfsetdash{}{0pt}%
\pgfsys@defobject{currentmarker}{\pgfqpoint{0.000000in}{-0.048611in}}{\pgfqpoint{0.000000in}{0.000000in}}{%
\pgfpathmoveto{\pgfqpoint{0.000000in}{0.000000in}}%
\pgfpathlineto{\pgfqpoint{0.000000in}{-0.048611in}}%
\pgfusepath{stroke,fill}%
}%
\begin{pgfscope}%
\pgfsys@transformshift{2.779314in}{0.415495in}%
\pgfsys@useobject{currentmarker}{}%
\end{pgfscope}%
\end{pgfscope}%
\begin{pgfscope}%
\definecolor{textcolor}{rgb}{0.000000,0.000000,0.000000}%
\pgfsetstrokecolor{textcolor}%
\pgfsetfillcolor{textcolor}%
\pgftext[x=2.779314in,y=0.318273in,,top]{\color{textcolor}\rmfamily\fontsize{10.000000}{12.000000}\selectfont \num{1.1e+07}}%
\end{pgfscope}%
\begin{pgfscope}%
\definecolor{textcolor}{rgb}{0.000000,0.000000,0.000000}%
\pgfsetstrokecolor{textcolor}%
\pgfsetfillcolor{textcolor}%
\pgftext[x=1.664026in,y=0.122655in,,top]{\color{textcolor}\rmfamily\fontsize{10.000000}{12.000000}\selectfont Intensität in ADU}%
\end{pgfscope}%
\begin{pgfscope}%
\pgfsetrectcap%
\pgfsetmiterjoin%
\pgfsetlinewidth{0.803000pt}%
\definecolor{currentstroke}{rgb}{0.000000,0.000000,0.000000}%
\pgfsetstrokecolor{currentstroke}%
\pgfsetdash{}{0pt}%
\pgfpathmoveto{\pgfqpoint{0.277338in}{0.415495in}}%
\pgfpathlineto{\pgfqpoint{0.277338in}{0.484829in}}%
\pgfpathlineto{\pgfqpoint{0.277338in}{0.554164in}}%
\pgfpathlineto{\pgfqpoint{3.050715in}{0.554164in}}%
\pgfpathlineto{\pgfqpoint{3.050715in}{0.484829in}}%
\pgfpathlineto{\pgfqpoint{3.050715in}{0.415495in}}%
\pgfpathlineto{\pgfqpoint{0.277338in}{0.415495in}}%
\pgfpathclose%
\pgfusepath{stroke}%
\end{pgfscope}%
\begin{pgfscope}%
\pgfsetbuttcap%
\pgfsetmiterjoin%
\definecolor{currentfill}{rgb}{1.000000,1.000000,1.000000}%
\pgfsetfillcolor{currentfill}%
\pgfsetlinewidth{0.000000pt}%
\definecolor{currentstroke}{rgb}{0.000000,0.000000,0.000000}%
\pgfsetstrokecolor{currentstroke}%
\pgfsetstrokeopacity{0.000000}%
\pgfsetdash{}{0pt}%
\pgfpathmoveto{\pgfqpoint{3.350715in}{0.415495in}}%
\pgfpathlineto{\pgfqpoint{6.124092in}{0.415495in}}%
\pgfpathlineto{\pgfqpoint{6.124092in}{0.554164in}}%
\pgfpathlineto{\pgfqpoint{3.350715in}{0.554164in}}%
\pgfpathlineto{\pgfqpoint{3.350715in}{0.415495in}}%
\pgfpathclose%
\pgfusepath{fill}%
\end{pgfscope}%
\begin{pgfscope}%
\pgfpathrectangle{\pgfqpoint{3.350715in}{0.415495in}}{\pgfqpoint{2.773377in}{0.138669in}}%
\pgfusepath{clip}%
\pgfsetbuttcap%
\pgfsetmiterjoin%
\definecolor{currentfill}{rgb}{1.000000,1.000000,1.000000}%
\pgfsetfillcolor{currentfill}%
\pgfsetlinewidth{0.010037pt}%
\definecolor{currentstroke}{rgb}{1.000000,1.000000,1.000000}%
\pgfsetstrokecolor{currentstroke}%
\pgfsetdash{}{0pt}%
\pgfusepath{stroke,fill}%
\end{pgfscope}%
\begin{pgfscope}%
\pgfsys@transformshift{3.347222in}{0.420131in}%
\pgftext[left,bottom]{\includegraphics[interpolate=true,width=2.763889in,height=0.138889in]{butterfly_two_shots-img3.png}}%
\end{pgfscope}%
\begin{pgfscope}%
\pgfsetbuttcap%
\pgfsetroundjoin%
\definecolor{currentfill}{rgb}{0.000000,0.000000,0.000000}%
\pgfsetfillcolor{currentfill}%
\pgfsetlinewidth{0.803000pt}%
\definecolor{currentstroke}{rgb}{0.000000,0.000000,0.000000}%
\pgfsetstrokecolor{currentstroke}%
\pgfsetdash{}{0pt}%
\pgfsys@defobject{currentmarker}{\pgfqpoint{0.000000in}{-0.048611in}}{\pgfqpoint{0.000000in}{0.000000in}}{%
\pgfpathmoveto{\pgfqpoint{0.000000in}{0.000000in}}%
\pgfpathlineto{\pgfqpoint{0.000000in}{-0.048611in}}%
\pgfusepath{stroke,fill}%
}%
\begin{pgfscope}%
\pgfsys@transformshift{3.581721in}{0.415495in}%
\pgfsys@useobject{currentmarker}{}%
\end{pgfscope}%
\end{pgfscope}%
\begin{pgfscope}%
\definecolor{textcolor}{rgb}{0.000000,0.000000,0.000000}%
\pgfsetstrokecolor{textcolor}%
\pgfsetfillcolor{textcolor}%
\pgftext[x=3.581721in,y=0.318273in,,top]{\color{textcolor}\rmfamily\fontsize{10.000000}{12.000000}\selectfont \num{0}}%
\end{pgfscope}%
\begin{pgfscope}%
\pgfsetbuttcap%
\pgfsetroundjoin%
\definecolor{currentfill}{rgb}{0.000000,0.000000,0.000000}%
\pgfsetfillcolor{currentfill}%
\pgfsetlinewidth{0.803000pt}%
\definecolor{currentstroke}{rgb}{0.000000,0.000000,0.000000}%
\pgfsetstrokecolor{currentstroke}%
\pgfsetdash{}{0pt}%
\pgfsys@defobject{currentmarker}{\pgfqpoint{0.000000in}{-0.048611in}}{\pgfqpoint{0.000000in}{0.000000in}}{%
\pgfpathmoveto{\pgfqpoint{0.000000in}{0.000000in}}%
\pgfpathlineto{\pgfqpoint{0.000000in}{-0.048611in}}%
\pgfusepath{stroke,fill}%
}%
\begin{pgfscope}%
\pgfsys@transformshift{4.051670in}{0.415495in}%
\pgfsys@useobject{currentmarker}{}%
\end{pgfscope}%
\end{pgfscope}%
\begin{pgfscope}%
\definecolor{textcolor}{rgb}{0.000000,0.000000,0.000000}%
\pgfsetstrokecolor{textcolor}%
\pgfsetfillcolor{textcolor}%
\pgftext[x=4.051670in,y=0.318273in,,top]{\color{textcolor}\rmfamily\fontsize{10.000000}{12.000000}\selectfont \num{250}}%
\end{pgfscope}%
\begin{pgfscope}%
\pgfsetbuttcap%
\pgfsetroundjoin%
\definecolor{currentfill}{rgb}{0.000000,0.000000,0.000000}%
\pgfsetfillcolor{currentfill}%
\pgfsetlinewidth{0.803000pt}%
\definecolor{currentstroke}{rgb}{0.000000,0.000000,0.000000}%
\pgfsetstrokecolor{currentstroke}%
\pgfsetdash{}{0pt}%
\pgfsys@defobject{currentmarker}{\pgfqpoint{0.000000in}{-0.048611in}}{\pgfqpoint{0.000000in}{0.000000in}}{%
\pgfpathmoveto{\pgfqpoint{0.000000in}{0.000000in}}%
\pgfpathlineto{\pgfqpoint{0.000000in}{-0.048611in}}%
\pgfusepath{stroke,fill}%
}%
\begin{pgfscope}%
\pgfsys@transformshift{4.521619in}{0.415495in}%
\pgfsys@useobject{currentmarker}{}%
\end{pgfscope}%
\end{pgfscope}%
\begin{pgfscope}%
\definecolor{textcolor}{rgb}{0.000000,0.000000,0.000000}%
\pgfsetstrokecolor{textcolor}%
\pgfsetfillcolor{textcolor}%
\pgftext[x=4.521619in,y=0.318273in,,top]{\color{textcolor}\rmfamily\fontsize{10.000000}{12.000000}\selectfont \num{500}}%
\end{pgfscope}%
\begin{pgfscope}%
\pgfsetbuttcap%
\pgfsetroundjoin%
\definecolor{currentfill}{rgb}{0.000000,0.000000,0.000000}%
\pgfsetfillcolor{currentfill}%
\pgfsetlinewidth{0.803000pt}%
\definecolor{currentstroke}{rgb}{0.000000,0.000000,0.000000}%
\pgfsetstrokecolor{currentstroke}%
\pgfsetdash{}{0pt}%
\pgfsys@defobject{currentmarker}{\pgfqpoint{0.000000in}{-0.048611in}}{\pgfqpoint{0.000000in}{0.000000in}}{%
\pgfpathmoveto{\pgfqpoint{0.000000in}{0.000000in}}%
\pgfpathlineto{\pgfqpoint{0.000000in}{-0.048611in}}%
\pgfusepath{stroke,fill}%
}%
\begin{pgfscope}%
\pgfsys@transformshift{4.991568in}{0.415495in}%
\pgfsys@useobject{currentmarker}{}%
\end{pgfscope}%
\end{pgfscope}%
\begin{pgfscope}%
\definecolor{textcolor}{rgb}{0.000000,0.000000,0.000000}%
\pgfsetstrokecolor{textcolor}%
\pgfsetfillcolor{textcolor}%
\pgftext[x=4.991568in,y=0.318273in,,top]{\color{textcolor}\rmfamily\fontsize{10.000000}{12.000000}\selectfont \num{750}}%
\end{pgfscope}%
\begin{pgfscope}%
\pgfsetbuttcap%
\pgfsetroundjoin%
\definecolor{currentfill}{rgb}{0.000000,0.000000,0.000000}%
\pgfsetfillcolor{currentfill}%
\pgfsetlinewidth{0.803000pt}%
\definecolor{currentstroke}{rgb}{0.000000,0.000000,0.000000}%
\pgfsetstrokecolor{currentstroke}%
\pgfsetdash{}{0pt}%
\pgfsys@defobject{currentmarker}{\pgfqpoint{0.000000in}{-0.048611in}}{\pgfqpoint{0.000000in}{0.000000in}}{%
\pgfpathmoveto{\pgfqpoint{0.000000in}{0.000000in}}%
\pgfpathlineto{\pgfqpoint{0.000000in}{-0.048611in}}%
\pgfusepath{stroke,fill}%
}%
\begin{pgfscope}%
\pgfsys@transformshift{5.461517in}{0.415495in}%
\pgfsys@useobject{currentmarker}{}%
\end{pgfscope}%
\end{pgfscope}%
\begin{pgfscope}%
\definecolor{textcolor}{rgb}{0.000000,0.000000,0.000000}%
\pgfsetstrokecolor{textcolor}%
\pgfsetfillcolor{textcolor}%
\pgftext[x=5.461517in,y=0.318273in,,top]{\color{textcolor}\rmfamily\fontsize{10.000000}{12.000000}\selectfont \num{1000}}%
\end{pgfscope}%
\begin{pgfscope}%
\pgfsetbuttcap%
\pgfsetroundjoin%
\definecolor{currentfill}{rgb}{0.000000,0.000000,0.000000}%
\pgfsetfillcolor{currentfill}%
\pgfsetlinewidth{0.803000pt}%
\definecolor{currentstroke}{rgb}{0.000000,0.000000,0.000000}%
\pgfsetstrokecolor{currentstroke}%
\pgfsetdash{}{0pt}%
\pgfsys@defobject{currentmarker}{\pgfqpoint{0.000000in}{-0.048611in}}{\pgfqpoint{0.000000in}{0.000000in}}{%
\pgfpathmoveto{\pgfqpoint{0.000000in}{0.000000in}}%
\pgfpathlineto{\pgfqpoint{0.000000in}{-0.048611in}}%
\pgfusepath{stroke,fill}%
}%
\begin{pgfscope}%
\pgfsys@transformshift{5.931466in}{0.415495in}%
\pgfsys@useobject{currentmarker}{}%
\end{pgfscope}%
\end{pgfscope}%
\begin{pgfscope}%
\definecolor{textcolor}{rgb}{0.000000,0.000000,0.000000}%
\pgfsetstrokecolor{textcolor}%
\pgfsetfillcolor{textcolor}%
\pgftext[x=5.931466in,y=0.318273in,,top]{\color{textcolor}\rmfamily\fontsize{10.000000}{12.000000}\selectfont \num{1250}}%
\end{pgfscope}%
\begin{pgfscope}%
\definecolor{textcolor}{rgb}{0.000000,0.000000,0.000000}%
\pgfsetstrokecolor{textcolor}%
\pgfsetfillcolor{textcolor}%
\pgftext[x=4.737403in,y=0.140062in,,top]{\color{textcolor}\rmfamily\fontsize{10.000000}{12.000000}\selectfont Intensität in ADU}%
\end{pgfscope}%
\begin{pgfscope}%
\pgfsetrectcap%
\pgfsetmiterjoin%
\pgfsetlinewidth{0.803000pt}%
\definecolor{currentstroke}{rgb}{0.000000,0.000000,0.000000}%
\pgfsetstrokecolor{currentstroke}%
\pgfsetdash{}{0pt}%
\pgfpathmoveto{\pgfqpoint{3.350715in}{0.415495in}}%
\pgfpathlineto{\pgfqpoint{3.350715in}{0.484829in}}%
\pgfpathlineto{\pgfqpoint{3.350715in}{0.554164in}}%
\pgfpathlineto{\pgfqpoint{6.124092in}{0.554164in}}%
\pgfpathlineto{\pgfqpoint{6.124092in}{0.484829in}}%
\pgfpathlineto{\pgfqpoint{6.124092in}{0.415495in}}%
\pgfpathlineto{\pgfqpoint{3.350715in}{0.415495in}}%
\pgfpathclose%
\pgfusepath{stroke}%
\end{pgfscope}%
\end{pgfpicture}%
\makeatother%
\endgroup%

    \label{fig:butterfly_moench_sum}
    \caption{Das direkte Strahprofil auf dem Detektor, das mit der \gls{rzp} für Gd fokussierst wurde. Abgebildet sind (a) die Summe von 26026 Pulsen und (b) ein einzelner Puls. Der hellste Bereich liegt um \SI{1189}{\eV}; die Photonenenergie nimmt von unten nach oben ab. Die Bereiche, in den die Photonenenergie $h\nu_{\text{Gd, M5}} = \SI{1184,79}{\eV}$ bzw. die Zielphotonenenergie der \gls{rzp} $\SI{1189}{\eV}$, abgebildet wurden, sind mit dem rotem bzw. hellgrünen Viereck markiert.}
    \label{fig:butterfly_moench}
\end{figure}
\noindent
Es ist schwer, den Photonenfluss der \gls{pxs} genau anzugeben, weil er nicht nur von der Wahl der \gls{rzp} abhängt, sondern auch der Photonenenergie und der Bandbreite.

\noindent
Es kann beispielsweise der Photonenfluss der \gls{pxs} an der Photonenenergie $h\nu_\text{Gd, M5}$ abgeschätzt werden. So werden ca.\ \SI{617}{\photons} pro Puls innerhalb des roten Vierecks in Abb. \ref{fig:butterfly_moench} detektiert. In Hinblick auf die Pulsperiode \SI{10}{\milli\second} ergibt sich der Photonenfluss in Höhe von ca.\ \SI[per-mode = symbol]{6.2e4}{\photons\per\second}, wobei der Photonenfluss an der Strahllinie P04 höher als \SI[per-mode = symbol]{1e12}{\photons\per\second} ist.

\noindent
Für die Detektion des Streusignals an der \gls{pxs} werden also erheblich längere Belichtungszeiten des Detektors benötigt. In diesem Fall gibt es zwei Strategien für das Aufnahmeverfahren:

\noindent
Die Belichtungszeit wird hoch gesetzt, sodass eine Aufnahme das Streusignal erzeugt von mehreren Pulsen der \gls{pxs} enthält. In diesem Fall müssen sowohl das Detektordunkelrauschen, das typischerweise mit der Zunahme der Belichtungszeit steigt, als auch das Ausleserauschen klein gegenüber dem Streusignal sein.

\noindent
Die andere Methode basiert auf der Trennenung des Hintergrundrauschens vom Streusignal. Die Belichtungszeit wird möglichst klein gewählt, sodass jeder Puls der \gls{pxs} einzeln aufgenommen wird und das Detektordunkelrauschen gesenkt wird. In diesem Fall ist das Ausleserauschen nun dominierend und muss klein gegenüber dem Signal eines einzelnen detektierten Photons sein. Dann lassen sich einzelne Photonen-Ergebnisse vom Rauschen trennen.

\noindent
Die beiden Methoden haben jeweils gewisse Vor- und Nachteile. Die erste Aufnahmemethode ist einfach zu realisieren. Der Dynamikumfang des Detektors muss jedoch idealerweise groß genug sein, damit das gemessene Signal innerhalb des gesamten Sensors linear abgebildet wird. In der Wirklichkeit ist das oft schwer realisierbar. So werden typischerweise potenziell überbelichtete Sensorbereiche des Detektors vom Röntgenstrahl abgeschirmt, was die Normierung des Streusignal auf die einfallende Intensität erschwert. Die zweite Aufnahmemethode lässt sich gut mit einem Anregungs-Abfrage-Experiment (\emph{engl. pump-probe experiment}) kombinieren und bietet besseren Dynamikumfang. Die Photonenerkennung kann jedoch Artefakte verursachen.

\noindent
In dieser Arbeit wird die zweite Methode verwendet. Dazu wurde ein neuartiger Detektor an der Laborquelle in Betrieb genommen, wie im nächsten Kapitel beschrieben. Eine besondere Herausforderung war die Entwicklung und Überprüfung von nummerischen Verfahren zur Erkennung von Einzelphotonen-Ereignissen in den Detektorbildern.


% 11 mittig: "So dient ein vertikaler Spalt..." ich glaube, der Spalt ist horizontal, oder?
% 16 \gls{pxs}: X-ray inkonsequent in der Groß-/Kleinschreibung: zuvor großes X, kleines -ray
% 27 Ich glaube, das Dysprosium interessiert an dieser Stelle nicht. Würde ich umformulieren, um das auszulassen.
% 28-31 (Abb. 8a) die Skalierung ist nicht besonders gut gelungen (Auswahl der Zwischenschritte); es ist nicht erkennbar, ob linear oder logarithmisch skaliert ist
% 39 Die Art, wie Photonen s⁻¹ in der PDF dargestellt wird, ist irgendwie unglücklich. Kann man da einen Multiplikations-Operator dazwischen setzen?
% 54 Vllt kann man auch deinen ursprünglichen Satz beibehalten und lediglich das "diskutiert" am Ende gegen "gehen" ersetzen
