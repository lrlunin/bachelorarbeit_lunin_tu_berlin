\newacronym{qe}{QE}{Quanteneffizienz}
\newacronym{fdpa}{FDPA}{fehldetektierten Photonen pro Pixel pro Aufnahme}
\newacronym{photnenfluss}{PF}{Photonenfluss}
\chapter{Ergebnisse}
\label{text:auswertung}
Das Ziel dieser Arbeit ist es, ein Experiment zu entwerfen und durchzuführen, das nachweist, dass die Detektion resonanter Kleinwinkelstreuung von einer magnetisch heterogenen Probe an einer Laborquelle für weiche Röntgenstrahlen möglich ist. Als Röntgenquelle wird eine Laser-getriebene Plasma-Quelle, in der die Röntgenstrahlung aus dem breiten Emissionspektrum mit einer Reflexionszonenplatte nähe um die Gd-Resonanzenergie fokussiert wird, eingesetzt. Das Experiment soll zeigen, ob die zu erwartende geringe Kohärenz der Quelle für den Streuversuch ausreichend ist und ob magnetischer Streukontrast auch für unpolarisierte Strahlung möglich ist. Größte Herausforderung ist aber die Detektion des zu erwartenden extrem kleinen Streusignals in der Größenordung von nur einige wenigen Photonen pro Puls.

\noindent
Die Streuung wird an der Fe/Gd-Multilagenprobe mit Breite von magnetischen Domänen ca. \SI{300}{\nano\meter} beobachtet. Der MÖNCH-Detektor wird in \SI{607(7)}{\milli\meter} Abstand von der Probe in der Transmissionsgeometrie befestigt. Mit der \qtyproduct{10 x 10}{\milli\meter} Sensorgröße kann maximaler Streuwinkel $2\theta = \SI{0.47}{\degree}$ detektiert werden. Dem maximalen Streuwinkel entspricht der Betrag vom Streuvektor \SI{99}{\per\micro\meter}, wenn Photonen mit Wellenlänge $\lambda_\text{Gd, M5} = \SI{1.045}{\nano\meter}$ gestreut werden. So kann das erwarte ringförmige Streumuster der Probe mit Radius \SI{20}{\per\micro\meter}, der sich als die Fourier-Transformierte des Domänenmusters ergibt, in dem vorhanden experimentellem Aufbau beobachtet werden.

\noindent
Die wesentliche Neuerung des durchgeführten Experiments ist die Einzelpuls-basierte Detektion des Signals. Es soll gezeigt werden, dass diese Art der Detektion in Verbindung mit Algorithmen zum Auffinden einzelner Photonenereignisse in jedem Einzelbild des Detektors es möglich macht, das \glsfirst{snr} so zu verbessern, dass auch extrem kleine Signale, wie die Kleinwinkelstreuung, gemessen werden können. Für dieses Detektionsschema war es nötig, die Röntegenquelle und den Detektor elektrisch zu synchronisieren und den Detektor in das Datenerfassungssystem der Laborquelle zu integrieren. Die Algorithmen zum Trennen von Photonergeignissen und Hintergrundrauschen werden im Zuge dieses Experiments genau analysiert.

\noindent
Im Laufe des Experiments werden \num{10000} Dunkelbilder mit dem MÖNCH-Detektor aufgenommen, die zur Bestimmung des konstanten Offsets jedes Pixels und der mittleren Standardabweichung vom Detektorrauschen benutzt werden. Als Nächstes wird das Absorptionsspektrum aufgenommen, um die Identität zwischen der Kippung der Reflexionszonenplatte und der auf dem Detektor abgebildeten Photonenenergie festzustellen. Im Anschluss wird der Versuch der Detektion der magnetischen Streuung an der resonanten und nicht-resonanten Photonenenergie durchgeführt.   

\noindent
Die Auswertung der Messung wird durch die hohe Zahl an Aufnahmen erschwert. \num{50000} Aufnahmen sind insgesamt ca. \qty{180}{\giga\byte} groß. Jede Aufnahme wird einzeln also von anderen Aufnahmen unabhängig ausgewertet. So ist der Auswertungsvorgang leicht parallelisierbar. Benutzt wird die high-level API Bibliothek \textit{dask-image} \cite{dask-library}, die parallelisierte und optimierte Ausführung der eingegebenen Funktionen ohne großen technischen Aufwand ermöglicht.

\section{Dunkelbild Analyse}
Die \num{10000} aufgenommenen Dunkelbilder werden gemittelt und als ein einzelnes Bild gespeichert. Die Mittlung verringert das Rauschen um einen Faktor $\sqrt{10000} = 100$. So wird im gemittelten Bild i. W. nur der zeitlich konstante Offset in jedem Pixel behalten. Das gemittelte Bild ist in Abb. \ref{fig:capture_ped_diff}b dargestellt und wird vor der weiteren Analyse von jeder einzelnen Aufnahme subtrahiert.

\noindent
Exemplarisch soll hier die Rauschcharakteristik des Detektors für die im Experiment gewählte Belichtungszeit $\tau = \SI{1}{\micro\second}$ gezeigt werden. Dazu wird das gemittelte Bild von jedem der \num{10000} Dunkelbilder subtrahiert. Die Pixelwerte, die sich als Differenzen ergeben, werden in ein Histogramm eingetragen und diese Verteilung wird mit der Funktion
\begin{equation}
    G(W, \mu, \sigma_R, A) = \frac{A}{\sqrt{2\pi \sigma_R^2}}\exp\left[-\frac{(W - \mu)^2}{2\sigma_R^2}\right]
    \label{eq:gauss_funktion}
\end{equation}
angepasst, die auf die Fläche $A$ normierte Gauß-Funktion mit dem Mittelwert $\mu$ und Standardabweichung $\sigma_R$ ist.
\begin{figure}[H]
    \centering
    %% Creator: Matplotlib, PGF backend
%%
%% To include the figure in your LaTeX document, write
%%   \input{<filename>.pgf}
%%
%% Make sure the required packages are loaded in your preamble
%%   \usepackage{pgf}
%%
%% Also ensure that all the required font packages are loaded; for instance,
%% the lmodern package is sometimes necessary when using math font.
%%   \usepackage{lmodern}
%%
%% Figures using additional raster images can only be included by \input if
%% they are in the same directory as the main LaTeX file. For loading figures
%% from other directories you can use the `import` package
%%   \usepackage{import}
%%
%% and then include the figures with
%%   \import{<path to file>}{<filename>.pgf}
%%
%% Matplotlib used the following preamble
%%   \usepackage{amsmath} \usepackage[utf8]{inputenc} \usepackage[T1]{fontenc} \usepackage[output-decimal-marker={,},print-unity-mantissa=false]{siunitx} \sisetup{per-mode=fraction, separate-uncertainty = true, locale = DE} \usepackage[acronym, toc, section=section, nonumberlist, nopostdot]{glossaries-extra}
%%
\begingroup%
\makeatletter%
\begin{pgfpicture}%
\pgfpathrectangle{\pgfpointorigin}{\pgfqpoint{6.381121in}{4.408226in}}%
\pgfusepath{use as bounding box, clip}%
\begin{pgfscope}%
\pgfsetbuttcap%
\pgfsetmiterjoin%
\pgfsetlinewidth{0.000000pt}%
\definecolor{currentstroke}{rgb}{1.000000,1.000000,1.000000}%
\pgfsetstrokecolor{currentstroke}%
\pgfsetstrokeopacity{0.000000}%
\pgfsetdash{}{0pt}%
\pgfpathmoveto{\pgfqpoint{0.000000in}{0.000000in}}%
\pgfpathlineto{\pgfqpoint{6.381121in}{0.000000in}}%
\pgfpathlineto{\pgfqpoint{6.381121in}{4.408226in}}%
\pgfpathlineto{\pgfqpoint{0.000000in}{4.408226in}}%
\pgfpathlineto{\pgfqpoint{0.000000in}{0.000000in}}%
\pgfpathclose%
\pgfusepath{}%
\end{pgfscope}%
\begin{pgfscope}%
\pgfsetbuttcap%
\pgfsetmiterjoin%
\definecolor{currentfill}{rgb}{1.000000,1.000000,1.000000}%
\pgfsetfillcolor{currentfill}%
\pgfsetlinewidth{0.000000pt}%
\definecolor{currentstroke}{rgb}{0.000000,0.000000,0.000000}%
\pgfsetstrokecolor{currentstroke}%
\pgfsetstrokeopacity{0.000000}%
\pgfsetdash{}{0pt}%
\pgfpathmoveto{\pgfqpoint{0.552903in}{0.498088in}}%
\pgfpathlineto{\pgfqpoint{6.281121in}{0.498088in}}%
\pgfpathlineto{\pgfqpoint{6.281121in}{4.161984in}}%
\pgfpathlineto{\pgfqpoint{0.552903in}{4.161984in}}%
\pgfpathlineto{\pgfqpoint{0.552903in}{0.498088in}}%
\pgfpathclose%
\pgfusepath{fill}%
\end{pgfscope}%
\begin{pgfscope}%
\pgfsetbuttcap%
\pgfsetroundjoin%
\definecolor{currentfill}{rgb}{0.000000,0.000000,0.000000}%
\pgfsetfillcolor{currentfill}%
\pgfsetlinewidth{0.803000pt}%
\definecolor{currentstroke}{rgb}{0.000000,0.000000,0.000000}%
\pgfsetstrokecolor{currentstroke}%
\pgfsetdash{}{0pt}%
\pgfsys@defobject{currentmarker}{\pgfqpoint{0.000000in}{-0.048611in}}{\pgfqpoint{0.000000in}{0.000000in}}{%
\pgfpathmoveto{\pgfqpoint{0.000000in}{0.000000in}}%
\pgfpathlineto{\pgfqpoint{0.000000in}{-0.048611in}}%
\pgfusepath{stroke,fill}%
}%
\begin{pgfscope}%
\pgfsys@transformshift{0.786976in}{0.498088in}%
\pgfsys@useobject{currentmarker}{}%
\end{pgfscope}%
\end{pgfscope}%
\begin{pgfscope}%
\definecolor{textcolor}{rgb}{0.000000,0.000000,0.000000}%
\pgfsetstrokecolor{textcolor}%
\pgfsetfillcolor{textcolor}%
\pgftext[x=0.786976in,y=0.400866in,,top]{\color{textcolor}\rmfamily\fontsize{10.000000}{12.000000}\selectfont \(\displaystyle {\ensuremath{-}100}\)}%
\end{pgfscope}%
\begin{pgfscope}%
\pgfsetbuttcap%
\pgfsetroundjoin%
\definecolor{currentfill}{rgb}{0.000000,0.000000,0.000000}%
\pgfsetfillcolor{currentfill}%
\pgfsetlinewidth{0.803000pt}%
\definecolor{currentstroke}{rgb}{0.000000,0.000000,0.000000}%
\pgfsetstrokecolor{currentstroke}%
\pgfsetdash{}{0pt}%
\pgfsys@defobject{currentmarker}{\pgfqpoint{0.000000in}{-0.048611in}}{\pgfqpoint{0.000000in}{0.000000in}}{%
\pgfpathmoveto{\pgfqpoint{0.000000in}{0.000000in}}%
\pgfpathlineto{\pgfqpoint{0.000000in}{-0.048611in}}%
\pgfusepath{stroke,fill}%
}%
\begin{pgfscope}%
\pgfsys@transformshift{1.444485in}{0.498088in}%
\pgfsys@useobject{currentmarker}{}%
\end{pgfscope}%
\end{pgfscope}%
\begin{pgfscope}%
\definecolor{textcolor}{rgb}{0.000000,0.000000,0.000000}%
\pgfsetstrokecolor{textcolor}%
\pgfsetfillcolor{textcolor}%
\pgftext[x=1.444485in,y=0.400866in,,top]{\color{textcolor}\rmfamily\fontsize{10.000000}{12.000000}\selectfont \(\displaystyle {\ensuremath{-}75}\)}%
\end{pgfscope}%
\begin{pgfscope}%
\pgfsetbuttcap%
\pgfsetroundjoin%
\definecolor{currentfill}{rgb}{0.000000,0.000000,0.000000}%
\pgfsetfillcolor{currentfill}%
\pgfsetlinewidth{0.803000pt}%
\definecolor{currentstroke}{rgb}{0.000000,0.000000,0.000000}%
\pgfsetstrokecolor{currentstroke}%
\pgfsetdash{}{0pt}%
\pgfsys@defobject{currentmarker}{\pgfqpoint{0.000000in}{-0.048611in}}{\pgfqpoint{0.000000in}{0.000000in}}{%
\pgfpathmoveto{\pgfqpoint{0.000000in}{0.000000in}}%
\pgfpathlineto{\pgfqpoint{0.000000in}{-0.048611in}}%
\pgfusepath{stroke,fill}%
}%
\begin{pgfscope}%
\pgfsys@transformshift{2.101994in}{0.498088in}%
\pgfsys@useobject{currentmarker}{}%
\end{pgfscope}%
\end{pgfscope}%
\begin{pgfscope}%
\definecolor{textcolor}{rgb}{0.000000,0.000000,0.000000}%
\pgfsetstrokecolor{textcolor}%
\pgfsetfillcolor{textcolor}%
\pgftext[x=2.101994in,y=0.400866in,,top]{\color{textcolor}\rmfamily\fontsize{10.000000}{12.000000}\selectfont \(\displaystyle {\ensuremath{-}50}\)}%
\end{pgfscope}%
\begin{pgfscope}%
\pgfsetbuttcap%
\pgfsetroundjoin%
\definecolor{currentfill}{rgb}{0.000000,0.000000,0.000000}%
\pgfsetfillcolor{currentfill}%
\pgfsetlinewidth{0.803000pt}%
\definecolor{currentstroke}{rgb}{0.000000,0.000000,0.000000}%
\pgfsetstrokecolor{currentstroke}%
\pgfsetdash{}{0pt}%
\pgfsys@defobject{currentmarker}{\pgfqpoint{0.000000in}{-0.048611in}}{\pgfqpoint{0.000000in}{0.000000in}}{%
\pgfpathmoveto{\pgfqpoint{0.000000in}{0.000000in}}%
\pgfpathlineto{\pgfqpoint{0.000000in}{-0.048611in}}%
\pgfusepath{stroke,fill}%
}%
\begin{pgfscope}%
\pgfsys@transformshift{2.759503in}{0.498088in}%
\pgfsys@useobject{currentmarker}{}%
\end{pgfscope}%
\end{pgfscope}%
\begin{pgfscope}%
\definecolor{textcolor}{rgb}{0.000000,0.000000,0.000000}%
\pgfsetstrokecolor{textcolor}%
\pgfsetfillcolor{textcolor}%
\pgftext[x=2.759503in,y=0.400866in,,top]{\color{textcolor}\rmfamily\fontsize{10.000000}{12.000000}\selectfont \(\displaystyle {\ensuremath{-}25}\)}%
\end{pgfscope}%
\begin{pgfscope}%
\pgfsetbuttcap%
\pgfsetroundjoin%
\definecolor{currentfill}{rgb}{0.000000,0.000000,0.000000}%
\pgfsetfillcolor{currentfill}%
\pgfsetlinewidth{0.803000pt}%
\definecolor{currentstroke}{rgb}{0.000000,0.000000,0.000000}%
\pgfsetstrokecolor{currentstroke}%
\pgfsetdash{}{0pt}%
\pgfsys@defobject{currentmarker}{\pgfqpoint{0.000000in}{-0.048611in}}{\pgfqpoint{0.000000in}{0.000000in}}{%
\pgfpathmoveto{\pgfqpoint{0.000000in}{0.000000in}}%
\pgfpathlineto{\pgfqpoint{0.000000in}{-0.048611in}}%
\pgfusepath{stroke,fill}%
}%
\begin{pgfscope}%
\pgfsys@transformshift{3.417012in}{0.498088in}%
\pgfsys@useobject{currentmarker}{}%
\end{pgfscope}%
\end{pgfscope}%
\begin{pgfscope}%
\definecolor{textcolor}{rgb}{0.000000,0.000000,0.000000}%
\pgfsetstrokecolor{textcolor}%
\pgfsetfillcolor{textcolor}%
\pgftext[x=3.417012in,y=0.400866in,,top]{\color{textcolor}\rmfamily\fontsize{10.000000}{12.000000}\selectfont \(\displaystyle {0}\)}%
\end{pgfscope}%
\begin{pgfscope}%
\pgfsetbuttcap%
\pgfsetroundjoin%
\definecolor{currentfill}{rgb}{0.000000,0.000000,0.000000}%
\pgfsetfillcolor{currentfill}%
\pgfsetlinewidth{0.803000pt}%
\definecolor{currentstroke}{rgb}{0.000000,0.000000,0.000000}%
\pgfsetstrokecolor{currentstroke}%
\pgfsetdash{}{0pt}%
\pgfsys@defobject{currentmarker}{\pgfqpoint{0.000000in}{-0.048611in}}{\pgfqpoint{0.000000in}{0.000000in}}{%
\pgfpathmoveto{\pgfqpoint{0.000000in}{0.000000in}}%
\pgfpathlineto{\pgfqpoint{0.000000in}{-0.048611in}}%
\pgfusepath{stroke,fill}%
}%
\begin{pgfscope}%
\pgfsys@transformshift{4.074521in}{0.498088in}%
\pgfsys@useobject{currentmarker}{}%
\end{pgfscope}%
\end{pgfscope}%
\begin{pgfscope}%
\definecolor{textcolor}{rgb}{0.000000,0.000000,0.000000}%
\pgfsetstrokecolor{textcolor}%
\pgfsetfillcolor{textcolor}%
\pgftext[x=4.074521in,y=0.400866in,,top]{\color{textcolor}\rmfamily\fontsize{10.000000}{12.000000}\selectfont \(\displaystyle {25}\)}%
\end{pgfscope}%
\begin{pgfscope}%
\pgfsetbuttcap%
\pgfsetroundjoin%
\definecolor{currentfill}{rgb}{0.000000,0.000000,0.000000}%
\pgfsetfillcolor{currentfill}%
\pgfsetlinewidth{0.803000pt}%
\definecolor{currentstroke}{rgb}{0.000000,0.000000,0.000000}%
\pgfsetstrokecolor{currentstroke}%
\pgfsetdash{}{0pt}%
\pgfsys@defobject{currentmarker}{\pgfqpoint{0.000000in}{-0.048611in}}{\pgfqpoint{0.000000in}{0.000000in}}{%
\pgfpathmoveto{\pgfqpoint{0.000000in}{0.000000in}}%
\pgfpathlineto{\pgfqpoint{0.000000in}{-0.048611in}}%
\pgfusepath{stroke,fill}%
}%
\begin{pgfscope}%
\pgfsys@transformshift{4.732030in}{0.498088in}%
\pgfsys@useobject{currentmarker}{}%
\end{pgfscope}%
\end{pgfscope}%
\begin{pgfscope}%
\definecolor{textcolor}{rgb}{0.000000,0.000000,0.000000}%
\pgfsetstrokecolor{textcolor}%
\pgfsetfillcolor{textcolor}%
\pgftext[x=4.732030in,y=0.400866in,,top]{\color{textcolor}\rmfamily\fontsize{10.000000}{12.000000}\selectfont \(\displaystyle {50}\)}%
\end{pgfscope}%
\begin{pgfscope}%
\pgfsetbuttcap%
\pgfsetroundjoin%
\definecolor{currentfill}{rgb}{0.000000,0.000000,0.000000}%
\pgfsetfillcolor{currentfill}%
\pgfsetlinewidth{0.803000pt}%
\definecolor{currentstroke}{rgb}{0.000000,0.000000,0.000000}%
\pgfsetstrokecolor{currentstroke}%
\pgfsetdash{}{0pt}%
\pgfsys@defobject{currentmarker}{\pgfqpoint{0.000000in}{-0.048611in}}{\pgfqpoint{0.000000in}{0.000000in}}{%
\pgfpathmoveto{\pgfqpoint{0.000000in}{0.000000in}}%
\pgfpathlineto{\pgfqpoint{0.000000in}{-0.048611in}}%
\pgfusepath{stroke,fill}%
}%
\begin{pgfscope}%
\pgfsys@transformshift{5.389539in}{0.498088in}%
\pgfsys@useobject{currentmarker}{}%
\end{pgfscope}%
\end{pgfscope}%
\begin{pgfscope}%
\definecolor{textcolor}{rgb}{0.000000,0.000000,0.000000}%
\pgfsetstrokecolor{textcolor}%
\pgfsetfillcolor{textcolor}%
\pgftext[x=5.389539in,y=0.400866in,,top]{\color{textcolor}\rmfamily\fontsize{10.000000}{12.000000}\selectfont \(\displaystyle {75}\)}%
\end{pgfscope}%
\begin{pgfscope}%
\pgfsetbuttcap%
\pgfsetroundjoin%
\definecolor{currentfill}{rgb}{0.000000,0.000000,0.000000}%
\pgfsetfillcolor{currentfill}%
\pgfsetlinewidth{0.803000pt}%
\definecolor{currentstroke}{rgb}{0.000000,0.000000,0.000000}%
\pgfsetstrokecolor{currentstroke}%
\pgfsetdash{}{0pt}%
\pgfsys@defobject{currentmarker}{\pgfqpoint{0.000000in}{-0.048611in}}{\pgfqpoint{0.000000in}{0.000000in}}{%
\pgfpathmoveto{\pgfqpoint{0.000000in}{0.000000in}}%
\pgfpathlineto{\pgfqpoint{0.000000in}{-0.048611in}}%
\pgfusepath{stroke,fill}%
}%
\begin{pgfscope}%
\pgfsys@transformshift{6.047048in}{0.498088in}%
\pgfsys@useobject{currentmarker}{}%
\end{pgfscope}%
\end{pgfscope}%
\begin{pgfscope}%
\definecolor{textcolor}{rgb}{0.000000,0.000000,0.000000}%
\pgfsetstrokecolor{textcolor}%
\pgfsetfillcolor{textcolor}%
\pgftext[x=6.047048in,y=0.400866in,,top]{\color{textcolor}\rmfamily\fontsize{10.000000}{12.000000}\selectfont \(\displaystyle {100}\)}%
\end{pgfscope}%
\begin{pgfscope}%
\definecolor{textcolor}{rgb}{0.000000,0.000000,0.000000}%
\pgfsetstrokecolor{textcolor}%
\pgfsetfillcolor{textcolor}%
\pgftext[x=3.417012in,y=0.222655in,,top]{\color{textcolor}\rmfamily\fontsize{10.000000}{12.000000}\selectfont Pixelwert \(\displaystyle W\) in ADU}%
\end{pgfscope}%
\begin{pgfscope}%
\pgfsetbuttcap%
\pgfsetroundjoin%
\definecolor{currentfill}{rgb}{0.000000,0.000000,0.000000}%
\pgfsetfillcolor{currentfill}%
\pgfsetlinewidth{0.803000pt}%
\definecolor{currentstroke}{rgb}{0.000000,0.000000,0.000000}%
\pgfsetstrokecolor{currentstroke}%
\pgfsetdash{}{0pt}%
\pgfsys@defobject{currentmarker}{\pgfqpoint{-0.048611in}{0.000000in}}{\pgfqpoint{-0.000000in}{0.000000in}}{%
\pgfpathmoveto{\pgfqpoint{-0.000000in}{0.000000in}}%
\pgfpathlineto{\pgfqpoint{-0.048611in}{0.000000in}}%
\pgfusepath{stroke,fill}%
}%
\begin{pgfscope}%
\pgfsys@transformshift{0.552903in}{0.664615in}%
\pgfsys@useobject{currentmarker}{}%
\end{pgfscope}%
\end{pgfscope}%
\begin{pgfscope}%
\definecolor{textcolor}{rgb}{0.000000,0.000000,0.000000}%
\pgfsetstrokecolor{textcolor}%
\pgfsetfillcolor{textcolor}%
\pgftext[x=0.278211in, y=0.616791in, left, base]{\color{textcolor}\rmfamily\fontsize{10.000000}{12.000000}\selectfont \num{0.0}}%
\end{pgfscope}%
\begin{pgfscope}%
\pgfsetbuttcap%
\pgfsetroundjoin%
\definecolor{currentfill}{rgb}{0.000000,0.000000,0.000000}%
\pgfsetfillcolor{currentfill}%
\pgfsetlinewidth{0.803000pt}%
\definecolor{currentstroke}{rgb}{0.000000,0.000000,0.000000}%
\pgfsetstrokecolor{currentstroke}%
\pgfsetdash{}{0pt}%
\pgfsys@defobject{currentmarker}{\pgfqpoint{-0.048611in}{0.000000in}}{\pgfqpoint{-0.000000in}{0.000000in}}{%
\pgfpathmoveto{\pgfqpoint{-0.000000in}{0.000000in}}%
\pgfpathlineto{\pgfqpoint{-0.048611in}{0.000000in}}%
\pgfusepath{stroke,fill}%
}%
\begin{pgfscope}%
\pgfsys@transformshift{0.552903in}{1.186011in}%
\pgfsys@useobject{currentmarker}{}%
\end{pgfscope}%
\end{pgfscope}%
\begin{pgfscope}%
\definecolor{textcolor}{rgb}{0.000000,0.000000,0.000000}%
\pgfsetstrokecolor{textcolor}%
\pgfsetfillcolor{textcolor}%
\pgftext[x=0.278211in, y=1.138187in, left, base]{\color{textcolor}\rmfamily\fontsize{10.000000}{12.000000}\selectfont \num{0.5}}%
\end{pgfscope}%
\begin{pgfscope}%
\pgfsetbuttcap%
\pgfsetroundjoin%
\definecolor{currentfill}{rgb}{0.000000,0.000000,0.000000}%
\pgfsetfillcolor{currentfill}%
\pgfsetlinewidth{0.803000pt}%
\definecolor{currentstroke}{rgb}{0.000000,0.000000,0.000000}%
\pgfsetstrokecolor{currentstroke}%
\pgfsetdash{}{0pt}%
\pgfsys@defobject{currentmarker}{\pgfqpoint{-0.048611in}{0.000000in}}{\pgfqpoint{-0.000000in}{0.000000in}}{%
\pgfpathmoveto{\pgfqpoint{-0.000000in}{0.000000in}}%
\pgfpathlineto{\pgfqpoint{-0.048611in}{0.000000in}}%
\pgfusepath{stroke,fill}%
}%
\begin{pgfscope}%
\pgfsys@transformshift{0.552903in}{1.707408in}%
\pgfsys@useobject{currentmarker}{}%
\end{pgfscope}%
\end{pgfscope}%
\begin{pgfscope}%
\definecolor{textcolor}{rgb}{0.000000,0.000000,0.000000}%
\pgfsetstrokecolor{textcolor}%
\pgfsetfillcolor{textcolor}%
\pgftext[x=0.278211in, y=1.659583in, left, base]{\color{textcolor}\rmfamily\fontsize{10.000000}{12.000000}\selectfont \num{1.0}}%
\end{pgfscope}%
\begin{pgfscope}%
\pgfsetbuttcap%
\pgfsetroundjoin%
\definecolor{currentfill}{rgb}{0.000000,0.000000,0.000000}%
\pgfsetfillcolor{currentfill}%
\pgfsetlinewidth{0.803000pt}%
\definecolor{currentstroke}{rgb}{0.000000,0.000000,0.000000}%
\pgfsetstrokecolor{currentstroke}%
\pgfsetdash{}{0pt}%
\pgfsys@defobject{currentmarker}{\pgfqpoint{-0.048611in}{0.000000in}}{\pgfqpoint{-0.000000in}{0.000000in}}{%
\pgfpathmoveto{\pgfqpoint{-0.000000in}{0.000000in}}%
\pgfpathlineto{\pgfqpoint{-0.048611in}{0.000000in}}%
\pgfusepath{stroke,fill}%
}%
\begin{pgfscope}%
\pgfsys@transformshift{0.552903in}{2.228804in}%
\pgfsys@useobject{currentmarker}{}%
\end{pgfscope}%
\end{pgfscope}%
\begin{pgfscope}%
\definecolor{textcolor}{rgb}{0.000000,0.000000,0.000000}%
\pgfsetstrokecolor{textcolor}%
\pgfsetfillcolor{textcolor}%
\pgftext[x=0.278211in, y=2.180980in, left, base]{\color{textcolor}\rmfamily\fontsize{10.000000}{12.000000}\selectfont \num{1.5}}%
\end{pgfscope}%
\begin{pgfscope}%
\pgfsetbuttcap%
\pgfsetroundjoin%
\definecolor{currentfill}{rgb}{0.000000,0.000000,0.000000}%
\pgfsetfillcolor{currentfill}%
\pgfsetlinewidth{0.803000pt}%
\definecolor{currentstroke}{rgb}{0.000000,0.000000,0.000000}%
\pgfsetstrokecolor{currentstroke}%
\pgfsetdash{}{0pt}%
\pgfsys@defobject{currentmarker}{\pgfqpoint{-0.048611in}{0.000000in}}{\pgfqpoint{-0.000000in}{0.000000in}}{%
\pgfpathmoveto{\pgfqpoint{-0.000000in}{0.000000in}}%
\pgfpathlineto{\pgfqpoint{-0.048611in}{0.000000in}}%
\pgfusepath{stroke,fill}%
}%
\begin{pgfscope}%
\pgfsys@transformshift{0.552903in}{2.750201in}%
\pgfsys@useobject{currentmarker}{}%
\end{pgfscope}%
\end{pgfscope}%
\begin{pgfscope}%
\definecolor{textcolor}{rgb}{0.000000,0.000000,0.000000}%
\pgfsetstrokecolor{textcolor}%
\pgfsetfillcolor{textcolor}%
\pgftext[x=0.278211in, y=2.702376in, left, base]{\color{textcolor}\rmfamily\fontsize{10.000000}{12.000000}\selectfont \num{2.0}}%
\end{pgfscope}%
\begin{pgfscope}%
\pgfsetbuttcap%
\pgfsetroundjoin%
\definecolor{currentfill}{rgb}{0.000000,0.000000,0.000000}%
\pgfsetfillcolor{currentfill}%
\pgfsetlinewidth{0.803000pt}%
\definecolor{currentstroke}{rgb}{0.000000,0.000000,0.000000}%
\pgfsetstrokecolor{currentstroke}%
\pgfsetdash{}{0pt}%
\pgfsys@defobject{currentmarker}{\pgfqpoint{-0.048611in}{0.000000in}}{\pgfqpoint{-0.000000in}{0.000000in}}{%
\pgfpathmoveto{\pgfqpoint{-0.000000in}{0.000000in}}%
\pgfpathlineto{\pgfqpoint{-0.048611in}{0.000000in}}%
\pgfusepath{stroke,fill}%
}%
\begin{pgfscope}%
\pgfsys@transformshift{0.552903in}{3.271597in}%
\pgfsys@useobject{currentmarker}{}%
\end{pgfscope}%
\end{pgfscope}%
\begin{pgfscope}%
\definecolor{textcolor}{rgb}{0.000000,0.000000,0.000000}%
\pgfsetstrokecolor{textcolor}%
\pgfsetfillcolor{textcolor}%
\pgftext[x=0.278211in, y=3.223772in, left, base]{\color{textcolor}\rmfamily\fontsize{10.000000}{12.000000}\selectfont \num{2.5}}%
\end{pgfscope}%
\begin{pgfscope}%
\pgfsetbuttcap%
\pgfsetroundjoin%
\definecolor{currentfill}{rgb}{0.000000,0.000000,0.000000}%
\pgfsetfillcolor{currentfill}%
\pgfsetlinewidth{0.803000pt}%
\definecolor{currentstroke}{rgb}{0.000000,0.000000,0.000000}%
\pgfsetstrokecolor{currentstroke}%
\pgfsetdash{}{0pt}%
\pgfsys@defobject{currentmarker}{\pgfqpoint{-0.048611in}{0.000000in}}{\pgfqpoint{-0.000000in}{0.000000in}}{%
\pgfpathmoveto{\pgfqpoint{-0.000000in}{0.000000in}}%
\pgfpathlineto{\pgfqpoint{-0.048611in}{0.000000in}}%
\pgfusepath{stroke,fill}%
}%
\begin{pgfscope}%
\pgfsys@transformshift{0.552903in}{3.792993in}%
\pgfsys@useobject{currentmarker}{}%
\end{pgfscope}%
\end{pgfscope}%
\begin{pgfscope}%
\definecolor{textcolor}{rgb}{0.000000,0.000000,0.000000}%
\pgfsetstrokecolor{textcolor}%
\pgfsetfillcolor{textcolor}%
\pgftext[x=0.278211in, y=3.745169in, left, base]{\color{textcolor}\rmfamily\fontsize{10.000000}{12.000000}\selectfont \num{3.0}}%
\end{pgfscope}%
\begin{pgfscope}%
\definecolor{textcolor}{rgb}{0.000000,0.000000,0.000000}%
\pgfsetstrokecolor{textcolor}%
\pgfsetfillcolor{textcolor}%
\pgftext[x=0.222655in,y=2.330036in,,bottom,rotate=90.000000]{\color{textcolor}\rmfamily\fontsize{10.000000}{12.000000}\selectfont Pixelzahl}%
\end{pgfscope}%
\begin{pgfscope}%
\definecolor{textcolor}{rgb}{0.000000,0.000000,0.000000}%
\pgfsetstrokecolor{textcolor}%
\pgfsetfillcolor{textcolor}%
\pgftext[x=0.552903in,y=4.203651in,left,base]{\color{textcolor}\rmfamily\fontsize{10.000000}{12.000000}\selectfont \(\displaystyle \times{10^{7}}{}\)}%
\end{pgfscope}%
\begin{pgfscope}%
\pgfpathrectangle{\pgfqpoint{0.552903in}{0.498088in}}{\pgfqpoint{5.728219in}{3.663896in}}%
\pgfusepath{clip}%
\pgfsetrectcap%
\pgfsetroundjoin%
\pgfsetlinewidth{1.505625pt}%
\definecolor{currentstroke}{rgb}{1.000000,0.498039,0.054902}%
\pgfsetstrokecolor{currentstroke}%
\pgfsetdash{}{0pt}%
\pgfpathmoveto{\pgfqpoint{0.813276in}{0.664631in}}%
\pgfpathlineto{\pgfqpoint{1.128881in}{0.664878in}}%
\pgfpathlineto{\pgfqpoint{1.260382in}{0.665371in}}%
\pgfpathlineto{\pgfqpoint{1.339283in}{0.665999in}}%
\pgfpathlineto{\pgfqpoint{1.418185in}{0.667092in}}%
\pgfpathlineto{\pgfqpoint{1.470785in}{0.668221in}}%
\pgfpathlineto{\pgfqpoint{1.523386in}{0.669811in}}%
\pgfpathlineto{\pgfqpoint{1.575987in}{0.672029in}}%
\pgfpathlineto{\pgfqpoint{1.602287in}{0.673437in}}%
\pgfpathlineto{\pgfqpoint{1.628587in}{0.675086in}}%
\pgfpathlineto{\pgfqpoint{1.654888in}{0.677012in}}%
\pgfpathlineto{\pgfqpoint{1.681188in}{0.679256in}}%
\pgfpathlineto{\pgfqpoint{1.707488in}{0.681863in}}%
\pgfpathlineto{\pgfqpoint{1.733789in}{0.684883in}}%
\pgfpathlineto{\pgfqpoint{1.760089in}{0.688371in}}%
\pgfpathlineto{\pgfqpoint{1.786390in}{0.692391in}}%
\pgfpathlineto{\pgfqpoint{1.812690in}{0.697008in}}%
\pgfpathlineto{\pgfqpoint{1.838990in}{0.702299in}}%
\pgfpathlineto{\pgfqpoint{1.865291in}{0.708343in}}%
\pgfpathlineto{\pgfqpoint{1.891591in}{0.715230in}}%
\pgfpathlineto{\pgfqpoint{1.917891in}{0.723054in}}%
\pgfpathlineto{\pgfqpoint{1.944192in}{0.731918in}}%
\pgfpathlineto{\pgfqpoint{1.970492in}{0.741933in}}%
\pgfpathlineto{\pgfqpoint{1.996792in}{0.753214in}}%
\pgfpathlineto{\pgfqpoint{2.023093in}{0.765886in}}%
\pgfpathlineto{\pgfqpoint{2.049393in}{0.780080in}}%
\pgfpathlineto{\pgfqpoint{2.075694in}{0.795933in}}%
\pgfpathlineto{\pgfqpoint{2.101994in}{0.813588in}}%
\pgfpathlineto{\pgfqpoint{2.128294in}{0.833192in}}%
\pgfpathlineto{\pgfqpoint{2.154595in}{0.854896in}}%
\pgfpathlineto{\pgfqpoint{2.180895in}{0.878856in}}%
\pgfpathlineto{\pgfqpoint{2.207195in}{0.905227in}}%
\pgfpathlineto{\pgfqpoint{2.233496in}{0.934166in}}%
\pgfpathlineto{\pgfqpoint{2.259796in}{0.965827in}}%
\pgfpathlineto{\pgfqpoint{2.286096in}{1.000362in}}%
\pgfpathlineto{\pgfqpoint{2.312397in}{1.037917in}}%
\pgfpathlineto{\pgfqpoint{2.338697in}{1.078630in}}%
\pgfpathlineto{\pgfqpoint{2.364997in}{1.122631in}}%
\pgfpathlineto{\pgfqpoint{2.391298in}{1.170036in}}%
\pgfpathlineto{\pgfqpoint{2.417598in}{1.220947in}}%
\pgfpathlineto{\pgfqpoint{2.443899in}{1.275449in}}%
\pgfpathlineto{\pgfqpoint{2.470199in}{1.333606in}}%
\pgfpathlineto{\pgfqpoint{2.496499in}{1.395461in}}%
\pgfpathlineto{\pgfqpoint{2.522800in}{1.461030in}}%
\pgfpathlineto{\pgfqpoint{2.549100in}{1.530302in}}%
\pgfpathlineto{\pgfqpoint{2.575400in}{1.603238in}}%
\pgfpathlineto{\pgfqpoint{2.601701in}{1.679763in}}%
\pgfpathlineto{\pgfqpoint{2.628001in}{1.759771in}}%
\pgfpathlineto{\pgfqpoint{2.654301in}{1.843118in}}%
\pgfpathlineto{\pgfqpoint{2.680602in}{1.929624in}}%
\pgfpathlineto{\pgfqpoint{2.706902in}{2.019071in}}%
\pgfpathlineto{\pgfqpoint{2.733203in}{2.111200in}}%
\pgfpathlineto{\pgfqpoint{2.759503in}{2.205718in}}%
\pgfpathlineto{\pgfqpoint{2.785803in}{2.302289in}}%
\pgfpathlineto{\pgfqpoint{2.812104in}{2.400542in}}%
\pgfpathlineto{\pgfqpoint{2.864704in}{2.600431in}}%
\pgfpathlineto{\pgfqpoint{2.917305in}{2.801738in}}%
\pgfpathlineto{\pgfqpoint{2.943605in}{2.901659in}}%
\pgfpathlineto{\pgfqpoint{2.969906in}{3.000373in}}%
\pgfpathlineto{\pgfqpoint{2.996206in}{3.097319in}}%
\pgfpathlineto{\pgfqpoint{3.022507in}{3.191928in}}%
\pgfpathlineto{\pgfqpoint{3.048807in}{3.283624in}}%
\pgfpathlineto{\pgfqpoint{3.075107in}{3.371832in}}%
\pgfpathlineto{\pgfqpoint{3.101408in}{3.455985in}}%
\pgfpathlineto{\pgfqpoint{3.127708in}{3.535527in}}%
\pgfpathlineto{\pgfqpoint{3.154008in}{3.609923in}}%
\pgfpathlineto{\pgfqpoint{3.180309in}{3.678659in}}%
\pgfpathlineto{\pgfqpoint{3.206609in}{3.741256in}}%
\pgfpathlineto{\pgfqpoint{3.232909in}{3.797268in}}%
\pgfpathlineto{\pgfqpoint{3.259210in}{3.846290in}}%
\pgfpathlineto{\pgfqpoint{3.285510in}{3.887966in}}%
\pgfpathlineto{\pgfqpoint{3.311810in}{3.921990in}}%
\pgfpathlineto{\pgfqpoint{3.338111in}{3.948107in}}%
\pgfpathlineto{\pgfqpoint{3.364411in}{3.966123in}}%
\pgfpathlineto{\pgfqpoint{3.390712in}{3.975904in}}%
\pgfpathlineto{\pgfqpoint{3.417012in}{3.977375in}}%
\pgfpathlineto{\pgfqpoint{3.443312in}{3.970525in}}%
\pgfpathlineto{\pgfqpoint{3.469613in}{3.955407in}}%
\pgfpathlineto{\pgfqpoint{3.495913in}{3.932133in}}%
\pgfpathlineto{\pgfqpoint{3.522213in}{3.900877in}}%
\pgfpathlineto{\pgfqpoint{3.548514in}{3.861873in}}%
\pgfpathlineto{\pgfqpoint{3.574814in}{3.815407in}}%
\pgfpathlineto{\pgfqpoint{3.601114in}{3.761822in}}%
\pgfpathlineto{\pgfqpoint{3.627415in}{3.701503in}}%
\pgfpathlineto{\pgfqpoint{3.653715in}{3.634883in}}%
\pgfpathlineto{\pgfqpoint{3.680016in}{3.562430in}}%
\pgfpathlineto{\pgfqpoint{3.706316in}{3.484646in}}%
\pgfpathlineto{\pgfqpoint{3.732616in}{3.402059in}}%
\pgfpathlineto{\pgfqpoint{3.758917in}{3.315220in}}%
\pgfpathlineto{\pgfqpoint{3.785217in}{3.224691in}}%
\pgfpathlineto{\pgfqpoint{3.811517in}{3.131046in}}%
\pgfpathlineto{\pgfqpoint{3.837818in}{3.034860in}}%
\pgfpathlineto{\pgfqpoint{3.864118in}{2.936706in}}%
\pgfpathlineto{\pgfqpoint{3.916719in}{2.736737in}}%
\pgfpathlineto{\pgfqpoint{3.969320in}{2.535460in}}%
\pgfpathlineto{\pgfqpoint{3.995620in}{2.435586in}}%
\pgfpathlineto{\pgfqpoint{4.021920in}{2.336834in}}%
\pgfpathlineto{\pgfqpoint{4.048221in}{2.239625in}}%
\pgfpathlineto{\pgfqpoint{4.074521in}{2.144342in}}%
\pgfpathlineto{\pgfqpoint{4.100821in}{2.051332in}}%
\pgfpathlineto{\pgfqpoint{4.127122in}{1.960906in}}%
\pgfpathlineto{\pgfqpoint{4.153422in}{1.873334in}}%
\pgfpathlineto{\pgfqpoint{4.179722in}{1.788848in}}%
\pgfpathlineto{\pgfqpoint{4.206023in}{1.707642in}}%
\pgfpathlineto{\pgfqpoint{4.232323in}{1.629873in}}%
\pgfpathlineto{\pgfqpoint{4.258623in}{1.555659in}}%
\pgfpathlineto{\pgfqpoint{4.284924in}{1.485085in}}%
\pgfpathlineto{\pgfqpoint{4.311224in}{1.418205in}}%
\pgfpathlineto{\pgfqpoint{4.337525in}{1.355038in}}%
\pgfpathlineto{\pgfqpoint{4.363825in}{1.295578in}}%
\pgfpathlineto{\pgfqpoint{4.390125in}{1.239791in}}%
\pgfpathlineto{\pgfqpoint{4.416426in}{1.187620in}}%
\pgfpathlineto{\pgfqpoint{4.442726in}{1.138987in}}%
\pgfpathlineto{\pgfqpoint{4.469026in}{1.093796in}}%
\pgfpathlineto{\pgfqpoint{4.495327in}{1.051935in}}%
\pgfpathlineto{\pgfqpoint{4.521627in}{1.013280in}}%
\pgfpathlineto{\pgfqpoint{4.547927in}{0.977694in}}%
\pgfpathlineto{\pgfqpoint{4.574228in}{0.945034in}}%
\pgfpathlineto{\pgfqpoint{4.600528in}{0.915151in}}%
\pgfpathlineto{\pgfqpoint{4.626829in}{0.887891in}}%
\pgfpathlineto{\pgfqpoint{4.653129in}{0.863097in}}%
\pgfpathlineto{\pgfqpoint{4.679429in}{0.840613in}}%
\pgfpathlineto{\pgfqpoint{4.705730in}{0.820284in}}%
\pgfpathlineto{\pgfqpoint{4.732030in}{0.801958in}}%
\pgfpathlineto{\pgfqpoint{4.758330in}{0.785485in}}%
\pgfpathlineto{\pgfqpoint{4.784631in}{0.770721in}}%
\pgfpathlineto{\pgfqpoint{4.810931in}{0.757526in}}%
\pgfpathlineto{\pgfqpoint{4.837231in}{0.745768in}}%
\pgfpathlineto{\pgfqpoint{4.863532in}{0.735320in}}%
\pgfpathlineto{\pgfqpoint{4.889832in}{0.726062in}}%
\pgfpathlineto{\pgfqpoint{4.916132in}{0.717882in}}%
\pgfpathlineto{\pgfqpoint{4.942433in}{0.710676in}}%
\pgfpathlineto{\pgfqpoint{4.968733in}{0.704344in}}%
\pgfpathlineto{\pgfqpoint{4.995034in}{0.698797in}}%
\pgfpathlineto{\pgfqpoint{5.021334in}{0.693950in}}%
\pgfpathlineto{\pgfqpoint{5.047634in}{0.689728in}}%
\pgfpathlineto{\pgfqpoint{5.073935in}{0.686059in}}%
\pgfpathlineto{\pgfqpoint{5.100235in}{0.682880in}}%
\pgfpathlineto{\pgfqpoint{5.126535in}{0.680133in}}%
\pgfpathlineto{\pgfqpoint{5.152836in}{0.677767in}}%
\pgfpathlineto{\pgfqpoint{5.179136in}{0.675733in}}%
\pgfpathlineto{\pgfqpoint{5.205436in}{0.673990in}}%
\pgfpathlineto{\pgfqpoint{5.258037in}{0.671231in}}%
\pgfpathlineto{\pgfqpoint{5.310638in}{0.669237in}}%
\pgfpathlineto{\pgfqpoint{5.363239in}{0.667812in}}%
\pgfpathlineto{\pgfqpoint{5.415839in}{0.666804in}}%
\pgfpathlineto{\pgfqpoint{5.494740in}{0.665832in}}%
\pgfpathlineto{\pgfqpoint{5.599942in}{0.665152in}}%
\pgfpathlineto{\pgfqpoint{5.757744in}{0.664761in}}%
\pgfpathlineto{\pgfqpoint{6.020748in}{0.664629in}}%
\pgfpathlineto{\pgfqpoint{6.020748in}{0.664629in}}%
\pgfusepath{stroke}%
\end{pgfscope}%
\begin{pgfscope}%
\pgfpathrectangle{\pgfqpoint{0.552903in}{0.498088in}}{\pgfqpoint{5.728219in}{3.663896in}}%
\pgfusepath{clip}%
\pgfsetbuttcap%
\pgfsetroundjoin%
\definecolor{currentfill}{rgb}{0.121569,0.466667,0.705882}%
\pgfsetfillcolor{currentfill}%
\pgfsetlinewidth{1.003750pt}%
\definecolor{currentstroke}{rgb}{0.121569,0.466667,0.705882}%
\pgfsetstrokecolor{currentstroke}%
\pgfsetdash{}{0pt}%
\pgfsys@defobject{currentmarker}{\pgfqpoint{-0.013889in}{-0.013889in}}{\pgfqpoint{0.013889in}{0.013889in}}{%
\pgfpathmoveto{\pgfqpoint{0.000000in}{-0.013889in}}%
\pgfpathcurveto{\pgfqpoint{0.003683in}{-0.013889in}}{\pgfqpoint{0.007216in}{-0.012425in}}{\pgfqpoint{0.009821in}{-0.009821in}}%
\pgfpathcurveto{\pgfqpoint{0.012425in}{-0.007216in}}{\pgfqpoint{0.013889in}{-0.003683in}}{\pgfqpoint{0.013889in}{0.000000in}}%
\pgfpathcurveto{\pgfqpoint{0.013889in}{0.003683in}}{\pgfqpoint{0.012425in}{0.007216in}}{\pgfqpoint{0.009821in}{0.009821in}}%
\pgfpathcurveto{\pgfqpoint{0.007216in}{0.012425in}}{\pgfqpoint{0.003683in}{0.013889in}}{\pgfqpoint{0.000000in}{0.013889in}}%
\pgfpathcurveto{\pgfqpoint{-0.003683in}{0.013889in}}{\pgfqpoint{-0.007216in}{0.012425in}}{\pgfqpoint{-0.009821in}{0.009821in}}%
\pgfpathcurveto{\pgfqpoint{-0.012425in}{0.007216in}}{\pgfqpoint{-0.013889in}{0.003683in}}{\pgfqpoint{-0.013889in}{0.000000in}}%
\pgfpathcurveto{\pgfqpoint{-0.013889in}{-0.003683in}}{\pgfqpoint{-0.012425in}{-0.007216in}}{\pgfqpoint{-0.009821in}{-0.009821in}}%
\pgfpathcurveto{\pgfqpoint{-0.007216in}{-0.012425in}}{\pgfqpoint{-0.003683in}{-0.013889in}}{\pgfqpoint{0.000000in}{-0.013889in}}%
\pgfpathlineto{\pgfqpoint{0.000000in}{-0.013889in}}%
\pgfpathclose%
\pgfusepath{stroke,fill}%
}%
\begin{pgfscope}%
\pgfsys@transformshift{0.813276in}{0.665386in}%
\pgfsys@useobject{currentmarker}{}%
\end{pgfscope}%
\begin{pgfscope}%
\pgfsys@transformshift{0.839577in}{0.665472in}%
\pgfsys@useobject{currentmarker}{}%
\end{pgfscope}%
\begin{pgfscope}%
\pgfsys@transformshift{0.865877in}{0.665582in}%
\pgfsys@useobject{currentmarker}{}%
\end{pgfscope}%
\begin{pgfscope}%
\pgfsys@transformshift{0.892177in}{0.665685in}%
\pgfsys@useobject{currentmarker}{}%
\end{pgfscope}%
\begin{pgfscope}%
\pgfsys@transformshift{0.918478in}{0.665814in}%
\pgfsys@useobject{currentmarker}{}%
\end{pgfscope}%
\begin{pgfscope}%
\pgfsys@transformshift{0.944778in}{0.665963in}%
\pgfsys@useobject{currentmarker}{}%
\end{pgfscope}%
\begin{pgfscope}%
\pgfsys@transformshift{0.971078in}{0.666123in}%
\pgfsys@useobject{currentmarker}{}%
\end{pgfscope}%
\begin{pgfscope}%
\pgfsys@transformshift{0.997379in}{0.666295in}%
\pgfsys@useobject{currentmarker}{}%
\end{pgfscope}%
\begin{pgfscope}%
\pgfsys@transformshift{1.023679in}{0.666471in}%
\pgfsys@useobject{currentmarker}{}%
\end{pgfscope}%
\begin{pgfscope}%
\pgfsys@transformshift{1.049979in}{0.666715in}%
\pgfsys@useobject{currentmarker}{}%
\end{pgfscope}%
\begin{pgfscope}%
\pgfsys@transformshift{1.076280in}{0.666937in}%
\pgfsys@useobject{currentmarker}{}%
\end{pgfscope}%
\begin{pgfscope}%
\pgfsys@transformshift{1.102580in}{0.667237in}%
\pgfsys@useobject{currentmarker}{}%
\end{pgfscope}%
\begin{pgfscope}%
\pgfsys@transformshift{1.128881in}{0.667518in}%
\pgfsys@useobject{currentmarker}{}%
\end{pgfscope}%
\begin{pgfscope}%
\pgfsys@transformshift{1.155181in}{0.667857in}%
\pgfsys@useobject{currentmarker}{}%
\end{pgfscope}%
\begin{pgfscope}%
\pgfsys@transformshift{1.181481in}{0.668248in}%
\pgfsys@useobject{currentmarker}{}%
\end{pgfscope}%
\begin{pgfscope}%
\pgfsys@transformshift{1.207782in}{0.668657in}%
\pgfsys@useobject{currentmarker}{}%
\end{pgfscope}%
\begin{pgfscope}%
\pgfsys@transformshift{1.234082in}{0.669100in}%
\pgfsys@useobject{currentmarker}{}%
\end{pgfscope}%
\begin{pgfscope}%
\pgfsys@transformshift{1.260382in}{0.669637in}%
\pgfsys@useobject{currentmarker}{}%
\end{pgfscope}%
\begin{pgfscope}%
\pgfsys@transformshift{1.286683in}{0.670155in}%
\pgfsys@useobject{currentmarker}{}%
\end{pgfscope}%
\begin{pgfscope}%
\pgfsys@transformshift{1.312983in}{0.670849in}%
\pgfsys@useobject{currentmarker}{}%
\end{pgfscope}%
\begin{pgfscope}%
\pgfsys@transformshift{1.339283in}{0.671553in}%
\pgfsys@useobject{currentmarker}{}%
\end{pgfscope}%
\begin{pgfscope}%
\pgfsys@transformshift{1.365584in}{0.672350in}%
\pgfsys@useobject{currentmarker}{}%
\end{pgfscope}%
\begin{pgfscope}%
\pgfsys@transformshift{1.391884in}{0.673222in}%
\pgfsys@useobject{currentmarker}{}%
\end{pgfscope}%
\begin{pgfscope}%
\pgfsys@transformshift{1.418185in}{0.674203in}%
\pgfsys@useobject{currentmarker}{}%
\end{pgfscope}%
\begin{pgfscope}%
\pgfsys@transformshift{1.444485in}{0.675378in}%
\pgfsys@useobject{currentmarker}{}%
\end{pgfscope}%
\begin{pgfscope}%
\pgfsys@transformshift{1.470785in}{0.676535in}%
\pgfsys@useobject{currentmarker}{}%
\end{pgfscope}%
\begin{pgfscope}%
\pgfsys@transformshift{1.497086in}{0.678029in}%
\pgfsys@useobject{currentmarker}{}%
\end{pgfscope}%
\begin{pgfscope}%
\pgfsys@transformshift{1.523386in}{0.679537in}%
\pgfsys@useobject{currentmarker}{}%
\end{pgfscope}%
\begin{pgfscope}%
\pgfsys@transformshift{1.549686in}{0.681162in}%
\pgfsys@useobject{currentmarker}{}%
\end{pgfscope}%
\begin{pgfscope}%
\pgfsys@transformshift{1.575987in}{0.683145in}%
\pgfsys@useobject{currentmarker}{}%
\end{pgfscope}%
\begin{pgfscope}%
\pgfsys@transformshift{1.602287in}{0.685349in}%
\pgfsys@useobject{currentmarker}{}%
\end{pgfscope}%
\begin{pgfscope}%
\pgfsys@transformshift{1.628587in}{0.687839in}%
\pgfsys@useobject{currentmarker}{}%
\end{pgfscope}%
\begin{pgfscope}%
\pgfsys@transformshift{1.654888in}{0.690563in}%
\pgfsys@useobject{currentmarker}{}%
\end{pgfscope}%
\begin{pgfscope}%
\pgfsys@transformshift{1.681188in}{0.693662in}%
\pgfsys@useobject{currentmarker}{}%
\end{pgfscope}%
\begin{pgfscope}%
\pgfsys@transformshift{1.707488in}{0.697046in}%
\pgfsys@useobject{currentmarker}{}%
\end{pgfscope}%
\begin{pgfscope}%
\pgfsys@transformshift{1.733789in}{0.700901in}%
\pgfsys@useobject{currentmarker}{}%
\end{pgfscope}%
\begin{pgfscope}%
\pgfsys@transformshift{1.760089in}{0.705296in}%
\pgfsys@useobject{currentmarker}{}%
\end{pgfscope}%
\begin{pgfscope}%
\pgfsys@transformshift{1.786390in}{0.710255in}%
\pgfsys@useobject{currentmarker}{}%
\end{pgfscope}%
\begin{pgfscope}%
\pgfsys@transformshift{1.812690in}{0.715592in}%
\pgfsys@useobject{currentmarker}{}%
\end{pgfscope}%
\begin{pgfscope}%
\pgfsys@transformshift{1.838990in}{0.721563in}%
\pgfsys@useobject{currentmarker}{}%
\end{pgfscope}%
\begin{pgfscope}%
\pgfsys@transformshift{1.865291in}{0.728446in}%
\pgfsys@useobject{currentmarker}{}%
\end{pgfscope}%
\begin{pgfscope}%
\pgfsys@transformshift{1.891591in}{0.735828in}%
\pgfsys@useobject{currentmarker}{}%
\end{pgfscope}%
\begin{pgfscope}%
\pgfsys@transformshift{1.917891in}{0.744298in}%
\pgfsys@useobject{currentmarker}{}%
\end{pgfscope}%
\begin{pgfscope}%
\pgfsys@transformshift{1.944192in}{0.753828in}%
\pgfsys@useobject{currentmarker}{}%
\end{pgfscope}%
\begin{pgfscope}%
\pgfsys@transformshift{1.970492in}{0.764014in}%
\pgfsys@useobject{currentmarker}{}%
\end{pgfscope}%
\begin{pgfscope}%
\pgfsys@transformshift{1.996792in}{0.775853in}%
\pgfsys@useobject{currentmarker}{}%
\end{pgfscope}%
\begin{pgfscope}%
\pgfsys@transformshift{2.023093in}{0.788676in}%
\pgfsys@useobject{currentmarker}{}%
\end{pgfscope}%
\begin{pgfscope}%
\pgfsys@transformshift{2.049393in}{0.802850in}%
\pgfsys@useobject{currentmarker}{}%
\end{pgfscope}%
\begin{pgfscope}%
\pgfsys@transformshift{2.075694in}{0.818678in}%
\pgfsys@useobject{currentmarker}{}%
\end{pgfscope}%
\begin{pgfscope}%
\pgfsys@transformshift{2.101994in}{0.835895in}%
\pgfsys@useobject{currentmarker}{}%
\end{pgfscope}%
\begin{pgfscope}%
\pgfsys@transformshift{2.128294in}{0.855733in}%
\pgfsys@useobject{currentmarker}{}%
\end{pgfscope}%
\begin{pgfscope}%
\pgfsys@transformshift{2.154595in}{0.876878in}%
\pgfsys@useobject{currentmarker}{}%
\end{pgfscope}%
\begin{pgfscope}%
\pgfsys@transformshift{2.180895in}{0.900163in}%
\pgfsys@useobject{currentmarker}{}%
\end{pgfscope}%
\begin{pgfscope}%
\pgfsys@transformshift{2.207195in}{0.925356in}%
\pgfsys@useobject{currentmarker}{}%
\end{pgfscope}%
\begin{pgfscope}%
\pgfsys@transformshift{2.233496in}{0.953520in}%
\pgfsys@useobject{currentmarker}{}%
\end{pgfscope}%
\begin{pgfscope}%
\pgfsys@transformshift{2.259796in}{0.984152in}%
\pgfsys@useobject{currentmarker}{}%
\end{pgfscope}%
\begin{pgfscope}%
\pgfsys@transformshift{2.286096in}{1.017209in}%
\pgfsys@useobject{currentmarker}{}%
\end{pgfscope}%
\begin{pgfscope}%
\pgfsys@transformshift{2.312397in}{1.052957in}%
\pgfsys@useobject{currentmarker}{}%
\end{pgfscope}%
\begin{pgfscope}%
\pgfsys@transformshift{2.338697in}{1.092255in}%
\pgfsys@useobject{currentmarker}{}%
\end{pgfscope}%
\begin{pgfscope}%
\pgfsys@transformshift{2.364997in}{1.134291in}%
\pgfsys@useobject{currentmarker}{}%
\end{pgfscope}%
\begin{pgfscope}%
\pgfsys@transformshift{2.391298in}{1.179891in}%
\pgfsys@useobject{currentmarker}{}%
\end{pgfscope}%
\begin{pgfscope}%
\pgfsys@transformshift{2.417598in}{1.228387in}%
\pgfsys@useobject{currentmarker}{}%
\end{pgfscope}%
\begin{pgfscope}%
\pgfsys@transformshift{2.443899in}{1.280934in}%
\pgfsys@useobject{currentmarker}{}%
\end{pgfscope}%
\begin{pgfscope}%
\pgfsys@transformshift{2.470199in}{1.336736in}%
\pgfsys@useobject{currentmarker}{}%
\end{pgfscope}%
\begin{pgfscope}%
\pgfsys@transformshift{2.496499in}{1.396479in}%
\pgfsys@useobject{currentmarker}{}%
\end{pgfscope}%
\begin{pgfscope}%
\pgfsys@transformshift{2.522800in}{1.459941in}%
\pgfsys@useobject{currentmarker}{}%
\end{pgfscope}%
\begin{pgfscope}%
\pgfsys@transformshift{2.549100in}{1.527370in}%
\pgfsys@useobject{currentmarker}{}%
\end{pgfscope}%
\begin{pgfscope}%
\pgfsys@transformshift{2.575400in}{1.597816in}%
\pgfsys@useobject{currentmarker}{}%
\end{pgfscope}%
\begin{pgfscope}%
\pgfsys@transformshift{2.601701in}{1.671908in}%
\pgfsys@useobject{currentmarker}{}%
\end{pgfscope}%
\begin{pgfscope}%
\pgfsys@transformshift{2.628001in}{1.750353in}%
\pgfsys@useobject{currentmarker}{}%
\end{pgfscope}%
\begin{pgfscope}%
\pgfsys@transformshift{2.654301in}{1.831523in}%
\pgfsys@useobject{currentmarker}{}%
\end{pgfscope}%
\begin{pgfscope}%
\pgfsys@transformshift{2.680602in}{1.917427in}%
\pgfsys@useobject{currentmarker}{}%
\end{pgfscope}%
\begin{pgfscope}%
\pgfsys@transformshift{2.706902in}{2.005112in}%
\pgfsys@useobject{currentmarker}{}%
\end{pgfscope}%
\begin{pgfscope}%
\pgfsys@transformshift{2.733203in}{2.096272in}%
\pgfsys@useobject{currentmarker}{}%
\end{pgfscope}%
\begin{pgfscope}%
\pgfsys@transformshift{2.759503in}{2.189114in}%
\pgfsys@useobject{currentmarker}{}%
\end{pgfscope}%
\begin{pgfscope}%
\pgfsys@transformshift{2.785803in}{2.285433in}%
\pgfsys@useobject{currentmarker}{}%
\end{pgfscope}%
\begin{pgfscope}%
\pgfsys@transformshift{2.812104in}{2.384058in}%
\pgfsys@useobject{currentmarker}{}%
\end{pgfscope}%
\begin{pgfscope}%
\pgfsys@transformshift{2.838404in}{2.483204in}%
\pgfsys@useobject{currentmarker}{}%
\end{pgfscope}%
\begin{pgfscope}%
\pgfsys@transformshift{2.864704in}{2.584371in}%
\pgfsys@useobject{currentmarker}{}%
\end{pgfscope}%
\begin{pgfscope}%
\pgfsys@transformshift{2.891005in}{2.685582in}%
\pgfsys@useobject{currentmarker}{}%
\end{pgfscope}%
\begin{pgfscope}%
\pgfsys@transformshift{2.917305in}{2.787243in}%
\pgfsys@useobject{currentmarker}{}%
\end{pgfscope}%
\begin{pgfscope}%
\pgfsys@transformshift{2.943605in}{2.888610in}%
\pgfsys@useobject{currentmarker}{}%
\end{pgfscope}%
\begin{pgfscope}%
\pgfsys@transformshift{2.969906in}{2.988881in}%
\pgfsys@useobject{currentmarker}{}%
\end{pgfscope}%
\begin{pgfscope}%
\pgfsys@transformshift{2.996206in}{3.086884in}%
\pgfsys@useobject{currentmarker}{}%
\end{pgfscope}%
\begin{pgfscope}%
\pgfsys@transformshift{3.022507in}{3.183227in}%
\pgfsys@useobject{currentmarker}{}%
\end{pgfscope}%
\begin{pgfscope}%
\pgfsys@transformshift{3.048807in}{3.277491in}%
\pgfsys@useobject{currentmarker}{}%
\end{pgfscope}%
\begin{pgfscope}%
\pgfsys@transformshift{3.075107in}{3.367602in}%
\pgfsys@useobject{currentmarker}{}%
\end{pgfscope}%
\begin{pgfscope}%
\pgfsys@transformshift{3.101408in}{3.454426in}%
\pgfsys@useobject{currentmarker}{}%
\end{pgfscope}%
\begin{pgfscope}%
\pgfsys@transformshift{3.127708in}{3.535811in}%
\pgfsys@useobject{currentmarker}{}%
\end{pgfscope}%
\begin{pgfscope}%
\pgfsys@transformshift{3.154008in}{3.612036in}%
\pgfsys@useobject{currentmarker}{}%
\end{pgfscope}%
\begin{pgfscope}%
\pgfsys@transformshift{3.180309in}{3.683957in}%
\pgfsys@useobject{currentmarker}{}%
\end{pgfscope}%
\begin{pgfscope}%
\pgfsys@transformshift{3.206609in}{3.749482in}%
\pgfsys@useobject{currentmarker}{}%
\end{pgfscope}%
\begin{pgfscope}%
\pgfsys@transformshift{3.232909in}{3.806773in}%
\pgfsys@useobject{currentmarker}{}%
\end{pgfscope}%
\begin{pgfscope}%
\pgfsys@transformshift{3.259210in}{3.856313in}%
\pgfsys@useobject{currentmarker}{}%
\end{pgfscope}%
\begin{pgfscope}%
\pgfsys@transformshift{3.285510in}{3.901281in}%
\pgfsys@useobject{currentmarker}{}%
\end{pgfscope}%
\begin{pgfscope}%
\pgfsys@transformshift{3.311810in}{3.937378in}%
\pgfsys@useobject{currentmarker}{}%
\end{pgfscope}%
\begin{pgfscope}%
\pgfsys@transformshift{3.338111in}{3.963248in}%
\pgfsys@useobject{currentmarker}{}%
\end{pgfscope}%
\begin{pgfscope}%
\pgfsys@transformshift{3.364411in}{3.983944in}%
\pgfsys@useobject{currentmarker}{}%
\end{pgfscope}%
\begin{pgfscope}%
\pgfsys@transformshift{3.390712in}{3.991713in}%
\pgfsys@useobject{currentmarker}{}%
\end{pgfscope}%
\begin{pgfscope}%
\pgfsys@transformshift{3.417012in}{3.995443in}%
\pgfsys@useobject{currentmarker}{}%
\end{pgfscope}%
\begin{pgfscope}%
\pgfsys@transformshift{3.443312in}{3.985736in}%
\pgfsys@useobject{currentmarker}{}%
\end{pgfscope}%
\begin{pgfscope}%
\pgfsys@transformshift{3.469613in}{3.969253in}%
\pgfsys@useobject{currentmarker}{}%
\end{pgfscope}%
\begin{pgfscope}%
\pgfsys@transformshift{3.495913in}{3.945575in}%
\pgfsys@useobject{currentmarker}{}%
\end{pgfscope}%
\begin{pgfscope}%
\pgfsys@transformshift{3.522213in}{3.914473in}%
\pgfsys@useobject{currentmarker}{}%
\end{pgfscope}%
\begin{pgfscope}%
\pgfsys@transformshift{3.548514in}{3.872129in}%
\pgfsys@useobject{currentmarker}{}%
\end{pgfscope}%
\begin{pgfscope}%
\pgfsys@transformshift{3.574814in}{3.823765in}%
\pgfsys@useobject{currentmarker}{}%
\end{pgfscope}%
\begin{pgfscope}%
\pgfsys@transformshift{3.601114in}{3.769346in}%
\pgfsys@useobject{currentmarker}{}%
\end{pgfscope}%
\begin{pgfscope}%
\pgfsys@transformshift{3.627415in}{3.706666in}%
\pgfsys@useobject{currentmarker}{}%
\end{pgfscope}%
\begin{pgfscope}%
\pgfsys@transformshift{3.653715in}{3.637972in}%
\pgfsys@useobject{currentmarker}{}%
\end{pgfscope}%
\begin{pgfscope}%
\pgfsys@transformshift{3.680016in}{3.561853in}%
\pgfsys@useobject{currentmarker}{}%
\end{pgfscope}%
\begin{pgfscope}%
\pgfsys@transformshift{3.706316in}{3.481386in}%
\pgfsys@useobject{currentmarker}{}%
\end{pgfscope}%
\begin{pgfscope}%
\pgfsys@transformshift{3.732616in}{3.397136in}%
\pgfsys@useobject{currentmarker}{}%
\end{pgfscope}%
\begin{pgfscope}%
\pgfsys@transformshift{3.758917in}{3.308196in}%
\pgfsys@useobject{currentmarker}{}%
\end{pgfscope}%
\begin{pgfscope}%
\pgfsys@transformshift{3.785217in}{3.215823in}%
\pgfsys@useobject{currentmarker}{}%
\end{pgfscope}%
\begin{pgfscope}%
\pgfsys@transformshift{3.811517in}{3.120533in}%
\pgfsys@useobject{currentmarker}{}%
\end{pgfscope}%
\begin{pgfscope}%
\pgfsys@transformshift{3.837818in}{3.022949in}%
\pgfsys@useobject{currentmarker}{}%
\end{pgfscope}%
\begin{pgfscope}%
\pgfsys@transformshift{3.864118in}{2.921939in}%
\pgfsys@useobject{currentmarker}{}%
\end{pgfscope}%
\begin{pgfscope}%
\pgfsys@transformshift{3.890418in}{2.821600in}%
\pgfsys@useobject{currentmarker}{}%
\end{pgfscope}%
\begin{pgfscope}%
\pgfsys@transformshift{3.916719in}{2.720455in}%
\pgfsys@useobject{currentmarker}{}%
\end{pgfscope}%
\begin{pgfscope}%
\pgfsys@transformshift{3.943019in}{2.619738in}%
\pgfsys@useobject{currentmarker}{}%
\end{pgfscope}%
\begin{pgfscope}%
\pgfsys@transformshift{3.969320in}{2.517694in}%
\pgfsys@useobject{currentmarker}{}%
\end{pgfscope}%
\begin{pgfscope}%
\pgfsys@transformshift{3.995620in}{2.418771in}%
\pgfsys@useobject{currentmarker}{}%
\end{pgfscope}%
\begin{pgfscope}%
\pgfsys@transformshift{4.021920in}{2.319815in}%
\pgfsys@useobject{currentmarker}{}%
\end{pgfscope}%
\begin{pgfscope}%
\pgfsys@transformshift{4.048221in}{2.223646in}%
\pgfsys@useobject{currentmarker}{}%
\end{pgfscope}%
\begin{pgfscope}%
\pgfsys@transformshift{4.074521in}{2.129304in}%
\pgfsys@useobject{currentmarker}{}%
\end{pgfscope}%
\begin{pgfscope}%
\pgfsys@transformshift{4.100821in}{2.036524in}%
\pgfsys@useobject{currentmarker}{}%
\end{pgfscope}%
\begin{pgfscope}%
\pgfsys@transformshift{4.127122in}{1.948470in}%
\pgfsys@useobject{currentmarker}{}%
\end{pgfscope}%
\begin{pgfscope}%
\pgfsys@transformshift{4.153422in}{1.862652in}%
\pgfsys@useobject{currentmarker}{}%
\end{pgfscope}%
\begin{pgfscope}%
\pgfsys@transformshift{4.179722in}{1.780003in}%
\pgfsys@useobject{currentmarker}{}%
\end{pgfscope}%
\begin{pgfscope}%
\pgfsys@transformshift{4.206023in}{1.700322in}%
\pgfsys@useobject{currentmarker}{}%
\end{pgfscope}%
\begin{pgfscope}%
\pgfsys@transformshift{4.232323in}{1.624954in}%
\pgfsys@useobject{currentmarker}{}%
\end{pgfscope}%
\begin{pgfscope}%
\pgfsys@transformshift{4.258623in}{1.553312in}%
\pgfsys@useobject{currentmarker}{}%
\end{pgfscope}%
\begin{pgfscope}%
\pgfsys@transformshift{4.284924in}{1.484700in}%
\pgfsys@useobject{currentmarker}{}%
\end{pgfscope}%
\begin{pgfscope}%
\pgfsys@transformshift{4.311224in}{1.420156in}%
\pgfsys@useobject{currentmarker}{}%
\end{pgfscope}%
\begin{pgfscope}%
\pgfsys@transformshift{4.337525in}{1.359169in}%
\pgfsys@useobject{currentmarker}{}%
\end{pgfscope}%
\begin{pgfscope}%
\pgfsys@transformshift{4.363825in}{1.302138in}%
\pgfsys@useobject{currentmarker}{}%
\end{pgfscope}%
\begin{pgfscope}%
\pgfsys@transformshift{4.390125in}{1.248332in}%
\pgfsys@useobject{currentmarker}{}%
\end{pgfscope}%
\begin{pgfscope}%
\pgfsys@transformshift{4.416426in}{1.198375in}%
\pgfsys@useobject{currentmarker}{}%
\end{pgfscope}%
\begin{pgfscope}%
\pgfsys@transformshift{4.442726in}{1.151941in}%
\pgfsys@useobject{currentmarker}{}%
\end{pgfscope}%
\begin{pgfscope}%
\pgfsys@transformshift{4.469026in}{1.108756in}%
\pgfsys@useobject{currentmarker}{}%
\end{pgfscope}%
\begin{pgfscope}%
\pgfsys@transformshift{4.495327in}{1.068082in}%
\pgfsys@useobject{currentmarker}{}%
\end{pgfscope}%
\begin{pgfscope}%
\pgfsys@transformshift{4.521627in}{1.030942in}%
\pgfsys@useobject{currentmarker}{}%
\end{pgfscope}%
\begin{pgfscope}%
\pgfsys@transformshift{4.547927in}{0.997214in}%
\pgfsys@useobject{currentmarker}{}%
\end{pgfscope}%
\begin{pgfscope}%
\pgfsys@transformshift{4.574228in}{0.965304in}%
\pgfsys@useobject{currentmarker}{}%
\end{pgfscope}%
\begin{pgfscope}%
\pgfsys@transformshift{4.600528in}{0.936383in}%
\pgfsys@useobject{currentmarker}{}%
\end{pgfscope}%
\begin{pgfscope}%
\pgfsys@transformshift{4.626829in}{0.910142in}%
\pgfsys@useobject{currentmarker}{}%
\end{pgfscope}%
\begin{pgfscope}%
\pgfsys@transformshift{4.653129in}{0.886002in}%
\pgfsys@useobject{currentmarker}{}%
\end{pgfscope}%
\begin{pgfscope}%
\pgfsys@transformshift{4.679429in}{0.864307in}%
\pgfsys@useobject{currentmarker}{}%
\end{pgfscope}%
\begin{pgfscope}%
\pgfsys@transformshift{4.705730in}{0.843990in}%
\pgfsys@useobject{currentmarker}{}%
\end{pgfscope}%
\begin{pgfscope}%
\pgfsys@transformshift{4.732030in}{0.825652in}%
\pgfsys@useobject{currentmarker}{}%
\end{pgfscope}%
\begin{pgfscope}%
\pgfsys@transformshift{4.758330in}{0.809426in}%
\pgfsys@useobject{currentmarker}{}%
\end{pgfscope}%
\begin{pgfscope}%
\pgfsys@transformshift{4.784631in}{0.794500in}%
\pgfsys@useobject{currentmarker}{}%
\end{pgfscope}%
\begin{pgfscope}%
\pgfsys@transformshift{4.810931in}{0.781131in}%
\pgfsys@useobject{currentmarker}{}%
\end{pgfscope}%
\begin{pgfscope}%
\pgfsys@transformshift{4.837231in}{0.768853in}%
\pgfsys@useobject{currentmarker}{}%
\end{pgfscope}%
\begin{pgfscope}%
\pgfsys@transformshift{4.863532in}{0.757948in}%
\pgfsys@useobject{currentmarker}{}%
\end{pgfscope}%
\begin{pgfscope}%
\pgfsys@transformshift{4.889832in}{0.748308in}%
\pgfsys@useobject{currentmarker}{}%
\end{pgfscope}%
\begin{pgfscope}%
\pgfsys@transformshift{4.916132in}{0.739260in}%
\pgfsys@useobject{currentmarker}{}%
\end{pgfscope}%
\begin{pgfscope}%
\pgfsys@transformshift{4.942433in}{0.731494in}%
\pgfsys@useobject{currentmarker}{}%
\end{pgfscope}%
\begin{pgfscope}%
\pgfsys@transformshift{4.968733in}{0.724568in}%
\pgfsys@useobject{currentmarker}{}%
\end{pgfscope}%
\begin{pgfscope}%
\pgfsys@transformshift{4.995034in}{0.718016in}%
\pgfsys@useobject{currentmarker}{}%
\end{pgfscope}%
\begin{pgfscope}%
\pgfsys@transformshift{5.021334in}{0.712346in}%
\pgfsys@useobject{currentmarker}{}%
\end{pgfscope}%
\begin{pgfscope}%
\pgfsys@transformshift{5.047634in}{0.707315in}%
\pgfsys@useobject{currentmarker}{}%
\end{pgfscope}%
\begin{pgfscope}%
\pgfsys@transformshift{5.073935in}{0.702933in}%
\pgfsys@useobject{currentmarker}{}%
\end{pgfscope}%
\begin{pgfscope}%
\pgfsys@transformshift{5.100235in}{0.698599in}%
\pgfsys@useobject{currentmarker}{}%
\end{pgfscope}%
\begin{pgfscope}%
\pgfsys@transformshift{5.126535in}{0.695086in}%
\pgfsys@useobject{currentmarker}{}%
\end{pgfscope}%
\begin{pgfscope}%
\pgfsys@transformshift{5.152836in}{0.691778in}%
\pgfsys@useobject{currentmarker}{}%
\end{pgfscope}%
\begin{pgfscope}%
\pgfsys@transformshift{5.179136in}{0.688908in}%
\pgfsys@useobject{currentmarker}{}%
\end{pgfscope}%
\begin{pgfscope}%
\pgfsys@transformshift{5.205436in}{0.686318in}%
\pgfsys@useobject{currentmarker}{}%
\end{pgfscope}%
\begin{pgfscope}%
\pgfsys@transformshift{5.231737in}{0.683986in}%
\pgfsys@useobject{currentmarker}{}%
\end{pgfscope}%
\begin{pgfscope}%
\pgfsys@transformshift{5.258037in}{0.682028in}%
\pgfsys@useobject{currentmarker}{}%
\end{pgfscope}%
\begin{pgfscope}%
\pgfsys@transformshift{5.284338in}{0.680210in}%
\pgfsys@useobject{currentmarker}{}%
\end{pgfscope}%
\begin{pgfscope}%
\pgfsys@transformshift{5.310638in}{0.678492in}%
\pgfsys@useobject{currentmarker}{}%
\end{pgfscope}%
\begin{pgfscope}%
\pgfsys@transformshift{5.336938in}{0.677058in}%
\pgfsys@useobject{currentmarker}{}%
\end{pgfscope}%
\begin{pgfscope}%
\pgfsys@transformshift{5.363239in}{0.675754in}%
\pgfsys@useobject{currentmarker}{}%
\end{pgfscope}%
\begin{pgfscope}%
\pgfsys@transformshift{5.389539in}{0.674647in}%
\pgfsys@useobject{currentmarker}{}%
\end{pgfscope}%
\begin{pgfscope}%
\pgfsys@transformshift{5.415839in}{0.673608in}%
\pgfsys@useobject{currentmarker}{}%
\end{pgfscope}%
\begin{pgfscope}%
\pgfsys@transformshift{5.442140in}{0.672642in}%
\pgfsys@useobject{currentmarker}{}%
\end{pgfscope}%
\begin{pgfscope}%
\pgfsys@transformshift{5.468440in}{0.671762in}%
\pgfsys@useobject{currentmarker}{}%
\end{pgfscope}%
\begin{pgfscope}%
\pgfsys@transformshift{5.494740in}{0.671029in}%
\pgfsys@useobject{currentmarker}{}%
\end{pgfscope}%
\begin{pgfscope}%
\pgfsys@transformshift{5.521041in}{0.670367in}%
\pgfsys@useobject{currentmarker}{}%
\end{pgfscope}%
\begin{pgfscope}%
\pgfsys@transformshift{5.547341in}{0.669765in}%
\pgfsys@useobject{currentmarker}{}%
\end{pgfscope}%
\begin{pgfscope}%
\pgfsys@transformshift{5.573642in}{0.669188in}%
\pgfsys@useobject{currentmarker}{}%
\end{pgfscope}%
\begin{pgfscope}%
\pgfsys@transformshift{5.599942in}{0.668780in}%
\pgfsys@useobject{currentmarker}{}%
\end{pgfscope}%
\begin{pgfscope}%
\pgfsys@transformshift{5.626242in}{0.668331in}%
\pgfsys@useobject{currentmarker}{}%
\end{pgfscope}%
\begin{pgfscope}%
\pgfsys@transformshift{5.652543in}{0.667921in}%
\pgfsys@useobject{currentmarker}{}%
\end{pgfscope}%
\begin{pgfscope}%
\pgfsys@transformshift{5.678843in}{0.667581in}%
\pgfsys@useobject{currentmarker}{}%
\end{pgfscope}%
\begin{pgfscope}%
\pgfsys@transformshift{5.705143in}{0.667291in}%
\pgfsys@useobject{currentmarker}{}%
\end{pgfscope}%
\begin{pgfscope}%
\pgfsys@transformshift{5.731444in}{0.666993in}%
\pgfsys@useobject{currentmarker}{}%
\end{pgfscope}%
\begin{pgfscope}%
\pgfsys@transformshift{5.757744in}{0.666740in}%
\pgfsys@useobject{currentmarker}{}%
\end{pgfscope}%
\begin{pgfscope}%
\pgfsys@transformshift{5.784044in}{0.666532in}%
\pgfsys@useobject{currentmarker}{}%
\end{pgfscope}%
\begin{pgfscope}%
\pgfsys@transformshift{5.810345in}{0.666314in}%
\pgfsys@useobject{currentmarker}{}%
\end{pgfscope}%
\begin{pgfscope}%
\pgfsys@transformshift{5.836645in}{0.666148in}%
\pgfsys@useobject{currentmarker}{}%
\end{pgfscope}%
\begin{pgfscope}%
\pgfsys@transformshift{5.862945in}{0.665956in}%
\pgfsys@useobject{currentmarker}{}%
\end{pgfscope}%
\begin{pgfscope}%
\pgfsys@transformshift{5.889246in}{0.665831in}%
\pgfsys@useobject{currentmarker}{}%
\end{pgfscope}%
\begin{pgfscope}%
\pgfsys@transformshift{5.915546in}{0.665685in}%
\pgfsys@useobject{currentmarker}{}%
\end{pgfscope}%
\begin{pgfscope}%
\pgfsys@transformshift{5.941847in}{0.665558in}%
\pgfsys@useobject{currentmarker}{}%
\end{pgfscope}%
\begin{pgfscope}%
\pgfsys@transformshift{5.968147in}{0.665487in}%
\pgfsys@useobject{currentmarker}{}%
\end{pgfscope}%
\begin{pgfscope}%
\pgfsys@transformshift{5.994447in}{0.665373in}%
\pgfsys@useobject{currentmarker}{}%
\end{pgfscope}%
\begin{pgfscope}%
\pgfsys@transformshift{6.020748in}{0.665295in}%
\pgfsys@useobject{currentmarker}{}%
\end{pgfscope}%
\end{pgfscope}%
\begin{pgfscope}%
\pgfsetrectcap%
\pgfsetmiterjoin%
\pgfsetlinewidth{0.803000pt}%
\definecolor{currentstroke}{rgb}{0.000000,0.000000,0.000000}%
\pgfsetstrokecolor{currentstroke}%
\pgfsetdash{}{0pt}%
\pgfpathmoveto{\pgfqpoint{0.552903in}{0.498088in}}%
\pgfpathlineto{\pgfqpoint{0.552903in}{4.161984in}}%
\pgfusepath{stroke}%
\end{pgfscope}%
\begin{pgfscope}%
\pgfsetrectcap%
\pgfsetmiterjoin%
\pgfsetlinewidth{0.803000pt}%
\definecolor{currentstroke}{rgb}{0.000000,0.000000,0.000000}%
\pgfsetstrokecolor{currentstroke}%
\pgfsetdash{}{0pt}%
\pgfpathmoveto{\pgfqpoint{6.281121in}{0.498088in}}%
\pgfpathlineto{\pgfqpoint{6.281121in}{4.161984in}}%
\pgfusepath{stroke}%
\end{pgfscope}%
\begin{pgfscope}%
\pgfsetrectcap%
\pgfsetmiterjoin%
\pgfsetlinewidth{0.803000pt}%
\definecolor{currentstroke}{rgb}{0.000000,0.000000,0.000000}%
\pgfsetstrokecolor{currentstroke}%
\pgfsetdash{}{0pt}%
\pgfpathmoveto{\pgfqpoint{0.552903in}{0.498088in}}%
\pgfpathlineto{\pgfqpoint{6.281121in}{0.498088in}}%
\pgfusepath{stroke}%
\end{pgfscope}%
\begin{pgfscope}%
\pgfsetrectcap%
\pgfsetmiterjoin%
\pgfsetlinewidth{0.803000pt}%
\definecolor{currentstroke}{rgb}{0.000000,0.000000,0.000000}%
\pgfsetstrokecolor{currentstroke}%
\pgfsetdash{}{0pt}%
\pgfpathmoveto{\pgfqpoint{0.552903in}{4.161984in}}%
\pgfpathlineto{\pgfqpoint{6.281121in}{4.161984in}}%
\pgfusepath{stroke}%
\end{pgfscope}%
\begin{pgfscope}%
\pgfsetbuttcap%
\pgfsetmiterjoin%
\definecolor{currentfill}{rgb}{1.000000,1.000000,1.000000}%
\pgfsetfillcolor{currentfill}%
\pgfsetfillopacity{0.800000}%
\pgfsetlinewidth{1.003750pt}%
\definecolor{currentstroke}{rgb}{0.800000,0.800000,0.800000}%
\pgfsetstrokecolor{currentstroke}%
\pgfsetstrokeopacity{0.800000}%
\pgfsetdash{}{0pt}%
\pgfpathmoveto{\pgfqpoint{4.199421in}{3.345663in}}%
\pgfpathlineto{\pgfqpoint{6.183899in}{3.345663in}}%
\pgfpathquadraticcurveto{\pgfqpoint{6.211677in}{3.345663in}}{\pgfqpoint{6.211677in}{3.373440in}}%
\pgfpathlineto{\pgfqpoint{6.211677in}{4.064762in}}%
\pgfpathquadraticcurveto{\pgfqpoint{6.211677in}{4.092540in}}{\pgfqpoint{6.183899in}{4.092540in}}%
\pgfpathlineto{\pgfqpoint{4.199421in}{4.092540in}}%
\pgfpathquadraticcurveto{\pgfqpoint{4.171643in}{4.092540in}}{\pgfqpoint{4.171643in}{4.064762in}}%
\pgfpathlineto{\pgfqpoint{4.171643in}{3.373440in}}%
\pgfpathquadraticcurveto{\pgfqpoint{4.171643in}{3.345663in}}{\pgfqpoint{4.199421in}{3.345663in}}%
\pgfpathlineto{\pgfqpoint{4.199421in}{3.345663in}}%
\pgfpathclose%
\pgfusepath{stroke,fill}%
\end{pgfscope}%
\begin{pgfscope}%
\pgfsetrectcap%
\pgfsetroundjoin%
\pgfsetlinewidth{1.505625pt}%
\definecolor{currentstroke}{rgb}{1.000000,0.498039,0.054902}%
\pgfsetstrokecolor{currentstroke}%
\pgfsetdash{}{0pt}%
\pgfpathmoveto{\pgfqpoint{4.227198in}{3.816718in}}%
\pgfpathlineto{\pgfqpoint{4.366087in}{3.816718in}}%
\pgfpathlineto{\pgfqpoint{4.504976in}{3.816718in}}%
\pgfusepath{stroke}%
\end{pgfscope}%
\begin{pgfscope}%
\definecolor{textcolor}{rgb}{0.000000,0.000000,0.000000}%
\pgfsetstrokecolor{textcolor}%
\pgfsetfillcolor{textcolor}%
\pgftext[x=4.616087in, y=3.941328in, left, base]{\color{textcolor}\rmfamily\fontsize{10.000000}{12.000000}\selectfont Gauß-Anpassung mit}%
\end{pgfscope}%
\begin{pgfscope}%
\definecolor{textcolor}{rgb}{0.000000,0.000000,0.000000}%
\pgfsetstrokecolor{textcolor}%
\pgfsetfillcolor{textcolor}%
\pgftext[x=4.616087in, y=3.789329in, left, base]{\color{textcolor}\rmfamily\fontsize{10.000000}{12.000000}\selectfont  \(\displaystyle \mu\) = \SI{-0.32(2)}{ADU},}%
\end{pgfscope}%
\begin{pgfscope}%
\definecolor{textcolor}{rgb}{0.000000,0.000000,0.000000}%
\pgfsetstrokecolor{textcolor}%
\pgfsetfillcolor{textcolor}%
\pgftext[x=4.616087in, y=3.629607in, left, base]{\color{textcolor}\rmfamily\fontsize{10.000000}{12.000000}\selectfont \(\displaystyle \sigma = \SI{19.94(2)}{ADU}\)}%
\end{pgfscope}%
\begin{pgfscope}%
\pgfsetbuttcap%
\pgfsetroundjoin%
\definecolor{currentfill}{rgb}{0.121569,0.466667,0.705882}%
\pgfsetfillcolor{currentfill}%
\pgfsetlinewidth{1.003750pt}%
\definecolor{currentstroke}{rgb}{0.121569,0.466667,0.705882}%
\pgfsetstrokecolor{currentstroke}%
\pgfsetdash{}{0pt}%
\pgfsys@defobject{currentmarker}{\pgfqpoint{-0.013889in}{-0.013889in}}{\pgfqpoint{0.013889in}{0.013889in}}{%
\pgfpathmoveto{\pgfqpoint{0.000000in}{-0.013889in}}%
\pgfpathcurveto{\pgfqpoint{0.003683in}{-0.013889in}}{\pgfqpoint{0.007216in}{-0.012425in}}{\pgfqpoint{0.009821in}{-0.009821in}}%
\pgfpathcurveto{\pgfqpoint{0.012425in}{-0.007216in}}{\pgfqpoint{0.013889in}{-0.003683in}}{\pgfqpoint{0.013889in}{0.000000in}}%
\pgfpathcurveto{\pgfqpoint{0.013889in}{0.003683in}}{\pgfqpoint{0.012425in}{0.007216in}}{\pgfqpoint{0.009821in}{0.009821in}}%
\pgfpathcurveto{\pgfqpoint{0.007216in}{0.012425in}}{\pgfqpoint{0.003683in}{0.013889in}}{\pgfqpoint{0.000000in}{0.013889in}}%
\pgfpathcurveto{\pgfqpoint{-0.003683in}{0.013889in}}{\pgfqpoint{-0.007216in}{0.012425in}}{\pgfqpoint{-0.009821in}{0.009821in}}%
\pgfpathcurveto{\pgfqpoint{-0.012425in}{0.007216in}}{\pgfqpoint{-0.013889in}{0.003683in}}{\pgfqpoint{-0.013889in}{0.000000in}}%
\pgfpathcurveto{\pgfqpoint{-0.013889in}{-0.003683in}}{\pgfqpoint{-0.012425in}{-0.007216in}}{\pgfqpoint{-0.009821in}{-0.009821in}}%
\pgfpathcurveto{\pgfqpoint{-0.007216in}{-0.012425in}}{\pgfqpoint{-0.003683in}{-0.013889in}}{\pgfqpoint{0.000000in}{-0.013889in}}%
\pgfpathlineto{\pgfqpoint{0.000000in}{-0.013889in}}%
\pgfpathclose%
\pgfusepath{stroke,fill}%
}%
\begin{pgfscope}%
\pgfsys@transformshift{4.366087in}{3.476829in}%
\pgfsys@useobject{currentmarker}{}%
\end{pgfscope}%
\end{pgfscope}%
\begin{pgfscope}%
\definecolor{textcolor}{rgb}{0.000000,0.000000,0.000000}%
\pgfsetstrokecolor{textcolor}%
\pgfsetfillcolor{textcolor}%
\pgftext[x=4.616087in,y=3.428218in,left,base]{\color{textcolor}\rmfamily\fontsize{10.000000}{12.000000}\selectfont Dunkelbilder}%
\end{pgfscope}%
\end{pgfpicture}%
\makeatother%
\endgroup%

    \caption{Histogramm von \num{10000} Dunkelbildern, das mit Fit $G(W,\mu,\sigma, A)$ mit $\mu= \SI{-0.32(2)}{\adu}$, $\sigma_R = \SI{19.9(1)}{\adu}$ und $A = \num{1.59e9}$ angepasst wird.}
    \label{fig:noise_hist_fit}
\end{figure}
\noindent
So wurde die Standardabweichung vom Rauschen
\begin{equation}
    \sigma_R = \SI{19.9(1)}{\adu}
    \label{eq:sigma_r}
\end{equation}
ermittelt und wird in der weiteren Analyse benutzt.

\section{Energiekalibration}
Zuerst muss die Photonenenergie kalibriert werden. Die \gls{rzp} dispergiert das \gls{pxs} Spektrum entlang der vertikalen Richtung. Das Einstellen der Photonenenergie am Experiment erfolgt über eine Kippung der \gls{rzp} mithilfe eines Schrittmotors, der mit dem Namen $\varphi_\text{\gls{rzp}}$ bezeichnet wird. Die Motorpositionen sind in den willkürlichen Einheiten angegeben.

\noindent
Um die Motorpositionen dem auf die Probe abgebildeten Photonenenergiebereich zuzuordnen, wird das Absorptionspektrum der Probe über die Motorposition $\varphi_\text{\gls{rzp}}$ im Intervall von \num{-115} bis \num{-50} aufgenommen. Diese Messung wird mit einer CCD-Kamera durchgeführt, die bereits in die Labor-Steuerungssoftware integriert ist und somit die mit der Motorpositionsveränderung synchronisierten Aufnahmen ermöglicht.

\noindent
Das vermessene Absorptionsspektrum wird mit dem Referenzabsorptionsspektrum von Gd im weichen Röntgenstrahlungsbereich verglichen. In dem Energieintervall nimmt das Spektrum zwei Maxima an den Gd M4 und Gd M5 Absorptionslinien an. Die gegebenen \cite[Abb. 2]{prieto-x-ray-2005} Absoprtionskoeffzienten $\beta \text{ (anti-)par.}$ entsprechen der (anti-)parallel zirkular polarisierten Strahlung relativ zur Magnetisierungachse des Atoms. Der Absoprtionskoeffzient für die linear polarisierte Strahlung $\bar{\beta}$ ergibt sich als Mittelwert davon.

\noindent
Die Energieachse wird linear zu der Motorposition $\varphi_\text{\gls{rzp}}$ angepasst, indem die Absorptionslinien Gd M5 und Gd M4 den zweien Peaks in dem vermessenen Spektrum zugeordnet werden. Das vermessene Spektrum und die Referenz werden über die Energieachse in Abb. \ref{fig:rzp_phi_ev} aufgetragen.
\begin{figure}[H]
    \centering
    %% Creator: Matplotlib, PGF backend
%%
%% To include the figure in your LaTeX document, write
%%   \input{<filename>.pgf}
%%
%% Make sure the required packages are loaded in your preamble
%%   \usepackage{pgf}
%%
%% Also ensure that all the required font packages are loaded; for instance,
%% the lmodern package is sometimes necessary when using math font.
%%   \usepackage{lmodern}
%%
%% Figures using additional raster images can only be included by \input if
%% they are in the same directory as the main LaTeX file. For loading figures
%% from other directories you can use the `import` package
%%   \usepackage{import}
%%
%% and then include the figures with
%%   \import{<path to file>}{<filename>.pgf}
%%
%% Matplotlib used the following preamble
%%   \usepackage{amsmath} \usepackage[utf8]{inputenc} \usepackage[T1]{fontenc} \usepackage[output-decimal-marker={,},print-unity-mantissa=false]{siunitx} \sisetup{per-mode=fraction, separate-uncertainty = true, locale = DE} \usepackage[acronym, toc, section=section, nonumberlist, nopostdot]{glossaries-extra}
%%
\begingroup%
\makeatletter%
\begin{pgfpicture}%
\pgfpathrectangle{\pgfpointorigin}{\pgfqpoint{5.598360in}{3.412478in}}%
\pgfusepath{use as bounding box, clip}%
\begin{pgfscope}%
\pgfsetbuttcap%
\pgfsetmiterjoin%
\pgfsetlinewidth{0.000000pt}%
\definecolor{currentstroke}{rgb}{1.000000,1.000000,1.000000}%
\pgfsetstrokecolor{currentstroke}%
\pgfsetstrokeopacity{0.000000}%
\pgfsetdash{}{0pt}%
\pgfpathmoveto{\pgfqpoint{0.000000in}{0.000000in}}%
\pgfpathlineto{\pgfqpoint{5.598360in}{0.000000in}}%
\pgfpathlineto{\pgfqpoint{5.598360in}{3.412478in}}%
\pgfpathlineto{\pgfqpoint{0.000000in}{3.412478in}}%
\pgfpathlineto{\pgfqpoint{0.000000in}{0.000000in}}%
\pgfpathclose%
\pgfusepath{}%
\end{pgfscope}%
\begin{pgfscope}%
\pgfsetbuttcap%
\pgfsetmiterjoin%
\definecolor{currentfill}{rgb}{1.000000,1.000000,1.000000}%
\pgfsetfillcolor{currentfill}%
\pgfsetlinewidth{0.000000pt}%
\definecolor{currentstroke}{rgb}{0.000000,0.000000,0.000000}%
\pgfsetstrokecolor{currentstroke}%
\pgfsetstrokeopacity{0.000000}%
\pgfsetdash{}{0pt}%
\pgfpathmoveto{\pgfqpoint{0.452903in}{0.398095in}}%
\pgfpathlineto{\pgfqpoint{5.474903in}{0.398095in}}%
\pgfpathlineto{\pgfqpoint{5.474903in}{3.040595in}}%
\pgfpathlineto{\pgfqpoint{0.452903in}{3.040595in}}%
\pgfpathlineto{\pgfqpoint{0.452903in}{0.398095in}}%
\pgfpathclose%
\pgfusepath{fill}%
\end{pgfscope}%
\begin{pgfscope}%
\pgfsetbuttcap%
\pgfsetmiterjoin%
\definecolor{currentfill}{rgb}{1.000000,1.000000,1.000000}%
\pgfsetfillcolor{currentfill}%
\pgfsetlinewidth{0.000000pt}%
\definecolor{currentstroke}{rgb}{0.000000,0.000000,0.000000}%
\pgfsetstrokecolor{currentstroke}%
\pgfsetstrokeopacity{0.000000}%
\pgfsetdash{}{0pt}%
\pgfpathmoveto{\pgfqpoint{0.452903in}{3.040595in}}%
\pgfpathlineto{\pgfqpoint{5.474903in}{3.040595in}}%
\pgfpathlineto{\pgfqpoint{5.474903in}{3.040595in}}%
\pgfpathlineto{\pgfqpoint{0.452903in}{3.040595in}}%
\pgfpathlineto{\pgfqpoint{0.452903in}{3.040595in}}%
\pgfpathclose%
\pgfusepath{fill}%
\end{pgfscope}%
\begin{pgfscope}%
\pgfsetbuttcap%
\pgfsetroundjoin%
\definecolor{currentfill}{rgb}{0.000000,0.000000,0.000000}%
\pgfsetfillcolor{currentfill}%
\pgfsetlinewidth{0.803000pt}%
\definecolor{currentstroke}{rgb}{0.000000,0.000000,0.000000}%
\pgfsetstrokecolor{currentstroke}%
\pgfsetdash{}{0pt}%
\pgfsys@defobject{currentmarker}{\pgfqpoint{0.000000in}{0.000000in}}{\pgfqpoint{0.000000in}{0.048611in}}{%
\pgfpathmoveto{\pgfqpoint{0.000000in}{0.000000in}}%
\pgfpathlineto{\pgfqpoint{0.000000in}{0.048611in}}%
\pgfusepath{stroke,fill}%
}%
\begin{pgfscope}%
\pgfsys@transformshift{5.437703in}{3.040595in}%
\pgfsys@useobject{currentmarker}{}%
\end{pgfscope}%
\end{pgfscope}%
\begin{pgfscope}%
\definecolor{textcolor}{rgb}{0.000000,0.000000,0.000000}%
\pgfsetstrokecolor{textcolor}%
\pgfsetfillcolor{textcolor}%
\pgftext[x=5.437703in,y=3.137817in,,bottom]{\color{textcolor}\rmfamily\fontsize{10.000000}{12.000000}\selectfont \(\displaystyle {1160}\)}%
\end{pgfscope}%
\begin{pgfscope}%
\pgfsetbuttcap%
\pgfsetroundjoin%
\definecolor{currentfill}{rgb}{0.000000,0.000000,0.000000}%
\pgfsetfillcolor{currentfill}%
\pgfsetlinewidth{0.803000pt}%
\definecolor{currentstroke}{rgb}{0.000000,0.000000,0.000000}%
\pgfsetstrokecolor{currentstroke}%
\pgfsetdash{}{0pt}%
\pgfsys@defobject{currentmarker}{\pgfqpoint{0.000000in}{0.000000in}}{\pgfqpoint{0.000000in}{0.048611in}}{%
\pgfpathmoveto{\pgfqpoint{0.000000in}{0.000000in}}%
\pgfpathlineto{\pgfqpoint{0.000000in}{0.048611in}}%
\pgfusepath{stroke,fill}%
}%
\begin{pgfscope}%
\pgfsys@transformshift{4.693703in}{3.040595in}%
\pgfsys@useobject{currentmarker}{}%
\end{pgfscope}%
\end{pgfscope}%
\begin{pgfscope}%
\definecolor{textcolor}{rgb}{0.000000,0.000000,0.000000}%
\pgfsetstrokecolor{textcolor}%
\pgfsetfillcolor{textcolor}%
\pgftext[x=4.693703in,y=3.137817in,,bottom]{\color{textcolor}\rmfamily\fontsize{10.000000}{12.000000}\selectfont \(\displaystyle {1170}\)}%
\end{pgfscope}%
\begin{pgfscope}%
\pgfsetbuttcap%
\pgfsetroundjoin%
\definecolor{currentfill}{rgb}{0.000000,0.000000,0.000000}%
\pgfsetfillcolor{currentfill}%
\pgfsetlinewidth{0.803000pt}%
\definecolor{currentstroke}{rgb}{0.000000,0.000000,0.000000}%
\pgfsetstrokecolor{currentstroke}%
\pgfsetdash{}{0pt}%
\pgfsys@defobject{currentmarker}{\pgfqpoint{0.000000in}{0.000000in}}{\pgfqpoint{0.000000in}{0.048611in}}{%
\pgfpathmoveto{\pgfqpoint{0.000000in}{0.000000in}}%
\pgfpathlineto{\pgfqpoint{0.000000in}{0.048611in}}%
\pgfusepath{stroke,fill}%
}%
\begin{pgfscope}%
\pgfsys@transformshift{3.949703in}{3.040595in}%
\pgfsys@useobject{currentmarker}{}%
\end{pgfscope}%
\end{pgfscope}%
\begin{pgfscope}%
\definecolor{textcolor}{rgb}{0.000000,0.000000,0.000000}%
\pgfsetstrokecolor{textcolor}%
\pgfsetfillcolor{textcolor}%
\pgftext[x=3.949703in,y=3.137817in,,bottom]{\color{textcolor}\rmfamily\fontsize{10.000000}{12.000000}\selectfont \(\displaystyle {1180}\)}%
\end{pgfscope}%
\begin{pgfscope}%
\pgfsetbuttcap%
\pgfsetroundjoin%
\definecolor{currentfill}{rgb}{0.000000,0.000000,0.000000}%
\pgfsetfillcolor{currentfill}%
\pgfsetlinewidth{0.803000pt}%
\definecolor{currentstroke}{rgb}{0.000000,0.000000,0.000000}%
\pgfsetstrokecolor{currentstroke}%
\pgfsetdash{}{0pt}%
\pgfsys@defobject{currentmarker}{\pgfqpoint{0.000000in}{0.000000in}}{\pgfqpoint{0.000000in}{0.048611in}}{%
\pgfpathmoveto{\pgfqpoint{0.000000in}{0.000000in}}%
\pgfpathlineto{\pgfqpoint{0.000000in}{0.048611in}}%
\pgfusepath{stroke,fill}%
}%
\begin{pgfscope}%
\pgfsys@transformshift{3.205703in}{3.040595in}%
\pgfsys@useobject{currentmarker}{}%
\end{pgfscope}%
\end{pgfscope}%
\begin{pgfscope}%
\definecolor{textcolor}{rgb}{0.000000,0.000000,0.000000}%
\pgfsetstrokecolor{textcolor}%
\pgfsetfillcolor{textcolor}%
\pgftext[x=3.205703in,y=3.137817in,,bottom]{\color{textcolor}\rmfamily\fontsize{10.000000}{12.000000}\selectfont \(\displaystyle {1190}\)}%
\end{pgfscope}%
\begin{pgfscope}%
\pgfsetbuttcap%
\pgfsetroundjoin%
\definecolor{currentfill}{rgb}{0.000000,0.000000,0.000000}%
\pgfsetfillcolor{currentfill}%
\pgfsetlinewidth{0.803000pt}%
\definecolor{currentstroke}{rgb}{0.000000,0.000000,0.000000}%
\pgfsetstrokecolor{currentstroke}%
\pgfsetdash{}{0pt}%
\pgfsys@defobject{currentmarker}{\pgfqpoint{0.000000in}{0.000000in}}{\pgfqpoint{0.000000in}{0.048611in}}{%
\pgfpathmoveto{\pgfqpoint{0.000000in}{0.000000in}}%
\pgfpathlineto{\pgfqpoint{0.000000in}{0.048611in}}%
\pgfusepath{stroke,fill}%
}%
\begin{pgfscope}%
\pgfsys@transformshift{2.461703in}{3.040595in}%
\pgfsys@useobject{currentmarker}{}%
\end{pgfscope}%
\end{pgfscope}%
\begin{pgfscope}%
\definecolor{textcolor}{rgb}{0.000000,0.000000,0.000000}%
\pgfsetstrokecolor{textcolor}%
\pgfsetfillcolor{textcolor}%
\pgftext[x=2.461703in,y=3.137817in,,bottom]{\color{textcolor}\rmfamily\fontsize{10.000000}{12.000000}\selectfont \(\displaystyle {1200}\)}%
\end{pgfscope}%
\begin{pgfscope}%
\pgfsetbuttcap%
\pgfsetroundjoin%
\definecolor{currentfill}{rgb}{0.000000,0.000000,0.000000}%
\pgfsetfillcolor{currentfill}%
\pgfsetlinewidth{0.803000pt}%
\definecolor{currentstroke}{rgb}{0.000000,0.000000,0.000000}%
\pgfsetstrokecolor{currentstroke}%
\pgfsetdash{}{0pt}%
\pgfsys@defobject{currentmarker}{\pgfqpoint{0.000000in}{0.000000in}}{\pgfqpoint{0.000000in}{0.048611in}}{%
\pgfpathmoveto{\pgfqpoint{0.000000in}{0.000000in}}%
\pgfpathlineto{\pgfqpoint{0.000000in}{0.048611in}}%
\pgfusepath{stroke,fill}%
}%
\begin{pgfscope}%
\pgfsys@transformshift{1.717703in}{3.040595in}%
\pgfsys@useobject{currentmarker}{}%
\end{pgfscope}%
\end{pgfscope}%
\begin{pgfscope}%
\definecolor{textcolor}{rgb}{0.000000,0.000000,0.000000}%
\pgfsetstrokecolor{textcolor}%
\pgfsetfillcolor{textcolor}%
\pgftext[x=1.717703in,y=3.137817in,,bottom]{\color{textcolor}\rmfamily\fontsize{10.000000}{12.000000}\selectfont \(\displaystyle {1210}\)}%
\end{pgfscope}%
\begin{pgfscope}%
\pgfsetbuttcap%
\pgfsetroundjoin%
\definecolor{currentfill}{rgb}{0.000000,0.000000,0.000000}%
\pgfsetfillcolor{currentfill}%
\pgfsetlinewidth{0.803000pt}%
\definecolor{currentstroke}{rgb}{0.000000,0.000000,0.000000}%
\pgfsetstrokecolor{currentstroke}%
\pgfsetdash{}{0pt}%
\pgfsys@defobject{currentmarker}{\pgfqpoint{0.000000in}{0.000000in}}{\pgfqpoint{0.000000in}{0.048611in}}{%
\pgfpathmoveto{\pgfqpoint{0.000000in}{0.000000in}}%
\pgfpathlineto{\pgfqpoint{0.000000in}{0.048611in}}%
\pgfusepath{stroke,fill}%
}%
\begin{pgfscope}%
\pgfsys@transformshift{0.973703in}{3.040595in}%
\pgfsys@useobject{currentmarker}{}%
\end{pgfscope}%
\end{pgfscope}%
\begin{pgfscope}%
\definecolor{textcolor}{rgb}{0.000000,0.000000,0.000000}%
\pgfsetstrokecolor{textcolor}%
\pgfsetfillcolor{textcolor}%
\pgftext[x=0.973703in,y=3.137817in,,bottom]{\color{textcolor}\rmfamily\fontsize{10.000000}{12.000000}\selectfont \(\displaystyle {1220}\)}%
\end{pgfscope}%
\begin{pgfscope}%
\definecolor{textcolor}{rgb}{0.000000,0.000000,0.000000}%
\pgfsetstrokecolor{textcolor}%
\pgfsetfillcolor{textcolor}%
\pgftext[x=2.963903in,y=3.316028in,,base]{\color{textcolor}\rmfamily\fontsize{10.000000}{12.000000}\selectfont Photonenenergie \(\displaystyle h\nu\) in eV}%
\end{pgfscope}%
\begin{pgfscope}%
\pgfsetrectcap%
\pgfsetmiterjoin%
\pgfsetlinewidth{0.803000pt}%
\definecolor{currentstroke}{rgb}{0.000000,0.000000,0.000000}%
\pgfsetstrokecolor{currentstroke}%
\pgfsetdash{}{0pt}%
\pgfpathmoveto{\pgfqpoint{5.474903in}{3.040595in}}%
\pgfpathlineto{\pgfqpoint{0.452903in}{3.040595in}}%
\pgfusepath{stroke}%
\end{pgfscope}%
\begin{pgfscope}%
\pgfsetbuttcap%
\pgfsetroundjoin%
\definecolor{currentfill}{rgb}{0.000000,0.000000,0.000000}%
\pgfsetfillcolor{currentfill}%
\pgfsetlinewidth{0.803000pt}%
\definecolor{currentstroke}{rgb}{0.000000,0.000000,0.000000}%
\pgfsetstrokecolor{currentstroke}%
\pgfsetdash{}{0pt}%
\pgfsys@defobject{currentmarker}{\pgfqpoint{0.000000in}{-0.048611in}}{\pgfqpoint{0.000000in}{0.000000in}}{%
\pgfpathmoveto{\pgfqpoint{0.000000in}{0.000000in}}%
\pgfpathlineto{\pgfqpoint{0.000000in}{-0.048611in}}%
\pgfusepath{stroke,fill}%
}%
\begin{pgfscope}%
\pgfsys@transformshift{0.452903in}{0.398095in}%
\pgfsys@useobject{currentmarker}{}%
\end{pgfscope}%
\end{pgfscope}%
\begin{pgfscope}%
\definecolor{textcolor}{rgb}{0.000000,0.000000,0.000000}%
\pgfsetstrokecolor{textcolor}%
\pgfsetfillcolor{textcolor}%
\pgftext[x=0.452903in,y=0.300872in,,top]{\color{textcolor}\rmfamily\fontsize{10.000000}{12.000000}\selectfont \(\displaystyle {\ensuremath{-}95}\)}%
\end{pgfscope}%
\begin{pgfscope}%
\pgfsetbuttcap%
\pgfsetroundjoin%
\definecolor{currentfill}{rgb}{0.000000,0.000000,0.000000}%
\pgfsetfillcolor{currentfill}%
\pgfsetlinewidth{0.803000pt}%
\definecolor{currentstroke}{rgb}{0.000000,0.000000,0.000000}%
\pgfsetstrokecolor{currentstroke}%
\pgfsetdash{}{0pt}%
\pgfsys@defobject{currentmarker}{\pgfqpoint{0.000000in}{-0.048611in}}{\pgfqpoint{0.000000in}{0.000000in}}{%
\pgfpathmoveto{\pgfqpoint{0.000000in}{0.000000in}}%
\pgfpathlineto{\pgfqpoint{0.000000in}{-0.048611in}}%
\pgfusepath{stroke,fill}%
}%
\begin{pgfscope}%
\pgfsys@transformshift{1.010903in}{0.398095in}%
\pgfsys@useobject{currentmarker}{}%
\end{pgfscope}%
\end{pgfscope}%
\begin{pgfscope}%
\definecolor{textcolor}{rgb}{0.000000,0.000000,0.000000}%
\pgfsetstrokecolor{textcolor}%
\pgfsetfillcolor{textcolor}%
\pgftext[x=1.010903in,y=0.300872in,,top]{\color{textcolor}\rmfamily\fontsize{10.000000}{12.000000}\selectfont \(\displaystyle {\ensuremath{-}90}\)}%
\end{pgfscope}%
\begin{pgfscope}%
\pgfsetbuttcap%
\pgfsetroundjoin%
\definecolor{currentfill}{rgb}{0.000000,0.000000,0.000000}%
\pgfsetfillcolor{currentfill}%
\pgfsetlinewidth{0.803000pt}%
\definecolor{currentstroke}{rgb}{0.000000,0.000000,0.000000}%
\pgfsetstrokecolor{currentstroke}%
\pgfsetdash{}{0pt}%
\pgfsys@defobject{currentmarker}{\pgfqpoint{0.000000in}{-0.048611in}}{\pgfqpoint{0.000000in}{0.000000in}}{%
\pgfpathmoveto{\pgfqpoint{0.000000in}{0.000000in}}%
\pgfpathlineto{\pgfqpoint{0.000000in}{-0.048611in}}%
\pgfusepath{stroke,fill}%
}%
\begin{pgfscope}%
\pgfsys@transformshift{1.568903in}{0.398095in}%
\pgfsys@useobject{currentmarker}{}%
\end{pgfscope}%
\end{pgfscope}%
\begin{pgfscope}%
\definecolor{textcolor}{rgb}{0.000000,0.000000,0.000000}%
\pgfsetstrokecolor{textcolor}%
\pgfsetfillcolor{textcolor}%
\pgftext[x=1.568903in,y=0.300872in,,top]{\color{textcolor}\rmfamily\fontsize{10.000000}{12.000000}\selectfont \(\displaystyle {\ensuremath{-}85}\)}%
\end{pgfscope}%
\begin{pgfscope}%
\pgfsetbuttcap%
\pgfsetroundjoin%
\definecolor{currentfill}{rgb}{0.000000,0.000000,0.000000}%
\pgfsetfillcolor{currentfill}%
\pgfsetlinewidth{0.803000pt}%
\definecolor{currentstroke}{rgb}{0.000000,0.000000,0.000000}%
\pgfsetstrokecolor{currentstroke}%
\pgfsetdash{}{0pt}%
\pgfsys@defobject{currentmarker}{\pgfqpoint{0.000000in}{-0.048611in}}{\pgfqpoint{0.000000in}{0.000000in}}{%
\pgfpathmoveto{\pgfqpoint{0.000000in}{0.000000in}}%
\pgfpathlineto{\pgfqpoint{0.000000in}{-0.048611in}}%
\pgfusepath{stroke,fill}%
}%
\begin{pgfscope}%
\pgfsys@transformshift{2.126903in}{0.398095in}%
\pgfsys@useobject{currentmarker}{}%
\end{pgfscope}%
\end{pgfscope}%
\begin{pgfscope}%
\definecolor{textcolor}{rgb}{0.000000,0.000000,0.000000}%
\pgfsetstrokecolor{textcolor}%
\pgfsetfillcolor{textcolor}%
\pgftext[x=2.126903in,y=0.300872in,,top]{\color{textcolor}\rmfamily\fontsize{10.000000}{12.000000}\selectfont \(\displaystyle {\ensuremath{-}80}\)}%
\end{pgfscope}%
\begin{pgfscope}%
\pgfsetbuttcap%
\pgfsetroundjoin%
\definecolor{currentfill}{rgb}{0.000000,0.000000,0.000000}%
\pgfsetfillcolor{currentfill}%
\pgfsetlinewidth{0.803000pt}%
\definecolor{currentstroke}{rgb}{0.000000,0.000000,0.000000}%
\pgfsetstrokecolor{currentstroke}%
\pgfsetdash{}{0pt}%
\pgfsys@defobject{currentmarker}{\pgfqpoint{0.000000in}{-0.048611in}}{\pgfqpoint{0.000000in}{0.000000in}}{%
\pgfpathmoveto{\pgfqpoint{0.000000in}{0.000000in}}%
\pgfpathlineto{\pgfqpoint{0.000000in}{-0.048611in}}%
\pgfusepath{stroke,fill}%
}%
\begin{pgfscope}%
\pgfsys@transformshift{2.684903in}{0.398095in}%
\pgfsys@useobject{currentmarker}{}%
\end{pgfscope}%
\end{pgfscope}%
\begin{pgfscope}%
\definecolor{textcolor}{rgb}{0.000000,0.000000,0.000000}%
\pgfsetstrokecolor{textcolor}%
\pgfsetfillcolor{textcolor}%
\pgftext[x=2.684903in,y=0.300872in,,top]{\color{textcolor}\rmfamily\fontsize{10.000000}{12.000000}\selectfont \(\displaystyle {\ensuremath{-}75}\)}%
\end{pgfscope}%
\begin{pgfscope}%
\pgfsetbuttcap%
\pgfsetroundjoin%
\definecolor{currentfill}{rgb}{0.000000,0.000000,0.000000}%
\pgfsetfillcolor{currentfill}%
\pgfsetlinewidth{0.803000pt}%
\definecolor{currentstroke}{rgb}{0.000000,0.000000,0.000000}%
\pgfsetstrokecolor{currentstroke}%
\pgfsetdash{}{0pt}%
\pgfsys@defobject{currentmarker}{\pgfqpoint{0.000000in}{-0.048611in}}{\pgfqpoint{0.000000in}{0.000000in}}{%
\pgfpathmoveto{\pgfqpoint{0.000000in}{0.000000in}}%
\pgfpathlineto{\pgfqpoint{0.000000in}{-0.048611in}}%
\pgfusepath{stroke,fill}%
}%
\begin{pgfscope}%
\pgfsys@transformshift{3.242903in}{0.398095in}%
\pgfsys@useobject{currentmarker}{}%
\end{pgfscope}%
\end{pgfscope}%
\begin{pgfscope}%
\definecolor{textcolor}{rgb}{0.000000,0.000000,0.000000}%
\pgfsetstrokecolor{textcolor}%
\pgfsetfillcolor{textcolor}%
\pgftext[x=3.242903in,y=0.300872in,,top]{\color{textcolor}\rmfamily\fontsize{10.000000}{12.000000}\selectfont \(\displaystyle {\ensuremath{-}70}\)}%
\end{pgfscope}%
\begin{pgfscope}%
\pgfsetbuttcap%
\pgfsetroundjoin%
\definecolor{currentfill}{rgb}{0.000000,0.000000,0.000000}%
\pgfsetfillcolor{currentfill}%
\pgfsetlinewidth{0.803000pt}%
\definecolor{currentstroke}{rgb}{0.000000,0.000000,0.000000}%
\pgfsetstrokecolor{currentstroke}%
\pgfsetdash{}{0pt}%
\pgfsys@defobject{currentmarker}{\pgfqpoint{0.000000in}{-0.048611in}}{\pgfqpoint{0.000000in}{0.000000in}}{%
\pgfpathmoveto{\pgfqpoint{0.000000in}{0.000000in}}%
\pgfpathlineto{\pgfqpoint{0.000000in}{-0.048611in}}%
\pgfusepath{stroke,fill}%
}%
\begin{pgfscope}%
\pgfsys@transformshift{3.800903in}{0.398095in}%
\pgfsys@useobject{currentmarker}{}%
\end{pgfscope}%
\end{pgfscope}%
\begin{pgfscope}%
\definecolor{textcolor}{rgb}{0.000000,0.000000,0.000000}%
\pgfsetstrokecolor{textcolor}%
\pgfsetfillcolor{textcolor}%
\pgftext[x=3.800903in,y=0.300872in,,top]{\color{textcolor}\rmfamily\fontsize{10.000000}{12.000000}\selectfont \(\displaystyle {\ensuremath{-}65}\)}%
\end{pgfscope}%
\begin{pgfscope}%
\pgfsetbuttcap%
\pgfsetroundjoin%
\definecolor{currentfill}{rgb}{0.000000,0.000000,0.000000}%
\pgfsetfillcolor{currentfill}%
\pgfsetlinewidth{0.803000pt}%
\definecolor{currentstroke}{rgb}{0.000000,0.000000,0.000000}%
\pgfsetstrokecolor{currentstroke}%
\pgfsetdash{}{0pt}%
\pgfsys@defobject{currentmarker}{\pgfqpoint{0.000000in}{-0.048611in}}{\pgfqpoint{0.000000in}{0.000000in}}{%
\pgfpathmoveto{\pgfqpoint{0.000000in}{0.000000in}}%
\pgfpathlineto{\pgfqpoint{0.000000in}{-0.048611in}}%
\pgfusepath{stroke,fill}%
}%
\begin{pgfscope}%
\pgfsys@transformshift{4.358903in}{0.398095in}%
\pgfsys@useobject{currentmarker}{}%
\end{pgfscope}%
\end{pgfscope}%
\begin{pgfscope}%
\definecolor{textcolor}{rgb}{0.000000,0.000000,0.000000}%
\pgfsetstrokecolor{textcolor}%
\pgfsetfillcolor{textcolor}%
\pgftext[x=4.358903in,y=0.300872in,,top]{\color{textcolor}\rmfamily\fontsize{10.000000}{12.000000}\selectfont \(\displaystyle {\ensuremath{-}60}\)}%
\end{pgfscope}%
\begin{pgfscope}%
\pgfsetbuttcap%
\pgfsetroundjoin%
\definecolor{currentfill}{rgb}{0.000000,0.000000,0.000000}%
\pgfsetfillcolor{currentfill}%
\pgfsetlinewidth{0.803000pt}%
\definecolor{currentstroke}{rgb}{0.000000,0.000000,0.000000}%
\pgfsetstrokecolor{currentstroke}%
\pgfsetdash{}{0pt}%
\pgfsys@defobject{currentmarker}{\pgfqpoint{0.000000in}{-0.048611in}}{\pgfqpoint{0.000000in}{0.000000in}}{%
\pgfpathmoveto{\pgfqpoint{0.000000in}{0.000000in}}%
\pgfpathlineto{\pgfqpoint{0.000000in}{-0.048611in}}%
\pgfusepath{stroke,fill}%
}%
\begin{pgfscope}%
\pgfsys@transformshift{4.916903in}{0.398095in}%
\pgfsys@useobject{currentmarker}{}%
\end{pgfscope}%
\end{pgfscope}%
\begin{pgfscope}%
\definecolor{textcolor}{rgb}{0.000000,0.000000,0.000000}%
\pgfsetstrokecolor{textcolor}%
\pgfsetfillcolor{textcolor}%
\pgftext[x=4.916903in,y=0.300872in,,top]{\color{textcolor}\rmfamily\fontsize{10.000000}{12.000000}\selectfont \(\displaystyle {\ensuremath{-}55}\)}%
\end{pgfscope}%
\begin{pgfscope}%
\pgfsetbuttcap%
\pgfsetroundjoin%
\definecolor{currentfill}{rgb}{0.000000,0.000000,0.000000}%
\pgfsetfillcolor{currentfill}%
\pgfsetlinewidth{0.803000pt}%
\definecolor{currentstroke}{rgb}{0.000000,0.000000,0.000000}%
\pgfsetstrokecolor{currentstroke}%
\pgfsetdash{}{0pt}%
\pgfsys@defobject{currentmarker}{\pgfqpoint{0.000000in}{-0.048611in}}{\pgfqpoint{0.000000in}{0.000000in}}{%
\pgfpathmoveto{\pgfqpoint{0.000000in}{0.000000in}}%
\pgfpathlineto{\pgfqpoint{0.000000in}{-0.048611in}}%
\pgfusepath{stroke,fill}%
}%
\begin{pgfscope}%
\pgfsys@transformshift{5.474903in}{0.398095in}%
\pgfsys@useobject{currentmarker}{}%
\end{pgfscope}%
\end{pgfscope}%
\begin{pgfscope}%
\definecolor{textcolor}{rgb}{0.000000,0.000000,0.000000}%
\pgfsetstrokecolor{textcolor}%
\pgfsetfillcolor{textcolor}%
\pgftext[x=5.474903in,y=0.300872in,,top]{\color{textcolor}\rmfamily\fontsize{10.000000}{12.000000}\selectfont \(\displaystyle {\ensuremath{-}50}\)}%
\end{pgfscope}%
\begin{pgfscope}%
\definecolor{textcolor}{rgb}{0.000000,0.000000,0.000000}%
\pgfsetstrokecolor{textcolor}%
\pgfsetfillcolor{textcolor}%
\pgftext[x=2.963903in,y=0.122662in,,top]{\color{textcolor}\rmfamily\fontsize{10.000000}{12.000000}\selectfont Motorposition \(\displaystyle \varphi_{\text{RZP}}\) in \si{\degree}}%
\end{pgfscope}%
\begin{pgfscope}%
\pgfsetbuttcap%
\pgfsetroundjoin%
\definecolor{currentfill}{rgb}{0.000000,0.000000,0.000000}%
\pgfsetfillcolor{currentfill}%
\pgfsetlinewidth{0.803000pt}%
\definecolor{currentstroke}{rgb}{0.000000,0.000000,0.000000}%
\pgfsetstrokecolor{currentstroke}%
\pgfsetdash{}{0pt}%
\pgfsys@defobject{currentmarker}{\pgfqpoint{-0.048611in}{0.000000in}}{\pgfqpoint{-0.000000in}{0.000000in}}{%
\pgfpathmoveto{\pgfqpoint{-0.000000in}{0.000000in}}%
\pgfpathlineto{\pgfqpoint{-0.048611in}{0.000000in}}%
\pgfusepath{stroke,fill}%
}%
\begin{pgfscope}%
\pgfsys@transformshift{0.452903in}{0.505513in}%
\pgfsys@useobject{currentmarker}{}%
\end{pgfscope}%
\end{pgfscope}%
\begin{pgfscope}%
\definecolor{textcolor}{rgb}{0.000000,0.000000,0.000000}%
\pgfsetstrokecolor{textcolor}%
\pgfsetfillcolor{textcolor}%
\pgftext[x=0.178211in, y=0.457689in, left, base]{\color{textcolor}\rmfamily\fontsize{10.000000}{12.000000}\selectfont \num{0.0}}%
\end{pgfscope}%
\begin{pgfscope}%
\pgfsetbuttcap%
\pgfsetroundjoin%
\definecolor{currentfill}{rgb}{0.000000,0.000000,0.000000}%
\pgfsetfillcolor{currentfill}%
\pgfsetlinewidth{0.803000pt}%
\definecolor{currentstroke}{rgb}{0.000000,0.000000,0.000000}%
\pgfsetstrokecolor{currentstroke}%
\pgfsetdash{}{0pt}%
\pgfsys@defobject{currentmarker}{\pgfqpoint{-0.048611in}{0.000000in}}{\pgfqpoint{-0.000000in}{0.000000in}}{%
\pgfpathmoveto{\pgfqpoint{-0.000000in}{0.000000in}}%
\pgfpathlineto{\pgfqpoint{-0.048611in}{0.000000in}}%
\pgfusepath{stroke,fill}%
}%
\begin{pgfscope}%
\pgfsys@transformshift{0.452903in}{0.935188in}%
\pgfsys@useobject{currentmarker}{}%
\end{pgfscope}%
\end{pgfscope}%
\begin{pgfscope}%
\definecolor{textcolor}{rgb}{0.000000,0.000000,0.000000}%
\pgfsetstrokecolor{textcolor}%
\pgfsetfillcolor{textcolor}%
\pgftext[x=0.178211in, y=0.887364in, left, base]{\color{textcolor}\rmfamily\fontsize{10.000000}{12.000000}\selectfont \num{0.2}}%
\end{pgfscope}%
\begin{pgfscope}%
\pgfsetbuttcap%
\pgfsetroundjoin%
\definecolor{currentfill}{rgb}{0.000000,0.000000,0.000000}%
\pgfsetfillcolor{currentfill}%
\pgfsetlinewidth{0.803000pt}%
\definecolor{currentstroke}{rgb}{0.000000,0.000000,0.000000}%
\pgfsetstrokecolor{currentstroke}%
\pgfsetdash{}{0pt}%
\pgfsys@defobject{currentmarker}{\pgfqpoint{-0.048611in}{0.000000in}}{\pgfqpoint{-0.000000in}{0.000000in}}{%
\pgfpathmoveto{\pgfqpoint{-0.000000in}{0.000000in}}%
\pgfpathlineto{\pgfqpoint{-0.048611in}{0.000000in}}%
\pgfusepath{stroke,fill}%
}%
\begin{pgfscope}%
\pgfsys@transformshift{0.452903in}{1.364863in}%
\pgfsys@useobject{currentmarker}{}%
\end{pgfscope}%
\end{pgfscope}%
\begin{pgfscope}%
\definecolor{textcolor}{rgb}{0.000000,0.000000,0.000000}%
\pgfsetstrokecolor{textcolor}%
\pgfsetfillcolor{textcolor}%
\pgftext[x=0.178211in, y=1.317038in, left, base]{\color{textcolor}\rmfamily\fontsize{10.000000}{12.000000}\selectfont \num{0.4}}%
\end{pgfscope}%
\begin{pgfscope}%
\pgfsetbuttcap%
\pgfsetroundjoin%
\definecolor{currentfill}{rgb}{0.000000,0.000000,0.000000}%
\pgfsetfillcolor{currentfill}%
\pgfsetlinewidth{0.803000pt}%
\definecolor{currentstroke}{rgb}{0.000000,0.000000,0.000000}%
\pgfsetstrokecolor{currentstroke}%
\pgfsetdash{}{0pt}%
\pgfsys@defobject{currentmarker}{\pgfqpoint{-0.048611in}{0.000000in}}{\pgfqpoint{-0.000000in}{0.000000in}}{%
\pgfpathmoveto{\pgfqpoint{-0.000000in}{0.000000in}}%
\pgfpathlineto{\pgfqpoint{-0.048611in}{0.000000in}}%
\pgfusepath{stroke,fill}%
}%
\begin{pgfscope}%
\pgfsys@transformshift{0.452903in}{1.794538in}%
\pgfsys@useobject{currentmarker}{}%
\end{pgfscope}%
\end{pgfscope}%
\begin{pgfscope}%
\definecolor{textcolor}{rgb}{0.000000,0.000000,0.000000}%
\pgfsetstrokecolor{textcolor}%
\pgfsetfillcolor{textcolor}%
\pgftext[x=0.178211in, y=1.746713in, left, base]{\color{textcolor}\rmfamily\fontsize{10.000000}{12.000000}\selectfont \num{0.6}}%
\end{pgfscope}%
\begin{pgfscope}%
\pgfsetbuttcap%
\pgfsetroundjoin%
\definecolor{currentfill}{rgb}{0.000000,0.000000,0.000000}%
\pgfsetfillcolor{currentfill}%
\pgfsetlinewidth{0.803000pt}%
\definecolor{currentstroke}{rgb}{0.000000,0.000000,0.000000}%
\pgfsetstrokecolor{currentstroke}%
\pgfsetdash{}{0pt}%
\pgfsys@defobject{currentmarker}{\pgfqpoint{-0.048611in}{0.000000in}}{\pgfqpoint{-0.000000in}{0.000000in}}{%
\pgfpathmoveto{\pgfqpoint{-0.000000in}{0.000000in}}%
\pgfpathlineto{\pgfqpoint{-0.048611in}{0.000000in}}%
\pgfusepath{stroke,fill}%
}%
\begin{pgfscope}%
\pgfsys@transformshift{0.452903in}{2.224212in}%
\pgfsys@useobject{currentmarker}{}%
\end{pgfscope}%
\end{pgfscope}%
\begin{pgfscope}%
\definecolor{textcolor}{rgb}{0.000000,0.000000,0.000000}%
\pgfsetstrokecolor{textcolor}%
\pgfsetfillcolor{textcolor}%
\pgftext[x=0.178211in, y=2.176388in, left, base]{\color{textcolor}\rmfamily\fontsize{10.000000}{12.000000}\selectfont \num{0.8}}%
\end{pgfscope}%
\begin{pgfscope}%
\pgfsetbuttcap%
\pgfsetroundjoin%
\definecolor{currentfill}{rgb}{0.000000,0.000000,0.000000}%
\pgfsetfillcolor{currentfill}%
\pgfsetlinewidth{0.803000pt}%
\definecolor{currentstroke}{rgb}{0.000000,0.000000,0.000000}%
\pgfsetstrokecolor{currentstroke}%
\pgfsetdash{}{0pt}%
\pgfsys@defobject{currentmarker}{\pgfqpoint{-0.048611in}{0.000000in}}{\pgfqpoint{-0.000000in}{0.000000in}}{%
\pgfpathmoveto{\pgfqpoint{-0.000000in}{0.000000in}}%
\pgfpathlineto{\pgfqpoint{-0.048611in}{0.000000in}}%
\pgfusepath{stroke,fill}%
}%
\begin{pgfscope}%
\pgfsys@transformshift{0.452903in}{2.653887in}%
\pgfsys@useobject{currentmarker}{}%
\end{pgfscope}%
\end{pgfscope}%
\begin{pgfscope}%
\definecolor{textcolor}{rgb}{0.000000,0.000000,0.000000}%
\pgfsetstrokecolor{textcolor}%
\pgfsetfillcolor{textcolor}%
\pgftext[x=0.178211in, y=2.606063in, left, base]{\color{textcolor}\rmfamily\fontsize{10.000000}{12.000000}\selectfont \num{1.0}}%
\end{pgfscope}%
\begin{pgfscope}%
\definecolor{textcolor}{rgb}{0.000000,0.000000,0.000000}%
\pgfsetstrokecolor{textcolor}%
\pgfsetfillcolor{textcolor}%
\pgftext[x=0.122655in,y=1.719345in,,bottom,rotate=90.000000]{\color{textcolor}\rmfamily\fontsize{10.000000}{12.000000}\selectfont Absorptionrate}%
\end{pgfscope}%
\begin{pgfscope}%
\pgfpathrectangle{\pgfqpoint{0.452903in}{0.398095in}}{\pgfqpoint{5.022000in}{2.642500in}}%
\pgfusepath{clip}%
\pgfsetrectcap%
\pgfsetroundjoin%
\pgfsetlinewidth{1.003750pt}%
\definecolor{currentstroke}{rgb}{0.000000,0.000000,0.000000}%
\pgfsetstrokecolor{currentstroke}%
\pgfsetdash{}{0pt}%
\pgfpathmoveto{\pgfqpoint{3.577703in}{0.398095in}}%
\pgfpathlineto{\pgfqpoint{3.577703in}{3.040595in}}%
\pgfusepath{stroke}%
\end{pgfscope}%
\begin{pgfscope}%
\pgfpathrectangle{\pgfqpoint{0.452903in}{0.398095in}}{\pgfqpoint{5.022000in}{2.642500in}}%
\pgfusepath{clip}%
\pgfsetrectcap%
\pgfsetroundjoin%
\pgfsetlinewidth{1.003750pt}%
\definecolor{currentstroke}{rgb}{0.000000,0.000000,0.000000}%
\pgfsetstrokecolor{currentstroke}%
\pgfsetdash{}{0pt}%
\pgfpathmoveto{\pgfqpoint{1.345703in}{0.398095in}}%
\pgfpathlineto{\pgfqpoint{1.345703in}{3.040595in}}%
\pgfusepath{stroke}%
\end{pgfscope}%
\begin{pgfscope}%
\pgfpathrectangle{\pgfqpoint{0.452903in}{0.398095in}}{\pgfqpoint{5.022000in}{2.642500in}}%
\pgfusepath{clip}%
\pgfsetrectcap%
\pgfsetroundjoin%
\pgfsetlinewidth{1.003750pt}%
\definecolor{currentstroke}{rgb}{0.000000,0.000000,0.000000}%
\pgfsetstrokecolor{currentstroke}%
\pgfsetdash{}{0pt}%
\pgfpathmoveto{\pgfqpoint{5.363303in}{0.398095in}}%
\pgfpathlineto{\pgfqpoint{5.363303in}{3.040595in}}%
\pgfusepath{stroke}%
\end{pgfscope}%
\begin{pgfscope}%
\pgfpathrectangle{\pgfqpoint{0.452903in}{0.398095in}}{\pgfqpoint{5.022000in}{2.642500in}}%
\pgfusepath{clip}%
\pgfsetrectcap%
\pgfsetroundjoin%
\pgfsetlinewidth{1.505625pt}%
\definecolor{currentstroke}{rgb}{0.121569,0.466667,0.705882}%
\pgfsetstrokecolor{currentstroke}%
\pgfsetdash{}{0pt}%
\pgfpathmoveto{\pgfqpoint{5.475173in}{0.714888in}}%
\pgfpathlineto{\pgfqpoint{5.418956in}{0.741605in}}%
\pgfpathlineto{\pgfqpoint{5.363197in}{0.739045in}}%
\pgfpathlineto{\pgfqpoint{5.307429in}{0.706559in}}%
\pgfpathlineto{\pgfqpoint{5.251837in}{0.710725in}}%
\pgfpathlineto{\pgfqpoint{5.195622in}{0.757829in}}%
\pgfpathlineto{\pgfqpoint{5.139735in}{0.756197in}}%
\pgfpathlineto{\pgfqpoint{5.084182in}{0.747729in}}%
\pgfpathlineto{\pgfqpoint{5.028051in}{0.708614in}}%
\pgfpathlineto{\pgfqpoint{4.972902in}{0.720135in}}%
\pgfpathlineto{\pgfqpoint{4.916551in}{0.736066in}}%
\pgfpathlineto{\pgfqpoint{4.860868in}{0.688072in}}%
\pgfpathlineto{\pgfqpoint{4.804961in}{0.734410in}}%
\pgfpathlineto{\pgfqpoint{4.749025in}{0.716308in}}%
\pgfpathlineto{\pgfqpoint{4.693642in}{0.716027in}}%
\pgfpathlineto{\pgfqpoint{4.637730in}{0.721106in}}%
\pgfpathlineto{\pgfqpoint{4.581673in}{0.710018in}}%
\pgfpathlineto{\pgfqpoint{4.526257in}{0.718279in}}%
\pgfpathlineto{\pgfqpoint{4.470392in}{0.702555in}}%
\pgfpathlineto{\pgfqpoint{4.414920in}{0.719629in}}%
\pgfpathlineto{\pgfqpoint{4.358727in}{0.734660in}}%
\pgfpathlineto{\pgfqpoint{4.302679in}{0.717379in}}%
\pgfpathlineto{\pgfqpoint{4.246949in}{0.682144in}}%
\pgfpathlineto{\pgfqpoint{4.191042in}{0.675260in}}%
\pgfpathlineto{\pgfqpoint{4.135795in}{0.612434in}}%
\pgfpathlineto{\pgfqpoint{4.079921in}{0.566971in}}%
\pgfpathlineto{\pgfqpoint{4.024122in}{0.536029in}}%
\pgfpathlineto{\pgfqpoint{3.968237in}{0.505513in}}%
\pgfpathlineto{\pgfqpoint{3.912461in}{0.521924in}}%
\pgfpathlineto{\pgfqpoint{3.856689in}{0.544521in}}%
\pgfpathlineto{\pgfqpoint{3.800670in}{0.672690in}}%
\pgfpathlineto{\pgfqpoint{3.744775in}{0.784525in}}%
\pgfpathlineto{\pgfqpoint{3.689367in}{1.310203in}}%
\pgfpathlineto{\pgfqpoint{3.633029in}{1.782846in}}%
\pgfpathlineto{\pgfqpoint{3.577159in}{2.653887in}}%
\pgfpathlineto{\pgfqpoint{3.521865in}{2.472411in}}%
\pgfpathlineto{\pgfqpoint{3.465856in}{1.694426in}}%
\pgfpathlineto{\pgfqpoint{3.409878in}{1.259696in}}%
\pgfpathlineto{\pgfqpoint{3.354074in}{1.008551in}}%
\pgfpathlineto{\pgfqpoint{3.298336in}{0.791412in}}%
\pgfpathlineto{\pgfqpoint{3.242849in}{0.780617in}}%
\pgfpathlineto{\pgfqpoint{3.186740in}{0.736630in}}%
\pgfpathlineto{\pgfqpoint{3.130961in}{0.783035in}}%
\pgfpathlineto{\pgfqpoint{3.074948in}{0.745221in}}%
\pgfpathlineto{\pgfqpoint{3.019683in}{0.687576in}}%
\pgfpathlineto{\pgfqpoint{2.963859in}{0.707042in}}%
\pgfpathlineto{\pgfqpoint{2.907837in}{0.691531in}}%
\pgfpathlineto{\pgfqpoint{2.851915in}{0.686346in}}%
\pgfpathlineto{\pgfqpoint{2.796419in}{0.685381in}}%
\pgfpathlineto{\pgfqpoint{2.740221in}{0.687571in}}%
\pgfpathlineto{\pgfqpoint{2.684870in}{0.710977in}}%
\pgfpathlineto{\pgfqpoint{2.628836in}{0.700209in}}%
\pgfpathlineto{\pgfqpoint{2.572830in}{0.720656in}}%
\pgfpathlineto{\pgfqpoint{2.517137in}{0.736528in}}%
\pgfpathlineto{\pgfqpoint{2.461255in}{0.747545in}}%
\pgfpathlineto{\pgfqpoint{2.405858in}{0.755520in}}%
\pgfpathlineto{\pgfqpoint{2.349892in}{0.758000in}}%
\pgfpathlineto{\pgfqpoint{2.293940in}{0.773358in}}%
\pgfpathlineto{\pgfqpoint{2.238107in}{0.775570in}}%
\pgfpathlineto{\pgfqpoint{2.182461in}{0.761222in}}%
\pgfpathlineto{\pgfqpoint{2.126926in}{0.745058in}}%
\pgfpathlineto{\pgfqpoint{2.126889in}{0.750047in}}%
\pgfpathlineto{\pgfqpoint{2.071262in}{0.735859in}}%
\pgfpathlineto{\pgfqpoint{2.015034in}{0.703936in}}%
\pgfpathlineto{\pgfqpoint{1.959656in}{0.674418in}}%
\pgfpathlineto{\pgfqpoint{1.903587in}{0.654609in}}%
\pgfpathlineto{\pgfqpoint{1.847585in}{0.654075in}}%
\pgfpathlineto{\pgfqpoint{1.791985in}{0.623387in}}%
\pgfpathlineto{\pgfqpoint{1.736202in}{0.609463in}}%
\pgfpathlineto{\pgfqpoint{1.680396in}{0.632365in}}%
\pgfpathlineto{\pgfqpoint{1.624724in}{0.663955in}}%
\pgfpathlineto{\pgfqpoint{1.568624in}{0.714373in}}%
\pgfpathlineto{\pgfqpoint{1.512713in}{0.856917in}}%
\pgfpathlineto{\pgfqpoint{1.456783in}{1.005575in}}%
\pgfpathlineto{\pgfqpoint{1.401618in}{1.079556in}}%
\pgfpathlineto{\pgfqpoint{1.345579in}{1.194038in}}%
\pgfpathlineto{\pgfqpoint{1.289698in}{1.173110in}}%
\pgfpathlineto{\pgfqpoint{1.233865in}{1.066121in}}%
\pgfpathlineto{\pgfqpoint{1.177970in}{1.023541in}}%
\pgfpathlineto{\pgfqpoint{1.122008in}{0.864642in}}%
\pgfpathlineto{\pgfqpoint{1.066682in}{0.805837in}}%
\pgfpathlineto{\pgfqpoint{1.010598in}{0.755128in}}%
\pgfpathlineto{\pgfqpoint{0.954779in}{0.729465in}}%
\pgfpathlineto{\pgfqpoint{0.898763in}{0.733370in}}%
\pgfpathlineto{\pgfqpoint{0.842993in}{0.703689in}}%
\pgfpathlineto{\pgfqpoint{0.787766in}{0.700136in}}%
\pgfpathlineto{\pgfqpoint{0.731574in}{0.724512in}}%
\pgfpathlineto{\pgfqpoint{0.675797in}{0.726181in}}%
\pgfpathlineto{\pgfqpoint{0.620342in}{0.752758in}}%
\pgfpathlineto{\pgfqpoint{0.564008in}{0.743250in}}%
\pgfpathlineto{\pgfqpoint{0.508650in}{0.748288in}}%
\pgfpathlineto{\pgfqpoint{0.452627in}{0.776287in}}%
\pgfpathlineto{\pgfqpoint{0.450903in}{0.775053in}}%
\pgfusepath{stroke}%
\end{pgfscope}%
\begin{pgfscope}%
\pgfpathrectangle{\pgfqpoint{0.452903in}{0.398095in}}{\pgfqpoint{5.022000in}{2.642500in}}%
\pgfusepath{clip}%
\pgfsetrectcap%
\pgfsetroundjoin%
\pgfsetlinewidth{1.505625pt}%
\definecolor{currentstroke}{rgb}{1.000000,0.498039,0.054902}%
\pgfsetstrokecolor{currentstroke}%
\pgfsetdash{}{0pt}%
\pgfpathmoveto{\pgfqpoint{5.389523in}{0.570927in}}%
\pgfpathlineto{\pgfqpoint{4.620181in}{0.570950in}}%
\pgfpathlineto{\pgfqpoint{4.209865in}{0.575290in}}%
\pgfpathlineto{\pgfqpoint{4.158576in}{0.576597in}}%
\pgfpathlineto{\pgfqpoint{4.132931in}{0.577333in}}%
\pgfpathlineto{\pgfqpoint{4.107286in}{0.578499in}}%
\pgfpathlineto{\pgfqpoint{4.030352in}{0.589840in}}%
\pgfpathlineto{\pgfqpoint{4.004708in}{0.595140in}}%
\pgfpathlineto{\pgfqpoint{3.850839in}{0.639254in}}%
\pgfpathlineto{\pgfqpoint{3.825194in}{0.664835in}}%
\pgfpathlineto{\pgfqpoint{3.799550in}{0.721105in}}%
\pgfpathlineto{\pgfqpoint{3.773905in}{0.789733in}}%
\pgfpathlineto{\pgfqpoint{3.748260in}{0.866044in}}%
\pgfpathlineto{\pgfqpoint{3.722615in}{0.964132in}}%
\pgfpathlineto{\pgfqpoint{3.696971in}{1.103809in}}%
\pgfpathlineto{\pgfqpoint{3.671326in}{1.292262in}}%
\pgfpathlineto{\pgfqpoint{3.645681in}{1.573286in}}%
\pgfpathlineto{\pgfqpoint{3.620036in}{1.958714in}}%
\pgfpathlineto{\pgfqpoint{3.594392in}{2.376822in}}%
\pgfpathlineto{\pgfqpoint{3.568747in}{2.653887in}}%
\pgfpathlineto{\pgfqpoint{3.543102in}{2.480103in}}%
\pgfpathlineto{\pgfqpoint{3.517458in}{2.100469in}}%
\pgfpathlineto{\pgfqpoint{3.491813in}{1.717050in}}%
\pgfpathlineto{\pgfqpoint{3.466168in}{1.254015in}}%
\pgfpathlineto{\pgfqpoint{3.440523in}{1.016787in}}%
\pgfpathlineto{\pgfqpoint{3.414879in}{0.908395in}}%
\pgfpathlineto{\pgfqpoint{3.363589in}{0.832410in}}%
\pgfpathlineto{\pgfqpoint{3.337944in}{0.800895in}}%
\pgfpathlineto{\pgfqpoint{3.312300in}{0.795802in}}%
\pgfpathlineto{\pgfqpoint{3.235365in}{0.812430in}}%
\pgfpathlineto{\pgfqpoint{3.209721in}{0.815306in}}%
\pgfpathlineto{\pgfqpoint{3.184076in}{0.776342in}}%
\pgfpathlineto{\pgfqpoint{3.132786in}{0.680964in}}%
\pgfpathlineto{\pgfqpoint{3.107142in}{0.664904in}}%
\pgfpathlineto{\pgfqpoint{3.081497in}{0.652705in}}%
\pgfpathlineto{\pgfqpoint{3.004563in}{0.646908in}}%
\pgfpathlineto{\pgfqpoint{2.978918in}{0.640885in}}%
\pgfpathlineto{\pgfqpoint{2.876339in}{0.607432in}}%
\pgfpathlineto{\pgfqpoint{2.850694in}{0.603444in}}%
\pgfpathlineto{\pgfqpoint{2.825050in}{0.600701in}}%
\pgfpathlineto{\pgfqpoint{2.773760in}{0.597198in}}%
\pgfpathlineto{\pgfqpoint{2.696826in}{0.592186in}}%
\pgfpathlineto{\pgfqpoint{2.671181in}{0.591318in}}%
\pgfpathlineto{\pgfqpoint{2.337800in}{0.588568in}}%
\pgfpathlineto{\pgfqpoint{2.183931in}{0.587380in}}%
\pgfpathlineto{\pgfqpoint{2.158286in}{0.587590in}}%
\pgfpathlineto{\pgfqpoint{2.132642in}{0.588904in}}%
\pgfpathlineto{\pgfqpoint{1.953129in}{0.600230in}}%
\pgfpathlineto{\pgfqpoint{1.876194in}{0.605924in}}%
\pgfpathlineto{\pgfqpoint{1.850550in}{0.607857in}}%
\pgfpathlineto{\pgfqpoint{1.824905in}{0.612652in}}%
\pgfpathlineto{\pgfqpoint{1.773615in}{0.622784in}}%
\pgfpathlineto{\pgfqpoint{1.747971in}{0.629178in}}%
\pgfpathlineto{\pgfqpoint{1.568458in}{0.685560in}}%
\pgfpathlineto{\pgfqpoint{1.542813in}{0.716382in}}%
\pgfpathlineto{\pgfqpoint{1.517168in}{0.777020in}}%
\pgfpathlineto{\pgfqpoint{1.491523in}{0.849354in}}%
\pgfpathlineto{\pgfqpoint{1.465879in}{0.942886in}}%
\pgfpathlineto{\pgfqpoint{1.440234in}{1.075548in}}%
\pgfpathlineto{\pgfqpoint{1.414589in}{1.162664in}}%
\pgfpathlineto{\pgfqpoint{1.388944in}{1.202332in}}%
\pgfpathlineto{\pgfqpoint{1.363300in}{1.155711in}}%
\pgfpathlineto{\pgfqpoint{1.312010in}{1.091858in}}%
\pgfpathlineto{\pgfqpoint{1.286365in}{1.081794in}}%
\pgfpathlineto{\pgfqpoint{1.260721in}{1.111235in}}%
\pgfpathlineto{\pgfqpoint{1.235076in}{1.138145in}}%
\pgfpathlineto{\pgfqpoint{1.209431in}{1.126166in}}%
\pgfpathlineto{\pgfqpoint{1.132497in}{0.937658in}}%
\pgfpathlineto{\pgfqpoint{1.106852in}{0.881534in}}%
\pgfpathlineto{\pgfqpoint{1.081208in}{0.826517in}}%
\pgfpathlineto{\pgfqpoint{1.055563in}{0.810235in}}%
\pgfpathlineto{\pgfqpoint{0.952984in}{0.753772in}}%
\pgfpathlineto{\pgfqpoint{0.927339in}{0.742006in}}%
\pgfpathlineto{\pgfqpoint{0.901694in}{0.733783in}}%
\pgfpathlineto{\pgfqpoint{0.747826in}{0.686624in}}%
\pgfpathlineto{\pgfqpoint{0.722181in}{0.682392in}}%
\pgfpathlineto{\pgfqpoint{0.696536in}{0.679936in}}%
\pgfpathlineto{\pgfqpoint{0.517023in}{0.670139in}}%
\pgfpathlineto{\pgfqpoint{0.491379in}{0.669248in}}%
\pgfpathlineto{\pgfqpoint{0.465734in}{0.668602in}}%
\pgfpathlineto{\pgfqpoint{0.450903in}{0.668470in}}%
\pgfpathlineto{\pgfqpoint{0.450903in}{0.668470in}}%
\pgfusepath{stroke}%
\end{pgfscope}%
\begin{pgfscope}%
\pgfsetrectcap%
\pgfsetmiterjoin%
\pgfsetlinewidth{0.803000pt}%
\definecolor{currentstroke}{rgb}{0.000000,0.000000,0.000000}%
\pgfsetstrokecolor{currentstroke}%
\pgfsetdash{}{0pt}%
\pgfpathmoveto{\pgfqpoint{0.452903in}{0.398095in}}%
\pgfpathlineto{\pgfqpoint{0.452903in}{3.040595in}}%
\pgfusepath{stroke}%
\end{pgfscope}%
\begin{pgfscope}%
\pgfsetrectcap%
\pgfsetmiterjoin%
\pgfsetlinewidth{0.803000pt}%
\definecolor{currentstroke}{rgb}{0.000000,0.000000,0.000000}%
\pgfsetstrokecolor{currentstroke}%
\pgfsetdash{}{0pt}%
\pgfpathmoveto{\pgfqpoint{5.474903in}{0.398095in}}%
\pgfpathlineto{\pgfqpoint{5.474903in}{3.040595in}}%
\pgfusepath{stroke}%
\end{pgfscope}%
\begin{pgfscope}%
\pgfsetrectcap%
\pgfsetmiterjoin%
\pgfsetlinewidth{0.803000pt}%
\definecolor{currentstroke}{rgb}{0.000000,0.000000,0.000000}%
\pgfsetstrokecolor{currentstroke}%
\pgfsetdash{}{0pt}%
\pgfpathmoveto{\pgfqpoint{0.452903in}{0.398095in}}%
\pgfpathlineto{\pgfqpoint{5.474903in}{0.398095in}}%
\pgfusepath{stroke}%
\end{pgfscope}%
\begin{pgfscope}%
\pgfsetrectcap%
\pgfsetmiterjoin%
\pgfsetlinewidth{0.803000pt}%
\definecolor{currentstroke}{rgb}{0.000000,0.000000,0.000000}%
\pgfsetstrokecolor{currentstroke}%
\pgfsetdash{}{0pt}%
\pgfpathmoveto{\pgfqpoint{0.452903in}{3.040595in}}%
\pgfpathlineto{\pgfqpoint{5.474903in}{3.040595in}}%
\pgfusepath{stroke}%
\end{pgfscope}%
\begin{pgfscope}%
\definecolor{textcolor}{rgb}{0.000000,0.000000,0.000000}%
\pgfsetstrokecolor{textcolor}%
\pgfsetfillcolor{textcolor}%
\pgftext[x=3.521903in,y=2.825757in,right,base]{\color{textcolor}\rmfamily\fontsize{10.000000}{12.000000}\selectfont Gd M5}%
\end{pgfscope}%
\begin{pgfscope}%
\definecolor{textcolor}{rgb}{0.000000,0.000000,0.000000}%
\pgfsetstrokecolor{textcolor}%
\pgfsetfillcolor{textcolor}%
\pgftext[x=1.289903in,y=2.825757in,right,base]{\color{textcolor}\rmfamily\fontsize{10.000000}{12.000000}\selectfont Gd M4}%
\end{pgfscope}%
\begin{pgfscope}%
\definecolor{textcolor}{rgb}{0.000000,0.000000,0.000000}%
\pgfsetstrokecolor{textcolor}%
\pgfsetfillcolor{textcolor}%
\pgftext[x=5.307503in,y=2.825757in,right,base]{\color{textcolor}\rmfamily\fontsize{10.000000}{12.000000}\selectfont Off-Resonanz}%
\end{pgfscope}%
\begin{pgfscope}%
\pgfsetbuttcap%
\pgfsetmiterjoin%
\definecolor{currentfill}{rgb}{1.000000,1.000000,1.000000}%
\pgfsetfillcolor{currentfill}%
\pgfsetfillopacity{0.800000}%
\pgfsetlinewidth{1.003750pt}%
\definecolor{currentstroke}{rgb}{0.800000,0.800000,0.800000}%
\pgfsetstrokecolor{currentstroke}%
\pgfsetstrokeopacity{0.800000}%
\pgfsetdash{}{0pt}%
\pgfpathmoveto{\pgfqpoint{0.550125in}{1.415857in}}%
\pgfpathlineto{\pgfqpoint{2.549621in}{1.415857in}}%
\pgfpathquadraticcurveto{\pgfqpoint{2.577399in}{1.415857in}}{\pgfqpoint{2.577399in}{1.443634in}}%
\pgfpathlineto{\pgfqpoint{2.577399in}{1.995055in}}%
\pgfpathquadraticcurveto{\pgfqpoint{2.577399in}{2.022833in}}{\pgfqpoint{2.549621in}{2.022833in}}%
\pgfpathlineto{\pgfqpoint{0.550125in}{2.022833in}}%
\pgfpathquadraticcurveto{\pgfqpoint{0.522347in}{2.022833in}}{\pgfqpoint{0.522347in}{1.995055in}}%
\pgfpathlineto{\pgfqpoint{0.522347in}{1.443634in}}%
\pgfpathquadraticcurveto{\pgfqpoint{0.522347in}{1.415857in}}{\pgfqpoint{0.550125in}{1.415857in}}%
\pgfpathlineto{\pgfqpoint{0.550125in}{1.415857in}}%
\pgfpathclose%
\pgfusepath{stroke,fill}%
\end{pgfscope}%
\begin{pgfscope}%
\pgfsetrectcap%
\pgfsetroundjoin%
\pgfsetlinewidth{1.505625pt}%
\definecolor{currentstroke}{rgb}{0.121569,0.466667,0.705882}%
\pgfsetstrokecolor{currentstroke}%
\pgfsetdash{}{0pt}%
\pgfpathmoveto{\pgfqpoint{0.577903in}{1.918666in}}%
\pgfpathlineto{\pgfqpoint{0.716792in}{1.918666in}}%
\pgfpathlineto{\pgfqpoint{0.855680in}{1.918666in}}%
\pgfusepath{stroke}%
\end{pgfscope}%
\begin{pgfscope}%
\definecolor{textcolor}{rgb}{0.000000,0.000000,0.000000}%
\pgfsetstrokecolor{textcolor}%
\pgfsetfillcolor{textcolor}%
\pgftext[x=0.966792in,y=1.870055in,left,base]{\color{textcolor}\rmfamily\fontsize{10.000000}{12.000000}\selectfont normierte Messdaten}%
\end{pgfscope}%
\begin{pgfscope}%
\pgfsetrectcap%
\pgfsetroundjoin%
\pgfsetlinewidth{1.505625pt}%
\definecolor{currentstroke}{rgb}{1.000000,0.498039,0.054902}%
\pgfsetstrokecolor{currentstroke}%
\pgfsetdash{}{0pt}%
\pgfpathmoveto{\pgfqpoint{0.577903in}{1.623295in}}%
\pgfpathlineto{\pgfqpoint{0.716792in}{1.623295in}}%
\pgfpathlineto{\pgfqpoint{0.855680in}{1.623295in}}%
\pgfusepath{stroke}%
\end{pgfscope}%
\begin{pgfscope}%
\definecolor{textcolor}{rgb}{0.000000,0.000000,0.000000}%
\pgfsetstrokecolor{textcolor}%
\pgfsetfillcolor{textcolor}%
\pgftext[x=0.966792in, y=1.658101in, left, base]{\color{textcolor}\rmfamily\fontsize{10.000000}{12.000000}\selectfont normierte Referenzwert \(\displaystyle \bar{\beta}\)}%
\end{pgfscope}%
\begin{pgfscope}%
\definecolor{textcolor}{rgb}{0.000000,0.000000,0.000000}%
\pgfsetstrokecolor{textcolor}%
\pgfsetfillcolor{textcolor}%
\pgftext[x=0.966792in, y=1.506126in, left, base]{\color{textcolor}\rmfamily\fontsize{10.000000}{12.000000}\selectfont \cite[Abb. 2]{prieto-x-ray-2005}}%
\end{pgfscope}%
\end{pgfpicture}%
\makeatother%
\endgroup%

    \caption{Absorptionsspektrum nähe Resonanzphotonenenergien von Gd (blau) der zu untersuchenden Probe und (orange) Referenzwert für Gd.}
    \label{fig:rzp_phi_ev}
\end{figure}
\noindent
Die Streubilder werden somit an der Motorposition $\varphi_\text{\gls{rzp}} = \num{-67}$ aufgenommen, wo der höchste \gls{xmcd}-Kontrast erwartet wird. An der Motorposition $\varphi_\text{\gls{rzp}} = \num{-52}$ wird eine Kontrollmessung an der Photonenenergie $h\nu_\text{Gd, Off-Res} \approx \SI{1163}{\eV}$ gemacht, die nachgewiesen weit von beiden Resonanzenergien $h\nu_\text{Gd, M5}$ und $h\nu_\text{Gd, M4}$ liegt und dem zufolge keine resonante magnetische Streuung beobachtet werden soll. Das erwartete Ein-Photon-Signal an der Photonenenenergie $h\nu_\text{Gd, Off-Res}$ 
\begin{equation}
    W_\text{Gd, Off-Res} = \SI{176(1)}{\adu}
\end{equation}
ergibt sich nach Gl. (\ref{eq:adu_to_ev}) und unterscheidet sich lediglich um $\SI{2}{\percent}$ von $W_\text{Gd, M5} = \SI{180(1)}{\adu}$, was im Endeffekt dieselben Parameter vom Auswertungsverfahren für die Photonendetektion zulässt.

\section{Punktspreizfunktion einzelner Photonenereignisse}
\label{text:punktspreizfunktion}
Um die Punktspreizfunktion eines isolierten Photons zu untersuchen, werden die einzeln detektierten Photonen im \qtyproduct{100 x 100}{\px} Direktstrahlbereich gesucht. Zunächst werden diejenigen Pixel betrachtet, deren Werte im Intervall von \SI{80}{\adu} bis \SI{200}{\adu} liegen. Um die Überlagerung der Ladungsverteilung der benachbarten Photonen-Ereignisse auszuschließen, werden nur diejenigen Photonen-Ereignisse erfasst, die keine Photonen-Ereignisse im \SI{5}{px}-Umkreis haben. Um die Fehldetektion des Rauschens möglichst zu verringern, wird eine Nebenbedingung auferlegt und zwar, dass die Summe über die \qtyproduct{3 x 3}{\px}-Umgebung um ein  Photonen-Ereignis im Intervall von $(180-3\sigma_{3\times 3})$ \si{\adu} bis $(180+3\sigma_{3\times 3})$ \si{\adu} liegen muss, wobei $\sigma_{3\times 3}$ die Standardabweichung dieser Summe ist und sich als $\sigma_{3\times 3} = \sqrt{9}\sigma_R = 3 \sigma_R$ berechnet.
% \noindent
% Das Quadrat der Standardabweichung (Varianz) $\sigma_\Sigma^2$ von Summe von $i$ Zufallsvariablen, die voneinander nicht abhängen, ergibt sich als die Summe der Varianzen jeder $i$-ten Zufallsvariable $\sigma_i^2$
% \begin{equation}
%     \sigma_\Sigma^2 = \sum_{i} \sigma_i^2.
%     \label{eq:summ_varianz}
% \end{equation}
% \noindent
% Somit kann die Standardabweichung der Summe über \qtyproduct{3 x 3}{\px}-Umgebung
% \begin{equation}
%     \sigma_{3\times 3}^2 = \sum_{i=1}^9\sigma_R^2 \Rightarrow \sigma_{3\times 3} = 3 \sigma_R
% \end{equation}
% berechnet werden.

\noindent
Zum Schluss werden \qtyproduct{5 x 5}{\px}-Bereiche um den Pixel, welche die Bedingungen im oberen Paragraph erfüllen, ausgeschnitten. Die \qtyproduct{5 x 5}{\px}-Bereiche werden über die Anzahl der Bereiche gemittelt und die Standardabweichung jedes Pixels $\sigma_{S+R}$ im \qtyproduct{5 x 5}{\px}-Bereich wird berechnet. Die Standardabweichung $\sigma_{S+R}$ enthält allerdings die Standardabweichung des Detektorrauschens $\sigma_{R}$ und die Standardabweichung der ADU-Wert Verteilung eines Photons $\sigma_{S}$. Diese sind voneinander unabhängig. Daher kann kann die Standardabweichung $\sigma_{S}$ wie folgt ermittelt werden:
\begin{equation}
    \sigma_{S+R}^2 = \sigma_{S}^2 + \sigma_{R}^2 \Rightarrow \sigma_{S} = \sqrt{\sigma_{S+R}^2 - \sigma_{R}^2}
    \label{eq:std_entkopplung}
\end{equation}
\begin{figure}[H]
    \centering
    %% Creator: Matplotlib, PGF backend
%%
%% To include the figure in your LaTeX document, write
%%   \input{<filename>.pgf}
%%
%% Make sure the required packages are loaded in your preamble
%%   \usepackage{pgf}
%%
%% Also ensure that all the required font packages are loaded; for instance,
%% the lmodern package is sometimes necessary when using math font.
%%   \usepackage{lmodern}
%%
%% Figures using additional raster images can only be included by \input if
%% they are in the same directory as the main LaTeX file. For loading figures
%% from other directories you can use the `import` package
%%   \usepackage{import}
%%
%% and then include the figures with
%%   \import{<path to file>}{<filename>.pgf}
%%
%% Matplotlib used the following preamble
%%   \usepackage{amsmath} \usepackage[utf8]{inputenc} \usepackage[T1]{fontenc} \usepackage[output-decimal-marker={,},print-unity-mantissa=false]{siunitx} \sisetup{per-mode=fraction, separate-uncertainty = true, locale = DE} \usepackage[acronym, toc, section=section, nonumberlist, nopostdot]{glossaries-extra} \DeclareSIUnit\adu{\text{ADU}} \DeclareSIUnit\px{\text{px}} \DeclareSIUnit\photons{\text{Pho\-to\-nen}} \DeclareSIUnit\photon{\text{Pho\-ton}}
%%
\begingroup%
\makeatletter%
\begin{pgfpicture}%
\pgfpathrectangle{\pgfpointorigin}{\pgfqpoint{6.517066in}{6.392126in}}%
\pgfusepath{use as bounding box, clip}%
\begin{pgfscope}%
\pgfsetbuttcap%
\pgfsetmiterjoin%
\pgfsetlinewidth{0.000000pt}%
\definecolor{currentstroke}{rgb}{1.000000,1.000000,1.000000}%
\pgfsetstrokecolor{currentstroke}%
\pgfsetstrokeopacity{0.000000}%
\pgfsetdash{}{0pt}%
\pgfpathmoveto{\pgfqpoint{0.000000in}{0.000000in}}%
\pgfpathlineto{\pgfqpoint{6.517066in}{0.000000in}}%
\pgfpathlineto{\pgfqpoint{6.517066in}{6.392126in}}%
\pgfpathlineto{\pgfqpoint{0.000000in}{6.392126in}}%
\pgfpathlineto{\pgfqpoint{0.000000in}{0.000000in}}%
\pgfpathclose%
\pgfusepath{}%
\end{pgfscope}%
\begin{pgfscope}%
\pgfsetbuttcap%
\pgfsetmiterjoin%
\pgfsetlinewidth{0.000000pt}%
\definecolor{currentstroke}{rgb}{1.000000,1.000000,1.000000}%
\pgfsetstrokecolor{currentstroke}%
\pgfsetstrokeopacity{0.000000}%
\pgfsetdash{}{0pt}%
\pgfpathmoveto{\pgfqpoint{-0.040795in}{3.437576in}}%
\pgfpathlineto{\pgfqpoint{6.439205in}{3.437576in}}%
\pgfpathlineto{\pgfqpoint{6.439205in}{6.562576in}}%
\pgfpathlineto{\pgfqpoint{-0.040795in}{6.562576in}}%
\pgfpathlineto{\pgfqpoint{-0.040795in}{3.437576in}}%
\pgfpathclose%
\pgfusepath{}%
\end{pgfscope}%
\begin{pgfscope}%
\pgfsetbuttcap%
\pgfsetmiterjoin%
\definecolor{currentfill}{rgb}{1.000000,1.000000,1.000000}%
\pgfsetfillcolor{currentfill}%
\pgfsetlinewidth{0.000000pt}%
\definecolor{currentstroke}{rgb}{0.000000,0.000000,0.000000}%
\pgfsetstrokecolor{currentstroke}%
\pgfsetstrokeopacity{0.000000}%
\pgfsetdash{}{0pt}%
\pgfpathmoveto{\pgfqpoint{0.275398in}{5.050076in}}%
\pgfpathlineto{\pgfqpoint{1.306988in}{5.050076in}}%
\pgfpathlineto{\pgfqpoint{1.306988in}{6.081667in}}%
\pgfpathlineto{\pgfqpoint{0.275398in}{6.081667in}}%
\pgfpathlineto{\pgfqpoint{0.275398in}{5.050076in}}%
\pgfpathclose%
\pgfusepath{fill}%
\end{pgfscope}%
\begin{pgfscope}%
\pgfsys@transformshift{0.276000in}{5.050126in}%
\pgftext[left,bottom]{\includegraphics[interpolate=true,width=1.032000in,height=1.032000in]{examples_average_std_5x5_hotspot-img0.png}}%
\end{pgfscope}%
\begin{pgfscope}%
\pgfsetrectcap%
\pgfsetmiterjoin%
\pgfsetlinewidth{0.803000pt}%
\definecolor{currentstroke}{rgb}{0.000000,0.000000,0.000000}%
\pgfsetstrokecolor{currentstroke}%
\pgfsetdash{}{0pt}%
\pgfpathmoveto{\pgfqpoint{0.275398in}{5.050076in}}%
\pgfpathlineto{\pgfqpoint{0.275398in}{6.081667in}}%
\pgfusepath{stroke}%
\end{pgfscope}%
\begin{pgfscope}%
\pgfsetrectcap%
\pgfsetmiterjoin%
\pgfsetlinewidth{0.803000pt}%
\definecolor{currentstroke}{rgb}{0.000000,0.000000,0.000000}%
\pgfsetstrokecolor{currentstroke}%
\pgfsetdash{}{0pt}%
\pgfpathmoveto{\pgfqpoint{1.306988in}{5.050076in}}%
\pgfpathlineto{\pgfqpoint{1.306988in}{6.081667in}}%
\pgfusepath{stroke}%
\end{pgfscope}%
\begin{pgfscope}%
\pgfsetrectcap%
\pgfsetmiterjoin%
\pgfsetlinewidth{0.803000pt}%
\definecolor{currentstroke}{rgb}{0.000000,0.000000,0.000000}%
\pgfsetstrokecolor{currentstroke}%
\pgfsetdash{}{0pt}%
\pgfpathmoveto{\pgfqpoint{0.275398in}{5.050076in}}%
\pgfpathlineto{\pgfqpoint{1.306988in}{5.050076in}}%
\pgfusepath{stroke}%
\end{pgfscope}%
\begin{pgfscope}%
\pgfsetrectcap%
\pgfsetmiterjoin%
\pgfsetlinewidth{0.803000pt}%
\definecolor{currentstroke}{rgb}{0.000000,0.000000,0.000000}%
\pgfsetstrokecolor{currentstroke}%
\pgfsetdash{}{0pt}%
\pgfpathmoveto{\pgfqpoint{0.275398in}{6.081667in}}%
\pgfpathlineto{\pgfqpoint{1.306988in}{6.081667in}}%
\pgfusepath{stroke}%
\end{pgfscope}%
\begin{pgfscope}%
\definecolor{textcolor}{rgb}{0.000000,0.000000,0.000000}%
\pgfsetstrokecolor{textcolor}%
\pgfsetfillcolor{textcolor}%
\pgftext[x=0.069080in,y=6.287985in,left,base]{\color{textcolor}\rmfamily\fontsize{10.000000}{12.000000}\selectfont (a)}%
\end{pgfscope}%
\begin{pgfscope}%
\pgfsetbuttcap%
\pgfsetmiterjoin%
\definecolor{currentfill}{rgb}{1.000000,1.000000,1.000000}%
\pgfsetfillcolor{currentfill}%
\pgfsetlinewidth{0.000000pt}%
\definecolor{currentstroke}{rgb}{0.000000,0.000000,0.000000}%
\pgfsetstrokecolor{currentstroke}%
\pgfsetstrokeopacity{0.000000}%
\pgfsetdash{}{0pt}%
\pgfpathmoveto{\pgfqpoint{1.406988in}{5.050076in}}%
\pgfpathlineto{\pgfqpoint{2.438579in}{5.050076in}}%
\pgfpathlineto{\pgfqpoint{2.438579in}{6.081667in}}%
\pgfpathlineto{\pgfqpoint{1.406988in}{6.081667in}}%
\pgfpathlineto{\pgfqpoint{1.406988in}{5.050076in}}%
\pgfpathclose%
\pgfusepath{fill}%
\end{pgfscope}%
\begin{pgfscope}%
\pgfsys@transformshift{1.406000in}{5.050126in}%
\pgftext[left,bottom]{\includegraphics[interpolate=true,width=1.032000in,height=1.032000in]{examples_average_std_5x5_hotspot-img1.png}}%
\end{pgfscope}%
\begin{pgfscope}%
\pgfsetrectcap%
\pgfsetmiterjoin%
\pgfsetlinewidth{0.803000pt}%
\definecolor{currentstroke}{rgb}{0.000000,0.000000,0.000000}%
\pgfsetstrokecolor{currentstroke}%
\pgfsetdash{}{0pt}%
\pgfpathmoveto{\pgfqpoint{1.406988in}{5.050076in}}%
\pgfpathlineto{\pgfqpoint{1.406988in}{6.081667in}}%
\pgfusepath{stroke}%
\end{pgfscope}%
\begin{pgfscope}%
\pgfsetrectcap%
\pgfsetmiterjoin%
\pgfsetlinewidth{0.803000pt}%
\definecolor{currentstroke}{rgb}{0.000000,0.000000,0.000000}%
\pgfsetstrokecolor{currentstroke}%
\pgfsetdash{}{0pt}%
\pgfpathmoveto{\pgfqpoint{2.438579in}{5.050076in}}%
\pgfpathlineto{\pgfqpoint{2.438579in}{6.081667in}}%
\pgfusepath{stroke}%
\end{pgfscope}%
\begin{pgfscope}%
\pgfsetrectcap%
\pgfsetmiterjoin%
\pgfsetlinewidth{0.803000pt}%
\definecolor{currentstroke}{rgb}{0.000000,0.000000,0.000000}%
\pgfsetstrokecolor{currentstroke}%
\pgfsetdash{}{0pt}%
\pgfpathmoveto{\pgfqpoint{1.406988in}{5.050076in}}%
\pgfpathlineto{\pgfqpoint{2.438579in}{5.050076in}}%
\pgfusepath{stroke}%
\end{pgfscope}%
\begin{pgfscope}%
\pgfsetrectcap%
\pgfsetmiterjoin%
\pgfsetlinewidth{0.803000pt}%
\definecolor{currentstroke}{rgb}{0.000000,0.000000,0.000000}%
\pgfsetstrokecolor{currentstroke}%
\pgfsetdash{}{0pt}%
\pgfpathmoveto{\pgfqpoint{1.406988in}{6.081667in}}%
\pgfpathlineto{\pgfqpoint{2.438579in}{6.081667in}}%
\pgfusepath{stroke}%
\end{pgfscope}%
\begin{pgfscope}%
\pgfsetbuttcap%
\pgfsetmiterjoin%
\definecolor{currentfill}{rgb}{1.000000,1.000000,1.000000}%
\pgfsetfillcolor{currentfill}%
\pgfsetlinewidth{0.000000pt}%
\definecolor{currentstroke}{rgb}{0.000000,0.000000,0.000000}%
\pgfsetstrokecolor{currentstroke}%
\pgfsetstrokeopacity{0.000000}%
\pgfsetdash{}{0pt}%
\pgfpathmoveto{\pgfqpoint{2.538579in}{5.050076in}}%
\pgfpathlineto{\pgfqpoint{3.570170in}{5.050076in}}%
\pgfpathlineto{\pgfqpoint{3.570170in}{6.081667in}}%
\pgfpathlineto{\pgfqpoint{2.538579in}{6.081667in}}%
\pgfpathlineto{\pgfqpoint{2.538579in}{5.050076in}}%
\pgfpathclose%
\pgfusepath{fill}%
\end{pgfscope}%
\begin{pgfscope}%
\pgfsys@transformshift{2.538000in}{5.050126in}%
\pgftext[left,bottom]{\includegraphics[interpolate=true,width=1.032000in,height=1.032000in]{examples_average_std_5x5_hotspot-img2.png}}%
\end{pgfscope}%
\begin{pgfscope}%
\pgfsetrectcap%
\pgfsetmiterjoin%
\pgfsetlinewidth{0.803000pt}%
\definecolor{currentstroke}{rgb}{0.000000,0.000000,0.000000}%
\pgfsetstrokecolor{currentstroke}%
\pgfsetdash{}{0pt}%
\pgfpathmoveto{\pgfqpoint{2.538579in}{5.050076in}}%
\pgfpathlineto{\pgfqpoint{2.538579in}{6.081667in}}%
\pgfusepath{stroke}%
\end{pgfscope}%
\begin{pgfscope}%
\pgfsetrectcap%
\pgfsetmiterjoin%
\pgfsetlinewidth{0.803000pt}%
\definecolor{currentstroke}{rgb}{0.000000,0.000000,0.000000}%
\pgfsetstrokecolor{currentstroke}%
\pgfsetdash{}{0pt}%
\pgfpathmoveto{\pgfqpoint{3.570170in}{5.050076in}}%
\pgfpathlineto{\pgfqpoint{3.570170in}{6.081667in}}%
\pgfusepath{stroke}%
\end{pgfscope}%
\begin{pgfscope}%
\pgfsetrectcap%
\pgfsetmiterjoin%
\pgfsetlinewidth{0.803000pt}%
\definecolor{currentstroke}{rgb}{0.000000,0.000000,0.000000}%
\pgfsetstrokecolor{currentstroke}%
\pgfsetdash{}{0pt}%
\pgfpathmoveto{\pgfqpoint{2.538579in}{5.050076in}}%
\pgfpathlineto{\pgfqpoint{3.570170in}{5.050076in}}%
\pgfusepath{stroke}%
\end{pgfscope}%
\begin{pgfscope}%
\pgfsetrectcap%
\pgfsetmiterjoin%
\pgfsetlinewidth{0.803000pt}%
\definecolor{currentstroke}{rgb}{0.000000,0.000000,0.000000}%
\pgfsetstrokecolor{currentstroke}%
\pgfsetdash{}{0pt}%
\pgfpathmoveto{\pgfqpoint{2.538579in}{6.081667in}}%
\pgfpathlineto{\pgfqpoint{3.570170in}{6.081667in}}%
\pgfusepath{stroke}%
\end{pgfscope}%
\begin{pgfscope}%
\pgfsetbuttcap%
\pgfsetmiterjoin%
\definecolor{currentfill}{rgb}{1.000000,1.000000,1.000000}%
\pgfsetfillcolor{currentfill}%
\pgfsetlinewidth{0.000000pt}%
\definecolor{currentstroke}{rgb}{0.000000,0.000000,0.000000}%
\pgfsetstrokecolor{currentstroke}%
\pgfsetstrokeopacity{0.000000}%
\pgfsetdash{}{0pt}%
\pgfpathmoveto{\pgfqpoint{3.670170in}{5.050076in}}%
\pgfpathlineto{\pgfqpoint{4.701760in}{5.050076in}}%
\pgfpathlineto{\pgfqpoint{4.701760in}{6.081667in}}%
\pgfpathlineto{\pgfqpoint{3.670170in}{6.081667in}}%
\pgfpathlineto{\pgfqpoint{3.670170in}{5.050076in}}%
\pgfpathclose%
\pgfusepath{fill}%
\end{pgfscope}%
\begin{pgfscope}%
\pgfsys@transformshift{3.670000in}{5.050126in}%
\pgftext[left,bottom]{\includegraphics[interpolate=true,width=1.032000in,height=1.032000in]{examples_average_std_5x5_hotspot-img3.png}}%
\end{pgfscope}%
\begin{pgfscope}%
\pgfsetrectcap%
\pgfsetmiterjoin%
\pgfsetlinewidth{0.803000pt}%
\definecolor{currentstroke}{rgb}{0.000000,0.000000,0.000000}%
\pgfsetstrokecolor{currentstroke}%
\pgfsetdash{}{0pt}%
\pgfpathmoveto{\pgfqpoint{3.670170in}{5.050076in}}%
\pgfpathlineto{\pgfqpoint{3.670170in}{6.081667in}}%
\pgfusepath{stroke}%
\end{pgfscope}%
\begin{pgfscope}%
\pgfsetrectcap%
\pgfsetmiterjoin%
\pgfsetlinewidth{0.803000pt}%
\definecolor{currentstroke}{rgb}{0.000000,0.000000,0.000000}%
\pgfsetstrokecolor{currentstroke}%
\pgfsetdash{}{0pt}%
\pgfpathmoveto{\pgfqpoint{4.701760in}{5.050076in}}%
\pgfpathlineto{\pgfqpoint{4.701760in}{6.081667in}}%
\pgfusepath{stroke}%
\end{pgfscope}%
\begin{pgfscope}%
\pgfsetrectcap%
\pgfsetmiterjoin%
\pgfsetlinewidth{0.803000pt}%
\definecolor{currentstroke}{rgb}{0.000000,0.000000,0.000000}%
\pgfsetstrokecolor{currentstroke}%
\pgfsetdash{}{0pt}%
\pgfpathmoveto{\pgfqpoint{3.670170in}{5.050076in}}%
\pgfpathlineto{\pgfqpoint{4.701760in}{5.050076in}}%
\pgfusepath{stroke}%
\end{pgfscope}%
\begin{pgfscope}%
\pgfsetrectcap%
\pgfsetmiterjoin%
\pgfsetlinewidth{0.803000pt}%
\definecolor{currentstroke}{rgb}{0.000000,0.000000,0.000000}%
\pgfsetstrokecolor{currentstroke}%
\pgfsetdash{}{0pt}%
\pgfpathmoveto{\pgfqpoint{3.670170in}{6.081667in}}%
\pgfpathlineto{\pgfqpoint{4.701760in}{6.081667in}}%
\pgfusepath{stroke}%
\end{pgfscope}%
\begin{pgfscope}%
\pgfsetbuttcap%
\pgfsetmiterjoin%
\definecolor{currentfill}{rgb}{1.000000,1.000000,1.000000}%
\pgfsetfillcolor{currentfill}%
\pgfsetlinewidth{0.000000pt}%
\definecolor{currentstroke}{rgb}{0.000000,0.000000,0.000000}%
\pgfsetstrokecolor{currentstroke}%
\pgfsetstrokeopacity{0.000000}%
\pgfsetdash{}{0pt}%
\pgfpathmoveto{\pgfqpoint{4.801760in}{5.050076in}}%
\pgfpathlineto{\pgfqpoint{5.833351in}{5.050076in}}%
\pgfpathlineto{\pgfqpoint{5.833351in}{6.081667in}}%
\pgfpathlineto{\pgfqpoint{4.801760in}{6.081667in}}%
\pgfpathlineto{\pgfqpoint{4.801760in}{5.050076in}}%
\pgfpathclose%
\pgfusepath{fill}%
\end{pgfscope}%
\begin{pgfscope}%
\pgfsys@transformshift{4.802000in}{5.050126in}%
\pgftext[left,bottom]{\includegraphics[interpolate=true,width=1.032000in,height=1.032000in]{examples_average_std_5x5_hotspot-img4.png}}%
\end{pgfscope}%
\begin{pgfscope}%
\pgfsetrectcap%
\pgfsetmiterjoin%
\pgfsetlinewidth{0.803000pt}%
\definecolor{currentstroke}{rgb}{0.000000,0.000000,0.000000}%
\pgfsetstrokecolor{currentstroke}%
\pgfsetdash{}{0pt}%
\pgfpathmoveto{\pgfqpoint{4.801760in}{5.050076in}}%
\pgfpathlineto{\pgfqpoint{4.801760in}{6.081667in}}%
\pgfusepath{stroke}%
\end{pgfscope}%
\begin{pgfscope}%
\pgfsetrectcap%
\pgfsetmiterjoin%
\pgfsetlinewidth{0.803000pt}%
\definecolor{currentstroke}{rgb}{0.000000,0.000000,0.000000}%
\pgfsetstrokecolor{currentstroke}%
\pgfsetdash{}{0pt}%
\pgfpathmoveto{\pgfqpoint{5.833351in}{5.050076in}}%
\pgfpathlineto{\pgfqpoint{5.833351in}{6.081667in}}%
\pgfusepath{stroke}%
\end{pgfscope}%
\begin{pgfscope}%
\pgfsetrectcap%
\pgfsetmiterjoin%
\pgfsetlinewidth{0.803000pt}%
\definecolor{currentstroke}{rgb}{0.000000,0.000000,0.000000}%
\pgfsetstrokecolor{currentstroke}%
\pgfsetdash{}{0pt}%
\pgfpathmoveto{\pgfqpoint{4.801760in}{5.050076in}}%
\pgfpathlineto{\pgfqpoint{5.833351in}{5.050076in}}%
\pgfusepath{stroke}%
\end{pgfscope}%
\begin{pgfscope}%
\pgfsetrectcap%
\pgfsetmiterjoin%
\pgfsetlinewidth{0.803000pt}%
\definecolor{currentstroke}{rgb}{0.000000,0.000000,0.000000}%
\pgfsetstrokecolor{currentstroke}%
\pgfsetdash{}{0pt}%
\pgfpathmoveto{\pgfqpoint{4.801760in}{6.081667in}}%
\pgfpathlineto{\pgfqpoint{5.833351in}{6.081667in}}%
\pgfusepath{stroke}%
\end{pgfscope}%
\begin{pgfscope}%
\pgfsetbuttcap%
\pgfsetmiterjoin%
\definecolor{currentfill}{rgb}{1.000000,1.000000,1.000000}%
\pgfsetfillcolor{currentfill}%
\pgfsetlinewidth{0.000000pt}%
\definecolor{currentstroke}{rgb}{0.000000,0.000000,0.000000}%
\pgfsetstrokecolor{currentstroke}%
\pgfsetstrokeopacity{0.000000}%
\pgfsetdash{}{0pt}%
\pgfpathmoveto{\pgfqpoint{0.275398in}{3.918486in}}%
\pgfpathlineto{\pgfqpoint{1.306988in}{3.918486in}}%
\pgfpathlineto{\pgfqpoint{1.306988in}{4.950076in}}%
\pgfpathlineto{\pgfqpoint{0.275398in}{4.950076in}}%
\pgfpathlineto{\pgfqpoint{0.275398in}{3.918486in}}%
\pgfpathclose%
\pgfusepath{fill}%
\end{pgfscope}%
\begin{pgfscope}%
\pgfsys@transformshift{0.276000in}{3.918126in}%
\pgftext[left,bottom]{\includegraphics[interpolate=true,width=1.032000in,height=1.032000in]{examples_average_std_5x5_hotspot-img5.png}}%
\end{pgfscope}%
\begin{pgfscope}%
\pgfsetrectcap%
\pgfsetmiterjoin%
\pgfsetlinewidth{0.803000pt}%
\definecolor{currentstroke}{rgb}{0.000000,0.000000,0.000000}%
\pgfsetstrokecolor{currentstroke}%
\pgfsetdash{}{0pt}%
\pgfpathmoveto{\pgfqpoint{0.275398in}{3.918486in}}%
\pgfpathlineto{\pgfqpoint{0.275398in}{4.950076in}}%
\pgfusepath{stroke}%
\end{pgfscope}%
\begin{pgfscope}%
\pgfsetrectcap%
\pgfsetmiterjoin%
\pgfsetlinewidth{0.803000pt}%
\definecolor{currentstroke}{rgb}{0.000000,0.000000,0.000000}%
\pgfsetstrokecolor{currentstroke}%
\pgfsetdash{}{0pt}%
\pgfpathmoveto{\pgfqpoint{1.306988in}{3.918486in}}%
\pgfpathlineto{\pgfqpoint{1.306988in}{4.950076in}}%
\pgfusepath{stroke}%
\end{pgfscope}%
\begin{pgfscope}%
\pgfsetrectcap%
\pgfsetmiterjoin%
\pgfsetlinewidth{0.803000pt}%
\definecolor{currentstroke}{rgb}{0.000000,0.000000,0.000000}%
\pgfsetstrokecolor{currentstroke}%
\pgfsetdash{}{0pt}%
\pgfpathmoveto{\pgfqpoint{0.275398in}{3.918486in}}%
\pgfpathlineto{\pgfqpoint{1.306988in}{3.918486in}}%
\pgfusepath{stroke}%
\end{pgfscope}%
\begin{pgfscope}%
\pgfsetrectcap%
\pgfsetmiterjoin%
\pgfsetlinewidth{0.803000pt}%
\definecolor{currentstroke}{rgb}{0.000000,0.000000,0.000000}%
\pgfsetstrokecolor{currentstroke}%
\pgfsetdash{}{0pt}%
\pgfpathmoveto{\pgfqpoint{0.275398in}{4.950076in}}%
\pgfpathlineto{\pgfqpoint{1.306988in}{4.950076in}}%
\pgfusepath{stroke}%
\end{pgfscope}%
\begin{pgfscope}%
\pgfsetbuttcap%
\pgfsetmiterjoin%
\definecolor{currentfill}{rgb}{1.000000,1.000000,1.000000}%
\pgfsetfillcolor{currentfill}%
\pgfsetlinewidth{0.000000pt}%
\definecolor{currentstroke}{rgb}{0.000000,0.000000,0.000000}%
\pgfsetstrokecolor{currentstroke}%
\pgfsetstrokeopacity{0.000000}%
\pgfsetdash{}{0pt}%
\pgfpathmoveto{\pgfqpoint{1.406988in}{3.918486in}}%
\pgfpathlineto{\pgfqpoint{2.438579in}{3.918486in}}%
\pgfpathlineto{\pgfqpoint{2.438579in}{4.950076in}}%
\pgfpathlineto{\pgfqpoint{1.406988in}{4.950076in}}%
\pgfpathlineto{\pgfqpoint{1.406988in}{3.918486in}}%
\pgfpathclose%
\pgfusepath{fill}%
\end{pgfscope}%
\begin{pgfscope}%
\pgfsys@transformshift{1.406000in}{3.918126in}%
\pgftext[left,bottom]{\includegraphics[interpolate=true,width=1.032000in,height=1.032000in]{examples_average_std_5x5_hotspot-img6.png}}%
\end{pgfscope}%
\begin{pgfscope}%
\pgfsetrectcap%
\pgfsetmiterjoin%
\pgfsetlinewidth{0.803000pt}%
\definecolor{currentstroke}{rgb}{0.000000,0.000000,0.000000}%
\pgfsetstrokecolor{currentstroke}%
\pgfsetdash{}{0pt}%
\pgfpathmoveto{\pgfqpoint{1.406988in}{3.918486in}}%
\pgfpathlineto{\pgfqpoint{1.406988in}{4.950076in}}%
\pgfusepath{stroke}%
\end{pgfscope}%
\begin{pgfscope}%
\pgfsetrectcap%
\pgfsetmiterjoin%
\pgfsetlinewidth{0.803000pt}%
\definecolor{currentstroke}{rgb}{0.000000,0.000000,0.000000}%
\pgfsetstrokecolor{currentstroke}%
\pgfsetdash{}{0pt}%
\pgfpathmoveto{\pgfqpoint{2.438579in}{3.918486in}}%
\pgfpathlineto{\pgfqpoint{2.438579in}{4.950076in}}%
\pgfusepath{stroke}%
\end{pgfscope}%
\begin{pgfscope}%
\pgfsetrectcap%
\pgfsetmiterjoin%
\pgfsetlinewidth{0.803000pt}%
\definecolor{currentstroke}{rgb}{0.000000,0.000000,0.000000}%
\pgfsetstrokecolor{currentstroke}%
\pgfsetdash{}{0pt}%
\pgfpathmoveto{\pgfqpoint{1.406988in}{3.918486in}}%
\pgfpathlineto{\pgfqpoint{2.438579in}{3.918486in}}%
\pgfusepath{stroke}%
\end{pgfscope}%
\begin{pgfscope}%
\pgfsetrectcap%
\pgfsetmiterjoin%
\pgfsetlinewidth{0.803000pt}%
\definecolor{currentstroke}{rgb}{0.000000,0.000000,0.000000}%
\pgfsetstrokecolor{currentstroke}%
\pgfsetdash{}{0pt}%
\pgfpathmoveto{\pgfqpoint{1.406988in}{4.950076in}}%
\pgfpathlineto{\pgfqpoint{2.438579in}{4.950076in}}%
\pgfusepath{stroke}%
\end{pgfscope}%
\begin{pgfscope}%
\pgfsetbuttcap%
\pgfsetmiterjoin%
\definecolor{currentfill}{rgb}{1.000000,1.000000,1.000000}%
\pgfsetfillcolor{currentfill}%
\pgfsetlinewidth{0.000000pt}%
\definecolor{currentstroke}{rgb}{0.000000,0.000000,0.000000}%
\pgfsetstrokecolor{currentstroke}%
\pgfsetstrokeopacity{0.000000}%
\pgfsetdash{}{0pt}%
\pgfpathmoveto{\pgfqpoint{2.538579in}{3.918486in}}%
\pgfpathlineto{\pgfqpoint{3.570170in}{3.918486in}}%
\pgfpathlineto{\pgfqpoint{3.570170in}{4.950076in}}%
\pgfpathlineto{\pgfqpoint{2.538579in}{4.950076in}}%
\pgfpathlineto{\pgfqpoint{2.538579in}{3.918486in}}%
\pgfpathclose%
\pgfusepath{fill}%
\end{pgfscope}%
\begin{pgfscope}%
\pgfsys@transformshift{2.538000in}{3.918126in}%
\pgftext[left,bottom]{\includegraphics[interpolate=true,width=1.032000in,height=1.032000in]{examples_average_std_5x5_hotspot-img7.png}}%
\end{pgfscope}%
\begin{pgfscope}%
\pgfsetrectcap%
\pgfsetmiterjoin%
\pgfsetlinewidth{0.803000pt}%
\definecolor{currentstroke}{rgb}{0.000000,0.000000,0.000000}%
\pgfsetstrokecolor{currentstroke}%
\pgfsetdash{}{0pt}%
\pgfpathmoveto{\pgfqpoint{2.538579in}{3.918486in}}%
\pgfpathlineto{\pgfqpoint{2.538579in}{4.950076in}}%
\pgfusepath{stroke}%
\end{pgfscope}%
\begin{pgfscope}%
\pgfsetrectcap%
\pgfsetmiterjoin%
\pgfsetlinewidth{0.803000pt}%
\definecolor{currentstroke}{rgb}{0.000000,0.000000,0.000000}%
\pgfsetstrokecolor{currentstroke}%
\pgfsetdash{}{0pt}%
\pgfpathmoveto{\pgfqpoint{3.570170in}{3.918486in}}%
\pgfpathlineto{\pgfqpoint{3.570170in}{4.950076in}}%
\pgfusepath{stroke}%
\end{pgfscope}%
\begin{pgfscope}%
\pgfsetrectcap%
\pgfsetmiterjoin%
\pgfsetlinewidth{0.803000pt}%
\definecolor{currentstroke}{rgb}{0.000000,0.000000,0.000000}%
\pgfsetstrokecolor{currentstroke}%
\pgfsetdash{}{0pt}%
\pgfpathmoveto{\pgfqpoint{2.538579in}{3.918486in}}%
\pgfpathlineto{\pgfqpoint{3.570170in}{3.918486in}}%
\pgfusepath{stroke}%
\end{pgfscope}%
\begin{pgfscope}%
\pgfsetrectcap%
\pgfsetmiterjoin%
\pgfsetlinewidth{0.803000pt}%
\definecolor{currentstroke}{rgb}{0.000000,0.000000,0.000000}%
\pgfsetstrokecolor{currentstroke}%
\pgfsetdash{}{0pt}%
\pgfpathmoveto{\pgfqpoint{2.538579in}{4.950076in}}%
\pgfpathlineto{\pgfqpoint{3.570170in}{4.950076in}}%
\pgfusepath{stroke}%
\end{pgfscope}%
\begin{pgfscope}%
\pgfsetbuttcap%
\pgfsetmiterjoin%
\definecolor{currentfill}{rgb}{1.000000,1.000000,1.000000}%
\pgfsetfillcolor{currentfill}%
\pgfsetlinewidth{0.000000pt}%
\definecolor{currentstroke}{rgb}{0.000000,0.000000,0.000000}%
\pgfsetstrokecolor{currentstroke}%
\pgfsetstrokeopacity{0.000000}%
\pgfsetdash{}{0pt}%
\pgfpathmoveto{\pgfqpoint{3.670170in}{3.918486in}}%
\pgfpathlineto{\pgfqpoint{4.701760in}{3.918486in}}%
\pgfpathlineto{\pgfqpoint{4.701760in}{4.950076in}}%
\pgfpathlineto{\pgfqpoint{3.670170in}{4.950076in}}%
\pgfpathlineto{\pgfqpoint{3.670170in}{3.918486in}}%
\pgfpathclose%
\pgfusepath{fill}%
\end{pgfscope}%
\begin{pgfscope}%
\pgfsys@transformshift{3.670000in}{3.918126in}%
\pgftext[left,bottom]{\includegraphics[interpolate=true,width=1.032000in,height=1.032000in]{examples_average_std_5x5_hotspot-img8.png}}%
\end{pgfscope}%
\begin{pgfscope}%
\pgfsetrectcap%
\pgfsetmiterjoin%
\pgfsetlinewidth{0.803000pt}%
\definecolor{currentstroke}{rgb}{0.000000,0.000000,0.000000}%
\pgfsetstrokecolor{currentstroke}%
\pgfsetdash{}{0pt}%
\pgfpathmoveto{\pgfqpoint{3.670170in}{3.918486in}}%
\pgfpathlineto{\pgfqpoint{3.670170in}{4.950076in}}%
\pgfusepath{stroke}%
\end{pgfscope}%
\begin{pgfscope}%
\pgfsetrectcap%
\pgfsetmiterjoin%
\pgfsetlinewidth{0.803000pt}%
\definecolor{currentstroke}{rgb}{0.000000,0.000000,0.000000}%
\pgfsetstrokecolor{currentstroke}%
\pgfsetdash{}{0pt}%
\pgfpathmoveto{\pgfqpoint{4.701760in}{3.918486in}}%
\pgfpathlineto{\pgfqpoint{4.701760in}{4.950076in}}%
\pgfusepath{stroke}%
\end{pgfscope}%
\begin{pgfscope}%
\pgfsetrectcap%
\pgfsetmiterjoin%
\pgfsetlinewidth{0.803000pt}%
\definecolor{currentstroke}{rgb}{0.000000,0.000000,0.000000}%
\pgfsetstrokecolor{currentstroke}%
\pgfsetdash{}{0pt}%
\pgfpathmoveto{\pgfqpoint{3.670170in}{3.918486in}}%
\pgfpathlineto{\pgfqpoint{4.701760in}{3.918486in}}%
\pgfusepath{stroke}%
\end{pgfscope}%
\begin{pgfscope}%
\pgfsetrectcap%
\pgfsetmiterjoin%
\pgfsetlinewidth{0.803000pt}%
\definecolor{currentstroke}{rgb}{0.000000,0.000000,0.000000}%
\pgfsetstrokecolor{currentstroke}%
\pgfsetdash{}{0pt}%
\pgfpathmoveto{\pgfqpoint{3.670170in}{4.950076in}}%
\pgfpathlineto{\pgfqpoint{4.701760in}{4.950076in}}%
\pgfusepath{stroke}%
\end{pgfscope}%
\begin{pgfscope}%
\pgfsetbuttcap%
\pgfsetmiterjoin%
\definecolor{currentfill}{rgb}{1.000000,1.000000,1.000000}%
\pgfsetfillcolor{currentfill}%
\pgfsetlinewidth{0.000000pt}%
\definecolor{currentstroke}{rgb}{0.000000,0.000000,0.000000}%
\pgfsetstrokecolor{currentstroke}%
\pgfsetstrokeopacity{0.000000}%
\pgfsetdash{}{0pt}%
\pgfpathmoveto{\pgfqpoint{4.801760in}{3.918486in}}%
\pgfpathlineto{\pgfqpoint{5.833351in}{3.918486in}}%
\pgfpathlineto{\pgfqpoint{5.833351in}{4.950076in}}%
\pgfpathlineto{\pgfqpoint{4.801760in}{4.950076in}}%
\pgfpathlineto{\pgfqpoint{4.801760in}{3.918486in}}%
\pgfpathclose%
\pgfusepath{fill}%
\end{pgfscope}%
\begin{pgfscope}%
\pgfsys@transformshift{4.802000in}{3.918126in}%
\pgftext[left,bottom]{\includegraphics[interpolate=true,width=1.032000in,height=1.032000in]{examples_average_std_5x5_hotspot-img9.png}}%
\end{pgfscope}%
\begin{pgfscope}%
\pgfsetrectcap%
\pgfsetmiterjoin%
\pgfsetlinewidth{0.803000pt}%
\definecolor{currentstroke}{rgb}{0.000000,0.000000,0.000000}%
\pgfsetstrokecolor{currentstroke}%
\pgfsetdash{}{0pt}%
\pgfpathmoveto{\pgfqpoint{4.801760in}{3.918486in}}%
\pgfpathlineto{\pgfqpoint{4.801760in}{4.950076in}}%
\pgfusepath{stroke}%
\end{pgfscope}%
\begin{pgfscope}%
\pgfsetrectcap%
\pgfsetmiterjoin%
\pgfsetlinewidth{0.803000pt}%
\definecolor{currentstroke}{rgb}{0.000000,0.000000,0.000000}%
\pgfsetstrokecolor{currentstroke}%
\pgfsetdash{}{0pt}%
\pgfpathmoveto{\pgfqpoint{5.833351in}{3.918486in}}%
\pgfpathlineto{\pgfqpoint{5.833351in}{4.950076in}}%
\pgfusepath{stroke}%
\end{pgfscope}%
\begin{pgfscope}%
\pgfsetrectcap%
\pgfsetmiterjoin%
\pgfsetlinewidth{0.803000pt}%
\definecolor{currentstroke}{rgb}{0.000000,0.000000,0.000000}%
\pgfsetstrokecolor{currentstroke}%
\pgfsetdash{}{0pt}%
\pgfpathmoveto{\pgfqpoint{4.801760in}{3.918486in}}%
\pgfpathlineto{\pgfqpoint{5.833351in}{3.918486in}}%
\pgfusepath{stroke}%
\end{pgfscope}%
\begin{pgfscope}%
\pgfsetrectcap%
\pgfsetmiterjoin%
\pgfsetlinewidth{0.803000pt}%
\definecolor{currentstroke}{rgb}{0.000000,0.000000,0.000000}%
\pgfsetstrokecolor{currentstroke}%
\pgfsetdash{}{0pt}%
\pgfpathmoveto{\pgfqpoint{4.801760in}{4.950076in}}%
\pgfpathlineto{\pgfqpoint{5.833351in}{4.950076in}}%
\pgfusepath{stroke}%
\end{pgfscope}%
\begin{pgfscope}%
\pgfsetbuttcap%
\pgfsetmiterjoin%
\definecolor{currentfill}{rgb}{1.000000,1.000000,1.000000}%
\pgfsetfillcolor{currentfill}%
\pgfsetlinewidth{0.000000pt}%
\definecolor{currentstroke}{rgb}{0.000000,0.000000,0.000000}%
\pgfsetstrokecolor{currentstroke}%
\pgfsetstrokeopacity{0.000000}%
\pgfsetdash{}{0pt}%
\pgfpathmoveto{\pgfqpoint{5.933351in}{3.918486in}}%
\pgfpathlineto{\pgfqpoint{6.033351in}{3.918486in}}%
\pgfpathlineto{\pgfqpoint{6.033351in}{6.081667in}}%
\pgfpathlineto{\pgfqpoint{5.933351in}{6.081667in}}%
\pgfpathlineto{\pgfqpoint{5.933351in}{3.918486in}}%
\pgfpathclose%
\pgfusepath{fill}%
\end{pgfscope}%
\begin{pgfscope}%
\pgfpathrectangle{\pgfqpoint{5.933351in}{3.918486in}}{\pgfqpoint{0.100000in}{2.163181in}}%
\pgfusepath{clip}%
\pgfsetbuttcap%
\pgfsetmiterjoin%
\definecolor{currentfill}{rgb}{1.000000,1.000000,1.000000}%
\pgfsetfillcolor{currentfill}%
\pgfsetlinewidth{0.010037pt}%
\definecolor{currentstroke}{rgb}{1.000000,1.000000,1.000000}%
\pgfsetstrokecolor{currentstroke}%
\pgfsetdash{}{0pt}%
\pgfusepath{stroke,fill}%
\end{pgfscope}%
\begin{pgfscope}%
\pgfsys@transformshift{5.934000in}{3.918126in}%
\pgftext[left,bottom]{\includegraphics[interpolate=true,width=0.100000in,height=2.164000in]{examples_average_std_5x5_hotspot-img10.png}}%
\end{pgfscope}%
\begin{pgfscope}%
\pgfsetbuttcap%
\pgfsetroundjoin%
\definecolor{currentfill}{rgb}{0.000000,0.000000,0.000000}%
\pgfsetfillcolor{currentfill}%
\pgfsetlinewidth{0.803000pt}%
\definecolor{currentstroke}{rgb}{0.000000,0.000000,0.000000}%
\pgfsetstrokecolor{currentstroke}%
\pgfsetdash{}{0pt}%
\pgfsys@defobject{currentmarker}{\pgfqpoint{0.000000in}{0.000000in}}{\pgfqpoint{0.048611in}{0.000000in}}{%
\pgfpathmoveto{\pgfqpoint{0.000000in}{0.000000in}}%
\pgfpathlineto{\pgfqpoint{0.048611in}{0.000000in}}%
\pgfusepath{stroke,fill}%
}%
\begin{pgfscope}%
\pgfsys@transformshift{6.033351in}{3.918486in}%
\pgfsys@useobject{currentmarker}{}%
\end{pgfscope}%
\end{pgfscope}%
\begin{pgfscope}%
\definecolor{textcolor}{rgb}{0.000000,0.000000,0.000000}%
\pgfsetstrokecolor{textcolor}%
\pgfsetfillcolor{textcolor}%
\pgftext[x=6.130573in, y=3.870658in, left, base]{\color{textcolor}\rmfamily\fontsize{10.000000}{12.000000}\selectfont -25}%
\end{pgfscope}%
\begin{pgfscope}%
\pgfsetbuttcap%
\pgfsetroundjoin%
\definecolor{currentfill}{rgb}{0.000000,0.000000,0.000000}%
\pgfsetfillcolor{currentfill}%
\pgfsetlinewidth{0.803000pt}%
\definecolor{currentstroke}{rgb}{0.000000,0.000000,0.000000}%
\pgfsetstrokecolor{currentstroke}%
\pgfsetdash{}{0pt}%
\pgfsys@defobject{currentmarker}{\pgfqpoint{0.000000in}{0.000000in}}{\pgfqpoint{0.048611in}{0.000000in}}{%
\pgfpathmoveto{\pgfqpoint{0.000000in}{0.000000in}}%
\pgfpathlineto{\pgfqpoint{0.048611in}{0.000000in}}%
\pgfusepath{stroke,fill}%
}%
\begin{pgfscope}%
\pgfsys@transformshift{6.033351in}{4.267386in}%
\pgfsys@useobject{currentmarker}{}%
\end{pgfscope}%
\end{pgfscope}%
\begin{pgfscope}%
\definecolor{textcolor}{rgb}{0.000000,0.000000,0.000000}%
\pgfsetstrokecolor{textcolor}%
\pgfsetfillcolor{textcolor}%
\pgftext[x=6.130573in, y=4.219558in, left, base]{\color{textcolor}\rmfamily\fontsize{10.000000}{12.000000}\selectfont 0}%
\end{pgfscope}%
\begin{pgfscope}%
\pgfsetbuttcap%
\pgfsetroundjoin%
\definecolor{currentfill}{rgb}{0.000000,0.000000,0.000000}%
\pgfsetfillcolor{currentfill}%
\pgfsetlinewidth{0.803000pt}%
\definecolor{currentstroke}{rgb}{0.000000,0.000000,0.000000}%
\pgfsetstrokecolor{currentstroke}%
\pgfsetdash{}{0pt}%
\pgfsys@defobject{currentmarker}{\pgfqpoint{0.000000in}{0.000000in}}{\pgfqpoint{0.048611in}{0.000000in}}{%
\pgfpathmoveto{\pgfqpoint{0.000000in}{0.000000in}}%
\pgfpathlineto{\pgfqpoint{0.048611in}{0.000000in}}%
\pgfusepath{stroke,fill}%
}%
\begin{pgfscope}%
\pgfsys@transformshift{6.033351in}{4.616286in}%
\pgfsys@useobject{currentmarker}{}%
\end{pgfscope}%
\end{pgfscope}%
\begin{pgfscope}%
\definecolor{textcolor}{rgb}{0.000000,0.000000,0.000000}%
\pgfsetstrokecolor{textcolor}%
\pgfsetfillcolor{textcolor}%
\pgftext[x=6.130573in, y=4.568458in, left, base]{\color{textcolor}\rmfamily\fontsize{10.000000}{12.000000}\selectfont 25}%
\end{pgfscope}%
\begin{pgfscope}%
\pgfsetbuttcap%
\pgfsetroundjoin%
\definecolor{currentfill}{rgb}{0.000000,0.000000,0.000000}%
\pgfsetfillcolor{currentfill}%
\pgfsetlinewidth{0.803000pt}%
\definecolor{currentstroke}{rgb}{0.000000,0.000000,0.000000}%
\pgfsetstrokecolor{currentstroke}%
\pgfsetdash{}{0pt}%
\pgfsys@defobject{currentmarker}{\pgfqpoint{0.000000in}{0.000000in}}{\pgfqpoint{0.048611in}{0.000000in}}{%
\pgfpathmoveto{\pgfqpoint{0.000000in}{0.000000in}}%
\pgfpathlineto{\pgfqpoint{0.048611in}{0.000000in}}%
\pgfusepath{stroke,fill}%
}%
\begin{pgfscope}%
\pgfsys@transformshift{6.033351in}{4.965186in}%
\pgfsys@useobject{currentmarker}{}%
\end{pgfscope}%
\end{pgfscope}%
\begin{pgfscope}%
\definecolor{textcolor}{rgb}{0.000000,0.000000,0.000000}%
\pgfsetstrokecolor{textcolor}%
\pgfsetfillcolor{textcolor}%
\pgftext[x=6.130573in, y=4.917359in, left, base]{\color{textcolor}\rmfamily\fontsize{10.000000}{12.000000}\selectfont 50}%
\end{pgfscope}%
\begin{pgfscope}%
\pgfsetbuttcap%
\pgfsetroundjoin%
\definecolor{currentfill}{rgb}{0.000000,0.000000,0.000000}%
\pgfsetfillcolor{currentfill}%
\pgfsetlinewidth{0.803000pt}%
\definecolor{currentstroke}{rgb}{0.000000,0.000000,0.000000}%
\pgfsetstrokecolor{currentstroke}%
\pgfsetdash{}{0pt}%
\pgfsys@defobject{currentmarker}{\pgfqpoint{0.000000in}{0.000000in}}{\pgfqpoint{0.048611in}{0.000000in}}{%
\pgfpathmoveto{\pgfqpoint{0.000000in}{0.000000in}}%
\pgfpathlineto{\pgfqpoint{0.048611in}{0.000000in}}%
\pgfusepath{stroke,fill}%
}%
\begin{pgfscope}%
\pgfsys@transformshift{6.033351in}{5.314087in}%
\pgfsys@useobject{currentmarker}{}%
\end{pgfscope}%
\end{pgfscope}%
\begin{pgfscope}%
\definecolor{textcolor}{rgb}{0.000000,0.000000,0.000000}%
\pgfsetstrokecolor{textcolor}%
\pgfsetfillcolor{textcolor}%
\pgftext[x=6.130573in, y=5.266259in, left, base]{\color{textcolor}\rmfamily\fontsize{10.000000}{12.000000}\selectfont 75}%
\end{pgfscope}%
\begin{pgfscope}%
\pgfsetbuttcap%
\pgfsetroundjoin%
\definecolor{currentfill}{rgb}{0.000000,0.000000,0.000000}%
\pgfsetfillcolor{currentfill}%
\pgfsetlinewidth{0.803000pt}%
\definecolor{currentstroke}{rgb}{0.000000,0.000000,0.000000}%
\pgfsetstrokecolor{currentstroke}%
\pgfsetdash{}{0pt}%
\pgfsys@defobject{currentmarker}{\pgfqpoint{0.000000in}{0.000000in}}{\pgfqpoint{0.048611in}{0.000000in}}{%
\pgfpathmoveto{\pgfqpoint{0.000000in}{0.000000in}}%
\pgfpathlineto{\pgfqpoint{0.048611in}{0.000000in}}%
\pgfusepath{stroke,fill}%
}%
\begin{pgfscope}%
\pgfsys@transformshift{6.033351in}{5.662987in}%
\pgfsys@useobject{currentmarker}{}%
\end{pgfscope}%
\end{pgfscope}%
\begin{pgfscope}%
\definecolor{textcolor}{rgb}{0.000000,0.000000,0.000000}%
\pgfsetstrokecolor{textcolor}%
\pgfsetfillcolor{textcolor}%
\pgftext[x=6.130573in, y=5.615159in, left, base]{\color{textcolor}\rmfamily\fontsize{10.000000}{12.000000}\selectfont 100}%
\end{pgfscope}%
\begin{pgfscope}%
\pgfsetbuttcap%
\pgfsetroundjoin%
\definecolor{currentfill}{rgb}{0.000000,0.000000,0.000000}%
\pgfsetfillcolor{currentfill}%
\pgfsetlinewidth{0.803000pt}%
\definecolor{currentstroke}{rgb}{0.000000,0.000000,0.000000}%
\pgfsetstrokecolor{currentstroke}%
\pgfsetdash{}{0pt}%
\pgfsys@defobject{currentmarker}{\pgfqpoint{0.000000in}{0.000000in}}{\pgfqpoint{0.048611in}{0.000000in}}{%
\pgfpathmoveto{\pgfqpoint{0.000000in}{0.000000in}}%
\pgfpathlineto{\pgfqpoint{0.048611in}{0.000000in}}%
\pgfusepath{stroke,fill}%
}%
\begin{pgfscope}%
\pgfsys@transformshift{6.033351in}{6.011887in}%
\pgfsys@useobject{currentmarker}{}%
\end{pgfscope}%
\end{pgfscope}%
\begin{pgfscope}%
\definecolor{textcolor}{rgb}{0.000000,0.000000,0.000000}%
\pgfsetstrokecolor{textcolor}%
\pgfsetfillcolor{textcolor}%
\pgftext[x=6.130573in, y=5.964059in, left, base]{\color{textcolor}\rmfamily\fontsize{10.000000}{12.000000}\selectfont 125}%
\end{pgfscope}%
\begin{pgfscope}%
\definecolor{textcolor}{rgb}{0.000000,0.000000,0.000000}%
\pgfsetstrokecolor{textcolor}%
\pgfsetfillcolor{textcolor}%
\pgftext[x=6.394411in,y=5.000076in,,top,rotate=90.000000]{\color{textcolor}\rmfamily\fontsize{10.000000}{12.000000}\selectfont Intensität in ADU}%
\end{pgfscope}%
\begin{pgfscope}%
\pgfsetrectcap%
\pgfsetmiterjoin%
\pgfsetlinewidth{0.803000pt}%
\definecolor{currentstroke}{rgb}{0.000000,0.000000,0.000000}%
\pgfsetstrokecolor{currentstroke}%
\pgfsetdash{}{0pt}%
\pgfpathmoveto{\pgfqpoint{5.933351in}{3.918486in}}%
\pgfpathlineto{\pgfqpoint{5.983351in}{3.918486in}}%
\pgfpathlineto{\pgfqpoint{6.033351in}{3.918486in}}%
\pgfpathlineto{\pgfqpoint{6.033351in}{6.081667in}}%
\pgfpathlineto{\pgfqpoint{5.983351in}{6.081667in}}%
\pgfpathlineto{\pgfqpoint{5.933351in}{6.081667in}}%
\pgfpathlineto{\pgfqpoint{5.933351in}{3.918486in}}%
\pgfpathclose%
\pgfusepath{stroke}%
\end{pgfscope}%
\begin{pgfscope}%
\pgfsetbuttcap%
\pgfsetmiterjoin%
\pgfsetlinewidth{0.000000pt}%
\definecolor{currentstroke}{rgb}{1.000000,1.000000,1.000000}%
\pgfsetstrokecolor{currentstroke}%
\pgfsetstrokeopacity{0.000000}%
\pgfsetdash{}{0pt}%
\pgfpathmoveto{\pgfqpoint{-0.040795in}{0.312576in}}%
\pgfpathlineto{\pgfqpoint{6.439205in}{0.312576in}}%
\pgfpathlineto{\pgfqpoint{6.439205in}{3.437576in}}%
\pgfpathlineto{\pgfqpoint{-0.040795in}{3.437576in}}%
\pgfpathlineto{\pgfqpoint{-0.040795in}{0.312576in}}%
\pgfpathclose%
\pgfusepath{}%
\end{pgfscope}%
\begin{pgfscope}%
\pgfsetbuttcap%
\pgfsetmiterjoin%
\definecolor{currentfill}{rgb}{1.000000,1.000000,1.000000}%
\pgfsetfillcolor{currentfill}%
\pgfsetlinewidth{0.000000pt}%
\definecolor{currentstroke}{rgb}{0.000000,0.000000,0.000000}%
\pgfsetstrokecolor{currentstroke}%
\pgfsetstrokeopacity{0.000000}%
\pgfsetdash{}{0pt}%
\pgfpathmoveto{\pgfqpoint{0.275398in}{0.598088in}}%
\pgfpathlineto{\pgfqpoint{3.029374in}{0.598088in}}%
\pgfpathlineto{\pgfqpoint{3.029374in}{3.352065in}}%
\pgfpathlineto{\pgfqpoint{0.275398in}{3.352065in}}%
\pgfpathlineto{\pgfqpoint{0.275398in}{0.598088in}}%
\pgfpathclose%
\pgfusepath{fill}%
\end{pgfscope}%
\begin{pgfscope}%
\pgfpathrectangle{\pgfqpoint{0.275398in}{0.598088in}}{\pgfqpoint{2.753977in}{2.753977in}}%
\pgfusepath{clip}%
\pgfsys@transformcm{2.754000}{0.000000}{0.000000}{-2.754000}{0.275398in}{3.352088in}%
\pgftext[left,bottom]{\includegraphics[interpolate=false,width=1.000000in,height=1.000000in]{examples_average_std_5x5_hotspot-img11.png}}%
\end{pgfscope}%
\begin{pgfscope}%
\pgfsetrectcap%
\pgfsetmiterjoin%
\pgfsetlinewidth{0.803000pt}%
\definecolor{currentstroke}{rgb}{0.000000,0.000000,0.000000}%
\pgfsetstrokecolor{currentstroke}%
\pgfsetdash{}{0pt}%
\pgfpathmoveto{\pgfqpoint{0.275398in}{0.598088in}}%
\pgfpathlineto{\pgfqpoint{0.275398in}{3.352065in}}%
\pgfusepath{stroke}%
\end{pgfscope}%
\begin{pgfscope}%
\pgfsetrectcap%
\pgfsetmiterjoin%
\pgfsetlinewidth{0.803000pt}%
\definecolor{currentstroke}{rgb}{0.000000,0.000000,0.000000}%
\pgfsetstrokecolor{currentstroke}%
\pgfsetdash{}{0pt}%
\pgfpathmoveto{\pgfqpoint{3.029374in}{0.598088in}}%
\pgfpathlineto{\pgfqpoint{3.029374in}{3.352065in}}%
\pgfusepath{stroke}%
\end{pgfscope}%
\begin{pgfscope}%
\pgfsetrectcap%
\pgfsetmiterjoin%
\pgfsetlinewidth{0.803000pt}%
\definecolor{currentstroke}{rgb}{0.000000,0.000000,0.000000}%
\pgfsetstrokecolor{currentstroke}%
\pgfsetdash{}{0pt}%
\pgfpathmoveto{\pgfqpoint{0.275398in}{0.598088in}}%
\pgfpathlineto{\pgfqpoint{3.029374in}{0.598088in}}%
\pgfusepath{stroke}%
\end{pgfscope}%
\begin{pgfscope}%
\pgfsetrectcap%
\pgfsetmiterjoin%
\pgfsetlinewidth{0.803000pt}%
\definecolor{currentstroke}{rgb}{0.000000,0.000000,0.000000}%
\pgfsetstrokecolor{currentstroke}%
\pgfsetdash{}{0pt}%
\pgfpathmoveto{\pgfqpoint{0.275398in}{3.352065in}}%
\pgfpathlineto{\pgfqpoint{3.029374in}{3.352065in}}%
\pgfusepath{stroke}%
\end{pgfscope}%
\begin{pgfscope}%
\definecolor{textcolor}{rgb}{0.000000,0.000000,0.000000}%
\pgfsetstrokecolor{textcolor}%
\pgfsetfillcolor{textcolor}%
\pgftext[x=0.000000in,y=3.627462in,left,base]{\color{textcolor}\rmfamily\fontsize{10.000000}{12.000000}\selectfont (b)}%
\end{pgfscope}%
\begin{pgfscope}%
\definecolor{textcolor}{rgb}{0.000000,1.000000,0.000000}%
\pgfsetstrokecolor{textcolor}%
\pgfsetfillcolor{textcolor}%
\pgftext[x=0.550795in,y=3.076667in,,]{\color{textcolor}\rmfamily\fontsize{10.000000}{12.000000}\selectfont \num{-0.3}}%
\end{pgfscope}%
\begin{pgfscope}%
\definecolor{textcolor}{rgb}{0.000000,1.000000,0.000000}%
\pgfsetstrokecolor{textcolor}%
\pgfsetfillcolor{textcolor}%
\pgftext[x=1.101591in,y=3.076667in,,]{\color{textcolor}\rmfamily\fontsize{10.000000}{12.000000}\selectfont \num{-0.4}}%
\end{pgfscope}%
\begin{pgfscope}%
\definecolor{textcolor}{rgb}{0.000000,1.000000,0.000000}%
\pgfsetstrokecolor{textcolor}%
\pgfsetfillcolor{textcolor}%
\pgftext[x=1.652386in,y=3.076667in,,]{\color{textcolor}\rmfamily\fontsize{10.000000}{12.000000}\selectfont \num{0.2}}%
\end{pgfscope}%
\begin{pgfscope}%
\definecolor{textcolor}{rgb}{0.000000,1.000000,0.000000}%
\pgfsetstrokecolor{textcolor}%
\pgfsetfillcolor{textcolor}%
\pgftext[x=2.203181in,y=3.076667in,,]{\color{textcolor}\rmfamily\fontsize{10.000000}{12.000000}\selectfont \num{-0.2}}%
\end{pgfscope}%
\begin{pgfscope}%
\definecolor{textcolor}{rgb}{0.000000,1.000000,0.000000}%
\pgfsetstrokecolor{textcolor}%
\pgfsetfillcolor{textcolor}%
\pgftext[x=2.753977in,y=3.076667in,,]{\color{textcolor}\rmfamily\fontsize{10.000000}{12.000000}\selectfont \num{0.2}}%
\end{pgfscope}%
\begin{pgfscope}%
\definecolor{textcolor}{rgb}{0.000000,1.000000,0.000000}%
\pgfsetstrokecolor{textcolor}%
\pgfsetfillcolor{textcolor}%
\pgftext[x=0.550795in,y=2.525872in,,]{\color{textcolor}\rmfamily\fontsize{10.000000}{12.000000}\selectfont \num{-0.4}}%
\end{pgfscope}%
\begin{pgfscope}%
\definecolor{textcolor}{rgb}{0.000000,1.000000,0.000000}%
\pgfsetstrokecolor{textcolor}%
\pgfsetfillcolor{textcolor}%
\pgftext[x=1.101591in,y=2.525872in,,]{\color{textcolor}\rmfamily\fontsize{10.000000}{12.000000}\selectfont \num{1.2}}%
\end{pgfscope}%
\begin{pgfscope}%
\definecolor{textcolor}{rgb}{0.000000,1.000000,0.000000}%
\pgfsetstrokecolor{textcolor}%
\pgfsetfillcolor{textcolor}%
\pgftext[x=1.652386in,y=2.525872in,,]{\color{textcolor}\rmfamily\fontsize{10.000000}{12.000000}\selectfont \num{14.5}}%
\end{pgfscope}%
\begin{pgfscope}%
\definecolor{textcolor}{rgb}{0.000000,1.000000,0.000000}%
\pgfsetstrokecolor{textcolor}%
\pgfsetfillcolor{textcolor}%
\pgftext[x=2.203181in,y=2.525872in,,]{\color{textcolor}\rmfamily\fontsize{10.000000}{12.000000}\selectfont \num{2.0}}%
\end{pgfscope}%
\begin{pgfscope}%
\definecolor{textcolor}{rgb}{0.000000,1.000000,0.000000}%
\pgfsetstrokecolor{textcolor}%
\pgfsetfillcolor{textcolor}%
\pgftext[x=2.753977in,y=2.525872in,,]{\color{textcolor}\rmfamily\fontsize{10.000000}{12.000000}\selectfont \num{0.8}}%
\end{pgfscope}%
\begin{pgfscope}%
\definecolor{textcolor}{rgb}{0.000000,1.000000,0.000000}%
\pgfsetstrokecolor{textcolor}%
\pgfsetfillcolor{textcolor}%
\pgftext[x=0.550795in,y=1.975076in,,]{\color{textcolor}\rmfamily\fontsize{10.000000}{12.000000}\selectfont \num{0.0}}%
\end{pgfscope}%
\begin{pgfscope}%
\definecolor{textcolor}{rgb}{0.000000,1.000000,0.000000}%
\pgfsetstrokecolor{textcolor}%
\pgfsetfillcolor{textcolor}%
\pgftext[x=1.101591in,y=1.975076in,,]{\color{textcolor}\rmfamily\fontsize{10.000000}{12.000000}\selectfont \num{13.3}}%
\end{pgfscope}%
\begin{pgfscope}%
\definecolor{textcolor}{rgb}{0.000000,0.000000,1.000000}%
\pgfsetstrokecolor{textcolor}%
\pgfsetfillcolor{textcolor}%
\pgftext[x=1.652386in,y=1.975076in,,]{\color{textcolor}\rmfamily\fontsize{10.000000}{12.000000}\selectfont \num{115.7}}%
\end{pgfscope}%
\begin{pgfscope}%
\definecolor{textcolor}{rgb}{0.000000,1.000000,0.000000}%
\pgfsetstrokecolor{textcolor}%
\pgfsetfillcolor{textcolor}%
\pgftext[x=2.203181in,y=1.975076in,,]{\color{textcolor}\rmfamily\fontsize{10.000000}{12.000000}\selectfont \num{14.0}}%
\end{pgfscope}%
\begin{pgfscope}%
\definecolor{textcolor}{rgb}{0.000000,1.000000,0.000000}%
\pgfsetstrokecolor{textcolor}%
\pgfsetfillcolor{textcolor}%
\pgftext[x=2.753977in,y=1.975076in,,]{\color{textcolor}\rmfamily\fontsize{10.000000}{12.000000}\selectfont \num{-0.5}}%
\end{pgfscope}%
\begin{pgfscope}%
\definecolor{textcolor}{rgb}{0.000000,1.000000,0.000000}%
\pgfsetstrokecolor{textcolor}%
\pgfsetfillcolor{textcolor}%
\pgftext[x=0.550795in,y=1.424281in,,]{\color{textcolor}\rmfamily\fontsize{10.000000}{12.000000}\selectfont \num{-0.3}}%
\end{pgfscope}%
\begin{pgfscope}%
\definecolor{textcolor}{rgb}{0.000000,1.000000,0.000000}%
\pgfsetstrokecolor{textcolor}%
\pgfsetfillcolor{textcolor}%
\pgftext[x=1.101591in,y=1.424281in,,]{\color{textcolor}\rmfamily\fontsize{10.000000}{12.000000}\selectfont \num{2.8}}%
\end{pgfscope}%
\begin{pgfscope}%
\definecolor{textcolor}{rgb}{0.000000,1.000000,0.000000}%
\pgfsetstrokecolor{textcolor}%
\pgfsetfillcolor{textcolor}%
\pgftext[x=1.652386in,y=1.424281in,,]{\color{textcolor}\rmfamily\fontsize{10.000000}{12.000000}\selectfont \num{13.5}}%
\end{pgfscope}%
\begin{pgfscope}%
\definecolor{textcolor}{rgb}{0.000000,1.000000,0.000000}%
\pgfsetstrokecolor{textcolor}%
\pgfsetfillcolor{textcolor}%
\pgftext[x=2.203181in,y=1.424281in,,]{\color{textcolor}\rmfamily\fontsize{10.000000}{12.000000}\selectfont \num{2.6}}%
\end{pgfscope}%
\begin{pgfscope}%
\definecolor{textcolor}{rgb}{0.000000,1.000000,0.000000}%
\pgfsetstrokecolor{textcolor}%
\pgfsetfillcolor{textcolor}%
\pgftext[x=2.753977in,y=1.424281in,,]{\color{textcolor}\rmfamily\fontsize{10.000000}{12.000000}\selectfont \num{0.0}}%
\end{pgfscope}%
\begin{pgfscope}%
\definecolor{textcolor}{rgb}{0.000000,1.000000,0.000000}%
\pgfsetstrokecolor{textcolor}%
\pgfsetfillcolor{textcolor}%
\pgftext[x=0.550795in,y=0.873486in,,]{\color{textcolor}\rmfamily\fontsize{10.000000}{12.000000}\selectfont \num{-0.2}}%
\end{pgfscope}%
\begin{pgfscope}%
\definecolor{textcolor}{rgb}{0.000000,1.000000,0.000000}%
\pgfsetstrokecolor{textcolor}%
\pgfsetfillcolor{textcolor}%
\pgftext[x=1.101591in,y=0.873486in,,]{\color{textcolor}\rmfamily\fontsize{10.000000}{12.000000}\selectfont \num{0.6}}%
\end{pgfscope}%
\begin{pgfscope}%
\definecolor{textcolor}{rgb}{0.000000,1.000000,0.000000}%
\pgfsetstrokecolor{textcolor}%
\pgfsetfillcolor{textcolor}%
\pgftext[x=1.652386in,y=0.873486in,,]{\color{textcolor}\rmfamily\fontsize{10.000000}{12.000000}\selectfont \num{0.7}}%
\end{pgfscope}%
\begin{pgfscope}%
\definecolor{textcolor}{rgb}{0.000000,1.000000,0.000000}%
\pgfsetstrokecolor{textcolor}%
\pgfsetfillcolor{textcolor}%
\pgftext[x=2.203181in,y=0.873486in,,]{\color{textcolor}\rmfamily\fontsize{10.000000}{12.000000}\selectfont \num{-0.0}}%
\end{pgfscope}%
\begin{pgfscope}%
\definecolor{textcolor}{rgb}{0.000000,1.000000,0.000000}%
\pgfsetstrokecolor{textcolor}%
\pgfsetfillcolor{textcolor}%
\pgftext[x=2.753977in,y=0.873486in,,]{\color{textcolor}\rmfamily\fontsize{10.000000}{12.000000}\selectfont \num{-0.4}}%
\end{pgfscope}%
\begin{pgfscope}%
\pgfsetbuttcap%
\pgfsetmiterjoin%
\definecolor{currentfill}{rgb}{1.000000,1.000000,1.000000}%
\pgfsetfillcolor{currentfill}%
\pgfsetlinewidth{0.000000pt}%
\definecolor{currentstroke}{rgb}{0.000000,0.000000,0.000000}%
\pgfsetstrokecolor{currentstroke}%
\pgfsetstrokeopacity{0.000000}%
\pgfsetdash{}{0pt}%
\pgfpathmoveto{\pgfqpoint{3.279374in}{0.598088in}}%
\pgfpathlineto{\pgfqpoint{6.033351in}{0.598088in}}%
\pgfpathlineto{\pgfqpoint{6.033351in}{3.352065in}}%
\pgfpathlineto{\pgfqpoint{3.279374in}{3.352065in}}%
\pgfpathlineto{\pgfqpoint{3.279374in}{0.598088in}}%
\pgfpathclose%
\pgfusepath{fill}%
\end{pgfscope}%
\begin{pgfscope}%
\pgfpathrectangle{\pgfqpoint{3.279374in}{0.598088in}}{\pgfqpoint{2.753977in}{2.753977in}}%
\pgfusepath{clip}%
\pgfsys@transformcm{2.754000}{0.000000}{0.000000}{-2.754000}{3.279374in}{3.352088in}%
\pgftext[left,bottom]{\includegraphics[interpolate=false,width=1.000000in,height=1.000000in]{examples_average_std_5x5_hotspot-img12.png}}%
\end{pgfscope}%
\begin{pgfscope}%
\pgfsetrectcap%
\pgfsetmiterjoin%
\pgfsetlinewidth{0.803000pt}%
\definecolor{currentstroke}{rgb}{0.000000,0.000000,0.000000}%
\pgfsetstrokecolor{currentstroke}%
\pgfsetdash{}{0pt}%
\pgfpathmoveto{\pgfqpoint{3.279374in}{0.598088in}}%
\pgfpathlineto{\pgfqpoint{3.279374in}{3.352065in}}%
\pgfusepath{stroke}%
\end{pgfscope}%
\begin{pgfscope}%
\pgfsetrectcap%
\pgfsetmiterjoin%
\pgfsetlinewidth{0.803000pt}%
\definecolor{currentstroke}{rgb}{0.000000,0.000000,0.000000}%
\pgfsetstrokecolor{currentstroke}%
\pgfsetdash{}{0pt}%
\pgfpathmoveto{\pgfqpoint{6.033351in}{0.598088in}}%
\pgfpathlineto{\pgfqpoint{6.033351in}{3.352065in}}%
\pgfusepath{stroke}%
\end{pgfscope}%
\begin{pgfscope}%
\pgfsetrectcap%
\pgfsetmiterjoin%
\pgfsetlinewidth{0.803000pt}%
\definecolor{currentstroke}{rgb}{0.000000,0.000000,0.000000}%
\pgfsetstrokecolor{currentstroke}%
\pgfsetdash{}{0pt}%
\pgfpathmoveto{\pgfqpoint{3.279374in}{0.598088in}}%
\pgfpathlineto{\pgfqpoint{6.033351in}{0.598088in}}%
\pgfusepath{stroke}%
\end{pgfscope}%
\begin{pgfscope}%
\pgfsetrectcap%
\pgfsetmiterjoin%
\pgfsetlinewidth{0.803000pt}%
\definecolor{currentstroke}{rgb}{0.000000,0.000000,0.000000}%
\pgfsetstrokecolor{currentstroke}%
\pgfsetdash{}{0pt}%
\pgfpathmoveto{\pgfqpoint{3.279374in}{3.352065in}}%
\pgfpathlineto{\pgfqpoint{6.033351in}{3.352065in}}%
\pgfusepath{stroke}%
\end{pgfscope}%
\begin{pgfscope}%
\definecolor{textcolor}{rgb}{0.000000,0.000000,0.000000}%
\pgfsetstrokecolor{textcolor}%
\pgfsetfillcolor{textcolor}%
\pgftext[x=3.003977in,y=3.627462in,left,base]{\color{textcolor}\rmfamily\fontsize{10.000000}{12.000000}\selectfont (c)}%
\end{pgfscope}%
\begin{pgfscope}%
\definecolor{textcolor}{rgb}{0.000000,0.000000,1.000000}%
\pgfsetstrokecolor{textcolor}%
\pgfsetfillcolor{textcolor}%
\pgftext[x=3.554772in,y=3.076667in,,]{\color{textcolor}\rmfamily\fontsize{10.000000}{12.000000}\selectfont \num{15.5}}%
\end{pgfscope}%
\begin{pgfscope}%
\definecolor{textcolor}{rgb}{0.000000,0.000000,1.000000}%
\pgfsetstrokecolor{textcolor}%
\pgfsetfillcolor{textcolor}%
\pgftext[x=4.105567in,y=3.076667in,,]{\color{textcolor}\rmfamily\fontsize{10.000000}{12.000000}\selectfont \num{15.5}}%
\end{pgfscope}%
\begin{pgfscope}%
\definecolor{textcolor}{rgb}{0.000000,0.000000,1.000000}%
\pgfsetstrokecolor{textcolor}%
\pgfsetfillcolor{textcolor}%
\pgftext[x=4.656363in,y=3.076667in,,]{\color{textcolor}\rmfamily\fontsize{10.000000}{12.000000}\selectfont \num{14.5}}%
\end{pgfscope}%
\begin{pgfscope}%
\definecolor{textcolor}{rgb}{0.000000,0.000000,1.000000}%
\pgfsetstrokecolor{textcolor}%
\pgfsetfillcolor{textcolor}%
\pgftext[x=5.207158in,y=3.076667in,,]{\color{textcolor}\rmfamily\fontsize{10.000000}{12.000000}\selectfont \num{15.4}}%
\end{pgfscope}%
\begin{pgfscope}%
\definecolor{textcolor}{rgb}{0.000000,0.000000,1.000000}%
\pgfsetstrokecolor{textcolor}%
\pgfsetfillcolor{textcolor}%
\pgftext[x=5.757953in,y=3.076667in,,]{\color{textcolor}\rmfamily\fontsize{10.000000}{12.000000}\selectfont \num{15.2}}%
\end{pgfscope}%
\begin{pgfscope}%
\definecolor{textcolor}{rgb}{0.000000,0.000000,1.000000}%
\pgfsetstrokecolor{textcolor}%
\pgfsetfillcolor{textcolor}%
\pgftext[x=3.554772in,y=2.525872in,,]{\color{textcolor}\rmfamily\fontsize{10.000000}{12.000000}\selectfont \num{14.8}}%
\end{pgfscope}%
\begin{pgfscope}%
\definecolor{textcolor}{rgb}{0.000000,0.000000,1.000000}%
\pgfsetstrokecolor{textcolor}%
\pgfsetfillcolor{textcolor}%
\pgftext[x=4.105567in,y=2.525872in,,]{\color{textcolor}\rmfamily\fontsize{10.000000}{12.000000}\selectfont \num{14.9}}%
\end{pgfscope}%
\begin{pgfscope}%
\definecolor{textcolor}{rgb}{0.000000,0.000000,1.000000}%
\pgfsetstrokecolor{textcolor}%
\pgfsetfillcolor{textcolor}%
\pgftext[x=4.656363in,y=2.525872in,,]{\color{textcolor}\rmfamily\fontsize{10.000000}{12.000000}\selectfont \num{20.6}}%
\end{pgfscope}%
\begin{pgfscope}%
\definecolor{textcolor}{rgb}{0.000000,0.000000,1.000000}%
\pgfsetstrokecolor{textcolor}%
\pgfsetfillcolor{textcolor}%
\pgftext[x=5.207158in,y=2.525872in,,]{\color{textcolor}\rmfamily\fontsize{10.000000}{12.000000}\selectfont \num{14.8}}%
\end{pgfscope}%
\begin{pgfscope}%
\definecolor{textcolor}{rgb}{0.000000,0.000000,1.000000}%
\pgfsetstrokecolor{textcolor}%
\pgfsetfillcolor{textcolor}%
\pgftext[x=5.757953in,y=2.525872in,,]{\color{textcolor}\rmfamily\fontsize{10.000000}{12.000000}\selectfont \num{15.5}}%
\end{pgfscope}%
\begin{pgfscope}%
\definecolor{textcolor}{rgb}{0.000000,0.000000,1.000000}%
\pgfsetstrokecolor{textcolor}%
\pgfsetfillcolor{textcolor}%
\pgftext[x=3.554772in,y=1.975076in,,]{\color{textcolor}\rmfamily\fontsize{10.000000}{12.000000}\selectfont \num{15.0}}%
\end{pgfscope}%
\begin{pgfscope}%
\definecolor{textcolor}{rgb}{0.000000,0.000000,1.000000}%
\pgfsetstrokecolor{textcolor}%
\pgfsetfillcolor{textcolor}%
\pgftext[x=4.105567in,y=1.975076in,,]{\color{textcolor}\rmfamily\fontsize{10.000000}{12.000000}\selectfont \num{21.5}}%
\end{pgfscope}%
\begin{pgfscope}%
\definecolor{textcolor}{rgb}{0.000000,1.000000,0.000000}%
\pgfsetstrokecolor{textcolor}%
\pgfsetfillcolor{textcolor}%
\pgftext[x=4.656363in,y=1.975076in,,]{\color{textcolor}\rmfamily\fontsize{10.000000}{12.000000}\selectfont \num{4.1}}%
\end{pgfscope}%
\begin{pgfscope}%
\definecolor{textcolor}{rgb}{0.000000,0.000000,1.000000}%
\pgfsetstrokecolor{textcolor}%
\pgfsetfillcolor{textcolor}%
\pgftext[x=5.207158in,y=1.975076in,,]{\color{textcolor}\rmfamily\fontsize{10.000000}{12.000000}\selectfont \num{20.4}}%
\end{pgfscope}%
\begin{pgfscope}%
\definecolor{textcolor}{rgb}{0.000000,0.000000,1.000000}%
\pgfsetstrokecolor{textcolor}%
\pgfsetfillcolor{textcolor}%
\pgftext[x=5.757953in,y=1.975076in,,]{\color{textcolor}\rmfamily\fontsize{10.000000}{12.000000}\selectfont \num{15.7}}%
\end{pgfscope}%
\begin{pgfscope}%
\definecolor{textcolor}{rgb}{0.000000,0.000000,1.000000}%
\pgfsetstrokecolor{textcolor}%
\pgfsetfillcolor{textcolor}%
\pgftext[x=3.554772in,y=1.424281in,,]{\color{textcolor}\rmfamily\fontsize{10.000000}{12.000000}\selectfont \num{15.9}}%
\end{pgfscope}%
\begin{pgfscope}%
\definecolor{textcolor}{rgb}{0.000000,0.000000,1.000000}%
\pgfsetstrokecolor{textcolor}%
\pgfsetfillcolor{textcolor}%
\pgftext[x=4.105567in,y=1.424281in,,]{\color{textcolor}\rmfamily\fontsize{10.000000}{12.000000}\selectfont \num{14.3}}%
\end{pgfscope}%
\begin{pgfscope}%
\definecolor{textcolor}{rgb}{0.000000,0.000000,1.000000}%
\pgfsetstrokecolor{textcolor}%
\pgfsetfillcolor{textcolor}%
\pgftext[x=4.656363in,y=1.424281in,,]{\color{textcolor}\rmfamily\fontsize{10.000000}{12.000000}\selectfont \num{20.6}}%
\end{pgfscope}%
\begin{pgfscope}%
\definecolor{textcolor}{rgb}{0.000000,0.000000,1.000000}%
\pgfsetstrokecolor{textcolor}%
\pgfsetfillcolor{textcolor}%
\pgftext[x=5.207158in,y=1.424281in,,]{\color{textcolor}\rmfamily\fontsize{10.000000}{12.000000}\selectfont \num{14.4}}%
\end{pgfscope}%
\begin{pgfscope}%
\definecolor{textcolor}{rgb}{0.000000,0.000000,1.000000}%
\pgfsetstrokecolor{textcolor}%
\pgfsetfillcolor{textcolor}%
\pgftext[x=5.757953in,y=1.424281in,,]{\color{textcolor}\rmfamily\fontsize{10.000000}{12.000000}\selectfont \num{14.4}}%
\end{pgfscope}%
\begin{pgfscope}%
\definecolor{textcolor}{rgb}{0.000000,0.000000,1.000000}%
\pgfsetstrokecolor{textcolor}%
\pgfsetfillcolor{textcolor}%
\pgftext[x=3.554772in,y=0.873486in,,]{\color{textcolor}\rmfamily\fontsize{10.000000}{12.000000}\selectfont \num{15.5}}%
\end{pgfscope}%
\begin{pgfscope}%
\definecolor{textcolor}{rgb}{0.000000,0.000000,1.000000}%
\pgfsetstrokecolor{textcolor}%
\pgfsetfillcolor{textcolor}%
\pgftext[x=4.105567in,y=0.873486in,,]{\color{textcolor}\rmfamily\fontsize{10.000000}{12.000000}\selectfont \num{14.8}}%
\end{pgfscope}%
\begin{pgfscope}%
\definecolor{textcolor}{rgb}{0.000000,0.000000,1.000000}%
\pgfsetstrokecolor{textcolor}%
\pgfsetfillcolor{textcolor}%
\pgftext[x=4.656363in,y=0.873486in,,]{\color{textcolor}\rmfamily\fontsize{10.000000}{12.000000}\selectfont \num{15.4}}%
\end{pgfscope}%
\begin{pgfscope}%
\definecolor{textcolor}{rgb}{0.000000,0.000000,1.000000}%
\pgfsetstrokecolor{textcolor}%
\pgfsetfillcolor{textcolor}%
\pgftext[x=5.207158in,y=0.873486in,,]{\color{textcolor}\rmfamily\fontsize{10.000000}{12.000000}\selectfont \num{15.2}}%
\end{pgfscope}%
\begin{pgfscope}%
\definecolor{textcolor}{rgb}{0.000000,0.000000,1.000000}%
\pgfsetstrokecolor{textcolor}%
\pgfsetfillcolor{textcolor}%
\pgftext[x=5.757953in,y=0.873486in,,]{\color{textcolor}\rmfamily\fontsize{10.000000}{12.000000}\selectfont \num{15.0}}%
\end{pgfscope}%
\begin{pgfscope}%
\pgfsetbuttcap%
\pgfsetmiterjoin%
\definecolor{currentfill}{rgb}{1.000000,1.000000,1.000000}%
\pgfsetfillcolor{currentfill}%
\pgfsetlinewidth{0.000000pt}%
\definecolor{currentstroke}{rgb}{0.000000,0.000000,0.000000}%
\pgfsetstrokecolor{currentstroke}%
\pgfsetstrokeopacity{0.000000}%
\pgfsetdash{}{0pt}%
\pgfpathmoveto{\pgfqpoint{0.275398in}{0.398088in}}%
\pgfpathlineto{\pgfqpoint{3.029374in}{0.398088in}}%
\pgfpathlineto{\pgfqpoint{3.029374in}{0.498088in}}%
\pgfpathlineto{\pgfqpoint{0.275398in}{0.498088in}}%
\pgfpathlineto{\pgfqpoint{0.275398in}{0.398088in}}%
\pgfpathclose%
\pgfusepath{fill}%
\end{pgfscope}%
\begin{pgfscope}%
\pgfpathrectangle{\pgfqpoint{0.275398in}{0.398088in}}{\pgfqpoint{2.753977in}{0.100000in}}%
\pgfusepath{clip}%
\pgfsetbuttcap%
\pgfsetmiterjoin%
\definecolor{currentfill}{rgb}{1.000000,1.000000,1.000000}%
\pgfsetfillcolor{currentfill}%
\pgfsetlinewidth{0.010037pt}%
\definecolor{currentstroke}{rgb}{1.000000,1.000000,1.000000}%
\pgfsetstrokecolor{currentstroke}%
\pgfsetdash{}{0pt}%
\pgfusepath{stroke,fill}%
\end{pgfscope}%
\begin{pgfscope}%
\pgfsys@transformshift{0.276000in}{0.398126in}%
\pgftext[left,bottom]{\includegraphics[interpolate=true,width=2.754000in,height=0.100000in]{examples_average_std_5x5_hotspot-img13.png}}%
\end{pgfscope}%
\begin{pgfscope}%
\pgfsetbuttcap%
\pgfsetroundjoin%
\definecolor{currentfill}{rgb}{0.000000,0.000000,0.000000}%
\pgfsetfillcolor{currentfill}%
\pgfsetlinewidth{0.803000pt}%
\definecolor{currentstroke}{rgb}{0.000000,0.000000,0.000000}%
\pgfsetstrokecolor{currentstroke}%
\pgfsetdash{}{0pt}%
\pgfsys@defobject{currentmarker}{\pgfqpoint{0.000000in}{-0.048611in}}{\pgfqpoint{0.000000in}{0.000000in}}{%
\pgfpathmoveto{\pgfqpoint{0.000000in}{0.000000in}}%
\pgfpathlineto{\pgfqpoint{0.000000in}{-0.048611in}}%
\pgfusepath{stroke,fill}%
}%
\begin{pgfscope}%
\pgfsys@transformshift{0.287600in}{0.398088in}%
\pgfsys@useobject{currentmarker}{}%
\end{pgfscope}%
\end{pgfscope}%
\begin{pgfscope}%
\definecolor{textcolor}{rgb}{0.000000,0.000000,0.000000}%
\pgfsetstrokecolor{textcolor}%
\pgfsetfillcolor{textcolor}%
\pgftext[x=0.287600in,y=0.300866in,,top]{\color{textcolor}\rmfamily\fontsize{10.000000}{12.000000}\selectfont 0}%
\end{pgfscope}%
\begin{pgfscope}%
\pgfsetbuttcap%
\pgfsetroundjoin%
\definecolor{currentfill}{rgb}{0.000000,0.000000,0.000000}%
\pgfsetfillcolor{currentfill}%
\pgfsetlinewidth{0.803000pt}%
\definecolor{currentstroke}{rgb}{0.000000,0.000000,0.000000}%
\pgfsetstrokecolor{currentstroke}%
\pgfsetdash{}{0pt}%
\pgfsys@defobject{currentmarker}{\pgfqpoint{0.000000in}{-0.048611in}}{\pgfqpoint{0.000000in}{0.000000in}}{%
\pgfpathmoveto{\pgfqpoint{0.000000in}{0.000000in}}%
\pgfpathlineto{\pgfqpoint{0.000000in}{-0.048611in}}%
\pgfusepath{stroke,fill}%
}%
\begin{pgfscope}%
\pgfsys@transformshift{0.761502in}{0.398088in}%
\pgfsys@useobject{currentmarker}{}%
\end{pgfscope}%
\end{pgfscope}%
\begin{pgfscope}%
\definecolor{textcolor}{rgb}{0.000000,0.000000,0.000000}%
\pgfsetstrokecolor{textcolor}%
\pgfsetfillcolor{textcolor}%
\pgftext[x=0.761502in,y=0.300866in,,top]{\color{textcolor}\rmfamily\fontsize{10.000000}{12.000000}\selectfont 20}%
\end{pgfscope}%
\begin{pgfscope}%
\pgfsetbuttcap%
\pgfsetroundjoin%
\definecolor{currentfill}{rgb}{0.000000,0.000000,0.000000}%
\pgfsetfillcolor{currentfill}%
\pgfsetlinewidth{0.803000pt}%
\definecolor{currentstroke}{rgb}{0.000000,0.000000,0.000000}%
\pgfsetstrokecolor{currentstroke}%
\pgfsetdash{}{0pt}%
\pgfsys@defobject{currentmarker}{\pgfqpoint{0.000000in}{-0.048611in}}{\pgfqpoint{0.000000in}{0.000000in}}{%
\pgfpathmoveto{\pgfqpoint{0.000000in}{0.000000in}}%
\pgfpathlineto{\pgfqpoint{0.000000in}{-0.048611in}}%
\pgfusepath{stroke,fill}%
}%
\begin{pgfscope}%
\pgfsys@transformshift{1.235403in}{0.398088in}%
\pgfsys@useobject{currentmarker}{}%
\end{pgfscope}%
\end{pgfscope}%
\begin{pgfscope}%
\definecolor{textcolor}{rgb}{0.000000,0.000000,0.000000}%
\pgfsetstrokecolor{textcolor}%
\pgfsetfillcolor{textcolor}%
\pgftext[x=1.235403in,y=0.300866in,,top]{\color{textcolor}\rmfamily\fontsize{10.000000}{12.000000}\selectfont 40}%
\end{pgfscope}%
\begin{pgfscope}%
\pgfsetbuttcap%
\pgfsetroundjoin%
\definecolor{currentfill}{rgb}{0.000000,0.000000,0.000000}%
\pgfsetfillcolor{currentfill}%
\pgfsetlinewidth{0.803000pt}%
\definecolor{currentstroke}{rgb}{0.000000,0.000000,0.000000}%
\pgfsetstrokecolor{currentstroke}%
\pgfsetdash{}{0pt}%
\pgfsys@defobject{currentmarker}{\pgfqpoint{0.000000in}{-0.048611in}}{\pgfqpoint{0.000000in}{0.000000in}}{%
\pgfpathmoveto{\pgfqpoint{0.000000in}{0.000000in}}%
\pgfpathlineto{\pgfqpoint{0.000000in}{-0.048611in}}%
\pgfusepath{stroke,fill}%
}%
\begin{pgfscope}%
\pgfsys@transformshift{1.709305in}{0.398088in}%
\pgfsys@useobject{currentmarker}{}%
\end{pgfscope}%
\end{pgfscope}%
\begin{pgfscope}%
\definecolor{textcolor}{rgb}{0.000000,0.000000,0.000000}%
\pgfsetstrokecolor{textcolor}%
\pgfsetfillcolor{textcolor}%
\pgftext[x=1.709305in,y=0.300866in,,top]{\color{textcolor}\rmfamily\fontsize{10.000000}{12.000000}\selectfont 60}%
\end{pgfscope}%
\begin{pgfscope}%
\pgfsetbuttcap%
\pgfsetroundjoin%
\definecolor{currentfill}{rgb}{0.000000,0.000000,0.000000}%
\pgfsetfillcolor{currentfill}%
\pgfsetlinewidth{0.803000pt}%
\definecolor{currentstroke}{rgb}{0.000000,0.000000,0.000000}%
\pgfsetstrokecolor{currentstroke}%
\pgfsetdash{}{0pt}%
\pgfsys@defobject{currentmarker}{\pgfqpoint{0.000000in}{-0.048611in}}{\pgfqpoint{0.000000in}{0.000000in}}{%
\pgfpathmoveto{\pgfqpoint{0.000000in}{0.000000in}}%
\pgfpathlineto{\pgfqpoint{0.000000in}{-0.048611in}}%
\pgfusepath{stroke,fill}%
}%
\begin{pgfscope}%
\pgfsys@transformshift{2.183207in}{0.398088in}%
\pgfsys@useobject{currentmarker}{}%
\end{pgfscope}%
\end{pgfscope}%
\begin{pgfscope}%
\definecolor{textcolor}{rgb}{0.000000,0.000000,0.000000}%
\pgfsetstrokecolor{textcolor}%
\pgfsetfillcolor{textcolor}%
\pgftext[x=2.183207in,y=0.300866in,,top]{\color{textcolor}\rmfamily\fontsize{10.000000}{12.000000}\selectfont 80}%
\end{pgfscope}%
\begin{pgfscope}%
\pgfsetbuttcap%
\pgfsetroundjoin%
\definecolor{currentfill}{rgb}{0.000000,0.000000,0.000000}%
\pgfsetfillcolor{currentfill}%
\pgfsetlinewidth{0.803000pt}%
\definecolor{currentstroke}{rgb}{0.000000,0.000000,0.000000}%
\pgfsetstrokecolor{currentstroke}%
\pgfsetdash{}{0pt}%
\pgfsys@defobject{currentmarker}{\pgfqpoint{0.000000in}{-0.048611in}}{\pgfqpoint{0.000000in}{0.000000in}}{%
\pgfpathmoveto{\pgfqpoint{0.000000in}{0.000000in}}%
\pgfpathlineto{\pgfqpoint{0.000000in}{-0.048611in}}%
\pgfusepath{stroke,fill}%
}%
\begin{pgfscope}%
\pgfsys@transformshift{2.657108in}{0.398088in}%
\pgfsys@useobject{currentmarker}{}%
\end{pgfscope}%
\end{pgfscope}%
\begin{pgfscope}%
\definecolor{textcolor}{rgb}{0.000000,0.000000,0.000000}%
\pgfsetstrokecolor{textcolor}%
\pgfsetfillcolor{textcolor}%
\pgftext[x=2.657108in,y=0.300866in,,top]{\color{textcolor}\rmfamily\fontsize{10.000000}{12.000000}\selectfont 100}%
\end{pgfscope}%
\begin{pgfscope}%
\definecolor{textcolor}{rgb}{0.000000,0.000000,0.000000}%
\pgfsetstrokecolor{textcolor}%
\pgfsetfillcolor{textcolor}%
\pgftext[x=1.652386in,y=0.122655in,,top]{\color{textcolor}\rmfamily\fontsize{10.000000}{12.000000}\selectfont Intensität in ADU}%
\end{pgfscope}%
\begin{pgfscope}%
\pgfsetrectcap%
\pgfsetmiterjoin%
\pgfsetlinewidth{0.803000pt}%
\definecolor{currentstroke}{rgb}{0.000000,0.000000,0.000000}%
\pgfsetstrokecolor{currentstroke}%
\pgfsetdash{}{0pt}%
\pgfpathmoveto{\pgfqpoint{0.275398in}{0.398088in}}%
\pgfpathlineto{\pgfqpoint{0.275398in}{0.448088in}}%
\pgfpathlineto{\pgfqpoint{0.275398in}{0.498088in}}%
\pgfpathlineto{\pgfqpoint{3.029374in}{0.498088in}}%
\pgfpathlineto{\pgfqpoint{3.029374in}{0.448088in}}%
\pgfpathlineto{\pgfqpoint{3.029374in}{0.398088in}}%
\pgfpathlineto{\pgfqpoint{0.275398in}{0.398088in}}%
\pgfpathclose%
\pgfusepath{stroke}%
\end{pgfscope}%
\begin{pgfscope}%
\pgfsetbuttcap%
\pgfsetmiterjoin%
\definecolor{currentfill}{rgb}{1.000000,1.000000,1.000000}%
\pgfsetfillcolor{currentfill}%
\pgfsetlinewidth{0.000000pt}%
\definecolor{currentstroke}{rgb}{0.000000,0.000000,0.000000}%
\pgfsetstrokecolor{currentstroke}%
\pgfsetstrokeopacity{0.000000}%
\pgfsetdash{}{0pt}%
\pgfpathmoveto{\pgfqpoint{3.279374in}{0.398088in}}%
\pgfpathlineto{\pgfqpoint{6.033351in}{0.398088in}}%
\pgfpathlineto{\pgfqpoint{6.033351in}{0.498088in}}%
\pgfpathlineto{\pgfqpoint{3.279374in}{0.498088in}}%
\pgfpathlineto{\pgfqpoint{3.279374in}{0.398088in}}%
\pgfpathclose%
\pgfusepath{fill}%
\end{pgfscope}%
\begin{pgfscope}%
\pgfpathrectangle{\pgfqpoint{3.279374in}{0.398088in}}{\pgfqpoint{2.753977in}{0.100000in}}%
\pgfusepath{clip}%
\pgfsetbuttcap%
\pgfsetmiterjoin%
\definecolor{currentfill}{rgb}{1.000000,1.000000,1.000000}%
\pgfsetfillcolor{currentfill}%
\pgfsetlinewidth{0.010037pt}%
\definecolor{currentstroke}{rgb}{1.000000,1.000000,1.000000}%
\pgfsetstrokecolor{currentstroke}%
\pgfsetdash{}{0pt}%
\pgfusepath{stroke,fill}%
\end{pgfscope}%
\begin{pgfscope}%
\pgfsys@transformshift{3.280000in}{0.398126in}%
\pgftext[left,bottom]{\includegraphics[interpolate=true,width=2.754000in,height=0.100000in]{examples_average_std_5x5_hotspot-img14.png}}%
\end{pgfscope}%
\begin{pgfscope}%
\pgfsetbuttcap%
\pgfsetroundjoin%
\definecolor{currentfill}{rgb}{0.000000,0.000000,0.000000}%
\pgfsetfillcolor{currentfill}%
\pgfsetlinewidth{0.803000pt}%
\definecolor{currentstroke}{rgb}{0.000000,0.000000,0.000000}%
\pgfsetstrokecolor{currentstroke}%
\pgfsetdash{}{0pt}%
\pgfsys@defobject{currentmarker}{\pgfqpoint{0.000000in}{-0.048611in}}{\pgfqpoint{0.000000in}{0.000000in}}{%
\pgfpathmoveto{\pgfqpoint{0.000000in}{0.000000in}}%
\pgfpathlineto{\pgfqpoint{0.000000in}{-0.048611in}}%
\pgfusepath{stroke,fill}%
}%
\begin{pgfscope}%
\pgfsys@transformshift{3.416390in}{0.398088in}%
\pgfsys@useobject{currentmarker}{}%
\end{pgfscope}%
\end{pgfscope}%
\begin{pgfscope}%
\definecolor{textcolor}{rgb}{0.000000,0.000000,0.000000}%
\pgfsetstrokecolor{textcolor}%
\pgfsetfillcolor{textcolor}%
\pgftext[x=3.416390in,y=0.300866in,,top]{\color{textcolor}\rmfamily\fontsize{10.000000}{12.000000}\selectfont 5}%
\end{pgfscope}%
\begin{pgfscope}%
\pgfsetbuttcap%
\pgfsetroundjoin%
\definecolor{currentfill}{rgb}{0.000000,0.000000,0.000000}%
\pgfsetfillcolor{currentfill}%
\pgfsetlinewidth{0.803000pt}%
\definecolor{currentstroke}{rgb}{0.000000,0.000000,0.000000}%
\pgfsetstrokecolor{currentstroke}%
\pgfsetdash{}{0pt}%
\pgfsys@defobject{currentmarker}{\pgfqpoint{0.000000in}{-0.048611in}}{\pgfqpoint{0.000000in}{0.000000in}}{%
\pgfpathmoveto{\pgfqpoint{0.000000in}{0.000000in}}%
\pgfpathlineto{\pgfqpoint{0.000000in}{-0.048611in}}%
\pgfusepath{stroke,fill}%
}%
\begin{pgfscope}%
\pgfsys@transformshift{4.210481in}{0.398088in}%
\pgfsys@useobject{currentmarker}{}%
\end{pgfscope}%
\end{pgfscope}%
\begin{pgfscope}%
\definecolor{textcolor}{rgb}{0.000000,0.000000,0.000000}%
\pgfsetstrokecolor{textcolor}%
\pgfsetfillcolor{textcolor}%
\pgftext[x=4.210481in,y=0.300866in,,top]{\color{textcolor}\rmfamily\fontsize{10.000000}{12.000000}\selectfont 10}%
\end{pgfscope}%
\begin{pgfscope}%
\pgfsetbuttcap%
\pgfsetroundjoin%
\definecolor{currentfill}{rgb}{0.000000,0.000000,0.000000}%
\pgfsetfillcolor{currentfill}%
\pgfsetlinewidth{0.803000pt}%
\definecolor{currentstroke}{rgb}{0.000000,0.000000,0.000000}%
\pgfsetstrokecolor{currentstroke}%
\pgfsetdash{}{0pt}%
\pgfsys@defobject{currentmarker}{\pgfqpoint{0.000000in}{-0.048611in}}{\pgfqpoint{0.000000in}{0.000000in}}{%
\pgfpathmoveto{\pgfqpoint{0.000000in}{0.000000in}}%
\pgfpathlineto{\pgfqpoint{0.000000in}{-0.048611in}}%
\pgfusepath{stroke,fill}%
}%
\begin{pgfscope}%
\pgfsys@transformshift{5.004572in}{0.398088in}%
\pgfsys@useobject{currentmarker}{}%
\end{pgfscope}%
\end{pgfscope}%
\begin{pgfscope}%
\definecolor{textcolor}{rgb}{0.000000,0.000000,0.000000}%
\pgfsetstrokecolor{textcolor}%
\pgfsetfillcolor{textcolor}%
\pgftext[x=5.004572in,y=0.300866in,,top]{\color{textcolor}\rmfamily\fontsize{10.000000}{12.000000}\selectfont 15}%
\end{pgfscope}%
\begin{pgfscope}%
\pgfsetbuttcap%
\pgfsetroundjoin%
\definecolor{currentfill}{rgb}{0.000000,0.000000,0.000000}%
\pgfsetfillcolor{currentfill}%
\pgfsetlinewidth{0.803000pt}%
\definecolor{currentstroke}{rgb}{0.000000,0.000000,0.000000}%
\pgfsetstrokecolor{currentstroke}%
\pgfsetdash{}{0pt}%
\pgfsys@defobject{currentmarker}{\pgfqpoint{0.000000in}{-0.048611in}}{\pgfqpoint{0.000000in}{0.000000in}}{%
\pgfpathmoveto{\pgfqpoint{0.000000in}{0.000000in}}%
\pgfpathlineto{\pgfqpoint{0.000000in}{-0.048611in}}%
\pgfusepath{stroke,fill}%
}%
\begin{pgfscope}%
\pgfsys@transformshift{5.798663in}{0.398088in}%
\pgfsys@useobject{currentmarker}{}%
\end{pgfscope}%
\end{pgfscope}%
\begin{pgfscope}%
\definecolor{textcolor}{rgb}{0.000000,0.000000,0.000000}%
\pgfsetstrokecolor{textcolor}%
\pgfsetfillcolor{textcolor}%
\pgftext[x=5.798663in,y=0.300866in,,top]{\color{textcolor}\rmfamily\fontsize{10.000000}{12.000000}\selectfont 20}%
\end{pgfscope}%
\begin{pgfscope}%
\definecolor{textcolor}{rgb}{0.000000,0.000000,0.000000}%
\pgfsetstrokecolor{textcolor}%
\pgfsetfillcolor{textcolor}%
\pgftext[x=4.656363in,y=0.122655in,,top]{\color{textcolor}\rmfamily\fontsize{10.000000}{12.000000}\selectfont Intensität in ADU}%
\end{pgfscope}%
\begin{pgfscope}%
\pgfsetrectcap%
\pgfsetmiterjoin%
\pgfsetlinewidth{0.803000pt}%
\definecolor{currentstroke}{rgb}{0.000000,0.000000,0.000000}%
\pgfsetstrokecolor{currentstroke}%
\pgfsetdash{}{0pt}%
\pgfpathmoveto{\pgfqpoint{3.279374in}{0.398088in}}%
\pgfpathlineto{\pgfqpoint{3.279374in}{0.448088in}}%
\pgfpathlineto{\pgfqpoint{3.279374in}{0.498088in}}%
\pgfpathlineto{\pgfqpoint{6.033351in}{0.498088in}}%
\pgfpathlineto{\pgfqpoint{6.033351in}{0.448088in}}%
\pgfpathlineto{\pgfqpoint{6.033351in}{0.398088in}}%
\pgfpathlineto{\pgfqpoint{3.279374in}{0.398088in}}%
\pgfpathclose%
\pgfusepath{stroke}%
\end{pgfscope}%
\end{pgfpicture}%
\makeatother%
\endgroup%

    \caption{(a) Beispiele der isolierten einzelnen Photonen, und wie ihre Gesamtintensität über die benachbarte Pixels verteilt wird. Als die Bedingung für die Einzelphotondetektion wurde der Schwellenwert \SI{90}{\adu} von unten und \SI{180}{\adu} von oben festgelegt. Außerdem wird die Nebenbedingung auferlegt, dass die Intensität über \qtyproduct{3 x 3}{\px}-Bereich mehr als \SI{161}{\adu} und weniger als \SI{200}{\adu} beträgt, um lediglich einzelne Photonen zu betrachten. Das über 534 \qtyproduct{5 x 5}{\px}-Bereiche gemitteltes Bild (b) zeigt, dass die Intensität hauptsächlich innerhalb des Kreuzes verteilt wird. Die Gesamtintensität des Zentralpixels und des Kreuzes beträgt \SI{179.6}{\adu}, was gut mit dem Erwartungswert für ein Photon mit der Energie $h\nu_\text{Gd, M5}$ übereinstimmt. In (c) ist die Standardabweichung jedes Pixels $\sigma_{S}$ gezeigt, wobei die Standardabweichung des Detektorrauschens $\sigma_R = \SI{19.94}{\adu}$ nach Gl. (\ref{eq:std_entkopplung}) abgezogen wird.}
    \label{fig:examples_average_std_5x5_hotspot}
\end{figure}
\noindent
Es werden insgesamt 2944 Photonen-Ereignisse gefunden und gemittelt. Zehn Beispiele davon sind in Abb. \ref{fig:examples_average_std_5x5_hotspot}a dargestellt. Der resultierende gemittelte \qtyproduct{5 x 5}{\px}-Bereich in Abb. \ref{fig:examples_average_std_5x5_hotspot}b zeigt, dass der ADU-Wert eines Photons schließlich über das zentrale Pixel und auf die angrenzenden Pixel in vertikaler bzw. horizontaler Richtung verteilt wird, wobei der ADU-Wert im zentralen Pixel relativ konstant bleibt und 
\begin{equation}
    W_\text{Gd, M5, reell}  = \SI{116(4)}{\adu} 
\end{equation}
beträgt. Die dargestellten Beispiele der \qtyproduct{5 x 5}{\px}-Bereich in Abb. \ref{fig:examples_average_std_5x5_hotspot}a und die Standardabweichung der Ladungsverteilung $\sigma_{S}$ in Abb. \ref{fig:examples_average_std_5x5_hotspot}c lässt ablesen, dass der gesamte ADU-Wert eines Photons in einem einzelnen Photon-Ereignis nicht simultan über die benachbarten Pixel verteilt wird, sondern nur in einem davon. 

\noindent
In Anbetracht der Tatsache, dass das hellste Pixel nur \SI{116(4)}{\adu} von dem erwarteten gesamten Ein-Photon-Signal $W_\text{GD, M5} = \SI{180(1)}{\adu}$ enthält, würde die weitere Erhöhung des Schwellenwertes $s_V$ über \SI{116}{\adu} zum signifikanten Verlust an Zahl von detektierten Photonen (und folglich von Signal) führen.

\noindent
Aus diesem Grund wird der Schwellenwert $s_V = \SI{100}{\adu}$ in der Auswertung benutzt. Die Senkung des Schwellenwertes verringert hingegen die Selektivtät des Algorithmuses.

\section{Auswertung der Streubilder mit Schwellenwert-Algorithmus}
\label{text:streuung_counting}
Die Messdaten wurden zuerst mit dem Schwellenwert-Algorithmus verarbeitet, dessen Funktionsprinzip in Abschnitt \ref{text:threshold_algorithm} beschrieben wurde. Die Einzelschritte dessen Anwendung werden exemplarisch in Abb. \ref{fig:capture_ped_diff} am Beispiel eines aufgenommenen Streubildes mit dem Schwellenwert $s_V = \SI{100}{\adu}$ demonstriert.
\begin{figure}[H]
    \centering
    %% Creator: Matplotlib, PGF backend
%%
%% To include the figure in your LaTeX document, write
%%   \input{<filename>.pgf}
%%
%% Make sure the required packages are loaded in your preamble
%%   \usepackage{pgf}
%%
%% Also ensure that all the required font packages are loaded; for instance,
%% the lmodern package is sometimes necessary when using math font.
%%   \usepackage{lmodern}
%%
%% Figures using additional raster images can only be included by \input if
%% they are in the same directory as the main LaTeX file. For loading figures
%% from other directories you can use the `import` package
%%   \usepackage{import}
%%
%% and then include the figures with
%%   \import{<path to file>}{<filename>.pgf}
%%
%% Matplotlib used the following preamble
%%   \usepackage{amsmath} \usepackage[utf8]{inputenc} \usepackage[T1]{fontenc} \usepackage[output-decimal-marker={,},print-unity-mantissa=false]{siunitx} \sisetup{per-mode=fraction, separate-uncertainty = true, locale = DE} \usepackage[acronym, toc, section=section, nonumberlist, nopostdot]{glossaries-extra}
%%
\begingroup%
\makeatletter%
\begin{pgfpicture}%
\pgfpathrectangle{\pgfpointorigin}{\pgfqpoint{5.886400in}{2.465419in}}%
\pgfusepath{use as bounding box, clip}%
\begin{pgfscope}%
\pgfsetbuttcap%
\pgfsetmiterjoin%
\pgfsetlinewidth{0.000000pt}%
\definecolor{currentstroke}{rgb}{1.000000,1.000000,1.000000}%
\pgfsetstrokecolor{currentstroke}%
\pgfsetstrokeopacity{0.000000}%
\pgfsetdash{}{0pt}%
\pgfpathmoveto{\pgfqpoint{0.000000in}{0.000000in}}%
\pgfpathlineto{\pgfqpoint{5.886400in}{0.000000in}}%
\pgfpathlineto{\pgfqpoint{5.886400in}{2.465419in}}%
\pgfpathlineto{\pgfqpoint{0.000000in}{2.465419in}}%
\pgfpathlineto{\pgfqpoint{0.000000in}{0.000000in}}%
\pgfpathclose%
\pgfusepath{}%
\end{pgfscope}%
\begin{pgfscope}%
\pgfsetbuttcap%
\pgfsetmiterjoin%
\pgfsetlinewidth{0.000000pt}%
\definecolor{currentstroke}{rgb}{1.000000,1.000000,1.000000}%
\pgfsetstrokecolor{currentstroke}%
\pgfsetstrokeopacity{0.000000}%
\pgfsetdash{}{0pt}%
\pgfpathmoveto{\pgfqpoint{-0.377600in}{-1.556873in}}%
\pgfpathlineto{\pgfqpoint{3.942400in}{-1.556873in}}%
\pgfpathlineto{\pgfqpoint{3.942400in}{3.943127in}}%
\pgfpathlineto{\pgfqpoint{-0.377600in}{3.943127in}}%
\pgfpathlineto{\pgfqpoint{-0.377600in}{-1.556873in}}%
\pgfpathclose%
\pgfusepath{}%
\end{pgfscope}%
\begin{pgfscope}%
\pgfsetbuttcap%
\pgfsetmiterjoin%
\definecolor{currentfill}{rgb}{1.000000,1.000000,1.000000}%
\pgfsetfillcolor{currentfill}%
\pgfsetlinewidth{0.000000pt}%
\definecolor{currentstroke}{rgb}{0.000000,0.000000,0.000000}%
\pgfsetstrokecolor{currentstroke}%
\pgfsetstrokeopacity{0.000000}%
\pgfsetdash{}{0pt}%
\pgfpathmoveto{\pgfqpoint{0.162400in}{0.544877in}}%
\pgfpathlineto{\pgfqpoint{1.786400in}{0.544877in}}%
\pgfpathlineto{\pgfqpoint{1.786400in}{2.168877in}}%
\pgfpathlineto{\pgfqpoint{0.162400in}{2.168877in}}%
\pgfpathlineto{\pgfqpoint{0.162400in}{0.544877in}}%
\pgfpathclose%
\pgfusepath{fill}%
\end{pgfscope}%
\begin{pgfscope}%
\pgfsys@transformshift{0.162000in}{0.545419in}%
\pgftext[left,bottom]{\includegraphics[interpolate=true,width=1.624000in,height=1.624000in]{capture_ped_diff-img0.png}}%
\end{pgfscope}%
\begin{pgfscope}%
\pgfsetrectcap%
\pgfsetmiterjoin%
\pgfsetlinewidth{0.803000pt}%
\definecolor{currentstroke}{rgb}{0.000000,0.000000,0.000000}%
\pgfsetstrokecolor{currentstroke}%
\pgfsetdash{}{0pt}%
\pgfpathmoveto{\pgfqpoint{0.162400in}{0.544877in}}%
\pgfpathlineto{\pgfqpoint{0.162400in}{2.168877in}}%
\pgfusepath{stroke}%
\end{pgfscope}%
\begin{pgfscope}%
\pgfsetrectcap%
\pgfsetmiterjoin%
\pgfsetlinewidth{0.803000pt}%
\definecolor{currentstroke}{rgb}{0.000000,0.000000,0.000000}%
\pgfsetstrokecolor{currentstroke}%
\pgfsetdash{}{0pt}%
\pgfpathmoveto{\pgfqpoint{1.786400in}{0.544877in}}%
\pgfpathlineto{\pgfqpoint{1.786400in}{2.168877in}}%
\pgfusepath{stroke}%
\end{pgfscope}%
\begin{pgfscope}%
\pgfsetrectcap%
\pgfsetmiterjoin%
\pgfsetlinewidth{0.803000pt}%
\definecolor{currentstroke}{rgb}{0.000000,0.000000,0.000000}%
\pgfsetstrokecolor{currentstroke}%
\pgfsetdash{}{0pt}%
\pgfpathmoveto{\pgfqpoint{0.162400in}{0.544877in}}%
\pgfpathlineto{\pgfqpoint{1.786400in}{0.544877in}}%
\pgfusepath{stroke}%
\end{pgfscope}%
\begin{pgfscope}%
\pgfsetrectcap%
\pgfsetmiterjoin%
\pgfsetlinewidth{0.803000pt}%
\definecolor{currentstroke}{rgb}{0.000000,0.000000,0.000000}%
\pgfsetstrokecolor{currentstroke}%
\pgfsetdash{}{0pt}%
\pgfpathmoveto{\pgfqpoint{0.162400in}{2.168877in}}%
\pgfpathlineto{\pgfqpoint{1.786400in}{2.168877in}}%
\pgfusepath{stroke}%
\end{pgfscope}%
\begin{pgfscope}%
\definecolor{textcolor}{rgb}{0.000000,0.000000,0.000000}%
\pgfsetstrokecolor{textcolor}%
\pgfsetfillcolor{textcolor}%
\pgftext[x=0.000000in,y=2.331277in,left,base]{\color{textcolor}\rmfamily\fontsize{10.000000}{12.000000}\selectfont (a)}%
\end{pgfscope}%
\begin{pgfscope}%
\pgfsetbuttcap%
\pgfsetmiterjoin%
\definecolor{currentfill}{rgb}{1.000000,1.000000,1.000000}%
\pgfsetfillcolor{currentfill}%
\pgfsetlinewidth{0.000000pt}%
\definecolor{currentstroke}{rgb}{0.000000,0.000000,0.000000}%
\pgfsetstrokecolor{currentstroke}%
\pgfsetstrokeopacity{0.000000}%
\pgfsetdash{}{0pt}%
\pgfpathmoveto{\pgfqpoint{1.886400in}{0.544877in}}%
\pgfpathlineto{\pgfqpoint{3.510400in}{0.544877in}}%
\pgfpathlineto{\pgfqpoint{3.510400in}{2.168877in}}%
\pgfpathlineto{\pgfqpoint{1.886400in}{2.168877in}}%
\pgfpathlineto{\pgfqpoint{1.886400in}{0.544877in}}%
\pgfpathclose%
\pgfusepath{fill}%
\end{pgfscope}%
\begin{pgfscope}%
\pgfsys@transformshift{1.886000in}{0.545419in}%
\pgftext[left,bottom]{\includegraphics[interpolate=true,width=1.624000in,height=1.624000in]{capture_ped_diff-img1.png}}%
\end{pgfscope}%
\begin{pgfscope}%
\pgfsetrectcap%
\pgfsetmiterjoin%
\pgfsetlinewidth{0.803000pt}%
\definecolor{currentstroke}{rgb}{0.000000,0.000000,0.000000}%
\pgfsetstrokecolor{currentstroke}%
\pgfsetdash{}{0pt}%
\pgfpathmoveto{\pgfqpoint{1.886400in}{0.544877in}}%
\pgfpathlineto{\pgfqpoint{1.886400in}{2.168877in}}%
\pgfusepath{stroke}%
\end{pgfscope}%
\begin{pgfscope}%
\pgfsetrectcap%
\pgfsetmiterjoin%
\pgfsetlinewidth{0.803000pt}%
\definecolor{currentstroke}{rgb}{0.000000,0.000000,0.000000}%
\pgfsetstrokecolor{currentstroke}%
\pgfsetdash{}{0pt}%
\pgfpathmoveto{\pgfqpoint{3.510400in}{0.544877in}}%
\pgfpathlineto{\pgfqpoint{3.510400in}{2.168877in}}%
\pgfusepath{stroke}%
\end{pgfscope}%
\begin{pgfscope}%
\pgfsetrectcap%
\pgfsetmiterjoin%
\pgfsetlinewidth{0.803000pt}%
\definecolor{currentstroke}{rgb}{0.000000,0.000000,0.000000}%
\pgfsetstrokecolor{currentstroke}%
\pgfsetdash{}{0pt}%
\pgfpathmoveto{\pgfqpoint{1.886400in}{0.544877in}}%
\pgfpathlineto{\pgfqpoint{3.510400in}{0.544877in}}%
\pgfusepath{stroke}%
\end{pgfscope}%
\begin{pgfscope}%
\pgfsetrectcap%
\pgfsetmiterjoin%
\pgfsetlinewidth{0.803000pt}%
\definecolor{currentstroke}{rgb}{0.000000,0.000000,0.000000}%
\pgfsetstrokecolor{currentstroke}%
\pgfsetdash{}{0pt}%
\pgfpathmoveto{\pgfqpoint{1.886400in}{2.168877in}}%
\pgfpathlineto{\pgfqpoint{3.510400in}{2.168877in}}%
\pgfusepath{stroke}%
\end{pgfscope}%
\begin{pgfscope}%
\definecolor{textcolor}{rgb}{0.000000,0.000000,0.000000}%
\pgfsetstrokecolor{textcolor}%
\pgfsetfillcolor{textcolor}%
\pgftext[x=1.724000in,y=2.331277in,left,base]{\color{textcolor}\rmfamily\fontsize{10.000000}{12.000000}\selectfont (b)}%
\end{pgfscope}%
\begin{pgfscope}%
\pgfsetbuttcap%
\pgfsetmiterjoin%
\definecolor{currentfill}{rgb}{1.000000,1.000000,1.000000}%
\pgfsetfillcolor{currentfill}%
\pgfsetlinewidth{0.000000pt}%
\definecolor{currentstroke}{rgb}{0.000000,0.000000,0.000000}%
\pgfsetstrokecolor{currentstroke}%
\pgfsetstrokeopacity{0.000000}%
\pgfsetdash{}{0pt}%
\pgfpathmoveto{\pgfqpoint{0.162400in}{0.244877in}}%
\pgfpathlineto{\pgfqpoint{3.510400in}{0.244877in}}%
\pgfpathlineto{\pgfqpoint{3.510400in}{0.344877in}}%
\pgfpathlineto{\pgfqpoint{0.162400in}{0.344877in}}%
\pgfpathlineto{\pgfqpoint{0.162400in}{0.244877in}}%
\pgfpathclose%
\pgfusepath{fill}%
\end{pgfscope}%
\begin{pgfscope}%
\pgfpathrectangle{\pgfqpoint{0.162400in}{0.244877in}}{\pgfqpoint{3.348000in}{0.100000in}}%
\pgfusepath{clip}%
\pgfsetbuttcap%
\pgfsetmiterjoin%
\definecolor{currentfill}{rgb}{1.000000,1.000000,1.000000}%
\pgfsetfillcolor{currentfill}%
\pgfsetlinewidth{0.010037pt}%
\definecolor{currentstroke}{rgb}{1.000000,1.000000,1.000000}%
\pgfsetstrokecolor{currentstroke}%
\pgfsetdash{}{0pt}%
\pgfusepath{stroke,fill}%
\end{pgfscope}%
\begin{pgfscope}%
\pgfpathrectangle{\pgfqpoint{0.162400in}{0.244877in}}{\pgfqpoint{3.348000in}{0.100000in}}%
\pgfusepath{clip}%
\pgfsetbuttcap%
\pgfsetmiterjoin%
\definecolor{currentfill}{rgb}{1.000000,1.000000,1.000000}%
\pgfsetfillcolor{currentfill}%
\pgfsetlinewidth{0.010037pt}%
\definecolor{currentstroke}{rgb}{1.000000,1.000000,1.000000}%
\pgfsetstrokecolor{currentstroke}%
\pgfsetdash{}{0pt}%
\pgfusepath{stroke,fill}%
\end{pgfscope}%
\begin{pgfscope}%
\pgfsys@transformshift{0.162000in}{0.245419in}%
\pgftext[left,bottom]{\includegraphics[interpolate=true,width=3.348000in,height=0.100000in]{capture_ped_diff-img2.png}}%
\end{pgfscope}%
\begin{pgfscope}%
\pgfsetbuttcap%
\pgfsetroundjoin%
\definecolor{currentfill}{rgb}{0.000000,0.000000,0.000000}%
\pgfsetfillcolor{currentfill}%
\pgfsetlinewidth{0.803000pt}%
\definecolor{currentstroke}{rgb}{0.000000,0.000000,0.000000}%
\pgfsetstrokecolor{currentstroke}%
\pgfsetdash{}{0pt}%
\pgfsys@defobject{currentmarker}{\pgfqpoint{0.000000in}{-0.048611in}}{\pgfqpoint{0.000000in}{0.000000in}}{%
\pgfpathmoveto{\pgfqpoint{0.000000in}{0.000000in}}%
\pgfpathlineto{\pgfqpoint{0.000000in}{-0.048611in}}%
\pgfusepath{stroke,fill}%
}%
\begin{pgfscope}%
\pgfsys@transformshift{0.162400in}{0.244877in}%
\pgfsys@useobject{currentmarker}{}%
\end{pgfscope}%
\end{pgfscope}%
\begin{pgfscope}%
\definecolor{textcolor}{rgb}{0.000000,0.000000,0.000000}%
\pgfsetstrokecolor{textcolor}%
\pgfsetfillcolor{textcolor}%
\pgftext[x=0.162400in,y=0.147655in,,top]{\color{textcolor}\rmfamily\fontsize{10.000000}{12.000000}\selectfont 5000}%
\end{pgfscope}%
\begin{pgfscope}%
\pgfsetbuttcap%
\pgfsetroundjoin%
\definecolor{currentfill}{rgb}{0.000000,0.000000,0.000000}%
\pgfsetfillcolor{currentfill}%
\pgfsetlinewidth{0.803000pt}%
\definecolor{currentstroke}{rgb}{0.000000,0.000000,0.000000}%
\pgfsetstrokecolor{currentstroke}%
\pgfsetdash{}{0pt}%
\pgfsys@defobject{currentmarker}{\pgfqpoint{0.000000in}{-0.048611in}}{\pgfqpoint{0.000000in}{0.000000in}}{%
\pgfpathmoveto{\pgfqpoint{0.000000in}{0.000000in}}%
\pgfpathlineto{\pgfqpoint{0.000000in}{-0.048611in}}%
\pgfusepath{stroke,fill}%
}%
\begin{pgfscope}%
\pgfsys@transformshift{0.832000in}{0.244877in}%
\pgfsys@useobject{currentmarker}{}%
\end{pgfscope}%
\end{pgfscope}%
\begin{pgfscope}%
\definecolor{textcolor}{rgb}{0.000000,0.000000,0.000000}%
\pgfsetstrokecolor{textcolor}%
\pgfsetfillcolor{textcolor}%
\pgftext[x=0.832000in,y=0.147655in,,top]{\color{textcolor}\rmfamily\fontsize{10.000000}{12.000000}\selectfont 5200}%
\end{pgfscope}%
\begin{pgfscope}%
\pgfsetbuttcap%
\pgfsetroundjoin%
\definecolor{currentfill}{rgb}{0.000000,0.000000,0.000000}%
\pgfsetfillcolor{currentfill}%
\pgfsetlinewidth{0.803000pt}%
\definecolor{currentstroke}{rgb}{0.000000,0.000000,0.000000}%
\pgfsetstrokecolor{currentstroke}%
\pgfsetdash{}{0pt}%
\pgfsys@defobject{currentmarker}{\pgfqpoint{0.000000in}{-0.048611in}}{\pgfqpoint{0.000000in}{0.000000in}}{%
\pgfpathmoveto{\pgfqpoint{0.000000in}{0.000000in}}%
\pgfpathlineto{\pgfqpoint{0.000000in}{-0.048611in}}%
\pgfusepath{stroke,fill}%
}%
\begin{pgfscope}%
\pgfsys@transformshift{1.501600in}{0.244877in}%
\pgfsys@useobject{currentmarker}{}%
\end{pgfscope}%
\end{pgfscope}%
\begin{pgfscope}%
\definecolor{textcolor}{rgb}{0.000000,0.000000,0.000000}%
\pgfsetstrokecolor{textcolor}%
\pgfsetfillcolor{textcolor}%
\pgftext[x=1.501600in,y=0.147655in,,top]{\color{textcolor}\rmfamily\fontsize{10.000000}{12.000000}\selectfont 5400}%
\end{pgfscope}%
\begin{pgfscope}%
\pgfsetbuttcap%
\pgfsetroundjoin%
\definecolor{currentfill}{rgb}{0.000000,0.000000,0.000000}%
\pgfsetfillcolor{currentfill}%
\pgfsetlinewidth{0.803000pt}%
\definecolor{currentstroke}{rgb}{0.000000,0.000000,0.000000}%
\pgfsetstrokecolor{currentstroke}%
\pgfsetdash{}{0pt}%
\pgfsys@defobject{currentmarker}{\pgfqpoint{0.000000in}{-0.048611in}}{\pgfqpoint{0.000000in}{0.000000in}}{%
\pgfpathmoveto{\pgfqpoint{0.000000in}{0.000000in}}%
\pgfpathlineto{\pgfqpoint{0.000000in}{-0.048611in}}%
\pgfusepath{stroke,fill}%
}%
\begin{pgfscope}%
\pgfsys@transformshift{2.171200in}{0.244877in}%
\pgfsys@useobject{currentmarker}{}%
\end{pgfscope}%
\end{pgfscope}%
\begin{pgfscope}%
\definecolor{textcolor}{rgb}{0.000000,0.000000,0.000000}%
\pgfsetstrokecolor{textcolor}%
\pgfsetfillcolor{textcolor}%
\pgftext[x=2.171200in,y=0.147655in,,top]{\color{textcolor}\rmfamily\fontsize{10.000000}{12.000000}\selectfont 5600}%
\end{pgfscope}%
\begin{pgfscope}%
\pgfsetbuttcap%
\pgfsetroundjoin%
\definecolor{currentfill}{rgb}{0.000000,0.000000,0.000000}%
\pgfsetfillcolor{currentfill}%
\pgfsetlinewidth{0.803000pt}%
\definecolor{currentstroke}{rgb}{0.000000,0.000000,0.000000}%
\pgfsetstrokecolor{currentstroke}%
\pgfsetdash{}{0pt}%
\pgfsys@defobject{currentmarker}{\pgfqpoint{0.000000in}{-0.048611in}}{\pgfqpoint{0.000000in}{0.000000in}}{%
\pgfpathmoveto{\pgfqpoint{0.000000in}{0.000000in}}%
\pgfpathlineto{\pgfqpoint{0.000000in}{-0.048611in}}%
\pgfusepath{stroke,fill}%
}%
\begin{pgfscope}%
\pgfsys@transformshift{2.840800in}{0.244877in}%
\pgfsys@useobject{currentmarker}{}%
\end{pgfscope}%
\end{pgfscope}%
\begin{pgfscope}%
\definecolor{textcolor}{rgb}{0.000000,0.000000,0.000000}%
\pgfsetstrokecolor{textcolor}%
\pgfsetfillcolor{textcolor}%
\pgftext[x=2.840800in,y=0.147655in,,top]{\color{textcolor}\rmfamily\fontsize{10.000000}{12.000000}\selectfont 5800}%
\end{pgfscope}%
\begin{pgfscope}%
\pgfsetbuttcap%
\pgfsetroundjoin%
\definecolor{currentfill}{rgb}{0.000000,0.000000,0.000000}%
\pgfsetfillcolor{currentfill}%
\pgfsetlinewidth{0.803000pt}%
\definecolor{currentstroke}{rgb}{0.000000,0.000000,0.000000}%
\pgfsetstrokecolor{currentstroke}%
\pgfsetdash{}{0pt}%
\pgfsys@defobject{currentmarker}{\pgfqpoint{0.000000in}{-0.048611in}}{\pgfqpoint{0.000000in}{0.000000in}}{%
\pgfpathmoveto{\pgfqpoint{0.000000in}{0.000000in}}%
\pgfpathlineto{\pgfqpoint{0.000000in}{-0.048611in}}%
\pgfusepath{stroke,fill}%
}%
\begin{pgfscope}%
\pgfsys@transformshift{3.510400in}{0.244877in}%
\pgfsys@useobject{currentmarker}{}%
\end{pgfscope}%
\end{pgfscope}%
\begin{pgfscope}%
\definecolor{textcolor}{rgb}{0.000000,0.000000,0.000000}%
\pgfsetstrokecolor{textcolor}%
\pgfsetfillcolor{textcolor}%
\pgftext[x=3.510400in,y=0.147655in,,top]{\color{textcolor}\rmfamily\fontsize{10.000000}{12.000000}\selectfont 6000}%
\end{pgfscope}%
\begin{pgfscope}%
\pgfsetrectcap%
\pgfsetmiterjoin%
\pgfsetlinewidth{0.803000pt}%
\definecolor{currentstroke}{rgb}{0.000000,0.000000,0.000000}%
\pgfsetstrokecolor{currentstroke}%
\pgfsetdash{}{0pt}%
\pgfpathmoveto{\pgfqpoint{0.162400in}{0.244877in}}%
\pgfpathlineto{\pgfqpoint{0.162400in}{0.294877in}}%
\pgfpathlineto{\pgfqpoint{0.162400in}{0.344877in}}%
\pgfpathlineto{\pgfqpoint{3.510400in}{0.344877in}}%
\pgfpathlineto{\pgfqpoint{3.510400in}{0.294877in}}%
\pgfpathlineto{\pgfqpoint{3.510400in}{0.244877in}}%
\pgfpathlineto{\pgfqpoint{0.162400in}{0.244877in}}%
\pgfpathclose%
\pgfusepath{stroke}%
\end{pgfscope}%
\begin{pgfscope}%
\pgfsetbuttcap%
\pgfsetmiterjoin%
\pgfsetlinewidth{0.000000pt}%
\definecolor{currentstroke}{rgb}{1.000000,1.000000,1.000000}%
\pgfsetstrokecolor{currentstroke}%
\pgfsetstrokeopacity{0.000000}%
\pgfsetdash{}{0pt}%
\pgfpathmoveto{\pgfqpoint{3.942400in}{-1.556873in}}%
\pgfpathlineto{\pgfqpoint{6.102400in}{-1.556873in}}%
\pgfpathlineto{\pgfqpoint{6.102400in}{3.943127in}}%
\pgfpathlineto{\pgfqpoint{3.942400in}{3.943127in}}%
\pgfpathlineto{\pgfqpoint{3.942400in}{-1.556873in}}%
\pgfpathclose%
\pgfusepath{}%
\end{pgfscope}%
\begin{pgfscope}%
\pgfsetbuttcap%
\pgfsetmiterjoin%
\definecolor{currentfill}{rgb}{1.000000,1.000000,1.000000}%
\pgfsetfillcolor{currentfill}%
\pgfsetlinewidth{0.000000pt}%
\definecolor{currentstroke}{rgb}{0.000000,0.000000,0.000000}%
\pgfsetstrokecolor{currentstroke}%
\pgfsetstrokeopacity{0.000000}%
\pgfsetdash{}{0pt}%
\pgfpathmoveto{\pgfqpoint{4.212400in}{0.519877in}}%
\pgfpathlineto{\pgfqpoint{5.886400in}{0.519877in}}%
\pgfpathlineto{\pgfqpoint{5.886400in}{2.193877in}}%
\pgfpathlineto{\pgfqpoint{4.212400in}{2.193877in}}%
\pgfpathlineto{\pgfqpoint{4.212400in}{0.519877in}}%
\pgfpathclose%
\pgfusepath{fill}%
\end{pgfscope}%
\begin{pgfscope}%
\pgfsys@transformshift{4.212000in}{0.521419in}%
\pgftext[left,bottom]{\includegraphics[interpolate=true,width=1.674000in,height=1.674000in]{capture_ped_diff-img3.png}}%
\end{pgfscope}%
\begin{pgfscope}%
\pgfsetrectcap%
\pgfsetmiterjoin%
\pgfsetlinewidth{0.803000pt}%
\definecolor{currentstroke}{rgb}{0.000000,0.000000,0.000000}%
\pgfsetstrokecolor{currentstroke}%
\pgfsetdash{}{0pt}%
\pgfpathmoveto{\pgfqpoint{4.212400in}{0.519877in}}%
\pgfpathlineto{\pgfqpoint{4.212400in}{2.193877in}}%
\pgfusepath{stroke}%
\end{pgfscope}%
\begin{pgfscope}%
\pgfsetrectcap%
\pgfsetmiterjoin%
\pgfsetlinewidth{0.803000pt}%
\definecolor{currentstroke}{rgb}{0.000000,0.000000,0.000000}%
\pgfsetstrokecolor{currentstroke}%
\pgfsetdash{}{0pt}%
\pgfpathmoveto{\pgfqpoint{5.886400in}{0.519877in}}%
\pgfpathlineto{\pgfqpoint{5.886400in}{2.193877in}}%
\pgfusepath{stroke}%
\end{pgfscope}%
\begin{pgfscope}%
\pgfsetrectcap%
\pgfsetmiterjoin%
\pgfsetlinewidth{0.803000pt}%
\definecolor{currentstroke}{rgb}{0.000000,0.000000,0.000000}%
\pgfsetstrokecolor{currentstroke}%
\pgfsetdash{}{0pt}%
\pgfpathmoveto{\pgfqpoint{4.212400in}{0.519877in}}%
\pgfpathlineto{\pgfqpoint{5.886400in}{0.519877in}}%
\pgfusepath{stroke}%
\end{pgfscope}%
\begin{pgfscope}%
\pgfsetrectcap%
\pgfsetmiterjoin%
\pgfsetlinewidth{0.803000pt}%
\definecolor{currentstroke}{rgb}{0.000000,0.000000,0.000000}%
\pgfsetstrokecolor{currentstroke}%
\pgfsetdash{}{0pt}%
\pgfpathmoveto{\pgfqpoint{4.212400in}{2.193877in}}%
\pgfpathlineto{\pgfqpoint{5.886400in}{2.193877in}}%
\pgfusepath{stroke}%
\end{pgfscope}%
\begin{pgfscope}%
\definecolor{textcolor}{rgb}{0.000000,0.000000,0.000000}%
\pgfsetstrokecolor{textcolor}%
\pgfsetfillcolor{textcolor}%
\pgftext[x=4.045000in,y=2.361277in,left,base]{\color{textcolor}\rmfamily\fontsize{10.000000}{12.000000}\selectfont (c)}%
\end{pgfscope}%
\begin{pgfscope}%
\pgfsetbuttcap%
\pgfsetmiterjoin%
\definecolor{currentfill}{rgb}{1.000000,1.000000,1.000000}%
\pgfsetfillcolor{currentfill}%
\pgfsetlinewidth{0.000000pt}%
\definecolor{currentstroke}{rgb}{0.000000,0.000000,0.000000}%
\pgfsetstrokecolor{currentstroke}%
\pgfsetstrokeopacity{0.000000}%
\pgfsetdash{}{0pt}%
\pgfpathmoveto{\pgfqpoint{4.212400in}{0.219877in}}%
\pgfpathlineto{\pgfqpoint{5.886400in}{0.219877in}}%
\pgfpathlineto{\pgfqpoint{5.886400in}{0.319877in}}%
\pgfpathlineto{\pgfqpoint{4.212400in}{0.319877in}}%
\pgfpathlineto{\pgfqpoint{4.212400in}{0.219877in}}%
\pgfpathclose%
\pgfusepath{fill}%
\end{pgfscope}%
\begin{pgfscope}%
\pgfpathrectangle{\pgfqpoint{4.212400in}{0.219877in}}{\pgfqpoint{1.674000in}{0.100000in}}%
\pgfusepath{clip}%
\pgfsetbuttcap%
\pgfsetmiterjoin%
\definecolor{currentfill}{rgb}{1.000000,1.000000,1.000000}%
\pgfsetfillcolor{currentfill}%
\pgfsetlinewidth{0.010037pt}%
\definecolor{currentstroke}{rgb}{1.000000,1.000000,1.000000}%
\pgfsetstrokecolor{currentstroke}%
\pgfsetdash{}{0pt}%
\pgfusepath{stroke,fill}%
\end{pgfscope}%
\begin{pgfscope}%
\pgfsys@transformshift{4.212000in}{0.221419in}%
\pgftext[left,bottom]{\includegraphics[interpolate=true,width=1.674000in,height=0.100000in]{capture_ped_diff-img4.png}}%
\end{pgfscope}%
\begin{pgfscope}%
\pgfsetbuttcap%
\pgfsetroundjoin%
\definecolor{currentfill}{rgb}{0.000000,0.000000,0.000000}%
\pgfsetfillcolor{currentfill}%
\pgfsetlinewidth{0.803000pt}%
\definecolor{currentstroke}{rgb}{0.000000,0.000000,0.000000}%
\pgfsetstrokecolor{currentstroke}%
\pgfsetdash{}{0pt}%
\pgfsys@defobject{currentmarker}{\pgfqpoint{0.000000in}{-0.048611in}}{\pgfqpoint{0.000000in}{0.000000in}}{%
\pgfpathmoveto{\pgfqpoint{0.000000in}{0.000000in}}%
\pgfpathlineto{\pgfqpoint{0.000000in}{-0.048611in}}%
\pgfusepath{stroke,fill}%
}%
\begin{pgfscope}%
\pgfsys@transformshift{4.507224in}{0.219877in}%
\pgfsys@useobject{currentmarker}{}%
\end{pgfscope}%
\end{pgfscope}%
\begin{pgfscope}%
\definecolor{textcolor}{rgb}{0.000000,0.000000,0.000000}%
\pgfsetstrokecolor{textcolor}%
\pgfsetfillcolor{textcolor}%
\pgftext[x=4.507224in,y=0.122655in,,top]{\color{textcolor}\rmfamily\fontsize{10.000000}{12.000000}\selectfont -100}%
\end{pgfscope}%
\begin{pgfscope}%
\pgfsetbuttcap%
\pgfsetroundjoin%
\definecolor{currentfill}{rgb}{0.000000,0.000000,0.000000}%
\pgfsetfillcolor{currentfill}%
\pgfsetlinewidth{0.803000pt}%
\definecolor{currentstroke}{rgb}{0.000000,0.000000,0.000000}%
\pgfsetstrokecolor{currentstroke}%
\pgfsetdash{}{0pt}%
\pgfsys@defobject{currentmarker}{\pgfqpoint{0.000000in}{-0.048611in}}{\pgfqpoint{0.000000in}{0.000000in}}{%
\pgfpathmoveto{\pgfqpoint{0.000000in}{0.000000in}}%
\pgfpathlineto{\pgfqpoint{0.000000in}{-0.048611in}}%
\pgfusepath{stroke,fill}%
}%
\begin{pgfscope}%
\pgfsys@transformshift{5.006925in}{0.219877in}%
\pgfsys@useobject{currentmarker}{}%
\end{pgfscope}%
\end{pgfscope}%
\begin{pgfscope}%
\definecolor{textcolor}{rgb}{0.000000,0.000000,0.000000}%
\pgfsetstrokecolor{textcolor}%
\pgfsetfillcolor{textcolor}%
\pgftext[x=5.006925in,y=0.122655in,,top]{\color{textcolor}\rmfamily\fontsize{10.000000}{12.000000}\selectfont 0}%
\end{pgfscope}%
\begin{pgfscope}%
\pgfsetbuttcap%
\pgfsetroundjoin%
\definecolor{currentfill}{rgb}{0.000000,0.000000,0.000000}%
\pgfsetfillcolor{currentfill}%
\pgfsetlinewidth{0.803000pt}%
\definecolor{currentstroke}{rgb}{0.000000,0.000000,0.000000}%
\pgfsetstrokecolor{currentstroke}%
\pgfsetdash{}{0pt}%
\pgfsys@defobject{currentmarker}{\pgfqpoint{0.000000in}{-0.048611in}}{\pgfqpoint{0.000000in}{0.000000in}}{%
\pgfpathmoveto{\pgfqpoint{0.000000in}{0.000000in}}%
\pgfpathlineto{\pgfqpoint{0.000000in}{-0.048611in}}%
\pgfusepath{stroke,fill}%
}%
\begin{pgfscope}%
\pgfsys@transformshift{5.506627in}{0.219877in}%
\pgfsys@useobject{currentmarker}{}%
\end{pgfscope}%
\end{pgfscope}%
\begin{pgfscope}%
\definecolor{textcolor}{rgb}{0.000000,0.000000,0.000000}%
\pgfsetstrokecolor{textcolor}%
\pgfsetfillcolor{textcolor}%
\pgftext[x=5.506627in,y=0.122655in,,top]{\color{textcolor}\rmfamily\fontsize{10.000000}{12.000000}\selectfont 100}%
\end{pgfscope}%
\begin{pgfscope}%
\pgfsetrectcap%
\pgfsetmiterjoin%
\pgfsetlinewidth{0.803000pt}%
\definecolor{currentstroke}{rgb}{0.000000,0.000000,0.000000}%
\pgfsetstrokecolor{currentstroke}%
\pgfsetdash{}{0pt}%
\pgfpathmoveto{\pgfqpoint{4.212400in}{0.219877in}}%
\pgfpathlineto{\pgfqpoint{4.212400in}{0.269877in}}%
\pgfpathlineto{\pgfqpoint{4.212400in}{0.319877in}}%
\pgfpathlineto{\pgfqpoint{5.886400in}{0.319877in}}%
\pgfpathlineto{\pgfqpoint{5.886400in}{0.269877in}}%
\pgfpathlineto{\pgfqpoint{5.886400in}{0.219877in}}%
\pgfpathlineto{\pgfqpoint{4.212400in}{0.219877in}}%
\pgfpathclose%
\pgfusepath{stroke}%
\end{pgfscope}%
\end{pgfpicture}%
\makeatother%
\endgroup%

    \caption{(a) eine Einzelaufnahme der gestreuten Photonen, (b) ein gemitteltes Bild über \num{10000} Dunkelbilder.  Die resultierende Differenz (c) von ersten zwei Bildern und (d) angewendete Schwellenwert \SI{100}{\adu}.}
    \label{fig:capture_ped_diff}
\end{figure}
\noindent
Von dem mit dem \gls{moench03} aufgenommenen Streubild (Abb. \ref{fig:capture_ped_diff}a) wir der konstante Offset (Abb. \ref{fig:capture_ped_diff}b), der sich als Mittelung von \num{10000} Dunkelbildern ergibt, subtrahiert. Die resultierende Differenz (Abb. \ref{fig:capture_ped_diff}c) ist wesentlich klein gegenüber dem konstanten Offset und dem ursprünglichen Streubild, was die Wichtigkeit der regulären Aufnahme von Dunkelbildern zur Korrektur des konstanten Offsets unterstreicht. Diejenige Pixel, deren Werte den Schwellenwert $s_V$ überschreiten, werden in Abb. \ref{fig:capture_ped_diff}d dargestellt.

\noindent
Im nächsten Abschnitt wird das resultierende Signal-Rauschen-Verhältnis in Bezug auf den Schwellenwert $s_V$ disktutiert.

\subsection{Signal-zu-Rauschen-Verhältnis}
Um das Signal und Rauschen quantitativ auszuwerten, werden einige Begriffe und Hilfsvariablen definiert und ermittelt. Mit Variablen $N_A$ und $N_P$ wird die Zahl der erfassten Aufnahmen und Pixel bezeichnet.

\noindent
Zunächst wird analoge und digitale Photonenzahl definiert. Analoge Photonenzahl in einem Bereich der Aufnahme ergibt sich als Quotient der Summe aller Pixelwerte in dem Bereich und des Ein-Photon-Signals. Diese wurde folgendermaßen ermittelt: Es werden 300 Streubilder gemittelt, um das Detektorrauscehn zu verringern. Die Pixelwerte im \qtyproduct{100 x 100}{\px}-Bereich um den Direktstrahl werden aufsummiert und durch das Ein-Photon-Signal \SI{180}{\adu} geteilt.

\noindent
Digitale Photonenzahl in einem Bereich ergibt sich als die Gesamtzahl von Photonen, die mit dem Schwellenwert-Algorithmus mit einem bestimmten Wert $s_V$ in dem Bereich detektiert werden. Diese wurde folgendermaßen ermittelt: Es werden dieselben 300 Streubilder genommen und einzeln mit Schwellenwert-Algoirthmus ausgewertet. Als Nächstes wird die Zahl der detektierten Photonen in demselben im \qtyproduct{100 x 100}{\px}-Bereich gezählt. Dieses Verfahren wird für $s_V$ im Intervall von \SI{50}{\adu} bis \SI{160}{\adu} mit dem Schritt \SI{5}{\adu} wiederholt.

\noindent
Das Pixel kann fälschlicherweise als Photon bezeichnet werden, wenn das Detektorrauschen in dem Pixel den entsprechenden Schwellenwert $s_V$ überschreitet. Die Zahl von \gls{fdpa} wird als digitale Photonenzahl in Dunkelbilder in Bezug auf $s_V$ ermittelt. Diese wurde folgendermaßen ermittelt: Es werden 5000 Dunkelbilder genommen und einzeln mit Schwellenwert-Algoirthmus ausgewertet. Als Nächstes wird die Zahl der detektierten Photonen im ganzen Bild gezählt und durch die gesamte Pixelzahl \qtyproduct{400 x 400}{\px} und Aufnahmezahl geteilt, um einen reduzierten Wert zu bekommen. Dieses Verfahren wird für $s_V$ im Intervall von \SI{50}{\adu} bis \SI{160}{\adu} mit dem Schritt \SI{5}{\adu} wiederholt.

\noindent
Basierend auf den definierten Hilfsvariablen wird ein Maß und zwar \gls{qe} des Schwellenwert-Algorithmuses wie folgt formuliert:
\begin{equation}
    \text{\gls{qe}}(s_V) = \frac{\text{digitale Photonenzahl}(s_V) - N_A N_P \cdot \text{\gls{fdpa}}(s_V)}{\text{analoge Photonenzahl}}
\end{equation}
Die Zahl zeigt das Verhältnis zwischen der Zahl der Photonen, die an dem Detektor detektiert wurden und die mit dem Schwellenwert-Algorithmus als Photon gekennzeichnet wird.

\noindent
Als Nächstes wird der \gls{photnenfluss} in einem Bereich der Aufnahme definiert. Dieser ergibt sich als die Analoge Photonenzahl, die über die Anzahl der Aufnahmen und erfastten Pixel geteilt wird.
\begin{equation}
    \text{\gls{photnenfluss}} = \frac{\text{analoge Photonenzahl}}{N_A N_P}
\end{equation}

\noindent
Bei der Auswertung mit dem Schwellenwert-Algorithmus kommt das Schrotrauschen wegen des niedrigen detektierten Photonenflusses als die primäre Rauschenquelle.

\noindent
Die Erwartungswerte vom Signal 
\begin{equation}
        S_{\text{EW}}(s_V, N_A, N_P) = N_{A}N_{P}\left[\text{\gls{photnenfluss}}\cdot\text{\gls{qe}}(s_V)\right]
        \label{eq:signal_ew}
\end{equation}
und vom dazugehörigen Shrotrauschen 
\begin{equation}
        N_{\text{EW}}(s_V, N_A, N_P) = \sqrt{N_{A}N_{P}\left[\text{\gls{photnenfluss}}\cdot\text{\gls{qe}}(s_V) + \text{\gls{fdpa}}(s_V)\right]}
        \label{eq:noise_ew}
\end{equation}
lassen sich der Theorie nach so ausdrücken. Die Erwartungswerte hängen u. a. von dem Photonenfluss ab, der sich je nach Bereich in der Aufnahme unterscheiden kann. So ist das Signal-Rauschen-Verhältnis in der Aufnahme ortsabhängig.  

\noindent
Die ermittelten Hilfsvariablen sowie die Erwartungswerte von Signal und Shrotrauschen, die exemplarisch für den Bereich vom Direktstrahl berechnet wurden, werden in Abb. \ref{fig:qe_fehldetektiert_signal_noise} dargestellt.
\begin{figure}[H]
    \centering
    %% Creator: Matplotlib, PGF backend
%%
%% To include the figure in your LaTeX document, write
%%   \input{<filename>.pgf}
%%
%% Make sure the required packages are loaded in your preamble
%%   \usepackage{pgf}
%%
%% Also ensure that all the required font packages are loaded; for instance,
%% the lmodern package is sometimes necessary when using math font.
%%   \usepackage{lmodern}
%%
%% Figures using additional raster images can only be included by \input if
%% they are in the same directory as the main LaTeX file. For loading figures
%% from other directories you can use the `import` package
%%   \usepackage{import}
%%
%% and then include the figures with
%%   \import{<path to file>}{<filename>.pgf}
%%
%% Matplotlib used the following preamble
%%   \usepackage{amsmath} \usepackage[utf8]{inputenc} \usepackage[T1]{fontenc} \usepackage[output-decimal-marker={,},print-unity-mantissa=false]{siunitx} \sisetup{per-mode=fraction, separate-uncertainty = true, locale = DE} \usepackage[acronym, toc, section=section, nonumberlist, nopostdot]{glossaries-extra} \DeclareSIUnit\adu{\text{ADU}} \DeclareSIUnit\px{\text{px}} \DeclareSIUnit\photons{\text{Pho\-to\-nen}} \DeclareSIUnit\photon{\text{Pho\-ton}}
%%
\begingroup%
\makeatletter%
\begin{pgfpicture}%
\pgfpathrectangle{\pgfpointorigin}{\pgfqpoint{6.426886in}{4.400916in}}%
\pgfusepath{use as bounding box, clip}%
\begin{pgfscope}%
\pgfsetbuttcap%
\pgfsetmiterjoin%
\pgfsetlinewidth{0.000000pt}%
\definecolor{currentstroke}{rgb}{1.000000,1.000000,1.000000}%
\pgfsetstrokecolor{currentstroke}%
\pgfsetstrokeopacity{0.000000}%
\pgfsetdash{}{0pt}%
\pgfpathmoveto{\pgfqpoint{0.000000in}{0.000000in}}%
\pgfpathlineto{\pgfqpoint{6.426886in}{0.000000in}}%
\pgfpathlineto{\pgfqpoint{6.426886in}{4.400916in}}%
\pgfpathlineto{\pgfqpoint{0.000000in}{4.400916in}}%
\pgfpathlineto{\pgfqpoint{0.000000in}{0.000000in}}%
\pgfpathclose%
\pgfusepath{}%
\end{pgfscope}%
\begin{pgfscope}%
\pgfsetbuttcap%
\pgfsetmiterjoin%
\definecolor{currentfill}{rgb}{1.000000,1.000000,1.000000}%
\pgfsetfillcolor{currentfill}%
\pgfsetlinewidth{0.000000pt}%
\definecolor{currentstroke}{rgb}{0.000000,0.000000,0.000000}%
\pgfsetstrokecolor{currentstroke}%
\pgfsetstrokeopacity{0.000000}%
\pgfsetdash{}{0pt}%
\pgfpathmoveto{\pgfqpoint{0.444878in}{2.498808in}}%
\pgfpathlineto{\pgfqpoint{2.923825in}{2.498808in}}%
\pgfpathlineto{\pgfqpoint{2.923825in}{4.300916in}}%
\pgfpathlineto{\pgfqpoint{0.444878in}{4.300916in}}%
\pgfpathlineto{\pgfqpoint{0.444878in}{2.498808in}}%
\pgfpathclose%
\pgfusepath{fill}%
\end{pgfscope}%
\begin{pgfscope}%
\pgfsetbuttcap%
\pgfsetroundjoin%
\definecolor{currentfill}{rgb}{0.000000,0.000000,0.000000}%
\pgfsetfillcolor{currentfill}%
\pgfsetlinewidth{0.803000pt}%
\definecolor{currentstroke}{rgb}{0.000000,0.000000,0.000000}%
\pgfsetstrokecolor{currentstroke}%
\pgfsetdash{}{0pt}%
\pgfsys@defobject{currentmarker}{\pgfqpoint{0.000000in}{-0.048611in}}{\pgfqpoint{0.000000in}{0.000000in}}{%
\pgfpathmoveto{\pgfqpoint{0.000000in}{0.000000in}}%
\pgfpathlineto{\pgfqpoint{0.000000in}{-0.048611in}}%
\pgfusepath{stroke,fill}%
}%
\begin{pgfscope}%
\pgfsys@transformshift{0.546474in}{2.498808in}%
\pgfsys@useobject{currentmarker}{}%
\end{pgfscope}%
\end{pgfscope}%
\begin{pgfscope}%
\pgfsetbuttcap%
\pgfsetroundjoin%
\definecolor{currentfill}{rgb}{0.000000,0.000000,0.000000}%
\pgfsetfillcolor{currentfill}%
\pgfsetlinewidth{0.803000pt}%
\definecolor{currentstroke}{rgb}{0.000000,0.000000,0.000000}%
\pgfsetstrokecolor{currentstroke}%
\pgfsetdash{}{0pt}%
\pgfsys@defobject{currentmarker}{\pgfqpoint{0.000000in}{-0.048611in}}{\pgfqpoint{0.000000in}{0.000000in}}{%
\pgfpathmoveto{\pgfqpoint{0.000000in}{0.000000in}}%
\pgfpathlineto{\pgfqpoint{0.000000in}{-0.048611in}}%
\pgfusepath{stroke,fill}%
}%
\begin{pgfscope}%
\pgfsys@transformshift{1.054455in}{2.498808in}%
\pgfsys@useobject{currentmarker}{}%
\end{pgfscope}%
\end{pgfscope}%
\begin{pgfscope}%
\pgfsetbuttcap%
\pgfsetroundjoin%
\definecolor{currentfill}{rgb}{0.000000,0.000000,0.000000}%
\pgfsetfillcolor{currentfill}%
\pgfsetlinewidth{0.803000pt}%
\definecolor{currentstroke}{rgb}{0.000000,0.000000,0.000000}%
\pgfsetstrokecolor{currentstroke}%
\pgfsetdash{}{0pt}%
\pgfsys@defobject{currentmarker}{\pgfqpoint{0.000000in}{-0.048611in}}{\pgfqpoint{0.000000in}{0.000000in}}{%
\pgfpathmoveto{\pgfqpoint{0.000000in}{0.000000in}}%
\pgfpathlineto{\pgfqpoint{0.000000in}{-0.048611in}}%
\pgfusepath{stroke,fill}%
}%
\begin{pgfscope}%
\pgfsys@transformshift{1.562436in}{2.498808in}%
\pgfsys@useobject{currentmarker}{}%
\end{pgfscope}%
\end{pgfscope}%
\begin{pgfscope}%
\pgfsetbuttcap%
\pgfsetroundjoin%
\definecolor{currentfill}{rgb}{0.000000,0.000000,0.000000}%
\pgfsetfillcolor{currentfill}%
\pgfsetlinewidth{0.803000pt}%
\definecolor{currentstroke}{rgb}{0.000000,0.000000,0.000000}%
\pgfsetstrokecolor{currentstroke}%
\pgfsetdash{}{0pt}%
\pgfsys@defobject{currentmarker}{\pgfqpoint{0.000000in}{-0.048611in}}{\pgfqpoint{0.000000in}{0.000000in}}{%
\pgfpathmoveto{\pgfqpoint{0.000000in}{0.000000in}}%
\pgfpathlineto{\pgfqpoint{0.000000in}{-0.048611in}}%
\pgfusepath{stroke,fill}%
}%
\begin{pgfscope}%
\pgfsys@transformshift{2.070417in}{2.498808in}%
\pgfsys@useobject{currentmarker}{}%
\end{pgfscope}%
\end{pgfscope}%
\begin{pgfscope}%
\pgfsetbuttcap%
\pgfsetroundjoin%
\definecolor{currentfill}{rgb}{0.000000,0.000000,0.000000}%
\pgfsetfillcolor{currentfill}%
\pgfsetlinewidth{0.803000pt}%
\definecolor{currentstroke}{rgb}{0.000000,0.000000,0.000000}%
\pgfsetstrokecolor{currentstroke}%
\pgfsetdash{}{0pt}%
\pgfsys@defobject{currentmarker}{\pgfqpoint{0.000000in}{-0.048611in}}{\pgfqpoint{0.000000in}{0.000000in}}{%
\pgfpathmoveto{\pgfqpoint{0.000000in}{0.000000in}}%
\pgfpathlineto{\pgfqpoint{0.000000in}{-0.048611in}}%
\pgfusepath{stroke,fill}%
}%
\begin{pgfscope}%
\pgfsys@transformshift{2.578398in}{2.498808in}%
\pgfsys@useobject{currentmarker}{}%
\end{pgfscope}%
\end{pgfscope}%
\begin{pgfscope}%
\pgfsetbuttcap%
\pgfsetroundjoin%
\definecolor{currentfill}{rgb}{0.000000,0.000000,0.000000}%
\pgfsetfillcolor{currentfill}%
\pgfsetlinewidth{0.602250pt}%
\definecolor{currentstroke}{rgb}{0.000000,0.000000,0.000000}%
\pgfsetstrokecolor{currentstroke}%
\pgfsetdash{}{0pt}%
\pgfsys@defobject{currentmarker}{\pgfqpoint{0.000000in}{-0.027778in}}{\pgfqpoint{0.000000in}{0.000000in}}{%
\pgfpathmoveto{\pgfqpoint{0.000000in}{0.000000in}}%
\pgfpathlineto{\pgfqpoint{0.000000in}{-0.027778in}}%
\pgfusepath{stroke,fill}%
}%
\begin{pgfscope}%
\pgfsys@transformshift{0.444878in}{2.498808in}%
\pgfsys@useobject{currentmarker}{}%
\end{pgfscope}%
\end{pgfscope}%
\begin{pgfscope}%
\pgfsetbuttcap%
\pgfsetroundjoin%
\definecolor{currentfill}{rgb}{0.000000,0.000000,0.000000}%
\pgfsetfillcolor{currentfill}%
\pgfsetlinewidth{0.602250pt}%
\definecolor{currentstroke}{rgb}{0.000000,0.000000,0.000000}%
\pgfsetstrokecolor{currentstroke}%
\pgfsetdash{}{0pt}%
\pgfsys@defobject{currentmarker}{\pgfqpoint{0.000000in}{-0.027778in}}{\pgfqpoint{0.000000in}{0.000000in}}{%
\pgfpathmoveto{\pgfqpoint{0.000000in}{0.000000in}}%
\pgfpathlineto{\pgfqpoint{0.000000in}{-0.027778in}}%
\pgfusepath{stroke,fill}%
}%
\begin{pgfscope}%
\pgfsys@transformshift{0.648070in}{2.498808in}%
\pgfsys@useobject{currentmarker}{}%
\end{pgfscope}%
\end{pgfscope}%
\begin{pgfscope}%
\pgfsetbuttcap%
\pgfsetroundjoin%
\definecolor{currentfill}{rgb}{0.000000,0.000000,0.000000}%
\pgfsetfillcolor{currentfill}%
\pgfsetlinewidth{0.602250pt}%
\definecolor{currentstroke}{rgb}{0.000000,0.000000,0.000000}%
\pgfsetstrokecolor{currentstroke}%
\pgfsetdash{}{0pt}%
\pgfsys@defobject{currentmarker}{\pgfqpoint{0.000000in}{-0.027778in}}{\pgfqpoint{0.000000in}{0.000000in}}{%
\pgfpathmoveto{\pgfqpoint{0.000000in}{0.000000in}}%
\pgfpathlineto{\pgfqpoint{0.000000in}{-0.027778in}}%
\pgfusepath{stroke,fill}%
}%
\begin{pgfscope}%
\pgfsys@transformshift{0.749666in}{2.498808in}%
\pgfsys@useobject{currentmarker}{}%
\end{pgfscope}%
\end{pgfscope}%
\begin{pgfscope}%
\pgfsetbuttcap%
\pgfsetroundjoin%
\definecolor{currentfill}{rgb}{0.000000,0.000000,0.000000}%
\pgfsetfillcolor{currentfill}%
\pgfsetlinewidth{0.602250pt}%
\definecolor{currentstroke}{rgb}{0.000000,0.000000,0.000000}%
\pgfsetstrokecolor{currentstroke}%
\pgfsetdash{}{0pt}%
\pgfsys@defobject{currentmarker}{\pgfqpoint{0.000000in}{-0.027778in}}{\pgfqpoint{0.000000in}{0.000000in}}{%
\pgfpathmoveto{\pgfqpoint{0.000000in}{0.000000in}}%
\pgfpathlineto{\pgfqpoint{0.000000in}{-0.027778in}}%
\pgfusepath{stroke,fill}%
}%
\begin{pgfscope}%
\pgfsys@transformshift{0.851262in}{2.498808in}%
\pgfsys@useobject{currentmarker}{}%
\end{pgfscope}%
\end{pgfscope}%
\begin{pgfscope}%
\pgfsetbuttcap%
\pgfsetroundjoin%
\definecolor{currentfill}{rgb}{0.000000,0.000000,0.000000}%
\pgfsetfillcolor{currentfill}%
\pgfsetlinewidth{0.602250pt}%
\definecolor{currentstroke}{rgb}{0.000000,0.000000,0.000000}%
\pgfsetstrokecolor{currentstroke}%
\pgfsetdash{}{0pt}%
\pgfsys@defobject{currentmarker}{\pgfqpoint{0.000000in}{-0.027778in}}{\pgfqpoint{0.000000in}{0.000000in}}{%
\pgfpathmoveto{\pgfqpoint{0.000000in}{0.000000in}}%
\pgfpathlineto{\pgfqpoint{0.000000in}{-0.027778in}}%
\pgfusepath{stroke,fill}%
}%
\begin{pgfscope}%
\pgfsys@transformshift{0.952859in}{2.498808in}%
\pgfsys@useobject{currentmarker}{}%
\end{pgfscope}%
\end{pgfscope}%
\begin{pgfscope}%
\pgfsetbuttcap%
\pgfsetroundjoin%
\definecolor{currentfill}{rgb}{0.000000,0.000000,0.000000}%
\pgfsetfillcolor{currentfill}%
\pgfsetlinewidth{0.602250pt}%
\definecolor{currentstroke}{rgb}{0.000000,0.000000,0.000000}%
\pgfsetstrokecolor{currentstroke}%
\pgfsetdash{}{0pt}%
\pgfsys@defobject{currentmarker}{\pgfqpoint{0.000000in}{-0.027778in}}{\pgfqpoint{0.000000in}{0.000000in}}{%
\pgfpathmoveto{\pgfqpoint{0.000000in}{0.000000in}}%
\pgfpathlineto{\pgfqpoint{0.000000in}{-0.027778in}}%
\pgfusepath{stroke,fill}%
}%
\begin{pgfscope}%
\pgfsys@transformshift{1.156051in}{2.498808in}%
\pgfsys@useobject{currentmarker}{}%
\end{pgfscope}%
\end{pgfscope}%
\begin{pgfscope}%
\pgfsetbuttcap%
\pgfsetroundjoin%
\definecolor{currentfill}{rgb}{0.000000,0.000000,0.000000}%
\pgfsetfillcolor{currentfill}%
\pgfsetlinewidth{0.602250pt}%
\definecolor{currentstroke}{rgb}{0.000000,0.000000,0.000000}%
\pgfsetstrokecolor{currentstroke}%
\pgfsetdash{}{0pt}%
\pgfsys@defobject{currentmarker}{\pgfqpoint{0.000000in}{-0.027778in}}{\pgfqpoint{0.000000in}{0.000000in}}{%
\pgfpathmoveto{\pgfqpoint{0.000000in}{0.000000in}}%
\pgfpathlineto{\pgfqpoint{0.000000in}{-0.027778in}}%
\pgfusepath{stroke,fill}%
}%
\begin{pgfscope}%
\pgfsys@transformshift{1.257647in}{2.498808in}%
\pgfsys@useobject{currentmarker}{}%
\end{pgfscope}%
\end{pgfscope}%
\begin{pgfscope}%
\pgfsetbuttcap%
\pgfsetroundjoin%
\definecolor{currentfill}{rgb}{0.000000,0.000000,0.000000}%
\pgfsetfillcolor{currentfill}%
\pgfsetlinewidth{0.602250pt}%
\definecolor{currentstroke}{rgb}{0.000000,0.000000,0.000000}%
\pgfsetstrokecolor{currentstroke}%
\pgfsetdash{}{0pt}%
\pgfsys@defobject{currentmarker}{\pgfqpoint{0.000000in}{-0.027778in}}{\pgfqpoint{0.000000in}{0.000000in}}{%
\pgfpathmoveto{\pgfqpoint{0.000000in}{0.000000in}}%
\pgfpathlineto{\pgfqpoint{0.000000in}{-0.027778in}}%
\pgfusepath{stroke,fill}%
}%
\begin{pgfscope}%
\pgfsys@transformshift{1.359244in}{2.498808in}%
\pgfsys@useobject{currentmarker}{}%
\end{pgfscope}%
\end{pgfscope}%
\begin{pgfscope}%
\pgfsetbuttcap%
\pgfsetroundjoin%
\definecolor{currentfill}{rgb}{0.000000,0.000000,0.000000}%
\pgfsetfillcolor{currentfill}%
\pgfsetlinewidth{0.602250pt}%
\definecolor{currentstroke}{rgb}{0.000000,0.000000,0.000000}%
\pgfsetstrokecolor{currentstroke}%
\pgfsetdash{}{0pt}%
\pgfsys@defobject{currentmarker}{\pgfqpoint{0.000000in}{-0.027778in}}{\pgfqpoint{0.000000in}{0.000000in}}{%
\pgfpathmoveto{\pgfqpoint{0.000000in}{0.000000in}}%
\pgfpathlineto{\pgfqpoint{0.000000in}{-0.027778in}}%
\pgfusepath{stroke,fill}%
}%
\begin{pgfscope}%
\pgfsys@transformshift{1.460840in}{2.498808in}%
\pgfsys@useobject{currentmarker}{}%
\end{pgfscope}%
\end{pgfscope}%
\begin{pgfscope}%
\pgfsetbuttcap%
\pgfsetroundjoin%
\definecolor{currentfill}{rgb}{0.000000,0.000000,0.000000}%
\pgfsetfillcolor{currentfill}%
\pgfsetlinewidth{0.602250pt}%
\definecolor{currentstroke}{rgb}{0.000000,0.000000,0.000000}%
\pgfsetstrokecolor{currentstroke}%
\pgfsetdash{}{0pt}%
\pgfsys@defobject{currentmarker}{\pgfqpoint{0.000000in}{-0.027778in}}{\pgfqpoint{0.000000in}{0.000000in}}{%
\pgfpathmoveto{\pgfqpoint{0.000000in}{0.000000in}}%
\pgfpathlineto{\pgfqpoint{0.000000in}{-0.027778in}}%
\pgfusepath{stroke,fill}%
}%
\begin{pgfscope}%
\pgfsys@transformshift{1.664032in}{2.498808in}%
\pgfsys@useobject{currentmarker}{}%
\end{pgfscope}%
\end{pgfscope}%
\begin{pgfscope}%
\pgfsetbuttcap%
\pgfsetroundjoin%
\definecolor{currentfill}{rgb}{0.000000,0.000000,0.000000}%
\pgfsetfillcolor{currentfill}%
\pgfsetlinewidth{0.602250pt}%
\definecolor{currentstroke}{rgb}{0.000000,0.000000,0.000000}%
\pgfsetstrokecolor{currentstroke}%
\pgfsetdash{}{0pt}%
\pgfsys@defobject{currentmarker}{\pgfqpoint{0.000000in}{-0.027778in}}{\pgfqpoint{0.000000in}{0.000000in}}{%
\pgfpathmoveto{\pgfqpoint{0.000000in}{0.000000in}}%
\pgfpathlineto{\pgfqpoint{0.000000in}{-0.027778in}}%
\pgfusepath{stroke,fill}%
}%
\begin{pgfscope}%
\pgfsys@transformshift{1.765628in}{2.498808in}%
\pgfsys@useobject{currentmarker}{}%
\end{pgfscope}%
\end{pgfscope}%
\begin{pgfscope}%
\pgfsetbuttcap%
\pgfsetroundjoin%
\definecolor{currentfill}{rgb}{0.000000,0.000000,0.000000}%
\pgfsetfillcolor{currentfill}%
\pgfsetlinewidth{0.602250pt}%
\definecolor{currentstroke}{rgb}{0.000000,0.000000,0.000000}%
\pgfsetstrokecolor{currentstroke}%
\pgfsetdash{}{0pt}%
\pgfsys@defobject{currentmarker}{\pgfqpoint{0.000000in}{-0.027778in}}{\pgfqpoint{0.000000in}{0.000000in}}{%
\pgfpathmoveto{\pgfqpoint{0.000000in}{0.000000in}}%
\pgfpathlineto{\pgfqpoint{0.000000in}{-0.027778in}}%
\pgfusepath{stroke,fill}%
}%
\begin{pgfscope}%
\pgfsys@transformshift{1.867225in}{2.498808in}%
\pgfsys@useobject{currentmarker}{}%
\end{pgfscope}%
\end{pgfscope}%
\begin{pgfscope}%
\pgfsetbuttcap%
\pgfsetroundjoin%
\definecolor{currentfill}{rgb}{0.000000,0.000000,0.000000}%
\pgfsetfillcolor{currentfill}%
\pgfsetlinewidth{0.602250pt}%
\definecolor{currentstroke}{rgb}{0.000000,0.000000,0.000000}%
\pgfsetstrokecolor{currentstroke}%
\pgfsetdash{}{0pt}%
\pgfsys@defobject{currentmarker}{\pgfqpoint{0.000000in}{-0.027778in}}{\pgfqpoint{0.000000in}{0.000000in}}{%
\pgfpathmoveto{\pgfqpoint{0.000000in}{0.000000in}}%
\pgfpathlineto{\pgfqpoint{0.000000in}{-0.027778in}}%
\pgfusepath{stroke,fill}%
}%
\begin{pgfscope}%
\pgfsys@transformshift{1.968821in}{2.498808in}%
\pgfsys@useobject{currentmarker}{}%
\end{pgfscope}%
\end{pgfscope}%
\begin{pgfscope}%
\pgfsetbuttcap%
\pgfsetroundjoin%
\definecolor{currentfill}{rgb}{0.000000,0.000000,0.000000}%
\pgfsetfillcolor{currentfill}%
\pgfsetlinewidth{0.602250pt}%
\definecolor{currentstroke}{rgb}{0.000000,0.000000,0.000000}%
\pgfsetstrokecolor{currentstroke}%
\pgfsetdash{}{0pt}%
\pgfsys@defobject{currentmarker}{\pgfqpoint{0.000000in}{-0.027778in}}{\pgfqpoint{0.000000in}{0.000000in}}{%
\pgfpathmoveto{\pgfqpoint{0.000000in}{0.000000in}}%
\pgfpathlineto{\pgfqpoint{0.000000in}{-0.027778in}}%
\pgfusepath{stroke,fill}%
}%
\begin{pgfscope}%
\pgfsys@transformshift{2.172013in}{2.498808in}%
\pgfsys@useobject{currentmarker}{}%
\end{pgfscope}%
\end{pgfscope}%
\begin{pgfscope}%
\pgfsetbuttcap%
\pgfsetroundjoin%
\definecolor{currentfill}{rgb}{0.000000,0.000000,0.000000}%
\pgfsetfillcolor{currentfill}%
\pgfsetlinewidth{0.602250pt}%
\definecolor{currentstroke}{rgb}{0.000000,0.000000,0.000000}%
\pgfsetstrokecolor{currentstroke}%
\pgfsetdash{}{0pt}%
\pgfsys@defobject{currentmarker}{\pgfqpoint{0.000000in}{-0.027778in}}{\pgfqpoint{0.000000in}{0.000000in}}{%
\pgfpathmoveto{\pgfqpoint{0.000000in}{0.000000in}}%
\pgfpathlineto{\pgfqpoint{0.000000in}{-0.027778in}}%
\pgfusepath{stroke,fill}%
}%
\begin{pgfscope}%
\pgfsys@transformshift{2.273609in}{2.498808in}%
\pgfsys@useobject{currentmarker}{}%
\end{pgfscope}%
\end{pgfscope}%
\begin{pgfscope}%
\pgfsetbuttcap%
\pgfsetroundjoin%
\definecolor{currentfill}{rgb}{0.000000,0.000000,0.000000}%
\pgfsetfillcolor{currentfill}%
\pgfsetlinewidth{0.602250pt}%
\definecolor{currentstroke}{rgb}{0.000000,0.000000,0.000000}%
\pgfsetstrokecolor{currentstroke}%
\pgfsetdash{}{0pt}%
\pgfsys@defobject{currentmarker}{\pgfqpoint{0.000000in}{-0.027778in}}{\pgfqpoint{0.000000in}{0.000000in}}{%
\pgfpathmoveto{\pgfqpoint{0.000000in}{0.000000in}}%
\pgfpathlineto{\pgfqpoint{0.000000in}{-0.027778in}}%
\pgfusepath{stroke,fill}%
}%
\begin{pgfscope}%
\pgfsys@transformshift{2.375206in}{2.498808in}%
\pgfsys@useobject{currentmarker}{}%
\end{pgfscope}%
\end{pgfscope}%
\begin{pgfscope}%
\pgfsetbuttcap%
\pgfsetroundjoin%
\definecolor{currentfill}{rgb}{0.000000,0.000000,0.000000}%
\pgfsetfillcolor{currentfill}%
\pgfsetlinewidth{0.602250pt}%
\definecolor{currentstroke}{rgb}{0.000000,0.000000,0.000000}%
\pgfsetstrokecolor{currentstroke}%
\pgfsetdash{}{0pt}%
\pgfsys@defobject{currentmarker}{\pgfqpoint{0.000000in}{-0.027778in}}{\pgfqpoint{0.000000in}{0.000000in}}{%
\pgfpathmoveto{\pgfqpoint{0.000000in}{0.000000in}}%
\pgfpathlineto{\pgfqpoint{0.000000in}{-0.027778in}}%
\pgfusepath{stroke,fill}%
}%
\begin{pgfscope}%
\pgfsys@transformshift{2.476802in}{2.498808in}%
\pgfsys@useobject{currentmarker}{}%
\end{pgfscope}%
\end{pgfscope}%
\begin{pgfscope}%
\pgfsetbuttcap%
\pgfsetroundjoin%
\definecolor{currentfill}{rgb}{0.000000,0.000000,0.000000}%
\pgfsetfillcolor{currentfill}%
\pgfsetlinewidth{0.602250pt}%
\definecolor{currentstroke}{rgb}{0.000000,0.000000,0.000000}%
\pgfsetstrokecolor{currentstroke}%
\pgfsetdash{}{0pt}%
\pgfsys@defobject{currentmarker}{\pgfqpoint{0.000000in}{-0.027778in}}{\pgfqpoint{0.000000in}{0.000000in}}{%
\pgfpathmoveto{\pgfqpoint{0.000000in}{0.000000in}}%
\pgfpathlineto{\pgfqpoint{0.000000in}{-0.027778in}}%
\pgfusepath{stroke,fill}%
}%
\begin{pgfscope}%
\pgfsys@transformshift{2.679994in}{2.498808in}%
\pgfsys@useobject{currentmarker}{}%
\end{pgfscope}%
\end{pgfscope}%
\begin{pgfscope}%
\pgfsetbuttcap%
\pgfsetroundjoin%
\definecolor{currentfill}{rgb}{0.000000,0.000000,0.000000}%
\pgfsetfillcolor{currentfill}%
\pgfsetlinewidth{0.602250pt}%
\definecolor{currentstroke}{rgb}{0.000000,0.000000,0.000000}%
\pgfsetstrokecolor{currentstroke}%
\pgfsetdash{}{0pt}%
\pgfsys@defobject{currentmarker}{\pgfqpoint{0.000000in}{-0.027778in}}{\pgfqpoint{0.000000in}{0.000000in}}{%
\pgfpathmoveto{\pgfqpoint{0.000000in}{0.000000in}}%
\pgfpathlineto{\pgfqpoint{0.000000in}{-0.027778in}}%
\pgfusepath{stroke,fill}%
}%
\begin{pgfscope}%
\pgfsys@transformshift{2.781590in}{2.498808in}%
\pgfsys@useobject{currentmarker}{}%
\end{pgfscope}%
\end{pgfscope}%
\begin{pgfscope}%
\pgfsetbuttcap%
\pgfsetroundjoin%
\definecolor{currentfill}{rgb}{0.000000,0.000000,0.000000}%
\pgfsetfillcolor{currentfill}%
\pgfsetlinewidth{0.602250pt}%
\definecolor{currentstroke}{rgb}{0.000000,0.000000,0.000000}%
\pgfsetstrokecolor{currentstroke}%
\pgfsetdash{}{0pt}%
\pgfsys@defobject{currentmarker}{\pgfqpoint{0.000000in}{-0.027778in}}{\pgfqpoint{0.000000in}{0.000000in}}{%
\pgfpathmoveto{\pgfqpoint{0.000000in}{0.000000in}}%
\pgfpathlineto{\pgfqpoint{0.000000in}{-0.027778in}}%
\pgfusepath{stroke,fill}%
}%
\begin{pgfscope}%
\pgfsys@transformshift{2.883187in}{2.498808in}%
\pgfsys@useobject{currentmarker}{}%
\end{pgfscope}%
\end{pgfscope}%
\begin{pgfscope}%
\pgfsetbuttcap%
\pgfsetroundjoin%
\definecolor{currentfill}{rgb}{0.000000,0.000000,0.000000}%
\pgfsetfillcolor{currentfill}%
\pgfsetlinewidth{0.803000pt}%
\definecolor{currentstroke}{rgb}{0.000000,0.000000,0.000000}%
\pgfsetstrokecolor{currentstroke}%
\pgfsetdash{}{0pt}%
\pgfsys@defobject{currentmarker}{\pgfqpoint{-0.048611in}{0.000000in}}{\pgfqpoint{-0.000000in}{0.000000in}}{%
\pgfpathmoveto{\pgfqpoint{-0.000000in}{0.000000in}}%
\pgfpathlineto{\pgfqpoint{-0.048611in}{0.000000in}}%
\pgfusepath{stroke,fill}%
}%
\begin{pgfscope}%
\pgfsys@transformshift{0.444878in}{2.553686in}%
\pgfsys@useobject{currentmarker}{}%
\end{pgfscope}%
\end{pgfscope}%
\begin{pgfscope}%
\definecolor{textcolor}{rgb}{0.000000,0.000000,0.000000}%
\pgfsetstrokecolor{textcolor}%
\pgfsetfillcolor{textcolor}%
\pgftext[x=0.278211in, y=2.505858in, left, base]{\color{textcolor}\rmfamily\fontsize{10.000000}{12.000000}\selectfont \(\displaystyle {0}\)}%
\end{pgfscope}%
\begin{pgfscope}%
\pgfsetbuttcap%
\pgfsetroundjoin%
\definecolor{currentfill}{rgb}{0.000000,0.000000,0.000000}%
\pgfsetfillcolor{currentfill}%
\pgfsetlinewidth{0.803000pt}%
\definecolor{currentstroke}{rgb}{0.000000,0.000000,0.000000}%
\pgfsetstrokecolor{currentstroke}%
\pgfsetdash{}{0pt}%
\pgfsys@defobject{currentmarker}{\pgfqpoint{-0.048611in}{0.000000in}}{\pgfqpoint{-0.000000in}{0.000000in}}{%
\pgfpathmoveto{\pgfqpoint{-0.000000in}{0.000000in}}%
\pgfpathlineto{\pgfqpoint{-0.048611in}{0.000000in}}%
\pgfusepath{stroke,fill}%
}%
\begin{pgfscope}%
\pgfsys@transformshift{0.444878in}{3.068360in}%
\pgfsys@useobject{currentmarker}{}%
\end{pgfscope}%
\end{pgfscope}%
\begin{pgfscope}%
\definecolor{textcolor}{rgb}{0.000000,0.000000,0.000000}%
\pgfsetstrokecolor{textcolor}%
\pgfsetfillcolor{textcolor}%
\pgftext[x=0.278211in, y=3.020532in, left, base]{\color{textcolor}\rmfamily\fontsize{10.000000}{12.000000}\selectfont \(\displaystyle {2}\)}%
\end{pgfscope}%
\begin{pgfscope}%
\pgfsetbuttcap%
\pgfsetroundjoin%
\definecolor{currentfill}{rgb}{0.000000,0.000000,0.000000}%
\pgfsetfillcolor{currentfill}%
\pgfsetlinewidth{0.803000pt}%
\definecolor{currentstroke}{rgb}{0.000000,0.000000,0.000000}%
\pgfsetstrokecolor{currentstroke}%
\pgfsetdash{}{0pt}%
\pgfsys@defobject{currentmarker}{\pgfqpoint{-0.048611in}{0.000000in}}{\pgfqpoint{-0.000000in}{0.000000in}}{%
\pgfpathmoveto{\pgfqpoint{-0.000000in}{0.000000in}}%
\pgfpathlineto{\pgfqpoint{-0.048611in}{0.000000in}}%
\pgfusepath{stroke,fill}%
}%
\begin{pgfscope}%
\pgfsys@transformshift{0.444878in}{3.583034in}%
\pgfsys@useobject{currentmarker}{}%
\end{pgfscope}%
\end{pgfscope}%
\begin{pgfscope}%
\definecolor{textcolor}{rgb}{0.000000,0.000000,0.000000}%
\pgfsetstrokecolor{textcolor}%
\pgfsetfillcolor{textcolor}%
\pgftext[x=0.278211in, y=3.535206in, left, base]{\color{textcolor}\rmfamily\fontsize{10.000000}{12.000000}\selectfont \(\displaystyle {4}\)}%
\end{pgfscope}%
\begin{pgfscope}%
\pgfsetbuttcap%
\pgfsetroundjoin%
\definecolor{currentfill}{rgb}{0.000000,0.000000,0.000000}%
\pgfsetfillcolor{currentfill}%
\pgfsetlinewidth{0.803000pt}%
\definecolor{currentstroke}{rgb}{0.000000,0.000000,0.000000}%
\pgfsetstrokecolor{currentstroke}%
\pgfsetdash{}{0pt}%
\pgfsys@defobject{currentmarker}{\pgfqpoint{-0.048611in}{0.000000in}}{\pgfqpoint{-0.000000in}{0.000000in}}{%
\pgfpathmoveto{\pgfqpoint{-0.000000in}{0.000000in}}%
\pgfpathlineto{\pgfqpoint{-0.048611in}{0.000000in}}%
\pgfusepath{stroke,fill}%
}%
\begin{pgfscope}%
\pgfsys@transformshift{0.444878in}{4.097708in}%
\pgfsys@useobject{currentmarker}{}%
\end{pgfscope}%
\end{pgfscope}%
\begin{pgfscope}%
\definecolor{textcolor}{rgb}{0.000000,0.000000,0.000000}%
\pgfsetstrokecolor{textcolor}%
\pgfsetfillcolor{textcolor}%
\pgftext[x=0.278211in, y=4.049880in, left, base]{\color{textcolor}\rmfamily\fontsize{10.000000}{12.000000}\selectfont \(\displaystyle {6}\)}%
\end{pgfscope}%
\begin{pgfscope}%
\pgfsetbuttcap%
\pgfsetroundjoin%
\definecolor{currentfill}{rgb}{0.000000,0.000000,0.000000}%
\pgfsetfillcolor{currentfill}%
\pgfsetlinewidth{0.602250pt}%
\definecolor{currentstroke}{rgb}{0.000000,0.000000,0.000000}%
\pgfsetstrokecolor{currentstroke}%
\pgfsetdash{}{0pt}%
\pgfsys@defobject{currentmarker}{\pgfqpoint{-0.027778in}{0.000000in}}{\pgfqpoint{-0.000000in}{0.000000in}}{%
\pgfpathmoveto{\pgfqpoint{-0.000000in}{0.000000in}}%
\pgfpathlineto{\pgfqpoint{-0.027778in}{0.000000in}}%
\pgfusepath{stroke,fill}%
}%
\begin{pgfscope}%
\pgfsys@transformshift{0.444878in}{2.682354in}%
\pgfsys@useobject{currentmarker}{}%
\end{pgfscope}%
\end{pgfscope}%
\begin{pgfscope}%
\pgfsetbuttcap%
\pgfsetroundjoin%
\definecolor{currentfill}{rgb}{0.000000,0.000000,0.000000}%
\pgfsetfillcolor{currentfill}%
\pgfsetlinewidth{0.602250pt}%
\definecolor{currentstroke}{rgb}{0.000000,0.000000,0.000000}%
\pgfsetstrokecolor{currentstroke}%
\pgfsetdash{}{0pt}%
\pgfsys@defobject{currentmarker}{\pgfqpoint{-0.027778in}{0.000000in}}{\pgfqpoint{-0.000000in}{0.000000in}}{%
\pgfpathmoveto{\pgfqpoint{-0.000000in}{0.000000in}}%
\pgfpathlineto{\pgfqpoint{-0.027778in}{0.000000in}}%
\pgfusepath{stroke,fill}%
}%
\begin{pgfscope}%
\pgfsys@transformshift{0.444878in}{2.811023in}%
\pgfsys@useobject{currentmarker}{}%
\end{pgfscope}%
\end{pgfscope}%
\begin{pgfscope}%
\pgfsetbuttcap%
\pgfsetroundjoin%
\definecolor{currentfill}{rgb}{0.000000,0.000000,0.000000}%
\pgfsetfillcolor{currentfill}%
\pgfsetlinewidth{0.602250pt}%
\definecolor{currentstroke}{rgb}{0.000000,0.000000,0.000000}%
\pgfsetstrokecolor{currentstroke}%
\pgfsetdash{}{0pt}%
\pgfsys@defobject{currentmarker}{\pgfqpoint{-0.027778in}{0.000000in}}{\pgfqpoint{-0.000000in}{0.000000in}}{%
\pgfpathmoveto{\pgfqpoint{-0.000000in}{0.000000in}}%
\pgfpathlineto{\pgfqpoint{-0.027778in}{0.000000in}}%
\pgfusepath{stroke,fill}%
}%
\begin{pgfscope}%
\pgfsys@transformshift{0.444878in}{2.939691in}%
\pgfsys@useobject{currentmarker}{}%
\end{pgfscope}%
\end{pgfscope}%
\begin{pgfscope}%
\pgfsetbuttcap%
\pgfsetroundjoin%
\definecolor{currentfill}{rgb}{0.000000,0.000000,0.000000}%
\pgfsetfillcolor{currentfill}%
\pgfsetlinewidth{0.602250pt}%
\definecolor{currentstroke}{rgb}{0.000000,0.000000,0.000000}%
\pgfsetstrokecolor{currentstroke}%
\pgfsetdash{}{0pt}%
\pgfsys@defobject{currentmarker}{\pgfqpoint{-0.027778in}{0.000000in}}{\pgfqpoint{-0.000000in}{0.000000in}}{%
\pgfpathmoveto{\pgfqpoint{-0.000000in}{0.000000in}}%
\pgfpathlineto{\pgfqpoint{-0.027778in}{0.000000in}}%
\pgfusepath{stroke,fill}%
}%
\begin{pgfscope}%
\pgfsys@transformshift{0.444878in}{3.197028in}%
\pgfsys@useobject{currentmarker}{}%
\end{pgfscope}%
\end{pgfscope}%
\begin{pgfscope}%
\pgfsetbuttcap%
\pgfsetroundjoin%
\definecolor{currentfill}{rgb}{0.000000,0.000000,0.000000}%
\pgfsetfillcolor{currentfill}%
\pgfsetlinewidth{0.602250pt}%
\definecolor{currentstroke}{rgb}{0.000000,0.000000,0.000000}%
\pgfsetstrokecolor{currentstroke}%
\pgfsetdash{}{0pt}%
\pgfsys@defobject{currentmarker}{\pgfqpoint{-0.027778in}{0.000000in}}{\pgfqpoint{-0.000000in}{0.000000in}}{%
\pgfpathmoveto{\pgfqpoint{-0.000000in}{0.000000in}}%
\pgfpathlineto{\pgfqpoint{-0.027778in}{0.000000in}}%
\pgfusepath{stroke,fill}%
}%
\begin{pgfscope}%
\pgfsys@transformshift{0.444878in}{3.325697in}%
\pgfsys@useobject{currentmarker}{}%
\end{pgfscope}%
\end{pgfscope}%
\begin{pgfscope}%
\pgfsetbuttcap%
\pgfsetroundjoin%
\definecolor{currentfill}{rgb}{0.000000,0.000000,0.000000}%
\pgfsetfillcolor{currentfill}%
\pgfsetlinewidth{0.602250pt}%
\definecolor{currentstroke}{rgb}{0.000000,0.000000,0.000000}%
\pgfsetstrokecolor{currentstroke}%
\pgfsetdash{}{0pt}%
\pgfsys@defobject{currentmarker}{\pgfqpoint{-0.027778in}{0.000000in}}{\pgfqpoint{-0.000000in}{0.000000in}}{%
\pgfpathmoveto{\pgfqpoint{-0.000000in}{0.000000in}}%
\pgfpathlineto{\pgfqpoint{-0.027778in}{0.000000in}}%
\pgfusepath{stroke,fill}%
}%
\begin{pgfscope}%
\pgfsys@transformshift{0.444878in}{3.454366in}%
\pgfsys@useobject{currentmarker}{}%
\end{pgfscope}%
\end{pgfscope}%
\begin{pgfscope}%
\pgfsetbuttcap%
\pgfsetroundjoin%
\definecolor{currentfill}{rgb}{0.000000,0.000000,0.000000}%
\pgfsetfillcolor{currentfill}%
\pgfsetlinewidth{0.602250pt}%
\definecolor{currentstroke}{rgb}{0.000000,0.000000,0.000000}%
\pgfsetstrokecolor{currentstroke}%
\pgfsetdash{}{0pt}%
\pgfsys@defobject{currentmarker}{\pgfqpoint{-0.027778in}{0.000000in}}{\pgfqpoint{-0.000000in}{0.000000in}}{%
\pgfpathmoveto{\pgfqpoint{-0.000000in}{0.000000in}}%
\pgfpathlineto{\pgfqpoint{-0.027778in}{0.000000in}}%
\pgfusepath{stroke,fill}%
}%
\begin{pgfscope}%
\pgfsys@transformshift{0.444878in}{3.711703in}%
\pgfsys@useobject{currentmarker}{}%
\end{pgfscope}%
\end{pgfscope}%
\begin{pgfscope}%
\pgfsetbuttcap%
\pgfsetroundjoin%
\definecolor{currentfill}{rgb}{0.000000,0.000000,0.000000}%
\pgfsetfillcolor{currentfill}%
\pgfsetlinewidth{0.602250pt}%
\definecolor{currentstroke}{rgb}{0.000000,0.000000,0.000000}%
\pgfsetstrokecolor{currentstroke}%
\pgfsetdash{}{0pt}%
\pgfsys@defobject{currentmarker}{\pgfqpoint{-0.027778in}{0.000000in}}{\pgfqpoint{-0.000000in}{0.000000in}}{%
\pgfpathmoveto{\pgfqpoint{-0.000000in}{0.000000in}}%
\pgfpathlineto{\pgfqpoint{-0.027778in}{0.000000in}}%
\pgfusepath{stroke,fill}%
}%
\begin{pgfscope}%
\pgfsys@transformshift{0.444878in}{3.840371in}%
\pgfsys@useobject{currentmarker}{}%
\end{pgfscope}%
\end{pgfscope}%
\begin{pgfscope}%
\pgfsetbuttcap%
\pgfsetroundjoin%
\definecolor{currentfill}{rgb}{0.000000,0.000000,0.000000}%
\pgfsetfillcolor{currentfill}%
\pgfsetlinewidth{0.602250pt}%
\definecolor{currentstroke}{rgb}{0.000000,0.000000,0.000000}%
\pgfsetstrokecolor{currentstroke}%
\pgfsetdash{}{0pt}%
\pgfsys@defobject{currentmarker}{\pgfqpoint{-0.027778in}{0.000000in}}{\pgfqpoint{-0.000000in}{0.000000in}}{%
\pgfpathmoveto{\pgfqpoint{-0.000000in}{0.000000in}}%
\pgfpathlineto{\pgfqpoint{-0.027778in}{0.000000in}}%
\pgfusepath{stroke,fill}%
}%
\begin{pgfscope}%
\pgfsys@transformshift{0.444878in}{3.969040in}%
\pgfsys@useobject{currentmarker}{}%
\end{pgfscope}%
\end{pgfscope}%
\begin{pgfscope}%
\pgfsetbuttcap%
\pgfsetroundjoin%
\definecolor{currentfill}{rgb}{0.000000,0.000000,0.000000}%
\pgfsetfillcolor{currentfill}%
\pgfsetlinewidth{0.602250pt}%
\definecolor{currentstroke}{rgb}{0.000000,0.000000,0.000000}%
\pgfsetstrokecolor{currentstroke}%
\pgfsetdash{}{0pt}%
\pgfsys@defobject{currentmarker}{\pgfqpoint{-0.027778in}{0.000000in}}{\pgfqpoint{-0.000000in}{0.000000in}}{%
\pgfpathmoveto{\pgfqpoint{-0.000000in}{0.000000in}}%
\pgfpathlineto{\pgfqpoint{-0.027778in}{0.000000in}}%
\pgfusepath{stroke,fill}%
}%
\begin{pgfscope}%
\pgfsys@transformshift{0.444878in}{4.226377in}%
\pgfsys@useobject{currentmarker}{}%
\end{pgfscope}%
\end{pgfscope}%
\begin{pgfscope}%
\definecolor{textcolor}{rgb}{0.000000,0.000000,0.000000}%
\pgfsetstrokecolor{textcolor}%
\pgfsetfillcolor{textcolor}%
\pgftext[x=0.222655in,y=3.399862in,,bottom,rotate=90.000000]{\color{textcolor}\rmfamily\fontsize{10.000000}{12.000000}\selectfont Quantumeffizienz}%
\end{pgfscope}%
\begin{pgfscope}%
\pgfsetrectcap%
\pgfsetmiterjoin%
\pgfsetlinewidth{0.803000pt}%
\definecolor{currentstroke}{rgb}{0.000000,0.000000,0.000000}%
\pgfsetstrokecolor{currentstroke}%
\pgfsetdash{}{0pt}%
\pgfpathmoveto{\pgfqpoint{0.444878in}{2.498808in}}%
\pgfpathlineto{\pgfqpoint{0.444878in}{4.300916in}}%
\pgfusepath{stroke}%
\end{pgfscope}%
\begin{pgfscope}%
\pgfsetrectcap%
\pgfsetmiterjoin%
\pgfsetlinewidth{0.803000pt}%
\definecolor{currentstroke}{rgb}{0.000000,0.000000,0.000000}%
\pgfsetstrokecolor{currentstroke}%
\pgfsetdash{}{0pt}%
\pgfpathmoveto{\pgfqpoint{2.923825in}{2.498808in}}%
\pgfpathlineto{\pgfqpoint{2.923825in}{4.300916in}}%
\pgfusepath{stroke}%
\end{pgfscope}%
\begin{pgfscope}%
\pgfsetrectcap%
\pgfsetmiterjoin%
\pgfsetlinewidth{0.803000pt}%
\definecolor{currentstroke}{rgb}{0.000000,0.000000,0.000000}%
\pgfsetstrokecolor{currentstroke}%
\pgfsetdash{}{0pt}%
\pgfpathmoveto{\pgfqpoint{0.444878in}{2.498808in}}%
\pgfpathlineto{\pgfqpoint{2.923825in}{2.498808in}}%
\pgfusepath{stroke}%
\end{pgfscope}%
\begin{pgfscope}%
\pgfsetrectcap%
\pgfsetmiterjoin%
\pgfsetlinewidth{0.803000pt}%
\definecolor{currentstroke}{rgb}{0.000000,0.000000,0.000000}%
\pgfsetstrokecolor{currentstroke}%
\pgfsetdash{}{0pt}%
\pgfpathmoveto{\pgfqpoint{0.444878in}{4.300916in}}%
\pgfpathlineto{\pgfqpoint{2.923825in}{4.300916in}}%
\pgfusepath{stroke}%
\end{pgfscope}%
\begin{pgfscope}%
\pgfpathrectangle{\pgfqpoint{0.444878in}{2.498808in}}{\pgfqpoint{2.478947in}{1.802109in}}%
\pgfusepath{clip}%
\pgfsetrectcap%
\pgfsetroundjoin%
\pgfsetlinewidth{1.505625pt}%
\definecolor{currentstroke}{rgb}{0.121569,0.466667,0.705882}%
\pgfsetstrokecolor{currentstroke}%
\pgfsetdash{}{0pt}%
\pgfpathmoveto{\pgfqpoint{0.546474in}{4.219002in}}%
\pgfpathlineto{\pgfqpoint{0.648070in}{3.785170in}}%
\pgfpathlineto{\pgfqpoint{0.749666in}{3.496168in}}%
\pgfpathlineto{\pgfqpoint{0.851262in}{3.300965in}}%
\pgfpathlineto{\pgfqpoint{0.952859in}{3.162763in}}%
\pgfpathlineto{\pgfqpoint{1.054455in}{3.068269in}}%
\pgfpathlineto{\pgfqpoint{1.156051in}{2.997337in}}%
\pgfpathlineto{\pgfqpoint{1.257647in}{2.941407in}}%
\pgfpathlineto{\pgfqpoint{1.359244in}{2.892292in}}%
\pgfpathlineto{\pgfqpoint{1.460840in}{2.853221in}}%
\pgfpathlineto{\pgfqpoint{1.562436in}{2.816197in}}%
\pgfpathlineto{\pgfqpoint{1.664032in}{2.782327in}}%
\pgfpathlineto{\pgfqpoint{1.765628in}{2.752625in}}%
\pgfpathlineto{\pgfqpoint{1.867225in}{2.724519in}}%
\pgfpathlineto{\pgfqpoint{1.968821in}{2.699304in}}%
\pgfpathlineto{\pgfqpoint{2.070417in}{2.675948in}}%
\pgfpathlineto{\pgfqpoint{2.172013in}{2.655709in}}%
\pgfpathlineto{\pgfqpoint{2.273609in}{2.636915in}}%
\pgfpathlineto{\pgfqpoint{2.375206in}{2.620787in}}%
\pgfpathlineto{\pgfqpoint{2.476802in}{2.607288in}}%
\pgfpathlineto{\pgfqpoint{2.578398in}{2.596511in}}%
\pgfpathlineto{\pgfqpoint{2.679994in}{2.588082in}}%
\pgfpathlineto{\pgfqpoint{2.781590in}{2.580722in}}%
\pgfusepath{stroke}%
\end{pgfscope}%
\begin{pgfscope}%
\pgfsetbuttcap%
\pgfsetmiterjoin%
\definecolor{currentfill}{rgb}{1.000000,1.000000,1.000000}%
\pgfsetfillcolor{currentfill}%
\pgfsetlinewidth{0.000000pt}%
\definecolor{currentstroke}{rgb}{0.000000,0.000000,0.000000}%
\pgfsetstrokecolor{currentstroke}%
\pgfsetstrokeopacity{0.000000}%
\pgfsetdash{}{0pt}%
\pgfpathmoveto{\pgfqpoint{3.073825in}{2.498808in}}%
\pgfpathlineto{\pgfqpoint{5.552773in}{2.498808in}}%
\pgfpathlineto{\pgfqpoint{5.552773in}{4.300916in}}%
\pgfpathlineto{\pgfqpoint{3.073825in}{4.300916in}}%
\pgfpathlineto{\pgfqpoint{3.073825in}{2.498808in}}%
\pgfpathclose%
\pgfusepath{fill}%
\end{pgfscope}%
\begin{pgfscope}%
\pgfsetbuttcap%
\pgfsetroundjoin%
\definecolor{currentfill}{rgb}{0.000000,0.000000,0.000000}%
\pgfsetfillcolor{currentfill}%
\pgfsetlinewidth{0.803000pt}%
\definecolor{currentstroke}{rgb}{0.000000,0.000000,0.000000}%
\pgfsetstrokecolor{currentstroke}%
\pgfsetdash{}{0pt}%
\pgfsys@defobject{currentmarker}{\pgfqpoint{0.000000in}{-0.048611in}}{\pgfqpoint{0.000000in}{0.000000in}}{%
\pgfpathmoveto{\pgfqpoint{0.000000in}{0.000000in}}%
\pgfpathlineto{\pgfqpoint{0.000000in}{-0.048611in}}%
\pgfusepath{stroke,fill}%
}%
\begin{pgfscope}%
\pgfsys@transformshift{3.175421in}{2.498808in}%
\pgfsys@useobject{currentmarker}{}%
\end{pgfscope}%
\end{pgfscope}%
\begin{pgfscope}%
\pgfsetbuttcap%
\pgfsetroundjoin%
\definecolor{currentfill}{rgb}{0.000000,0.000000,0.000000}%
\pgfsetfillcolor{currentfill}%
\pgfsetlinewidth{0.803000pt}%
\definecolor{currentstroke}{rgb}{0.000000,0.000000,0.000000}%
\pgfsetstrokecolor{currentstroke}%
\pgfsetdash{}{0pt}%
\pgfsys@defobject{currentmarker}{\pgfqpoint{0.000000in}{-0.048611in}}{\pgfqpoint{0.000000in}{0.000000in}}{%
\pgfpathmoveto{\pgfqpoint{0.000000in}{0.000000in}}%
\pgfpathlineto{\pgfqpoint{0.000000in}{-0.048611in}}%
\pgfusepath{stroke,fill}%
}%
\begin{pgfscope}%
\pgfsys@transformshift{3.683402in}{2.498808in}%
\pgfsys@useobject{currentmarker}{}%
\end{pgfscope}%
\end{pgfscope}%
\begin{pgfscope}%
\pgfsetbuttcap%
\pgfsetroundjoin%
\definecolor{currentfill}{rgb}{0.000000,0.000000,0.000000}%
\pgfsetfillcolor{currentfill}%
\pgfsetlinewidth{0.803000pt}%
\definecolor{currentstroke}{rgb}{0.000000,0.000000,0.000000}%
\pgfsetstrokecolor{currentstroke}%
\pgfsetdash{}{0pt}%
\pgfsys@defobject{currentmarker}{\pgfqpoint{0.000000in}{-0.048611in}}{\pgfqpoint{0.000000in}{0.000000in}}{%
\pgfpathmoveto{\pgfqpoint{0.000000in}{0.000000in}}%
\pgfpathlineto{\pgfqpoint{0.000000in}{-0.048611in}}%
\pgfusepath{stroke,fill}%
}%
\begin{pgfscope}%
\pgfsys@transformshift{4.191383in}{2.498808in}%
\pgfsys@useobject{currentmarker}{}%
\end{pgfscope}%
\end{pgfscope}%
\begin{pgfscope}%
\pgfsetbuttcap%
\pgfsetroundjoin%
\definecolor{currentfill}{rgb}{0.000000,0.000000,0.000000}%
\pgfsetfillcolor{currentfill}%
\pgfsetlinewidth{0.803000pt}%
\definecolor{currentstroke}{rgb}{0.000000,0.000000,0.000000}%
\pgfsetstrokecolor{currentstroke}%
\pgfsetdash{}{0pt}%
\pgfsys@defobject{currentmarker}{\pgfqpoint{0.000000in}{-0.048611in}}{\pgfqpoint{0.000000in}{0.000000in}}{%
\pgfpathmoveto{\pgfqpoint{0.000000in}{0.000000in}}%
\pgfpathlineto{\pgfqpoint{0.000000in}{-0.048611in}}%
\pgfusepath{stroke,fill}%
}%
\begin{pgfscope}%
\pgfsys@transformshift{4.699364in}{2.498808in}%
\pgfsys@useobject{currentmarker}{}%
\end{pgfscope}%
\end{pgfscope}%
\begin{pgfscope}%
\pgfsetbuttcap%
\pgfsetroundjoin%
\definecolor{currentfill}{rgb}{0.000000,0.000000,0.000000}%
\pgfsetfillcolor{currentfill}%
\pgfsetlinewidth{0.803000pt}%
\definecolor{currentstroke}{rgb}{0.000000,0.000000,0.000000}%
\pgfsetstrokecolor{currentstroke}%
\pgfsetdash{}{0pt}%
\pgfsys@defobject{currentmarker}{\pgfqpoint{0.000000in}{-0.048611in}}{\pgfqpoint{0.000000in}{0.000000in}}{%
\pgfpathmoveto{\pgfqpoint{0.000000in}{0.000000in}}%
\pgfpathlineto{\pgfqpoint{0.000000in}{-0.048611in}}%
\pgfusepath{stroke,fill}%
}%
\begin{pgfscope}%
\pgfsys@transformshift{5.207346in}{2.498808in}%
\pgfsys@useobject{currentmarker}{}%
\end{pgfscope}%
\end{pgfscope}%
\begin{pgfscope}%
\pgfsetbuttcap%
\pgfsetroundjoin%
\definecolor{currentfill}{rgb}{0.000000,0.000000,0.000000}%
\pgfsetfillcolor{currentfill}%
\pgfsetlinewidth{0.602250pt}%
\definecolor{currentstroke}{rgb}{0.000000,0.000000,0.000000}%
\pgfsetstrokecolor{currentstroke}%
\pgfsetdash{}{0pt}%
\pgfsys@defobject{currentmarker}{\pgfqpoint{0.000000in}{-0.027778in}}{\pgfqpoint{0.000000in}{0.000000in}}{%
\pgfpathmoveto{\pgfqpoint{0.000000in}{0.000000in}}%
\pgfpathlineto{\pgfqpoint{0.000000in}{-0.027778in}}%
\pgfusepath{stroke,fill}%
}%
\begin{pgfscope}%
\pgfsys@transformshift{3.073825in}{2.498808in}%
\pgfsys@useobject{currentmarker}{}%
\end{pgfscope}%
\end{pgfscope}%
\begin{pgfscope}%
\pgfsetbuttcap%
\pgfsetroundjoin%
\definecolor{currentfill}{rgb}{0.000000,0.000000,0.000000}%
\pgfsetfillcolor{currentfill}%
\pgfsetlinewidth{0.602250pt}%
\definecolor{currentstroke}{rgb}{0.000000,0.000000,0.000000}%
\pgfsetstrokecolor{currentstroke}%
\pgfsetdash{}{0pt}%
\pgfsys@defobject{currentmarker}{\pgfqpoint{0.000000in}{-0.027778in}}{\pgfqpoint{0.000000in}{0.000000in}}{%
\pgfpathmoveto{\pgfqpoint{0.000000in}{0.000000in}}%
\pgfpathlineto{\pgfqpoint{0.000000in}{-0.027778in}}%
\pgfusepath{stroke,fill}%
}%
\begin{pgfscope}%
\pgfsys@transformshift{3.277018in}{2.498808in}%
\pgfsys@useobject{currentmarker}{}%
\end{pgfscope}%
\end{pgfscope}%
\begin{pgfscope}%
\pgfsetbuttcap%
\pgfsetroundjoin%
\definecolor{currentfill}{rgb}{0.000000,0.000000,0.000000}%
\pgfsetfillcolor{currentfill}%
\pgfsetlinewidth{0.602250pt}%
\definecolor{currentstroke}{rgb}{0.000000,0.000000,0.000000}%
\pgfsetstrokecolor{currentstroke}%
\pgfsetdash{}{0pt}%
\pgfsys@defobject{currentmarker}{\pgfqpoint{0.000000in}{-0.027778in}}{\pgfqpoint{0.000000in}{0.000000in}}{%
\pgfpathmoveto{\pgfqpoint{0.000000in}{0.000000in}}%
\pgfpathlineto{\pgfqpoint{0.000000in}{-0.027778in}}%
\pgfusepath{stroke,fill}%
}%
\begin{pgfscope}%
\pgfsys@transformshift{3.378614in}{2.498808in}%
\pgfsys@useobject{currentmarker}{}%
\end{pgfscope}%
\end{pgfscope}%
\begin{pgfscope}%
\pgfsetbuttcap%
\pgfsetroundjoin%
\definecolor{currentfill}{rgb}{0.000000,0.000000,0.000000}%
\pgfsetfillcolor{currentfill}%
\pgfsetlinewidth{0.602250pt}%
\definecolor{currentstroke}{rgb}{0.000000,0.000000,0.000000}%
\pgfsetstrokecolor{currentstroke}%
\pgfsetdash{}{0pt}%
\pgfsys@defobject{currentmarker}{\pgfqpoint{0.000000in}{-0.027778in}}{\pgfqpoint{0.000000in}{0.000000in}}{%
\pgfpathmoveto{\pgfqpoint{0.000000in}{0.000000in}}%
\pgfpathlineto{\pgfqpoint{0.000000in}{-0.027778in}}%
\pgfusepath{stroke,fill}%
}%
\begin{pgfscope}%
\pgfsys@transformshift{3.480210in}{2.498808in}%
\pgfsys@useobject{currentmarker}{}%
\end{pgfscope}%
\end{pgfscope}%
\begin{pgfscope}%
\pgfsetbuttcap%
\pgfsetroundjoin%
\definecolor{currentfill}{rgb}{0.000000,0.000000,0.000000}%
\pgfsetfillcolor{currentfill}%
\pgfsetlinewidth{0.602250pt}%
\definecolor{currentstroke}{rgb}{0.000000,0.000000,0.000000}%
\pgfsetstrokecolor{currentstroke}%
\pgfsetdash{}{0pt}%
\pgfsys@defobject{currentmarker}{\pgfqpoint{0.000000in}{-0.027778in}}{\pgfqpoint{0.000000in}{0.000000in}}{%
\pgfpathmoveto{\pgfqpoint{0.000000in}{0.000000in}}%
\pgfpathlineto{\pgfqpoint{0.000000in}{-0.027778in}}%
\pgfusepath{stroke,fill}%
}%
\begin{pgfscope}%
\pgfsys@transformshift{3.581806in}{2.498808in}%
\pgfsys@useobject{currentmarker}{}%
\end{pgfscope}%
\end{pgfscope}%
\begin{pgfscope}%
\pgfsetbuttcap%
\pgfsetroundjoin%
\definecolor{currentfill}{rgb}{0.000000,0.000000,0.000000}%
\pgfsetfillcolor{currentfill}%
\pgfsetlinewidth{0.602250pt}%
\definecolor{currentstroke}{rgb}{0.000000,0.000000,0.000000}%
\pgfsetstrokecolor{currentstroke}%
\pgfsetdash{}{0pt}%
\pgfsys@defobject{currentmarker}{\pgfqpoint{0.000000in}{-0.027778in}}{\pgfqpoint{0.000000in}{0.000000in}}{%
\pgfpathmoveto{\pgfqpoint{0.000000in}{0.000000in}}%
\pgfpathlineto{\pgfqpoint{0.000000in}{-0.027778in}}%
\pgfusepath{stroke,fill}%
}%
\begin{pgfscope}%
\pgfsys@transformshift{3.784999in}{2.498808in}%
\pgfsys@useobject{currentmarker}{}%
\end{pgfscope}%
\end{pgfscope}%
\begin{pgfscope}%
\pgfsetbuttcap%
\pgfsetroundjoin%
\definecolor{currentfill}{rgb}{0.000000,0.000000,0.000000}%
\pgfsetfillcolor{currentfill}%
\pgfsetlinewidth{0.602250pt}%
\definecolor{currentstroke}{rgb}{0.000000,0.000000,0.000000}%
\pgfsetstrokecolor{currentstroke}%
\pgfsetdash{}{0pt}%
\pgfsys@defobject{currentmarker}{\pgfqpoint{0.000000in}{-0.027778in}}{\pgfqpoint{0.000000in}{0.000000in}}{%
\pgfpathmoveto{\pgfqpoint{0.000000in}{0.000000in}}%
\pgfpathlineto{\pgfqpoint{0.000000in}{-0.027778in}}%
\pgfusepath{stroke,fill}%
}%
\begin{pgfscope}%
\pgfsys@transformshift{3.886595in}{2.498808in}%
\pgfsys@useobject{currentmarker}{}%
\end{pgfscope}%
\end{pgfscope}%
\begin{pgfscope}%
\pgfsetbuttcap%
\pgfsetroundjoin%
\definecolor{currentfill}{rgb}{0.000000,0.000000,0.000000}%
\pgfsetfillcolor{currentfill}%
\pgfsetlinewidth{0.602250pt}%
\definecolor{currentstroke}{rgb}{0.000000,0.000000,0.000000}%
\pgfsetstrokecolor{currentstroke}%
\pgfsetdash{}{0pt}%
\pgfsys@defobject{currentmarker}{\pgfqpoint{0.000000in}{-0.027778in}}{\pgfqpoint{0.000000in}{0.000000in}}{%
\pgfpathmoveto{\pgfqpoint{0.000000in}{0.000000in}}%
\pgfpathlineto{\pgfqpoint{0.000000in}{-0.027778in}}%
\pgfusepath{stroke,fill}%
}%
\begin{pgfscope}%
\pgfsys@transformshift{3.988191in}{2.498808in}%
\pgfsys@useobject{currentmarker}{}%
\end{pgfscope}%
\end{pgfscope}%
\begin{pgfscope}%
\pgfsetbuttcap%
\pgfsetroundjoin%
\definecolor{currentfill}{rgb}{0.000000,0.000000,0.000000}%
\pgfsetfillcolor{currentfill}%
\pgfsetlinewidth{0.602250pt}%
\definecolor{currentstroke}{rgb}{0.000000,0.000000,0.000000}%
\pgfsetstrokecolor{currentstroke}%
\pgfsetdash{}{0pt}%
\pgfsys@defobject{currentmarker}{\pgfqpoint{0.000000in}{-0.027778in}}{\pgfqpoint{0.000000in}{0.000000in}}{%
\pgfpathmoveto{\pgfqpoint{0.000000in}{0.000000in}}%
\pgfpathlineto{\pgfqpoint{0.000000in}{-0.027778in}}%
\pgfusepath{stroke,fill}%
}%
\begin{pgfscope}%
\pgfsys@transformshift{4.089787in}{2.498808in}%
\pgfsys@useobject{currentmarker}{}%
\end{pgfscope}%
\end{pgfscope}%
\begin{pgfscope}%
\pgfsetbuttcap%
\pgfsetroundjoin%
\definecolor{currentfill}{rgb}{0.000000,0.000000,0.000000}%
\pgfsetfillcolor{currentfill}%
\pgfsetlinewidth{0.602250pt}%
\definecolor{currentstroke}{rgb}{0.000000,0.000000,0.000000}%
\pgfsetstrokecolor{currentstroke}%
\pgfsetdash{}{0pt}%
\pgfsys@defobject{currentmarker}{\pgfqpoint{0.000000in}{-0.027778in}}{\pgfqpoint{0.000000in}{0.000000in}}{%
\pgfpathmoveto{\pgfqpoint{0.000000in}{0.000000in}}%
\pgfpathlineto{\pgfqpoint{0.000000in}{-0.027778in}}%
\pgfusepath{stroke,fill}%
}%
\begin{pgfscope}%
\pgfsys@transformshift{4.292980in}{2.498808in}%
\pgfsys@useobject{currentmarker}{}%
\end{pgfscope}%
\end{pgfscope}%
\begin{pgfscope}%
\pgfsetbuttcap%
\pgfsetroundjoin%
\definecolor{currentfill}{rgb}{0.000000,0.000000,0.000000}%
\pgfsetfillcolor{currentfill}%
\pgfsetlinewidth{0.602250pt}%
\definecolor{currentstroke}{rgb}{0.000000,0.000000,0.000000}%
\pgfsetstrokecolor{currentstroke}%
\pgfsetdash{}{0pt}%
\pgfsys@defobject{currentmarker}{\pgfqpoint{0.000000in}{-0.027778in}}{\pgfqpoint{0.000000in}{0.000000in}}{%
\pgfpathmoveto{\pgfqpoint{0.000000in}{0.000000in}}%
\pgfpathlineto{\pgfqpoint{0.000000in}{-0.027778in}}%
\pgfusepath{stroke,fill}%
}%
\begin{pgfscope}%
\pgfsys@transformshift{4.394576in}{2.498808in}%
\pgfsys@useobject{currentmarker}{}%
\end{pgfscope}%
\end{pgfscope}%
\begin{pgfscope}%
\pgfsetbuttcap%
\pgfsetroundjoin%
\definecolor{currentfill}{rgb}{0.000000,0.000000,0.000000}%
\pgfsetfillcolor{currentfill}%
\pgfsetlinewidth{0.602250pt}%
\definecolor{currentstroke}{rgb}{0.000000,0.000000,0.000000}%
\pgfsetstrokecolor{currentstroke}%
\pgfsetdash{}{0pt}%
\pgfsys@defobject{currentmarker}{\pgfqpoint{0.000000in}{-0.027778in}}{\pgfqpoint{0.000000in}{0.000000in}}{%
\pgfpathmoveto{\pgfqpoint{0.000000in}{0.000000in}}%
\pgfpathlineto{\pgfqpoint{0.000000in}{-0.027778in}}%
\pgfusepath{stroke,fill}%
}%
\begin{pgfscope}%
\pgfsys@transformshift{4.496172in}{2.498808in}%
\pgfsys@useobject{currentmarker}{}%
\end{pgfscope}%
\end{pgfscope}%
\begin{pgfscope}%
\pgfsetbuttcap%
\pgfsetroundjoin%
\definecolor{currentfill}{rgb}{0.000000,0.000000,0.000000}%
\pgfsetfillcolor{currentfill}%
\pgfsetlinewidth{0.602250pt}%
\definecolor{currentstroke}{rgb}{0.000000,0.000000,0.000000}%
\pgfsetstrokecolor{currentstroke}%
\pgfsetdash{}{0pt}%
\pgfsys@defobject{currentmarker}{\pgfqpoint{0.000000in}{-0.027778in}}{\pgfqpoint{0.000000in}{0.000000in}}{%
\pgfpathmoveto{\pgfqpoint{0.000000in}{0.000000in}}%
\pgfpathlineto{\pgfqpoint{0.000000in}{-0.027778in}}%
\pgfusepath{stroke,fill}%
}%
\begin{pgfscope}%
\pgfsys@transformshift{4.597768in}{2.498808in}%
\pgfsys@useobject{currentmarker}{}%
\end{pgfscope}%
\end{pgfscope}%
\begin{pgfscope}%
\pgfsetbuttcap%
\pgfsetroundjoin%
\definecolor{currentfill}{rgb}{0.000000,0.000000,0.000000}%
\pgfsetfillcolor{currentfill}%
\pgfsetlinewidth{0.602250pt}%
\definecolor{currentstroke}{rgb}{0.000000,0.000000,0.000000}%
\pgfsetstrokecolor{currentstroke}%
\pgfsetdash{}{0pt}%
\pgfsys@defobject{currentmarker}{\pgfqpoint{0.000000in}{-0.027778in}}{\pgfqpoint{0.000000in}{0.000000in}}{%
\pgfpathmoveto{\pgfqpoint{0.000000in}{0.000000in}}%
\pgfpathlineto{\pgfqpoint{0.000000in}{-0.027778in}}%
\pgfusepath{stroke,fill}%
}%
\begin{pgfscope}%
\pgfsys@transformshift{4.800961in}{2.498808in}%
\pgfsys@useobject{currentmarker}{}%
\end{pgfscope}%
\end{pgfscope}%
\begin{pgfscope}%
\pgfsetbuttcap%
\pgfsetroundjoin%
\definecolor{currentfill}{rgb}{0.000000,0.000000,0.000000}%
\pgfsetfillcolor{currentfill}%
\pgfsetlinewidth{0.602250pt}%
\definecolor{currentstroke}{rgb}{0.000000,0.000000,0.000000}%
\pgfsetstrokecolor{currentstroke}%
\pgfsetdash{}{0pt}%
\pgfsys@defobject{currentmarker}{\pgfqpoint{0.000000in}{-0.027778in}}{\pgfqpoint{0.000000in}{0.000000in}}{%
\pgfpathmoveto{\pgfqpoint{0.000000in}{0.000000in}}%
\pgfpathlineto{\pgfqpoint{0.000000in}{-0.027778in}}%
\pgfusepath{stroke,fill}%
}%
\begin{pgfscope}%
\pgfsys@transformshift{4.902557in}{2.498808in}%
\pgfsys@useobject{currentmarker}{}%
\end{pgfscope}%
\end{pgfscope}%
\begin{pgfscope}%
\pgfsetbuttcap%
\pgfsetroundjoin%
\definecolor{currentfill}{rgb}{0.000000,0.000000,0.000000}%
\pgfsetfillcolor{currentfill}%
\pgfsetlinewidth{0.602250pt}%
\definecolor{currentstroke}{rgb}{0.000000,0.000000,0.000000}%
\pgfsetstrokecolor{currentstroke}%
\pgfsetdash{}{0pt}%
\pgfsys@defobject{currentmarker}{\pgfqpoint{0.000000in}{-0.027778in}}{\pgfqpoint{0.000000in}{0.000000in}}{%
\pgfpathmoveto{\pgfqpoint{0.000000in}{0.000000in}}%
\pgfpathlineto{\pgfqpoint{0.000000in}{-0.027778in}}%
\pgfusepath{stroke,fill}%
}%
\begin{pgfscope}%
\pgfsys@transformshift{5.004153in}{2.498808in}%
\pgfsys@useobject{currentmarker}{}%
\end{pgfscope}%
\end{pgfscope}%
\begin{pgfscope}%
\pgfsetbuttcap%
\pgfsetroundjoin%
\definecolor{currentfill}{rgb}{0.000000,0.000000,0.000000}%
\pgfsetfillcolor{currentfill}%
\pgfsetlinewidth{0.602250pt}%
\definecolor{currentstroke}{rgb}{0.000000,0.000000,0.000000}%
\pgfsetstrokecolor{currentstroke}%
\pgfsetdash{}{0pt}%
\pgfsys@defobject{currentmarker}{\pgfqpoint{0.000000in}{-0.027778in}}{\pgfqpoint{0.000000in}{0.000000in}}{%
\pgfpathmoveto{\pgfqpoint{0.000000in}{0.000000in}}%
\pgfpathlineto{\pgfqpoint{0.000000in}{-0.027778in}}%
\pgfusepath{stroke,fill}%
}%
\begin{pgfscope}%
\pgfsys@transformshift{5.105749in}{2.498808in}%
\pgfsys@useobject{currentmarker}{}%
\end{pgfscope}%
\end{pgfscope}%
\begin{pgfscope}%
\pgfsetbuttcap%
\pgfsetroundjoin%
\definecolor{currentfill}{rgb}{0.000000,0.000000,0.000000}%
\pgfsetfillcolor{currentfill}%
\pgfsetlinewidth{0.602250pt}%
\definecolor{currentstroke}{rgb}{0.000000,0.000000,0.000000}%
\pgfsetstrokecolor{currentstroke}%
\pgfsetdash{}{0pt}%
\pgfsys@defobject{currentmarker}{\pgfqpoint{0.000000in}{-0.027778in}}{\pgfqpoint{0.000000in}{0.000000in}}{%
\pgfpathmoveto{\pgfqpoint{0.000000in}{0.000000in}}%
\pgfpathlineto{\pgfqpoint{0.000000in}{-0.027778in}}%
\pgfusepath{stroke,fill}%
}%
\begin{pgfscope}%
\pgfsys@transformshift{5.308942in}{2.498808in}%
\pgfsys@useobject{currentmarker}{}%
\end{pgfscope}%
\end{pgfscope}%
\begin{pgfscope}%
\pgfsetbuttcap%
\pgfsetroundjoin%
\definecolor{currentfill}{rgb}{0.000000,0.000000,0.000000}%
\pgfsetfillcolor{currentfill}%
\pgfsetlinewidth{0.602250pt}%
\definecolor{currentstroke}{rgb}{0.000000,0.000000,0.000000}%
\pgfsetstrokecolor{currentstroke}%
\pgfsetdash{}{0pt}%
\pgfsys@defobject{currentmarker}{\pgfqpoint{0.000000in}{-0.027778in}}{\pgfqpoint{0.000000in}{0.000000in}}{%
\pgfpathmoveto{\pgfqpoint{0.000000in}{0.000000in}}%
\pgfpathlineto{\pgfqpoint{0.000000in}{-0.027778in}}%
\pgfusepath{stroke,fill}%
}%
\begin{pgfscope}%
\pgfsys@transformshift{5.410538in}{2.498808in}%
\pgfsys@useobject{currentmarker}{}%
\end{pgfscope}%
\end{pgfscope}%
\begin{pgfscope}%
\pgfsetbuttcap%
\pgfsetroundjoin%
\definecolor{currentfill}{rgb}{0.000000,0.000000,0.000000}%
\pgfsetfillcolor{currentfill}%
\pgfsetlinewidth{0.602250pt}%
\definecolor{currentstroke}{rgb}{0.000000,0.000000,0.000000}%
\pgfsetstrokecolor{currentstroke}%
\pgfsetdash{}{0pt}%
\pgfsys@defobject{currentmarker}{\pgfqpoint{0.000000in}{-0.027778in}}{\pgfqpoint{0.000000in}{0.000000in}}{%
\pgfpathmoveto{\pgfqpoint{0.000000in}{0.000000in}}%
\pgfpathlineto{\pgfqpoint{0.000000in}{-0.027778in}}%
\pgfusepath{stroke,fill}%
}%
\begin{pgfscope}%
\pgfsys@transformshift{5.512134in}{2.498808in}%
\pgfsys@useobject{currentmarker}{}%
\end{pgfscope}%
\end{pgfscope}%
\begin{pgfscope}%
\pgfsetbuttcap%
\pgfsetroundjoin%
\definecolor{currentfill}{rgb}{0.000000,0.000000,0.000000}%
\pgfsetfillcolor{currentfill}%
\pgfsetlinewidth{0.803000pt}%
\definecolor{currentstroke}{rgb}{0.000000,0.000000,0.000000}%
\pgfsetstrokecolor{currentstroke}%
\pgfsetdash{}{0pt}%
\pgfsys@defobject{currentmarker}{\pgfqpoint{0.000000in}{0.000000in}}{\pgfqpoint{0.048611in}{0.000000in}}{%
\pgfpathmoveto{\pgfqpoint{0.000000in}{0.000000in}}%
\pgfpathlineto{\pgfqpoint{0.048611in}{0.000000in}}%
\pgfusepath{stroke,fill}%
}%
\begin{pgfscope}%
\pgfsys@transformshift{5.552773in}{2.696071in}%
\pgfsys@useobject{currentmarker}{}%
\end{pgfscope}%
\end{pgfscope}%
\begin{pgfscope}%
\definecolor{textcolor}{rgb}{0.000000,0.000000,0.000000}%
\pgfsetstrokecolor{textcolor}%
\pgfsetfillcolor{textcolor}%
\pgftext[x=5.649995in, y=2.648244in, left, base]{\color{textcolor}\rmfamily\fontsize{10.000000}{12.000000}\selectfont \(\displaystyle {10^{-6}}\)}%
\end{pgfscope}%
\begin{pgfscope}%
\pgfsetbuttcap%
\pgfsetroundjoin%
\definecolor{currentfill}{rgb}{0.000000,0.000000,0.000000}%
\pgfsetfillcolor{currentfill}%
\pgfsetlinewidth{0.803000pt}%
\definecolor{currentstroke}{rgb}{0.000000,0.000000,0.000000}%
\pgfsetstrokecolor{currentstroke}%
\pgfsetdash{}{0pt}%
\pgfsys@defobject{currentmarker}{\pgfqpoint{0.000000in}{0.000000in}}{\pgfqpoint{0.048611in}{0.000000in}}{%
\pgfpathmoveto{\pgfqpoint{0.000000in}{0.000000in}}%
\pgfpathlineto{\pgfqpoint{0.048611in}{0.000000in}}%
\pgfusepath{stroke,fill}%
}%
\begin{pgfscope}%
\pgfsys@transformshift{5.552773in}{3.079255in}%
\pgfsys@useobject{currentmarker}{}%
\end{pgfscope}%
\end{pgfscope}%
\begin{pgfscope}%
\definecolor{textcolor}{rgb}{0.000000,0.000000,0.000000}%
\pgfsetstrokecolor{textcolor}%
\pgfsetfillcolor{textcolor}%
\pgftext[x=5.649995in, y=3.031427in, left, base]{\color{textcolor}\rmfamily\fontsize{10.000000}{12.000000}\selectfont \(\displaystyle {10^{-5}}\)}%
\end{pgfscope}%
\begin{pgfscope}%
\pgfsetbuttcap%
\pgfsetroundjoin%
\definecolor{currentfill}{rgb}{0.000000,0.000000,0.000000}%
\pgfsetfillcolor{currentfill}%
\pgfsetlinewidth{0.803000pt}%
\definecolor{currentstroke}{rgb}{0.000000,0.000000,0.000000}%
\pgfsetstrokecolor{currentstroke}%
\pgfsetdash{}{0pt}%
\pgfsys@defobject{currentmarker}{\pgfqpoint{0.000000in}{0.000000in}}{\pgfqpoint{0.048611in}{0.000000in}}{%
\pgfpathmoveto{\pgfqpoint{0.000000in}{0.000000in}}%
\pgfpathlineto{\pgfqpoint{0.048611in}{0.000000in}}%
\pgfusepath{stroke,fill}%
}%
\begin{pgfscope}%
\pgfsys@transformshift{5.552773in}{3.462438in}%
\pgfsys@useobject{currentmarker}{}%
\end{pgfscope}%
\end{pgfscope}%
\begin{pgfscope}%
\definecolor{textcolor}{rgb}{0.000000,0.000000,0.000000}%
\pgfsetstrokecolor{textcolor}%
\pgfsetfillcolor{textcolor}%
\pgftext[x=5.649995in, y=3.414610in, left, base]{\color{textcolor}\rmfamily\fontsize{10.000000}{12.000000}\selectfont \(\displaystyle {10^{-4}}\)}%
\end{pgfscope}%
\begin{pgfscope}%
\pgfsetbuttcap%
\pgfsetroundjoin%
\definecolor{currentfill}{rgb}{0.000000,0.000000,0.000000}%
\pgfsetfillcolor{currentfill}%
\pgfsetlinewidth{0.803000pt}%
\definecolor{currentstroke}{rgb}{0.000000,0.000000,0.000000}%
\pgfsetstrokecolor{currentstroke}%
\pgfsetdash{}{0pt}%
\pgfsys@defobject{currentmarker}{\pgfqpoint{0.000000in}{0.000000in}}{\pgfqpoint{0.048611in}{0.000000in}}{%
\pgfpathmoveto{\pgfqpoint{0.000000in}{0.000000in}}%
\pgfpathlineto{\pgfqpoint{0.048611in}{0.000000in}}%
\pgfusepath{stroke,fill}%
}%
\begin{pgfscope}%
\pgfsys@transformshift{5.552773in}{3.845621in}%
\pgfsys@useobject{currentmarker}{}%
\end{pgfscope}%
\end{pgfscope}%
\begin{pgfscope}%
\definecolor{textcolor}{rgb}{0.000000,0.000000,0.000000}%
\pgfsetstrokecolor{textcolor}%
\pgfsetfillcolor{textcolor}%
\pgftext[x=5.649995in, y=3.797793in, left, base]{\color{textcolor}\rmfamily\fontsize{10.000000}{12.000000}\selectfont \(\displaystyle {10^{-3}}\)}%
\end{pgfscope}%
\begin{pgfscope}%
\pgfsetbuttcap%
\pgfsetroundjoin%
\definecolor{currentfill}{rgb}{0.000000,0.000000,0.000000}%
\pgfsetfillcolor{currentfill}%
\pgfsetlinewidth{0.803000pt}%
\definecolor{currentstroke}{rgb}{0.000000,0.000000,0.000000}%
\pgfsetstrokecolor{currentstroke}%
\pgfsetdash{}{0pt}%
\pgfsys@defobject{currentmarker}{\pgfqpoint{0.000000in}{0.000000in}}{\pgfqpoint{0.048611in}{0.000000in}}{%
\pgfpathmoveto{\pgfqpoint{0.000000in}{0.000000in}}%
\pgfpathlineto{\pgfqpoint{0.048611in}{0.000000in}}%
\pgfusepath{stroke,fill}%
}%
\begin{pgfscope}%
\pgfsys@transformshift{5.552773in}{4.228804in}%
\pgfsys@useobject{currentmarker}{}%
\end{pgfscope}%
\end{pgfscope}%
\begin{pgfscope}%
\definecolor{textcolor}{rgb}{0.000000,0.000000,0.000000}%
\pgfsetstrokecolor{textcolor}%
\pgfsetfillcolor{textcolor}%
\pgftext[x=5.649995in, y=4.180976in, left, base]{\color{textcolor}\rmfamily\fontsize{10.000000}{12.000000}\selectfont \(\displaystyle {10^{-2}}\)}%
\end{pgfscope}%
\begin{pgfscope}%
\pgfsetbuttcap%
\pgfsetroundjoin%
\definecolor{currentfill}{rgb}{0.000000,0.000000,0.000000}%
\pgfsetfillcolor{currentfill}%
\pgfsetlinewidth{0.602250pt}%
\definecolor{currentstroke}{rgb}{0.000000,0.000000,0.000000}%
\pgfsetstrokecolor{currentstroke}%
\pgfsetdash{}{0pt}%
\pgfsys@defobject{currentmarker}{\pgfqpoint{0.000000in}{0.000000in}}{\pgfqpoint{0.027778in}{0.000000in}}{%
\pgfpathmoveto{\pgfqpoint{0.000000in}{0.000000in}}%
\pgfpathlineto{\pgfqpoint{0.027778in}{0.000000in}}%
\pgfusepath{stroke,fill}%
}%
\begin{pgfscope}%
\pgfsys@transformshift{5.552773in}{2.543587in}%
\pgfsys@useobject{currentmarker}{}%
\end{pgfscope}%
\end{pgfscope}%
\begin{pgfscope}%
\pgfsetbuttcap%
\pgfsetroundjoin%
\definecolor{currentfill}{rgb}{0.000000,0.000000,0.000000}%
\pgfsetfillcolor{currentfill}%
\pgfsetlinewidth{0.602250pt}%
\definecolor{currentstroke}{rgb}{0.000000,0.000000,0.000000}%
\pgfsetstrokecolor{currentstroke}%
\pgfsetdash{}{0pt}%
\pgfsys@defobject{currentmarker}{\pgfqpoint{0.000000in}{0.000000in}}{\pgfqpoint{0.027778in}{0.000000in}}{%
\pgfpathmoveto{\pgfqpoint{0.000000in}{0.000000in}}%
\pgfpathlineto{\pgfqpoint{0.027778in}{0.000000in}}%
\pgfusepath{stroke,fill}%
}%
\begin{pgfscope}%
\pgfsys@transformshift{5.552773in}{2.580722in}%
\pgfsys@useobject{currentmarker}{}%
\end{pgfscope}%
\end{pgfscope}%
\begin{pgfscope}%
\pgfsetbuttcap%
\pgfsetroundjoin%
\definecolor{currentfill}{rgb}{0.000000,0.000000,0.000000}%
\pgfsetfillcolor{currentfill}%
\pgfsetlinewidth{0.602250pt}%
\definecolor{currentstroke}{rgb}{0.000000,0.000000,0.000000}%
\pgfsetstrokecolor{currentstroke}%
\pgfsetdash{}{0pt}%
\pgfsys@defobject{currentmarker}{\pgfqpoint{0.000000in}{0.000000in}}{\pgfqpoint{0.027778in}{0.000000in}}{%
\pgfpathmoveto{\pgfqpoint{0.000000in}{0.000000in}}%
\pgfpathlineto{\pgfqpoint{0.027778in}{0.000000in}}%
\pgfusepath{stroke,fill}%
}%
\begin{pgfscope}%
\pgfsys@transformshift{5.552773in}{2.611063in}%
\pgfsys@useobject{currentmarker}{}%
\end{pgfscope}%
\end{pgfscope}%
\begin{pgfscope}%
\pgfsetbuttcap%
\pgfsetroundjoin%
\definecolor{currentfill}{rgb}{0.000000,0.000000,0.000000}%
\pgfsetfillcolor{currentfill}%
\pgfsetlinewidth{0.602250pt}%
\definecolor{currentstroke}{rgb}{0.000000,0.000000,0.000000}%
\pgfsetstrokecolor{currentstroke}%
\pgfsetdash{}{0pt}%
\pgfsys@defobject{currentmarker}{\pgfqpoint{0.000000in}{0.000000in}}{\pgfqpoint{0.027778in}{0.000000in}}{%
\pgfpathmoveto{\pgfqpoint{0.000000in}{0.000000in}}%
\pgfpathlineto{\pgfqpoint{0.027778in}{0.000000in}}%
\pgfusepath{stroke,fill}%
}%
\begin{pgfscope}%
\pgfsys@transformshift{5.552773in}{2.636716in}%
\pgfsys@useobject{currentmarker}{}%
\end{pgfscope}%
\end{pgfscope}%
\begin{pgfscope}%
\pgfsetbuttcap%
\pgfsetroundjoin%
\definecolor{currentfill}{rgb}{0.000000,0.000000,0.000000}%
\pgfsetfillcolor{currentfill}%
\pgfsetlinewidth{0.602250pt}%
\definecolor{currentstroke}{rgb}{0.000000,0.000000,0.000000}%
\pgfsetstrokecolor{currentstroke}%
\pgfsetdash{}{0pt}%
\pgfsys@defobject{currentmarker}{\pgfqpoint{0.000000in}{0.000000in}}{\pgfqpoint{0.027778in}{0.000000in}}{%
\pgfpathmoveto{\pgfqpoint{0.000000in}{0.000000in}}%
\pgfpathlineto{\pgfqpoint{0.027778in}{0.000000in}}%
\pgfusepath{stroke,fill}%
}%
\begin{pgfscope}%
\pgfsys@transformshift{5.552773in}{2.658937in}%
\pgfsys@useobject{currentmarker}{}%
\end{pgfscope}%
\end{pgfscope}%
\begin{pgfscope}%
\pgfsetbuttcap%
\pgfsetroundjoin%
\definecolor{currentfill}{rgb}{0.000000,0.000000,0.000000}%
\pgfsetfillcolor{currentfill}%
\pgfsetlinewidth{0.602250pt}%
\definecolor{currentstroke}{rgb}{0.000000,0.000000,0.000000}%
\pgfsetstrokecolor{currentstroke}%
\pgfsetdash{}{0pt}%
\pgfsys@defobject{currentmarker}{\pgfqpoint{0.000000in}{0.000000in}}{\pgfqpoint{0.027778in}{0.000000in}}{%
\pgfpathmoveto{\pgfqpoint{0.000000in}{0.000000in}}%
\pgfpathlineto{\pgfqpoint{0.027778in}{0.000000in}}%
\pgfusepath{stroke,fill}%
}%
\begin{pgfscope}%
\pgfsys@transformshift{5.552773in}{2.678538in}%
\pgfsys@useobject{currentmarker}{}%
\end{pgfscope}%
\end{pgfscope}%
\begin{pgfscope}%
\pgfsetbuttcap%
\pgfsetroundjoin%
\definecolor{currentfill}{rgb}{0.000000,0.000000,0.000000}%
\pgfsetfillcolor{currentfill}%
\pgfsetlinewidth{0.602250pt}%
\definecolor{currentstroke}{rgb}{0.000000,0.000000,0.000000}%
\pgfsetstrokecolor{currentstroke}%
\pgfsetdash{}{0pt}%
\pgfsys@defobject{currentmarker}{\pgfqpoint{0.000000in}{0.000000in}}{\pgfqpoint{0.027778in}{0.000000in}}{%
\pgfpathmoveto{\pgfqpoint{0.000000in}{0.000000in}}%
\pgfpathlineto{\pgfqpoint{0.027778in}{0.000000in}}%
\pgfusepath{stroke,fill}%
}%
\begin{pgfscope}%
\pgfsys@transformshift{5.552773in}{2.811421in}%
\pgfsys@useobject{currentmarker}{}%
\end{pgfscope}%
\end{pgfscope}%
\begin{pgfscope}%
\pgfsetbuttcap%
\pgfsetroundjoin%
\definecolor{currentfill}{rgb}{0.000000,0.000000,0.000000}%
\pgfsetfillcolor{currentfill}%
\pgfsetlinewidth{0.602250pt}%
\definecolor{currentstroke}{rgb}{0.000000,0.000000,0.000000}%
\pgfsetstrokecolor{currentstroke}%
\pgfsetdash{}{0pt}%
\pgfsys@defobject{currentmarker}{\pgfqpoint{0.000000in}{0.000000in}}{\pgfqpoint{0.027778in}{0.000000in}}{%
\pgfpathmoveto{\pgfqpoint{0.000000in}{0.000000in}}%
\pgfpathlineto{\pgfqpoint{0.027778in}{0.000000in}}%
\pgfusepath{stroke,fill}%
}%
\begin{pgfscope}%
\pgfsys@transformshift{5.552773in}{2.878896in}%
\pgfsys@useobject{currentmarker}{}%
\end{pgfscope}%
\end{pgfscope}%
\begin{pgfscope}%
\pgfsetbuttcap%
\pgfsetroundjoin%
\definecolor{currentfill}{rgb}{0.000000,0.000000,0.000000}%
\pgfsetfillcolor{currentfill}%
\pgfsetlinewidth{0.602250pt}%
\definecolor{currentstroke}{rgb}{0.000000,0.000000,0.000000}%
\pgfsetstrokecolor{currentstroke}%
\pgfsetdash{}{0pt}%
\pgfsys@defobject{currentmarker}{\pgfqpoint{0.000000in}{0.000000in}}{\pgfqpoint{0.027778in}{0.000000in}}{%
\pgfpathmoveto{\pgfqpoint{0.000000in}{0.000000in}}%
\pgfpathlineto{\pgfqpoint{0.027778in}{0.000000in}}%
\pgfusepath{stroke,fill}%
}%
\begin{pgfscope}%
\pgfsys@transformshift{5.552773in}{2.926771in}%
\pgfsys@useobject{currentmarker}{}%
\end{pgfscope}%
\end{pgfscope}%
\begin{pgfscope}%
\pgfsetbuttcap%
\pgfsetroundjoin%
\definecolor{currentfill}{rgb}{0.000000,0.000000,0.000000}%
\pgfsetfillcolor{currentfill}%
\pgfsetlinewidth{0.602250pt}%
\definecolor{currentstroke}{rgb}{0.000000,0.000000,0.000000}%
\pgfsetstrokecolor{currentstroke}%
\pgfsetdash{}{0pt}%
\pgfsys@defobject{currentmarker}{\pgfqpoint{0.000000in}{0.000000in}}{\pgfqpoint{0.027778in}{0.000000in}}{%
\pgfpathmoveto{\pgfqpoint{0.000000in}{0.000000in}}%
\pgfpathlineto{\pgfqpoint{0.027778in}{0.000000in}}%
\pgfusepath{stroke,fill}%
}%
\begin{pgfscope}%
\pgfsys@transformshift{5.552773in}{2.963905in}%
\pgfsys@useobject{currentmarker}{}%
\end{pgfscope}%
\end{pgfscope}%
\begin{pgfscope}%
\pgfsetbuttcap%
\pgfsetroundjoin%
\definecolor{currentfill}{rgb}{0.000000,0.000000,0.000000}%
\pgfsetfillcolor{currentfill}%
\pgfsetlinewidth{0.602250pt}%
\definecolor{currentstroke}{rgb}{0.000000,0.000000,0.000000}%
\pgfsetstrokecolor{currentstroke}%
\pgfsetdash{}{0pt}%
\pgfsys@defobject{currentmarker}{\pgfqpoint{0.000000in}{0.000000in}}{\pgfqpoint{0.027778in}{0.000000in}}{%
\pgfpathmoveto{\pgfqpoint{0.000000in}{0.000000in}}%
\pgfpathlineto{\pgfqpoint{0.027778in}{0.000000in}}%
\pgfusepath{stroke,fill}%
}%
\begin{pgfscope}%
\pgfsys@transformshift{5.552773in}{2.994246in}%
\pgfsys@useobject{currentmarker}{}%
\end{pgfscope}%
\end{pgfscope}%
\begin{pgfscope}%
\pgfsetbuttcap%
\pgfsetroundjoin%
\definecolor{currentfill}{rgb}{0.000000,0.000000,0.000000}%
\pgfsetfillcolor{currentfill}%
\pgfsetlinewidth{0.602250pt}%
\definecolor{currentstroke}{rgb}{0.000000,0.000000,0.000000}%
\pgfsetstrokecolor{currentstroke}%
\pgfsetdash{}{0pt}%
\pgfsys@defobject{currentmarker}{\pgfqpoint{0.000000in}{0.000000in}}{\pgfqpoint{0.027778in}{0.000000in}}{%
\pgfpathmoveto{\pgfqpoint{0.000000in}{0.000000in}}%
\pgfpathlineto{\pgfqpoint{0.027778in}{0.000000in}}%
\pgfusepath{stroke,fill}%
}%
\begin{pgfscope}%
\pgfsys@transformshift{5.552773in}{3.019899in}%
\pgfsys@useobject{currentmarker}{}%
\end{pgfscope}%
\end{pgfscope}%
\begin{pgfscope}%
\pgfsetbuttcap%
\pgfsetroundjoin%
\definecolor{currentfill}{rgb}{0.000000,0.000000,0.000000}%
\pgfsetfillcolor{currentfill}%
\pgfsetlinewidth{0.602250pt}%
\definecolor{currentstroke}{rgb}{0.000000,0.000000,0.000000}%
\pgfsetstrokecolor{currentstroke}%
\pgfsetdash{}{0pt}%
\pgfsys@defobject{currentmarker}{\pgfqpoint{0.000000in}{0.000000in}}{\pgfqpoint{0.027778in}{0.000000in}}{%
\pgfpathmoveto{\pgfqpoint{0.000000in}{0.000000in}}%
\pgfpathlineto{\pgfqpoint{0.027778in}{0.000000in}}%
\pgfusepath{stroke,fill}%
}%
\begin{pgfscope}%
\pgfsys@transformshift{5.552773in}{3.042120in}%
\pgfsys@useobject{currentmarker}{}%
\end{pgfscope}%
\end{pgfscope}%
\begin{pgfscope}%
\pgfsetbuttcap%
\pgfsetroundjoin%
\definecolor{currentfill}{rgb}{0.000000,0.000000,0.000000}%
\pgfsetfillcolor{currentfill}%
\pgfsetlinewidth{0.602250pt}%
\definecolor{currentstroke}{rgb}{0.000000,0.000000,0.000000}%
\pgfsetstrokecolor{currentstroke}%
\pgfsetdash{}{0pt}%
\pgfsys@defobject{currentmarker}{\pgfqpoint{0.000000in}{0.000000in}}{\pgfqpoint{0.027778in}{0.000000in}}{%
\pgfpathmoveto{\pgfqpoint{0.000000in}{0.000000in}}%
\pgfpathlineto{\pgfqpoint{0.027778in}{0.000000in}}%
\pgfusepath{stroke,fill}%
}%
\begin{pgfscope}%
\pgfsys@transformshift{5.552773in}{3.061721in}%
\pgfsys@useobject{currentmarker}{}%
\end{pgfscope}%
\end{pgfscope}%
\begin{pgfscope}%
\pgfsetbuttcap%
\pgfsetroundjoin%
\definecolor{currentfill}{rgb}{0.000000,0.000000,0.000000}%
\pgfsetfillcolor{currentfill}%
\pgfsetlinewidth{0.602250pt}%
\definecolor{currentstroke}{rgb}{0.000000,0.000000,0.000000}%
\pgfsetstrokecolor{currentstroke}%
\pgfsetdash{}{0pt}%
\pgfsys@defobject{currentmarker}{\pgfqpoint{0.000000in}{0.000000in}}{\pgfqpoint{0.027778in}{0.000000in}}{%
\pgfpathmoveto{\pgfqpoint{0.000000in}{0.000000in}}%
\pgfpathlineto{\pgfqpoint{0.027778in}{0.000000in}}%
\pgfusepath{stroke,fill}%
}%
\begin{pgfscope}%
\pgfsys@transformshift{5.552773in}{3.194604in}%
\pgfsys@useobject{currentmarker}{}%
\end{pgfscope}%
\end{pgfscope}%
\begin{pgfscope}%
\pgfsetbuttcap%
\pgfsetroundjoin%
\definecolor{currentfill}{rgb}{0.000000,0.000000,0.000000}%
\pgfsetfillcolor{currentfill}%
\pgfsetlinewidth{0.602250pt}%
\definecolor{currentstroke}{rgb}{0.000000,0.000000,0.000000}%
\pgfsetstrokecolor{currentstroke}%
\pgfsetdash{}{0pt}%
\pgfsys@defobject{currentmarker}{\pgfqpoint{0.000000in}{0.000000in}}{\pgfqpoint{0.027778in}{0.000000in}}{%
\pgfpathmoveto{\pgfqpoint{0.000000in}{0.000000in}}%
\pgfpathlineto{\pgfqpoint{0.027778in}{0.000000in}}%
\pgfusepath{stroke,fill}%
}%
\begin{pgfscope}%
\pgfsys@transformshift{5.552773in}{3.262079in}%
\pgfsys@useobject{currentmarker}{}%
\end{pgfscope}%
\end{pgfscope}%
\begin{pgfscope}%
\pgfsetbuttcap%
\pgfsetroundjoin%
\definecolor{currentfill}{rgb}{0.000000,0.000000,0.000000}%
\pgfsetfillcolor{currentfill}%
\pgfsetlinewidth{0.602250pt}%
\definecolor{currentstroke}{rgb}{0.000000,0.000000,0.000000}%
\pgfsetstrokecolor{currentstroke}%
\pgfsetdash{}{0pt}%
\pgfsys@defobject{currentmarker}{\pgfqpoint{0.000000in}{0.000000in}}{\pgfqpoint{0.027778in}{0.000000in}}{%
\pgfpathmoveto{\pgfqpoint{0.000000in}{0.000000in}}%
\pgfpathlineto{\pgfqpoint{0.027778in}{0.000000in}}%
\pgfusepath{stroke,fill}%
}%
\begin{pgfscope}%
\pgfsys@transformshift{5.552773in}{3.309954in}%
\pgfsys@useobject{currentmarker}{}%
\end{pgfscope}%
\end{pgfscope}%
\begin{pgfscope}%
\pgfsetbuttcap%
\pgfsetroundjoin%
\definecolor{currentfill}{rgb}{0.000000,0.000000,0.000000}%
\pgfsetfillcolor{currentfill}%
\pgfsetlinewidth{0.602250pt}%
\definecolor{currentstroke}{rgb}{0.000000,0.000000,0.000000}%
\pgfsetstrokecolor{currentstroke}%
\pgfsetdash{}{0pt}%
\pgfsys@defobject{currentmarker}{\pgfqpoint{0.000000in}{0.000000in}}{\pgfqpoint{0.027778in}{0.000000in}}{%
\pgfpathmoveto{\pgfqpoint{0.000000in}{0.000000in}}%
\pgfpathlineto{\pgfqpoint{0.027778in}{0.000000in}}%
\pgfusepath{stroke,fill}%
}%
\begin{pgfscope}%
\pgfsys@transformshift{5.552773in}{3.347088in}%
\pgfsys@useobject{currentmarker}{}%
\end{pgfscope}%
\end{pgfscope}%
\begin{pgfscope}%
\pgfsetbuttcap%
\pgfsetroundjoin%
\definecolor{currentfill}{rgb}{0.000000,0.000000,0.000000}%
\pgfsetfillcolor{currentfill}%
\pgfsetlinewidth{0.602250pt}%
\definecolor{currentstroke}{rgb}{0.000000,0.000000,0.000000}%
\pgfsetstrokecolor{currentstroke}%
\pgfsetdash{}{0pt}%
\pgfsys@defobject{currentmarker}{\pgfqpoint{0.000000in}{0.000000in}}{\pgfqpoint{0.027778in}{0.000000in}}{%
\pgfpathmoveto{\pgfqpoint{0.000000in}{0.000000in}}%
\pgfpathlineto{\pgfqpoint{0.027778in}{0.000000in}}%
\pgfusepath{stroke,fill}%
}%
\begin{pgfscope}%
\pgfsys@transformshift{5.552773in}{3.377429in}%
\pgfsys@useobject{currentmarker}{}%
\end{pgfscope}%
\end{pgfscope}%
\begin{pgfscope}%
\pgfsetbuttcap%
\pgfsetroundjoin%
\definecolor{currentfill}{rgb}{0.000000,0.000000,0.000000}%
\pgfsetfillcolor{currentfill}%
\pgfsetlinewidth{0.602250pt}%
\definecolor{currentstroke}{rgb}{0.000000,0.000000,0.000000}%
\pgfsetstrokecolor{currentstroke}%
\pgfsetdash{}{0pt}%
\pgfsys@defobject{currentmarker}{\pgfqpoint{0.000000in}{0.000000in}}{\pgfqpoint{0.027778in}{0.000000in}}{%
\pgfpathmoveto{\pgfqpoint{0.000000in}{0.000000in}}%
\pgfpathlineto{\pgfqpoint{0.027778in}{0.000000in}}%
\pgfusepath{stroke,fill}%
}%
\begin{pgfscope}%
\pgfsys@transformshift{5.552773in}{3.403082in}%
\pgfsys@useobject{currentmarker}{}%
\end{pgfscope}%
\end{pgfscope}%
\begin{pgfscope}%
\pgfsetbuttcap%
\pgfsetroundjoin%
\definecolor{currentfill}{rgb}{0.000000,0.000000,0.000000}%
\pgfsetfillcolor{currentfill}%
\pgfsetlinewidth{0.602250pt}%
\definecolor{currentstroke}{rgb}{0.000000,0.000000,0.000000}%
\pgfsetstrokecolor{currentstroke}%
\pgfsetdash{}{0pt}%
\pgfsys@defobject{currentmarker}{\pgfqpoint{0.000000in}{0.000000in}}{\pgfqpoint{0.027778in}{0.000000in}}{%
\pgfpathmoveto{\pgfqpoint{0.000000in}{0.000000in}}%
\pgfpathlineto{\pgfqpoint{0.027778in}{0.000000in}}%
\pgfusepath{stroke,fill}%
}%
\begin{pgfscope}%
\pgfsys@transformshift{5.552773in}{3.425304in}%
\pgfsys@useobject{currentmarker}{}%
\end{pgfscope}%
\end{pgfscope}%
\begin{pgfscope}%
\pgfsetbuttcap%
\pgfsetroundjoin%
\definecolor{currentfill}{rgb}{0.000000,0.000000,0.000000}%
\pgfsetfillcolor{currentfill}%
\pgfsetlinewidth{0.602250pt}%
\definecolor{currentstroke}{rgb}{0.000000,0.000000,0.000000}%
\pgfsetstrokecolor{currentstroke}%
\pgfsetdash{}{0pt}%
\pgfsys@defobject{currentmarker}{\pgfqpoint{0.000000in}{0.000000in}}{\pgfqpoint{0.027778in}{0.000000in}}{%
\pgfpathmoveto{\pgfqpoint{0.000000in}{0.000000in}}%
\pgfpathlineto{\pgfqpoint{0.027778in}{0.000000in}}%
\pgfusepath{stroke,fill}%
}%
\begin{pgfscope}%
\pgfsys@transformshift{5.552773in}{3.444904in}%
\pgfsys@useobject{currentmarker}{}%
\end{pgfscope}%
\end{pgfscope}%
\begin{pgfscope}%
\pgfsetbuttcap%
\pgfsetroundjoin%
\definecolor{currentfill}{rgb}{0.000000,0.000000,0.000000}%
\pgfsetfillcolor{currentfill}%
\pgfsetlinewidth{0.602250pt}%
\definecolor{currentstroke}{rgb}{0.000000,0.000000,0.000000}%
\pgfsetstrokecolor{currentstroke}%
\pgfsetdash{}{0pt}%
\pgfsys@defobject{currentmarker}{\pgfqpoint{0.000000in}{0.000000in}}{\pgfqpoint{0.027778in}{0.000000in}}{%
\pgfpathmoveto{\pgfqpoint{0.000000in}{0.000000in}}%
\pgfpathlineto{\pgfqpoint{0.027778in}{0.000000in}}%
\pgfusepath{stroke,fill}%
}%
\begin{pgfscope}%
\pgfsys@transformshift{5.552773in}{3.577787in}%
\pgfsys@useobject{currentmarker}{}%
\end{pgfscope}%
\end{pgfscope}%
\begin{pgfscope}%
\pgfsetbuttcap%
\pgfsetroundjoin%
\definecolor{currentfill}{rgb}{0.000000,0.000000,0.000000}%
\pgfsetfillcolor{currentfill}%
\pgfsetlinewidth{0.602250pt}%
\definecolor{currentstroke}{rgb}{0.000000,0.000000,0.000000}%
\pgfsetstrokecolor{currentstroke}%
\pgfsetdash{}{0pt}%
\pgfsys@defobject{currentmarker}{\pgfqpoint{0.000000in}{0.000000in}}{\pgfqpoint{0.027778in}{0.000000in}}{%
\pgfpathmoveto{\pgfqpoint{0.000000in}{0.000000in}}%
\pgfpathlineto{\pgfqpoint{0.027778in}{0.000000in}}%
\pgfusepath{stroke,fill}%
}%
\begin{pgfscope}%
\pgfsys@transformshift{5.552773in}{3.645263in}%
\pgfsys@useobject{currentmarker}{}%
\end{pgfscope}%
\end{pgfscope}%
\begin{pgfscope}%
\pgfsetbuttcap%
\pgfsetroundjoin%
\definecolor{currentfill}{rgb}{0.000000,0.000000,0.000000}%
\pgfsetfillcolor{currentfill}%
\pgfsetlinewidth{0.602250pt}%
\definecolor{currentstroke}{rgb}{0.000000,0.000000,0.000000}%
\pgfsetstrokecolor{currentstroke}%
\pgfsetdash{}{0pt}%
\pgfsys@defobject{currentmarker}{\pgfqpoint{0.000000in}{0.000000in}}{\pgfqpoint{0.027778in}{0.000000in}}{%
\pgfpathmoveto{\pgfqpoint{0.000000in}{0.000000in}}%
\pgfpathlineto{\pgfqpoint{0.027778in}{0.000000in}}%
\pgfusepath{stroke,fill}%
}%
\begin{pgfscope}%
\pgfsys@transformshift{5.552773in}{3.693137in}%
\pgfsys@useobject{currentmarker}{}%
\end{pgfscope}%
\end{pgfscope}%
\begin{pgfscope}%
\pgfsetbuttcap%
\pgfsetroundjoin%
\definecolor{currentfill}{rgb}{0.000000,0.000000,0.000000}%
\pgfsetfillcolor{currentfill}%
\pgfsetlinewidth{0.602250pt}%
\definecolor{currentstroke}{rgb}{0.000000,0.000000,0.000000}%
\pgfsetstrokecolor{currentstroke}%
\pgfsetdash{}{0pt}%
\pgfsys@defobject{currentmarker}{\pgfqpoint{0.000000in}{0.000000in}}{\pgfqpoint{0.027778in}{0.000000in}}{%
\pgfpathmoveto{\pgfqpoint{0.000000in}{0.000000in}}%
\pgfpathlineto{\pgfqpoint{0.027778in}{0.000000in}}%
\pgfusepath{stroke,fill}%
}%
\begin{pgfscope}%
\pgfsys@transformshift{5.552773in}{3.730271in}%
\pgfsys@useobject{currentmarker}{}%
\end{pgfscope}%
\end{pgfscope}%
\begin{pgfscope}%
\pgfsetbuttcap%
\pgfsetroundjoin%
\definecolor{currentfill}{rgb}{0.000000,0.000000,0.000000}%
\pgfsetfillcolor{currentfill}%
\pgfsetlinewidth{0.602250pt}%
\definecolor{currentstroke}{rgb}{0.000000,0.000000,0.000000}%
\pgfsetstrokecolor{currentstroke}%
\pgfsetdash{}{0pt}%
\pgfsys@defobject{currentmarker}{\pgfqpoint{0.000000in}{0.000000in}}{\pgfqpoint{0.027778in}{0.000000in}}{%
\pgfpathmoveto{\pgfqpoint{0.000000in}{0.000000in}}%
\pgfpathlineto{\pgfqpoint{0.027778in}{0.000000in}}%
\pgfusepath{stroke,fill}%
}%
\begin{pgfscope}%
\pgfsys@transformshift{5.552773in}{3.760612in}%
\pgfsys@useobject{currentmarker}{}%
\end{pgfscope}%
\end{pgfscope}%
\begin{pgfscope}%
\pgfsetbuttcap%
\pgfsetroundjoin%
\definecolor{currentfill}{rgb}{0.000000,0.000000,0.000000}%
\pgfsetfillcolor{currentfill}%
\pgfsetlinewidth{0.602250pt}%
\definecolor{currentstroke}{rgb}{0.000000,0.000000,0.000000}%
\pgfsetstrokecolor{currentstroke}%
\pgfsetdash{}{0pt}%
\pgfsys@defobject{currentmarker}{\pgfqpoint{0.000000in}{0.000000in}}{\pgfqpoint{0.027778in}{0.000000in}}{%
\pgfpathmoveto{\pgfqpoint{0.000000in}{0.000000in}}%
\pgfpathlineto{\pgfqpoint{0.027778in}{0.000000in}}%
\pgfusepath{stroke,fill}%
}%
\begin{pgfscope}%
\pgfsys@transformshift{5.552773in}{3.786265in}%
\pgfsys@useobject{currentmarker}{}%
\end{pgfscope}%
\end{pgfscope}%
\begin{pgfscope}%
\pgfsetbuttcap%
\pgfsetroundjoin%
\definecolor{currentfill}{rgb}{0.000000,0.000000,0.000000}%
\pgfsetfillcolor{currentfill}%
\pgfsetlinewidth{0.602250pt}%
\definecolor{currentstroke}{rgb}{0.000000,0.000000,0.000000}%
\pgfsetstrokecolor{currentstroke}%
\pgfsetdash{}{0pt}%
\pgfsys@defobject{currentmarker}{\pgfqpoint{0.000000in}{0.000000in}}{\pgfqpoint{0.027778in}{0.000000in}}{%
\pgfpathmoveto{\pgfqpoint{0.000000in}{0.000000in}}%
\pgfpathlineto{\pgfqpoint{0.027778in}{0.000000in}}%
\pgfusepath{stroke,fill}%
}%
\begin{pgfscope}%
\pgfsys@transformshift{5.552773in}{3.808487in}%
\pgfsys@useobject{currentmarker}{}%
\end{pgfscope}%
\end{pgfscope}%
\begin{pgfscope}%
\pgfsetbuttcap%
\pgfsetroundjoin%
\definecolor{currentfill}{rgb}{0.000000,0.000000,0.000000}%
\pgfsetfillcolor{currentfill}%
\pgfsetlinewidth{0.602250pt}%
\definecolor{currentstroke}{rgb}{0.000000,0.000000,0.000000}%
\pgfsetstrokecolor{currentstroke}%
\pgfsetdash{}{0pt}%
\pgfsys@defobject{currentmarker}{\pgfqpoint{0.000000in}{0.000000in}}{\pgfqpoint{0.027778in}{0.000000in}}{%
\pgfpathmoveto{\pgfqpoint{0.000000in}{0.000000in}}%
\pgfpathlineto{\pgfqpoint{0.027778in}{0.000000in}}%
\pgfusepath{stroke,fill}%
}%
\begin{pgfscope}%
\pgfsys@transformshift{5.552773in}{3.828088in}%
\pgfsys@useobject{currentmarker}{}%
\end{pgfscope}%
\end{pgfscope}%
\begin{pgfscope}%
\pgfsetbuttcap%
\pgfsetroundjoin%
\definecolor{currentfill}{rgb}{0.000000,0.000000,0.000000}%
\pgfsetfillcolor{currentfill}%
\pgfsetlinewidth{0.602250pt}%
\definecolor{currentstroke}{rgb}{0.000000,0.000000,0.000000}%
\pgfsetstrokecolor{currentstroke}%
\pgfsetdash{}{0pt}%
\pgfsys@defobject{currentmarker}{\pgfqpoint{0.000000in}{0.000000in}}{\pgfqpoint{0.027778in}{0.000000in}}{%
\pgfpathmoveto{\pgfqpoint{0.000000in}{0.000000in}}%
\pgfpathlineto{\pgfqpoint{0.027778in}{0.000000in}}%
\pgfusepath{stroke,fill}%
}%
\begin{pgfscope}%
\pgfsys@transformshift{5.552773in}{3.960971in}%
\pgfsys@useobject{currentmarker}{}%
\end{pgfscope}%
\end{pgfscope}%
\begin{pgfscope}%
\pgfsetbuttcap%
\pgfsetroundjoin%
\definecolor{currentfill}{rgb}{0.000000,0.000000,0.000000}%
\pgfsetfillcolor{currentfill}%
\pgfsetlinewidth{0.602250pt}%
\definecolor{currentstroke}{rgb}{0.000000,0.000000,0.000000}%
\pgfsetstrokecolor{currentstroke}%
\pgfsetdash{}{0pt}%
\pgfsys@defobject{currentmarker}{\pgfqpoint{0.000000in}{0.000000in}}{\pgfqpoint{0.027778in}{0.000000in}}{%
\pgfpathmoveto{\pgfqpoint{0.000000in}{0.000000in}}%
\pgfpathlineto{\pgfqpoint{0.027778in}{0.000000in}}%
\pgfusepath{stroke,fill}%
}%
\begin{pgfscope}%
\pgfsys@transformshift{5.552773in}{4.028446in}%
\pgfsys@useobject{currentmarker}{}%
\end{pgfscope}%
\end{pgfscope}%
\begin{pgfscope}%
\pgfsetbuttcap%
\pgfsetroundjoin%
\definecolor{currentfill}{rgb}{0.000000,0.000000,0.000000}%
\pgfsetfillcolor{currentfill}%
\pgfsetlinewidth{0.602250pt}%
\definecolor{currentstroke}{rgb}{0.000000,0.000000,0.000000}%
\pgfsetstrokecolor{currentstroke}%
\pgfsetdash{}{0pt}%
\pgfsys@defobject{currentmarker}{\pgfqpoint{0.000000in}{0.000000in}}{\pgfqpoint{0.027778in}{0.000000in}}{%
\pgfpathmoveto{\pgfqpoint{0.000000in}{0.000000in}}%
\pgfpathlineto{\pgfqpoint{0.027778in}{0.000000in}}%
\pgfusepath{stroke,fill}%
}%
\begin{pgfscope}%
\pgfsys@transformshift{5.552773in}{4.076320in}%
\pgfsys@useobject{currentmarker}{}%
\end{pgfscope}%
\end{pgfscope}%
\begin{pgfscope}%
\pgfsetbuttcap%
\pgfsetroundjoin%
\definecolor{currentfill}{rgb}{0.000000,0.000000,0.000000}%
\pgfsetfillcolor{currentfill}%
\pgfsetlinewidth{0.602250pt}%
\definecolor{currentstroke}{rgb}{0.000000,0.000000,0.000000}%
\pgfsetstrokecolor{currentstroke}%
\pgfsetdash{}{0pt}%
\pgfsys@defobject{currentmarker}{\pgfqpoint{0.000000in}{0.000000in}}{\pgfqpoint{0.027778in}{0.000000in}}{%
\pgfpathmoveto{\pgfqpoint{0.000000in}{0.000000in}}%
\pgfpathlineto{\pgfqpoint{0.027778in}{0.000000in}}%
\pgfusepath{stroke,fill}%
}%
\begin{pgfscope}%
\pgfsys@transformshift{5.552773in}{4.113455in}%
\pgfsys@useobject{currentmarker}{}%
\end{pgfscope}%
\end{pgfscope}%
\begin{pgfscope}%
\pgfsetbuttcap%
\pgfsetroundjoin%
\definecolor{currentfill}{rgb}{0.000000,0.000000,0.000000}%
\pgfsetfillcolor{currentfill}%
\pgfsetlinewidth{0.602250pt}%
\definecolor{currentstroke}{rgb}{0.000000,0.000000,0.000000}%
\pgfsetstrokecolor{currentstroke}%
\pgfsetdash{}{0pt}%
\pgfsys@defobject{currentmarker}{\pgfqpoint{0.000000in}{0.000000in}}{\pgfqpoint{0.027778in}{0.000000in}}{%
\pgfpathmoveto{\pgfqpoint{0.000000in}{0.000000in}}%
\pgfpathlineto{\pgfqpoint{0.027778in}{0.000000in}}%
\pgfusepath{stroke,fill}%
}%
\begin{pgfscope}%
\pgfsys@transformshift{5.552773in}{4.143796in}%
\pgfsys@useobject{currentmarker}{}%
\end{pgfscope}%
\end{pgfscope}%
\begin{pgfscope}%
\pgfsetbuttcap%
\pgfsetroundjoin%
\definecolor{currentfill}{rgb}{0.000000,0.000000,0.000000}%
\pgfsetfillcolor{currentfill}%
\pgfsetlinewidth{0.602250pt}%
\definecolor{currentstroke}{rgb}{0.000000,0.000000,0.000000}%
\pgfsetstrokecolor{currentstroke}%
\pgfsetdash{}{0pt}%
\pgfsys@defobject{currentmarker}{\pgfqpoint{0.000000in}{0.000000in}}{\pgfqpoint{0.027778in}{0.000000in}}{%
\pgfpathmoveto{\pgfqpoint{0.000000in}{0.000000in}}%
\pgfpathlineto{\pgfqpoint{0.027778in}{0.000000in}}%
\pgfusepath{stroke,fill}%
}%
\begin{pgfscope}%
\pgfsys@transformshift{5.552773in}{4.169448in}%
\pgfsys@useobject{currentmarker}{}%
\end{pgfscope}%
\end{pgfscope}%
\begin{pgfscope}%
\pgfsetbuttcap%
\pgfsetroundjoin%
\definecolor{currentfill}{rgb}{0.000000,0.000000,0.000000}%
\pgfsetfillcolor{currentfill}%
\pgfsetlinewidth{0.602250pt}%
\definecolor{currentstroke}{rgb}{0.000000,0.000000,0.000000}%
\pgfsetstrokecolor{currentstroke}%
\pgfsetdash{}{0pt}%
\pgfsys@defobject{currentmarker}{\pgfqpoint{0.000000in}{0.000000in}}{\pgfqpoint{0.027778in}{0.000000in}}{%
\pgfpathmoveto{\pgfqpoint{0.000000in}{0.000000in}}%
\pgfpathlineto{\pgfqpoint{0.027778in}{0.000000in}}%
\pgfusepath{stroke,fill}%
}%
\begin{pgfscope}%
\pgfsys@transformshift{5.552773in}{4.191670in}%
\pgfsys@useobject{currentmarker}{}%
\end{pgfscope}%
\end{pgfscope}%
\begin{pgfscope}%
\pgfsetbuttcap%
\pgfsetroundjoin%
\definecolor{currentfill}{rgb}{0.000000,0.000000,0.000000}%
\pgfsetfillcolor{currentfill}%
\pgfsetlinewidth{0.602250pt}%
\definecolor{currentstroke}{rgb}{0.000000,0.000000,0.000000}%
\pgfsetstrokecolor{currentstroke}%
\pgfsetdash{}{0pt}%
\pgfsys@defobject{currentmarker}{\pgfqpoint{0.000000in}{0.000000in}}{\pgfqpoint{0.027778in}{0.000000in}}{%
\pgfpathmoveto{\pgfqpoint{0.000000in}{0.000000in}}%
\pgfpathlineto{\pgfqpoint{0.027778in}{0.000000in}}%
\pgfusepath{stroke,fill}%
}%
\begin{pgfscope}%
\pgfsys@transformshift{5.552773in}{4.211271in}%
\pgfsys@useobject{currentmarker}{}%
\end{pgfscope}%
\end{pgfscope}%
\begin{pgfscope}%
\definecolor{textcolor}{rgb}{0.000000,0.000000,0.000000}%
\pgfsetstrokecolor{textcolor}%
\pgfsetfillcolor{textcolor}%
\pgftext[x=5.993553in,y=3.399862in,,top,rotate=90.000000]{\color{textcolor}\rmfamily\fontsize{10.000000}{12.000000}\selectfont \begin{tabular}{c}fehldetektierte Photonen\\pro Pixel pro Aufnahme\end{tabular}}%
\end{pgfscope}%
\begin{pgfscope}%
\pgfsetrectcap%
\pgfsetmiterjoin%
\pgfsetlinewidth{0.803000pt}%
\definecolor{currentstroke}{rgb}{0.000000,0.000000,0.000000}%
\pgfsetstrokecolor{currentstroke}%
\pgfsetdash{}{0pt}%
\pgfpathmoveto{\pgfqpoint{3.073825in}{2.498808in}}%
\pgfpathlineto{\pgfqpoint{3.073825in}{4.300916in}}%
\pgfusepath{stroke}%
\end{pgfscope}%
\begin{pgfscope}%
\pgfsetrectcap%
\pgfsetmiterjoin%
\pgfsetlinewidth{0.803000pt}%
\definecolor{currentstroke}{rgb}{0.000000,0.000000,0.000000}%
\pgfsetstrokecolor{currentstroke}%
\pgfsetdash{}{0pt}%
\pgfpathmoveto{\pgfqpoint{5.552773in}{2.498808in}}%
\pgfpathlineto{\pgfqpoint{5.552773in}{4.300916in}}%
\pgfusepath{stroke}%
\end{pgfscope}%
\begin{pgfscope}%
\pgfsetrectcap%
\pgfsetmiterjoin%
\pgfsetlinewidth{0.803000pt}%
\definecolor{currentstroke}{rgb}{0.000000,0.000000,0.000000}%
\pgfsetstrokecolor{currentstroke}%
\pgfsetdash{}{0pt}%
\pgfpathmoveto{\pgfqpoint{3.073825in}{2.498808in}}%
\pgfpathlineto{\pgfqpoint{5.552773in}{2.498808in}}%
\pgfusepath{stroke}%
\end{pgfscope}%
\begin{pgfscope}%
\pgfsetrectcap%
\pgfsetmiterjoin%
\pgfsetlinewidth{0.803000pt}%
\definecolor{currentstroke}{rgb}{0.000000,0.000000,0.000000}%
\pgfsetstrokecolor{currentstroke}%
\pgfsetdash{}{0pt}%
\pgfpathmoveto{\pgfqpoint{3.073825in}{4.300916in}}%
\pgfpathlineto{\pgfqpoint{5.552773in}{4.300916in}}%
\pgfusepath{stroke}%
\end{pgfscope}%
\begin{pgfscope}%
\pgfpathrectangle{\pgfqpoint{3.073825in}{2.498808in}}{\pgfqpoint{2.478947in}{1.802109in}}%
\pgfusepath{clip}%
\pgfsetrectcap%
\pgfsetroundjoin%
\pgfsetlinewidth{1.505625pt}%
\definecolor{currentstroke}{rgb}{0.121569,0.466667,0.705882}%
\pgfsetstrokecolor{currentstroke}%
\pgfsetdash{}{0pt}%
\pgfpathmoveto{\pgfqpoint{3.175421in}{4.219002in}}%
\pgfpathlineto{\pgfqpoint{3.277018in}{4.127098in}}%
\pgfpathlineto{\pgfqpoint{3.378614in}{4.034391in}}%
\pgfpathlineto{\pgfqpoint{3.480210in}{3.942435in}}%
\pgfpathlineto{\pgfqpoint{3.581806in}{3.849987in}}%
\pgfpathlineto{\pgfqpoint{3.683402in}{3.758074in}}%
\pgfpathlineto{\pgfqpoint{3.784999in}{3.664562in}}%
\pgfpathlineto{\pgfqpoint{3.886595in}{3.569227in}}%
\pgfpathlineto{\pgfqpoint{3.988191in}{3.473956in}}%
\pgfpathlineto{\pgfqpoint{4.089787in}{3.379268in}}%
\pgfpathlineto{\pgfqpoint{4.191383in}{3.282908in}}%
\pgfpathlineto{\pgfqpoint{4.292980in}{3.176299in}}%
\pgfpathlineto{\pgfqpoint{4.394576in}{3.086933in}}%
\pgfpathlineto{\pgfqpoint{4.496172in}{2.995015in}}%
\pgfpathlineto{\pgfqpoint{4.597768in}{2.914793in}}%
\pgfpathlineto{\pgfqpoint{4.699364in}{2.842914in}}%
\pgfpathlineto{\pgfqpoint{4.800961in}{2.799444in}}%
\pgfpathlineto{\pgfqpoint{4.902557in}{2.736863in}}%
\pgfpathlineto{\pgfqpoint{5.004153in}{2.691383in}}%
\pgfpathlineto{\pgfqpoint{5.105749in}{2.660089in}}%
\pgfpathlineto{\pgfqpoint{5.207346in}{2.628596in}}%
\pgfpathlineto{\pgfqpoint{5.308942in}{2.614116in}}%
\pgfpathlineto{\pgfqpoint{5.410538in}{2.580722in}}%
\pgfusepath{stroke}%
\end{pgfscope}%
\begin{pgfscope}%
\pgfsetbuttcap%
\pgfsetmiterjoin%
\definecolor{currentfill}{rgb}{1.000000,1.000000,1.000000}%
\pgfsetfillcolor{currentfill}%
\pgfsetlinewidth{0.000000pt}%
\definecolor{currentstroke}{rgb}{0.000000,0.000000,0.000000}%
\pgfsetstrokecolor{currentstroke}%
\pgfsetstrokeopacity{0.000000}%
\pgfsetdash{}{0pt}%
\pgfpathmoveto{\pgfqpoint{0.444878in}{0.498088in}}%
\pgfpathlineto{\pgfqpoint{2.923825in}{0.498088in}}%
\pgfpathlineto{\pgfqpoint{2.923825in}{2.300197in}}%
\pgfpathlineto{\pgfqpoint{0.444878in}{2.300197in}}%
\pgfpathlineto{\pgfqpoint{0.444878in}{0.498088in}}%
\pgfpathclose%
\pgfusepath{fill}%
\end{pgfscope}%
\begin{pgfscope}%
\pgfpathrectangle{\pgfqpoint{0.444878in}{0.498088in}}{\pgfqpoint{2.478947in}{1.802109in}}%
\pgfusepath{clip}%
\pgfsetbuttcap%
\pgfsetroundjoin%
\pgfsetlinewidth{1.003750pt}%
\definecolor{currentstroke}{rgb}{0.000000,0.000000,0.000000}%
\pgfsetstrokecolor{currentstroke}%
\pgfsetdash{}{0pt}%
\pgfpathmoveto{\pgfqpoint{0.442878in}{0.925215in}}%
\pgfpathlineto{\pgfqpoint{1.562436in}{0.925215in}}%
\pgfusepath{stroke}%
\end{pgfscope}%
\begin{pgfscope}%
\pgfpathrectangle{\pgfqpoint{0.444878in}{0.498088in}}{\pgfqpoint{2.478947in}{1.802109in}}%
\pgfusepath{clip}%
\pgfsetbuttcap%
\pgfsetroundjoin%
\pgfsetlinewidth{1.003750pt}%
\definecolor{currentstroke}{rgb}{0.000000,0.000000,0.000000}%
\pgfsetstrokecolor{currentstroke}%
\pgfsetdash{}{0pt}%
\pgfpathmoveto{\pgfqpoint{0.442878in}{0.726842in}}%
\pgfpathlineto{\pgfqpoint{2.070417in}{0.726842in}}%
\pgfusepath{stroke}%
\end{pgfscope}%
\begin{pgfscope}%
\pgfpathrectangle{\pgfqpoint{0.444878in}{0.498088in}}{\pgfqpoint{2.478947in}{1.802109in}}%
\pgfusepath{clip}%
\pgfsetbuttcap%
\pgfsetroundjoin%
\pgfsetlinewidth{1.003750pt}%
\definecolor{currentstroke}{rgb}{0.000000,0.000000,0.000000}%
\pgfsetstrokecolor{currentstroke}%
\pgfsetdash{}{0pt}%
\pgfpathmoveto{\pgfqpoint{1.562436in}{0.496088in}}%
\pgfpathlineto{\pgfqpoint{1.562436in}{0.925215in}}%
\pgfusepath{stroke}%
\end{pgfscope}%
\begin{pgfscope}%
\pgfpathrectangle{\pgfqpoint{0.444878in}{0.498088in}}{\pgfqpoint{2.478947in}{1.802109in}}%
\pgfusepath{clip}%
\pgfsetbuttcap%
\pgfsetroundjoin%
\pgfsetlinewidth{1.003750pt}%
\definecolor{currentstroke}{rgb}{0.000000,0.000000,0.000000}%
\pgfsetstrokecolor{currentstroke}%
\pgfsetdash{}{0pt}%
\pgfpathmoveto{\pgfqpoint{2.070417in}{0.496088in}}%
\pgfpathlineto{\pgfqpoint{2.070417in}{0.726842in}}%
\pgfusepath{stroke}%
\end{pgfscope}%
\begin{pgfscope}%
\pgfsetbuttcap%
\pgfsetroundjoin%
\definecolor{currentfill}{rgb}{0.000000,0.000000,0.000000}%
\pgfsetfillcolor{currentfill}%
\pgfsetlinewidth{0.803000pt}%
\definecolor{currentstroke}{rgb}{0.000000,0.000000,0.000000}%
\pgfsetstrokecolor{currentstroke}%
\pgfsetdash{}{0pt}%
\pgfsys@defobject{currentmarker}{\pgfqpoint{0.000000in}{-0.048611in}}{\pgfqpoint{0.000000in}{0.000000in}}{%
\pgfpathmoveto{\pgfqpoint{0.000000in}{0.000000in}}%
\pgfpathlineto{\pgfqpoint{0.000000in}{-0.048611in}}%
\pgfusepath{stroke,fill}%
}%
\begin{pgfscope}%
\pgfsys@transformshift{0.546474in}{0.498088in}%
\pgfsys@useobject{currentmarker}{}%
\end{pgfscope}%
\end{pgfscope}%
\begin{pgfscope}%
\definecolor{textcolor}{rgb}{0.000000,0.000000,0.000000}%
\pgfsetstrokecolor{textcolor}%
\pgfsetfillcolor{textcolor}%
\pgftext[x=0.546474in,y=0.400866in,,top]{\color{textcolor}\rmfamily\fontsize{10.000000}{12.000000}\selectfont \(\displaystyle {50}\)}%
\end{pgfscope}%
\begin{pgfscope}%
\pgfsetbuttcap%
\pgfsetroundjoin%
\definecolor{currentfill}{rgb}{0.000000,0.000000,0.000000}%
\pgfsetfillcolor{currentfill}%
\pgfsetlinewidth{0.803000pt}%
\definecolor{currentstroke}{rgb}{0.000000,0.000000,0.000000}%
\pgfsetstrokecolor{currentstroke}%
\pgfsetdash{}{0pt}%
\pgfsys@defobject{currentmarker}{\pgfqpoint{0.000000in}{-0.048611in}}{\pgfqpoint{0.000000in}{0.000000in}}{%
\pgfpathmoveto{\pgfqpoint{0.000000in}{0.000000in}}%
\pgfpathlineto{\pgfqpoint{0.000000in}{-0.048611in}}%
\pgfusepath{stroke,fill}%
}%
\begin{pgfscope}%
\pgfsys@transformshift{1.054455in}{0.498088in}%
\pgfsys@useobject{currentmarker}{}%
\end{pgfscope}%
\end{pgfscope}%
\begin{pgfscope}%
\definecolor{textcolor}{rgb}{0.000000,0.000000,0.000000}%
\pgfsetstrokecolor{textcolor}%
\pgfsetfillcolor{textcolor}%
\pgftext[x=1.054455in,y=0.400866in,,top]{\color{textcolor}\rmfamily\fontsize{10.000000}{12.000000}\selectfont \(\displaystyle {75}\)}%
\end{pgfscope}%
\begin{pgfscope}%
\pgfsetbuttcap%
\pgfsetroundjoin%
\definecolor{currentfill}{rgb}{0.000000,0.000000,0.000000}%
\pgfsetfillcolor{currentfill}%
\pgfsetlinewidth{0.803000pt}%
\definecolor{currentstroke}{rgb}{0.000000,0.000000,0.000000}%
\pgfsetstrokecolor{currentstroke}%
\pgfsetdash{}{0pt}%
\pgfsys@defobject{currentmarker}{\pgfqpoint{0.000000in}{-0.048611in}}{\pgfqpoint{0.000000in}{0.000000in}}{%
\pgfpathmoveto{\pgfqpoint{0.000000in}{0.000000in}}%
\pgfpathlineto{\pgfqpoint{0.000000in}{-0.048611in}}%
\pgfusepath{stroke,fill}%
}%
\begin{pgfscope}%
\pgfsys@transformshift{1.562436in}{0.498088in}%
\pgfsys@useobject{currentmarker}{}%
\end{pgfscope}%
\end{pgfscope}%
\begin{pgfscope}%
\definecolor{textcolor}{rgb}{0.000000,0.000000,0.000000}%
\pgfsetstrokecolor{textcolor}%
\pgfsetfillcolor{textcolor}%
\pgftext[x=1.562436in,y=0.400866in,,top]{\color{textcolor}\rmfamily\fontsize{10.000000}{12.000000}\selectfont \(\displaystyle {100}\)}%
\end{pgfscope}%
\begin{pgfscope}%
\pgfsetbuttcap%
\pgfsetroundjoin%
\definecolor{currentfill}{rgb}{0.000000,0.000000,0.000000}%
\pgfsetfillcolor{currentfill}%
\pgfsetlinewidth{0.803000pt}%
\definecolor{currentstroke}{rgb}{0.000000,0.000000,0.000000}%
\pgfsetstrokecolor{currentstroke}%
\pgfsetdash{}{0pt}%
\pgfsys@defobject{currentmarker}{\pgfqpoint{0.000000in}{-0.048611in}}{\pgfqpoint{0.000000in}{0.000000in}}{%
\pgfpathmoveto{\pgfqpoint{0.000000in}{0.000000in}}%
\pgfpathlineto{\pgfqpoint{0.000000in}{-0.048611in}}%
\pgfusepath{stroke,fill}%
}%
\begin{pgfscope}%
\pgfsys@transformshift{2.070417in}{0.498088in}%
\pgfsys@useobject{currentmarker}{}%
\end{pgfscope}%
\end{pgfscope}%
\begin{pgfscope}%
\definecolor{textcolor}{rgb}{0.000000,0.000000,0.000000}%
\pgfsetstrokecolor{textcolor}%
\pgfsetfillcolor{textcolor}%
\pgftext[x=2.070417in,y=0.400866in,,top]{\color{textcolor}\rmfamily\fontsize{10.000000}{12.000000}\selectfont \(\displaystyle {125}\)}%
\end{pgfscope}%
\begin{pgfscope}%
\pgfsetbuttcap%
\pgfsetroundjoin%
\definecolor{currentfill}{rgb}{0.000000,0.000000,0.000000}%
\pgfsetfillcolor{currentfill}%
\pgfsetlinewidth{0.803000pt}%
\definecolor{currentstroke}{rgb}{0.000000,0.000000,0.000000}%
\pgfsetstrokecolor{currentstroke}%
\pgfsetdash{}{0pt}%
\pgfsys@defobject{currentmarker}{\pgfqpoint{0.000000in}{-0.048611in}}{\pgfqpoint{0.000000in}{0.000000in}}{%
\pgfpathmoveto{\pgfqpoint{0.000000in}{0.000000in}}%
\pgfpathlineto{\pgfqpoint{0.000000in}{-0.048611in}}%
\pgfusepath{stroke,fill}%
}%
\begin{pgfscope}%
\pgfsys@transformshift{2.578398in}{0.498088in}%
\pgfsys@useobject{currentmarker}{}%
\end{pgfscope}%
\end{pgfscope}%
\begin{pgfscope}%
\definecolor{textcolor}{rgb}{0.000000,0.000000,0.000000}%
\pgfsetstrokecolor{textcolor}%
\pgfsetfillcolor{textcolor}%
\pgftext[x=2.578398in,y=0.400866in,,top]{\color{textcolor}\rmfamily\fontsize{10.000000}{12.000000}\selectfont \(\displaystyle {150}\)}%
\end{pgfscope}%
\begin{pgfscope}%
\pgfsetbuttcap%
\pgfsetroundjoin%
\definecolor{currentfill}{rgb}{0.000000,0.000000,0.000000}%
\pgfsetfillcolor{currentfill}%
\pgfsetlinewidth{0.602250pt}%
\definecolor{currentstroke}{rgb}{0.000000,0.000000,0.000000}%
\pgfsetstrokecolor{currentstroke}%
\pgfsetdash{}{0pt}%
\pgfsys@defobject{currentmarker}{\pgfqpoint{0.000000in}{-0.027778in}}{\pgfqpoint{0.000000in}{0.000000in}}{%
\pgfpathmoveto{\pgfqpoint{0.000000in}{0.000000in}}%
\pgfpathlineto{\pgfqpoint{0.000000in}{-0.027778in}}%
\pgfusepath{stroke,fill}%
}%
\begin{pgfscope}%
\pgfsys@transformshift{0.444878in}{0.498088in}%
\pgfsys@useobject{currentmarker}{}%
\end{pgfscope}%
\end{pgfscope}%
\begin{pgfscope}%
\pgfsetbuttcap%
\pgfsetroundjoin%
\definecolor{currentfill}{rgb}{0.000000,0.000000,0.000000}%
\pgfsetfillcolor{currentfill}%
\pgfsetlinewidth{0.602250pt}%
\definecolor{currentstroke}{rgb}{0.000000,0.000000,0.000000}%
\pgfsetstrokecolor{currentstroke}%
\pgfsetdash{}{0pt}%
\pgfsys@defobject{currentmarker}{\pgfqpoint{0.000000in}{-0.027778in}}{\pgfqpoint{0.000000in}{0.000000in}}{%
\pgfpathmoveto{\pgfqpoint{0.000000in}{0.000000in}}%
\pgfpathlineto{\pgfqpoint{0.000000in}{-0.027778in}}%
\pgfusepath{stroke,fill}%
}%
\begin{pgfscope}%
\pgfsys@transformshift{0.648070in}{0.498088in}%
\pgfsys@useobject{currentmarker}{}%
\end{pgfscope}%
\end{pgfscope}%
\begin{pgfscope}%
\pgfsetbuttcap%
\pgfsetroundjoin%
\definecolor{currentfill}{rgb}{0.000000,0.000000,0.000000}%
\pgfsetfillcolor{currentfill}%
\pgfsetlinewidth{0.602250pt}%
\definecolor{currentstroke}{rgb}{0.000000,0.000000,0.000000}%
\pgfsetstrokecolor{currentstroke}%
\pgfsetdash{}{0pt}%
\pgfsys@defobject{currentmarker}{\pgfqpoint{0.000000in}{-0.027778in}}{\pgfqpoint{0.000000in}{0.000000in}}{%
\pgfpathmoveto{\pgfqpoint{0.000000in}{0.000000in}}%
\pgfpathlineto{\pgfqpoint{0.000000in}{-0.027778in}}%
\pgfusepath{stroke,fill}%
}%
\begin{pgfscope}%
\pgfsys@transformshift{0.749666in}{0.498088in}%
\pgfsys@useobject{currentmarker}{}%
\end{pgfscope}%
\end{pgfscope}%
\begin{pgfscope}%
\pgfsetbuttcap%
\pgfsetroundjoin%
\definecolor{currentfill}{rgb}{0.000000,0.000000,0.000000}%
\pgfsetfillcolor{currentfill}%
\pgfsetlinewidth{0.602250pt}%
\definecolor{currentstroke}{rgb}{0.000000,0.000000,0.000000}%
\pgfsetstrokecolor{currentstroke}%
\pgfsetdash{}{0pt}%
\pgfsys@defobject{currentmarker}{\pgfqpoint{0.000000in}{-0.027778in}}{\pgfqpoint{0.000000in}{0.000000in}}{%
\pgfpathmoveto{\pgfqpoint{0.000000in}{0.000000in}}%
\pgfpathlineto{\pgfqpoint{0.000000in}{-0.027778in}}%
\pgfusepath{stroke,fill}%
}%
\begin{pgfscope}%
\pgfsys@transformshift{0.851262in}{0.498088in}%
\pgfsys@useobject{currentmarker}{}%
\end{pgfscope}%
\end{pgfscope}%
\begin{pgfscope}%
\pgfsetbuttcap%
\pgfsetroundjoin%
\definecolor{currentfill}{rgb}{0.000000,0.000000,0.000000}%
\pgfsetfillcolor{currentfill}%
\pgfsetlinewidth{0.602250pt}%
\definecolor{currentstroke}{rgb}{0.000000,0.000000,0.000000}%
\pgfsetstrokecolor{currentstroke}%
\pgfsetdash{}{0pt}%
\pgfsys@defobject{currentmarker}{\pgfqpoint{0.000000in}{-0.027778in}}{\pgfqpoint{0.000000in}{0.000000in}}{%
\pgfpathmoveto{\pgfqpoint{0.000000in}{0.000000in}}%
\pgfpathlineto{\pgfqpoint{0.000000in}{-0.027778in}}%
\pgfusepath{stroke,fill}%
}%
\begin{pgfscope}%
\pgfsys@transformshift{0.952859in}{0.498088in}%
\pgfsys@useobject{currentmarker}{}%
\end{pgfscope}%
\end{pgfscope}%
\begin{pgfscope}%
\pgfsetbuttcap%
\pgfsetroundjoin%
\definecolor{currentfill}{rgb}{0.000000,0.000000,0.000000}%
\pgfsetfillcolor{currentfill}%
\pgfsetlinewidth{0.602250pt}%
\definecolor{currentstroke}{rgb}{0.000000,0.000000,0.000000}%
\pgfsetstrokecolor{currentstroke}%
\pgfsetdash{}{0pt}%
\pgfsys@defobject{currentmarker}{\pgfqpoint{0.000000in}{-0.027778in}}{\pgfqpoint{0.000000in}{0.000000in}}{%
\pgfpathmoveto{\pgfqpoint{0.000000in}{0.000000in}}%
\pgfpathlineto{\pgfqpoint{0.000000in}{-0.027778in}}%
\pgfusepath{stroke,fill}%
}%
\begin{pgfscope}%
\pgfsys@transformshift{1.156051in}{0.498088in}%
\pgfsys@useobject{currentmarker}{}%
\end{pgfscope}%
\end{pgfscope}%
\begin{pgfscope}%
\pgfsetbuttcap%
\pgfsetroundjoin%
\definecolor{currentfill}{rgb}{0.000000,0.000000,0.000000}%
\pgfsetfillcolor{currentfill}%
\pgfsetlinewidth{0.602250pt}%
\definecolor{currentstroke}{rgb}{0.000000,0.000000,0.000000}%
\pgfsetstrokecolor{currentstroke}%
\pgfsetdash{}{0pt}%
\pgfsys@defobject{currentmarker}{\pgfqpoint{0.000000in}{-0.027778in}}{\pgfqpoint{0.000000in}{0.000000in}}{%
\pgfpathmoveto{\pgfqpoint{0.000000in}{0.000000in}}%
\pgfpathlineto{\pgfqpoint{0.000000in}{-0.027778in}}%
\pgfusepath{stroke,fill}%
}%
\begin{pgfscope}%
\pgfsys@transformshift{1.257647in}{0.498088in}%
\pgfsys@useobject{currentmarker}{}%
\end{pgfscope}%
\end{pgfscope}%
\begin{pgfscope}%
\pgfsetbuttcap%
\pgfsetroundjoin%
\definecolor{currentfill}{rgb}{0.000000,0.000000,0.000000}%
\pgfsetfillcolor{currentfill}%
\pgfsetlinewidth{0.602250pt}%
\definecolor{currentstroke}{rgb}{0.000000,0.000000,0.000000}%
\pgfsetstrokecolor{currentstroke}%
\pgfsetdash{}{0pt}%
\pgfsys@defobject{currentmarker}{\pgfqpoint{0.000000in}{-0.027778in}}{\pgfqpoint{0.000000in}{0.000000in}}{%
\pgfpathmoveto{\pgfqpoint{0.000000in}{0.000000in}}%
\pgfpathlineto{\pgfqpoint{0.000000in}{-0.027778in}}%
\pgfusepath{stroke,fill}%
}%
\begin{pgfscope}%
\pgfsys@transformshift{1.359244in}{0.498088in}%
\pgfsys@useobject{currentmarker}{}%
\end{pgfscope}%
\end{pgfscope}%
\begin{pgfscope}%
\pgfsetbuttcap%
\pgfsetroundjoin%
\definecolor{currentfill}{rgb}{0.000000,0.000000,0.000000}%
\pgfsetfillcolor{currentfill}%
\pgfsetlinewidth{0.602250pt}%
\definecolor{currentstroke}{rgb}{0.000000,0.000000,0.000000}%
\pgfsetstrokecolor{currentstroke}%
\pgfsetdash{}{0pt}%
\pgfsys@defobject{currentmarker}{\pgfqpoint{0.000000in}{-0.027778in}}{\pgfqpoint{0.000000in}{0.000000in}}{%
\pgfpathmoveto{\pgfqpoint{0.000000in}{0.000000in}}%
\pgfpathlineto{\pgfqpoint{0.000000in}{-0.027778in}}%
\pgfusepath{stroke,fill}%
}%
\begin{pgfscope}%
\pgfsys@transformshift{1.460840in}{0.498088in}%
\pgfsys@useobject{currentmarker}{}%
\end{pgfscope}%
\end{pgfscope}%
\begin{pgfscope}%
\pgfsetbuttcap%
\pgfsetroundjoin%
\definecolor{currentfill}{rgb}{0.000000,0.000000,0.000000}%
\pgfsetfillcolor{currentfill}%
\pgfsetlinewidth{0.602250pt}%
\definecolor{currentstroke}{rgb}{0.000000,0.000000,0.000000}%
\pgfsetstrokecolor{currentstroke}%
\pgfsetdash{}{0pt}%
\pgfsys@defobject{currentmarker}{\pgfqpoint{0.000000in}{-0.027778in}}{\pgfqpoint{0.000000in}{0.000000in}}{%
\pgfpathmoveto{\pgfqpoint{0.000000in}{0.000000in}}%
\pgfpathlineto{\pgfqpoint{0.000000in}{-0.027778in}}%
\pgfusepath{stroke,fill}%
}%
\begin{pgfscope}%
\pgfsys@transformshift{1.664032in}{0.498088in}%
\pgfsys@useobject{currentmarker}{}%
\end{pgfscope}%
\end{pgfscope}%
\begin{pgfscope}%
\pgfsetbuttcap%
\pgfsetroundjoin%
\definecolor{currentfill}{rgb}{0.000000,0.000000,0.000000}%
\pgfsetfillcolor{currentfill}%
\pgfsetlinewidth{0.602250pt}%
\definecolor{currentstroke}{rgb}{0.000000,0.000000,0.000000}%
\pgfsetstrokecolor{currentstroke}%
\pgfsetdash{}{0pt}%
\pgfsys@defobject{currentmarker}{\pgfqpoint{0.000000in}{-0.027778in}}{\pgfqpoint{0.000000in}{0.000000in}}{%
\pgfpathmoveto{\pgfqpoint{0.000000in}{0.000000in}}%
\pgfpathlineto{\pgfqpoint{0.000000in}{-0.027778in}}%
\pgfusepath{stroke,fill}%
}%
\begin{pgfscope}%
\pgfsys@transformshift{1.765628in}{0.498088in}%
\pgfsys@useobject{currentmarker}{}%
\end{pgfscope}%
\end{pgfscope}%
\begin{pgfscope}%
\pgfsetbuttcap%
\pgfsetroundjoin%
\definecolor{currentfill}{rgb}{0.000000,0.000000,0.000000}%
\pgfsetfillcolor{currentfill}%
\pgfsetlinewidth{0.602250pt}%
\definecolor{currentstroke}{rgb}{0.000000,0.000000,0.000000}%
\pgfsetstrokecolor{currentstroke}%
\pgfsetdash{}{0pt}%
\pgfsys@defobject{currentmarker}{\pgfqpoint{0.000000in}{-0.027778in}}{\pgfqpoint{0.000000in}{0.000000in}}{%
\pgfpathmoveto{\pgfqpoint{0.000000in}{0.000000in}}%
\pgfpathlineto{\pgfqpoint{0.000000in}{-0.027778in}}%
\pgfusepath{stroke,fill}%
}%
\begin{pgfscope}%
\pgfsys@transformshift{1.867225in}{0.498088in}%
\pgfsys@useobject{currentmarker}{}%
\end{pgfscope}%
\end{pgfscope}%
\begin{pgfscope}%
\pgfsetbuttcap%
\pgfsetroundjoin%
\definecolor{currentfill}{rgb}{0.000000,0.000000,0.000000}%
\pgfsetfillcolor{currentfill}%
\pgfsetlinewidth{0.602250pt}%
\definecolor{currentstroke}{rgb}{0.000000,0.000000,0.000000}%
\pgfsetstrokecolor{currentstroke}%
\pgfsetdash{}{0pt}%
\pgfsys@defobject{currentmarker}{\pgfqpoint{0.000000in}{-0.027778in}}{\pgfqpoint{0.000000in}{0.000000in}}{%
\pgfpathmoveto{\pgfqpoint{0.000000in}{0.000000in}}%
\pgfpathlineto{\pgfqpoint{0.000000in}{-0.027778in}}%
\pgfusepath{stroke,fill}%
}%
\begin{pgfscope}%
\pgfsys@transformshift{1.968821in}{0.498088in}%
\pgfsys@useobject{currentmarker}{}%
\end{pgfscope}%
\end{pgfscope}%
\begin{pgfscope}%
\pgfsetbuttcap%
\pgfsetroundjoin%
\definecolor{currentfill}{rgb}{0.000000,0.000000,0.000000}%
\pgfsetfillcolor{currentfill}%
\pgfsetlinewidth{0.602250pt}%
\definecolor{currentstroke}{rgb}{0.000000,0.000000,0.000000}%
\pgfsetstrokecolor{currentstroke}%
\pgfsetdash{}{0pt}%
\pgfsys@defobject{currentmarker}{\pgfqpoint{0.000000in}{-0.027778in}}{\pgfqpoint{0.000000in}{0.000000in}}{%
\pgfpathmoveto{\pgfqpoint{0.000000in}{0.000000in}}%
\pgfpathlineto{\pgfqpoint{0.000000in}{-0.027778in}}%
\pgfusepath{stroke,fill}%
}%
\begin{pgfscope}%
\pgfsys@transformshift{2.172013in}{0.498088in}%
\pgfsys@useobject{currentmarker}{}%
\end{pgfscope}%
\end{pgfscope}%
\begin{pgfscope}%
\pgfsetbuttcap%
\pgfsetroundjoin%
\definecolor{currentfill}{rgb}{0.000000,0.000000,0.000000}%
\pgfsetfillcolor{currentfill}%
\pgfsetlinewidth{0.602250pt}%
\definecolor{currentstroke}{rgb}{0.000000,0.000000,0.000000}%
\pgfsetstrokecolor{currentstroke}%
\pgfsetdash{}{0pt}%
\pgfsys@defobject{currentmarker}{\pgfqpoint{0.000000in}{-0.027778in}}{\pgfqpoint{0.000000in}{0.000000in}}{%
\pgfpathmoveto{\pgfqpoint{0.000000in}{0.000000in}}%
\pgfpathlineto{\pgfqpoint{0.000000in}{-0.027778in}}%
\pgfusepath{stroke,fill}%
}%
\begin{pgfscope}%
\pgfsys@transformshift{2.273609in}{0.498088in}%
\pgfsys@useobject{currentmarker}{}%
\end{pgfscope}%
\end{pgfscope}%
\begin{pgfscope}%
\pgfsetbuttcap%
\pgfsetroundjoin%
\definecolor{currentfill}{rgb}{0.000000,0.000000,0.000000}%
\pgfsetfillcolor{currentfill}%
\pgfsetlinewidth{0.602250pt}%
\definecolor{currentstroke}{rgb}{0.000000,0.000000,0.000000}%
\pgfsetstrokecolor{currentstroke}%
\pgfsetdash{}{0pt}%
\pgfsys@defobject{currentmarker}{\pgfqpoint{0.000000in}{-0.027778in}}{\pgfqpoint{0.000000in}{0.000000in}}{%
\pgfpathmoveto{\pgfqpoint{0.000000in}{0.000000in}}%
\pgfpathlineto{\pgfqpoint{0.000000in}{-0.027778in}}%
\pgfusepath{stroke,fill}%
}%
\begin{pgfscope}%
\pgfsys@transformshift{2.375206in}{0.498088in}%
\pgfsys@useobject{currentmarker}{}%
\end{pgfscope}%
\end{pgfscope}%
\begin{pgfscope}%
\pgfsetbuttcap%
\pgfsetroundjoin%
\definecolor{currentfill}{rgb}{0.000000,0.000000,0.000000}%
\pgfsetfillcolor{currentfill}%
\pgfsetlinewidth{0.602250pt}%
\definecolor{currentstroke}{rgb}{0.000000,0.000000,0.000000}%
\pgfsetstrokecolor{currentstroke}%
\pgfsetdash{}{0pt}%
\pgfsys@defobject{currentmarker}{\pgfqpoint{0.000000in}{-0.027778in}}{\pgfqpoint{0.000000in}{0.000000in}}{%
\pgfpathmoveto{\pgfqpoint{0.000000in}{0.000000in}}%
\pgfpathlineto{\pgfqpoint{0.000000in}{-0.027778in}}%
\pgfusepath{stroke,fill}%
}%
\begin{pgfscope}%
\pgfsys@transformshift{2.476802in}{0.498088in}%
\pgfsys@useobject{currentmarker}{}%
\end{pgfscope}%
\end{pgfscope}%
\begin{pgfscope}%
\pgfsetbuttcap%
\pgfsetroundjoin%
\definecolor{currentfill}{rgb}{0.000000,0.000000,0.000000}%
\pgfsetfillcolor{currentfill}%
\pgfsetlinewidth{0.602250pt}%
\definecolor{currentstroke}{rgb}{0.000000,0.000000,0.000000}%
\pgfsetstrokecolor{currentstroke}%
\pgfsetdash{}{0pt}%
\pgfsys@defobject{currentmarker}{\pgfqpoint{0.000000in}{-0.027778in}}{\pgfqpoint{0.000000in}{0.000000in}}{%
\pgfpathmoveto{\pgfqpoint{0.000000in}{0.000000in}}%
\pgfpathlineto{\pgfqpoint{0.000000in}{-0.027778in}}%
\pgfusepath{stroke,fill}%
}%
\begin{pgfscope}%
\pgfsys@transformshift{2.679994in}{0.498088in}%
\pgfsys@useobject{currentmarker}{}%
\end{pgfscope}%
\end{pgfscope}%
\begin{pgfscope}%
\pgfsetbuttcap%
\pgfsetroundjoin%
\definecolor{currentfill}{rgb}{0.000000,0.000000,0.000000}%
\pgfsetfillcolor{currentfill}%
\pgfsetlinewidth{0.602250pt}%
\definecolor{currentstroke}{rgb}{0.000000,0.000000,0.000000}%
\pgfsetstrokecolor{currentstroke}%
\pgfsetdash{}{0pt}%
\pgfsys@defobject{currentmarker}{\pgfqpoint{0.000000in}{-0.027778in}}{\pgfqpoint{0.000000in}{0.000000in}}{%
\pgfpathmoveto{\pgfqpoint{0.000000in}{0.000000in}}%
\pgfpathlineto{\pgfqpoint{0.000000in}{-0.027778in}}%
\pgfusepath{stroke,fill}%
}%
\begin{pgfscope}%
\pgfsys@transformshift{2.781590in}{0.498088in}%
\pgfsys@useobject{currentmarker}{}%
\end{pgfscope}%
\end{pgfscope}%
\begin{pgfscope}%
\pgfsetbuttcap%
\pgfsetroundjoin%
\definecolor{currentfill}{rgb}{0.000000,0.000000,0.000000}%
\pgfsetfillcolor{currentfill}%
\pgfsetlinewidth{0.602250pt}%
\definecolor{currentstroke}{rgb}{0.000000,0.000000,0.000000}%
\pgfsetstrokecolor{currentstroke}%
\pgfsetdash{}{0pt}%
\pgfsys@defobject{currentmarker}{\pgfqpoint{0.000000in}{-0.027778in}}{\pgfqpoint{0.000000in}{0.000000in}}{%
\pgfpathmoveto{\pgfqpoint{0.000000in}{0.000000in}}%
\pgfpathlineto{\pgfqpoint{0.000000in}{-0.027778in}}%
\pgfusepath{stroke,fill}%
}%
\begin{pgfscope}%
\pgfsys@transformshift{2.883187in}{0.498088in}%
\pgfsys@useobject{currentmarker}{}%
\end{pgfscope}%
\end{pgfscope}%
\begin{pgfscope}%
\definecolor{textcolor}{rgb}{0.000000,0.000000,0.000000}%
\pgfsetstrokecolor{textcolor}%
\pgfsetfillcolor{textcolor}%
\pgftext[x=1.684351in,y=0.222655in,,top]{\color{textcolor}\rmfamily\fontsize{10.000000}{12.000000}\selectfont Schwellenwert \(\displaystyle s_V\) in ADU}%
\end{pgfscope}%
\begin{pgfscope}%
\pgfsetbuttcap%
\pgfsetroundjoin%
\definecolor{currentfill}{rgb}{0.000000,0.000000,0.000000}%
\pgfsetfillcolor{currentfill}%
\pgfsetlinewidth{0.803000pt}%
\definecolor{currentstroke}{rgb}{0.000000,0.000000,0.000000}%
\pgfsetstrokecolor{currentstroke}%
\pgfsetdash{}{0pt}%
\pgfsys@defobject{currentmarker}{\pgfqpoint{-0.048611in}{0.000000in}}{\pgfqpoint{-0.000000in}{0.000000in}}{%
\pgfpathmoveto{\pgfqpoint{-0.000000in}{0.000000in}}%
\pgfpathlineto{\pgfqpoint{-0.048611in}{0.000000in}}%
\pgfusepath{stroke,fill}%
}%
\begin{pgfscope}%
\pgfsys@transformshift{0.444878in}{0.551882in}%
\pgfsys@useobject{currentmarker}{}%
\end{pgfscope}%
\end{pgfscope}%
\begin{pgfscope}%
\definecolor{textcolor}{rgb}{0.000000,0.000000,0.000000}%
\pgfsetstrokecolor{textcolor}%
\pgfsetfillcolor{textcolor}%
\pgftext[x=0.278211in, y=0.504055in, left, base]{\color{textcolor}\rmfamily\fontsize{10.000000}{12.000000}\selectfont \num{0}}%
\end{pgfscope}%
\begin{pgfscope}%
\pgfsetbuttcap%
\pgfsetroundjoin%
\definecolor{currentfill}{rgb}{0.000000,0.000000,0.000000}%
\pgfsetfillcolor{currentfill}%
\pgfsetlinewidth{0.803000pt}%
\definecolor{currentstroke}{rgb}{0.000000,0.000000,0.000000}%
\pgfsetstrokecolor{currentstroke}%
\pgfsetdash{}{0pt}%
\pgfsys@defobject{currentmarker}{\pgfqpoint{-0.048611in}{0.000000in}}{\pgfqpoint{-0.000000in}{0.000000in}}{%
\pgfpathmoveto{\pgfqpoint{-0.000000in}{0.000000in}}%
\pgfpathlineto{\pgfqpoint{-0.048611in}{0.000000in}}%
\pgfusepath{stroke,fill}%
}%
\begin{pgfscope}%
\pgfsys@transformshift{0.444878in}{1.089825in}%
\pgfsys@useobject{currentmarker}{}%
\end{pgfscope}%
\end{pgfscope}%
\begin{pgfscope}%
\definecolor{textcolor}{rgb}{0.000000,0.000000,0.000000}%
\pgfsetstrokecolor{textcolor}%
\pgfsetfillcolor{textcolor}%
\pgftext[x=0.278211in, y=1.041997in, left, base]{\color{textcolor}\rmfamily\fontsize{10.000000}{12.000000}\selectfont \num{2}}%
\end{pgfscope}%
\begin{pgfscope}%
\pgfsetbuttcap%
\pgfsetroundjoin%
\definecolor{currentfill}{rgb}{0.000000,0.000000,0.000000}%
\pgfsetfillcolor{currentfill}%
\pgfsetlinewidth{0.803000pt}%
\definecolor{currentstroke}{rgb}{0.000000,0.000000,0.000000}%
\pgfsetstrokecolor{currentstroke}%
\pgfsetdash{}{0pt}%
\pgfsys@defobject{currentmarker}{\pgfqpoint{-0.048611in}{0.000000in}}{\pgfqpoint{-0.000000in}{0.000000in}}{%
\pgfpathmoveto{\pgfqpoint{-0.000000in}{0.000000in}}%
\pgfpathlineto{\pgfqpoint{-0.048611in}{0.000000in}}%
\pgfusepath{stroke,fill}%
}%
\begin{pgfscope}%
\pgfsys@transformshift{0.444878in}{1.627768in}%
\pgfsys@useobject{currentmarker}{}%
\end{pgfscope}%
\end{pgfscope}%
\begin{pgfscope}%
\definecolor{textcolor}{rgb}{0.000000,0.000000,0.000000}%
\pgfsetstrokecolor{textcolor}%
\pgfsetfillcolor{textcolor}%
\pgftext[x=0.278211in, y=1.579940in, left, base]{\color{textcolor}\rmfamily\fontsize{10.000000}{12.000000}\selectfont \num{4}}%
\end{pgfscope}%
\begin{pgfscope}%
\pgfsetbuttcap%
\pgfsetroundjoin%
\definecolor{currentfill}{rgb}{0.000000,0.000000,0.000000}%
\pgfsetfillcolor{currentfill}%
\pgfsetlinewidth{0.803000pt}%
\definecolor{currentstroke}{rgb}{0.000000,0.000000,0.000000}%
\pgfsetstrokecolor{currentstroke}%
\pgfsetdash{}{0pt}%
\pgfsys@defobject{currentmarker}{\pgfqpoint{-0.048611in}{0.000000in}}{\pgfqpoint{-0.000000in}{0.000000in}}{%
\pgfpathmoveto{\pgfqpoint{-0.000000in}{0.000000in}}%
\pgfpathlineto{\pgfqpoint{-0.048611in}{0.000000in}}%
\pgfusepath{stroke,fill}%
}%
\begin{pgfscope}%
\pgfsys@transformshift{0.444878in}{2.165711in}%
\pgfsys@useobject{currentmarker}{}%
\end{pgfscope}%
\end{pgfscope}%
\begin{pgfscope}%
\definecolor{textcolor}{rgb}{0.000000,0.000000,0.000000}%
\pgfsetstrokecolor{textcolor}%
\pgfsetfillcolor{textcolor}%
\pgftext[x=0.278211in, y=2.117883in, left, base]{\color{textcolor}\rmfamily\fontsize{10.000000}{12.000000}\selectfont \num{6}}%
\end{pgfscope}%
\begin{pgfscope}%
\pgfsetbuttcap%
\pgfsetroundjoin%
\definecolor{currentfill}{rgb}{0.000000,0.000000,0.000000}%
\pgfsetfillcolor{currentfill}%
\pgfsetlinewidth{0.602250pt}%
\definecolor{currentstroke}{rgb}{0.000000,0.000000,0.000000}%
\pgfsetstrokecolor{currentstroke}%
\pgfsetdash{}{0pt}%
\pgfsys@defobject{currentmarker}{\pgfqpoint{-0.027778in}{0.000000in}}{\pgfqpoint{-0.000000in}{0.000000in}}{%
\pgfpathmoveto{\pgfqpoint{-0.000000in}{0.000000in}}%
\pgfpathlineto{\pgfqpoint{-0.027778in}{0.000000in}}%
\pgfusepath{stroke,fill}%
}%
\begin{pgfscope}%
\pgfsys@transformshift{0.444878in}{0.686368in}%
\pgfsys@useobject{currentmarker}{}%
\end{pgfscope}%
\end{pgfscope}%
\begin{pgfscope}%
\pgfsetbuttcap%
\pgfsetroundjoin%
\definecolor{currentfill}{rgb}{0.000000,0.000000,0.000000}%
\pgfsetfillcolor{currentfill}%
\pgfsetlinewidth{0.602250pt}%
\definecolor{currentstroke}{rgb}{0.000000,0.000000,0.000000}%
\pgfsetstrokecolor{currentstroke}%
\pgfsetdash{}{0pt}%
\pgfsys@defobject{currentmarker}{\pgfqpoint{-0.027778in}{0.000000in}}{\pgfqpoint{-0.000000in}{0.000000in}}{%
\pgfpathmoveto{\pgfqpoint{-0.000000in}{0.000000in}}%
\pgfpathlineto{\pgfqpoint{-0.027778in}{0.000000in}}%
\pgfusepath{stroke,fill}%
}%
\begin{pgfscope}%
\pgfsys@transformshift{0.444878in}{0.820854in}%
\pgfsys@useobject{currentmarker}{}%
\end{pgfscope}%
\end{pgfscope}%
\begin{pgfscope}%
\pgfsetbuttcap%
\pgfsetroundjoin%
\definecolor{currentfill}{rgb}{0.000000,0.000000,0.000000}%
\pgfsetfillcolor{currentfill}%
\pgfsetlinewidth{0.602250pt}%
\definecolor{currentstroke}{rgb}{0.000000,0.000000,0.000000}%
\pgfsetstrokecolor{currentstroke}%
\pgfsetdash{}{0pt}%
\pgfsys@defobject{currentmarker}{\pgfqpoint{-0.027778in}{0.000000in}}{\pgfqpoint{-0.000000in}{0.000000in}}{%
\pgfpathmoveto{\pgfqpoint{-0.000000in}{0.000000in}}%
\pgfpathlineto{\pgfqpoint{-0.027778in}{0.000000in}}%
\pgfusepath{stroke,fill}%
}%
\begin{pgfscope}%
\pgfsys@transformshift{0.444878in}{0.955339in}%
\pgfsys@useobject{currentmarker}{}%
\end{pgfscope}%
\end{pgfscope}%
\begin{pgfscope}%
\pgfsetbuttcap%
\pgfsetroundjoin%
\definecolor{currentfill}{rgb}{0.000000,0.000000,0.000000}%
\pgfsetfillcolor{currentfill}%
\pgfsetlinewidth{0.602250pt}%
\definecolor{currentstroke}{rgb}{0.000000,0.000000,0.000000}%
\pgfsetstrokecolor{currentstroke}%
\pgfsetdash{}{0pt}%
\pgfsys@defobject{currentmarker}{\pgfqpoint{-0.027778in}{0.000000in}}{\pgfqpoint{-0.000000in}{0.000000in}}{%
\pgfpathmoveto{\pgfqpoint{-0.000000in}{0.000000in}}%
\pgfpathlineto{\pgfqpoint{-0.027778in}{0.000000in}}%
\pgfusepath{stroke,fill}%
}%
\begin{pgfscope}%
\pgfsys@transformshift{0.444878in}{1.224311in}%
\pgfsys@useobject{currentmarker}{}%
\end{pgfscope}%
\end{pgfscope}%
\begin{pgfscope}%
\pgfsetbuttcap%
\pgfsetroundjoin%
\definecolor{currentfill}{rgb}{0.000000,0.000000,0.000000}%
\pgfsetfillcolor{currentfill}%
\pgfsetlinewidth{0.602250pt}%
\definecolor{currentstroke}{rgb}{0.000000,0.000000,0.000000}%
\pgfsetstrokecolor{currentstroke}%
\pgfsetdash{}{0pt}%
\pgfsys@defobject{currentmarker}{\pgfqpoint{-0.027778in}{0.000000in}}{\pgfqpoint{-0.000000in}{0.000000in}}{%
\pgfpathmoveto{\pgfqpoint{-0.000000in}{0.000000in}}%
\pgfpathlineto{\pgfqpoint{-0.027778in}{0.000000in}}%
\pgfusepath{stroke,fill}%
}%
\begin{pgfscope}%
\pgfsys@transformshift{0.444878in}{1.358797in}%
\pgfsys@useobject{currentmarker}{}%
\end{pgfscope}%
\end{pgfscope}%
\begin{pgfscope}%
\pgfsetbuttcap%
\pgfsetroundjoin%
\definecolor{currentfill}{rgb}{0.000000,0.000000,0.000000}%
\pgfsetfillcolor{currentfill}%
\pgfsetlinewidth{0.602250pt}%
\definecolor{currentstroke}{rgb}{0.000000,0.000000,0.000000}%
\pgfsetstrokecolor{currentstroke}%
\pgfsetdash{}{0pt}%
\pgfsys@defobject{currentmarker}{\pgfqpoint{-0.027778in}{0.000000in}}{\pgfqpoint{-0.000000in}{0.000000in}}{%
\pgfpathmoveto{\pgfqpoint{-0.000000in}{0.000000in}}%
\pgfpathlineto{\pgfqpoint{-0.027778in}{0.000000in}}%
\pgfusepath{stroke,fill}%
}%
\begin{pgfscope}%
\pgfsys@transformshift{0.444878in}{1.493282in}%
\pgfsys@useobject{currentmarker}{}%
\end{pgfscope}%
\end{pgfscope}%
\begin{pgfscope}%
\pgfsetbuttcap%
\pgfsetroundjoin%
\definecolor{currentfill}{rgb}{0.000000,0.000000,0.000000}%
\pgfsetfillcolor{currentfill}%
\pgfsetlinewidth{0.602250pt}%
\definecolor{currentstroke}{rgb}{0.000000,0.000000,0.000000}%
\pgfsetstrokecolor{currentstroke}%
\pgfsetdash{}{0pt}%
\pgfsys@defobject{currentmarker}{\pgfqpoint{-0.027778in}{0.000000in}}{\pgfqpoint{-0.000000in}{0.000000in}}{%
\pgfpathmoveto{\pgfqpoint{-0.000000in}{0.000000in}}%
\pgfpathlineto{\pgfqpoint{-0.027778in}{0.000000in}}%
\pgfusepath{stroke,fill}%
}%
\begin{pgfscope}%
\pgfsys@transformshift{0.444878in}{1.762254in}%
\pgfsys@useobject{currentmarker}{}%
\end{pgfscope}%
\end{pgfscope}%
\begin{pgfscope}%
\pgfsetbuttcap%
\pgfsetroundjoin%
\definecolor{currentfill}{rgb}{0.000000,0.000000,0.000000}%
\pgfsetfillcolor{currentfill}%
\pgfsetlinewidth{0.602250pt}%
\definecolor{currentstroke}{rgb}{0.000000,0.000000,0.000000}%
\pgfsetstrokecolor{currentstroke}%
\pgfsetdash{}{0pt}%
\pgfsys@defobject{currentmarker}{\pgfqpoint{-0.027778in}{0.000000in}}{\pgfqpoint{-0.000000in}{0.000000in}}{%
\pgfpathmoveto{\pgfqpoint{-0.000000in}{0.000000in}}%
\pgfpathlineto{\pgfqpoint{-0.027778in}{0.000000in}}%
\pgfusepath{stroke,fill}%
}%
\begin{pgfscope}%
\pgfsys@transformshift{0.444878in}{1.896739in}%
\pgfsys@useobject{currentmarker}{}%
\end{pgfscope}%
\end{pgfscope}%
\begin{pgfscope}%
\pgfsetbuttcap%
\pgfsetroundjoin%
\definecolor{currentfill}{rgb}{0.000000,0.000000,0.000000}%
\pgfsetfillcolor{currentfill}%
\pgfsetlinewidth{0.602250pt}%
\definecolor{currentstroke}{rgb}{0.000000,0.000000,0.000000}%
\pgfsetstrokecolor{currentstroke}%
\pgfsetdash{}{0pt}%
\pgfsys@defobject{currentmarker}{\pgfqpoint{-0.027778in}{0.000000in}}{\pgfqpoint{-0.000000in}{0.000000in}}{%
\pgfpathmoveto{\pgfqpoint{-0.000000in}{0.000000in}}%
\pgfpathlineto{\pgfqpoint{-0.027778in}{0.000000in}}%
\pgfusepath{stroke,fill}%
}%
\begin{pgfscope}%
\pgfsys@transformshift{0.444878in}{2.031225in}%
\pgfsys@useobject{currentmarker}{}%
\end{pgfscope}%
\end{pgfscope}%
\begin{pgfscope}%
\pgfsetbuttcap%
\pgfsetroundjoin%
\definecolor{currentfill}{rgb}{0.000000,0.000000,0.000000}%
\pgfsetfillcolor{currentfill}%
\pgfsetlinewidth{0.602250pt}%
\definecolor{currentstroke}{rgb}{0.000000,0.000000,0.000000}%
\pgfsetstrokecolor{currentstroke}%
\pgfsetdash{}{0pt}%
\pgfsys@defobject{currentmarker}{\pgfqpoint{-0.027778in}{0.000000in}}{\pgfqpoint{-0.000000in}{0.000000in}}{%
\pgfpathmoveto{\pgfqpoint{-0.000000in}{0.000000in}}%
\pgfpathlineto{\pgfqpoint{-0.027778in}{0.000000in}}%
\pgfusepath{stroke,fill}%
}%
\begin{pgfscope}%
\pgfsys@transformshift{0.444878in}{2.300197in}%
\pgfsys@useobject{currentmarker}{}%
\end{pgfscope}%
\end{pgfscope}%
\begin{pgfscope}%
\definecolor{textcolor}{rgb}{0.000000,0.000000,0.000000}%
\pgfsetstrokecolor{textcolor}%
\pgfsetfillcolor{textcolor}%
\pgftext[x=0.222655in,y=1.399142in,,bottom,rotate=90.000000]{\color{textcolor}\rmfamily\fontsize{10.000000}{12.000000}\selectfont \(\displaystyle S_{\text{EW}}\) in Photonen}%
\end{pgfscope}%
\begin{pgfscope}%
\pgfsetrectcap%
\pgfsetmiterjoin%
\pgfsetlinewidth{0.803000pt}%
\definecolor{currentstroke}{rgb}{0.000000,0.000000,0.000000}%
\pgfsetstrokecolor{currentstroke}%
\pgfsetdash{}{0pt}%
\pgfpathmoveto{\pgfqpoint{0.444878in}{0.498088in}}%
\pgfpathlineto{\pgfqpoint{0.444878in}{2.300197in}}%
\pgfusepath{stroke}%
\end{pgfscope}%
\begin{pgfscope}%
\pgfsetrectcap%
\pgfsetmiterjoin%
\pgfsetlinewidth{0.803000pt}%
\definecolor{currentstroke}{rgb}{0.000000,0.000000,0.000000}%
\pgfsetstrokecolor{currentstroke}%
\pgfsetdash{}{0pt}%
\pgfpathmoveto{\pgfqpoint{2.923825in}{0.498088in}}%
\pgfpathlineto{\pgfqpoint{2.923825in}{2.300197in}}%
\pgfusepath{stroke}%
\end{pgfscope}%
\begin{pgfscope}%
\pgfsetrectcap%
\pgfsetmiterjoin%
\pgfsetlinewidth{0.803000pt}%
\definecolor{currentstroke}{rgb}{0.000000,0.000000,0.000000}%
\pgfsetstrokecolor{currentstroke}%
\pgfsetdash{}{0pt}%
\pgfpathmoveto{\pgfqpoint{0.444878in}{0.498088in}}%
\pgfpathlineto{\pgfqpoint{2.923825in}{0.498088in}}%
\pgfusepath{stroke}%
\end{pgfscope}%
\begin{pgfscope}%
\pgfsetrectcap%
\pgfsetmiterjoin%
\pgfsetlinewidth{0.803000pt}%
\definecolor{currentstroke}{rgb}{0.000000,0.000000,0.000000}%
\pgfsetstrokecolor{currentstroke}%
\pgfsetdash{}{0pt}%
\pgfpathmoveto{\pgfqpoint{0.444878in}{2.300197in}}%
\pgfpathlineto{\pgfqpoint{2.923825in}{2.300197in}}%
\pgfusepath{stroke}%
\end{pgfscope}%
\begin{pgfscope}%
\pgfpathrectangle{\pgfqpoint{0.444878in}{0.498088in}}{\pgfqpoint{2.478947in}{1.802109in}}%
\pgfusepath{clip}%
\pgfsetrectcap%
\pgfsetroundjoin%
\pgfsetlinewidth{1.505625pt}%
\definecolor{currentstroke}{rgb}{0.121569,0.466667,0.705882}%
\pgfsetstrokecolor{currentstroke}%
\pgfsetdash{}{0pt}%
\pgfpathmoveto{\pgfqpoint{0.546474in}{2.176873in}}%
\pgfpathlineto{\pgfqpoint{0.648070in}{1.878192in}}%
\pgfpathlineto{\pgfqpoint{0.749666in}{1.651213in}}%
\pgfpathlineto{\pgfqpoint{0.851262in}{1.478070in}}%
\pgfpathlineto{\pgfqpoint{0.952859in}{1.341616in}}%
\pgfpathlineto{\pgfqpoint{1.054455in}{1.241397in}}%
\pgfpathlineto{\pgfqpoint{1.156051in}{1.160278in}}%
\pgfpathlineto{\pgfqpoint{1.257647in}{1.092006in}}%
\pgfpathlineto{\pgfqpoint{1.359244in}{1.028325in}}%
\pgfpathlineto{\pgfqpoint{1.460840in}{0.976104in}}%
\pgfpathlineto{\pgfqpoint{1.562436in}{0.925215in}}%
\pgfpathlineto{\pgfqpoint{1.664032in}{0.877990in}}%
\pgfpathlineto{\pgfqpoint{1.765628in}{0.836039in}}%
\pgfpathlineto{\pgfqpoint{1.867225in}{0.796133in}}%
\pgfpathlineto{\pgfqpoint{1.968821in}{0.760196in}}%
\pgfpathlineto{\pgfqpoint{2.070417in}{0.726842in}}%
\pgfpathlineto{\pgfqpoint{2.172013in}{0.697891in}}%
\pgfpathlineto{\pgfqpoint{2.273609in}{0.671014in}}%
\pgfpathlineto{\pgfqpoint{2.375206in}{0.647934in}}%
\pgfpathlineto{\pgfqpoint{2.476802in}{0.628609in}}%
\pgfpathlineto{\pgfqpoint{2.578398in}{0.613181in}}%
\pgfpathlineto{\pgfqpoint{2.679994in}{0.601109in}}%
\pgfpathlineto{\pgfqpoint{2.781590in}{0.590574in}}%
\pgfusepath{stroke}%
\end{pgfscope}%
\begin{pgfscope}%
\pgfsetbuttcap%
\pgfsetmiterjoin%
\definecolor{currentfill}{rgb}{1.000000,1.000000,1.000000}%
\pgfsetfillcolor{currentfill}%
\pgfsetlinewidth{0.000000pt}%
\definecolor{currentstroke}{rgb}{0.000000,0.000000,0.000000}%
\pgfsetstrokecolor{currentstroke}%
\pgfsetstrokeopacity{0.000000}%
\pgfsetdash{}{0pt}%
\pgfpathmoveto{\pgfqpoint{3.073825in}{0.498088in}}%
\pgfpathlineto{\pgfqpoint{5.552773in}{0.498088in}}%
\pgfpathlineto{\pgfqpoint{5.552773in}{2.300197in}}%
\pgfpathlineto{\pgfqpoint{3.073825in}{2.300197in}}%
\pgfpathlineto{\pgfqpoint{3.073825in}{0.498088in}}%
\pgfpathclose%
\pgfusepath{fill}%
\end{pgfscope}%
\begin{pgfscope}%
\pgfpathrectangle{\pgfqpoint{3.073825in}{0.498088in}}{\pgfqpoint{2.478947in}{1.802109in}}%
\pgfusepath{clip}%
\pgfsetbuttcap%
\pgfsetroundjoin%
\pgfsetlinewidth{1.003750pt}%
\definecolor{currentstroke}{rgb}{0.000000,0.000000,0.000000}%
\pgfsetstrokecolor{currentstroke}%
\pgfsetdash{}{0pt}%
\pgfpathmoveto{\pgfqpoint{4.191383in}{0.871087in}}%
\pgfpathlineto{\pgfqpoint{5.554773in}{0.871087in}}%
\pgfusepath{stroke}%
\end{pgfscope}%
\begin{pgfscope}%
\pgfpathrectangle{\pgfqpoint{3.073825in}{0.498088in}}{\pgfqpoint{2.478947in}{1.802109in}}%
\pgfusepath{clip}%
\pgfsetbuttcap%
\pgfsetroundjoin%
\pgfsetlinewidth{1.003750pt}%
\definecolor{currentstroke}{rgb}{0.000000,0.000000,0.000000}%
\pgfsetstrokecolor{currentstroke}%
\pgfsetdash{}{0pt}%
\pgfpathmoveto{\pgfqpoint{4.699364in}{0.769055in}}%
\pgfpathlineto{\pgfqpoint{5.554773in}{0.769055in}}%
\pgfusepath{stroke}%
\end{pgfscope}%
\begin{pgfscope}%
\pgfpathrectangle{\pgfqpoint{3.073825in}{0.498088in}}{\pgfqpoint{2.478947in}{1.802109in}}%
\pgfusepath{clip}%
\pgfsetbuttcap%
\pgfsetroundjoin%
\pgfsetlinewidth{1.003750pt}%
\definecolor{currentstroke}{rgb}{0.000000,0.000000,0.000000}%
\pgfsetstrokecolor{currentstroke}%
\pgfsetdash{}{0pt}%
\pgfpathmoveto{\pgfqpoint{4.191383in}{0.496088in}}%
\pgfpathlineto{\pgfqpoint{4.191383in}{0.871087in}}%
\pgfusepath{stroke}%
\end{pgfscope}%
\begin{pgfscope}%
\pgfpathrectangle{\pgfqpoint{3.073825in}{0.498088in}}{\pgfqpoint{2.478947in}{1.802109in}}%
\pgfusepath{clip}%
\pgfsetbuttcap%
\pgfsetroundjoin%
\pgfsetlinewidth{1.003750pt}%
\definecolor{currentstroke}{rgb}{0.000000,0.000000,0.000000}%
\pgfsetstrokecolor{currentstroke}%
\pgfsetdash{}{0pt}%
\pgfpathmoveto{\pgfqpoint{4.699364in}{0.496088in}}%
\pgfpathlineto{\pgfqpoint{4.699364in}{0.769055in}}%
\pgfusepath{stroke}%
\end{pgfscope}%
\begin{pgfscope}%
\pgfsetbuttcap%
\pgfsetroundjoin%
\definecolor{currentfill}{rgb}{0.000000,0.000000,0.000000}%
\pgfsetfillcolor{currentfill}%
\pgfsetlinewidth{0.803000pt}%
\definecolor{currentstroke}{rgb}{0.000000,0.000000,0.000000}%
\pgfsetstrokecolor{currentstroke}%
\pgfsetdash{}{0pt}%
\pgfsys@defobject{currentmarker}{\pgfqpoint{0.000000in}{-0.048611in}}{\pgfqpoint{0.000000in}{0.000000in}}{%
\pgfpathmoveto{\pgfqpoint{0.000000in}{0.000000in}}%
\pgfpathlineto{\pgfqpoint{0.000000in}{-0.048611in}}%
\pgfusepath{stroke,fill}%
}%
\begin{pgfscope}%
\pgfsys@transformshift{3.175421in}{0.498088in}%
\pgfsys@useobject{currentmarker}{}%
\end{pgfscope}%
\end{pgfscope}%
\begin{pgfscope}%
\definecolor{textcolor}{rgb}{0.000000,0.000000,0.000000}%
\pgfsetstrokecolor{textcolor}%
\pgfsetfillcolor{textcolor}%
\pgftext[x=3.175421in,y=0.400866in,,top]{\color{textcolor}\rmfamily\fontsize{10.000000}{12.000000}\selectfont \(\displaystyle {50}\)}%
\end{pgfscope}%
\begin{pgfscope}%
\pgfsetbuttcap%
\pgfsetroundjoin%
\definecolor{currentfill}{rgb}{0.000000,0.000000,0.000000}%
\pgfsetfillcolor{currentfill}%
\pgfsetlinewidth{0.803000pt}%
\definecolor{currentstroke}{rgb}{0.000000,0.000000,0.000000}%
\pgfsetstrokecolor{currentstroke}%
\pgfsetdash{}{0pt}%
\pgfsys@defobject{currentmarker}{\pgfqpoint{0.000000in}{-0.048611in}}{\pgfqpoint{0.000000in}{0.000000in}}{%
\pgfpathmoveto{\pgfqpoint{0.000000in}{0.000000in}}%
\pgfpathlineto{\pgfqpoint{0.000000in}{-0.048611in}}%
\pgfusepath{stroke,fill}%
}%
\begin{pgfscope}%
\pgfsys@transformshift{3.683402in}{0.498088in}%
\pgfsys@useobject{currentmarker}{}%
\end{pgfscope}%
\end{pgfscope}%
\begin{pgfscope}%
\definecolor{textcolor}{rgb}{0.000000,0.000000,0.000000}%
\pgfsetstrokecolor{textcolor}%
\pgfsetfillcolor{textcolor}%
\pgftext[x=3.683402in,y=0.400866in,,top]{\color{textcolor}\rmfamily\fontsize{10.000000}{12.000000}\selectfont \(\displaystyle {75}\)}%
\end{pgfscope}%
\begin{pgfscope}%
\pgfsetbuttcap%
\pgfsetroundjoin%
\definecolor{currentfill}{rgb}{0.000000,0.000000,0.000000}%
\pgfsetfillcolor{currentfill}%
\pgfsetlinewidth{0.803000pt}%
\definecolor{currentstroke}{rgb}{0.000000,0.000000,0.000000}%
\pgfsetstrokecolor{currentstroke}%
\pgfsetdash{}{0pt}%
\pgfsys@defobject{currentmarker}{\pgfqpoint{0.000000in}{-0.048611in}}{\pgfqpoint{0.000000in}{0.000000in}}{%
\pgfpathmoveto{\pgfqpoint{0.000000in}{0.000000in}}%
\pgfpathlineto{\pgfqpoint{0.000000in}{-0.048611in}}%
\pgfusepath{stroke,fill}%
}%
\begin{pgfscope}%
\pgfsys@transformshift{4.191383in}{0.498088in}%
\pgfsys@useobject{currentmarker}{}%
\end{pgfscope}%
\end{pgfscope}%
\begin{pgfscope}%
\definecolor{textcolor}{rgb}{0.000000,0.000000,0.000000}%
\pgfsetstrokecolor{textcolor}%
\pgfsetfillcolor{textcolor}%
\pgftext[x=4.191383in,y=0.400866in,,top]{\color{textcolor}\rmfamily\fontsize{10.000000}{12.000000}\selectfont \(\displaystyle {100}\)}%
\end{pgfscope}%
\begin{pgfscope}%
\pgfsetbuttcap%
\pgfsetroundjoin%
\definecolor{currentfill}{rgb}{0.000000,0.000000,0.000000}%
\pgfsetfillcolor{currentfill}%
\pgfsetlinewidth{0.803000pt}%
\definecolor{currentstroke}{rgb}{0.000000,0.000000,0.000000}%
\pgfsetstrokecolor{currentstroke}%
\pgfsetdash{}{0pt}%
\pgfsys@defobject{currentmarker}{\pgfqpoint{0.000000in}{-0.048611in}}{\pgfqpoint{0.000000in}{0.000000in}}{%
\pgfpathmoveto{\pgfqpoint{0.000000in}{0.000000in}}%
\pgfpathlineto{\pgfqpoint{0.000000in}{-0.048611in}}%
\pgfusepath{stroke,fill}%
}%
\begin{pgfscope}%
\pgfsys@transformshift{4.699364in}{0.498088in}%
\pgfsys@useobject{currentmarker}{}%
\end{pgfscope}%
\end{pgfscope}%
\begin{pgfscope}%
\definecolor{textcolor}{rgb}{0.000000,0.000000,0.000000}%
\pgfsetstrokecolor{textcolor}%
\pgfsetfillcolor{textcolor}%
\pgftext[x=4.699364in,y=0.400866in,,top]{\color{textcolor}\rmfamily\fontsize{10.000000}{12.000000}\selectfont \(\displaystyle {125}\)}%
\end{pgfscope}%
\begin{pgfscope}%
\pgfsetbuttcap%
\pgfsetroundjoin%
\definecolor{currentfill}{rgb}{0.000000,0.000000,0.000000}%
\pgfsetfillcolor{currentfill}%
\pgfsetlinewidth{0.803000pt}%
\definecolor{currentstroke}{rgb}{0.000000,0.000000,0.000000}%
\pgfsetstrokecolor{currentstroke}%
\pgfsetdash{}{0pt}%
\pgfsys@defobject{currentmarker}{\pgfqpoint{0.000000in}{-0.048611in}}{\pgfqpoint{0.000000in}{0.000000in}}{%
\pgfpathmoveto{\pgfqpoint{0.000000in}{0.000000in}}%
\pgfpathlineto{\pgfqpoint{0.000000in}{-0.048611in}}%
\pgfusepath{stroke,fill}%
}%
\begin{pgfscope}%
\pgfsys@transformshift{5.207346in}{0.498088in}%
\pgfsys@useobject{currentmarker}{}%
\end{pgfscope}%
\end{pgfscope}%
\begin{pgfscope}%
\definecolor{textcolor}{rgb}{0.000000,0.000000,0.000000}%
\pgfsetstrokecolor{textcolor}%
\pgfsetfillcolor{textcolor}%
\pgftext[x=5.207346in,y=0.400866in,,top]{\color{textcolor}\rmfamily\fontsize{10.000000}{12.000000}\selectfont \(\displaystyle {150}\)}%
\end{pgfscope}%
\begin{pgfscope}%
\pgfsetbuttcap%
\pgfsetroundjoin%
\definecolor{currentfill}{rgb}{0.000000,0.000000,0.000000}%
\pgfsetfillcolor{currentfill}%
\pgfsetlinewidth{0.602250pt}%
\definecolor{currentstroke}{rgb}{0.000000,0.000000,0.000000}%
\pgfsetstrokecolor{currentstroke}%
\pgfsetdash{}{0pt}%
\pgfsys@defobject{currentmarker}{\pgfqpoint{0.000000in}{-0.027778in}}{\pgfqpoint{0.000000in}{0.000000in}}{%
\pgfpathmoveto{\pgfqpoint{0.000000in}{0.000000in}}%
\pgfpathlineto{\pgfqpoint{0.000000in}{-0.027778in}}%
\pgfusepath{stroke,fill}%
}%
\begin{pgfscope}%
\pgfsys@transformshift{3.073825in}{0.498088in}%
\pgfsys@useobject{currentmarker}{}%
\end{pgfscope}%
\end{pgfscope}%
\begin{pgfscope}%
\pgfsetbuttcap%
\pgfsetroundjoin%
\definecolor{currentfill}{rgb}{0.000000,0.000000,0.000000}%
\pgfsetfillcolor{currentfill}%
\pgfsetlinewidth{0.602250pt}%
\definecolor{currentstroke}{rgb}{0.000000,0.000000,0.000000}%
\pgfsetstrokecolor{currentstroke}%
\pgfsetdash{}{0pt}%
\pgfsys@defobject{currentmarker}{\pgfqpoint{0.000000in}{-0.027778in}}{\pgfqpoint{0.000000in}{0.000000in}}{%
\pgfpathmoveto{\pgfqpoint{0.000000in}{0.000000in}}%
\pgfpathlineto{\pgfqpoint{0.000000in}{-0.027778in}}%
\pgfusepath{stroke,fill}%
}%
\begin{pgfscope}%
\pgfsys@transformshift{3.277018in}{0.498088in}%
\pgfsys@useobject{currentmarker}{}%
\end{pgfscope}%
\end{pgfscope}%
\begin{pgfscope}%
\pgfsetbuttcap%
\pgfsetroundjoin%
\definecolor{currentfill}{rgb}{0.000000,0.000000,0.000000}%
\pgfsetfillcolor{currentfill}%
\pgfsetlinewidth{0.602250pt}%
\definecolor{currentstroke}{rgb}{0.000000,0.000000,0.000000}%
\pgfsetstrokecolor{currentstroke}%
\pgfsetdash{}{0pt}%
\pgfsys@defobject{currentmarker}{\pgfqpoint{0.000000in}{-0.027778in}}{\pgfqpoint{0.000000in}{0.000000in}}{%
\pgfpathmoveto{\pgfqpoint{0.000000in}{0.000000in}}%
\pgfpathlineto{\pgfqpoint{0.000000in}{-0.027778in}}%
\pgfusepath{stroke,fill}%
}%
\begin{pgfscope}%
\pgfsys@transformshift{3.378614in}{0.498088in}%
\pgfsys@useobject{currentmarker}{}%
\end{pgfscope}%
\end{pgfscope}%
\begin{pgfscope}%
\pgfsetbuttcap%
\pgfsetroundjoin%
\definecolor{currentfill}{rgb}{0.000000,0.000000,0.000000}%
\pgfsetfillcolor{currentfill}%
\pgfsetlinewidth{0.602250pt}%
\definecolor{currentstroke}{rgb}{0.000000,0.000000,0.000000}%
\pgfsetstrokecolor{currentstroke}%
\pgfsetdash{}{0pt}%
\pgfsys@defobject{currentmarker}{\pgfqpoint{0.000000in}{-0.027778in}}{\pgfqpoint{0.000000in}{0.000000in}}{%
\pgfpathmoveto{\pgfqpoint{0.000000in}{0.000000in}}%
\pgfpathlineto{\pgfqpoint{0.000000in}{-0.027778in}}%
\pgfusepath{stroke,fill}%
}%
\begin{pgfscope}%
\pgfsys@transformshift{3.480210in}{0.498088in}%
\pgfsys@useobject{currentmarker}{}%
\end{pgfscope}%
\end{pgfscope}%
\begin{pgfscope}%
\pgfsetbuttcap%
\pgfsetroundjoin%
\definecolor{currentfill}{rgb}{0.000000,0.000000,0.000000}%
\pgfsetfillcolor{currentfill}%
\pgfsetlinewidth{0.602250pt}%
\definecolor{currentstroke}{rgb}{0.000000,0.000000,0.000000}%
\pgfsetstrokecolor{currentstroke}%
\pgfsetdash{}{0pt}%
\pgfsys@defobject{currentmarker}{\pgfqpoint{0.000000in}{-0.027778in}}{\pgfqpoint{0.000000in}{0.000000in}}{%
\pgfpathmoveto{\pgfqpoint{0.000000in}{0.000000in}}%
\pgfpathlineto{\pgfqpoint{0.000000in}{-0.027778in}}%
\pgfusepath{stroke,fill}%
}%
\begin{pgfscope}%
\pgfsys@transformshift{3.581806in}{0.498088in}%
\pgfsys@useobject{currentmarker}{}%
\end{pgfscope}%
\end{pgfscope}%
\begin{pgfscope}%
\pgfsetbuttcap%
\pgfsetroundjoin%
\definecolor{currentfill}{rgb}{0.000000,0.000000,0.000000}%
\pgfsetfillcolor{currentfill}%
\pgfsetlinewidth{0.602250pt}%
\definecolor{currentstroke}{rgb}{0.000000,0.000000,0.000000}%
\pgfsetstrokecolor{currentstroke}%
\pgfsetdash{}{0pt}%
\pgfsys@defobject{currentmarker}{\pgfqpoint{0.000000in}{-0.027778in}}{\pgfqpoint{0.000000in}{0.000000in}}{%
\pgfpathmoveto{\pgfqpoint{0.000000in}{0.000000in}}%
\pgfpathlineto{\pgfqpoint{0.000000in}{-0.027778in}}%
\pgfusepath{stroke,fill}%
}%
\begin{pgfscope}%
\pgfsys@transformshift{3.784999in}{0.498088in}%
\pgfsys@useobject{currentmarker}{}%
\end{pgfscope}%
\end{pgfscope}%
\begin{pgfscope}%
\pgfsetbuttcap%
\pgfsetroundjoin%
\definecolor{currentfill}{rgb}{0.000000,0.000000,0.000000}%
\pgfsetfillcolor{currentfill}%
\pgfsetlinewidth{0.602250pt}%
\definecolor{currentstroke}{rgb}{0.000000,0.000000,0.000000}%
\pgfsetstrokecolor{currentstroke}%
\pgfsetdash{}{0pt}%
\pgfsys@defobject{currentmarker}{\pgfqpoint{0.000000in}{-0.027778in}}{\pgfqpoint{0.000000in}{0.000000in}}{%
\pgfpathmoveto{\pgfqpoint{0.000000in}{0.000000in}}%
\pgfpathlineto{\pgfqpoint{0.000000in}{-0.027778in}}%
\pgfusepath{stroke,fill}%
}%
\begin{pgfscope}%
\pgfsys@transformshift{3.886595in}{0.498088in}%
\pgfsys@useobject{currentmarker}{}%
\end{pgfscope}%
\end{pgfscope}%
\begin{pgfscope}%
\pgfsetbuttcap%
\pgfsetroundjoin%
\definecolor{currentfill}{rgb}{0.000000,0.000000,0.000000}%
\pgfsetfillcolor{currentfill}%
\pgfsetlinewidth{0.602250pt}%
\definecolor{currentstroke}{rgb}{0.000000,0.000000,0.000000}%
\pgfsetstrokecolor{currentstroke}%
\pgfsetdash{}{0pt}%
\pgfsys@defobject{currentmarker}{\pgfqpoint{0.000000in}{-0.027778in}}{\pgfqpoint{0.000000in}{0.000000in}}{%
\pgfpathmoveto{\pgfqpoint{0.000000in}{0.000000in}}%
\pgfpathlineto{\pgfqpoint{0.000000in}{-0.027778in}}%
\pgfusepath{stroke,fill}%
}%
\begin{pgfscope}%
\pgfsys@transformshift{3.988191in}{0.498088in}%
\pgfsys@useobject{currentmarker}{}%
\end{pgfscope}%
\end{pgfscope}%
\begin{pgfscope}%
\pgfsetbuttcap%
\pgfsetroundjoin%
\definecolor{currentfill}{rgb}{0.000000,0.000000,0.000000}%
\pgfsetfillcolor{currentfill}%
\pgfsetlinewidth{0.602250pt}%
\definecolor{currentstroke}{rgb}{0.000000,0.000000,0.000000}%
\pgfsetstrokecolor{currentstroke}%
\pgfsetdash{}{0pt}%
\pgfsys@defobject{currentmarker}{\pgfqpoint{0.000000in}{-0.027778in}}{\pgfqpoint{0.000000in}{0.000000in}}{%
\pgfpathmoveto{\pgfqpoint{0.000000in}{0.000000in}}%
\pgfpathlineto{\pgfqpoint{0.000000in}{-0.027778in}}%
\pgfusepath{stroke,fill}%
}%
\begin{pgfscope}%
\pgfsys@transformshift{4.089787in}{0.498088in}%
\pgfsys@useobject{currentmarker}{}%
\end{pgfscope}%
\end{pgfscope}%
\begin{pgfscope}%
\pgfsetbuttcap%
\pgfsetroundjoin%
\definecolor{currentfill}{rgb}{0.000000,0.000000,0.000000}%
\pgfsetfillcolor{currentfill}%
\pgfsetlinewidth{0.602250pt}%
\definecolor{currentstroke}{rgb}{0.000000,0.000000,0.000000}%
\pgfsetstrokecolor{currentstroke}%
\pgfsetdash{}{0pt}%
\pgfsys@defobject{currentmarker}{\pgfqpoint{0.000000in}{-0.027778in}}{\pgfqpoint{0.000000in}{0.000000in}}{%
\pgfpathmoveto{\pgfqpoint{0.000000in}{0.000000in}}%
\pgfpathlineto{\pgfqpoint{0.000000in}{-0.027778in}}%
\pgfusepath{stroke,fill}%
}%
\begin{pgfscope}%
\pgfsys@transformshift{4.292980in}{0.498088in}%
\pgfsys@useobject{currentmarker}{}%
\end{pgfscope}%
\end{pgfscope}%
\begin{pgfscope}%
\pgfsetbuttcap%
\pgfsetroundjoin%
\definecolor{currentfill}{rgb}{0.000000,0.000000,0.000000}%
\pgfsetfillcolor{currentfill}%
\pgfsetlinewidth{0.602250pt}%
\definecolor{currentstroke}{rgb}{0.000000,0.000000,0.000000}%
\pgfsetstrokecolor{currentstroke}%
\pgfsetdash{}{0pt}%
\pgfsys@defobject{currentmarker}{\pgfqpoint{0.000000in}{-0.027778in}}{\pgfqpoint{0.000000in}{0.000000in}}{%
\pgfpathmoveto{\pgfqpoint{0.000000in}{0.000000in}}%
\pgfpathlineto{\pgfqpoint{0.000000in}{-0.027778in}}%
\pgfusepath{stroke,fill}%
}%
\begin{pgfscope}%
\pgfsys@transformshift{4.394576in}{0.498088in}%
\pgfsys@useobject{currentmarker}{}%
\end{pgfscope}%
\end{pgfscope}%
\begin{pgfscope}%
\pgfsetbuttcap%
\pgfsetroundjoin%
\definecolor{currentfill}{rgb}{0.000000,0.000000,0.000000}%
\pgfsetfillcolor{currentfill}%
\pgfsetlinewidth{0.602250pt}%
\definecolor{currentstroke}{rgb}{0.000000,0.000000,0.000000}%
\pgfsetstrokecolor{currentstroke}%
\pgfsetdash{}{0pt}%
\pgfsys@defobject{currentmarker}{\pgfqpoint{0.000000in}{-0.027778in}}{\pgfqpoint{0.000000in}{0.000000in}}{%
\pgfpathmoveto{\pgfqpoint{0.000000in}{0.000000in}}%
\pgfpathlineto{\pgfqpoint{0.000000in}{-0.027778in}}%
\pgfusepath{stroke,fill}%
}%
\begin{pgfscope}%
\pgfsys@transformshift{4.496172in}{0.498088in}%
\pgfsys@useobject{currentmarker}{}%
\end{pgfscope}%
\end{pgfscope}%
\begin{pgfscope}%
\pgfsetbuttcap%
\pgfsetroundjoin%
\definecolor{currentfill}{rgb}{0.000000,0.000000,0.000000}%
\pgfsetfillcolor{currentfill}%
\pgfsetlinewidth{0.602250pt}%
\definecolor{currentstroke}{rgb}{0.000000,0.000000,0.000000}%
\pgfsetstrokecolor{currentstroke}%
\pgfsetdash{}{0pt}%
\pgfsys@defobject{currentmarker}{\pgfqpoint{0.000000in}{-0.027778in}}{\pgfqpoint{0.000000in}{0.000000in}}{%
\pgfpathmoveto{\pgfqpoint{0.000000in}{0.000000in}}%
\pgfpathlineto{\pgfqpoint{0.000000in}{-0.027778in}}%
\pgfusepath{stroke,fill}%
}%
\begin{pgfscope}%
\pgfsys@transformshift{4.597768in}{0.498088in}%
\pgfsys@useobject{currentmarker}{}%
\end{pgfscope}%
\end{pgfscope}%
\begin{pgfscope}%
\pgfsetbuttcap%
\pgfsetroundjoin%
\definecolor{currentfill}{rgb}{0.000000,0.000000,0.000000}%
\pgfsetfillcolor{currentfill}%
\pgfsetlinewidth{0.602250pt}%
\definecolor{currentstroke}{rgb}{0.000000,0.000000,0.000000}%
\pgfsetstrokecolor{currentstroke}%
\pgfsetdash{}{0pt}%
\pgfsys@defobject{currentmarker}{\pgfqpoint{0.000000in}{-0.027778in}}{\pgfqpoint{0.000000in}{0.000000in}}{%
\pgfpathmoveto{\pgfqpoint{0.000000in}{0.000000in}}%
\pgfpathlineto{\pgfqpoint{0.000000in}{-0.027778in}}%
\pgfusepath{stroke,fill}%
}%
\begin{pgfscope}%
\pgfsys@transformshift{4.800961in}{0.498088in}%
\pgfsys@useobject{currentmarker}{}%
\end{pgfscope}%
\end{pgfscope}%
\begin{pgfscope}%
\pgfsetbuttcap%
\pgfsetroundjoin%
\definecolor{currentfill}{rgb}{0.000000,0.000000,0.000000}%
\pgfsetfillcolor{currentfill}%
\pgfsetlinewidth{0.602250pt}%
\definecolor{currentstroke}{rgb}{0.000000,0.000000,0.000000}%
\pgfsetstrokecolor{currentstroke}%
\pgfsetdash{}{0pt}%
\pgfsys@defobject{currentmarker}{\pgfqpoint{0.000000in}{-0.027778in}}{\pgfqpoint{0.000000in}{0.000000in}}{%
\pgfpathmoveto{\pgfqpoint{0.000000in}{0.000000in}}%
\pgfpathlineto{\pgfqpoint{0.000000in}{-0.027778in}}%
\pgfusepath{stroke,fill}%
}%
\begin{pgfscope}%
\pgfsys@transformshift{4.902557in}{0.498088in}%
\pgfsys@useobject{currentmarker}{}%
\end{pgfscope}%
\end{pgfscope}%
\begin{pgfscope}%
\pgfsetbuttcap%
\pgfsetroundjoin%
\definecolor{currentfill}{rgb}{0.000000,0.000000,0.000000}%
\pgfsetfillcolor{currentfill}%
\pgfsetlinewidth{0.602250pt}%
\definecolor{currentstroke}{rgb}{0.000000,0.000000,0.000000}%
\pgfsetstrokecolor{currentstroke}%
\pgfsetdash{}{0pt}%
\pgfsys@defobject{currentmarker}{\pgfqpoint{0.000000in}{-0.027778in}}{\pgfqpoint{0.000000in}{0.000000in}}{%
\pgfpathmoveto{\pgfqpoint{0.000000in}{0.000000in}}%
\pgfpathlineto{\pgfqpoint{0.000000in}{-0.027778in}}%
\pgfusepath{stroke,fill}%
}%
\begin{pgfscope}%
\pgfsys@transformshift{5.004153in}{0.498088in}%
\pgfsys@useobject{currentmarker}{}%
\end{pgfscope}%
\end{pgfscope}%
\begin{pgfscope}%
\pgfsetbuttcap%
\pgfsetroundjoin%
\definecolor{currentfill}{rgb}{0.000000,0.000000,0.000000}%
\pgfsetfillcolor{currentfill}%
\pgfsetlinewidth{0.602250pt}%
\definecolor{currentstroke}{rgb}{0.000000,0.000000,0.000000}%
\pgfsetstrokecolor{currentstroke}%
\pgfsetdash{}{0pt}%
\pgfsys@defobject{currentmarker}{\pgfqpoint{0.000000in}{-0.027778in}}{\pgfqpoint{0.000000in}{0.000000in}}{%
\pgfpathmoveto{\pgfqpoint{0.000000in}{0.000000in}}%
\pgfpathlineto{\pgfqpoint{0.000000in}{-0.027778in}}%
\pgfusepath{stroke,fill}%
}%
\begin{pgfscope}%
\pgfsys@transformshift{5.105749in}{0.498088in}%
\pgfsys@useobject{currentmarker}{}%
\end{pgfscope}%
\end{pgfscope}%
\begin{pgfscope}%
\pgfsetbuttcap%
\pgfsetroundjoin%
\definecolor{currentfill}{rgb}{0.000000,0.000000,0.000000}%
\pgfsetfillcolor{currentfill}%
\pgfsetlinewidth{0.602250pt}%
\definecolor{currentstroke}{rgb}{0.000000,0.000000,0.000000}%
\pgfsetstrokecolor{currentstroke}%
\pgfsetdash{}{0pt}%
\pgfsys@defobject{currentmarker}{\pgfqpoint{0.000000in}{-0.027778in}}{\pgfqpoint{0.000000in}{0.000000in}}{%
\pgfpathmoveto{\pgfqpoint{0.000000in}{0.000000in}}%
\pgfpathlineto{\pgfqpoint{0.000000in}{-0.027778in}}%
\pgfusepath{stroke,fill}%
}%
\begin{pgfscope}%
\pgfsys@transformshift{5.308942in}{0.498088in}%
\pgfsys@useobject{currentmarker}{}%
\end{pgfscope}%
\end{pgfscope}%
\begin{pgfscope}%
\pgfsetbuttcap%
\pgfsetroundjoin%
\definecolor{currentfill}{rgb}{0.000000,0.000000,0.000000}%
\pgfsetfillcolor{currentfill}%
\pgfsetlinewidth{0.602250pt}%
\definecolor{currentstroke}{rgb}{0.000000,0.000000,0.000000}%
\pgfsetstrokecolor{currentstroke}%
\pgfsetdash{}{0pt}%
\pgfsys@defobject{currentmarker}{\pgfqpoint{0.000000in}{-0.027778in}}{\pgfqpoint{0.000000in}{0.000000in}}{%
\pgfpathmoveto{\pgfqpoint{0.000000in}{0.000000in}}%
\pgfpathlineto{\pgfqpoint{0.000000in}{-0.027778in}}%
\pgfusepath{stroke,fill}%
}%
\begin{pgfscope}%
\pgfsys@transformshift{5.410538in}{0.498088in}%
\pgfsys@useobject{currentmarker}{}%
\end{pgfscope}%
\end{pgfscope}%
\begin{pgfscope}%
\pgfsetbuttcap%
\pgfsetroundjoin%
\definecolor{currentfill}{rgb}{0.000000,0.000000,0.000000}%
\pgfsetfillcolor{currentfill}%
\pgfsetlinewidth{0.602250pt}%
\definecolor{currentstroke}{rgb}{0.000000,0.000000,0.000000}%
\pgfsetstrokecolor{currentstroke}%
\pgfsetdash{}{0pt}%
\pgfsys@defobject{currentmarker}{\pgfqpoint{0.000000in}{-0.027778in}}{\pgfqpoint{0.000000in}{0.000000in}}{%
\pgfpathmoveto{\pgfqpoint{0.000000in}{0.000000in}}%
\pgfpathlineto{\pgfqpoint{0.000000in}{-0.027778in}}%
\pgfusepath{stroke,fill}%
}%
\begin{pgfscope}%
\pgfsys@transformshift{5.512134in}{0.498088in}%
\pgfsys@useobject{currentmarker}{}%
\end{pgfscope}%
\end{pgfscope}%
\begin{pgfscope}%
\definecolor{textcolor}{rgb}{0.000000,0.000000,0.000000}%
\pgfsetstrokecolor{textcolor}%
\pgfsetfillcolor{textcolor}%
\pgftext[x=4.313299in,y=0.222655in,,top]{\color{textcolor}\rmfamily\fontsize{10.000000}{12.000000}\selectfont Schwellenwert \(\displaystyle s_V\) in ADU}%
\end{pgfscope}%
\begin{pgfscope}%
\pgfsetbuttcap%
\pgfsetroundjoin%
\definecolor{currentfill}{rgb}{0.000000,0.000000,0.000000}%
\pgfsetfillcolor{currentfill}%
\pgfsetlinewidth{0.803000pt}%
\definecolor{currentstroke}{rgb}{0.000000,0.000000,0.000000}%
\pgfsetstrokecolor{currentstroke}%
\pgfsetdash{}{0pt}%
\pgfsys@defobject{currentmarker}{\pgfqpoint{0.000000in}{0.000000in}}{\pgfqpoint{0.048611in}{0.000000in}}{%
\pgfpathmoveto{\pgfqpoint{0.000000in}{0.000000in}}%
\pgfpathlineto{\pgfqpoint{0.048611in}{0.000000in}}%
\pgfusepath{stroke,fill}%
}%
\begin{pgfscope}%
\pgfsys@transformshift{5.552773in}{0.551882in}%
\pgfsys@useobject{currentmarker}{}%
\end{pgfscope}%
\end{pgfscope}%
\begin{pgfscope}%
\definecolor{textcolor}{rgb}{0.000000,0.000000,0.000000}%
\pgfsetstrokecolor{textcolor}%
\pgfsetfillcolor{textcolor}%
\pgftext[x=5.649995in, y=0.504055in, left, base]{\color{textcolor}\rmfamily\fontsize{10.000000}{12.000000}\selectfont \(\displaystyle {0}\)}%
\end{pgfscope}%
\begin{pgfscope}%
\pgfsetbuttcap%
\pgfsetroundjoin%
\definecolor{currentfill}{rgb}{0.000000,0.000000,0.000000}%
\pgfsetfillcolor{currentfill}%
\pgfsetlinewidth{0.803000pt}%
\definecolor{currentstroke}{rgb}{0.000000,0.000000,0.000000}%
\pgfsetstrokecolor{currentstroke}%
\pgfsetdash{}{0pt}%
\pgfsys@defobject{currentmarker}{\pgfqpoint{0.000000in}{0.000000in}}{\pgfqpoint{0.048611in}{0.000000in}}{%
\pgfpathmoveto{\pgfqpoint{0.000000in}{0.000000in}}%
\pgfpathlineto{\pgfqpoint{0.048611in}{0.000000in}}%
\pgfusepath{stroke,fill}%
}%
\begin{pgfscope}%
\pgfsys@transformshift{5.552773in}{1.089825in}%
\pgfsys@useobject{currentmarker}{}%
\end{pgfscope}%
\end{pgfscope}%
\begin{pgfscope}%
\definecolor{textcolor}{rgb}{0.000000,0.000000,0.000000}%
\pgfsetstrokecolor{textcolor}%
\pgfsetfillcolor{textcolor}%
\pgftext[x=5.649995in, y=1.041997in, left, base]{\color{textcolor}\rmfamily\fontsize{10.000000}{12.000000}\selectfont \(\displaystyle {2}\)}%
\end{pgfscope}%
\begin{pgfscope}%
\pgfsetbuttcap%
\pgfsetroundjoin%
\definecolor{currentfill}{rgb}{0.000000,0.000000,0.000000}%
\pgfsetfillcolor{currentfill}%
\pgfsetlinewidth{0.803000pt}%
\definecolor{currentstroke}{rgb}{0.000000,0.000000,0.000000}%
\pgfsetstrokecolor{currentstroke}%
\pgfsetdash{}{0pt}%
\pgfsys@defobject{currentmarker}{\pgfqpoint{0.000000in}{0.000000in}}{\pgfqpoint{0.048611in}{0.000000in}}{%
\pgfpathmoveto{\pgfqpoint{0.000000in}{0.000000in}}%
\pgfpathlineto{\pgfqpoint{0.048611in}{0.000000in}}%
\pgfusepath{stroke,fill}%
}%
\begin{pgfscope}%
\pgfsys@transformshift{5.552773in}{1.627768in}%
\pgfsys@useobject{currentmarker}{}%
\end{pgfscope}%
\end{pgfscope}%
\begin{pgfscope}%
\definecolor{textcolor}{rgb}{0.000000,0.000000,0.000000}%
\pgfsetstrokecolor{textcolor}%
\pgfsetfillcolor{textcolor}%
\pgftext[x=5.649995in, y=1.579940in, left, base]{\color{textcolor}\rmfamily\fontsize{10.000000}{12.000000}\selectfont \(\displaystyle {4}\)}%
\end{pgfscope}%
\begin{pgfscope}%
\pgfsetbuttcap%
\pgfsetroundjoin%
\definecolor{currentfill}{rgb}{0.000000,0.000000,0.000000}%
\pgfsetfillcolor{currentfill}%
\pgfsetlinewidth{0.803000pt}%
\definecolor{currentstroke}{rgb}{0.000000,0.000000,0.000000}%
\pgfsetstrokecolor{currentstroke}%
\pgfsetdash{}{0pt}%
\pgfsys@defobject{currentmarker}{\pgfqpoint{0.000000in}{0.000000in}}{\pgfqpoint{0.048611in}{0.000000in}}{%
\pgfpathmoveto{\pgfqpoint{0.000000in}{0.000000in}}%
\pgfpathlineto{\pgfqpoint{0.048611in}{0.000000in}}%
\pgfusepath{stroke,fill}%
}%
\begin{pgfscope}%
\pgfsys@transformshift{5.552773in}{2.165711in}%
\pgfsys@useobject{currentmarker}{}%
\end{pgfscope}%
\end{pgfscope}%
\begin{pgfscope}%
\definecolor{textcolor}{rgb}{0.000000,0.000000,0.000000}%
\pgfsetstrokecolor{textcolor}%
\pgfsetfillcolor{textcolor}%
\pgftext[x=5.649995in, y=2.117883in, left, base]{\color{textcolor}\rmfamily\fontsize{10.000000}{12.000000}\selectfont \(\displaystyle {6}\)}%
\end{pgfscope}%
\begin{pgfscope}%
\pgfsetbuttcap%
\pgfsetroundjoin%
\definecolor{currentfill}{rgb}{0.000000,0.000000,0.000000}%
\pgfsetfillcolor{currentfill}%
\pgfsetlinewidth{0.602250pt}%
\definecolor{currentstroke}{rgb}{0.000000,0.000000,0.000000}%
\pgfsetstrokecolor{currentstroke}%
\pgfsetdash{}{0pt}%
\pgfsys@defobject{currentmarker}{\pgfqpoint{0.000000in}{0.000000in}}{\pgfqpoint{0.027778in}{0.000000in}}{%
\pgfpathmoveto{\pgfqpoint{0.000000in}{0.000000in}}%
\pgfpathlineto{\pgfqpoint{0.027778in}{0.000000in}}%
\pgfusepath{stroke,fill}%
}%
\begin{pgfscope}%
\pgfsys@transformshift{5.552773in}{0.686368in}%
\pgfsys@useobject{currentmarker}{}%
\end{pgfscope}%
\end{pgfscope}%
\begin{pgfscope}%
\pgfsetbuttcap%
\pgfsetroundjoin%
\definecolor{currentfill}{rgb}{0.000000,0.000000,0.000000}%
\pgfsetfillcolor{currentfill}%
\pgfsetlinewidth{0.602250pt}%
\definecolor{currentstroke}{rgb}{0.000000,0.000000,0.000000}%
\pgfsetstrokecolor{currentstroke}%
\pgfsetdash{}{0pt}%
\pgfsys@defobject{currentmarker}{\pgfqpoint{0.000000in}{0.000000in}}{\pgfqpoint{0.027778in}{0.000000in}}{%
\pgfpathmoveto{\pgfqpoint{0.000000in}{0.000000in}}%
\pgfpathlineto{\pgfqpoint{0.027778in}{0.000000in}}%
\pgfusepath{stroke,fill}%
}%
\begin{pgfscope}%
\pgfsys@transformshift{5.552773in}{0.820854in}%
\pgfsys@useobject{currentmarker}{}%
\end{pgfscope}%
\end{pgfscope}%
\begin{pgfscope}%
\pgfsetbuttcap%
\pgfsetroundjoin%
\definecolor{currentfill}{rgb}{0.000000,0.000000,0.000000}%
\pgfsetfillcolor{currentfill}%
\pgfsetlinewidth{0.602250pt}%
\definecolor{currentstroke}{rgb}{0.000000,0.000000,0.000000}%
\pgfsetstrokecolor{currentstroke}%
\pgfsetdash{}{0pt}%
\pgfsys@defobject{currentmarker}{\pgfqpoint{0.000000in}{0.000000in}}{\pgfqpoint{0.027778in}{0.000000in}}{%
\pgfpathmoveto{\pgfqpoint{0.000000in}{0.000000in}}%
\pgfpathlineto{\pgfqpoint{0.027778in}{0.000000in}}%
\pgfusepath{stroke,fill}%
}%
\begin{pgfscope}%
\pgfsys@transformshift{5.552773in}{0.955339in}%
\pgfsys@useobject{currentmarker}{}%
\end{pgfscope}%
\end{pgfscope}%
\begin{pgfscope}%
\pgfsetbuttcap%
\pgfsetroundjoin%
\definecolor{currentfill}{rgb}{0.000000,0.000000,0.000000}%
\pgfsetfillcolor{currentfill}%
\pgfsetlinewidth{0.602250pt}%
\definecolor{currentstroke}{rgb}{0.000000,0.000000,0.000000}%
\pgfsetstrokecolor{currentstroke}%
\pgfsetdash{}{0pt}%
\pgfsys@defobject{currentmarker}{\pgfqpoint{0.000000in}{0.000000in}}{\pgfqpoint{0.027778in}{0.000000in}}{%
\pgfpathmoveto{\pgfqpoint{0.000000in}{0.000000in}}%
\pgfpathlineto{\pgfqpoint{0.027778in}{0.000000in}}%
\pgfusepath{stroke,fill}%
}%
\begin{pgfscope}%
\pgfsys@transformshift{5.552773in}{1.224311in}%
\pgfsys@useobject{currentmarker}{}%
\end{pgfscope}%
\end{pgfscope}%
\begin{pgfscope}%
\pgfsetbuttcap%
\pgfsetroundjoin%
\definecolor{currentfill}{rgb}{0.000000,0.000000,0.000000}%
\pgfsetfillcolor{currentfill}%
\pgfsetlinewidth{0.602250pt}%
\definecolor{currentstroke}{rgb}{0.000000,0.000000,0.000000}%
\pgfsetstrokecolor{currentstroke}%
\pgfsetdash{}{0pt}%
\pgfsys@defobject{currentmarker}{\pgfqpoint{0.000000in}{0.000000in}}{\pgfqpoint{0.027778in}{0.000000in}}{%
\pgfpathmoveto{\pgfqpoint{0.000000in}{0.000000in}}%
\pgfpathlineto{\pgfqpoint{0.027778in}{0.000000in}}%
\pgfusepath{stroke,fill}%
}%
\begin{pgfscope}%
\pgfsys@transformshift{5.552773in}{1.358797in}%
\pgfsys@useobject{currentmarker}{}%
\end{pgfscope}%
\end{pgfscope}%
\begin{pgfscope}%
\pgfsetbuttcap%
\pgfsetroundjoin%
\definecolor{currentfill}{rgb}{0.000000,0.000000,0.000000}%
\pgfsetfillcolor{currentfill}%
\pgfsetlinewidth{0.602250pt}%
\definecolor{currentstroke}{rgb}{0.000000,0.000000,0.000000}%
\pgfsetstrokecolor{currentstroke}%
\pgfsetdash{}{0pt}%
\pgfsys@defobject{currentmarker}{\pgfqpoint{0.000000in}{0.000000in}}{\pgfqpoint{0.027778in}{0.000000in}}{%
\pgfpathmoveto{\pgfqpoint{0.000000in}{0.000000in}}%
\pgfpathlineto{\pgfqpoint{0.027778in}{0.000000in}}%
\pgfusepath{stroke,fill}%
}%
\begin{pgfscope}%
\pgfsys@transformshift{5.552773in}{1.493282in}%
\pgfsys@useobject{currentmarker}{}%
\end{pgfscope}%
\end{pgfscope}%
\begin{pgfscope}%
\pgfsetbuttcap%
\pgfsetroundjoin%
\definecolor{currentfill}{rgb}{0.000000,0.000000,0.000000}%
\pgfsetfillcolor{currentfill}%
\pgfsetlinewidth{0.602250pt}%
\definecolor{currentstroke}{rgb}{0.000000,0.000000,0.000000}%
\pgfsetstrokecolor{currentstroke}%
\pgfsetdash{}{0pt}%
\pgfsys@defobject{currentmarker}{\pgfqpoint{0.000000in}{0.000000in}}{\pgfqpoint{0.027778in}{0.000000in}}{%
\pgfpathmoveto{\pgfqpoint{0.000000in}{0.000000in}}%
\pgfpathlineto{\pgfqpoint{0.027778in}{0.000000in}}%
\pgfusepath{stroke,fill}%
}%
\begin{pgfscope}%
\pgfsys@transformshift{5.552773in}{1.762254in}%
\pgfsys@useobject{currentmarker}{}%
\end{pgfscope}%
\end{pgfscope}%
\begin{pgfscope}%
\pgfsetbuttcap%
\pgfsetroundjoin%
\definecolor{currentfill}{rgb}{0.000000,0.000000,0.000000}%
\pgfsetfillcolor{currentfill}%
\pgfsetlinewidth{0.602250pt}%
\definecolor{currentstroke}{rgb}{0.000000,0.000000,0.000000}%
\pgfsetstrokecolor{currentstroke}%
\pgfsetdash{}{0pt}%
\pgfsys@defobject{currentmarker}{\pgfqpoint{0.000000in}{0.000000in}}{\pgfqpoint{0.027778in}{0.000000in}}{%
\pgfpathmoveto{\pgfqpoint{0.000000in}{0.000000in}}%
\pgfpathlineto{\pgfqpoint{0.027778in}{0.000000in}}%
\pgfusepath{stroke,fill}%
}%
\begin{pgfscope}%
\pgfsys@transformshift{5.552773in}{1.896739in}%
\pgfsys@useobject{currentmarker}{}%
\end{pgfscope}%
\end{pgfscope}%
\begin{pgfscope}%
\pgfsetbuttcap%
\pgfsetroundjoin%
\definecolor{currentfill}{rgb}{0.000000,0.000000,0.000000}%
\pgfsetfillcolor{currentfill}%
\pgfsetlinewidth{0.602250pt}%
\definecolor{currentstroke}{rgb}{0.000000,0.000000,0.000000}%
\pgfsetstrokecolor{currentstroke}%
\pgfsetdash{}{0pt}%
\pgfsys@defobject{currentmarker}{\pgfqpoint{0.000000in}{0.000000in}}{\pgfqpoint{0.027778in}{0.000000in}}{%
\pgfpathmoveto{\pgfqpoint{0.000000in}{0.000000in}}%
\pgfpathlineto{\pgfqpoint{0.027778in}{0.000000in}}%
\pgfusepath{stroke,fill}%
}%
\begin{pgfscope}%
\pgfsys@transformshift{5.552773in}{2.031225in}%
\pgfsys@useobject{currentmarker}{}%
\end{pgfscope}%
\end{pgfscope}%
\begin{pgfscope}%
\pgfsetbuttcap%
\pgfsetroundjoin%
\definecolor{currentfill}{rgb}{0.000000,0.000000,0.000000}%
\pgfsetfillcolor{currentfill}%
\pgfsetlinewidth{0.602250pt}%
\definecolor{currentstroke}{rgb}{0.000000,0.000000,0.000000}%
\pgfsetstrokecolor{currentstroke}%
\pgfsetdash{}{0pt}%
\pgfsys@defobject{currentmarker}{\pgfqpoint{0.000000in}{0.000000in}}{\pgfqpoint{0.027778in}{0.000000in}}{%
\pgfpathmoveto{\pgfqpoint{0.000000in}{0.000000in}}%
\pgfpathlineto{\pgfqpoint{0.027778in}{0.000000in}}%
\pgfusepath{stroke,fill}%
}%
\begin{pgfscope}%
\pgfsys@transformshift{5.552773in}{2.300197in}%
\pgfsys@useobject{currentmarker}{}%
\end{pgfscope}%
\end{pgfscope}%
\begin{pgfscope}%
\definecolor{textcolor}{rgb}{0.000000,0.000000,0.000000}%
\pgfsetstrokecolor{textcolor}%
\pgfsetfillcolor{textcolor}%
\pgftext[x=5.774995in,y=1.399142in,,top,rotate=90.000000]{\color{textcolor}\rmfamily\fontsize{10.000000}{12.000000}\selectfont \(\displaystyle N_{\text{EW}}\) in Photonen}%
\end{pgfscope}%
\begin{pgfscope}%
\pgfsetrectcap%
\pgfsetmiterjoin%
\pgfsetlinewidth{0.803000pt}%
\definecolor{currentstroke}{rgb}{0.000000,0.000000,0.000000}%
\pgfsetstrokecolor{currentstroke}%
\pgfsetdash{}{0pt}%
\pgfpathmoveto{\pgfqpoint{3.073825in}{0.498088in}}%
\pgfpathlineto{\pgfqpoint{3.073825in}{2.300197in}}%
\pgfusepath{stroke}%
\end{pgfscope}%
\begin{pgfscope}%
\pgfsetrectcap%
\pgfsetmiterjoin%
\pgfsetlinewidth{0.803000pt}%
\definecolor{currentstroke}{rgb}{0.000000,0.000000,0.000000}%
\pgfsetstrokecolor{currentstroke}%
\pgfsetdash{}{0pt}%
\pgfpathmoveto{\pgfqpoint{5.552773in}{0.498088in}}%
\pgfpathlineto{\pgfqpoint{5.552773in}{2.300197in}}%
\pgfusepath{stroke}%
\end{pgfscope}%
\begin{pgfscope}%
\pgfsetrectcap%
\pgfsetmiterjoin%
\pgfsetlinewidth{0.803000pt}%
\definecolor{currentstroke}{rgb}{0.000000,0.000000,0.000000}%
\pgfsetstrokecolor{currentstroke}%
\pgfsetdash{}{0pt}%
\pgfpathmoveto{\pgfqpoint{3.073825in}{0.498088in}}%
\pgfpathlineto{\pgfqpoint{5.552773in}{0.498088in}}%
\pgfusepath{stroke}%
\end{pgfscope}%
\begin{pgfscope}%
\pgfsetrectcap%
\pgfsetmiterjoin%
\pgfsetlinewidth{0.803000pt}%
\definecolor{currentstroke}{rgb}{0.000000,0.000000,0.000000}%
\pgfsetstrokecolor{currentstroke}%
\pgfsetdash{}{0pt}%
\pgfpathmoveto{\pgfqpoint{3.073825in}{2.300197in}}%
\pgfpathlineto{\pgfqpoint{5.552773in}{2.300197in}}%
\pgfusepath{stroke}%
\end{pgfscope}%
\begin{pgfscope}%
\pgfpathrectangle{\pgfqpoint{3.073825in}{0.498088in}}{\pgfqpoint{2.478947in}{1.802109in}}%
\pgfusepath{clip}%
\pgfsetrectcap%
\pgfsetroundjoin%
\pgfsetlinewidth{1.505625pt}%
\definecolor{currentstroke}{rgb}{0.121569,0.466667,0.705882}%
\pgfsetstrokecolor{currentstroke}%
\pgfsetdash{}{0pt}%
\pgfpathmoveto{\pgfqpoint{3.175421in}{1.471839in}}%
\pgfpathlineto{\pgfqpoint{3.277018in}{1.321505in}}%
\pgfpathlineto{\pgfqpoint{3.378614in}{1.208119in}}%
\pgfpathlineto{\pgfqpoint{3.480210in}{1.123531in}}%
\pgfpathlineto{\pgfqpoint{3.581806in}{1.058811in}}%
\pgfpathlineto{\pgfqpoint{3.683402in}{1.011349in}}%
\pgfpathlineto{\pgfqpoint{3.784999in}{0.974096in}}%
\pgfpathlineto{\pgfqpoint{3.886595in}{0.943704in}}%
\pgfpathlineto{\pgfqpoint{3.988191in}{0.916301in}}%
\pgfpathlineto{\pgfqpoint{4.089787in}{0.893550in}}%
\pgfpathlineto{\pgfqpoint{4.191383in}{0.871087in}}%
\pgfpathlineto{\pgfqpoint{4.292980in}{0.849357in}}%
\pgfpathlineto{\pgfqpoint{4.394576in}{0.829163in}}%
\pgfpathlineto{\pgfqpoint{4.496172in}{0.808705in}}%
\pgfpathlineto{\pgfqpoint{4.597768in}{0.788931in}}%
\pgfpathlineto{\pgfqpoint{4.699364in}{0.769055in}}%
\pgfpathlineto{\pgfqpoint{4.800961in}{0.750258in}}%
\pgfpathlineto{\pgfqpoint{4.902557in}{0.731043in}}%
\pgfpathlineto{\pgfqpoint{5.004153in}{0.712747in}}%
\pgfpathlineto{\pgfqpoint{5.105749in}{0.695661in}}%
\pgfpathlineto{\pgfqpoint{5.207346in}{0.680399in}}%
\pgfpathlineto{\pgfqpoint{5.308942in}{0.667065in}}%
\pgfpathlineto{\pgfqpoint{5.410538in}{0.654003in}}%
\pgfusepath{stroke}%
\end{pgfscope}%
\end{pgfpicture}%
\makeatother%
\endgroup%

    \caption{Im Bezug auf Schwellenwert $s_V$ sind (oben links) Quanteneffizienz und (oben rechts) Zahl der fehldetektierten Photonen aufgetragen. Somit werden Erwartungswerte von Signal (unten links) und Rauschen (unten rechts) der Summe \SI{300}{\captures} $S/N_{\text{EW}}(s_V, N_A = 300, N_P = 1)$ aufgetragen. Man kann sehen, dass das Signal und Rauschen für den Schwellenwert $s_V$ im Inervall \qtyrange{100}{125}{\adu} vergleichbar sind.}
    \label{fig:qe_fehldetektiert_signal_noise}
\end{figure}
\noindent
Die \gls{qe}, die für die kleinen Werte $s_V$ die Werte größer \num{1} annimmt, ergibt physikalisch kein Sinn und kann durch das signifkante Detektorrauschen erklärt werden. Die Erwartungswerte $S_{\text{EW}}$ und $N_{\text{EW}}$ betragen im Bereich von $s_V = \text{\SIrange{100}{125}{\adu}}$ ca. \SI{1}{\photon} und sind nahezu gleich. So ist das \gls{snr} in dem Fall ca. 1.

\noindent
In Analogie dazu werden die Erwartungswerte $S_{\text{EW}}$ und $N_{\text{EW}}$ pro ein Pixel für die ausgewerteten Summen von \SI{50000}{\captures} im Bereich des Streuringes gesucht.

\noindent
Für ein erfasstes Pixel in der Summe ergeben sich die Erwartungswerte von Signal und Rauschen \begin{equation}
    \begin{split}
        S_\text{EW} &= \SI{1.97}{\photons}\\
        N_\text{EW} &= \SI{1.91}{\photons}\\
    \end{split}
\end{equation}
und damit ist das \gls{snr} pro ein Pixel ca. 1.

\noindent
Das \gls{snr} lässt sich bei den konstanten Parametern $N_A, s_V$ und $\text{\gls{photnenfluss}}$ mit
\begin{equation}
    \text{\gls{snr}} \approx \sqrt{N_P}
    \label{eq:snr_pixel}
\end{equation}
nähern und kann durch die Erfassung von mehreren Pixeln, z.B. über die radiale Integration des Streubildes, erhöht werden.

\section{Kleinwinkelstreuung}
Nun werden \SI{50000}{\captures} mit dem Schwellenwert-Algorithmus einzeln ausgewertet und aufsummiert. In Abb. \ref{fig:th_50_100_125_150_170_180_200_220_260} sind die Summen von ausgewerteten Aufnahmen in log-Skala dargestellt. Für die Auswertung jeder Summen wird ein Schwellenwert $s_V$ aus dem Intervall von \SIrange{50}{260}{\adu} genommen, um den Überblick auf die Abhängigkeit des Ergebnisses vom Schwellenwert zu geben. Der hellste rechförmige Fleck im Zentrum ist der Direktstrahl, der durch die Probe transmittiert wird. In Abb. \ref{fig:th_50_100_125_150_170_180_200_220_260}b, c und d können die ringförmigen Muster beobachtet werden, die dem erwarteten Strering entsprechen.
\begin{figure}[H]
    \centering
    %% Creator: Matplotlib, PGF backend
%%
%% To include the figure in your LaTeX document, write
%%   \input{<filename>.pgf}
%%
%% Make sure the required packages are loaded in your preamble
%%   \usepackage{pgf}
%%
%% Also ensure that all the required font packages are loaded; for instance,
%% the lmodern package is sometimes necessary when using math font.
%%   \usepackage{lmodern}
%%
%% Figures using additional raster images can only be included by \input if
%% they are in the same directory as the main LaTeX file. For loading figures
%% from other directories you can use the `import` package
%%   \usepackage{import}
%%
%% and then include the figures with
%%   \import{<path to file>}{<filename>.pgf}
%%
%% Matplotlib used the following preamble
%%   \usepackage{amsmath} \usepackage[utf8]{inputenc} \usepackage[T1]{fontenc} \usepackage[output-decimal-marker={,},print-unity-mantissa=false]{siunitx} \sisetup{per-mode=fraction, separate-uncertainty = true, locale = DE} \usepackage[acronym, toc, section=section, nonumberlist, nopostdot]{glossaries-extra}
%%
\begingroup%
\makeatletter%
\begin{pgfpicture}%
\pgfpathrectangle{\pgfpointorigin}{\pgfqpoint{6.196487in}{5.780024in}}%
\pgfusepath{use as bounding box, clip}%
\begin{pgfscope}%
\pgfsetbuttcap%
\pgfsetmiterjoin%
\pgfsetlinewidth{0.000000pt}%
\definecolor{currentstroke}{rgb}{1.000000,1.000000,1.000000}%
\pgfsetstrokecolor{currentstroke}%
\pgfsetstrokeopacity{0.000000}%
\pgfsetdash{}{0pt}%
\pgfpathmoveto{\pgfqpoint{0.000000in}{0.000000in}}%
\pgfpathlineto{\pgfqpoint{6.196487in}{0.000000in}}%
\pgfpathlineto{\pgfqpoint{6.196487in}{5.780024in}}%
\pgfpathlineto{\pgfqpoint{0.000000in}{5.780024in}}%
\pgfpathlineto{\pgfqpoint{0.000000in}{0.000000in}}%
\pgfpathclose%
\pgfusepath{}%
\end{pgfscope}%
\begin{pgfscope}%
\pgfsetbuttcap%
\pgfsetmiterjoin%
\definecolor{currentfill}{rgb}{1.000000,1.000000,1.000000}%
\pgfsetfillcolor{currentfill}%
\pgfsetlinewidth{0.000000pt}%
\definecolor{currentstroke}{rgb}{0.000000,0.000000,0.000000}%
\pgfsetstrokecolor{currentstroke}%
\pgfsetstrokeopacity{0.000000}%
\pgfsetdash{}{0pt}%
\pgfpathmoveto{\pgfqpoint{0.158099in}{3.936799in}}%
\pgfpathlineto{\pgfqpoint{1.739084in}{3.936799in}}%
\pgfpathlineto{\pgfqpoint{1.739084in}{5.517785in}}%
\pgfpathlineto{\pgfqpoint{0.158099in}{5.517785in}}%
\pgfpathlineto{\pgfqpoint{0.158099in}{3.936799in}}%
\pgfpathclose%
\pgfusepath{fill}%
\end{pgfscope}%
\begin{pgfscope}%
\pgfsys@transformshift{0.158000in}{3.936024in}%
\pgftext[left,bottom]{\includegraphics[interpolate=true,width=1.582000in,height=1.582000in]{th_50_100_125_150_170_180_200_220_260-img0.png}}%
\end{pgfscope}%
\begin{pgfscope}%
\pgfsetrectcap%
\pgfsetmiterjoin%
\pgfsetlinewidth{0.803000pt}%
\definecolor{currentstroke}{rgb}{0.000000,0.000000,0.000000}%
\pgfsetstrokecolor{currentstroke}%
\pgfsetdash{}{0pt}%
\pgfpathmoveto{\pgfqpoint{0.158099in}{3.936799in}}%
\pgfpathlineto{\pgfqpoint{0.158099in}{5.517785in}}%
\pgfusepath{stroke}%
\end{pgfscope}%
\begin{pgfscope}%
\pgfsetrectcap%
\pgfsetmiterjoin%
\pgfsetlinewidth{0.803000pt}%
\definecolor{currentstroke}{rgb}{0.000000,0.000000,0.000000}%
\pgfsetstrokecolor{currentstroke}%
\pgfsetdash{}{0pt}%
\pgfpathmoveto{\pgfqpoint{1.739084in}{3.936799in}}%
\pgfpathlineto{\pgfqpoint{1.739084in}{5.517785in}}%
\pgfusepath{stroke}%
\end{pgfscope}%
\begin{pgfscope}%
\pgfsetrectcap%
\pgfsetmiterjoin%
\pgfsetlinewidth{0.803000pt}%
\definecolor{currentstroke}{rgb}{0.000000,0.000000,0.000000}%
\pgfsetstrokecolor{currentstroke}%
\pgfsetdash{}{0pt}%
\pgfpathmoveto{\pgfqpoint{0.158099in}{3.936799in}}%
\pgfpathlineto{\pgfqpoint{1.739084in}{3.936799in}}%
\pgfusepath{stroke}%
\end{pgfscope}%
\begin{pgfscope}%
\pgfsetrectcap%
\pgfsetmiterjoin%
\pgfsetlinewidth{0.803000pt}%
\definecolor{currentstroke}{rgb}{0.000000,0.000000,0.000000}%
\pgfsetstrokecolor{currentstroke}%
\pgfsetdash{}{0pt}%
\pgfpathmoveto{\pgfqpoint{0.158099in}{5.517785in}}%
\pgfpathlineto{\pgfqpoint{1.739084in}{5.517785in}}%
\pgfusepath{stroke}%
\end{pgfscope}%
\begin{pgfscope}%
\definecolor{textcolor}{rgb}{0.000000,0.000000,0.000000}%
\pgfsetstrokecolor{textcolor}%
\pgfsetfillcolor{textcolor}%
\pgftext[x=0.000000in,y=5.675883in,left,base]{\color{textcolor}\rmfamily\fontsize{10.000000}{12.000000}\selectfont (a)}%
\end{pgfscope}%
\begin{pgfscope}%
\pgfsetbuttcap%
\pgfsetmiterjoin%
\definecolor{currentfill}{rgb}{1.000000,1.000000,1.000000}%
\pgfsetfillcolor{currentfill}%
\pgfsetlinewidth{0.000000pt}%
\definecolor{currentstroke}{rgb}{0.000000,0.000000,0.000000}%
\pgfsetstrokecolor{currentstroke}%
\pgfsetstrokeopacity{0.000000}%
\pgfsetdash{}{0pt}%
\pgfpathmoveto{\pgfqpoint{1.939084in}{3.936799in}}%
\pgfpathlineto{\pgfqpoint{3.520070in}{3.936799in}}%
\pgfpathlineto{\pgfqpoint{3.520070in}{5.517785in}}%
\pgfpathlineto{\pgfqpoint{1.939084in}{5.517785in}}%
\pgfpathlineto{\pgfqpoint{1.939084in}{3.936799in}}%
\pgfpathclose%
\pgfusepath{fill}%
\end{pgfscope}%
\begin{pgfscope}%
\pgfsys@transformshift{1.940000in}{3.936024in}%
\pgftext[left,bottom]{\includegraphics[interpolate=true,width=1.580000in,height=1.582000in]{th_50_100_125_150_170_180_200_220_260-img1.png}}%
\end{pgfscope}%
\begin{pgfscope}%
\pgfsetrectcap%
\pgfsetmiterjoin%
\pgfsetlinewidth{0.803000pt}%
\definecolor{currentstroke}{rgb}{0.000000,0.000000,0.000000}%
\pgfsetstrokecolor{currentstroke}%
\pgfsetdash{}{0pt}%
\pgfpathmoveto{\pgfqpoint{1.939084in}{3.936799in}}%
\pgfpathlineto{\pgfqpoint{1.939084in}{5.517785in}}%
\pgfusepath{stroke}%
\end{pgfscope}%
\begin{pgfscope}%
\pgfsetrectcap%
\pgfsetmiterjoin%
\pgfsetlinewidth{0.803000pt}%
\definecolor{currentstroke}{rgb}{0.000000,0.000000,0.000000}%
\pgfsetstrokecolor{currentstroke}%
\pgfsetdash{}{0pt}%
\pgfpathmoveto{\pgfqpoint{3.520070in}{3.936799in}}%
\pgfpathlineto{\pgfqpoint{3.520070in}{5.517785in}}%
\pgfusepath{stroke}%
\end{pgfscope}%
\begin{pgfscope}%
\pgfsetrectcap%
\pgfsetmiterjoin%
\pgfsetlinewidth{0.803000pt}%
\definecolor{currentstroke}{rgb}{0.000000,0.000000,0.000000}%
\pgfsetstrokecolor{currentstroke}%
\pgfsetdash{}{0pt}%
\pgfpathmoveto{\pgfqpoint{1.939084in}{3.936799in}}%
\pgfpathlineto{\pgfqpoint{3.520070in}{3.936799in}}%
\pgfusepath{stroke}%
\end{pgfscope}%
\begin{pgfscope}%
\pgfsetrectcap%
\pgfsetmiterjoin%
\pgfsetlinewidth{0.803000pt}%
\definecolor{currentstroke}{rgb}{0.000000,0.000000,0.000000}%
\pgfsetstrokecolor{currentstroke}%
\pgfsetdash{}{0pt}%
\pgfpathmoveto{\pgfqpoint{1.939084in}{5.517785in}}%
\pgfpathlineto{\pgfqpoint{3.520070in}{5.517785in}}%
\pgfusepath{stroke}%
\end{pgfscope}%
\begin{pgfscope}%
\definecolor{textcolor}{rgb}{0.000000,0.000000,0.000000}%
\pgfsetstrokecolor{textcolor}%
\pgfsetfillcolor{textcolor}%
\pgftext[x=1.780986in,y=5.675883in,left,base]{\color{textcolor}\rmfamily\fontsize{10.000000}{12.000000}\selectfont (b)}%
\end{pgfscope}%
\begin{pgfscope}%
\pgfsetbuttcap%
\pgfsetmiterjoin%
\definecolor{currentfill}{rgb}{1.000000,1.000000,1.000000}%
\pgfsetfillcolor{currentfill}%
\pgfsetlinewidth{0.000000pt}%
\definecolor{currentstroke}{rgb}{0.000000,0.000000,0.000000}%
\pgfsetstrokecolor{currentstroke}%
\pgfsetstrokeopacity{0.000000}%
\pgfsetdash{}{0pt}%
\pgfpathmoveto{\pgfqpoint{3.720070in}{3.936799in}}%
\pgfpathlineto{\pgfqpoint{5.301056in}{3.936799in}}%
\pgfpathlineto{\pgfqpoint{5.301056in}{5.517785in}}%
\pgfpathlineto{\pgfqpoint{3.720070in}{5.517785in}}%
\pgfpathlineto{\pgfqpoint{3.720070in}{3.936799in}}%
\pgfpathclose%
\pgfusepath{fill}%
\end{pgfscope}%
\begin{pgfscope}%
\pgfsys@transformshift{3.720000in}{3.936024in}%
\pgftext[left,bottom]{\includegraphics[interpolate=true,width=1.582000in,height=1.582000in]{th_50_100_125_150_170_180_200_220_260-img2.png}}%
\end{pgfscope}%
\begin{pgfscope}%
\pgfsetrectcap%
\pgfsetmiterjoin%
\pgfsetlinewidth{0.803000pt}%
\definecolor{currentstroke}{rgb}{0.000000,0.000000,0.000000}%
\pgfsetstrokecolor{currentstroke}%
\pgfsetdash{}{0pt}%
\pgfpathmoveto{\pgfqpoint{3.720070in}{3.936799in}}%
\pgfpathlineto{\pgfqpoint{3.720070in}{5.517785in}}%
\pgfusepath{stroke}%
\end{pgfscope}%
\begin{pgfscope}%
\pgfsetrectcap%
\pgfsetmiterjoin%
\pgfsetlinewidth{0.803000pt}%
\definecolor{currentstroke}{rgb}{0.000000,0.000000,0.000000}%
\pgfsetstrokecolor{currentstroke}%
\pgfsetdash{}{0pt}%
\pgfpathmoveto{\pgfqpoint{5.301056in}{3.936799in}}%
\pgfpathlineto{\pgfqpoint{5.301056in}{5.517785in}}%
\pgfusepath{stroke}%
\end{pgfscope}%
\begin{pgfscope}%
\pgfsetrectcap%
\pgfsetmiterjoin%
\pgfsetlinewidth{0.803000pt}%
\definecolor{currentstroke}{rgb}{0.000000,0.000000,0.000000}%
\pgfsetstrokecolor{currentstroke}%
\pgfsetdash{}{0pt}%
\pgfpathmoveto{\pgfqpoint{3.720070in}{3.936799in}}%
\pgfpathlineto{\pgfqpoint{5.301056in}{3.936799in}}%
\pgfusepath{stroke}%
\end{pgfscope}%
\begin{pgfscope}%
\pgfsetrectcap%
\pgfsetmiterjoin%
\pgfsetlinewidth{0.803000pt}%
\definecolor{currentstroke}{rgb}{0.000000,0.000000,0.000000}%
\pgfsetstrokecolor{currentstroke}%
\pgfsetdash{}{0pt}%
\pgfpathmoveto{\pgfqpoint{3.720070in}{5.517785in}}%
\pgfpathlineto{\pgfqpoint{5.301056in}{5.517785in}}%
\pgfusepath{stroke}%
\end{pgfscope}%
\begin{pgfscope}%
\definecolor{textcolor}{rgb}{0.000000,0.000000,0.000000}%
\pgfsetstrokecolor{textcolor}%
\pgfsetfillcolor{textcolor}%
\pgftext[x=3.561972in,y=5.675883in,left,base]{\color{textcolor}\rmfamily\fontsize{10.000000}{12.000000}\selectfont (c)}%
\end{pgfscope}%
\begin{pgfscope}%
\pgfsetbuttcap%
\pgfsetmiterjoin%
\definecolor{currentfill}{rgb}{1.000000,1.000000,1.000000}%
\pgfsetfillcolor{currentfill}%
\pgfsetlinewidth{0.000000pt}%
\definecolor{currentstroke}{rgb}{0.000000,0.000000,0.000000}%
\pgfsetstrokecolor{currentstroke}%
\pgfsetstrokeopacity{0.000000}%
\pgfsetdash{}{0pt}%
\pgfpathmoveto{\pgfqpoint{0.158099in}{2.005813in}}%
\pgfpathlineto{\pgfqpoint{1.739084in}{2.005813in}}%
\pgfpathlineto{\pgfqpoint{1.739084in}{3.586799in}}%
\pgfpathlineto{\pgfqpoint{0.158099in}{3.586799in}}%
\pgfpathlineto{\pgfqpoint{0.158099in}{2.005813in}}%
\pgfpathclose%
\pgfusepath{fill}%
\end{pgfscope}%
\begin{pgfscope}%
\pgfsys@transformshift{0.158000in}{2.006024in}%
\pgftext[left,bottom]{\includegraphics[interpolate=true,width=1.582000in,height=1.580000in]{th_50_100_125_150_170_180_200_220_260-img3.png}}%
\end{pgfscope}%
\begin{pgfscope}%
\pgfsetrectcap%
\pgfsetmiterjoin%
\pgfsetlinewidth{0.803000pt}%
\definecolor{currentstroke}{rgb}{0.000000,0.000000,0.000000}%
\pgfsetstrokecolor{currentstroke}%
\pgfsetdash{}{0pt}%
\pgfpathmoveto{\pgfqpoint{0.158099in}{2.005813in}}%
\pgfpathlineto{\pgfqpoint{0.158099in}{3.586799in}}%
\pgfusepath{stroke}%
\end{pgfscope}%
\begin{pgfscope}%
\pgfsetrectcap%
\pgfsetmiterjoin%
\pgfsetlinewidth{0.803000pt}%
\definecolor{currentstroke}{rgb}{0.000000,0.000000,0.000000}%
\pgfsetstrokecolor{currentstroke}%
\pgfsetdash{}{0pt}%
\pgfpathmoveto{\pgfqpoint{1.739084in}{2.005813in}}%
\pgfpathlineto{\pgfqpoint{1.739084in}{3.586799in}}%
\pgfusepath{stroke}%
\end{pgfscope}%
\begin{pgfscope}%
\pgfsetrectcap%
\pgfsetmiterjoin%
\pgfsetlinewidth{0.803000pt}%
\definecolor{currentstroke}{rgb}{0.000000,0.000000,0.000000}%
\pgfsetstrokecolor{currentstroke}%
\pgfsetdash{}{0pt}%
\pgfpathmoveto{\pgfqpoint{0.158099in}{2.005813in}}%
\pgfpathlineto{\pgfqpoint{1.739084in}{2.005813in}}%
\pgfusepath{stroke}%
\end{pgfscope}%
\begin{pgfscope}%
\pgfsetrectcap%
\pgfsetmiterjoin%
\pgfsetlinewidth{0.803000pt}%
\definecolor{currentstroke}{rgb}{0.000000,0.000000,0.000000}%
\pgfsetstrokecolor{currentstroke}%
\pgfsetdash{}{0pt}%
\pgfpathmoveto{\pgfqpoint{0.158099in}{3.586799in}}%
\pgfpathlineto{\pgfqpoint{1.739084in}{3.586799in}}%
\pgfusepath{stroke}%
\end{pgfscope}%
\begin{pgfscope}%
\definecolor{textcolor}{rgb}{0.000000,0.000000,0.000000}%
\pgfsetstrokecolor{textcolor}%
\pgfsetfillcolor{textcolor}%
\pgftext[x=0.000000in,y=3.744897in,left,base]{\color{textcolor}\rmfamily\fontsize{10.000000}{12.000000}\selectfont (d)}%
\end{pgfscope}%
\begin{pgfscope}%
\pgfsetbuttcap%
\pgfsetmiterjoin%
\definecolor{currentfill}{rgb}{1.000000,1.000000,1.000000}%
\pgfsetfillcolor{currentfill}%
\pgfsetlinewidth{0.000000pt}%
\definecolor{currentstroke}{rgb}{0.000000,0.000000,0.000000}%
\pgfsetstrokecolor{currentstroke}%
\pgfsetstrokeopacity{0.000000}%
\pgfsetdash{}{0pt}%
\pgfpathmoveto{\pgfqpoint{1.939084in}{2.005813in}}%
\pgfpathlineto{\pgfqpoint{3.520070in}{2.005813in}}%
\pgfpathlineto{\pgfqpoint{3.520070in}{3.586799in}}%
\pgfpathlineto{\pgfqpoint{1.939084in}{3.586799in}}%
\pgfpathlineto{\pgfqpoint{1.939084in}{2.005813in}}%
\pgfpathclose%
\pgfusepath{fill}%
\end{pgfscope}%
\begin{pgfscope}%
\pgfsys@transformshift{1.940000in}{2.006024in}%
\pgftext[left,bottom]{\includegraphics[interpolate=true,width=1.580000in,height=1.580000in]{th_50_100_125_150_170_180_200_220_260-img4.png}}%
\end{pgfscope}%
\begin{pgfscope}%
\pgfsetrectcap%
\pgfsetmiterjoin%
\pgfsetlinewidth{0.803000pt}%
\definecolor{currentstroke}{rgb}{0.000000,0.000000,0.000000}%
\pgfsetstrokecolor{currentstroke}%
\pgfsetdash{}{0pt}%
\pgfpathmoveto{\pgfqpoint{1.939084in}{2.005813in}}%
\pgfpathlineto{\pgfqpoint{1.939084in}{3.586799in}}%
\pgfusepath{stroke}%
\end{pgfscope}%
\begin{pgfscope}%
\pgfsetrectcap%
\pgfsetmiterjoin%
\pgfsetlinewidth{0.803000pt}%
\definecolor{currentstroke}{rgb}{0.000000,0.000000,0.000000}%
\pgfsetstrokecolor{currentstroke}%
\pgfsetdash{}{0pt}%
\pgfpathmoveto{\pgfqpoint{3.520070in}{2.005813in}}%
\pgfpathlineto{\pgfqpoint{3.520070in}{3.586799in}}%
\pgfusepath{stroke}%
\end{pgfscope}%
\begin{pgfscope}%
\pgfsetrectcap%
\pgfsetmiterjoin%
\pgfsetlinewidth{0.803000pt}%
\definecolor{currentstroke}{rgb}{0.000000,0.000000,0.000000}%
\pgfsetstrokecolor{currentstroke}%
\pgfsetdash{}{0pt}%
\pgfpathmoveto{\pgfqpoint{1.939084in}{2.005813in}}%
\pgfpathlineto{\pgfqpoint{3.520070in}{2.005813in}}%
\pgfusepath{stroke}%
\end{pgfscope}%
\begin{pgfscope}%
\pgfsetrectcap%
\pgfsetmiterjoin%
\pgfsetlinewidth{0.803000pt}%
\definecolor{currentstroke}{rgb}{0.000000,0.000000,0.000000}%
\pgfsetstrokecolor{currentstroke}%
\pgfsetdash{}{0pt}%
\pgfpathmoveto{\pgfqpoint{1.939084in}{3.586799in}}%
\pgfpathlineto{\pgfqpoint{3.520070in}{3.586799in}}%
\pgfusepath{stroke}%
\end{pgfscope}%
\begin{pgfscope}%
\definecolor{textcolor}{rgb}{0.000000,0.000000,0.000000}%
\pgfsetstrokecolor{textcolor}%
\pgfsetfillcolor{textcolor}%
\pgftext[x=1.780986in,y=3.744897in,left,base]{\color{textcolor}\rmfamily\fontsize{10.000000}{12.000000}\selectfont (e)}%
\end{pgfscope}%
\begin{pgfscope}%
\pgfsetbuttcap%
\pgfsetmiterjoin%
\definecolor{currentfill}{rgb}{1.000000,1.000000,1.000000}%
\pgfsetfillcolor{currentfill}%
\pgfsetlinewidth{0.000000pt}%
\definecolor{currentstroke}{rgb}{0.000000,0.000000,0.000000}%
\pgfsetstrokecolor{currentstroke}%
\pgfsetstrokeopacity{0.000000}%
\pgfsetdash{}{0pt}%
\pgfpathmoveto{\pgfqpoint{3.720070in}{2.005813in}}%
\pgfpathlineto{\pgfqpoint{5.301056in}{2.005813in}}%
\pgfpathlineto{\pgfqpoint{5.301056in}{3.586799in}}%
\pgfpathlineto{\pgfqpoint{3.720070in}{3.586799in}}%
\pgfpathlineto{\pgfqpoint{3.720070in}{2.005813in}}%
\pgfpathclose%
\pgfusepath{fill}%
\end{pgfscope}%
\begin{pgfscope}%
\pgfsys@transformshift{3.806000in}{2.006024in}%
\pgftext[left,bottom]{\includegraphics[interpolate=true,width=1.496000in,height=1.568000in]{th_50_100_125_150_170_180_200_220_260-img5.png}}%
\end{pgfscope}%
\begin{pgfscope}%
\pgfsetrectcap%
\pgfsetmiterjoin%
\pgfsetlinewidth{0.803000pt}%
\definecolor{currentstroke}{rgb}{0.000000,0.000000,0.000000}%
\pgfsetstrokecolor{currentstroke}%
\pgfsetdash{}{0pt}%
\pgfpathmoveto{\pgfqpoint{3.720070in}{2.005813in}}%
\pgfpathlineto{\pgfqpoint{3.720070in}{3.586799in}}%
\pgfusepath{stroke}%
\end{pgfscope}%
\begin{pgfscope}%
\pgfsetrectcap%
\pgfsetmiterjoin%
\pgfsetlinewidth{0.803000pt}%
\definecolor{currentstroke}{rgb}{0.000000,0.000000,0.000000}%
\pgfsetstrokecolor{currentstroke}%
\pgfsetdash{}{0pt}%
\pgfpathmoveto{\pgfqpoint{5.301056in}{2.005813in}}%
\pgfpathlineto{\pgfqpoint{5.301056in}{3.586799in}}%
\pgfusepath{stroke}%
\end{pgfscope}%
\begin{pgfscope}%
\pgfsetrectcap%
\pgfsetmiterjoin%
\pgfsetlinewidth{0.803000pt}%
\definecolor{currentstroke}{rgb}{0.000000,0.000000,0.000000}%
\pgfsetstrokecolor{currentstroke}%
\pgfsetdash{}{0pt}%
\pgfpathmoveto{\pgfqpoint{3.720070in}{2.005813in}}%
\pgfpathlineto{\pgfqpoint{5.301056in}{2.005813in}}%
\pgfusepath{stroke}%
\end{pgfscope}%
\begin{pgfscope}%
\pgfsetrectcap%
\pgfsetmiterjoin%
\pgfsetlinewidth{0.803000pt}%
\definecolor{currentstroke}{rgb}{0.000000,0.000000,0.000000}%
\pgfsetstrokecolor{currentstroke}%
\pgfsetdash{}{0pt}%
\pgfpathmoveto{\pgfqpoint{3.720070in}{3.586799in}}%
\pgfpathlineto{\pgfqpoint{5.301056in}{3.586799in}}%
\pgfusepath{stroke}%
\end{pgfscope}%
\begin{pgfscope}%
\definecolor{textcolor}{rgb}{0.000000,0.000000,0.000000}%
\pgfsetstrokecolor{textcolor}%
\pgfsetfillcolor{textcolor}%
\pgftext[x=3.561972in,y=3.744897in,left,base]{\color{textcolor}\rmfamily\fontsize{10.000000}{12.000000}\selectfont (f)}%
\end{pgfscope}%
\begin{pgfscope}%
\pgfsetbuttcap%
\pgfsetmiterjoin%
\definecolor{currentfill}{rgb}{1.000000,1.000000,1.000000}%
\pgfsetfillcolor{currentfill}%
\pgfsetlinewidth{0.000000pt}%
\definecolor{currentstroke}{rgb}{0.000000,0.000000,0.000000}%
\pgfsetstrokecolor{currentstroke}%
\pgfsetstrokeopacity{0.000000}%
\pgfsetdash{}{0pt}%
\pgfpathmoveto{\pgfqpoint{0.158099in}{0.074827in}}%
\pgfpathlineto{\pgfqpoint{1.739084in}{0.074827in}}%
\pgfpathlineto{\pgfqpoint{1.739084in}{1.655813in}}%
\pgfpathlineto{\pgfqpoint{0.158099in}{1.655813in}}%
\pgfpathlineto{\pgfqpoint{0.158099in}{0.074827in}}%
\pgfpathclose%
\pgfusepath{fill}%
\end{pgfscope}%
\begin{pgfscope}%
\pgfsys@transformshift{0.244000in}{0.074024in}%
\pgftext[left,bottom]{\includegraphics[interpolate=true,width=1.476000in,height=1.568000in]{th_50_100_125_150_170_180_200_220_260-img6.png}}%
\end{pgfscope}%
\begin{pgfscope}%
\pgfsetrectcap%
\pgfsetmiterjoin%
\pgfsetlinewidth{0.803000pt}%
\definecolor{currentstroke}{rgb}{0.000000,0.000000,0.000000}%
\pgfsetstrokecolor{currentstroke}%
\pgfsetdash{}{0pt}%
\pgfpathmoveto{\pgfqpoint{0.158099in}{0.074827in}}%
\pgfpathlineto{\pgfqpoint{0.158099in}{1.655813in}}%
\pgfusepath{stroke}%
\end{pgfscope}%
\begin{pgfscope}%
\pgfsetrectcap%
\pgfsetmiterjoin%
\pgfsetlinewidth{0.803000pt}%
\definecolor{currentstroke}{rgb}{0.000000,0.000000,0.000000}%
\pgfsetstrokecolor{currentstroke}%
\pgfsetdash{}{0pt}%
\pgfpathmoveto{\pgfqpoint{1.739084in}{0.074827in}}%
\pgfpathlineto{\pgfqpoint{1.739084in}{1.655813in}}%
\pgfusepath{stroke}%
\end{pgfscope}%
\begin{pgfscope}%
\pgfsetrectcap%
\pgfsetmiterjoin%
\pgfsetlinewidth{0.803000pt}%
\definecolor{currentstroke}{rgb}{0.000000,0.000000,0.000000}%
\pgfsetstrokecolor{currentstroke}%
\pgfsetdash{}{0pt}%
\pgfpathmoveto{\pgfqpoint{0.158099in}{0.074827in}}%
\pgfpathlineto{\pgfqpoint{1.739084in}{0.074827in}}%
\pgfusepath{stroke}%
\end{pgfscope}%
\begin{pgfscope}%
\pgfsetrectcap%
\pgfsetmiterjoin%
\pgfsetlinewidth{0.803000pt}%
\definecolor{currentstroke}{rgb}{0.000000,0.000000,0.000000}%
\pgfsetstrokecolor{currentstroke}%
\pgfsetdash{}{0pt}%
\pgfpathmoveto{\pgfqpoint{0.158099in}{1.655813in}}%
\pgfpathlineto{\pgfqpoint{1.739084in}{1.655813in}}%
\pgfusepath{stroke}%
\end{pgfscope}%
\begin{pgfscope}%
\definecolor{textcolor}{rgb}{0.000000,0.000000,0.000000}%
\pgfsetstrokecolor{textcolor}%
\pgfsetfillcolor{textcolor}%
\pgftext[x=0.000000in,y=1.813912in,left,base]{\color{textcolor}\rmfamily\fontsize{10.000000}{12.000000}\selectfont (g)}%
\end{pgfscope}%
\begin{pgfscope}%
\pgfsetbuttcap%
\pgfsetmiterjoin%
\definecolor{currentfill}{rgb}{1.000000,1.000000,1.000000}%
\pgfsetfillcolor{currentfill}%
\pgfsetlinewidth{0.000000pt}%
\definecolor{currentstroke}{rgb}{0.000000,0.000000,0.000000}%
\pgfsetstrokecolor{currentstroke}%
\pgfsetstrokeopacity{0.000000}%
\pgfsetdash{}{0pt}%
\pgfpathmoveto{\pgfqpoint{1.939084in}{0.074827in}}%
\pgfpathlineto{\pgfqpoint{3.520070in}{0.074827in}}%
\pgfpathlineto{\pgfqpoint{3.520070in}{1.655813in}}%
\pgfpathlineto{\pgfqpoint{1.939084in}{1.655813in}}%
\pgfpathlineto{\pgfqpoint{1.939084in}{0.074827in}}%
\pgfpathclose%
\pgfusepath{fill}%
\end{pgfscope}%
\begin{pgfscope}%
\pgfsys@transformshift{2.024000in}{0.114024in}%
\pgftext[left,bottom]{\includegraphics[interpolate=true,width=1.476000in,height=1.528000in]{th_50_100_125_150_170_180_200_220_260-img7.png}}%
\end{pgfscope}%
\begin{pgfscope}%
\pgfsetrectcap%
\pgfsetmiterjoin%
\pgfsetlinewidth{0.803000pt}%
\definecolor{currentstroke}{rgb}{0.000000,0.000000,0.000000}%
\pgfsetstrokecolor{currentstroke}%
\pgfsetdash{}{0pt}%
\pgfpathmoveto{\pgfqpoint{1.939084in}{0.074827in}}%
\pgfpathlineto{\pgfqpoint{1.939084in}{1.655813in}}%
\pgfusepath{stroke}%
\end{pgfscope}%
\begin{pgfscope}%
\pgfsetrectcap%
\pgfsetmiterjoin%
\pgfsetlinewidth{0.803000pt}%
\definecolor{currentstroke}{rgb}{0.000000,0.000000,0.000000}%
\pgfsetstrokecolor{currentstroke}%
\pgfsetdash{}{0pt}%
\pgfpathmoveto{\pgfqpoint{3.520070in}{0.074827in}}%
\pgfpathlineto{\pgfqpoint{3.520070in}{1.655813in}}%
\pgfusepath{stroke}%
\end{pgfscope}%
\begin{pgfscope}%
\pgfsetrectcap%
\pgfsetmiterjoin%
\pgfsetlinewidth{0.803000pt}%
\definecolor{currentstroke}{rgb}{0.000000,0.000000,0.000000}%
\pgfsetstrokecolor{currentstroke}%
\pgfsetdash{}{0pt}%
\pgfpathmoveto{\pgfqpoint{1.939084in}{0.074827in}}%
\pgfpathlineto{\pgfqpoint{3.520070in}{0.074827in}}%
\pgfusepath{stroke}%
\end{pgfscope}%
\begin{pgfscope}%
\pgfsetrectcap%
\pgfsetmiterjoin%
\pgfsetlinewidth{0.803000pt}%
\definecolor{currentstroke}{rgb}{0.000000,0.000000,0.000000}%
\pgfsetstrokecolor{currentstroke}%
\pgfsetdash{}{0pt}%
\pgfpathmoveto{\pgfqpoint{1.939084in}{1.655813in}}%
\pgfpathlineto{\pgfqpoint{3.520070in}{1.655813in}}%
\pgfusepath{stroke}%
\end{pgfscope}%
\begin{pgfscope}%
\definecolor{textcolor}{rgb}{0.000000,0.000000,0.000000}%
\pgfsetstrokecolor{textcolor}%
\pgfsetfillcolor{textcolor}%
\pgftext[x=1.780986in,y=1.813912in,left,base]{\color{textcolor}\rmfamily\fontsize{10.000000}{12.000000}\selectfont (h)}%
\end{pgfscope}%
\begin{pgfscope}%
\pgfsetbuttcap%
\pgfsetmiterjoin%
\definecolor{currentfill}{rgb}{1.000000,1.000000,1.000000}%
\pgfsetfillcolor{currentfill}%
\pgfsetlinewidth{0.000000pt}%
\definecolor{currentstroke}{rgb}{0.000000,0.000000,0.000000}%
\pgfsetstrokecolor{currentstroke}%
\pgfsetstrokeopacity{0.000000}%
\pgfsetdash{}{0pt}%
\pgfpathmoveto{\pgfqpoint{3.720070in}{0.074827in}}%
\pgfpathlineto{\pgfqpoint{5.301056in}{0.074827in}}%
\pgfpathlineto{\pgfqpoint{5.301056in}{1.655813in}}%
\pgfpathlineto{\pgfqpoint{3.720070in}{1.655813in}}%
\pgfpathlineto{\pgfqpoint{3.720070in}{0.074827in}}%
\pgfpathclose%
\pgfusepath{fill}%
\end{pgfscope}%
\begin{pgfscope}%
\pgfsys@transformshift{3.806000in}{0.114024in}%
\pgftext[left,bottom]{\includegraphics[interpolate=true,width=1.476000in,height=1.528000in]{th_50_100_125_150_170_180_200_220_260-img8.png}}%
\end{pgfscope}%
\begin{pgfscope}%
\pgfsetrectcap%
\pgfsetmiterjoin%
\pgfsetlinewidth{0.803000pt}%
\definecolor{currentstroke}{rgb}{0.000000,0.000000,0.000000}%
\pgfsetstrokecolor{currentstroke}%
\pgfsetdash{}{0pt}%
\pgfpathmoveto{\pgfqpoint{3.720070in}{0.074827in}}%
\pgfpathlineto{\pgfqpoint{3.720070in}{1.655813in}}%
\pgfusepath{stroke}%
\end{pgfscope}%
\begin{pgfscope}%
\pgfsetrectcap%
\pgfsetmiterjoin%
\pgfsetlinewidth{0.803000pt}%
\definecolor{currentstroke}{rgb}{0.000000,0.000000,0.000000}%
\pgfsetstrokecolor{currentstroke}%
\pgfsetdash{}{0pt}%
\pgfpathmoveto{\pgfqpoint{5.301056in}{0.074827in}}%
\pgfpathlineto{\pgfqpoint{5.301056in}{1.655813in}}%
\pgfusepath{stroke}%
\end{pgfscope}%
\begin{pgfscope}%
\pgfsetrectcap%
\pgfsetmiterjoin%
\pgfsetlinewidth{0.803000pt}%
\definecolor{currentstroke}{rgb}{0.000000,0.000000,0.000000}%
\pgfsetstrokecolor{currentstroke}%
\pgfsetdash{}{0pt}%
\pgfpathmoveto{\pgfqpoint{3.720070in}{0.074827in}}%
\pgfpathlineto{\pgfqpoint{5.301056in}{0.074827in}}%
\pgfusepath{stroke}%
\end{pgfscope}%
\begin{pgfscope}%
\pgfsetrectcap%
\pgfsetmiterjoin%
\pgfsetlinewidth{0.803000pt}%
\definecolor{currentstroke}{rgb}{0.000000,0.000000,0.000000}%
\pgfsetstrokecolor{currentstroke}%
\pgfsetdash{}{0pt}%
\pgfpathmoveto{\pgfqpoint{3.720070in}{1.655813in}}%
\pgfpathlineto{\pgfqpoint{5.301056in}{1.655813in}}%
\pgfusepath{stroke}%
\end{pgfscope}%
\begin{pgfscope}%
\definecolor{textcolor}{rgb}{0.000000,0.000000,0.000000}%
\pgfsetstrokecolor{textcolor}%
\pgfsetfillcolor{textcolor}%
\pgftext[x=3.561972in,y=1.813912in,left,base]{\color{textcolor}\rmfamily\fontsize{10.000000}{12.000000}\selectfont (i)}%
\end{pgfscope}%
\begin{pgfscope}%
\pgfsetbuttcap%
\pgfsetmiterjoin%
\definecolor{currentfill}{rgb}{1.000000,1.000000,1.000000}%
\pgfsetfillcolor{currentfill}%
\pgfsetlinewidth{0.000000pt}%
\definecolor{currentstroke}{rgb}{0.000000,0.000000,0.000000}%
\pgfsetstrokecolor{currentstroke}%
\pgfsetstrokeopacity{0.000000}%
\pgfsetdash{}{0pt}%
\pgfpathmoveto{\pgfqpoint{5.501056in}{0.074827in}}%
\pgfpathlineto{\pgfqpoint{5.643345in}{0.074827in}}%
\pgfpathlineto{\pgfqpoint{5.643345in}{5.517785in}}%
\pgfpathlineto{\pgfqpoint{5.501056in}{5.517785in}}%
\pgfpathlineto{\pgfqpoint{5.501056in}{0.074827in}}%
\pgfpathclose%
\pgfusepath{fill}%
\end{pgfscope}%
\begin{pgfscope}%
\pgfpathrectangle{\pgfqpoint{5.501056in}{0.074827in}}{\pgfqpoint{0.142289in}{5.442957in}}%
\pgfusepath{clip}%
\pgfsetbuttcap%
\pgfsetmiterjoin%
\definecolor{currentfill}{rgb}{1.000000,1.000000,1.000000}%
\pgfsetfillcolor{currentfill}%
\pgfsetlinewidth{0.010037pt}%
\definecolor{currentstroke}{rgb}{1.000000,1.000000,1.000000}%
\pgfsetstrokecolor{currentstroke}%
\pgfsetdash{}{0pt}%
\pgfusepath{stroke,fill}%
\end{pgfscope}%
\begin{pgfscope}%
\pgfsys@transformshift{5.502000in}{0.074024in}%
\pgftext[left,bottom]{\includegraphics[interpolate=true,width=0.142000in,height=5.444000in]{th_50_100_125_150_170_180_200_220_260-img9.png}}%
\end{pgfscope}%
\begin{pgfscope}%
\pgfsetbuttcap%
\pgfsetroundjoin%
\definecolor{currentfill}{rgb}{0.000000,0.000000,0.000000}%
\pgfsetfillcolor{currentfill}%
\pgfsetlinewidth{0.803000pt}%
\definecolor{currentstroke}{rgb}{0.000000,0.000000,0.000000}%
\pgfsetstrokecolor{currentstroke}%
\pgfsetdash{}{0pt}%
\pgfsys@defobject{currentmarker}{\pgfqpoint{0.000000in}{0.000000in}}{\pgfqpoint{0.048611in}{0.000000in}}{%
\pgfpathmoveto{\pgfqpoint{0.000000in}{0.000000in}}%
\pgfpathlineto{\pgfqpoint{0.048611in}{0.000000in}}%
\pgfusepath{stroke,fill}%
}%
\begin{pgfscope}%
\pgfsys@transformshift{5.643345in}{0.074827in}%
\pgfsys@useobject{currentmarker}{}%
\end{pgfscope}%
\end{pgfscope}%
\begin{pgfscope}%
\definecolor{textcolor}{rgb}{0.000000,0.000000,0.000000}%
\pgfsetstrokecolor{textcolor}%
\pgfsetfillcolor{textcolor}%
\pgftext[x=5.740567in, y=0.027000in, left, base]{\color{textcolor}\rmfamily\fontsize{10.000000}{12.000000}\selectfont 1}%
\end{pgfscope}%
\begin{pgfscope}%
\pgfsetbuttcap%
\pgfsetroundjoin%
\definecolor{currentfill}{rgb}{0.000000,0.000000,0.000000}%
\pgfsetfillcolor{currentfill}%
\pgfsetlinewidth{0.803000pt}%
\definecolor{currentstroke}{rgb}{0.000000,0.000000,0.000000}%
\pgfsetstrokecolor{currentstroke}%
\pgfsetdash{}{0pt}%
\pgfsys@defobject{currentmarker}{\pgfqpoint{0.000000in}{0.000000in}}{\pgfqpoint{0.048611in}{0.000000in}}{%
\pgfpathmoveto{\pgfqpoint{0.000000in}{0.000000in}}%
\pgfpathlineto{\pgfqpoint{0.048611in}{0.000000in}}%
\pgfusepath{stroke,fill}%
}%
\begin{pgfscope}%
\pgfsys@transformshift{5.643345in}{1.889146in}%
\pgfsys@useobject{currentmarker}{}%
\end{pgfscope}%
\end{pgfscope}%
\begin{pgfscope}%
\definecolor{textcolor}{rgb}{0.000000,0.000000,0.000000}%
\pgfsetstrokecolor{textcolor}%
\pgfsetfillcolor{textcolor}%
\pgftext[x=5.740567in, y=1.841319in, left, base]{\color{textcolor}\rmfamily\fontsize{10.000000}{12.000000}\selectfont 10}%
\end{pgfscope}%
\begin{pgfscope}%
\pgfsetbuttcap%
\pgfsetroundjoin%
\definecolor{currentfill}{rgb}{0.000000,0.000000,0.000000}%
\pgfsetfillcolor{currentfill}%
\pgfsetlinewidth{0.803000pt}%
\definecolor{currentstroke}{rgb}{0.000000,0.000000,0.000000}%
\pgfsetstrokecolor{currentstroke}%
\pgfsetdash{}{0pt}%
\pgfsys@defobject{currentmarker}{\pgfqpoint{0.000000in}{0.000000in}}{\pgfqpoint{0.048611in}{0.000000in}}{%
\pgfpathmoveto{\pgfqpoint{0.000000in}{0.000000in}}%
\pgfpathlineto{\pgfqpoint{0.048611in}{0.000000in}}%
\pgfusepath{stroke,fill}%
}%
\begin{pgfscope}%
\pgfsys@transformshift{5.643345in}{3.703466in}%
\pgfsys@useobject{currentmarker}{}%
\end{pgfscope}%
\end{pgfscope}%
\begin{pgfscope}%
\definecolor{textcolor}{rgb}{0.000000,0.000000,0.000000}%
\pgfsetstrokecolor{textcolor}%
\pgfsetfillcolor{textcolor}%
\pgftext[x=5.740567in, y=3.655638in, left, base]{\color{textcolor}\rmfamily\fontsize{10.000000}{12.000000}\selectfont 100}%
\end{pgfscope}%
\begin{pgfscope}%
\pgfsetbuttcap%
\pgfsetroundjoin%
\definecolor{currentfill}{rgb}{0.000000,0.000000,0.000000}%
\pgfsetfillcolor{currentfill}%
\pgfsetlinewidth{0.803000pt}%
\definecolor{currentstroke}{rgb}{0.000000,0.000000,0.000000}%
\pgfsetstrokecolor{currentstroke}%
\pgfsetdash{}{0pt}%
\pgfsys@defobject{currentmarker}{\pgfqpoint{0.000000in}{0.000000in}}{\pgfqpoint{0.048611in}{0.000000in}}{%
\pgfpathmoveto{\pgfqpoint{0.000000in}{0.000000in}}%
\pgfpathlineto{\pgfqpoint{0.048611in}{0.000000in}}%
\pgfusepath{stroke,fill}%
}%
\begin{pgfscope}%
\pgfsys@transformshift{5.643345in}{5.517785in}%
\pgfsys@useobject{currentmarker}{}%
\end{pgfscope}%
\end{pgfscope}%
\begin{pgfscope}%
\definecolor{textcolor}{rgb}{0.000000,0.000000,0.000000}%
\pgfsetstrokecolor{textcolor}%
\pgfsetfillcolor{textcolor}%
\pgftext[x=5.740567in, y=5.469957in, left, base]{\color{textcolor}\rmfamily\fontsize{10.000000}{12.000000}\selectfont 1000}%
\end{pgfscope}%
\begin{pgfscope}%
\pgfsetbuttcap%
\pgfsetroundjoin%
\definecolor{currentfill}{rgb}{0.000000,0.000000,0.000000}%
\pgfsetfillcolor{currentfill}%
\pgfsetlinewidth{0.602250pt}%
\definecolor{currentstroke}{rgb}{0.000000,0.000000,0.000000}%
\pgfsetstrokecolor{currentstroke}%
\pgfsetdash{}{0pt}%
\pgfsys@defobject{currentmarker}{\pgfqpoint{0.000000in}{0.000000in}}{\pgfqpoint{0.027778in}{0.000000in}}{%
\pgfpathmoveto{\pgfqpoint{0.000000in}{0.000000in}}%
\pgfpathlineto{\pgfqpoint{0.027778in}{0.000000in}}%
\pgfusepath{stroke,fill}%
}%
\begin{pgfscope}%
\pgfsys@transformshift{5.643345in}{0.620992in}%
\pgfsys@useobject{currentmarker}{}%
\end{pgfscope}%
\end{pgfscope}%
\begin{pgfscope}%
\pgfsetbuttcap%
\pgfsetroundjoin%
\definecolor{currentfill}{rgb}{0.000000,0.000000,0.000000}%
\pgfsetfillcolor{currentfill}%
\pgfsetlinewidth{0.602250pt}%
\definecolor{currentstroke}{rgb}{0.000000,0.000000,0.000000}%
\pgfsetstrokecolor{currentstroke}%
\pgfsetdash{}{0pt}%
\pgfsys@defobject{currentmarker}{\pgfqpoint{0.000000in}{0.000000in}}{\pgfqpoint{0.027778in}{0.000000in}}{%
\pgfpathmoveto{\pgfqpoint{0.000000in}{0.000000in}}%
\pgfpathlineto{\pgfqpoint{0.027778in}{0.000000in}}%
\pgfusepath{stroke,fill}%
}%
\begin{pgfscope}%
\pgfsys@transformshift{5.643345in}{0.940478in}%
\pgfsys@useobject{currentmarker}{}%
\end{pgfscope}%
\end{pgfscope}%
\begin{pgfscope}%
\pgfsetbuttcap%
\pgfsetroundjoin%
\definecolor{currentfill}{rgb}{0.000000,0.000000,0.000000}%
\pgfsetfillcolor{currentfill}%
\pgfsetlinewidth{0.602250pt}%
\definecolor{currentstroke}{rgb}{0.000000,0.000000,0.000000}%
\pgfsetstrokecolor{currentstroke}%
\pgfsetdash{}{0pt}%
\pgfsys@defobject{currentmarker}{\pgfqpoint{0.000000in}{0.000000in}}{\pgfqpoint{0.027778in}{0.000000in}}{%
\pgfpathmoveto{\pgfqpoint{0.000000in}{0.000000in}}%
\pgfpathlineto{\pgfqpoint{0.027778in}{0.000000in}}%
\pgfusepath{stroke,fill}%
}%
\begin{pgfscope}%
\pgfsys@transformshift{5.643345in}{1.167156in}%
\pgfsys@useobject{currentmarker}{}%
\end{pgfscope}%
\end{pgfscope}%
\begin{pgfscope}%
\pgfsetbuttcap%
\pgfsetroundjoin%
\definecolor{currentfill}{rgb}{0.000000,0.000000,0.000000}%
\pgfsetfillcolor{currentfill}%
\pgfsetlinewidth{0.602250pt}%
\definecolor{currentstroke}{rgb}{0.000000,0.000000,0.000000}%
\pgfsetstrokecolor{currentstroke}%
\pgfsetdash{}{0pt}%
\pgfsys@defobject{currentmarker}{\pgfqpoint{0.000000in}{0.000000in}}{\pgfqpoint{0.027778in}{0.000000in}}{%
\pgfpathmoveto{\pgfqpoint{0.000000in}{0.000000in}}%
\pgfpathlineto{\pgfqpoint{0.027778in}{0.000000in}}%
\pgfusepath{stroke,fill}%
}%
\begin{pgfscope}%
\pgfsys@transformshift{5.643345in}{1.342982in}%
\pgfsys@useobject{currentmarker}{}%
\end{pgfscope}%
\end{pgfscope}%
\begin{pgfscope}%
\pgfsetbuttcap%
\pgfsetroundjoin%
\definecolor{currentfill}{rgb}{0.000000,0.000000,0.000000}%
\pgfsetfillcolor{currentfill}%
\pgfsetlinewidth{0.602250pt}%
\definecolor{currentstroke}{rgb}{0.000000,0.000000,0.000000}%
\pgfsetstrokecolor{currentstroke}%
\pgfsetdash{}{0pt}%
\pgfsys@defobject{currentmarker}{\pgfqpoint{0.000000in}{0.000000in}}{\pgfqpoint{0.027778in}{0.000000in}}{%
\pgfpathmoveto{\pgfqpoint{0.000000in}{0.000000in}}%
\pgfpathlineto{\pgfqpoint{0.027778in}{0.000000in}}%
\pgfusepath{stroke,fill}%
}%
\begin{pgfscope}%
\pgfsys@transformshift{5.643345in}{1.486642in}%
\pgfsys@useobject{currentmarker}{}%
\end{pgfscope}%
\end{pgfscope}%
\begin{pgfscope}%
\pgfsetbuttcap%
\pgfsetroundjoin%
\definecolor{currentfill}{rgb}{0.000000,0.000000,0.000000}%
\pgfsetfillcolor{currentfill}%
\pgfsetlinewidth{0.602250pt}%
\definecolor{currentstroke}{rgb}{0.000000,0.000000,0.000000}%
\pgfsetstrokecolor{currentstroke}%
\pgfsetdash{}{0pt}%
\pgfsys@defobject{currentmarker}{\pgfqpoint{0.000000in}{0.000000in}}{\pgfqpoint{0.027778in}{0.000000in}}{%
\pgfpathmoveto{\pgfqpoint{0.000000in}{0.000000in}}%
\pgfpathlineto{\pgfqpoint{0.027778in}{0.000000in}}%
\pgfusepath{stroke,fill}%
}%
\begin{pgfscope}%
\pgfsys@transformshift{5.643345in}{1.608105in}%
\pgfsys@useobject{currentmarker}{}%
\end{pgfscope}%
\end{pgfscope}%
\begin{pgfscope}%
\pgfsetbuttcap%
\pgfsetroundjoin%
\definecolor{currentfill}{rgb}{0.000000,0.000000,0.000000}%
\pgfsetfillcolor{currentfill}%
\pgfsetlinewidth{0.602250pt}%
\definecolor{currentstroke}{rgb}{0.000000,0.000000,0.000000}%
\pgfsetstrokecolor{currentstroke}%
\pgfsetdash{}{0pt}%
\pgfsys@defobject{currentmarker}{\pgfqpoint{0.000000in}{0.000000in}}{\pgfqpoint{0.027778in}{0.000000in}}{%
\pgfpathmoveto{\pgfqpoint{0.000000in}{0.000000in}}%
\pgfpathlineto{\pgfqpoint{0.027778in}{0.000000in}}%
\pgfusepath{stroke,fill}%
}%
\begin{pgfscope}%
\pgfsys@transformshift{5.643345in}{1.713321in}%
\pgfsys@useobject{currentmarker}{}%
\end{pgfscope}%
\end{pgfscope}%
\begin{pgfscope}%
\pgfsetbuttcap%
\pgfsetroundjoin%
\definecolor{currentfill}{rgb}{0.000000,0.000000,0.000000}%
\pgfsetfillcolor{currentfill}%
\pgfsetlinewidth{0.602250pt}%
\definecolor{currentstroke}{rgb}{0.000000,0.000000,0.000000}%
\pgfsetstrokecolor{currentstroke}%
\pgfsetdash{}{0pt}%
\pgfsys@defobject{currentmarker}{\pgfqpoint{0.000000in}{0.000000in}}{\pgfqpoint{0.027778in}{0.000000in}}{%
\pgfpathmoveto{\pgfqpoint{0.000000in}{0.000000in}}%
\pgfpathlineto{\pgfqpoint{0.027778in}{0.000000in}}%
\pgfusepath{stroke,fill}%
}%
\begin{pgfscope}%
\pgfsys@transformshift{5.643345in}{1.806128in}%
\pgfsys@useobject{currentmarker}{}%
\end{pgfscope}%
\end{pgfscope}%
\begin{pgfscope}%
\pgfsetbuttcap%
\pgfsetroundjoin%
\definecolor{currentfill}{rgb}{0.000000,0.000000,0.000000}%
\pgfsetfillcolor{currentfill}%
\pgfsetlinewidth{0.602250pt}%
\definecolor{currentstroke}{rgb}{0.000000,0.000000,0.000000}%
\pgfsetstrokecolor{currentstroke}%
\pgfsetdash{}{0pt}%
\pgfsys@defobject{currentmarker}{\pgfqpoint{0.000000in}{0.000000in}}{\pgfqpoint{0.027778in}{0.000000in}}{%
\pgfpathmoveto{\pgfqpoint{0.000000in}{0.000000in}}%
\pgfpathlineto{\pgfqpoint{0.027778in}{0.000000in}}%
\pgfusepath{stroke,fill}%
}%
\begin{pgfscope}%
\pgfsys@transformshift{5.643345in}{2.435311in}%
\pgfsys@useobject{currentmarker}{}%
\end{pgfscope}%
\end{pgfscope}%
\begin{pgfscope}%
\pgfsetbuttcap%
\pgfsetroundjoin%
\definecolor{currentfill}{rgb}{0.000000,0.000000,0.000000}%
\pgfsetfillcolor{currentfill}%
\pgfsetlinewidth{0.602250pt}%
\definecolor{currentstroke}{rgb}{0.000000,0.000000,0.000000}%
\pgfsetstrokecolor{currentstroke}%
\pgfsetdash{}{0pt}%
\pgfsys@defobject{currentmarker}{\pgfqpoint{0.000000in}{0.000000in}}{\pgfqpoint{0.027778in}{0.000000in}}{%
\pgfpathmoveto{\pgfqpoint{0.000000in}{0.000000in}}%
\pgfpathlineto{\pgfqpoint{0.027778in}{0.000000in}}%
\pgfusepath{stroke,fill}%
}%
\begin{pgfscope}%
\pgfsys@transformshift{5.643345in}{2.754797in}%
\pgfsys@useobject{currentmarker}{}%
\end{pgfscope}%
\end{pgfscope}%
\begin{pgfscope}%
\pgfsetbuttcap%
\pgfsetroundjoin%
\definecolor{currentfill}{rgb}{0.000000,0.000000,0.000000}%
\pgfsetfillcolor{currentfill}%
\pgfsetlinewidth{0.602250pt}%
\definecolor{currentstroke}{rgb}{0.000000,0.000000,0.000000}%
\pgfsetstrokecolor{currentstroke}%
\pgfsetdash{}{0pt}%
\pgfsys@defobject{currentmarker}{\pgfqpoint{0.000000in}{0.000000in}}{\pgfqpoint{0.027778in}{0.000000in}}{%
\pgfpathmoveto{\pgfqpoint{0.000000in}{0.000000in}}%
\pgfpathlineto{\pgfqpoint{0.027778in}{0.000000in}}%
\pgfusepath{stroke,fill}%
}%
\begin{pgfscope}%
\pgfsys@transformshift{5.643345in}{2.981475in}%
\pgfsys@useobject{currentmarker}{}%
\end{pgfscope}%
\end{pgfscope}%
\begin{pgfscope}%
\pgfsetbuttcap%
\pgfsetroundjoin%
\definecolor{currentfill}{rgb}{0.000000,0.000000,0.000000}%
\pgfsetfillcolor{currentfill}%
\pgfsetlinewidth{0.602250pt}%
\definecolor{currentstroke}{rgb}{0.000000,0.000000,0.000000}%
\pgfsetstrokecolor{currentstroke}%
\pgfsetdash{}{0pt}%
\pgfsys@defobject{currentmarker}{\pgfqpoint{0.000000in}{0.000000in}}{\pgfqpoint{0.027778in}{0.000000in}}{%
\pgfpathmoveto{\pgfqpoint{0.000000in}{0.000000in}}%
\pgfpathlineto{\pgfqpoint{0.027778in}{0.000000in}}%
\pgfusepath{stroke,fill}%
}%
\begin{pgfscope}%
\pgfsys@transformshift{5.643345in}{3.157301in}%
\pgfsys@useobject{currentmarker}{}%
\end{pgfscope}%
\end{pgfscope}%
\begin{pgfscope}%
\pgfsetbuttcap%
\pgfsetroundjoin%
\definecolor{currentfill}{rgb}{0.000000,0.000000,0.000000}%
\pgfsetfillcolor{currentfill}%
\pgfsetlinewidth{0.602250pt}%
\definecolor{currentstroke}{rgb}{0.000000,0.000000,0.000000}%
\pgfsetstrokecolor{currentstroke}%
\pgfsetdash{}{0pt}%
\pgfsys@defobject{currentmarker}{\pgfqpoint{0.000000in}{0.000000in}}{\pgfqpoint{0.027778in}{0.000000in}}{%
\pgfpathmoveto{\pgfqpoint{0.000000in}{0.000000in}}%
\pgfpathlineto{\pgfqpoint{0.027778in}{0.000000in}}%
\pgfusepath{stroke,fill}%
}%
\begin{pgfscope}%
\pgfsys@transformshift{5.643345in}{3.300961in}%
\pgfsys@useobject{currentmarker}{}%
\end{pgfscope}%
\end{pgfscope}%
\begin{pgfscope}%
\pgfsetbuttcap%
\pgfsetroundjoin%
\definecolor{currentfill}{rgb}{0.000000,0.000000,0.000000}%
\pgfsetfillcolor{currentfill}%
\pgfsetlinewidth{0.602250pt}%
\definecolor{currentstroke}{rgb}{0.000000,0.000000,0.000000}%
\pgfsetstrokecolor{currentstroke}%
\pgfsetdash{}{0pt}%
\pgfsys@defobject{currentmarker}{\pgfqpoint{0.000000in}{0.000000in}}{\pgfqpoint{0.027778in}{0.000000in}}{%
\pgfpathmoveto{\pgfqpoint{0.000000in}{0.000000in}}%
\pgfpathlineto{\pgfqpoint{0.027778in}{0.000000in}}%
\pgfusepath{stroke,fill}%
}%
\begin{pgfscope}%
\pgfsys@transformshift{5.643345in}{3.422424in}%
\pgfsys@useobject{currentmarker}{}%
\end{pgfscope}%
\end{pgfscope}%
\begin{pgfscope}%
\pgfsetbuttcap%
\pgfsetroundjoin%
\definecolor{currentfill}{rgb}{0.000000,0.000000,0.000000}%
\pgfsetfillcolor{currentfill}%
\pgfsetlinewidth{0.602250pt}%
\definecolor{currentstroke}{rgb}{0.000000,0.000000,0.000000}%
\pgfsetstrokecolor{currentstroke}%
\pgfsetdash{}{0pt}%
\pgfsys@defobject{currentmarker}{\pgfqpoint{0.000000in}{0.000000in}}{\pgfqpoint{0.027778in}{0.000000in}}{%
\pgfpathmoveto{\pgfqpoint{0.000000in}{0.000000in}}%
\pgfpathlineto{\pgfqpoint{0.027778in}{0.000000in}}%
\pgfusepath{stroke,fill}%
}%
\begin{pgfscope}%
\pgfsys@transformshift{5.643345in}{3.527640in}%
\pgfsys@useobject{currentmarker}{}%
\end{pgfscope}%
\end{pgfscope}%
\begin{pgfscope}%
\pgfsetbuttcap%
\pgfsetroundjoin%
\definecolor{currentfill}{rgb}{0.000000,0.000000,0.000000}%
\pgfsetfillcolor{currentfill}%
\pgfsetlinewidth{0.602250pt}%
\definecolor{currentstroke}{rgb}{0.000000,0.000000,0.000000}%
\pgfsetstrokecolor{currentstroke}%
\pgfsetdash{}{0pt}%
\pgfsys@defobject{currentmarker}{\pgfqpoint{0.000000in}{0.000000in}}{\pgfqpoint{0.027778in}{0.000000in}}{%
\pgfpathmoveto{\pgfqpoint{0.000000in}{0.000000in}}%
\pgfpathlineto{\pgfqpoint{0.027778in}{0.000000in}}%
\pgfusepath{stroke,fill}%
}%
\begin{pgfscope}%
\pgfsys@transformshift{5.643345in}{3.620447in}%
\pgfsys@useobject{currentmarker}{}%
\end{pgfscope}%
\end{pgfscope}%
\begin{pgfscope}%
\pgfsetbuttcap%
\pgfsetroundjoin%
\definecolor{currentfill}{rgb}{0.000000,0.000000,0.000000}%
\pgfsetfillcolor{currentfill}%
\pgfsetlinewidth{0.602250pt}%
\definecolor{currentstroke}{rgb}{0.000000,0.000000,0.000000}%
\pgfsetstrokecolor{currentstroke}%
\pgfsetdash{}{0pt}%
\pgfsys@defobject{currentmarker}{\pgfqpoint{0.000000in}{0.000000in}}{\pgfqpoint{0.027778in}{0.000000in}}{%
\pgfpathmoveto{\pgfqpoint{0.000000in}{0.000000in}}%
\pgfpathlineto{\pgfqpoint{0.027778in}{0.000000in}}%
\pgfusepath{stroke,fill}%
}%
\begin{pgfscope}%
\pgfsys@transformshift{5.643345in}{4.249630in}%
\pgfsys@useobject{currentmarker}{}%
\end{pgfscope}%
\end{pgfscope}%
\begin{pgfscope}%
\pgfsetbuttcap%
\pgfsetroundjoin%
\definecolor{currentfill}{rgb}{0.000000,0.000000,0.000000}%
\pgfsetfillcolor{currentfill}%
\pgfsetlinewidth{0.602250pt}%
\definecolor{currentstroke}{rgb}{0.000000,0.000000,0.000000}%
\pgfsetstrokecolor{currentstroke}%
\pgfsetdash{}{0pt}%
\pgfsys@defobject{currentmarker}{\pgfqpoint{0.000000in}{0.000000in}}{\pgfqpoint{0.027778in}{0.000000in}}{%
\pgfpathmoveto{\pgfqpoint{0.000000in}{0.000000in}}%
\pgfpathlineto{\pgfqpoint{0.027778in}{0.000000in}}%
\pgfusepath{stroke,fill}%
}%
\begin{pgfscope}%
\pgfsys@transformshift{5.643345in}{4.569116in}%
\pgfsys@useobject{currentmarker}{}%
\end{pgfscope}%
\end{pgfscope}%
\begin{pgfscope}%
\pgfsetbuttcap%
\pgfsetroundjoin%
\definecolor{currentfill}{rgb}{0.000000,0.000000,0.000000}%
\pgfsetfillcolor{currentfill}%
\pgfsetlinewidth{0.602250pt}%
\definecolor{currentstroke}{rgb}{0.000000,0.000000,0.000000}%
\pgfsetstrokecolor{currentstroke}%
\pgfsetdash{}{0pt}%
\pgfsys@defobject{currentmarker}{\pgfqpoint{0.000000in}{0.000000in}}{\pgfqpoint{0.027778in}{0.000000in}}{%
\pgfpathmoveto{\pgfqpoint{0.000000in}{0.000000in}}%
\pgfpathlineto{\pgfqpoint{0.027778in}{0.000000in}}%
\pgfusepath{stroke,fill}%
}%
\begin{pgfscope}%
\pgfsys@transformshift{5.643345in}{4.795794in}%
\pgfsys@useobject{currentmarker}{}%
\end{pgfscope}%
\end{pgfscope}%
\begin{pgfscope}%
\pgfsetbuttcap%
\pgfsetroundjoin%
\definecolor{currentfill}{rgb}{0.000000,0.000000,0.000000}%
\pgfsetfillcolor{currentfill}%
\pgfsetlinewidth{0.602250pt}%
\definecolor{currentstroke}{rgb}{0.000000,0.000000,0.000000}%
\pgfsetstrokecolor{currentstroke}%
\pgfsetdash{}{0pt}%
\pgfsys@defobject{currentmarker}{\pgfqpoint{0.000000in}{0.000000in}}{\pgfqpoint{0.027778in}{0.000000in}}{%
\pgfpathmoveto{\pgfqpoint{0.000000in}{0.000000in}}%
\pgfpathlineto{\pgfqpoint{0.027778in}{0.000000in}}%
\pgfusepath{stroke,fill}%
}%
\begin{pgfscope}%
\pgfsys@transformshift{5.643345in}{4.971620in}%
\pgfsys@useobject{currentmarker}{}%
\end{pgfscope}%
\end{pgfscope}%
\begin{pgfscope}%
\pgfsetbuttcap%
\pgfsetroundjoin%
\definecolor{currentfill}{rgb}{0.000000,0.000000,0.000000}%
\pgfsetfillcolor{currentfill}%
\pgfsetlinewidth{0.602250pt}%
\definecolor{currentstroke}{rgb}{0.000000,0.000000,0.000000}%
\pgfsetstrokecolor{currentstroke}%
\pgfsetdash{}{0pt}%
\pgfsys@defobject{currentmarker}{\pgfqpoint{0.000000in}{0.000000in}}{\pgfqpoint{0.027778in}{0.000000in}}{%
\pgfpathmoveto{\pgfqpoint{0.000000in}{0.000000in}}%
\pgfpathlineto{\pgfqpoint{0.027778in}{0.000000in}}%
\pgfusepath{stroke,fill}%
}%
\begin{pgfscope}%
\pgfsys@transformshift{5.643345in}{5.115280in}%
\pgfsys@useobject{currentmarker}{}%
\end{pgfscope}%
\end{pgfscope}%
\begin{pgfscope}%
\pgfsetbuttcap%
\pgfsetroundjoin%
\definecolor{currentfill}{rgb}{0.000000,0.000000,0.000000}%
\pgfsetfillcolor{currentfill}%
\pgfsetlinewidth{0.602250pt}%
\definecolor{currentstroke}{rgb}{0.000000,0.000000,0.000000}%
\pgfsetstrokecolor{currentstroke}%
\pgfsetdash{}{0pt}%
\pgfsys@defobject{currentmarker}{\pgfqpoint{0.000000in}{0.000000in}}{\pgfqpoint{0.027778in}{0.000000in}}{%
\pgfpathmoveto{\pgfqpoint{0.000000in}{0.000000in}}%
\pgfpathlineto{\pgfqpoint{0.027778in}{0.000000in}}%
\pgfusepath{stroke,fill}%
}%
\begin{pgfscope}%
\pgfsys@transformshift{5.643345in}{5.236743in}%
\pgfsys@useobject{currentmarker}{}%
\end{pgfscope}%
\end{pgfscope}%
\begin{pgfscope}%
\pgfsetbuttcap%
\pgfsetroundjoin%
\definecolor{currentfill}{rgb}{0.000000,0.000000,0.000000}%
\pgfsetfillcolor{currentfill}%
\pgfsetlinewidth{0.602250pt}%
\definecolor{currentstroke}{rgb}{0.000000,0.000000,0.000000}%
\pgfsetstrokecolor{currentstroke}%
\pgfsetdash{}{0pt}%
\pgfsys@defobject{currentmarker}{\pgfqpoint{0.000000in}{0.000000in}}{\pgfqpoint{0.027778in}{0.000000in}}{%
\pgfpathmoveto{\pgfqpoint{0.000000in}{0.000000in}}%
\pgfpathlineto{\pgfqpoint{0.027778in}{0.000000in}}%
\pgfusepath{stroke,fill}%
}%
\begin{pgfscope}%
\pgfsys@transformshift{5.643345in}{5.341959in}%
\pgfsys@useobject{currentmarker}{}%
\end{pgfscope}%
\end{pgfscope}%
\begin{pgfscope}%
\pgfsetbuttcap%
\pgfsetroundjoin%
\definecolor{currentfill}{rgb}{0.000000,0.000000,0.000000}%
\pgfsetfillcolor{currentfill}%
\pgfsetlinewidth{0.602250pt}%
\definecolor{currentstroke}{rgb}{0.000000,0.000000,0.000000}%
\pgfsetstrokecolor{currentstroke}%
\pgfsetdash{}{0pt}%
\pgfsys@defobject{currentmarker}{\pgfqpoint{0.000000in}{0.000000in}}{\pgfqpoint{0.027778in}{0.000000in}}{%
\pgfpathmoveto{\pgfqpoint{0.000000in}{0.000000in}}%
\pgfpathlineto{\pgfqpoint{0.027778in}{0.000000in}}%
\pgfusepath{stroke,fill}%
}%
\begin{pgfscope}%
\pgfsys@transformshift{5.643345in}{5.434766in}%
\pgfsys@useobject{currentmarker}{}%
\end{pgfscope}%
\end{pgfscope}%
\begin{pgfscope}%
\definecolor{textcolor}{rgb}{0.000000,0.000000,0.000000}%
\pgfsetstrokecolor{textcolor}%
\pgfsetfillcolor{textcolor}%
\pgftext[x=6.073832in,y=2.796306in,,top,rotate=90.000000]{\color{textcolor}\rmfamily\fontsize{10.000000}{12.000000}\selectfont Photonen}%
\end{pgfscope}%
\begin{pgfscope}%
\pgfsetrectcap%
\pgfsetmiterjoin%
\pgfsetlinewidth{0.803000pt}%
\definecolor{currentstroke}{rgb}{0.000000,0.000000,0.000000}%
\pgfsetstrokecolor{currentstroke}%
\pgfsetdash{}{0pt}%
\pgfpathmoveto{\pgfqpoint{5.501056in}{0.074827in}}%
\pgfpathlineto{\pgfqpoint{5.572200in}{0.074827in}}%
\pgfpathlineto{\pgfqpoint{5.643345in}{0.074827in}}%
\pgfpathlineto{\pgfqpoint{5.643345in}{5.517785in}}%
\pgfpathlineto{\pgfqpoint{5.572200in}{5.517785in}}%
\pgfpathlineto{\pgfqpoint{5.501056in}{5.517785in}}%
\pgfpathlineto{\pgfqpoint{5.501056in}{0.074827in}}%
\pgfpathclose%
\pgfusepath{stroke}%
\end{pgfscope}%
\end{pgfpicture}%
\makeatother%
\endgroup%

    \caption{Anzahl von den detektierten Photonen mithilfe des Schwellenwert-Algorithmuses mit dem Schwellenwert $s_V$ (a) \SI{50}{\adu}, (b) \SI{100}{\adu}, (c) \SI{125}{\adu}, (d) \SI{150}{\adu}, (e) \SI{170}{\adu}, (f) \SI{180}{\adu}, (g) \SI{200}{\adu}, (h) \SI{220}{\adu} und (g) \SI{260}{\adu}. Aufsummiert werden jeweils \num{50000} Aufnahmen. Die dunklere horizontale Linie im Direktstrahlzentrum entspricht der Gd M5 Absorptionskante.}
    \label{fig:th_50_100_125_150_170_180_200_220_260}
\end{figure}
\noindent
Bei dem Einsatz der höheren Schwellenwerte $s_V = \SI{150}{\adu}$, $\SI{170}{\adu}$ und $\SI{180}{\adu}$ (Abb. \ref{fig:th_50_100_125_150_170_180_200_220_260}d, e und f), wodurch weniger Rauschen als Photonen fehldetektiert wird, sind die Summen wesentlich kontrastarmer im Vergleich mit den eingesetzten Schwellenwerten $s_V = \SI{100}{\adu}$ (Abb. \ref{fig:th_50_100_125_150_170_180_200_220_260}b), was auf die Spreizung des Ein-Photon-Signals zurückzuführen ist. Als Nächstes wird das Streusignal in dieser Summe ausgewertet. 

% \noindent
% Bei der Summe von $N_P = 2\pi\cdot\SI{75}{\pixel} \approx \SI{471}{\pixel}$ ergeben sich die folgenden Erwartungswerte
% \begin{equation}
%     \begin{split}
%         S_\text{EW} &= \SI{926}{\photons}\\
%         N_\text{EW} &= \SI{42}{\photons},
%     \end{split}
% \end{equation}
% und das 
% \begin{equation}
%     \text{\gls{snr}} \approx \num{22.28},
% \end{equation}
% \noindent
% das mit dem Wert $\sqrt{471}$ gut übereinstimmt.
\subsection{Auswertung des Streusignals}
Wichtig zu beachten, dass die Summen der Aufnahmen $I_\text{ges}$, die mit Schwellenwert-Algorithmus bearbeitet sind, enthalten die fehldetektierten Photonen. Daher gilt:
\begin{equation}
    I_\text{ges} = N_A N_P \left[\text{\gls{photnenfluss}}\cdot \text{\gls{qe}} + \text{\gls{fdpa}}\right] = \underbrace{N_A N_P \left[\text{\gls{photnenfluss}}\cdot \text{\gls{qe}}\right]}_{S} + N_P N_A\cdot\text{\gls{fdpa}}
\end{equation}
So muss die \gls{fdpa} (vgl. Gl. (\ref{eq:signal_ew})) von jedem Pixel als Offset abgezogen werden:
\begin{equation}
    S = I_\text{ges} - \underbrace{N_A\cdot\text{\gls{fdpa}}(s_V)}_{\Delta_\text{\gls{fdpa}}}
    \label{eq:signal_int_fpda}
\end{equation}

\noindent
Es können die Pixel in den Summen von Aufnahmen (Abb. \ref{fig:th_180_450_600}) gefunden werden, wo sich die Photonen bei der Erhöhung von $s_V$ bis zu \SI{600}{\adu} detektieren lassen. Solche Pixel werden aussortiert, weil es sich dabei um hochenergetische kosmische Strahlung oder um die von der \gls{pxs} emittierten Elektronen handelt.
\begin{figure}[H]
    \centering
    %% Creator: Matplotlib, PGF backend
%%
%% To include the figure in your LaTeX document, write
%%   \input{<filename>.pgf}
%%
%% Make sure the required packages are loaded in your preamble
%%   \usepackage{pgf}
%%
%% Also ensure that all the required font packages are loaded; for instance,
%% the lmodern package is sometimes necessary when using math font.
%%   \usepackage{lmodern}
%%
%% Figures using additional raster images can only be included by \input if
%% they are in the same directory as the main LaTeX file. For loading figures
%% from other directories you can use the `import` package
%%   \usepackage{import}
%%
%% and then include the figures with
%%   \import{<path to file>}{<filename>.pgf}
%%
%% Matplotlib used the following preamble
%%   \usepackage[utf8]{inputenc} \usepackage[T1]{fontenc} \usepackage[ngerman]{babel} \usepackage{hyperref} \usepackage[sorting=none]{biblatex} \usepackage{amsmath} \usepackage[output-decimal-marker={,}]{siunitx} \sisetup{per-mode=fraction, separate-uncertainty = true, locale = DE} \usepackage[acronym, toc, section=section, nonumberlist, nopostdot]{glossaries-extra} \usepackage{lmodern}
%%
\begingroup%
\makeatletter%
\begin{pgfpicture}%
\pgfpathrectangle{\pgfpointorigin}{\pgfqpoint{6.180778in}{2.299710in}}%
\pgfusepath{use as bounding box, clip}%
\begin{pgfscope}%
\pgfsetbuttcap%
\pgfsetmiterjoin%
\pgfsetlinewidth{0.000000pt}%
\definecolor{currentstroke}{rgb}{1.000000,1.000000,1.000000}%
\pgfsetstrokecolor{currentstroke}%
\pgfsetstrokeopacity{0.000000}%
\pgfsetdash{}{0pt}%
\pgfpathmoveto{\pgfqpoint{0.000000in}{0.000000in}}%
\pgfpathlineto{\pgfqpoint{6.180778in}{0.000000in}}%
\pgfpathlineto{\pgfqpoint{6.180778in}{2.299710in}}%
\pgfpathlineto{\pgfqpoint{0.000000in}{2.299710in}}%
\pgfpathlineto{\pgfqpoint{0.000000in}{0.000000in}}%
\pgfpathclose%
\pgfusepath{}%
\end{pgfscope}%
\begin{pgfscope}%
\pgfsetbuttcap%
\pgfsetmiterjoin%
\definecolor{currentfill}{rgb}{1.000000,1.000000,1.000000}%
\pgfsetfillcolor{currentfill}%
\pgfsetlinewidth{0.000000pt}%
\definecolor{currentstroke}{rgb}{0.000000,0.000000,0.000000}%
\pgfsetstrokecolor{currentstroke}%
\pgfsetstrokeopacity{0.000000}%
\pgfsetdash{}{0pt}%
\pgfpathmoveto{\pgfqpoint{0.048611in}{0.061342in}}%
\pgfpathlineto{\pgfqpoint{1.848810in}{0.061342in}}%
\pgfpathlineto{\pgfqpoint{1.848810in}{1.768703in}}%
\pgfpathlineto{\pgfqpoint{0.048611in}{1.768703in}}%
\pgfpathlineto{\pgfqpoint{0.048611in}{0.061342in}}%
\pgfpathclose%
\pgfusepath{fill}%
\end{pgfscope}%
\begin{pgfscope}%
\pgfsys@transformshift{0.142000in}{0.063710in}%
\pgftext[left,bottom]{\includegraphics[interpolate=true,width=1.614000in,height=1.692000in]{th_180_450_600-img0.png}}%
\end{pgfscope}%
\begin{pgfscope}%
\pgfpathrectangle{\pgfqpoint{0.048611in}{0.061342in}}{\pgfqpoint{1.800199in}{1.707361in}}%
\pgfusepath{clip}%
\pgfsetbuttcap%
\pgfsetmiterjoin%
\pgfsetlinewidth{1.003750pt}%
\definecolor{currentstroke}{rgb}{0.000000,0.501961,0.000000}%
\pgfsetstrokecolor{currentstroke}%
\pgfsetdash{}{0pt}%
\pgfpathmoveto{\pgfqpoint{0.247803in}{0.722944in}}%
\pgfpathcurveto{\pgfqpoint{0.257237in}{0.722944in}}{\pgfqpoint{0.266285in}{0.726692in}}{\pgfqpoint{0.272955in}{0.733362in}}%
\pgfpathcurveto{\pgfqpoint{0.279625in}{0.740033in}}{\pgfqpoint{0.283373in}{0.749081in}}{\pgfqpoint{0.283373in}{0.758514in}}%
\pgfpathcurveto{\pgfqpoint{0.283373in}{0.767947in}}{\pgfqpoint{0.279625in}{0.776996in}}{\pgfqpoint{0.272955in}{0.783666in}}%
\pgfpathcurveto{\pgfqpoint{0.266285in}{0.790336in}}{\pgfqpoint{0.257237in}{0.794084in}}{\pgfqpoint{0.247803in}{0.794084in}}%
\pgfpathcurveto{\pgfqpoint{0.238370in}{0.794084in}}{\pgfqpoint{0.229322in}{0.790336in}}{\pgfqpoint{0.222651in}{0.783666in}}%
\pgfpathcurveto{\pgfqpoint{0.215981in}{0.776996in}}{\pgfqpoint{0.212233in}{0.767947in}}{\pgfqpoint{0.212233in}{0.758514in}}%
\pgfpathcurveto{\pgfqpoint{0.212233in}{0.749081in}}{\pgfqpoint{0.215981in}{0.740033in}}{\pgfqpoint{0.222651in}{0.733362in}}%
\pgfpathcurveto{\pgfqpoint{0.229322in}{0.726692in}}{\pgfqpoint{0.238370in}{0.722944in}}{\pgfqpoint{0.247803in}{0.722944in}}%
\pgfpathlineto{\pgfqpoint{0.247803in}{0.722944in}}%
\pgfpathclose%
\pgfusepath{stroke}%
\end{pgfscope}%
\begin{pgfscope}%
\pgfpathrectangle{\pgfqpoint{0.048611in}{0.061342in}}{\pgfqpoint{1.800199in}{1.707361in}}%
\pgfusepath{clip}%
\pgfsetbuttcap%
\pgfsetmiterjoin%
\pgfsetlinewidth{1.003750pt}%
\definecolor{currentstroke}{rgb}{0.000000,0.501961,0.000000}%
\pgfsetstrokecolor{currentstroke}%
\pgfsetdash{}{0pt}%
\pgfpathmoveto{\pgfqpoint{1.201080in}{1.683335in}}%
\pgfpathcurveto{\pgfqpoint{1.210513in}{1.683335in}}{\pgfqpoint{1.219562in}{1.687083in}}{\pgfqpoint{1.226232in}{1.693753in}}%
\pgfpathcurveto{\pgfqpoint{1.232902in}{1.700423in}}{\pgfqpoint{1.236650in}{1.709472in}}{\pgfqpoint{1.236650in}{1.718905in}}%
\pgfpathcurveto{\pgfqpoint{1.236650in}{1.728338in}}{\pgfqpoint{1.232902in}{1.737386in}}{\pgfqpoint{1.226232in}{1.744057in}}%
\pgfpathcurveto{\pgfqpoint{1.219562in}{1.750727in}}{\pgfqpoint{1.210513in}{1.754475in}}{\pgfqpoint{1.201080in}{1.754475in}}%
\pgfpathcurveto{\pgfqpoint{1.191647in}{1.754475in}}{\pgfqpoint{1.182599in}{1.750727in}}{\pgfqpoint{1.175928in}{1.744057in}}%
\pgfpathcurveto{\pgfqpoint{1.169258in}{1.737386in}}{\pgfqpoint{1.165510in}{1.728338in}}{\pgfqpoint{1.165510in}{1.718905in}}%
\pgfpathcurveto{\pgfqpoint{1.165510in}{1.709472in}}{\pgfqpoint{1.169258in}{1.700423in}}{\pgfqpoint{1.175928in}{1.693753in}}%
\pgfpathcurveto{\pgfqpoint{1.182599in}{1.687083in}}{\pgfqpoint{1.191647in}{1.683335in}}{\pgfqpoint{1.201080in}{1.683335in}}%
\pgfpathlineto{\pgfqpoint{1.201080in}{1.683335in}}%
\pgfpathclose%
\pgfusepath{stroke}%
\end{pgfscope}%
\begin{pgfscope}%
\pgfpathrectangle{\pgfqpoint{0.048611in}{0.061342in}}{\pgfqpoint{1.800199in}{1.707361in}}%
\pgfusepath{clip}%
\pgfsetbuttcap%
\pgfsetmiterjoin%
\pgfsetlinewidth{1.003750pt}%
\definecolor{currentstroke}{rgb}{0.000000,0.501961,0.000000}%
\pgfsetstrokecolor{currentstroke}%
\pgfsetdash{}{0pt}%
\pgfpathmoveto{\pgfqpoint{1.599464in}{1.156898in}}%
\pgfpathcurveto{\pgfqpoint{1.608898in}{1.156898in}}{\pgfqpoint{1.617946in}{1.160646in}}{\pgfqpoint{1.624616in}{1.167317in}}%
\pgfpathcurveto{\pgfqpoint{1.631287in}{1.173987in}}{\pgfqpoint{1.635034in}{1.183035in}}{\pgfqpoint{1.635034in}{1.192468in}}%
\pgfpathcurveto{\pgfqpoint{1.635034in}{1.201902in}}{\pgfqpoint{1.631287in}{1.210950in}}{\pgfqpoint{1.624616in}{1.217620in}}%
\pgfpathcurveto{\pgfqpoint{1.617946in}{1.224291in}}{\pgfqpoint{1.608898in}{1.228039in}}{\pgfqpoint{1.599464in}{1.228039in}}%
\pgfpathcurveto{\pgfqpoint{1.590031in}{1.228039in}}{\pgfqpoint{1.580983in}{1.224291in}}{\pgfqpoint{1.574313in}{1.217620in}}%
\pgfpathcurveto{\pgfqpoint{1.567642in}{1.210950in}}{\pgfqpoint{1.563894in}{1.201902in}}{\pgfqpoint{1.563894in}{1.192468in}}%
\pgfpathcurveto{\pgfqpoint{1.563894in}{1.183035in}}{\pgfqpoint{1.567642in}{1.173987in}}{\pgfqpoint{1.574313in}{1.167317in}}%
\pgfpathcurveto{\pgfqpoint{1.580983in}{1.160646in}}{\pgfqpoint{1.590031in}{1.156898in}}{\pgfqpoint{1.599464in}{1.156898in}}%
\pgfpathlineto{\pgfqpoint{1.599464in}{1.156898in}}%
\pgfpathclose%
\pgfusepath{stroke}%
\end{pgfscope}%
\begin{pgfscope}%
\pgfpathrectangle{\pgfqpoint{0.048611in}{0.061342in}}{\pgfqpoint{1.800199in}{1.707361in}}%
\pgfusepath{clip}%
\pgfsetbuttcap%
\pgfsetmiterjoin%
\pgfsetlinewidth{1.003750pt}%
\definecolor{currentstroke}{rgb}{0.000000,0.501961,0.000000}%
\pgfsetstrokecolor{currentstroke}%
\pgfsetdash{}{0pt}%
\pgfpathmoveto{\pgfqpoint{1.350474in}{0.075570in}}%
\pgfpathcurveto{\pgfqpoint{1.359908in}{0.075570in}}{\pgfqpoint{1.368956in}{0.079317in}}{\pgfqpoint{1.375626in}{0.085988in}}%
\pgfpathcurveto{\pgfqpoint{1.382296in}{0.092658in}}{\pgfqpoint{1.386044in}{0.101706in}}{\pgfqpoint{1.386044in}{0.111140in}}%
\pgfpathcurveto{\pgfqpoint{1.386044in}{0.120573in}}{\pgfqpoint{1.382296in}{0.129621in}}{\pgfqpoint{1.375626in}{0.136291in}}%
\pgfpathcurveto{\pgfqpoint{1.368956in}{0.142962in}}{\pgfqpoint{1.359908in}{0.146710in}}{\pgfqpoint{1.350474in}{0.146710in}}%
\pgfpathcurveto{\pgfqpoint{1.341041in}{0.146710in}}{\pgfqpoint{1.331993in}{0.142962in}}{\pgfqpoint{1.325322in}{0.136291in}}%
\pgfpathcurveto{\pgfqpoint{1.318652in}{0.129621in}}{\pgfqpoint{1.314904in}{0.120573in}}{\pgfqpoint{1.314904in}{0.111140in}}%
\pgfpathcurveto{\pgfqpoint{1.314904in}{0.101706in}}{\pgfqpoint{1.318652in}{0.092658in}}{\pgfqpoint{1.325322in}{0.085988in}}%
\pgfpathcurveto{\pgfqpoint{1.331993in}{0.079317in}}{\pgfqpoint{1.341041in}{0.075570in}}{\pgfqpoint{1.350474in}{0.075570in}}%
\pgfpathlineto{\pgfqpoint{1.350474in}{0.075570in}}%
\pgfpathclose%
\pgfusepath{stroke}%
\end{pgfscope}%
\begin{pgfscope}%
\pgfpathrectangle{\pgfqpoint{0.048611in}{0.061342in}}{\pgfqpoint{1.800199in}{1.707361in}}%
\pgfusepath{clip}%
\pgfsetbuttcap%
\pgfsetmiterjoin%
\pgfsetlinewidth{1.003750pt}%
\definecolor{currentstroke}{rgb}{0.000000,0.501961,0.000000}%
\pgfsetstrokecolor{currentstroke}%
\pgfsetdash{}{0pt}%
\pgfpathmoveto{\pgfqpoint{1.336246in}{1.035960in}}%
\pgfpathcurveto{\pgfqpoint{1.345679in}{1.035960in}}{\pgfqpoint{1.354728in}{1.039708in}}{\pgfqpoint{1.361398in}{1.046379in}}%
\pgfpathcurveto{\pgfqpoint{1.368068in}{1.053049in}}{\pgfqpoint{1.371816in}{1.062097in}}{\pgfqpoint{1.371816in}{1.071530in}}%
\pgfpathcurveto{\pgfqpoint{1.371816in}{1.080964in}}{\pgfqpoint{1.368068in}{1.090012in}}{\pgfqpoint{1.361398in}{1.096682in}}%
\pgfpathcurveto{\pgfqpoint{1.354728in}{1.103353in}}{\pgfqpoint{1.345679in}{1.107100in}}{\pgfqpoint{1.336246in}{1.107100in}}%
\pgfpathcurveto{\pgfqpoint{1.326813in}{1.107100in}}{\pgfqpoint{1.317765in}{1.103353in}}{\pgfqpoint{1.311094in}{1.096682in}}%
\pgfpathcurveto{\pgfqpoint{1.304424in}{1.090012in}}{\pgfqpoint{1.300676in}{1.080964in}}{\pgfqpoint{1.300676in}{1.071530in}}%
\pgfpathcurveto{\pgfqpoint{1.300676in}{1.062097in}}{\pgfqpoint{1.304424in}{1.053049in}}{\pgfqpoint{1.311094in}{1.046379in}}%
\pgfpathcurveto{\pgfqpoint{1.317765in}{1.039708in}}{\pgfqpoint{1.326813in}{1.035960in}}{\pgfqpoint{1.336246in}{1.035960in}}%
\pgfpathlineto{\pgfqpoint{1.336246in}{1.035960in}}%
\pgfpathclose%
\pgfusepath{stroke}%
\end{pgfscope}%
\begin{pgfscope}%
\pgfpathrectangle{\pgfqpoint{0.048611in}{0.061342in}}{\pgfqpoint{1.800199in}{1.707361in}}%
\pgfusepath{clip}%
\pgfsetbuttcap%
\pgfsetmiterjoin%
\pgfsetlinewidth{1.003750pt}%
\definecolor{currentstroke}{rgb}{0.000000,0.501961,0.000000}%
\pgfsetstrokecolor{currentstroke}%
\pgfsetdash{}{0pt}%
\pgfpathmoveto{\pgfqpoint{1.613692in}{0.829654in}}%
\pgfpathcurveto{\pgfqpoint{1.623126in}{0.829654in}}{\pgfqpoint{1.632174in}{0.833402in}}{\pgfqpoint{1.638844in}{0.840072in}}%
\pgfpathcurveto{\pgfqpoint{1.645515in}{0.846743in}}{\pgfqpoint{1.649262in}{0.855791in}}{\pgfqpoint{1.649262in}{0.865224in}}%
\pgfpathcurveto{\pgfqpoint{1.649262in}{0.874657in}}{\pgfqpoint{1.645515in}{0.883706in}}{\pgfqpoint{1.638844in}{0.890376in}}%
\pgfpathcurveto{\pgfqpoint{1.632174in}{0.897046in}}{\pgfqpoint{1.623126in}{0.900794in}}{\pgfqpoint{1.613692in}{0.900794in}}%
\pgfpathcurveto{\pgfqpoint{1.604259in}{0.900794in}}{\pgfqpoint{1.595211in}{0.897046in}}{\pgfqpoint{1.588541in}{0.890376in}}%
\pgfpathcurveto{\pgfqpoint{1.581870in}{0.883706in}}{\pgfqpoint{1.578122in}{0.874657in}}{\pgfqpoint{1.578122in}{0.865224in}}%
\pgfpathcurveto{\pgfqpoint{1.578122in}{0.855791in}}{\pgfqpoint{1.581870in}{0.846743in}}{\pgfqpoint{1.588541in}{0.840072in}}%
\pgfpathcurveto{\pgfqpoint{1.595211in}{0.833402in}}{\pgfqpoint{1.604259in}{0.829654in}}{\pgfqpoint{1.613692in}{0.829654in}}%
\pgfpathlineto{\pgfqpoint{1.613692in}{0.829654in}}%
\pgfpathclose%
\pgfusepath{stroke}%
\end{pgfscope}%
\begin{pgfscope}%
\pgfpathrectangle{\pgfqpoint{0.048611in}{0.061342in}}{\pgfqpoint{1.800199in}{1.707361in}}%
\pgfusepath{clip}%
\pgfsetbuttcap%
\pgfsetmiterjoin%
\pgfsetlinewidth{1.003750pt}%
\definecolor{currentstroke}{rgb}{0.000000,0.501961,0.000000}%
\pgfsetstrokecolor{currentstroke}%
\pgfsetdash{}{0pt}%
\pgfpathmoveto{\pgfqpoint{1.542552in}{0.331674in}}%
\pgfpathcurveto{\pgfqpoint{1.551986in}{0.331674in}}{\pgfqpoint{1.561034in}{0.335422in}}{\pgfqpoint{1.567704in}{0.342092in}}%
\pgfpathcurveto{\pgfqpoint{1.574375in}{0.348762in}}{\pgfqpoint{1.578122in}{0.357811in}}{\pgfqpoint{1.578122in}{0.367244in}}%
\pgfpathcurveto{\pgfqpoint{1.578122in}{0.376677in}}{\pgfqpoint{1.574375in}{0.385725in}}{\pgfqpoint{1.567704in}{0.392396in}}%
\pgfpathcurveto{\pgfqpoint{1.561034in}{0.399066in}}{\pgfqpoint{1.551986in}{0.402814in}}{\pgfqpoint{1.542552in}{0.402814in}}%
\pgfpathcurveto{\pgfqpoint{1.533119in}{0.402814in}}{\pgfqpoint{1.524071in}{0.399066in}}{\pgfqpoint{1.517401in}{0.392396in}}%
\pgfpathcurveto{\pgfqpoint{1.510730in}{0.385725in}}{\pgfqpoint{1.506982in}{0.376677in}}{\pgfqpoint{1.506982in}{0.367244in}}%
\pgfpathcurveto{\pgfqpoint{1.506982in}{0.357811in}}{\pgfqpoint{1.510730in}{0.348762in}}{\pgfqpoint{1.517401in}{0.342092in}}%
\pgfpathcurveto{\pgfqpoint{1.524071in}{0.335422in}}{\pgfqpoint{1.533119in}{0.331674in}}{\pgfqpoint{1.542552in}{0.331674in}}%
\pgfpathlineto{\pgfqpoint{1.542552in}{0.331674in}}%
\pgfpathclose%
\pgfusepath{stroke}%
\end{pgfscope}%
\begin{pgfscope}%
\pgfpathrectangle{\pgfqpoint{0.048611in}{0.061342in}}{\pgfqpoint{1.800199in}{1.707361in}}%
\pgfusepath{clip}%
\pgfsetbuttcap%
\pgfsetmiterjoin%
\pgfsetlinewidth{1.003750pt}%
\definecolor{currentstroke}{rgb}{0.000000,0.501961,0.000000}%
\pgfsetstrokecolor{currentstroke}%
\pgfsetdash{}{0pt}%
\pgfpathmoveto{\pgfqpoint{0.141093in}{1.491257in}}%
\pgfpathcurveto{\pgfqpoint{0.150526in}{1.491257in}}{\pgfqpoint{0.159575in}{1.495005in}}{\pgfqpoint{0.166245in}{1.501675in}}%
\pgfpathcurveto{\pgfqpoint{0.172915in}{1.508345in}}{\pgfqpoint{0.176663in}{1.517393in}}{\pgfqpoint{0.176663in}{1.526827in}}%
\pgfpathcurveto{\pgfqpoint{0.176663in}{1.536260in}}{\pgfqpoint{0.172915in}{1.545308in}}{\pgfqpoint{0.166245in}{1.551979in}}%
\pgfpathcurveto{\pgfqpoint{0.159575in}{1.558649in}}{\pgfqpoint{0.150526in}{1.562397in}}{\pgfqpoint{0.141093in}{1.562397in}}%
\pgfpathcurveto{\pgfqpoint{0.131660in}{1.562397in}}{\pgfqpoint{0.122612in}{1.558649in}}{\pgfqpoint{0.115941in}{1.551979in}}%
\pgfpathcurveto{\pgfqpoint{0.109271in}{1.545308in}}{\pgfqpoint{0.105523in}{1.536260in}}{\pgfqpoint{0.105523in}{1.526827in}}%
\pgfpathcurveto{\pgfqpoint{0.105523in}{1.517393in}}{\pgfqpoint{0.109271in}{1.508345in}}{\pgfqpoint{0.115941in}{1.501675in}}%
\pgfpathcurveto{\pgfqpoint{0.122612in}{1.495005in}}{\pgfqpoint{0.131660in}{1.491257in}}{\pgfqpoint{0.141093in}{1.491257in}}%
\pgfpathlineto{\pgfqpoint{0.141093in}{1.491257in}}%
\pgfpathclose%
\pgfusepath{stroke}%
\end{pgfscope}%
\begin{pgfscope}%
\pgfpathrectangle{\pgfqpoint{0.048611in}{0.061342in}}{\pgfqpoint{1.800199in}{1.707361in}}%
\pgfusepath{clip}%
\pgfsetbuttcap%
\pgfsetmiterjoin%
\pgfsetlinewidth{1.003750pt}%
\definecolor{currentstroke}{rgb}{0.000000,0.501961,0.000000}%
\pgfsetstrokecolor{currentstroke}%
\pgfsetdash{}{0pt}%
\pgfpathmoveto{\pgfqpoint{1.129940in}{0.417042in}}%
\pgfpathcurveto{\pgfqpoint{1.139373in}{0.417042in}}{\pgfqpoint{1.148422in}{0.420790in}}{\pgfqpoint{1.155092in}{0.427460in}}%
\pgfpathcurveto{\pgfqpoint{1.161762in}{0.434130in}}{\pgfqpoint{1.165510in}{0.443179in}}{\pgfqpoint{1.165510in}{0.452612in}}%
\pgfpathcurveto{\pgfqpoint{1.165510in}{0.462045in}}{\pgfqpoint{1.161762in}{0.471093in}}{\pgfqpoint{1.155092in}{0.477764in}}%
\pgfpathcurveto{\pgfqpoint{1.148422in}{0.484434in}}{\pgfqpoint{1.139373in}{0.488182in}}{\pgfqpoint{1.129940in}{0.488182in}}%
\pgfpathcurveto{\pgfqpoint{1.120507in}{0.488182in}}{\pgfqpoint{1.111459in}{0.484434in}}{\pgfqpoint{1.104788in}{0.477764in}}%
\pgfpathcurveto{\pgfqpoint{1.098118in}{0.471093in}}{\pgfqpoint{1.094370in}{0.462045in}}{\pgfqpoint{1.094370in}{0.452612in}}%
\pgfpathcurveto{\pgfqpoint{1.094370in}{0.443179in}}{\pgfqpoint{1.098118in}{0.434130in}}{\pgfqpoint{1.104788in}{0.427460in}}%
\pgfpathcurveto{\pgfqpoint{1.111459in}{0.420790in}}{\pgfqpoint{1.120507in}{0.417042in}}{\pgfqpoint{1.129940in}{0.417042in}}%
\pgfpathlineto{\pgfqpoint{1.129940in}{0.417042in}}%
\pgfpathclose%
\pgfusepath{stroke}%
\end{pgfscope}%
\begin{pgfscope}%
\pgfpathrectangle{\pgfqpoint{0.048611in}{0.061342in}}{\pgfqpoint{1.800199in}{1.707361in}}%
\pgfusepath{clip}%
\pgfsetbuttcap%
\pgfsetmiterjoin%
\pgfsetlinewidth{1.003750pt}%
\definecolor{currentstroke}{rgb}{0.000000,0.501961,0.000000}%
\pgfsetstrokecolor{currentstroke}%
\pgfsetdash{}{0pt}%
\pgfpathmoveto{\pgfqpoint{1.727517in}{1.676221in}}%
\pgfpathcurveto{\pgfqpoint{1.736950in}{1.676221in}}{\pgfqpoint{1.745998in}{1.679969in}}{\pgfqpoint{1.752668in}{1.686639in}}%
\pgfpathcurveto{\pgfqpoint{1.759339in}{1.693309in}}{\pgfqpoint{1.763087in}{1.702358in}}{\pgfqpoint{1.763087in}{1.711791in}}%
\pgfpathcurveto{\pgfqpoint{1.763087in}{1.721224in}}{\pgfqpoint{1.759339in}{1.730272in}}{\pgfqpoint{1.752668in}{1.736943in}}%
\pgfpathcurveto{\pgfqpoint{1.745998in}{1.743613in}}{\pgfqpoint{1.736950in}{1.747361in}}{\pgfqpoint{1.727517in}{1.747361in}}%
\pgfpathcurveto{\pgfqpoint{1.718083in}{1.747361in}}{\pgfqpoint{1.709035in}{1.743613in}}{\pgfqpoint{1.702365in}{1.736943in}}%
\pgfpathcurveto{\pgfqpoint{1.695694in}{1.730272in}}{\pgfqpoint{1.691947in}{1.721224in}}{\pgfqpoint{1.691947in}{1.711791in}}%
\pgfpathcurveto{\pgfqpoint{1.691947in}{1.702358in}}{\pgfqpoint{1.695694in}{1.693309in}}{\pgfqpoint{1.702365in}{1.686639in}}%
\pgfpathcurveto{\pgfqpoint{1.709035in}{1.679969in}}{\pgfqpoint{1.718083in}{1.676221in}}{\pgfqpoint{1.727517in}{1.676221in}}%
\pgfpathlineto{\pgfqpoint{1.727517in}{1.676221in}}%
\pgfpathclose%
\pgfusepath{stroke}%
\end{pgfscope}%
\begin{pgfscope}%
\pgfpathrectangle{\pgfqpoint{0.048611in}{0.061342in}}{\pgfqpoint{1.800199in}{1.707361in}}%
\pgfusepath{clip}%
\pgfsetbuttcap%
\pgfsetmiterjoin%
\pgfsetlinewidth{1.003750pt}%
\definecolor{currentstroke}{rgb}{0.000000,0.501961,0.000000}%
\pgfsetstrokecolor{currentstroke}%
\pgfsetdash{}{0pt}%
\pgfpathmoveto{\pgfqpoint{1.649262in}{1.263609in}}%
\pgfpathcurveto{\pgfqpoint{1.658696in}{1.263609in}}{\pgfqpoint{1.667744in}{1.267356in}}{\pgfqpoint{1.674414in}{1.274027in}}%
\pgfpathcurveto{\pgfqpoint{1.681085in}{1.280697in}}{\pgfqpoint{1.684833in}{1.289745in}}{\pgfqpoint{1.684833in}{1.299179in}}%
\pgfpathcurveto{\pgfqpoint{1.684833in}{1.308612in}}{\pgfqpoint{1.681085in}{1.317660in}}{\pgfqpoint{1.674414in}{1.324330in}}%
\pgfpathcurveto{\pgfqpoint{1.667744in}{1.331001in}}{\pgfqpoint{1.658696in}{1.334749in}}{\pgfqpoint{1.649262in}{1.334749in}}%
\pgfpathcurveto{\pgfqpoint{1.639829in}{1.334749in}}{\pgfqpoint{1.630781in}{1.331001in}}{\pgfqpoint{1.624111in}{1.324330in}}%
\pgfpathcurveto{\pgfqpoint{1.617440in}{1.317660in}}{\pgfqpoint{1.613692in}{1.308612in}}{\pgfqpoint{1.613692in}{1.299179in}}%
\pgfpathcurveto{\pgfqpoint{1.613692in}{1.289745in}}{\pgfqpoint{1.617440in}{1.280697in}}{\pgfqpoint{1.624111in}{1.274027in}}%
\pgfpathcurveto{\pgfqpoint{1.630781in}{1.267356in}}{\pgfqpoint{1.639829in}{1.263609in}}{\pgfqpoint{1.649262in}{1.263609in}}%
\pgfpathlineto{\pgfqpoint{1.649262in}{1.263609in}}%
\pgfpathclose%
\pgfusepath{stroke}%
\end{pgfscope}%
\begin{pgfscope}%
\pgfsetrectcap%
\pgfsetmiterjoin%
\pgfsetlinewidth{0.803000pt}%
\definecolor{currentstroke}{rgb}{0.000000,0.000000,0.000000}%
\pgfsetstrokecolor{currentstroke}%
\pgfsetdash{}{0pt}%
\pgfpathmoveto{\pgfqpoint{0.048611in}{0.061342in}}%
\pgfpathlineto{\pgfqpoint{0.048611in}{1.768703in}}%
\pgfusepath{stroke}%
\end{pgfscope}%
\begin{pgfscope}%
\pgfsetrectcap%
\pgfsetmiterjoin%
\pgfsetlinewidth{0.803000pt}%
\definecolor{currentstroke}{rgb}{0.000000,0.000000,0.000000}%
\pgfsetstrokecolor{currentstroke}%
\pgfsetdash{}{0pt}%
\pgfpathmoveto{\pgfqpoint{1.848810in}{0.061342in}}%
\pgfpathlineto{\pgfqpoint{1.848810in}{1.768703in}}%
\pgfusepath{stroke}%
\end{pgfscope}%
\begin{pgfscope}%
\pgfsetrectcap%
\pgfsetmiterjoin%
\pgfsetlinewidth{0.803000pt}%
\definecolor{currentstroke}{rgb}{0.000000,0.000000,0.000000}%
\pgfsetstrokecolor{currentstroke}%
\pgfsetdash{}{0pt}%
\pgfpathmoveto{\pgfqpoint{0.048611in}{0.061342in}}%
\pgfpathlineto{\pgfqpoint{1.848810in}{0.061342in}}%
\pgfusepath{stroke}%
\end{pgfscope}%
\begin{pgfscope}%
\pgfsetrectcap%
\pgfsetmiterjoin%
\pgfsetlinewidth{0.803000pt}%
\definecolor{currentstroke}{rgb}{0.000000,0.000000,0.000000}%
\pgfsetstrokecolor{currentstroke}%
\pgfsetdash{}{0pt}%
\pgfpathmoveto{\pgfqpoint{0.048611in}{1.768703in}}%
\pgfpathlineto{\pgfqpoint{1.848810in}{1.768703in}}%
\pgfusepath{stroke}%
\end{pgfscope}%
\begin{pgfscope}%
\definecolor{textcolor}{rgb}{0.000000,0.000000,0.000000}%
\pgfsetstrokecolor{textcolor}%
\pgfsetfillcolor{textcolor}%
\pgftext[x=0.048611in,y=2.195543in,left,base]{\color{textcolor}\rmfamily\fontsize{10.000000}{12.000000}\selectfont (a)}%
\end{pgfscope}%
\begin{pgfscope}%
\pgfsetbuttcap%
\pgfsetmiterjoin%
\definecolor{currentfill}{rgb}{1.000000,1.000000,1.000000}%
\pgfsetfillcolor{currentfill}%
\pgfsetlinewidth{0.000000pt}%
\definecolor{currentstroke}{rgb}{0.000000,0.000000,0.000000}%
\pgfsetstrokecolor{currentstroke}%
\pgfsetstrokeopacity{0.000000}%
\pgfsetdash{}{0pt}%
\pgfpathmoveto{\pgfqpoint{1.948810in}{0.061342in}}%
\pgfpathlineto{\pgfqpoint{3.749010in}{0.061342in}}%
\pgfpathlineto{\pgfqpoint{3.749010in}{1.768703in}}%
\pgfpathlineto{\pgfqpoint{1.948810in}{1.768703in}}%
\pgfpathlineto{\pgfqpoint{1.948810in}{0.061342in}}%
\pgfpathclose%
\pgfusepath{fill}%
\end{pgfscope}%
\begin{pgfscope}%
\pgfsys@transformshift{2.042000in}{0.105710in}%
\pgftext[left,bottom]{\includegraphics[interpolate=true,width=1.592000in,height=1.636000in]{th_180_450_600-img1.png}}%
\end{pgfscope}%
\begin{pgfscope}%
\pgfpathrectangle{\pgfqpoint{1.948810in}{0.061342in}}{\pgfqpoint{1.800199in}{1.707361in}}%
\pgfusepath{clip}%
\pgfsetbuttcap%
\pgfsetmiterjoin%
\pgfsetlinewidth{1.003750pt}%
\definecolor{currentstroke}{rgb}{0.000000,0.501961,0.000000}%
\pgfsetstrokecolor{currentstroke}%
\pgfsetdash{}{0pt}%
\pgfpathmoveto{\pgfqpoint{2.148003in}{0.722944in}}%
\pgfpathcurveto{\pgfqpoint{2.157436in}{0.722944in}}{\pgfqpoint{2.166484in}{0.726692in}}{\pgfqpoint{2.173154in}{0.733362in}}%
\pgfpathcurveto{\pgfqpoint{2.179825in}{0.740033in}}{\pgfqpoint{2.183573in}{0.749081in}}{\pgfqpoint{2.183573in}{0.758514in}}%
\pgfpathcurveto{\pgfqpoint{2.183573in}{0.767947in}}{\pgfqpoint{2.179825in}{0.776996in}}{\pgfqpoint{2.173154in}{0.783666in}}%
\pgfpathcurveto{\pgfqpoint{2.166484in}{0.790336in}}{\pgfqpoint{2.157436in}{0.794084in}}{\pgfqpoint{2.148003in}{0.794084in}}%
\pgfpathcurveto{\pgfqpoint{2.138569in}{0.794084in}}{\pgfqpoint{2.129521in}{0.790336in}}{\pgfqpoint{2.122851in}{0.783666in}}%
\pgfpathcurveto{\pgfqpoint{2.116180in}{0.776996in}}{\pgfqpoint{2.112432in}{0.767947in}}{\pgfqpoint{2.112432in}{0.758514in}}%
\pgfpathcurveto{\pgfqpoint{2.112432in}{0.749081in}}{\pgfqpoint{2.116180in}{0.740033in}}{\pgfqpoint{2.122851in}{0.733362in}}%
\pgfpathcurveto{\pgfqpoint{2.129521in}{0.726692in}}{\pgfqpoint{2.138569in}{0.722944in}}{\pgfqpoint{2.148003in}{0.722944in}}%
\pgfpathlineto{\pgfqpoint{2.148003in}{0.722944in}}%
\pgfpathclose%
\pgfusepath{stroke}%
\end{pgfscope}%
\begin{pgfscope}%
\pgfpathrectangle{\pgfqpoint{1.948810in}{0.061342in}}{\pgfqpoint{1.800199in}{1.707361in}}%
\pgfusepath{clip}%
\pgfsetbuttcap%
\pgfsetmiterjoin%
\pgfsetlinewidth{1.003750pt}%
\definecolor{currentstroke}{rgb}{0.000000,0.501961,0.000000}%
\pgfsetstrokecolor{currentstroke}%
\pgfsetdash{}{0pt}%
\pgfpathmoveto{\pgfqpoint{3.101279in}{1.683335in}}%
\pgfpathcurveto{\pgfqpoint{3.110713in}{1.683335in}}{\pgfqpoint{3.119761in}{1.687083in}}{\pgfqpoint{3.126431in}{1.693753in}}%
\pgfpathcurveto{\pgfqpoint{3.133101in}{1.700423in}}{\pgfqpoint{3.136849in}{1.709472in}}{\pgfqpoint{3.136849in}{1.718905in}}%
\pgfpathcurveto{\pgfqpoint{3.136849in}{1.728338in}}{\pgfqpoint{3.133101in}{1.737386in}}{\pgfqpoint{3.126431in}{1.744057in}}%
\pgfpathcurveto{\pgfqpoint{3.119761in}{1.750727in}}{\pgfqpoint{3.110713in}{1.754475in}}{\pgfqpoint{3.101279in}{1.754475in}}%
\pgfpathcurveto{\pgfqpoint{3.091846in}{1.754475in}}{\pgfqpoint{3.082798in}{1.750727in}}{\pgfqpoint{3.076128in}{1.744057in}}%
\pgfpathcurveto{\pgfqpoint{3.069457in}{1.737386in}}{\pgfqpoint{3.065709in}{1.728338in}}{\pgfqpoint{3.065709in}{1.718905in}}%
\pgfpathcurveto{\pgfqpoint{3.065709in}{1.709472in}}{\pgfqpoint{3.069457in}{1.700423in}}{\pgfqpoint{3.076128in}{1.693753in}}%
\pgfpathcurveto{\pgfqpoint{3.082798in}{1.687083in}}{\pgfqpoint{3.091846in}{1.683335in}}{\pgfqpoint{3.101279in}{1.683335in}}%
\pgfpathlineto{\pgfqpoint{3.101279in}{1.683335in}}%
\pgfpathclose%
\pgfusepath{stroke}%
\end{pgfscope}%
\begin{pgfscope}%
\pgfpathrectangle{\pgfqpoint{1.948810in}{0.061342in}}{\pgfqpoint{1.800199in}{1.707361in}}%
\pgfusepath{clip}%
\pgfsetbuttcap%
\pgfsetmiterjoin%
\pgfsetlinewidth{1.003750pt}%
\definecolor{currentstroke}{rgb}{0.000000,0.501961,0.000000}%
\pgfsetstrokecolor{currentstroke}%
\pgfsetdash{}{0pt}%
\pgfpathmoveto{\pgfqpoint{3.499664in}{1.156898in}}%
\pgfpathcurveto{\pgfqpoint{3.509097in}{1.156898in}}{\pgfqpoint{3.518145in}{1.160646in}}{\pgfqpoint{3.524815in}{1.167317in}}%
\pgfpathcurveto{\pgfqpoint{3.531486in}{1.173987in}}{\pgfqpoint{3.535234in}{1.183035in}}{\pgfqpoint{3.535234in}{1.192468in}}%
\pgfpathcurveto{\pgfqpoint{3.535234in}{1.201902in}}{\pgfqpoint{3.531486in}{1.210950in}}{\pgfqpoint{3.524815in}{1.217620in}}%
\pgfpathcurveto{\pgfqpoint{3.518145in}{1.224291in}}{\pgfqpoint{3.509097in}{1.228039in}}{\pgfqpoint{3.499664in}{1.228039in}}%
\pgfpathcurveto{\pgfqpoint{3.490230in}{1.228039in}}{\pgfqpoint{3.481182in}{1.224291in}}{\pgfqpoint{3.474512in}{1.217620in}}%
\pgfpathcurveto{\pgfqpoint{3.467842in}{1.210950in}}{\pgfqpoint{3.464094in}{1.201902in}}{\pgfqpoint{3.464094in}{1.192468in}}%
\pgfpathcurveto{\pgfqpoint{3.464094in}{1.183035in}}{\pgfqpoint{3.467842in}{1.173987in}}{\pgfqpoint{3.474512in}{1.167317in}}%
\pgfpathcurveto{\pgfqpoint{3.481182in}{1.160646in}}{\pgfqpoint{3.490230in}{1.156898in}}{\pgfqpoint{3.499664in}{1.156898in}}%
\pgfpathlineto{\pgfqpoint{3.499664in}{1.156898in}}%
\pgfpathclose%
\pgfusepath{stroke}%
\end{pgfscope}%
\begin{pgfscope}%
\pgfpathrectangle{\pgfqpoint{1.948810in}{0.061342in}}{\pgfqpoint{1.800199in}{1.707361in}}%
\pgfusepath{clip}%
\pgfsetbuttcap%
\pgfsetmiterjoin%
\pgfsetlinewidth{1.003750pt}%
\definecolor{currentstroke}{rgb}{0.000000,0.501961,0.000000}%
\pgfsetstrokecolor{currentstroke}%
\pgfsetdash{}{0pt}%
\pgfpathmoveto{\pgfqpoint{3.250673in}{0.075570in}}%
\pgfpathcurveto{\pgfqpoint{3.260107in}{0.075570in}}{\pgfqpoint{3.269155in}{0.079317in}}{\pgfqpoint{3.275825in}{0.085988in}}%
\pgfpathcurveto{\pgfqpoint{3.282496in}{0.092658in}}{\pgfqpoint{3.286243in}{0.101706in}}{\pgfqpoint{3.286243in}{0.111140in}}%
\pgfpathcurveto{\pgfqpoint{3.286243in}{0.120573in}}{\pgfqpoint{3.282496in}{0.129621in}}{\pgfqpoint{3.275825in}{0.136291in}}%
\pgfpathcurveto{\pgfqpoint{3.269155in}{0.142962in}}{\pgfqpoint{3.260107in}{0.146710in}}{\pgfqpoint{3.250673in}{0.146710in}}%
\pgfpathcurveto{\pgfqpoint{3.241240in}{0.146710in}}{\pgfqpoint{3.232192in}{0.142962in}}{\pgfqpoint{3.225522in}{0.136291in}}%
\pgfpathcurveto{\pgfqpoint{3.218851in}{0.129621in}}{\pgfqpoint{3.215103in}{0.120573in}}{\pgfqpoint{3.215103in}{0.111140in}}%
\pgfpathcurveto{\pgfqpoint{3.215103in}{0.101706in}}{\pgfqpoint{3.218851in}{0.092658in}}{\pgfqpoint{3.225522in}{0.085988in}}%
\pgfpathcurveto{\pgfqpoint{3.232192in}{0.079317in}}{\pgfqpoint{3.241240in}{0.075570in}}{\pgfqpoint{3.250673in}{0.075570in}}%
\pgfpathlineto{\pgfqpoint{3.250673in}{0.075570in}}%
\pgfpathclose%
\pgfusepath{stroke}%
\end{pgfscope}%
\begin{pgfscope}%
\pgfpathrectangle{\pgfqpoint{1.948810in}{0.061342in}}{\pgfqpoint{1.800199in}{1.707361in}}%
\pgfusepath{clip}%
\pgfsetbuttcap%
\pgfsetmiterjoin%
\pgfsetlinewidth{1.003750pt}%
\definecolor{currentstroke}{rgb}{0.000000,0.501961,0.000000}%
\pgfsetstrokecolor{currentstroke}%
\pgfsetdash{}{0pt}%
\pgfpathmoveto{\pgfqpoint{3.236445in}{1.035960in}}%
\pgfpathcurveto{\pgfqpoint{3.245879in}{1.035960in}}{\pgfqpoint{3.254927in}{1.039708in}}{\pgfqpoint{3.261597in}{1.046379in}}%
\pgfpathcurveto{\pgfqpoint{3.268268in}{1.053049in}}{\pgfqpoint{3.272015in}{1.062097in}}{\pgfqpoint{3.272015in}{1.071530in}}%
\pgfpathcurveto{\pgfqpoint{3.272015in}{1.080964in}}{\pgfqpoint{3.268268in}{1.090012in}}{\pgfqpoint{3.261597in}{1.096682in}}%
\pgfpathcurveto{\pgfqpoint{3.254927in}{1.103353in}}{\pgfqpoint{3.245879in}{1.107100in}}{\pgfqpoint{3.236445in}{1.107100in}}%
\pgfpathcurveto{\pgfqpoint{3.227012in}{1.107100in}}{\pgfqpoint{3.217964in}{1.103353in}}{\pgfqpoint{3.211294in}{1.096682in}}%
\pgfpathcurveto{\pgfqpoint{3.204623in}{1.090012in}}{\pgfqpoint{3.200875in}{1.080964in}}{\pgfqpoint{3.200875in}{1.071530in}}%
\pgfpathcurveto{\pgfqpoint{3.200875in}{1.062097in}}{\pgfqpoint{3.204623in}{1.053049in}}{\pgfqpoint{3.211294in}{1.046379in}}%
\pgfpathcurveto{\pgfqpoint{3.217964in}{1.039708in}}{\pgfqpoint{3.227012in}{1.035960in}}{\pgfqpoint{3.236445in}{1.035960in}}%
\pgfpathlineto{\pgfqpoint{3.236445in}{1.035960in}}%
\pgfpathclose%
\pgfusepath{stroke}%
\end{pgfscope}%
\begin{pgfscope}%
\pgfpathrectangle{\pgfqpoint{1.948810in}{0.061342in}}{\pgfqpoint{1.800199in}{1.707361in}}%
\pgfusepath{clip}%
\pgfsetbuttcap%
\pgfsetmiterjoin%
\pgfsetlinewidth{1.003750pt}%
\definecolor{currentstroke}{rgb}{0.000000,0.501961,0.000000}%
\pgfsetstrokecolor{currentstroke}%
\pgfsetdash{}{0pt}%
\pgfpathmoveto{\pgfqpoint{3.513892in}{0.829654in}}%
\pgfpathcurveto{\pgfqpoint{3.523325in}{0.829654in}}{\pgfqpoint{3.532373in}{0.833402in}}{\pgfqpoint{3.539043in}{0.840072in}}%
\pgfpathcurveto{\pgfqpoint{3.545714in}{0.846743in}}{\pgfqpoint{3.549462in}{0.855791in}}{\pgfqpoint{3.549462in}{0.865224in}}%
\pgfpathcurveto{\pgfqpoint{3.549462in}{0.874657in}}{\pgfqpoint{3.545714in}{0.883706in}}{\pgfqpoint{3.539043in}{0.890376in}}%
\pgfpathcurveto{\pgfqpoint{3.532373in}{0.897046in}}{\pgfqpoint{3.523325in}{0.900794in}}{\pgfqpoint{3.513892in}{0.900794in}}%
\pgfpathcurveto{\pgfqpoint{3.504458in}{0.900794in}}{\pgfqpoint{3.495410in}{0.897046in}}{\pgfqpoint{3.488740in}{0.890376in}}%
\pgfpathcurveto{\pgfqpoint{3.482070in}{0.883706in}}{\pgfqpoint{3.478322in}{0.874657in}}{\pgfqpoint{3.478322in}{0.865224in}}%
\pgfpathcurveto{\pgfqpoint{3.478322in}{0.855791in}}{\pgfqpoint{3.482070in}{0.846743in}}{\pgfqpoint{3.488740in}{0.840072in}}%
\pgfpathcurveto{\pgfqpoint{3.495410in}{0.833402in}}{\pgfqpoint{3.504458in}{0.829654in}}{\pgfqpoint{3.513892in}{0.829654in}}%
\pgfpathlineto{\pgfqpoint{3.513892in}{0.829654in}}%
\pgfpathclose%
\pgfusepath{stroke}%
\end{pgfscope}%
\begin{pgfscope}%
\pgfpathrectangle{\pgfqpoint{1.948810in}{0.061342in}}{\pgfqpoint{1.800199in}{1.707361in}}%
\pgfusepath{clip}%
\pgfsetbuttcap%
\pgfsetmiterjoin%
\pgfsetlinewidth{1.003750pt}%
\definecolor{currentstroke}{rgb}{0.000000,0.501961,0.000000}%
\pgfsetstrokecolor{currentstroke}%
\pgfsetdash{}{0pt}%
\pgfpathmoveto{\pgfqpoint{3.442752in}{0.331674in}}%
\pgfpathcurveto{\pgfqpoint{3.452185in}{0.331674in}}{\pgfqpoint{3.461233in}{0.335422in}}{\pgfqpoint{3.467903in}{0.342092in}}%
\pgfpathcurveto{\pgfqpoint{3.474574in}{0.348762in}}{\pgfqpoint{3.478322in}{0.357811in}}{\pgfqpoint{3.478322in}{0.367244in}}%
\pgfpathcurveto{\pgfqpoint{3.478322in}{0.376677in}}{\pgfqpoint{3.474574in}{0.385725in}}{\pgfqpoint{3.467903in}{0.392396in}}%
\pgfpathcurveto{\pgfqpoint{3.461233in}{0.399066in}}{\pgfqpoint{3.452185in}{0.402814in}}{\pgfqpoint{3.442752in}{0.402814in}}%
\pgfpathcurveto{\pgfqpoint{3.433318in}{0.402814in}}{\pgfqpoint{3.424270in}{0.399066in}}{\pgfqpoint{3.417600in}{0.392396in}}%
\pgfpathcurveto{\pgfqpoint{3.410929in}{0.385725in}}{\pgfqpoint{3.407182in}{0.376677in}}{\pgfqpoint{3.407182in}{0.367244in}}%
\pgfpathcurveto{\pgfqpoint{3.407182in}{0.357811in}}{\pgfqpoint{3.410929in}{0.348762in}}{\pgfqpoint{3.417600in}{0.342092in}}%
\pgfpathcurveto{\pgfqpoint{3.424270in}{0.335422in}}{\pgfqpoint{3.433318in}{0.331674in}}{\pgfqpoint{3.442752in}{0.331674in}}%
\pgfpathlineto{\pgfqpoint{3.442752in}{0.331674in}}%
\pgfpathclose%
\pgfusepath{stroke}%
\end{pgfscope}%
\begin{pgfscope}%
\pgfpathrectangle{\pgfqpoint{1.948810in}{0.061342in}}{\pgfqpoint{1.800199in}{1.707361in}}%
\pgfusepath{clip}%
\pgfsetbuttcap%
\pgfsetmiterjoin%
\pgfsetlinewidth{1.003750pt}%
\definecolor{currentstroke}{rgb}{0.000000,0.501961,0.000000}%
\pgfsetstrokecolor{currentstroke}%
\pgfsetdash{}{0pt}%
\pgfpathmoveto{\pgfqpoint{2.041292in}{1.491257in}}%
\pgfpathcurveto{\pgfqpoint{2.050726in}{1.491257in}}{\pgfqpoint{2.059774in}{1.495005in}}{\pgfqpoint{2.066444in}{1.501675in}}%
\pgfpathcurveto{\pgfqpoint{2.073115in}{1.508345in}}{\pgfqpoint{2.076862in}{1.517393in}}{\pgfqpoint{2.076862in}{1.526827in}}%
\pgfpathcurveto{\pgfqpoint{2.076862in}{1.536260in}}{\pgfqpoint{2.073115in}{1.545308in}}{\pgfqpoint{2.066444in}{1.551979in}}%
\pgfpathcurveto{\pgfqpoint{2.059774in}{1.558649in}}{\pgfqpoint{2.050726in}{1.562397in}}{\pgfqpoint{2.041292in}{1.562397in}}%
\pgfpathcurveto{\pgfqpoint{2.031859in}{1.562397in}}{\pgfqpoint{2.022811in}{1.558649in}}{\pgfqpoint{2.016141in}{1.551979in}}%
\pgfpathcurveto{\pgfqpoint{2.009470in}{1.545308in}}{\pgfqpoint{2.005722in}{1.536260in}}{\pgfqpoint{2.005722in}{1.526827in}}%
\pgfpathcurveto{\pgfqpoint{2.005722in}{1.517393in}}{\pgfqpoint{2.009470in}{1.508345in}}{\pgfqpoint{2.016141in}{1.501675in}}%
\pgfpathcurveto{\pgfqpoint{2.022811in}{1.495005in}}{\pgfqpoint{2.031859in}{1.491257in}}{\pgfqpoint{2.041292in}{1.491257in}}%
\pgfpathlineto{\pgfqpoint{2.041292in}{1.491257in}}%
\pgfpathclose%
\pgfusepath{stroke}%
\end{pgfscope}%
\begin{pgfscope}%
\pgfpathrectangle{\pgfqpoint{1.948810in}{0.061342in}}{\pgfqpoint{1.800199in}{1.707361in}}%
\pgfusepath{clip}%
\pgfsetbuttcap%
\pgfsetmiterjoin%
\pgfsetlinewidth{1.003750pt}%
\definecolor{currentstroke}{rgb}{0.000000,0.501961,0.000000}%
\pgfsetstrokecolor{currentstroke}%
\pgfsetdash{}{0pt}%
\pgfpathmoveto{\pgfqpoint{3.030139in}{0.417042in}}%
\pgfpathcurveto{\pgfqpoint{3.039573in}{0.417042in}}{\pgfqpoint{3.048621in}{0.420790in}}{\pgfqpoint{3.055291in}{0.427460in}}%
\pgfpathcurveto{\pgfqpoint{3.061961in}{0.434130in}}{\pgfqpoint{3.065709in}{0.443179in}}{\pgfqpoint{3.065709in}{0.452612in}}%
\pgfpathcurveto{\pgfqpoint{3.065709in}{0.462045in}}{\pgfqpoint{3.061961in}{0.471093in}}{\pgfqpoint{3.055291in}{0.477764in}}%
\pgfpathcurveto{\pgfqpoint{3.048621in}{0.484434in}}{\pgfqpoint{3.039573in}{0.488182in}}{\pgfqpoint{3.030139in}{0.488182in}}%
\pgfpathcurveto{\pgfqpoint{3.020706in}{0.488182in}}{\pgfqpoint{3.011658in}{0.484434in}}{\pgfqpoint{3.004987in}{0.477764in}}%
\pgfpathcurveto{\pgfqpoint{2.998317in}{0.471093in}}{\pgfqpoint{2.994569in}{0.462045in}}{\pgfqpoint{2.994569in}{0.452612in}}%
\pgfpathcurveto{\pgfqpoint{2.994569in}{0.443179in}}{\pgfqpoint{2.998317in}{0.434130in}}{\pgfqpoint{3.004987in}{0.427460in}}%
\pgfpathcurveto{\pgfqpoint{3.011658in}{0.420790in}}{\pgfqpoint{3.020706in}{0.417042in}}{\pgfqpoint{3.030139in}{0.417042in}}%
\pgfpathlineto{\pgfqpoint{3.030139in}{0.417042in}}%
\pgfpathclose%
\pgfusepath{stroke}%
\end{pgfscope}%
\begin{pgfscope}%
\pgfpathrectangle{\pgfqpoint{1.948810in}{0.061342in}}{\pgfqpoint{1.800199in}{1.707361in}}%
\pgfusepath{clip}%
\pgfsetbuttcap%
\pgfsetmiterjoin%
\pgfsetlinewidth{1.003750pt}%
\definecolor{currentstroke}{rgb}{0.000000,0.501961,0.000000}%
\pgfsetstrokecolor{currentstroke}%
\pgfsetdash{}{0pt}%
\pgfpathmoveto{\pgfqpoint{3.627716in}{1.676221in}}%
\pgfpathcurveto{\pgfqpoint{3.637149in}{1.676221in}}{\pgfqpoint{3.646197in}{1.679969in}}{\pgfqpoint{3.652868in}{1.686639in}}%
\pgfpathcurveto{\pgfqpoint{3.659538in}{1.693309in}}{\pgfqpoint{3.663286in}{1.702358in}}{\pgfqpoint{3.663286in}{1.711791in}}%
\pgfpathcurveto{\pgfqpoint{3.663286in}{1.721224in}}{\pgfqpoint{3.659538in}{1.730272in}}{\pgfqpoint{3.652868in}{1.736943in}}%
\pgfpathcurveto{\pgfqpoint{3.646197in}{1.743613in}}{\pgfqpoint{3.637149in}{1.747361in}}{\pgfqpoint{3.627716in}{1.747361in}}%
\pgfpathcurveto{\pgfqpoint{3.618283in}{1.747361in}}{\pgfqpoint{3.609234in}{1.743613in}}{\pgfqpoint{3.602564in}{1.736943in}}%
\pgfpathcurveto{\pgfqpoint{3.595894in}{1.730272in}}{\pgfqpoint{3.592146in}{1.721224in}}{\pgfqpoint{3.592146in}{1.711791in}}%
\pgfpathcurveto{\pgfqpoint{3.592146in}{1.702358in}}{\pgfqpoint{3.595894in}{1.693309in}}{\pgfqpoint{3.602564in}{1.686639in}}%
\pgfpathcurveto{\pgfqpoint{3.609234in}{1.679969in}}{\pgfqpoint{3.618283in}{1.676221in}}{\pgfqpoint{3.627716in}{1.676221in}}%
\pgfpathlineto{\pgfqpoint{3.627716in}{1.676221in}}%
\pgfpathclose%
\pgfusepath{stroke}%
\end{pgfscope}%
\begin{pgfscope}%
\pgfpathrectangle{\pgfqpoint{1.948810in}{0.061342in}}{\pgfqpoint{1.800199in}{1.707361in}}%
\pgfusepath{clip}%
\pgfsetbuttcap%
\pgfsetmiterjoin%
\pgfsetlinewidth{1.003750pt}%
\definecolor{currentstroke}{rgb}{0.000000,0.501961,0.000000}%
\pgfsetstrokecolor{currentstroke}%
\pgfsetdash{}{0pt}%
\pgfpathmoveto{\pgfqpoint{3.549462in}{1.263609in}}%
\pgfpathcurveto{\pgfqpoint{3.558895in}{1.263609in}}{\pgfqpoint{3.567943in}{1.267356in}}{\pgfqpoint{3.574614in}{1.274027in}}%
\pgfpathcurveto{\pgfqpoint{3.581284in}{1.280697in}}{\pgfqpoint{3.585032in}{1.289745in}}{\pgfqpoint{3.585032in}{1.299179in}}%
\pgfpathcurveto{\pgfqpoint{3.585032in}{1.308612in}}{\pgfqpoint{3.581284in}{1.317660in}}{\pgfqpoint{3.574614in}{1.324330in}}%
\pgfpathcurveto{\pgfqpoint{3.567943in}{1.331001in}}{\pgfqpoint{3.558895in}{1.334749in}}{\pgfqpoint{3.549462in}{1.334749in}}%
\pgfpathcurveto{\pgfqpoint{3.540028in}{1.334749in}}{\pgfqpoint{3.530980in}{1.331001in}}{\pgfqpoint{3.524310in}{1.324330in}}%
\pgfpathcurveto{\pgfqpoint{3.517640in}{1.317660in}}{\pgfqpoint{3.513892in}{1.308612in}}{\pgfqpoint{3.513892in}{1.299179in}}%
\pgfpathcurveto{\pgfqpoint{3.513892in}{1.289745in}}{\pgfqpoint{3.517640in}{1.280697in}}{\pgfqpoint{3.524310in}{1.274027in}}%
\pgfpathcurveto{\pgfqpoint{3.530980in}{1.267356in}}{\pgfqpoint{3.540028in}{1.263609in}}{\pgfqpoint{3.549462in}{1.263609in}}%
\pgfpathlineto{\pgfqpoint{3.549462in}{1.263609in}}%
\pgfpathclose%
\pgfusepath{stroke}%
\end{pgfscope}%
\begin{pgfscope}%
\pgfsetrectcap%
\pgfsetmiterjoin%
\pgfsetlinewidth{0.803000pt}%
\definecolor{currentstroke}{rgb}{0.000000,0.000000,0.000000}%
\pgfsetstrokecolor{currentstroke}%
\pgfsetdash{}{0pt}%
\pgfpathmoveto{\pgfqpoint{1.948810in}{0.061342in}}%
\pgfpathlineto{\pgfqpoint{1.948810in}{1.768703in}}%
\pgfusepath{stroke}%
\end{pgfscope}%
\begin{pgfscope}%
\pgfsetrectcap%
\pgfsetmiterjoin%
\pgfsetlinewidth{0.803000pt}%
\definecolor{currentstroke}{rgb}{0.000000,0.000000,0.000000}%
\pgfsetstrokecolor{currentstroke}%
\pgfsetdash{}{0pt}%
\pgfpathmoveto{\pgfqpoint{3.749010in}{0.061342in}}%
\pgfpathlineto{\pgfqpoint{3.749010in}{1.768703in}}%
\pgfusepath{stroke}%
\end{pgfscope}%
\begin{pgfscope}%
\pgfsetrectcap%
\pgfsetmiterjoin%
\pgfsetlinewidth{0.803000pt}%
\definecolor{currentstroke}{rgb}{0.000000,0.000000,0.000000}%
\pgfsetstrokecolor{currentstroke}%
\pgfsetdash{}{0pt}%
\pgfpathmoveto{\pgfqpoint{1.948810in}{0.061342in}}%
\pgfpathlineto{\pgfqpoint{3.749010in}{0.061342in}}%
\pgfusepath{stroke}%
\end{pgfscope}%
\begin{pgfscope}%
\pgfsetrectcap%
\pgfsetmiterjoin%
\pgfsetlinewidth{0.803000pt}%
\definecolor{currentstroke}{rgb}{0.000000,0.000000,0.000000}%
\pgfsetstrokecolor{currentstroke}%
\pgfsetdash{}{0pt}%
\pgfpathmoveto{\pgfqpoint{1.948810in}{1.768703in}}%
\pgfpathlineto{\pgfqpoint{3.749010in}{1.768703in}}%
\pgfusepath{stroke}%
\end{pgfscope}%
\begin{pgfscope}%
\definecolor{textcolor}{rgb}{0.000000,0.000000,0.000000}%
\pgfsetstrokecolor{textcolor}%
\pgfsetfillcolor{textcolor}%
\pgftext[x=1.948810in,y=2.195543in,left,base]{\color{textcolor}\rmfamily\fontsize{10.000000}{12.000000}\selectfont (b)}%
\end{pgfscope}%
\begin{pgfscope}%
\pgfsetbuttcap%
\pgfsetmiterjoin%
\definecolor{currentfill}{rgb}{1.000000,1.000000,1.000000}%
\pgfsetfillcolor{currentfill}%
\pgfsetlinewidth{0.000000pt}%
\definecolor{currentstroke}{rgb}{0.000000,0.000000,0.000000}%
\pgfsetstrokecolor{currentstroke}%
\pgfsetstrokeopacity{0.000000}%
\pgfsetdash{}{0pt}%
\pgfpathmoveto{\pgfqpoint{3.849010in}{0.061342in}}%
\pgfpathlineto{\pgfqpoint{5.649209in}{0.061342in}}%
\pgfpathlineto{\pgfqpoint{5.649209in}{1.768703in}}%
\pgfpathlineto{\pgfqpoint{3.849010in}{1.768703in}}%
\pgfpathlineto{\pgfqpoint{3.849010in}{0.061342in}}%
\pgfpathclose%
\pgfusepath{fill}%
\end{pgfscope}%
\begin{pgfscope}%
\pgfsys@transformshift{3.942000in}{0.105710in}%
\pgftext[left,bottom]{\includegraphics[interpolate=true,width=1.594000in,height=1.636000in]{th_180_450_600-img2.png}}%
\end{pgfscope}%
\begin{pgfscope}%
\pgfpathrectangle{\pgfqpoint{3.849010in}{0.061342in}}{\pgfqpoint{1.800199in}{1.707361in}}%
\pgfusepath{clip}%
\pgfsetbuttcap%
\pgfsetmiterjoin%
\pgfsetlinewidth{1.003750pt}%
\definecolor{currentstroke}{rgb}{0.000000,0.501961,0.000000}%
\pgfsetstrokecolor{currentstroke}%
\pgfsetdash{}{0pt}%
\pgfpathmoveto{\pgfqpoint{4.048202in}{0.722944in}}%
\pgfpathcurveto{\pgfqpoint{4.057635in}{0.722944in}}{\pgfqpoint{4.066683in}{0.726692in}}{\pgfqpoint{4.073354in}{0.733362in}}%
\pgfpathcurveto{\pgfqpoint{4.080024in}{0.740033in}}{\pgfqpoint{4.083772in}{0.749081in}}{\pgfqpoint{4.083772in}{0.758514in}}%
\pgfpathcurveto{\pgfqpoint{4.083772in}{0.767947in}}{\pgfqpoint{4.080024in}{0.776996in}}{\pgfqpoint{4.073354in}{0.783666in}}%
\pgfpathcurveto{\pgfqpoint{4.066683in}{0.790336in}}{\pgfqpoint{4.057635in}{0.794084in}}{\pgfqpoint{4.048202in}{0.794084in}}%
\pgfpathcurveto{\pgfqpoint{4.038768in}{0.794084in}}{\pgfqpoint{4.029720in}{0.790336in}}{\pgfqpoint{4.023050in}{0.783666in}}%
\pgfpathcurveto{\pgfqpoint{4.016380in}{0.776996in}}{\pgfqpoint{4.012632in}{0.767947in}}{\pgfqpoint{4.012632in}{0.758514in}}%
\pgfpathcurveto{\pgfqpoint{4.012632in}{0.749081in}}{\pgfqpoint{4.016380in}{0.740033in}}{\pgfqpoint{4.023050in}{0.733362in}}%
\pgfpathcurveto{\pgfqpoint{4.029720in}{0.726692in}}{\pgfqpoint{4.038768in}{0.722944in}}{\pgfqpoint{4.048202in}{0.722944in}}%
\pgfpathlineto{\pgfqpoint{4.048202in}{0.722944in}}%
\pgfpathclose%
\pgfusepath{stroke}%
\end{pgfscope}%
\begin{pgfscope}%
\pgfpathrectangle{\pgfqpoint{3.849010in}{0.061342in}}{\pgfqpoint{1.800199in}{1.707361in}}%
\pgfusepath{clip}%
\pgfsetbuttcap%
\pgfsetmiterjoin%
\pgfsetlinewidth{1.003750pt}%
\definecolor{currentstroke}{rgb}{0.000000,0.501961,0.000000}%
\pgfsetstrokecolor{currentstroke}%
\pgfsetdash{}{0pt}%
\pgfpathmoveto{\pgfqpoint{5.001479in}{1.683335in}}%
\pgfpathcurveto{\pgfqpoint{5.010912in}{1.683335in}}{\pgfqpoint{5.019960in}{1.687083in}}{\pgfqpoint{5.026630in}{1.693753in}}%
\pgfpathcurveto{\pgfqpoint{5.033301in}{1.700423in}}{\pgfqpoint{5.037049in}{1.709472in}}{\pgfqpoint{5.037049in}{1.718905in}}%
\pgfpathcurveto{\pgfqpoint{5.037049in}{1.728338in}}{\pgfqpoint{5.033301in}{1.737386in}}{\pgfqpoint{5.026630in}{1.744057in}}%
\pgfpathcurveto{\pgfqpoint{5.019960in}{1.750727in}}{\pgfqpoint{5.010912in}{1.754475in}}{\pgfqpoint{5.001479in}{1.754475in}}%
\pgfpathcurveto{\pgfqpoint{4.992045in}{1.754475in}}{\pgfqpoint{4.982997in}{1.750727in}}{\pgfqpoint{4.976327in}{1.744057in}}%
\pgfpathcurveto{\pgfqpoint{4.969656in}{1.737386in}}{\pgfqpoint{4.965909in}{1.728338in}}{\pgfqpoint{4.965909in}{1.718905in}}%
\pgfpathcurveto{\pgfqpoint{4.965909in}{1.709472in}}{\pgfqpoint{4.969656in}{1.700423in}}{\pgfqpoint{4.976327in}{1.693753in}}%
\pgfpathcurveto{\pgfqpoint{4.982997in}{1.687083in}}{\pgfqpoint{4.992045in}{1.683335in}}{\pgfqpoint{5.001479in}{1.683335in}}%
\pgfpathlineto{\pgfqpoint{5.001479in}{1.683335in}}%
\pgfpathclose%
\pgfusepath{stroke}%
\end{pgfscope}%
\begin{pgfscope}%
\pgfpathrectangle{\pgfqpoint{3.849010in}{0.061342in}}{\pgfqpoint{1.800199in}{1.707361in}}%
\pgfusepath{clip}%
\pgfsetbuttcap%
\pgfsetmiterjoin%
\pgfsetlinewidth{1.003750pt}%
\definecolor{currentstroke}{rgb}{0.000000,0.501961,0.000000}%
\pgfsetstrokecolor{currentstroke}%
\pgfsetdash{}{0pt}%
\pgfpathmoveto{\pgfqpoint{5.399863in}{1.156898in}}%
\pgfpathcurveto{\pgfqpoint{5.409296in}{1.156898in}}{\pgfqpoint{5.418344in}{1.160646in}}{\pgfqpoint{5.425015in}{1.167317in}}%
\pgfpathcurveto{\pgfqpoint{5.431685in}{1.173987in}}{\pgfqpoint{5.435433in}{1.183035in}}{\pgfqpoint{5.435433in}{1.192468in}}%
\pgfpathcurveto{\pgfqpoint{5.435433in}{1.201902in}}{\pgfqpoint{5.431685in}{1.210950in}}{\pgfqpoint{5.425015in}{1.217620in}}%
\pgfpathcurveto{\pgfqpoint{5.418344in}{1.224291in}}{\pgfqpoint{5.409296in}{1.228039in}}{\pgfqpoint{5.399863in}{1.228039in}}%
\pgfpathcurveto{\pgfqpoint{5.390430in}{1.228039in}}{\pgfqpoint{5.381381in}{1.224291in}}{\pgfqpoint{5.374711in}{1.217620in}}%
\pgfpathcurveto{\pgfqpoint{5.368041in}{1.210950in}}{\pgfqpoint{5.364293in}{1.201902in}}{\pgfqpoint{5.364293in}{1.192468in}}%
\pgfpathcurveto{\pgfqpoint{5.364293in}{1.183035in}}{\pgfqpoint{5.368041in}{1.173987in}}{\pgfqpoint{5.374711in}{1.167317in}}%
\pgfpathcurveto{\pgfqpoint{5.381381in}{1.160646in}}{\pgfqpoint{5.390430in}{1.156898in}}{\pgfqpoint{5.399863in}{1.156898in}}%
\pgfpathlineto{\pgfqpoint{5.399863in}{1.156898in}}%
\pgfpathclose%
\pgfusepath{stroke}%
\end{pgfscope}%
\begin{pgfscope}%
\pgfpathrectangle{\pgfqpoint{3.849010in}{0.061342in}}{\pgfqpoint{1.800199in}{1.707361in}}%
\pgfusepath{clip}%
\pgfsetbuttcap%
\pgfsetmiterjoin%
\pgfsetlinewidth{1.003750pt}%
\definecolor{currentstroke}{rgb}{0.000000,0.501961,0.000000}%
\pgfsetstrokecolor{currentstroke}%
\pgfsetdash{}{0pt}%
\pgfpathmoveto{\pgfqpoint{5.150873in}{0.075570in}}%
\pgfpathcurveto{\pgfqpoint{5.160306in}{0.075570in}}{\pgfqpoint{5.169354in}{0.079317in}}{\pgfqpoint{5.176025in}{0.085988in}}%
\pgfpathcurveto{\pgfqpoint{5.182695in}{0.092658in}}{\pgfqpoint{5.186443in}{0.101706in}}{\pgfqpoint{5.186443in}{0.111140in}}%
\pgfpathcurveto{\pgfqpoint{5.186443in}{0.120573in}}{\pgfqpoint{5.182695in}{0.129621in}}{\pgfqpoint{5.176025in}{0.136291in}}%
\pgfpathcurveto{\pgfqpoint{5.169354in}{0.142962in}}{\pgfqpoint{5.160306in}{0.146710in}}{\pgfqpoint{5.150873in}{0.146710in}}%
\pgfpathcurveto{\pgfqpoint{5.141439in}{0.146710in}}{\pgfqpoint{5.132391in}{0.142962in}}{\pgfqpoint{5.125721in}{0.136291in}}%
\pgfpathcurveto{\pgfqpoint{5.119051in}{0.129621in}}{\pgfqpoint{5.115303in}{0.120573in}}{\pgfqpoint{5.115303in}{0.111140in}}%
\pgfpathcurveto{\pgfqpoint{5.115303in}{0.101706in}}{\pgfqpoint{5.119051in}{0.092658in}}{\pgfqpoint{5.125721in}{0.085988in}}%
\pgfpathcurveto{\pgfqpoint{5.132391in}{0.079317in}}{\pgfqpoint{5.141439in}{0.075570in}}{\pgfqpoint{5.150873in}{0.075570in}}%
\pgfpathlineto{\pgfqpoint{5.150873in}{0.075570in}}%
\pgfpathclose%
\pgfusepath{stroke}%
\end{pgfscope}%
\begin{pgfscope}%
\pgfpathrectangle{\pgfqpoint{3.849010in}{0.061342in}}{\pgfqpoint{1.800199in}{1.707361in}}%
\pgfusepath{clip}%
\pgfsetbuttcap%
\pgfsetmiterjoin%
\pgfsetlinewidth{1.003750pt}%
\definecolor{currentstroke}{rgb}{0.000000,0.501961,0.000000}%
\pgfsetstrokecolor{currentstroke}%
\pgfsetdash{}{0pt}%
\pgfpathmoveto{\pgfqpoint{5.136645in}{1.035960in}}%
\pgfpathcurveto{\pgfqpoint{5.146078in}{1.035960in}}{\pgfqpoint{5.155126in}{1.039708in}}{\pgfqpoint{5.161797in}{1.046379in}}%
\pgfpathcurveto{\pgfqpoint{5.168467in}{1.053049in}}{\pgfqpoint{5.172215in}{1.062097in}}{\pgfqpoint{5.172215in}{1.071530in}}%
\pgfpathcurveto{\pgfqpoint{5.172215in}{1.080964in}}{\pgfqpoint{5.168467in}{1.090012in}}{\pgfqpoint{5.161797in}{1.096682in}}%
\pgfpathcurveto{\pgfqpoint{5.155126in}{1.103353in}}{\pgfqpoint{5.146078in}{1.107100in}}{\pgfqpoint{5.136645in}{1.107100in}}%
\pgfpathcurveto{\pgfqpoint{5.127211in}{1.107100in}}{\pgfqpoint{5.118163in}{1.103353in}}{\pgfqpoint{5.111493in}{1.096682in}}%
\pgfpathcurveto{\pgfqpoint{5.104823in}{1.090012in}}{\pgfqpoint{5.101075in}{1.080964in}}{\pgfqpoint{5.101075in}{1.071530in}}%
\pgfpathcurveto{\pgfqpoint{5.101075in}{1.062097in}}{\pgfqpoint{5.104823in}{1.053049in}}{\pgfqpoint{5.111493in}{1.046379in}}%
\pgfpathcurveto{\pgfqpoint{5.118163in}{1.039708in}}{\pgfqpoint{5.127211in}{1.035960in}}{\pgfqpoint{5.136645in}{1.035960in}}%
\pgfpathlineto{\pgfqpoint{5.136645in}{1.035960in}}%
\pgfpathclose%
\pgfusepath{stroke}%
\end{pgfscope}%
\begin{pgfscope}%
\pgfpathrectangle{\pgfqpoint{3.849010in}{0.061342in}}{\pgfqpoint{1.800199in}{1.707361in}}%
\pgfusepath{clip}%
\pgfsetbuttcap%
\pgfsetmiterjoin%
\pgfsetlinewidth{1.003750pt}%
\definecolor{currentstroke}{rgb}{0.000000,0.501961,0.000000}%
\pgfsetstrokecolor{currentstroke}%
\pgfsetdash{}{0pt}%
\pgfpathmoveto{\pgfqpoint{5.414091in}{0.829654in}}%
\pgfpathcurveto{\pgfqpoint{5.423524in}{0.829654in}}{\pgfqpoint{5.432572in}{0.833402in}}{\pgfqpoint{5.439243in}{0.840072in}}%
\pgfpathcurveto{\pgfqpoint{5.445913in}{0.846743in}}{\pgfqpoint{5.449661in}{0.855791in}}{\pgfqpoint{5.449661in}{0.865224in}}%
\pgfpathcurveto{\pgfqpoint{5.449661in}{0.874657in}}{\pgfqpoint{5.445913in}{0.883706in}}{\pgfqpoint{5.439243in}{0.890376in}}%
\pgfpathcurveto{\pgfqpoint{5.432572in}{0.897046in}}{\pgfqpoint{5.423524in}{0.900794in}}{\pgfqpoint{5.414091in}{0.900794in}}%
\pgfpathcurveto{\pgfqpoint{5.404658in}{0.900794in}}{\pgfqpoint{5.395609in}{0.897046in}}{\pgfqpoint{5.388939in}{0.890376in}}%
\pgfpathcurveto{\pgfqpoint{5.382269in}{0.883706in}}{\pgfqpoint{5.378521in}{0.874657in}}{\pgfqpoint{5.378521in}{0.865224in}}%
\pgfpathcurveto{\pgfqpoint{5.378521in}{0.855791in}}{\pgfqpoint{5.382269in}{0.846743in}}{\pgfqpoint{5.388939in}{0.840072in}}%
\pgfpathcurveto{\pgfqpoint{5.395609in}{0.833402in}}{\pgfqpoint{5.404658in}{0.829654in}}{\pgfqpoint{5.414091in}{0.829654in}}%
\pgfpathlineto{\pgfqpoint{5.414091in}{0.829654in}}%
\pgfpathclose%
\pgfusepath{stroke}%
\end{pgfscope}%
\begin{pgfscope}%
\pgfpathrectangle{\pgfqpoint{3.849010in}{0.061342in}}{\pgfqpoint{1.800199in}{1.707361in}}%
\pgfusepath{clip}%
\pgfsetbuttcap%
\pgfsetmiterjoin%
\pgfsetlinewidth{1.003750pt}%
\definecolor{currentstroke}{rgb}{0.000000,0.501961,0.000000}%
\pgfsetstrokecolor{currentstroke}%
\pgfsetdash{}{0pt}%
\pgfpathmoveto{\pgfqpoint{5.342951in}{0.331674in}}%
\pgfpathcurveto{\pgfqpoint{5.352384in}{0.331674in}}{\pgfqpoint{5.361432in}{0.335422in}}{\pgfqpoint{5.368103in}{0.342092in}}%
\pgfpathcurveto{\pgfqpoint{5.374773in}{0.348762in}}{\pgfqpoint{5.378521in}{0.357811in}}{\pgfqpoint{5.378521in}{0.367244in}}%
\pgfpathcurveto{\pgfqpoint{5.378521in}{0.376677in}}{\pgfqpoint{5.374773in}{0.385725in}}{\pgfqpoint{5.368103in}{0.392396in}}%
\pgfpathcurveto{\pgfqpoint{5.361432in}{0.399066in}}{\pgfqpoint{5.352384in}{0.402814in}}{\pgfqpoint{5.342951in}{0.402814in}}%
\pgfpathcurveto{\pgfqpoint{5.333518in}{0.402814in}}{\pgfqpoint{5.324469in}{0.399066in}}{\pgfqpoint{5.317799in}{0.392396in}}%
\pgfpathcurveto{\pgfqpoint{5.311129in}{0.385725in}}{\pgfqpoint{5.307381in}{0.376677in}}{\pgfqpoint{5.307381in}{0.367244in}}%
\pgfpathcurveto{\pgfqpoint{5.307381in}{0.357811in}}{\pgfqpoint{5.311129in}{0.348762in}}{\pgfqpoint{5.317799in}{0.342092in}}%
\pgfpathcurveto{\pgfqpoint{5.324469in}{0.335422in}}{\pgfqpoint{5.333518in}{0.331674in}}{\pgfqpoint{5.342951in}{0.331674in}}%
\pgfpathlineto{\pgfqpoint{5.342951in}{0.331674in}}%
\pgfpathclose%
\pgfusepath{stroke}%
\end{pgfscope}%
\begin{pgfscope}%
\pgfpathrectangle{\pgfqpoint{3.849010in}{0.061342in}}{\pgfqpoint{1.800199in}{1.707361in}}%
\pgfusepath{clip}%
\pgfsetbuttcap%
\pgfsetmiterjoin%
\pgfsetlinewidth{1.003750pt}%
\definecolor{currentstroke}{rgb}{0.000000,0.501961,0.000000}%
\pgfsetstrokecolor{currentstroke}%
\pgfsetdash{}{0pt}%
\pgfpathmoveto{\pgfqpoint{3.941492in}{1.491257in}}%
\pgfpathcurveto{\pgfqpoint{3.950925in}{1.491257in}}{\pgfqpoint{3.959973in}{1.495005in}}{\pgfqpoint{3.966643in}{1.501675in}}%
\pgfpathcurveto{\pgfqpoint{3.973314in}{1.508345in}}{\pgfqpoint{3.977062in}{1.517393in}}{\pgfqpoint{3.977062in}{1.526827in}}%
\pgfpathcurveto{\pgfqpoint{3.977062in}{1.536260in}}{\pgfqpoint{3.973314in}{1.545308in}}{\pgfqpoint{3.966643in}{1.551979in}}%
\pgfpathcurveto{\pgfqpoint{3.959973in}{1.558649in}}{\pgfqpoint{3.950925in}{1.562397in}}{\pgfqpoint{3.941492in}{1.562397in}}%
\pgfpathcurveto{\pgfqpoint{3.932058in}{1.562397in}}{\pgfqpoint{3.923010in}{1.558649in}}{\pgfqpoint{3.916340in}{1.551979in}}%
\pgfpathcurveto{\pgfqpoint{3.909670in}{1.545308in}}{\pgfqpoint{3.905922in}{1.536260in}}{\pgfqpoint{3.905922in}{1.526827in}}%
\pgfpathcurveto{\pgfqpoint{3.905922in}{1.517393in}}{\pgfqpoint{3.909670in}{1.508345in}}{\pgfqpoint{3.916340in}{1.501675in}}%
\pgfpathcurveto{\pgfqpoint{3.923010in}{1.495005in}}{\pgfqpoint{3.932058in}{1.491257in}}{\pgfqpoint{3.941492in}{1.491257in}}%
\pgfpathlineto{\pgfqpoint{3.941492in}{1.491257in}}%
\pgfpathclose%
\pgfusepath{stroke}%
\end{pgfscope}%
\begin{pgfscope}%
\pgfpathrectangle{\pgfqpoint{3.849010in}{0.061342in}}{\pgfqpoint{1.800199in}{1.707361in}}%
\pgfusepath{clip}%
\pgfsetbuttcap%
\pgfsetmiterjoin%
\pgfsetlinewidth{1.003750pt}%
\definecolor{currentstroke}{rgb}{0.000000,0.501961,0.000000}%
\pgfsetstrokecolor{currentstroke}%
\pgfsetdash{}{0pt}%
\pgfpathmoveto{\pgfqpoint{4.930339in}{0.417042in}}%
\pgfpathcurveto{\pgfqpoint{4.939772in}{0.417042in}}{\pgfqpoint{4.948820in}{0.420790in}}{\pgfqpoint{4.955490in}{0.427460in}}%
\pgfpathcurveto{\pgfqpoint{4.962161in}{0.434130in}}{\pgfqpoint{4.965909in}{0.443179in}}{\pgfqpoint{4.965909in}{0.452612in}}%
\pgfpathcurveto{\pgfqpoint{4.965909in}{0.462045in}}{\pgfqpoint{4.962161in}{0.471093in}}{\pgfqpoint{4.955490in}{0.477764in}}%
\pgfpathcurveto{\pgfqpoint{4.948820in}{0.484434in}}{\pgfqpoint{4.939772in}{0.488182in}}{\pgfqpoint{4.930339in}{0.488182in}}%
\pgfpathcurveto{\pgfqpoint{4.920905in}{0.488182in}}{\pgfqpoint{4.911857in}{0.484434in}}{\pgfqpoint{4.905187in}{0.477764in}}%
\pgfpathcurveto{\pgfqpoint{4.898516in}{0.471093in}}{\pgfqpoint{4.894768in}{0.462045in}}{\pgfqpoint{4.894768in}{0.452612in}}%
\pgfpathcurveto{\pgfqpoint{4.894768in}{0.443179in}}{\pgfqpoint{4.898516in}{0.434130in}}{\pgfqpoint{4.905187in}{0.427460in}}%
\pgfpathcurveto{\pgfqpoint{4.911857in}{0.420790in}}{\pgfqpoint{4.920905in}{0.417042in}}{\pgfqpoint{4.930339in}{0.417042in}}%
\pgfpathlineto{\pgfqpoint{4.930339in}{0.417042in}}%
\pgfpathclose%
\pgfusepath{stroke}%
\end{pgfscope}%
\begin{pgfscope}%
\pgfpathrectangle{\pgfqpoint{3.849010in}{0.061342in}}{\pgfqpoint{1.800199in}{1.707361in}}%
\pgfusepath{clip}%
\pgfsetbuttcap%
\pgfsetmiterjoin%
\pgfsetlinewidth{1.003750pt}%
\definecolor{currentstroke}{rgb}{0.000000,0.501961,0.000000}%
\pgfsetstrokecolor{currentstroke}%
\pgfsetdash{}{0pt}%
\pgfpathmoveto{\pgfqpoint{5.527915in}{1.676221in}}%
\pgfpathcurveto{\pgfqpoint{5.537348in}{1.676221in}}{\pgfqpoint{5.546396in}{1.679969in}}{\pgfqpoint{5.553067in}{1.686639in}}%
\pgfpathcurveto{\pgfqpoint{5.559737in}{1.693309in}}{\pgfqpoint{5.563485in}{1.702358in}}{\pgfqpoint{5.563485in}{1.711791in}}%
\pgfpathcurveto{\pgfqpoint{5.563485in}{1.721224in}}{\pgfqpoint{5.559737in}{1.730272in}}{\pgfqpoint{5.553067in}{1.736943in}}%
\pgfpathcurveto{\pgfqpoint{5.546396in}{1.743613in}}{\pgfqpoint{5.537348in}{1.747361in}}{\pgfqpoint{5.527915in}{1.747361in}}%
\pgfpathcurveto{\pgfqpoint{5.518482in}{1.747361in}}{\pgfqpoint{5.509434in}{1.743613in}}{\pgfqpoint{5.502763in}{1.736943in}}%
\pgfpathcurveto{\pgfqpoint{5.496093in}{1.730272in}}{\pgfqpoint{5.492345in}{1.721224in}}{\pgfqpoint{5.492345in}{1.711791in}}%
\pgfpathcurveto{\pgfqpoint{5.492345in}{1.702358in}}{\pgfqpoint{5.496093in}{1.693309in}}{\pgfqpoint{5.502763in}{1.686639in}}%
\pgfpathcurveto{\pgfqpoint{5.509434in}{1.679969in}}{\pgfqpoint{5.518482in}{1.676221in}}{\pgfqpoint{5.527915in}{1.676221in}}%
\pgfpathlineto{\pgfqpoint{5.527915in}{1.676221in}}%
\pgfpathclose%
\pgfusepath{stroke}%
\end{pgfscope}%
\begin{pgfscope}%
\pgfpathrectangle{\pgfqpoint{3.849010in}{0.061342in}}{\pgfqpoint{1.800199in}{1.707361in}}%
\pgfusepath{clip}%
\pgfsetbuttcap%
\pgfsetmiterjoin%
\pgfsetlinewidth{1.003750pt}%
\definecolor{currentstroke}{rgb}{0.000000,0.501961,0.000000}%
\pgfsetstrokecolor{currentstroke}%
\pgfsetdash{}{0pt}%
\pgfpathmoveto{\pgfqpoint{5.449661in}{1.263609in}}%
\pgfpathcurveto{\pgfqpoint{5.459094in}{1.263609in}}{\pgfqpoint{5.468142in}{1.267356in}}{\pgfqpoint{5.474813in}{1.274027in}}%
\pgfpathcurveto{\pgfqpoint{5.481483in}{1.280697in}}{\pgfqpoint{5.485231in}{1.289745in}}{\pgfqpoint{5.485231in}{1.299179in}}%
\pgfpathcurveto{\pgfqpoint{5.485231in}{1.308612in}}{\pgfqpoint{5.481483in}{1.317660in}}{\pgfqpoint{5.474813in}{1.324330in}}%
\pgfpathcurveto{\pgfqpoint{5.468142in}{1.331001in}}{\pgfqpoint{5.459094in}{1.334749in}}{\pgfqpoint{5.449661in}{1.334749in}}%
\pgfpathcurveto{\pgfqpoint{5.440228in}{1.334749in}}{\pgfqpoint{5.431179in}{1.331001in}}{\pgfqpoint{5.424509in}{1.324330in}}%
\pgfpathcurveto{\pgfqpoint{5.417839in}{1.317660in}}{\pgfqpoint{5.414091in}{1.308612in}}{\pgfqpoint{5.414091in}{1.299179in}}%
\pgfpathcurveto{\pgfqpoint{5.414091in}{1.289745in}}{\pgfqpoint{5.417839in}{1.280697in}}{\pgfqpoint{5.424509in}{1.274027in}}%
\pgfpathcurveto{\pgfqpoint{5.431179in}{1.267356in}}{\pgfqpoint{5.440228in}{1.263609in}}{\pgfqpoint{5.449661in}{1.263609in}}%
\pgfpathlineto{\pgfqpoint{5.449661in}{1.263609in}}%
\pgfpathclose%
\pgfusepath{stroke}%
\end{pgfscope}%
\begin{pgfscope}%
\pgfsetrectcap%
\pgfsetmiterjoin%
\pgfsetlinewidth{0.803000pt}%
\definecolor{currentstroke}{rgb}{0.000000,0.000000,0.000000}%
\pgfsetstrokecolor{currentstroke}%
\pgfsetdash{}{0pt}%
\pgfpathmoveto{\pgfqpoint{3.849010in}{0.061342in}}%
\pgfpathlineto{\pgfqpoint{3.849010in}{1.768703in}}%
\pgfusepath{stroke}%
\end{pgfscope}%
\begin{pgfscope}%
\pgfsetrectcap%
\pgfsetmiterjoin%
\pgfsetlinewidth{0.803000pt}%
\definecolor{currentstroke}{rgb}{0.000000,0.000000,0.000000}%
\pgfsetstrokecolor{currentstroke}%
\pgfsetdash{}{0pt}%
\pgfpathmoveto{\pgfqpoint{5.649209in}{0.061342in}}%
\pgfpathlineto{\pgfqpoint{5.649209in}{1.768703in}}%
\pgfusepath{stroke}%
\end{pgfscope}%
\begin{pgfscope}%
\pgfsetrectcap%
\pgfsetmiterjoin%
\pgfsetlinewidth{0.803000pt}%
\definecolor{currentstroke}{rgb}{0.000000,0.000000,0.000000}%
\pgfsetstrokecolor{currentstroke}%
\pgfsetdash{}{0pt}%
\pgfpathmoveto{\pgfqpoint{3.849010in}{0.061342in}}%
\pgfpathlineto{\pgfqpoint{5.649209in}{0.061342in}}%
\pgfusepath{stroke}%
\end{pgfscope}%
\begin{pgfscope}%
\pgfsetrectcap%
\pgfsetmiterjoin%
\pgfsetlinewidth{0.803000pt}%
\definecolor{currentstroke}{rgb}{0.000000,0.000000,0.000000}%
\pgfsetstrokecolor{currentstroke}%
\pgfsetdash{}{0pt}%
\pgfpathmoveto{\pgfqpoint{3.849010in}{1.768703in}}%
\pgfpathlineto{\pgfqpoint{5.649209in}{1.768703in}}%
\pgfusepath{stroke}%
\end{pgfscope}%
\begin{pgfscope}%
\definecolor{textcolor}{rgb}{0.000000,0.000000,0.000000}%
\pgfsetstrokecolor{textcolor}%
\pgfsetfillcolor{textcolor}%
\pgftext[x=3.849010in,y=2.195543in,left,base]{\color{textcolor}\rmfamily\fontsize{10.000000}{12.000000}\selectfont (c)}%
\end{pgfscope}%
\begin{pgfscope}%
\pgfsetbuttcap%
\pgfsetmiterjoin%
\definecolor{currentfill}{rgb}{1.000000,1.000000,1.000000}%
\pgfsetfillcolor{currentfill}%
\pgfsetlinewidth{0.000000pt}%
\definecolor{currentstroke}{rgb}{0.000000,0.000000,0.000000}%
\pgfsetstrokecolor{currentstroke}%
\pgfsetstrokeopacity{0.000000}%
\pgfsetdash{}{0pt}%
\pgfpathmoveto{\pgfqpoint{5.749209in}{0.061342in}}%
\pgfpathlineto{\pgfqpoint{5.875223in}{0.061342in}}%
\pgfpathlineto{\pgfqpoint{5.875223in}{1.768703in}}%
\pgfpathlineto{\pgfqpoint{5.749209in}{1.768703in}}%
\pgfpathlineto{\pgfqpoint{5.749209in}{0.061342in}}%
\pgfpathclose%
\pgfusepath{fill}%
\end{pgfscope}%
\begin{pgfscope}%
\pgfpathrectangle{\pgfqpoint{5.749209in}{0.061342in}}{\pgfqpoint{0.126014in}{1.707361in}}%
\pgfusepath{clip}%
\pgfsetbuttcap%
\pgfsetmiterjoin%
\definecolor{currentfill}{rgb}{1.000000,1.000000,1.000000}%
\pgfsetfillcolor{currentfill}%
\pgfsetlinewidth{0.010037pt}%
\definecolor{currentstroke}{rgb}{1.000000,1.000000,1.000000}%
\pgfsetstrokecolor{currentstroke}%
\pgfsetdash{}{0pt}%
\pgfusepath{stroke,fill}%
\end{pgfscope}%
\begin{pgfscope}%
\pgfsys@transformshift{5.750000in}{0.063710in}%
\pgftext[left,bottom]{\includegraphics[interpolate=true,width=0.126000in,height=1.706000in]{th_180_450_600-img3.png}}%
\end{pgfscope}%
\begin{pgfscope}%
\pgfsetbuttcap%
\pgfsetroundjoin%
\definecolor{currentfill}{rgb}{0.000000,0.000000,0.000000}%
\pgfsetfillcolor{currentfill}%
\pgfsetlinewidth{0.803000pt}%
\definecolor{currentstroke}{rgb}{0.000000,0.000000,0.000000}%
\pgfsetstrokecolor{currentstroke}%
\pgfsetdash{}{0pt}%
\pgfsys@defobject{currentmarker}{\pgfqpoint{0.000000in}{0.000000in}}{\pgfqpoint{0.048611in}{0.000000in}}{%
\pgfpathmoveto{\pgfqpoint{0.000000in}{0.000000in}}%
\pgfpathlineto{\pgfqpoint{0.048611in}{0.000000in}}%
\pgfusepath{stroke,fill}%
}%
\begin{pgfscope}%
\pgfsys@transformshift{5.875223in}{0.061342in}%
\pgfsys@useobject{currentmarker}{}%
\end{pgfscope}%
\end{pgfscope}%
\begin{pgfscope}%
\definecolor{textcolor}{rgb}{0.000000,0.000000,0.000000}%
\pgfsetstrokecolor{textcolor}%
\pgfsetfillcolor{textcolor}%
\pgftext[x=5.972445in,y=0.061342in,left,]{\color{textcolor}\rmfamily\fontsize{10.000000}{12.000000}\selectfont 1}%
\end{pgfscope}%
\begin{pgfscope}%
\pgfsetbuttcap%
\pgfsetroundjoin%
\definecolor{currentfill}{rgb}{0.000000,0.000000,0.000000}%
\pgfsetfillcolor{currentfill}%
\pgfsetlinewidth{0.803000pt}%
\definecolor{currentstroke}{rgb}{0.000000,0.000000,0.000000}%
\pgfsetstrokecolor{currentstroke}%
\pgfsetdash{}{0pt}%
\pgfsys@defobject{currentmarker}{\pgfqpoint{0.000000in}{0.000000in}}{\pgfqpoint{0.048611in}{0.000000in}}{%
\pgfpathmoveto{\pgfqpoint{0.000000in}{0.000000in}}%
\pgfpathlineto{\pgfqpoint{0.048611in}{0.000000in}}%
\pgfusepath{stroke,fill}%
}%
\begin{pgfscope}%
\pgfsys@transformshift{5.875223in}{0.773353in}%
\pgfsys@useobject{currentmarker}{}%
\end{pgfscope}%
\end{pgfscope}%
\begin{pgfscope}%
\definecolor{textcolor}{rgb}{0.000000,0.000000,0.000000}%
\pgfsetstrokecolor{textcolor}%
\pgfsetfillcolor{textcolor}%
\pgftext[x=5.972445in,y=0.773353in,left,]{\color{textcolor}\rmfamily\fontsize{10.000000}{12.000000}\selectfont 10}%
\end{pgfscope}%
\begin{pgfscope}%
\pgfsetbuttcap%
\pgfsetroundjoin%
\definecolor{currentfill}{rgb}{0.000000,0.000000,0.000000}%
\pgfsetfillcolor{currentfill}%
\pgfsetlinewidth{0.803000pt}%
\definecolor{currentstroke}{rgb}{0.000000,0.000000,0.000000}%
\pgfsetstrokecolor{currentstroke}%
\pgfsetdash{}{0pt}%
\pgfsys@defobject{currentmarker}{\pgfqpoint{0.000000in}{0.000000in}}{\pgfqpoint{0.048611in}{0.000000in}}{%
\pgfpathmoveto{\pgfqpoint{0.000000in}{0.000000in}}%
\pgfpathlineto{\pgfqpoint{0.048611in}{0.000000in}}%
\pgfusepath{stroke,fill}%
}%
\begin{pgfscope}%
\pgfsys@transformshift{5.875223in}{1.485365in}%
\pgfsys@useobject{currentmarker}{}%
\end{pgfscope}%
\end{pgfscope}%
\begin{pgfscope}%
\definecolor{textcolor}{rgb}{0.000000,0.000000,0.000000}%
\pgfsetstrokecolor{textcolor}%
\pgfsetfillcolor{textcolor}%
\pgftext[x=5.972445in,y=1.485365in,left,]{\color{textcolor}\rmfamily\fontsize{10.000000}{12.000000}\selectfont 100}%
\end{pgfscope}%
\begin{pgfscope}%
\pgfsetbuttcap%
\pgfsetroundjoin%
\definecolor{currentfill}{rgb}{0.000000,0.000000,0.000000}%
\pgfsetfillcolor{currentfill}%
\pgfsetlinewidth{0.602250pt}%
\definecolor{currentstroke}{rgb}{0.000000,0.000000,0.000000}%
\pgfsetstrokecolor{currentstroke}%
\pgfsetdash{}{0pt}%
\pgfsys@defobject{currentmarker}{\pgfqpoint{0.000000in}{0.000000in}}{\pgfqpoint{0.027778in}{0.000000in}}{%
\pgfpathmoveto{\pgfqpoint{0.000000in}{0.000000in}}%
\pgfpathlineto{\pgfqpoint{0.027778in}{0.000000in}}%
\pgfusepath{stroke,fill}%
}%
\begin{pgfscope}%
\pgfsys@transformshift{5.875223in}{0.275678in}%
\pgfsys@useobject{currentmarker}{}%
\end{pgfscope}%
\end{pgfscope}%
\begin{pgfscope}%
\pgfsetbuttcap%
\pgfsetroundjoin%
\definecolor{currentfill}{rgb}{0.000000,0.000000,0.000000}%
\pgfsetfillcolor{currentfill}%
\pgfsetlinewidth{0.602250pt}%
\definecolor{currentstroke}{rgb}{0.000000,0.000000,0.000000}%
\pgfsetstrokecolor{currentstroke}%
\pgfsetdash{}{0pt}%
\pgfsys@defobject{currentmarker}{\pgfqpoint{0.000000in}{0.000000in}}{\pgfqpoint{0.027778in}{0.000000in}}{%
\pgfpathmoveto{\pgfqpoint{0.000000in}{0.000000in}}%
\pgfpathlineto{\pgfqpoint{0.027778in}{0.000000in}}%
\pgfusepath{stroke,fill}%
}%
\begin{pgfscope}%
\pgfsys@transformshift{5.875223in}{0.401057in}%
\pgfsys@useobject{currentmarker}{}%
\end{pgfscope}%
\end{pgfscope}%
\begin{pgfscope}%
\pgfsetbuttcap%
\pgfsetroundjoin%
\definecolor{currentfill}{rgb}{0.000000,0.000000,0.000000}%
\pgfsetfillcolor{currentfill}%
\pgfsetlinewidth{0.602250pt}%
\definecolor{currentstroke}{rgb}{0.000000,0.000000,0.000000}%
\pgfsetstrokecolor{currentstroke}%
\pgfsetdash{}{0pt}%
\pgfsys@defobject{currentmarker}{\pgfqpoint{0.000000in}{0.000000in}}{\pgfqpoint{0.027778in}{0.000000in}}{%
\pgfpathmoveto{\pgfqpoint{0.000000in}{0.000000in}}%
\pgfpathlineto{\pgfqpoint{0.027778in}{0.000000in}}%
\pgfusepath{stroke,fill}%
}%
\begin{pgfscope}%
\pgfsys@transformshift{5.875223in}{0.490015in}%
\pgfsys@useobject{currentmarker}{}%
\end{pgfscope}%
\end{pgfscope}%
\begin{pgfscope}%
\pgfsetbuttcap%
\pgfsetroundjoin%
\definecolor{currentfill}{rgb}{0.000000,0.000000,0.000000}%
\pgfsetfillcolor{currentfill}%
\pgfsetlinewidth{0.602250pt}%
\definecolor{currentstroke}{rgb}{0.000000,0.000000,0.000000}%
\pgfsetstrokecolor{currentstroke}%
\pgfsetdash{}{0pt}%
\pgfsys@defobject{currentmarker}{\pgfqpoint{0.000000in}{0.000000in}}{\pgfqpoint{0.027778in}{0.000000in}}{%
\pgfpathmoveto{\pgfqpoint{0.000000in}{0.000000in}}%
\pgfpathlineto{\pgfqpoint{0.027778in}{0.000000in}}%
\pgfusepath{stroke,fill}%
}%
\begin{pgfscope}%
\pgfsys@transformshift{5.875223in}{0.559016in}%
\pgfsys@useobject{currentmarker}{}%
\end{pgfscope}%
\end{pgfscope}%
\begin{pgfscope}%
\pgfsetbuttcap%
\pgfsetroundjoin%
\definecolor{currentfill}{rgb}{0.000000,0.000000,0.000000}%
\pgfsetfillcolor{currentfill}%
\pgfsetlinewidth{0.602250pt}%
\definecolor{currentstroke}{rgb}{0.000000,0.000000,0.000000}%
\pgfsetstrokecolor{currentstroke}%
\pgfsetdash{}{0pt}%
\pgfsys@defobject{currentmarker}{\pgfqpoint{0.000000in}{0.000000in}}{\pgfqpoint{0.027778in}{0.000000in}}{%
\pgfpathmoveto{\pgfqpoint{0.000000in}{0.000000in}}%
\pgfpathlineto{\pgfqpoint{0.027778in}{0.000000in}}%
\pgfusepath{stroke,fill}%
}%
\begin{pgfscope}%
\pgfsys@transformshift{5.875223in}{0.615394in}%
\pgfsys@useobject{currentmarker}{}%
\end{pgfscope}%
\end{pgfscope}%
\begin{pgfscope}%
\pgfsetbuttcap%
\pgfsetroundjoin%
\definecolor{currentfill}{rgb}{0.000000,0.000000,0.000000}%
\pgfsetfillcolor{currentfill}%
\pgfsetlinewidth{0.602250pt}%
\definecolor{currentstroke}{rgb}{0.000000,0.000000,0.000000}%
\pgfsetstrokecolor{currentstroke}%
\pgfsetdash{}{0pt}%
\pgfsys@defobject{currentmarker}{\pgfqpoint{0.000000in}{0.000000in}}{\pgfqpoint{0.027778in}{0.000000in}}{%
\pgfpathmoveto{\pgfqpoint{0.000000in}{0.000000in}}%
\pgfpathlineto{\pgfqpoint{0.027778in}{0.000000in}}%
\pgfusepath{stroke,fill}%
}%
\begin{pgfscope}%
\pgfsys@transformshift{5.875223in}{0.663061in}%
\pgfsys@useobject{currentmarker}{}%
\end{pgfscope}%
\end{pgfscope}%
\begin{pgfscope}%
\pgfsetbuttcap%
\pgfsetroundjoin%
\definecolor{currentfill}{rgb}{0.000000,0.000000,0.000000}%
\pgfsetfillcolor{currentfill}%
\pgfsetlinewidth{0.602250pt}%
\definecolor{currentstroke}{rgb}{0.000000,0.000000,0.000000}%
\pgfsetstrokecolor{currentstroke}%
\pgfsetdash{}{0pt}%
\pgfsys@defobject{currentmarker}{\pgfqpoint{0.000000in}{0.000000in}}{\pgfqpoint{0.027778in}{0.000000in}}{%
\pgfpathmoveto{\pgfqpoint{0.000000in}{0.000000in}}%
\pgfpathlineto{\pgfqpoint{0.027778in}{0.000000in}}%
\pgfusepath{stroke,fill}%
}%
\begin{pgfscope}%
\pgfsys@transformshift{5.875223in}{0.704352in}%
\pgfsys@useobject{currentmarker}{}%
\end{pgfscope}%
\end{pgfscope}%
\begin{pgfscope}%
\pgfsetbuttcap%
\pgfsetroundjoin%
\definecolor{currentfill}{rgb}{0.000000,0.000000,0.000000}%
\pgfsetfillcolor{currentfill}%
\pgfsetlinewidth{0.602250pt}%
\definecolor{currentstroke}{rgb}{0.000000,0.000000,0.000000}%
\pgfsetstrokecolor{currentstroke}%
\pgfsetdash{}{0pt}%
\pgfsys@defobject{currentmarker}{\pgfqpoint{0.000000in}{0.000000in}}{\pgfqpoint{0.027778in}{0.000000in}}{%
\pgfpathmoveto{\pgfqpoint{0.000000in}{0.000000in}}%
\pgfpathlineto{\pgfqpoint{0.027778in}{0.000000in}}%
\pgfusepath{stroke,fill}%
}%
\begin{pgfscope}%
\pgfsys@transformshift{5.875223in}{0.740773in}%
\pgfsys@useobject{currentmarker}{}%
\end{pgfscope}%
\end{pgfscope}%
\begin{pgfscope}%
\pgfsetbuttcap%
\pgfsetroundjoin%
\definecolor{currentfill}{rgb}{0.000000,0.000000,0.000000}%
\pgfsetfillcolor{currentfill}%
\pgfsetlinewidth{0.602250pt}%
\definecolor{currentstroke}{rgb}{0.000000,0.000000,0.000000}%
\pgfsetstrokecolor{currentstroke}%
\pgfsetdash{}{0pt}%
\pgfsys@defobject{currentmarker}{\pgfqpoint{0.000000in}{0.000000in}}{\pgfqpoint{0.027778in}{0.000000in}}{%
\pgfpathmoveto{\pgfqpoint{0.000000in}{0.000000in}}%
\pgfpathlineto{\pgfqpoint{0.027778in}{0.000000in}}%
\pgfusepath{stroke,fill}%
}%
\begin{pgfscope}%
\pgfsys@transformshift{5.875223in}{0.987690in}%
\pgfsys@useobject{currentmarker}{}%
\end{pgfscope}%
\end{pgfscope}%
\begin{pgfscope}%
\pgfsetbuttcap%
\pgfsetroundjoin%
\definecolor{currentfill}{rgb}{0.000000,0.000000,0.000000}%
\pgfsetfillcolor{currentfill}%
\pgfsetlinewidth{0.602250pt}%
\definecolor{currentstroke}{rgb}{0.000000,0.000000,0.000000}%
\pgfsetstrokecolor{currentstroke}%
\pgfsetdash{}{0pt}%
\pgfsys@defobject{currentmarker}{\pgfqpoint{0.000000in}{0.000000in}}{\pgfqpoint{0.027778in}{0.000000in}}{%
\pgfpathmoveto{\pgfqpoint{0.000000in}{0.000000in}}%
\pgfpathlineto{\pgfqpoint{0.027778in}{0.000000in}}%
\pgfusepath{stroke,fill}%
}%
\begin{pgfscope}%
\pgfsys@transformshift{5.875223in}{1.113069in}%
\pgfsys@useobject{currentmarker}{}%
\end{pgfscope}%
\end{pgfscope}%
\begin{pgfscope}%
\pgfsetbuttcap%
\pgfsetroundjoin%
\definecolor{currentfill}{rgb}{0.000000,0.000000,0.000000}%
\pgfsetfillcolor{currentfill}%
\pgfsetlinewidth{0.602250pt}%
\definecolor{currentstroke}{rgb}{0.000000,0.000000,0.000000}%
\pgfsetstrokecolor{currentstroke}%
\pgfsetdash{}{0pt}%
\pgfsys@defobject{currentmarker}{\pgfqpoint{0.000000in}{0.000000in}}{\pgfqpoint{0.027778in}{0.000000in}}{%
\pgfpathmoveto{\pgfqpoint{0.000000in}{0.000000in}}%
\pgfpathlineto{\pgfqpoint{0.027778in}{0.000000in}}%
\pgfusepath{stroke,fill}%
}%
\begin{pgfscope}%
\pgfsys@transformshift{5.875223in}{1.202027in}%
\pgfsys@useobject{currentmarker}{}%
\end{pgfscope}%
\end{pgfscope}%
\begin{pgfscope}%
\pgfsetbuttcap%
\pgfsetroundjoin%
\definecolor{currentfill}{rgb}{0.000000,0.000000,0.000000}%
\pgfsetfillcolor{currentfill}%
\pgfsetlinewidth{0.602250pt}%
\definecolor{currentstroke}{rgb}{0.000000,0.000000,0.000000}%
\pgfsetstrokecolor{currentstroke}%
\pgfsetdash{}{0pt}%
\pgfsys@defobject{currentmarker}{\pgfqpoint{0.000000in}{0.000000in}}{\pgfqpoint{0.027778in}{0.000000in}}{%
\pgfpathmoveto{\pgfqpoint{0.000000in}{0.000000in}}%
\pgfpathlineto{\pgfqpoint{0.027778in}{0.000000in}}%
\pgfusepath{stroke,fill}%
}%
\begin{pgfscope}%
\pgfsys@transformshift{5.875223in}{1.271028in}%
\pgfsys@useobject{currentmarker}{}%
\end{pgfscope}%
\end{pgfscope}%
\begin{pgfscope}%
\pgfsetbuttcap%
\pgfsetroundjoin%
\definecolor{currentfill}{rgb}{0.000000,0.000000,0.000000}%
\pgfsetfillcolor{currentfill}%
\pgfsetlinewidth{0.602250pt}%
\definecolor{currentstroke}{rgb}{0.000000,0.000000,0.000000}%
\pgfsetstrokecolor{currentstroke}%
\pgfsetdash{}{0pt}%
\pgfsys@defobject{currentmarker}{\pgfqpoint{0.000000in}{0.000000in}}{\pgfqpoint{0.027778in}{0.000000in}}{%
\pgfpathmoveto{\pgfqpoint{0.000000in}{0.000000in}}%
\pgfpathlineto{\pgfqpoint{0.027778in}{0.000000in}}%
\pgfusepath{stroke,fill}%
}%
\begin{pgfscope}%
\pgfsys@transformshift{5.875223in}{1.327406in}%
\pgfsys@useobject{currentmarker}{}%
\end{pgfscope}%
\end{pgfscope}%
\begin{pgfscope}%
\pgfsetbuttcap%
\pgfsetroundjoin%
\definecolor{currentfill}{rgb}{0.000000,0.000000,0.000000}%
\pgfsetfillcolor{currentfill}%
\pgfsetlinewidth{0.602250pt}%
\definecolor{currentstroke}{rgb}{0.000000,0.000000,0.000000}%
\pgfsetstrokecolor{currentstroke}%
\pgfsetdash{}{0pt}%
\pgfsys@defobject{currentmarker}{\pgfqpoint{0.000000in}{0.000000in}}{\pgfqpoint{0.027778in}{0.000000in}}{%
\pgfpathmoveto{\pgfqpoint{0.000000in}{0.000000in}}%
\pgfpathlineto{\pgfqpoint{0.027778in}{0.000000in}}%
\pgfusepath{stroke,fill}%
}%
\begin{pgfscope}%
\pgfsys@transformshift{5.875223in}{1.375073in}%
\pgfsys@useobject{currentmarker}{}%
\end{pgfscope}%
\end{pgfscope}%
\begin{pgfscope}%
\pgfsetbuttcap%
\pgfsetroundjoin%
\definecolor{currentfill}{rgb}{0.000000,0.000000,0.000000}%
\pgfsetfillcolor{currentfill}%
\pgfsetlinewidth{0.602250pt}%
\definecolor{currentstroke}{rgb}{0.000000,0.000000,0.000000}%
\pgfsetstrokecolor{currentstroke}%
\pgfsetdash{}{0pt}%
\pgfsys@defobject{currentmarker}{\pgfqpoint{0.000000in}{0.000000in}}{\pgfqpoint{0.027778in}{0.000000in}}{%
\pgfpathmoveto{\pgfqpoint{0.000000in}{0.000000in}}%
\pgfpathlineto{\pgfqpoint{0.027778in}{0.000000in}}%
\pgfusepath{stroke,fill}%
}%
\begin{pgfscope}%
\pgfsys@transformshift{5.875223in}{1.416364in}%
\pgfsys@useobject{currentmarker}{}%
\end{pgfscope}%
\end{pgfscope}%
\begin{pgfscope}%
\pgfsetbuttcap%
\pgfsetroundjoin%
\definecolor{currentfill}{rgb}{0.000000,0.000000,0.000000}%
\pgfsetfillcolor{currentfill}%
\pgfsetlinewidth{0.602250pt}%
\definecolor{currentstroke}{rgb}{0.000000,0.000000,0.000000}%
\pgfsetstrokecolor{currentstroke}%
\pgfsetdash{}{0pt}%
\pgfsys@defobject{currentmarker}{\pgfqpoint{0.000000in}{0.000000in}}{\pgfqpoint{0.027778in}{0.000000in}}{%
\pgfpathmoveto{\pgfqpoint{0.000000in}{0.000000in}}%
\pgfpathlineto{\pgfqpoint{0.027778in}{0.000000in}}%
\pgfusepath{stroke,fill}%
}%
\begin{pgfscope}%
\pgfsys@transformshift{5.875223in}{1.452785in}%
\pgfsys@useobject{currentmarker}{}%
\end{pgfscope}%
\end{pgfscope}%
\begin{pgfscope}%
\pgfsetbuttcap%
\pgfsetroundjoin%
\definecolor{currentfill}{rgb}{0.000000,0.000000,0.000000}%
\pgfsetfillcolor{currentfill}%
\pgfsetlinewidth{0.602250pt}%
\definecolor{currentstroke}{rgb}{0.000000,0.000000,0.000000}%
\pgfsetstrokecolor{currentstroke}%
\pgfsetdash{}{0pt}%
\pgfsys@defobject{currentmarker}{\pgfqpoint{0.000000in}{0.000000in}}{\pgfqpoint{0.027778in}{0.000000in}}{%
\pgfpathmoveto{\pgfqpoint{0.000000in}{0.000000in}}%
\pgfpathlineto{\pgfqpoint{0.027778in}{0.000000in}}%
\pgfusepath{stroke,fill}%
}%
\begin{pgfscope}%
\pgfsys@transformshift{5.875223in}{1.699702in}%
\pgfsys@useobject{currentmarker}{}%
\end{pgfscope}%
\end{pgfscope}%
\begin{pgfscope}%
\pgfsetrectcap%
\pgfsetmiterjoin%
\pgfsetlinewidth{0.803000pt}%
\definecolor{currentstroke}{rgb}{0.000000,0.000000,0.000000}%
\pgfsetstrokecolor{currentstroke}%
\pgfsetdash{}{0pt}%
\pgfpathmoveto{\pgfqpoint{5.749209in}{0.061342in}}%
\pgfpathlineto{\pgfqpoint{5.812216in}{0.061342in}}%
\pgfpathlineto{\pgfqpoint{5.875223in}{0.061342in}}%
\pgfpathlineto{\pgfqpoint{5.875223in}{1.768703in}}%
\pgfpathlineto{\pgfqpoint{5.812216in}{1.768703in}}%
\pgfpathlineto{\pgfqpoint{5.749209in}{1.768703in}}%
\pgfpathlineto{\pgfqpoint{5.749209in}{0.061342in}}%
\pgfpathclose%
\pgfusepath{stroke}%
\end{pgfscope}%
\end{pgfpicture}%
\makeatother%
\endgroup%

    \caption{Die Summe von ausgewerteten \num{50000} Aufnahmen mit dem Schwellenwert $s_V$ (a) \SI{180}{\adu}, (b) \SI{450}{\adu} und (c) \SI{600}{\adu}. Diejenigen Photonen, die mit allen Schwellenwerten detektiert wurden, werden grün eingekreist.}
    \label{fig:th_180_450_600}
\end{figure}
\noindent
Die ausgewertete Summe von \SI{50000}{\captures} wird in Polarkoordinaten transformiert. Dafür ist es nötig, den Mittelpunktes der Transformation von Koordinatensystem festzulegen. Die resonante magnetische Streuung ist sehr energieselektiv und findet überwiegend an der Resonanzfrequenz statt, die sich in den Streubildern als eine schmale horizontale dunklere Linie im Direktstrahl identifizieren lässt. Die Absorptionslinie bestimmt also die vertikale Position des Mittelpunktes. Die horizontale Koordinate des Mittelpunktes wird durch die Anpassung eines elliptischen Umrisses ermittelt (Abb. \ref{fig:th-100-200-maske-radial-transform}a).
\begin{figure}[H]
    \centering
    %% Creator: Matplotlib, PGF backend
%%
%% To include the figure in your LaTeX document, write
%%   \input{<filename>.pgf}
%%
%% Make sure the required packages are loaded in your preamble
%%   \usepackage{pgf}
%%
%% Also ensure that all the required font packages are loaded; for instance,
%% the lmodern package is sometimes necessary when using math font.
%%   \usepackage{lmodern}
%%
%% Figures using additional raster images can only be included by \input if
%% they are in the same directory as the main LaTeX file. For loading figures
%% from other directories you can use the `import` package
%%   \usepackage{import}
%%
%% and then include the figures with
%%   \import{<path to file>}{<filename>.pgf}
%%
%% Matplotlib used the following preamble
%%   \usepackage{amsmath} \usepackage[utf8]{inputenc} \usepackage[T1]{fontenc} \usepackage[output-decimal-marker={,},print-unity-mantissa=false]{siunitx} \sisetup{per-mode=fraction, separate-uncertainty = true, locale = DE} \usepackage[acronym, toc, section=section, nonumberlist, nopostdot]{glossaries-extra} \DeclareSIUnit\adu{\text{ADU}} \DeclareSIUnit\px{\text{px}} \DeclareSIUnit\photons{\text{Pho\-to\-nen}} \DeclareSIUnit\photon{\text{Pho\-ton}}
%%
\begingroup%
\makeatletter%
\begin{pgfpicture}%
\pgfpathrectangle{\pgfpointorigin}{\pgfqpoint{6.068283in}{6.036411in}}%
\pgfusepath{use as bounding box, clip}%
\begin{pgfscope}%
\pgfsetbuttcap%
\pgfsetmiterjoin%
\pgfsetlinewidth{0.000000pt}%
\definecolor{currentstroke}{rgb}{1.000000,1.000000,1.000000}%
\pgfsetstrokecolor{currentstroke}%
\pgfsetstrokeopacity{0.000000}%
\pgfsetdash{}{0pt}%
\pgfpathmoveto{\pgfqpoint{0.000000in}{0.000000in}}%
\pgfpathlineto{\pgfqpoint{6.068283in}{0.000000in}}%
\pgfpathlineto{\pgfqpoint{6.068283in}{6.036411in}}%
\pgfpathlineto{\pgfqpoint{0.000000in}{6.036411in}}%
\pgfpathlineto{\pgfqpoint{0.000000in}{0.000000in}}%
\pgfpathclose%
\pgfusepath{}%
\end{pgfscope}%
\begin{pgfscope}%
\pgfsetbuttcap%
\pgfsetmiterjoin%
\pgfsetlinewidth{0.000000pt}%
\definecolor{currentstroke}{rgb}{1.000000,1.000000,1.000000}%
\pgfsetstrokecolor{currentstroke}%
\pgfsetstrokeopacity{0.000000}%
\pgfsetdash{}{0pt}%
\pgfpathmoveto{\pgfqpoint{-0.243853in}{3.081900in}}%
\pgfpathlineto{\pgfqpoint{6.236147in}{3.081900in}}%
\pgfpathlineto{\pgfqpoint{6.236147in}{6.081900in}}%
\pgfpathlineto{\pgfqpoint{-0.243853in}{6.081900in}}%
\pgfpathlineto{\pgfqpoint{-0.243853in}{3.081900in}}%
\pgfpathclose%
\pgfusepath{}%
\end{pgfscope}%
\begin{pgfscope}%
\pgfsetbuttcap%
\pgfsetmiterjoin%
\definecolor{currentfill}{rgb}{1.000000,1.000000,1.000000}%
\pgfsetfillcolor{currentfill}%
\pgfsetlinewidth{0.000000pt}%
\definecolor{currentstroke}{rgb}{0.000000,0.000000,0.000000}%
\pgfsetstrokecolor{currentstroke}%
\pgfsetstrokeopacity{0.000000}%
\pgfsetdash{}{0pt}%
\pgfpathmoveto{\pgfqpoint{0.501376in}{3.456592in}}%
\pgfpathlineto{\pgfqpoint{2.751993in}{3.456592in}}%
\pgfpathlineto{\pgfqpoint{2.751993in}{5.707208in}}%
\pgfpathlineto{\pgfqpoint{0.501376in}{5.707208in}}%
\pgfpathlineto{\pgfqpoint{0.501376in}{3.456592in}}%
\pgfpathclose%
\pgfusepath{fill}%
\end{pgfscope}%
\begin{pgfscope}%
\pgfsys@transformshift{0.502000in}{3.456411in}%
\pgftext[left,bottom]{\includegraphics[interpolate=true,width=2.250000in,height=2.252000in]{th_100_200_masked_radial_transform-img0.png}}%
\end{pgfscope}%
\begin{pgfscope}%
\pgfsetrectcap%
\pgfsetmiterjoin%
\pgfsetlinewidth{0.803000pt}%
\definecolor{currentstroke}{rgb}{0.000000,0.000000,0.000000}%
\pgfsetstrokecolor{currentstroke}%
\pgfsetdash{}{0pt}%
\pgfpathmoveto{\pgfqpoint{0.501376in}{3.456592in}}%
\pgfpathlineto{\pgfqpoint{0.501376in}{5.707208in}}%
\pgfusepath{stroke}%
\end{pgfscope}%
\begin{pgfscope}%
\pgfsetrectcap%
\pgfsetmiterjoin%
\pgfsetlinewidth{0.803000pt}%
\definecolor{currentstroke}{rgb}{0.000000,0.000000,0.000000}%
\pgfsetstrokecolor{currentstroke}%
\pgfsetdash{}{0pt}%
\pgfpathmoveto{\pgfqpoint{2.751993in}{3.456592in}}%
\pgfpathlineto{\pgfqpoint{2.751993in}{5.707208in}}%
\pgfusepath{stroke}%
\end{pgfscope}%
\begin{pgfscope}%
\pgfsetrectcap%
\pgfsetmiterjoin%
\pgfsetlinewidth{0.803000pt}%
\definecolor{currentstroke}{rgb}{0.000000,0.000000,0.000000}%
\pgfsetstrokecolor{currentstroke}%
\pgfsetdash{}{0pt}%
\pgfpathmoveto{\pgfqpoint{0.501376in}{3.456592in}}%
\pgfpathlineto{\pgfqpoint{2.751993in}{3.456592in}}%
\pgfusepath{stroke}%
\end{pgfscope}%
\begin{pgfscope}%
\pgfsetrectcap%
\pgfsetmiterjoin%
\pgfsetlinewidth{0.803000pt}%
\definecolor{currentstroke}{rgb}{0.000000,0.000000,0.000000}%
\pgfsetstrokecolor{currentstroke}%
\pgfsetdash{}{0pt}%
\pgfpathmoveto{\pgfqpoint{0.501376in}{5.707208in}}%
\pgfpathlineto{\pgfqpoint{2.751993in}{5.707208in}}%
\pgfusepath{stroke}%
\end{pgfscope}%
\begin{pgfscope}%
\definecolor{textcolor}{rgb}{0.000000,0.000000,0.000000}%
\pgfsetstrokecolor{textcolor}%
\pgfsetfillcolor{textcolor}%
\pgftext[x=0.276315in,y=5.932270in,left,base]{\color{textcolor}\rmfamily\fontsize{10.000000}{12.000000}\selectfont (a)}%
\end{pgfscope}%
\begin{pgfscope}%
\pgfpathrectangle{\pgfqpoint{0.501376in}{3.456592in}}{\pgfqpoint{2.250617in}{2.250617in}}%
\pgfusepath{clip}%
\pgfsetbuttcap%
\pgfsetmiterjoin%
\definecolor{currentfill}{rgb}{0.000000,0.501961,0.000000}%
\pgfsetfillcolor{currentfill}%
\pgfsetlinewidth{1.003750pt}%
\definecolor{currentstroke}{rgb}{0.000000,0.501961,0.000000}%
\pgfsetstrokecolor{currentstroke}%
\pgfsetdash{}{0pt}%
\pgfpathmoveto{\pgfqpoint{1.579797in}{4.431859in}}%
\pgfpathcurveto{\pgfqpoint{1.584771in}{4.431859in}}{\pgfqpoint{1.589542in}{4.433835in}}{\pgfqpoint{1.593059in}{4.437352in}}%
\pgfpathcurveto{\pgfqpoint{1.596576in}{4.440869in}}{\pgfqpoint{1.598552in}{4.445640in}}{\pgfqpoint{1.598552in}{4.450614in}}%
\pgfpathcurveto{\pgfqpoint{1.598552in}{4.455588in}}{\pgfqpoint{1.596576in}{4.460359in}}{\pgfqpoint{1.593059in}{4.463876in}}%
\pgfpathcurveto{\pgfqpoint{1.589542in}{4.467393in}}{\pgfqpoint{1.584771in}{4.469369in}}{\pgfqpoint{1.579797in}{4.469369in}}%
\pgfpathcurveto{\pgfqpoint{1.574823in}{4.469369in}}{\pgfqpoint{1.570052in}{4.467393in}}{\pgfqpoint{1.566535in}{4.463876in}}%
\pgfpathcurveto{\pgfqpoint{1.563018in}{4.460359in}}{\pgfqpoint{1.561042in}{4.455588in}}{\pgfqpoint{1.561042in}{4.450614in}}%
\pgfpathcurveto{\pgfqpoint{1.561042in}{4.445640in}}{\pgfqpoint{1.563018in}{4.440869in}}{\pgfqpoint{1.566535in}{4.437352in}}%
\pgfpathcurveto{\pgfqpoint{1.570052in}{4.433835in}}{\pgfqpoint{1.574823in}{4.431859in}}{\pgfqpoint{1.579797in}{4.431859in}}%
\pgfpathlineto{\pgfqpoint{1.579797in}{4.431859in}}%
\pgfpathclose%
\pgfusepath{stroke,fill}%
\end{pgfscope}%
\begin{pgfscope}%
\pgfpathrectangle{\pgfqpoint{0.501376in}{3.456592in}}{\pgfqpoint{2.250617in}{2.250617in}}%
\pgfusepath{clip}%
\pgfsetbuttcap%
\pgfsetmiterjoin%
\pgfsetlinewidth{3.011250pt}%
\definecolor{currentstroke}{rgb}{0.000000,0.501961,0.000000}%
\pgfsetstrokecolor{currentstroke}%
\pgfsetdash{}{0pt}%
\pgfpathmoveto{\pgfqpoint{1.579797in}{3.559745in}}%
\pgfpathcurveto{\pgfqpoint{1.828493in}{3.559745in}}{\pgfqpoint{2.067036in}{3.653613in}}{\pgfqpoint{2.242891in}{3.820674in}}%
\pgfpathcurveto{\pgfqpoint{2.418746in}{3.987736in}}{\pgfqpoint{2.517554in}{4.214353in}}{\pgfqpoint{2.517554in}{4.450614in}}%
\pgfpathcurveto{\pgfqpoint{2.517554in}{4.686875in}}{\pgfqpoint{2.418746in}{4.913492in}}{\pgfqpoint{2.242891in}{5.080554in}}%
\pgfpathcurveto{\pgfqpoint{2.067036in}{5.247615in}}{\pgfqpoint{1.828493in}{5.341483in}}{\pgfqpoint{1.579797in}{5.341483in}}%
\pgfpathcurveto{\pgfqpoint{1.331101in}{5.341483in}}{\pgfqpoint{1.092557in}{5.247615in}}{\pgfqpoint{0.916703in}{5.080554in}}%
\pgfpathcurveto{\pgfqpoint{0.740848in}{4.913492in}}{\pgfqpoint{0.642040in}{4.686875in}}{\pgfqpoint{0.642040in}{4.450614in}}%
\pgfpathcurveto{\pgfqpoint{0.642040in}{4.214353in}}{\pgfqpoint{0.740848in}{3.987736in}}{\pgfqpoint{0.916703in}{3.820674in}}%
\pgfpathcurveto{\pgfqpoint{1.092557in}{3.653613in}}{\pgfqpoint{1.331101in}{3.559745in}}{\pgfqpoint{1.579797in}{3.559745in}}%
\pgfpathlineto{\pgfqpoint{1.579797in}{3.559745in}}%
\pgfpathclose%
\pgfusepath{stroke}%
\end{pgfscope}%
\begin{pgfscope}%
\pgfsetbuttcap%
\pgfsetmiterjoin%
\definecolor{currentfill}{rgb}{1.000000,1.000000,1.000000}%
\pgfsetfillcolor{currentfill}%
\pgfsetlinewidth{0.000000pt}%
\definecolor{currentstroke}{rgb}{0.000000,0.000000,0.000000}%
\pgfsetstrokecolor{currentstroke}%
\pgfsetstrokeopacity{0.000000}%
\pgfsetdash{}{0pt}%
\pgfpathmoveto{\pgfqpoint{2.951993in}{3.456592in}}%
\pgfpathlineto{\pgfqpoint{5.202609in}{3.456592in}}%
\pgfpathlineto{\pgfqpoint{5.202609in}{5.707208in}}%
\pgfpathlineto{\pgfqpoint{2.951993in}{5.707208in}}%
\pgfpathlineto{\pgfqpoint{2.951993in}{3.456592in}}%
\pgfpathclose%
\pgfusepath{fill}%
\end{pgfscope}%
\begin{pgfscope}%
\pgfsys@transformshift{2.952000in}{3.456411in}%
\pgftext[left,bottom]{\includegraphics[interpolate=true,width=2.250000in,height=2.252000in]{th_100_200_masked_radial_transform-img1.png}}%
\end{pgfscope}%
\begin{pgfscope}%
\pgfsetrectcap%
\pgfsetmiterjoin%
\pgfsetlinewidth{0.803000pt}%
\definecolor{currentstroke}{rgb}{0.000000,0.000000,0.000000}%
\pgfsetstrokecolor{currentstroke}%
\pgfsetdash{}{0pt}%
\pgfpathmoveto{\pgfqpoint{2.951993in}{3.456592in}}%
\pgfpathlineto{\pgfqpoint{2.951993in}{5.707208in}}%
\pgfusepath{stroke}%
\end{pgfscope}%
\begin{pgfscope}%
\pgfsetrectcap%
\pgfsetmiterjoin%
\pgfsetlinewidth{0.803000pt}%
\definecolor{currentstroke}{rgb}{0.000000,0.000000,0.000000}%
\pgfsetstrokecolor{currentstroke}%
\pgfsetdash{}{0pt}%
\pgfpathmoveto{\pgfqpoint{5.202609in}{3.456592in}}%
\pgfpathlineto{\pgfqpoint{5.202609in}{5.707208in}}%
\pgfusepath{stroke}%
\end{pgfscope}%
\begin{pgfscope}%
\pgfsetrectcap%
\pgfsetmiterjoin%
\pgfsetlinewidth{0.803000pt}%
\definecolor{currentstroke}{rgb}{0.000000,0.000000,0.000000}%
\pgfsetstrokecolor{currentstroke}%
\pgfsetdash{}{0pt}%
\pgfpathmoveto{\pgfqpoint{2.951993in}{3.456592in}}%
\pgfpathlineto{\pgfqpoint{5.202609in}{3.456592in}}%
\pgfusepath{stroke}%
\end{pgfscope}%
\begin{pgfscope}%
\pgfsetrectcap%
\pgfsetmiterjoin%
\pgfsetlinewidth{0.803000pt}%
\definecolor{currentstroke}{rgb}{0.000000,0.000000,0.000000}%
\pgfsetstrokecolor{currentstroke}%
\pgfsetdash{}{0pt}%
\pgfpathmoveto{\pgfqpoint{2.951993in}{5.707208in}}%
\pgfpathlineto{\pgfqpoint{5.202609in}{5.707208in}}%
\pgfusepath{stroke}%
\end{pgfscope}%
\begin{pgfscope}%
\definecolor{textcolor}{rgb}{0.000000,0.000000,0.000000}%
\pgfsetstrokecolor{textcolor}%
\pgfsetfillcolor{textcolor}%
\pgftext[x=2.726931in,y=5.932270in,left,base]{\color{textcolor}\rmfamily\fontsize{10.000000}{12.000000}\selectfont (b)}%
\end{pgfscope}%
\begin{pgfscope}%
\definecolor{textcolor}{rgb}{0.000000,0.000000,0.000000}%
\pgfsetstrokecolor{textcolor}%
\pgfsetfillcolor{textcolor}%
\pgftext[x=4.030413in,y=4.169287in,,base]{\color{textcolor}\rmfamily\fontsize{10.000000}{12.000000}\selectfont \(\displaystyle \varphi\)}%
\end{pgfscope}%
\begin{pgfscope}%
\pgfpathrectangle{\pgfqpoint{2.951993in}{3.456592in}}{\pgfqpoint{2.250617in}{2.250617in}}%
\pgfusepath{clip}%
\pgfsetbuttcap%
\pgfsetmiterjoin%
\definecolor{currentfill}{rgb}{0.000000,0.501961,0.000000}%
\pgfsetfillcolor{currentfill}%
\pgfsetlinewidth{1.003750pt}%
\definecolor{currentstroke}{rgb}{0.000000,0.501961,0.000000}%
\pgfsetstrokecolor{currentstroke}%
\pgfsetdash{}{0pt}%
\pgfpathmoveto{\pgfqpoint{4.030413in}{4.431859in}}%
\pgfpathcurveto{\pgfqpoint{4.035387in}{4.431859in}}{\pgfqpoint{4.040158in}{4.433835in}}{\pgfqpoint{4.043675in}{4.437352in}}%
\pgfpathcurveto{\pgfqpoint{4.047192in}{4.440869in}}{\pgfqpoint{4.049168in}{4.445640in}}{\pgfqpoint{4.049168in}{4.450614in}}%
\pgfpathcurveto{\pgfqpoint{4.049168in}{4.455588in}}{\pgfqpoint{4.047192in}{4.460359in}}{\pgfqpoint{4.043675in}{4.463876in}}%
\pgfpathcurveto{\pgfqpoint{4.040158in}{4.467393in}}{\pgfqpoint{4.035387in}{4.469369in}}{\pgfqpoint{4.030413in}{4.469369in}}%
\pgfpathcurveto{\pgfqpoint{4.025439in}{4.469369in}}{\pgfqpoint{4.020669in}{4.467393in}}{\pgfqpoint{4.017151in}{4.463876in}}%
\pgfpathcurveto{\pgfqpoint{4.013634in}{4.460359in}}{\pgfqpoint{4.011658in}{4.455588in}}{\pgfqpoint{4.011658in}{4.450614in}}%
\pgfpathcurveto{\pgfqpoint{4.011658in}{4.445640in}}{\pgfqpoint{4.013634in}{4.440869in}}{\pgfqpoint{4.017151in}{4.437352in}}%
\pgfpathcurveto{\pgfqpoint{4.020669in}{4.433835in}}{\pgfqpoint{4.025439in}{4.431859in}}{\pgfqpoint{4.030413in}{4.431859in}}%
\pgfpathlineto{\pgfqpoint{4.030413in}{4.431859in}}%
\pgfpathclose%
\pgfusepath{stroke,fill}%
\end{pgfscope}%
\begin{pgfscope}%
\definecolor{textcolor}{rgb}{0.000000,0.000000,0.000000}%
\pgfsetstrokecolor{textcolor}%
\pgfsetfillcolor{textcolor}%
\pgftext[x=4.030413in,y=4.544390in,,base]{\color{textcolor}\rmfamily\fontsize{10.000000}{12.000000}\selectfont \(\displaystyle r = \SI{0}{px}\)}%
\end{pgfscope}%
\begin{pgfscope}%
\pgfsetbuttcap%
\pgfsetmiterjoin%
\definecolor{currentfill}{rgb}{1.000000,1.000000,1.000000}%
\pgfsetfillcolor{currentfill}%
\pgfsetlinewidth{0.000000pt}%
\definecolor{currentstroke}{rgb}{0.000000,0.000000,0.000000}%
\pgfsetstrokecolor{currentstroke}%
\pgfsetstrokeopacity{0.000000}%
\pgfsetdash{}{0pt}%
\pgfpathmoveto{\pgfqpoint{5.402609in}{3.456592in}}%
\pgfpathlineto{\pgfqpoint{5.515140in}{3.456592in}}%
\pgfpathlineto{\pgfqpoint{5.515140in}{5.707208in}}%
\pgfpathlineto{\pgfqpoint{5.402609in}{5.707208in}}%
\pgfpathlineto{\pgfqpoint{5.402609in}{3.456592in}}%
\pgfpathclose%
\pgfusepath{fill}%
\end{pgfscope}%
\begin{pgfscope}%
\pgfpathrectangle{\pgfqpoint{5.402609in}{3.456592in}}{\pgfqpoint{0.112531in}{2.250617in}}%
\pgfusepath{clip}%
\pgfsetbuttcap%
\pgfsetmiterjoin%
\definecolor{currentfill}{rgb}{1.000000,1.000000,1.000000}%
\pgfsetfillcolor{currentfill}%
\pgfsetlinewidth{0.010037pt}%
\definecolor{currentstroke}{rgb}{1.000000,1.000000,1.000000}%
\pgfsetstrokecolor{currentstroke}%
\pgfsetdash{}{0pt}%
\pgfusepath{stroke,fill}%
\end{pgfscope}%
\begin{pgfscope}%
\pgfpathrectangle{\pgfqpoint{5.402609in}{3.456592in}}{\pgfqpoint{0.112531in}{2.250617in}}%
\pgfusepath{clip}%
\pgfsetbuttcap%
\pgfsetmiterjoin%
\definecolor{currentfill}{rgb}{1.000000,1.000000,1.000000}%
\pgfsetfillcolor{currentfill}%
\pgfsetlinewidth{0.010037pt}%
\definecolor{currentstroke}{rgb}{1.000000,1.000000,1.000000}%
\pgfsetstrokecolor{currentstroke}%
\pgfsetdash{}{0pt}%
\pgfusepath{stroke,fill}%
\end{pgfscope}%
\begin{pgfscope}%
\pgfsys@transformshift{5.402000in}{3.456411in}%
\pgftext[left,bottom]{\includegraphics[interpolate=true,width=0.114000in,height=2.252000in]{th_100_200_masked_radial_transform-img2.png}}%
\end{pgfscope}%
\begin{pgfscope}%
\pgfsetbuttcap%
\pgfsetroundjoin%
\definecolor{currentfill}{rgb}{0.000000,0.000000,0.000000}%
\pgfsetfillcolor{currentfill}%
\pgfsetlinewidth{0.803000pt}%
\definecolor{currentstroke}{rgb}{0.000000,0.000000,0.000000}%
\pgfsetstrokecolor{currentstroke}%
\pgfsetdash{}{0pt}%
\pgfsys@defobject{currentmarker}{\pgfqpoint{0.000000in}{0.000000in}}{\pgfqpoint{0.048611in}{0.000000in}}{%
\pgfpathmoveto{\pgfqpoint{0.000000in}{0.000000in}}%
\pgfpathlineto{\pgfqpoint{0.048611in}{0.000000in}}%
\pgfusepath{stroke,fill}%
}%
\begin{pgfscope}%
\pgfsys@transformshift{5.515140in}{3.456592in}%
\pgfsys@useobject{currentmarker}{}%
\end{pgfscope}%
\end{pgfscope}%
\begin{pgfscope}%
\definecolor{textcolor}{rgb}{0.000000,0.000000,0.000000}%
\pgfsetstrokecolor{textcolor}%
\pgfsetfillcolor{textcolor}%
\pgftext[x=5.612362in, y=3.408764in, left, base]{\color{textcolor}\rmfamily\fontsize{10.000000}{12.000000}\selectfont 1}%
\end{pgfscope}%
\begin{pgfscope}%
\pgfsetbuttcap%
\pgfsetroundjoin%
\definecolor{currentfill}{rgb}{0.000000,0.000000,0.000000}%
\pgfsetfillcolor{currentfill}%
\pgfsetlinewidth{0.803000pt}%
\definecolor{currentstroke}{rgb}{0.000000,0.000000,0.000000}%
\pgfsetstrokecolor{currentstroke}%
\pgfsetdash{}{0pt}%
\pgfsys@defobject{currentmarker}{\pgfqpoint{0.000000in}{0.000000in}}{\pgfqpoint{0.048611in}{0.000000in}}{%
\pgfpathmoveto{\pgfqpoint{0.000000in}{0.000000in}}%
\pgfpathlineto{\pgfqpoint{0.048611in}{0.000000in}}%
\pgfusepath{stroke,fill}%
}%
\begin{pgfscope}%
\pgfsys@transformshift{5.515140in}{4.206797in}%
\pgfsys@useobject{currentmarker}{}%
\end{pgfscope}%
\end{pgfscope}%
\begin{pgfscope}%
\definecolor{textcolor}{rgb}{0.000000,0.000000,0.000000}%
\pgfsetstrokecolor{textcolor}%
\pgfsetfillcolor{textcolor}%
\pgftext[x=5.612362in, y=4.158969in, left, base]{\color{textcolor}\rmfamily\fontsize{10.000000}{12.000000}\selectfont 10}%
\end{pgfscope}%
\begin{pgfscope}%
\pgfsetbuttcap%
\pgfsetroundjoin%
\definecolor{currentfill}{rgb}{0.000000,0.000000,0.000000}%
\pgfsetfillcolor{currentfill}%
\pgfsetlinewidth{0.803000pt}%
\definecolor{currentstroke}{rgb}{0.000000,0.000000,0.000000}%
\pgfsetstrokecolor{currentstroke}%
\pgfsetdash{}{0pt}%
\pgfsys@defobject{currentmarker}{\pgfqpoint{0.000000in}{0.000000in}}{\pgfqpoint{0.048611in}{0.000000in}}{%
\pgfpathmoveto{\pgfqpoint{0.000000in}{0.000000in}}%
\pgfpathlineto{\pgfqpoint{0.048611in}{0.000000in}}%
\pgfusepath{stroke,fill}%
}%
\begin{pgfscope}%
\pgfsys@transformshift{5.515140in}{4.957003in}%
\pgfsys@useobject{currentmarker}{}%
\end{pgfscope}%
\end{pgfscope}%
\begin{pgfscope}%
\definecolor{textcolor}{rgb}{0.000000,0.000000,0.000000}%
\pgfsetstrokecolor{textcolor}%
\pgfsetfillcolor{textcolor}%
\pgftext[x=5.612362in, y=4.909175in, left, base]{\color{textcolor}\rmfamily\fontsize{10.000000}{12.000000}\selectfont 100}%
\end{pgfscope}%
\begin{pgfscope}%
\pgfsetbuttcap%
\pgfsetroundjoin%
\definecolor{currentfill}{rgb}{0.000000,0.000000,0.000000}%
\pgfsetfillcolor{currentfill}%
\pgfsetlinewidth{0.803000pt}%
\definecolor{currentstroke}{rgb}{0.000000,0.000000,0.000000}%
\pgfsetstrokecolor{currentstroke}%
\pgfsetdash{}{0pt}%
\pgfsys@defobject{currentmarker}{\pgfqpoint{0.000000in}{0.000000in}}{\pgfqpoint{0.048611in}{0.000000in}}{%
\pgfpathmoveto{\pgfqpoint{0.000000in}{0.000000in}}%
\pgfpathlineto{\pgfqpoint{0.048611in}{0.000000in}}%
\pgfusepath{stroke,fill}%
}%
\begin{pgfscope}%
\pgfsys@transformshift{5.515140in}{5.707208in}%
\pgfsys@useobject{currentmarker}{}%
\end{pgfscope}%
\end{pgfscope}%
\begin{pgfscope}%
\definecolor{textcolor}{rgb}{0.000000,0.000000,0.000000}%
\pgfsetstrokecolor{textcolor}%
\pgfsetfillcolor{textcolor}%
\pgftext[x=5.612362in, y=5.659380in, left, base]{\color{textcolor}\rmfamily\fontsize{10.000000}{12.000000}\selectfont 1000}%
\end{pgfscope}%
\begin{pgfscope}%
\pgfsetbuttcap%
\pgfsetroundjoin%
\definecolor{currentfill}{rgb}{0.000000,0.000000,0.000000}%
\pgfsetfillcolor{currentfill}%
\pgfsetlinewidth{0.602250pt}%
\definecolor{currentstroke}{rgb}{0.000000,0.000000,0.000000}%
\pgfsetstrokecolor{currentstroke}%
\pgfsetdash{}{0pt}%
\pgfsys@defobject{currentmarker}{\pgfqpoint{0.000000in}{0.000000in}}{\pgfqpoint{0.027778in}{0.000000in}}{%
\pgfpathmoveto{\pgfqpoint{0.000000in}{0.000000in}}%
\pgfpathlineto{\pgfqpoint{0.027778in}{0.000000in}}%
\pgfusepath{stroke,fill}%
}%
\begin{pgfscope}%
\pgfsys@transformshift{5.515140in}{3.682426in}%
\pgfsys@useobject{currentmarker}{}%
\end{pgfscope}%
\end{pgfscope}%
\begin{pgfscope}%
\pgfsetbuttcap%
\pgfsetroundjoin%
\definecolor{currentfill}{rgb}{0.000000,0.000000,0.000000}%
\pgfsetfillcolor{currentfill}%
\pgfsetlinewidth{0.602250pt}%
\definecolor{currentstroke}{rgb}{0.000000,0.000000,0.000000}%
\pgfsetstrokecolor{currentstroke}%
\pgfsetdash{}{0pt}%
\pgfsys@defobject{currentmarker}{\pgfqpoint{0.000000in}{0.000000in}}{\pgfqpoint{0.027778in}{0.000000in}}{%
\pgfpathmoveto{\pgfqpoint{0.000000in}{0.000000in}}%
\pgfpathlineto{\pgfqpoint{0.027778in}{0.000000in}}%
\pgfusepath{stroke,fill}%
}%
\begin{pgfscope}%
\pgfsys@transformshift{5.515140in}{3.814531in}%
\pgfsys@useobject{currentmarker}{}%
\end{pgfscope}%
\end{pgfscope}%
\begin{pgfscope}%
\pgfsetbuttcap%
\pgfsetroundjoin%
\definecolor{currentfill}{rgb}{0.000000,0.000000,0.000000}%
\pgfsetfillcolor{currentfill}%
\pgfsetlinewidth{0.602250pt}%
\definecolor{currentstroke}{rgb}{0.000000,0.000000,0.000000}%
\pgfsetstrokecolor{currentstroke}%
\pgfsetdash{}{0pt}%
\pgfsys@defobject{currentmarker}{\pgfqpoint{0.000000in}{0.000000in}}{\pgfqpoint{0.027778in}{0.000000in}}{%
\pgfpathmoveto{\pgfqpoint{0.000000in}{0.000000in}}%
\pgfpathlineto{\pgfqpoint{0.027778in}{0.000000in}}%
\pgfusepath{stroke,fill}%
}%
\begin{pgfscope}%
\pgfsys@transformshift{5.515140in}{3.908260in}%
\pgfsys@useobject{currentmarker}{}%
\end{pgfscope}%
\end{pgfscope}%
\begin{pgfscope}%
\pgfsetbuttcap%
\pgfsetroundjoin%
\definecolor{currentfill}{rgb}{0.000000,0.000000,0.000000}%
\pgfsetfillcolor{currentfill}%
\pgfsetlinewidth{0.602250pt}%
\definecolor{currentstroke}{rgb}{0.000000,0.000000,0.000000}%
\pgfsetstrokecolor{currentstroke}%
\pgfsetdash{}{0pt}%
\pgfsys@defobject{currentmarker}{\pgfqpoint{0.000000in}{0.000000in}}{\pgfqpoint{0.027778in}{0.000000in}}{%
\pgfpathmoveto{\pgfqpoint{0.000000in}{0.000000in}}%
\pgfpathlineto{\pgfqpoint{0.027778in}{0.000000in}}%
\pgfusepath{stroke,fill}%
}%
\begin{pgfscope}%
\pgfsys@transformshift{5.515140in}{3.980963in}%
\pgfsys@useobject{currentmarker}{}%
\end{pgfscope}%
\end{pgfscope}%
\begin{pgfscope}%
\pgfsetbuttcap%
\pgfsetroundjoin%
\definecolor{currentfill}{rgb}{0.000000,0.000000,0.000000}%
\pgfsetfillcolor{currentfill}%
\pgfsetlinewidth{0.602250pt}%
\definecolor{currentstroke}{rgb}{0.000000,0.000000,0.000000}%
\pgfsetstrokecolor{currentstroke}%
\pgfsetdash{}{0pt}%
\pgfsys@defobject{currentmarker}{\pgfqpoint{0.000000in}{0.000000in}}{\pgfqpoint{0.027778in}{0.000000in}}{%
\pgfpathmoveto{\pgfqpoint{0.000000in}{0.000000in}}%
\pgfpathlineto{\pgfqpoint{0.027778in}{0.000000in}}%
\pgfusepath{stroke,fill}%
}%
\begin{pgfscope}%
\pgfsys@transformshift{5.515140in}{4.040365in}%
\pgfsys@useobject{currentmarker}{}%
\end{pgfscope}%
\end{pgfscope}%
\begin{pgfscope}%
\pgfsetbuttcap%
\pgfsetroundjoin%
\definecolor{currentfill}{rgb}{0.000000,0.000000,0.000000}%
\pgfsetfillcolor{currentfill}%
\pgfsetlinewidth{0.602250pt}%
\definecolor{currentstroke}{rgb}{0.000000,0.000000,0.000000}%
\pgfsetstrokecolor{currentstroke}%
\pgfsetdash{}{0pt}%
\pgfsys@defobject{currentmarker}{\pgfqpoint{0.000000in}{0.000000in}}{\pgfqpoint{0.027778in}{0.000000in}}{%
\pgfpathmoveto{\pgfqpoint{0.000000in}{0.000000in}}%
\pgfpathlineto{\pgfqpoint{0.027778in}{0.000000in}}%
\pgfusepath{stroke,fill}%
}%
\begin{pgfscope}%
\pgfsys@transformshift{5.515140in}{4.090589in}%
\pgfsys@useobject{currentmarker}{}%
\end{pgfscope}%
\end{pgfscope}%
\begin{pgfscope}%
\pgfsetbuttcap%
\pgfsetroundjoin%
\definecolor{currentfill}{rgb}{0.000000,0.000000,0.000000}%
\pgfsetfillcolor{currentfill}%
\pgfsetlinewidth{0.602250pt}%
\definecolor{currentstroke}{rgb}{0.000000,0.000000,0.000000}%
\pgfsetstrokecolor{currentstroke}%
\pgfsetdash{}{0pt}%
\pgfsys@defobject{currentmarker}{\pgfqpoint{0.000000in}{0.000000in}}{\pgfqpoint{0.027778in}{0.000000in}}{%
\pgfpathmoveto{\pgfqpoint{0.000000in}{0.000000in}}%
\pgfpathlineto{\pgfqpoint{0.027778in}{0.000000in}}%
\pgfusepath{stroke,fill}%
}%
\begin{pgfscope}%
\pgfsys@transformshift{5.515140in}{4.134095in}%
\pgfsys@useobject{currentmarker}{}%
\end{pgfscope}%
\end{pgfscope}%
\begin{pgfscope}%
\pgfsetbuttcap%
\pgfsetroundjoin%
\definecolor{currentfill}{rgb}{0.000000,0.000000,0.000000}%
\pgfsetfillcolor{currentfill}%
\pgfsetlinewidth{0.602250pt}%
\definecolor{currentstroke}{rgb}{0.000000,0.000000,0.000000}%
\pgfsetstrokecolor{currentstroke}%
\pgfsetdash{}{0pt}%
\pgfsys@defobject{currentmarker}{\pgfqpoint{0.000000in}{0.000000in}}{\pgfqpoint{0.027778in}{0.000000in}}{%
\pgfpathmoveto{\pgfqpoint{0.000000in}{0.000000in}}%
\pgfpathlineto{\pgfqpoint{0.027778in}{0.000000in}}%
\pgfusepath{stroke,fill}%
}%
\begin{pgfscope}%
\pgfsys@transformshift{5.515140in}{4.172470in}%
\pgfsys@useobject{currentmarker}{}%
\end{pgfscope}%
\end{pgfscope}%
\begin{pgfscope}%
\pgfsetbuttcap%
\pgfsetroundjoin%
\definecolor{currentfill}{rgb}{0.000000,0.000000,0.000000}%
\pgfsetfillcolor{currentfill}%
\pgfsetlinewidth{0.602250pt}%
\definecolor{currentstroke}{rgb}{0.000000,0.000000,0.000000}%
\pgfsetstrokecolor{currentstroke}%
\pgfsetdash{}{0pt}%
\pgfsys@defobject{currentmarker}{\pgfqpoint{0.000000in}{0.000000in}}{\pgfqpoint{0.027778in}{0.000000in}}{%
\pgfpathmoveto{\pgfqpoint{0.000000in}{0.000000in}}%
\pgfpathlineto{\pgfqpoint{0.027778in}{0.000000in}}%
\pgfusepath{stroke,fill}%
}%
\begin{pgfscope}%
\pgfsys@transformshift{5.515140in}{4.432632in}%
\pgfsys@useobject{currentmarker}{}%
\end{pgfscope}%
\end{pgfscope}%
\begin{pgfscope}%
\pgfsetbuttcap%
\pgfsetroundjoin%
\definecolor{currentfill}{rgb}{0.000000,0.000000,0.000000}%
\pgfsetfillcolor{currentfill}%
\pgfsetlinewidth{0.602250pt}%
\definecolor{currentstroke}{rgb}{0.000000,0.000000,0.000000}%
\pgfsetstrokecolor{currentstroke}%
\pgfsetdash{}{0pt}%
\pgfsys@defobject{currentmarker}{\pgfqpoint{0.000000in}{0.000000in}}{\pgfqpoint{0.027778in}{0.000000in}}{%
\pgfpathmoveto{\pgfqpoint{0.000000in}{0.000000in}}%
\pgfpathlineto{\pgfqpoint{0.027778in}{0.000000in}}%
\pgfusepath{stroke,fill}%
}%
\begin{pgfscope}%
\pgfsys@transformshift{5.515140in}{4.564736in}%
\pgfsys@useobject{currentmarker}{}%
\end{pgfscope}%
\end{pgfscope}%
\begin{pgfscope}%
\pgfsetbuttcap%
\pgfsetroundjoin%
\definecolor{currentfill}{rgb}{0.000000,0.000000,0.000000}%
\pgfsetfillcolor{currentfill}%
\pgfsetlinewidth{0.602250pt}%
\definecolor{currentstroke}{rgb}{0.000000,0.000000,0.000000}%
\pgfsetstrokecolor{currentstroke}%
\pgfsetdash{}{0pt}%
\pgfsys@defobject{currentmarker}{\pgfqpoint{0.000000in}{0.000000in}}{\pgfqpoint{0.027778in}{0.000000in}}{%
\pgfpathmoveto{\pgfqpoint{0.000000in}{0.000000in}}%
\pgfpathlineto{\pgfqpoint{0.027778in}{0.000000in}}%
\pgfusepath{stroke,fill}%
}%
\begin{pgfscope}%
\pgfsys@transformshift{5.515140in}{4.658466in}%
\pgfsys@useobject{currentmarker}{}%
\end{pgfscope}%
\end{pgfscope}%
\begin{pgfscope}%
\pgfsetbuttcap%
\pgfsetroundjoin%
\definecolor{currentfill}{rgb}{0.000000,0.000000,0.000000}%
\pgfsetfillcolor{currentfill}%
\pgfsetlinewidth{0.602250pt}%
\definecolor{currentstroke}{rgb}{0.000000,0.000000,0.000000}%
\pgfsetstrokecolor{currentstroke}%
\pgfsetdash{}{0pt}%
\pgfsys@defobject{currentmarker}{\pgfqpoint{0.000000in}{0.000000in}}{\pgfqpoint{0.027778in}{0.000000in}}{%
\pgfpathmoveto{\pgfqpoint{0.000000in}{0.000000in}}%
\pgfpathlineto{\pgfqpoint{0.027778in}{0.000000in}}%
\pgfusepath{stroke,fill}%
}%
\begin{pgfscope}%
\pgfsys@transformshift{5.515140in}{4.731168in}%
\pgfsys@useobject{currentmarker}{}%
\end{pgfscope}%
\end{pgfscope}%
\begin{pgfscope}%
\pgfsetbuttcap%
\pgfsetroundjoin%
\definecolor{currentfill}{rgb}{0.000000,0.000000,0.000000}%
\pgfsetfillcolor{currentfill}%
\pgfsetlinewidth{0.602250pt}%
\definecolor{currentstroke}{rgb}{0.000000,0.000000,0.000000}%
\pgfsetstrokecolor{currentstroke}%
\pgfsetdash{}{0pt}%
\pgfsys@defobject{currentmarker}{\pgfqpoint{0.000000in}{0.000000in}}{\pgfqpoint{0.027778in}{0.000000in}}{%
\pgfpathmoveto{\pgfqpoint{0.000000in}{0.000000in}}%
\pgfpathlineto{\pgfqpoint{0.027778in}{0.000000in}}%
\pgfusepath{stroke,fill}%
}%
\begin{pgfscope}%
\pgfsys@transformshift{5.515140in}{4.790571in}%
\pgfsys@useobject{currentmarker}{}%
\end{pgfscope}%
\end{pgfscope}%
\begin{pgfscope}%
\pgfsetbuttcap%
\pgfsetroundjoin%
\definecolor{currentfill}{rgb}{0.000000,0.000000,0.000000}%
\pgfsetfillcolor{currentfill}%
\pgfsetlinewidth{0.602250pt}%
\definecolor{currentstroke}{rgb}{0.000000,0.000000,0.000000}%
\pgfsetstrokecolor{currentstroke}%
\pgfsetdash{}{0pt}%
\pgfsys@defobject{currentmarker}{\pgfqpoint{0.000000in}{0.000000in}}{\pgfqpoint{0.027778in}{0.000000in}}{%
\pgfpathmoveto{\pgfqpoint{0.000000in}{0.000000in}}%
\pgfpathlineto{\pgfqpoint{0.027778in}{0.000000in}}%
\pgfusepath{stroke,fill}%
}%
\begin{pgfscope}%
\pgfsys@transformshift{5.515140in}{4.840794in}%
\pgfsys@useobject{currentmarker}{}%
\end{pgfscope}%
\end{pgfscope}%
\begin{pgfscope}%
\pgfsetbuttcap%
\pgfsetroundjoin%
\definecolor{currentfill}{rgb}{0.000000,0.000000,0.000000}%
\pgfsetfillcolor{currentfill}%
\pgfsetlinewidth{0.602250pt}%
\definecolor{currentstroke}{rgb}{0.000000,0.000000,0.000000}%
\pgfsetstrokecolor{currentstroke}%
\pgfsetdash{}{0pt}%
\pgfsys@defobject{currentmarker}{\pgfqpoint{0.000000in}{0.000000in}}{\pgfqpoint{0.027778in}{0.000000in}}{%
\pgfpathmoveto{\pgfqpoint{0.000000in}{0.000000in}}%
\pgfpathlineto{\pgfqpoint{0.027778in}{0.000000in}}%
\pgfusepath{stroke,fill}%
}%
\begin{pgfscope}%
\pgfsys@transformshift{5.515140in}{4.884300in}%
\pgfsys@useobject{currentmarker}{}%
\end{pgfscope}%
\end{pgfscope}%
\begin{pgfscope}%
\pgfsetbuttcap%
\pgfsetroundjoin%
\definecolor{currentfill}{rgb}{0.000000,0.000000,0.000000}%
\pgfsetfillcolor{currentfill}%
\pgfsetlinewidth{0.602250pt}%
\definecolor{currentstroke}{rgb}{0.000000,0.000000,0.000000}%
\pgfsetstrokecolor{currentstroke}%
\pgfsetdash{}{0pt}%
\pgfsys@defobject{currentmarker}{\pgfqpoint{0.000000in}{0.000000in}}{\pgfqpoint{0.027778in}{0.000000in}}{%
\pgfpathmoveto{\pgfqpoint{0.000000in}{0.000000in}}%
\pgfpathlineto{\pgfqpoint{0.027778in}{0.000000in}}%
\pgfusepath{stroke,fill}%
}%
\begin{pgfscope}%
\pgfsys@transformshift{5.515140in}{4.922675in}%
\pgfsys@useobject{currentmarker}{}%
\end{pgfscope}%
\end{pgfscope}%
\begin{pgfscope}%
\pgfsetbuttcap%
\pgfsetroundjoin%
\definecolor{currentfill}{rgb}{0.000000,0.000000,0.000000}%
\pgfsetfillcolor{currentfill}%
\pgfsetlinewidth{0.602250pt}%
\definecolor{currentstroke}{rgb}{0.000000,0.000000,0.000000}%
\pgfsetstrokecolor{currentstroke}%
\pgfsetdash{}{0pt}%
\pgfsys@defobject{currentmarker}{\pgfqpoint{0.000000in}{0.000000in}}{\pgfqpoint{0.027778in}{0.000000in}}{%
\pgfpathmoveto{\pgfqpoint{0.000000in}{0.000000in}}%
\pgfpathlineto{\pgfqpoint{0.027778in}{0.000000in}}%
\pgfusepath{stroke,fill}%
}%
\begin{pgfscope}%
\pgfsys@transformshift{5.515140in}{5.182837in}%
\pgfsys@useobject{currentmarker}{}%
\end{pgfscope}%
\end{pgfscope}%
\begin{pgfscope}%
\pgfsetbuttcap%
\pgfsetroundjoin%
\definecolor{currentfill}{rgb}{0.000000,0.000000,0.000000}%
\pgfsetfillcolor{currentfill}%
\pgfsetlinewidth{0.602250pt}%
\definecolor{currentstroke}{rgb}{0.000000,0.000000,0.000000}%
\pgfsetstrokecolor{currentstroke}%
\pgfsetdash{}{0pt}%
\pgfsys@defobject{currentmarker}{\pgfqpoint{0.000000in}{0.000000in}}{\pgfqpoint{0.027778in}{0.000000in}}{%
\pgfpathmoveto{\pgfqpoint{0.000000in}{0.000000in}}%
\pgfpathlineto{\pgfqpoint{0.027778in}{0.000000in}}%
\pgfusepath{stroke,fill}%
}%
\begin{pgfscope}%
\pgfsys@transformshift{5.515140in}{5.314942in}%
\pgfsys@useobject{currentmarker}{}%
\end{pgfscope}%
\end{pgfscope}%
\begin{pgfscope}%
\pgfsetbuttcap%
\pgfsetroundjoin%
\definecolor{currentfill}{rgb}{0.000000,0.000000,0.000000}%
\pgfsetfillcolor{currentfill}%
\pgfsetlinewidth{0.602250pt}%
\definecolor{currentstroke}{rgb}{0.000000,0.000000,0.000000}%
\pgfsetstrokecolor{currentstroke}%
\pgfsetdash{}{0pt}%
\pgfsys@defobject{currentmarker}{\pgfqpoint{0.000000in}{0.000000in}}{\pgfqpoint{0.027778in}{0.000000in}}{%
\pgfpathmoveto{\pgfqpoint{0.000000in}{0.000000in}}%
\pgfpathlineto{\pgfqpoint{0.027778in}{0.000000in}}%
\pgfusepath{stroke,fill}%
}%
\begin{pgfscope}%
\pgfsys@transformshift{5.515140in}{5.408671in}%
\pgfsys@useobject{currentmarker}{}%
\end{pgfscope}%
\end{pgfscope}%
\begin{pgfscope}%
\pgfsetbuttcap%
\pgfsetroundjoin%
\definecolor{currentfill}{rgb}{0.000000,0.000000,0.000000}%
\pgfsetfillcolor{currentfill}%
\pgfsetlinewidth{0.602250pt}%
\definecolor{currentstroke}{rgb}{0.000000,0.000000,0.000000}%
\pgfsetstrokecolor{currentstroke}%
\pgfsetdash{}{0pt}%
\pgfsys@defobject{currentmarker}{\pgfqpoint{0.000000in}{0.000000in}}{\pgfqpoint{0.027778in}{0.000000in}}{%
\pgfpathmoveto{\pgfqpoint{0.000000in}{0.000000in}}%
\pgfpathlineto{\pgfqpoint{0.027778in}{0.000000in}}%
\pgfusepath{stroke,fill}%
}%
\begin{pgfscope}%
\pgfsys@transformshift{5.515140in}{5.481374in}%
\pgfsys@useobject{currentmarker}{}%
\end{pgfscope}%
\end{pgfscope}%
\begin{pgfscope}%
\pgfsetbuttcap%
\pgfsetroundjoin%
\definecolor{currentfill}{rgb}{0.000000,0.000000,0.000000}%
\pgfsetfillcolor{currentfill}%
\pgfsetlinewidth{0.602250pt}%
\definecolor{currentstroke}{rgb}{0.000000,0.000000,0.000000}%
\pgfsetstrokecolor{currentstroke}%
\pgfsetdash{}{0pt}%
\pgfsys@defobject{currentmarker}{\pgfqpoint{0.000000in}{0.000000in}}{\pgfqpoint{0.027778in}{0.000000in}}{%
\pgfpathmoveto{\pgfqpoint{0.000000in}{0.000000in}}%
\pgfpathlineto{\pgfqpoint{0.027778in}{0.000000in}}%
\pgfusepath{stroke,fill}%
}%
\begin{pgfscope}%
\pgfsys@transformshift{5.515140in}{5.540776in}%
\pgfsys@useobject{currentmarker}{}%
\end{pgfscope}%
\end{pgfscope}%
\begin{pgfscope}%
\pgfsetbuttcap%
\pgfsetroundjoin%
\definecolor{currentfill}{rgb}{0.000000,0.000000,0.000000}%
\pgfsetfillcolor{currentfill}%
\pgfsetlinewidth{0.602250pt}%
\definecolor{currentstroke}{rgb}{0.000000,0.000000,0.000000}%
\pgfsetstrokecolor{currentstroke}%
\pgfsetdash{}{0pt}%
\pgfsys@defobject{currentmarker}{\pgfqpoint{0.000000in}{0.000000in}}{\pgfqpoint{0.027778in}{0.000000in}}{%
\pgfpathmoveto{\pgfqpoint{0.000000in}{0.000000in}}%
\pgfpathlineto{\pgfqpoint{0.027778in}{0.000000in}}%
\pgfusepath{stroke,fill}%
}%
\begin{pgfscope}%
\pgfsys@transformshift{5.515140in}{5.591000in}%
\pgfsys@useobject{currentmarker}{}%
\end{pgfscope}%
\end{pgfscope}%
\begin{pgfscope}%
\pgfsetbuttcap%
\pgfsetroundjoin%
\definecolor{currentfill}{rgb}{0.000000,0.000000,0.000000}%
\pgfsetfillcolor{currentfill}%
\pgfsetlinewidth{0.602250pt}%
\definecolor{currentstroke}{rgb}{0.000000,0.000000,0.000000}%
\pgfsetstrokecolor{currentstroke}%
\pgfsetdash{}{0pt}%
\pgfsys@defobject{currentmarker}{\pgfqpoint{0.000000in}{0.000000in}}{\pgfqpoint{0.027778in}{0.000000in}}{%
\pgfpathmoveto{\pgfqpoint{0.000000in}{0.000000in}}%
\pgfpathlineto{\pgfqpoint{0.027778in}{0.000000in}}%
\pgfusepath{stroke,fill}%
}%
\begin{pgfscope}%
\pgfsys@transformshift{5.515140in}{5.634506in}%
\pgfsys@useobject{currentmarker}{}%
\end{pgfscope}%
\end{pgfscope}%
\begin{pgfscope}%
\pgfsetbuttcap%
\pgfsetroundjoin%
\definecolor{currentfill}{rgb}{0.000000,0.000000,0.000000}%
\pgfsetfillcolor{currentfill}%
\pgfsetlinewidth{0.602250pt}%
\definecolor{currentstroke}{rgb}{0.000000,0.000000,0.000000}%
\pgfsetstrokecolor{currentstroke}%
\pgfsetdash{}{0pt}%
\pgfsys@defobject{currentmarker}{\pgfqpoint{0.000000in}{0.000000in}}{\pgfqpoint{0.027778in}{0.000000in}}{%
\pgfpathmoveto{\pgfqpoint{0.000000in}{0.000000in}}%
\pgfpathlineto{\pgfqpoint{0.027778in}{0.000000in}}%
\pgfusepath{stroke,fill}%
}%
\begin{pgfscope}%
\pgfsys@transformshift{5.515140in}{5.672881in}%
\pgfsys@useobject{currentmarker}{}%
\end{pgfscope}%
\end{pgfscope}%
\begin{pgfscope}%
\definecolor{textcolor}{rgb}{0.000000,0.000000,0.000000}%
\pgfsetstrokecolor{textcolor}%
\pgfsetfillcolor{textcolor}%
\pgftext[x=5.945628in,y=4.581900in,,top,rotate=90.000000]{\color{textcolor}\rmfamily\fontsize{10.000000}{12.000000}\selectfont Photonen}%
\end{pgfscope}%
\begin{pgfscope}%
\pgfsetrectcap%
\pgfsetmiterjoin%
\pgfsetlinewidth{0.803000pt}%
\definecolor{currentstroke}{rgb}{0.000000,0.000000,0.000000}%
\pgfsetstrokecolor{currentstroke}%
\pgfsetdash{}{0pt}%
\pgfpathmoveto{\pgfqpoint{5.402609in}{3.456592in}}%
\pgfpathlineto{\pgfqpoint{5.458875in}{3.456592in}}%
\pgfpathlineto{\pgfqpoint{5.515140in}{3.456592in}}%
\pgfpathlineto{\pgfqpoint{5.515140in}{5.707208in}}%
\pgfpathlineto{\pgfqpoint{5.458875in}{5.707208in}}%
\pgfpathlineto{\pgfqpoint{5.402609in}{5.707208in}}%
\pgfpathlineto{\pgfqpoint{5.402609in}{3.456592in}}%
\pgfpathclose%
\pgfusepath{stroke}%
\end{pgfscope}%
\begin{pgfscope}%
\pgfsetbuttcap%
\pgfsetmiterjoin%
\pgfsetlinewidth{0.000000pt}%
\definecolor{currentstroke}{rgb}{1.000000,1.000000,1.000000}%
\pgfsetstrokecolor{currentstroke}%
\pgfsetstrokeopacity{0.000000}%
\pgfsetdash{}{0pt}%
\pgfpathmoveto{\pgfqpoint{-0.243853in}{0.081900in}}%
\pgfpathlineto{\pgfqpoint{6.236147in}{0.081900in}}%
\pgfpathlineto{\pgfqpoint{6.236147in}{3.081900in}}%
\pgfpathlineto{\pgfqpoint{-0.243853in}{3.081900in}}%
\pgfpathlineto{\pgfqpoint{-0.243853in}{0.081900in}}%
\pgfpathclose%
\pgfusepath{}%
\end{pgfscope}%
\begin{pgfscope}%
\pgfsetbuttcap%
\pgfsetmiterjoin%
\definecolor{currentfill}{rgb}{1.000000,1.000000,1.000000}%
\pgfsetfillcolor{currentfill}%
\pgfsetlinewidth{0.000000pt}%
\definecolor{currentstroke}{rgb}{0.000000,0.000000,0.000000}%
\pgfsetstrokecolor{currentstroke}%
\pgfsetstrokeopacity{0.000000}%
\pgfsetdash{}{0pt}%
\pgfpathmoveto{\pgfqpoint{0.501376in}{0.398095in}}%
\pgfpathlineto{\pgfqpoint{5.515140in}{0.398095in}}%
\pgfpathlineto{\pgfqpoint{5.515140in}{2.765705in}}%
\pgfpathlineto{\pgfqpoint{0.501376in}{2.765705in}}%
\pgfpathlineto{\pgfqpoint{0.501376in}{0.398095in}}%
\pgfpathclose%
\pgfusepath{fill}%
\end{pgfscope}%
\begin{pgfscope}%
\pgfpathrectangle{\pgfqpoint{0.501376in}{0.398095in}}{\pgfqpoint{5.013764in}{2.367611in}}%
\pgfusepath{clip}%
\pgfsetbuttcap%
\pgfsetmiterjoin%
\definecolor{currentfill}{rgb}{0.000000,1.000000,1.000000}%
\pgfsetfillcolor{currentfill}%
\pgfsetfillopacity{0.439216}%
\pgfsetlinewidth{0.000000pt}%
\definecolor{currentstroke}{rgb}{0.000000,0.000000,0.000000}%
\pgfsetstrokecolor{currentstroke}%
\pgfsetstrokeopacity{0.000000}%
\pgfsetdash{}{0pt}%
\pgfpathmoveto{\pgfqpoint{3.634979in}{0.398095in}}%
\pgfpathlineto{\pgfqpoint{3.634979in}{2.765705in}}%
\pgfpathlineto{\pgfqpoint{4.888420in}{2.765705in}}%
\pgfpathlineto{\pgfqpoint{4.888420in}{0.398095in}}%
\pgfpathlineto{\pgfqpoint{3.634979in}{0.398095in}}%
\pgfpathclose%
\pgfusepath{fill}%
\end{pgfscope}%
\begin{pgfscope}%
\pgfpathrectangle{\pgfqpoint{0.501376in}{0.398095in}}{\pgfqpoint{5.013764in}{2.367611in}}%
\pgfusepath{clip}%
\pgfsetbuttcap%
\pgfsetmiterjoin%
\definecolor{currentfill}{rgb}{1.000000,0.411765,0.705882}%
\pgfsetfillcolor{currentfill}%
\pgfsetfillopacity{0.439216}%
\pgfsetlinewidth{0.000000pt}%
\definecolor{currentstroke}{rgb}{0.000000,0.000000,0.000000}%
\pgfsetstrokecolor{currentstroke}%
\pgfsetstrokeopacity{0.000000}%
\pgfsetdash{}{0pt}%
\pgfpathmoveto{\pgfqpoint{4.888420in}{0.398095in}}%
\pgfpathlineto{\pgfqpoint{4.888420in}{2.765705in}}%
\pgfpathlineto{\pgfqpoint{5.515140in}{2.765705in}}%
\pgfpathlineto{\pgfqpoint{5.515140in}{0.398095in}}%
\pgfpathlineto{\pgfqpoint{4.888420in}{0.398095in}}%
\pgfpathclose%
\pgfusepath{fill}%
\end{pgfscope}%
\begin{pgfscope}%
\pgfpathrectangle{\pgfqpoint{0.501376in}{0.398095in}}{\pgfqpoint{5.013764in}{2.367611in}}%
\pgfusepath{clip}%
\pgfsetbuttcap%
\pgfsetmiterjoin%
\definecolor{currentfill}{rgb}{1.000000,0.411765,0.705882}%
\pgfsetfillcolor{currentfill}%
\pgfsetfillopacity{0.439216}%
\pgfsetlinewidth{0.000000pt}%
\definecolor{currentstroke}{rgb}{0.000000,0.000000,0.000000}%
\pgfsetstrokecolor{currentstroke}%
\pgfsetstrokeopacity{0.000000}%
\pgfsetdash{}{0pt}%
\pgfpathmoveto{\pgfqpoint{0.501376in}{0.398095in}}%
\pgfpathlineto{\pgfqpoint{0.501376in}{2.765705in}}%
\pgfpathlineto{\pgfqpoint{1.128097in}{2.765705in}}%
\pgfpathlineto{\pgfqpoint{1.128097in}{0.398095in}}%
\pgfpathlineto{\pgfqpoint{0.501376in}{0.398095in}}%
\pgfpathclose%
\pgfusepath{fill}%
\end{pgfscope}%
\begin{pgfscope}%
\pgfpathrectangle{\pgfqpoint{0.501376in}{0.398095in}}{\pgfqpoint{5.013764in}{2.367611in}}%
\pgfusepath{clip}%
\pgfsetbuttcap%
\pgfsetmiterjoin%
\definecolor{currentfill}{rgb}{0.000000,1.000000,1.000000}%
\pgfsetfillcolor{currentfill}%
\pgfsetfillopacity{0.439216}%
\pgfsetlinewidth{0.000000pt}%
\definecolor{currentstroke}{rgb}{0.000000,0.000000,0.000000}%
\pgfsetstrokecolor{currentstroke}%
\pgfsetstrokeopacity{0.000000}%
\pgfsetdash{}{0pt}%
\pgfpathmoveto{\pgfqpoint{1.128097in}{0.398095in}}%
\pgfpathlineto{\pgfqpoint{1.128097in}{2.765705in}}%
\pgfpathlineto{\pgfqpoint{2.381538in}{2.765705in}}%
\pgfpathlineto{\pgfqpoint{2.381538in}{0.398095in}}%
\pgfpathlineto{\pgfqpoint{1.128097in}{0.398095in}}%
\pgfpathclose%
\pgfusepath{fill}%
\end{pgfscope}%
\begin{pgfscope}%
\pgfpathrectangle{\pgfqpoint{0.501376in}{0.398095in}}{\pgfqpoint{5.013764in}{2.367611in}}%
\pgfusepath{clip}%
\pgfsetbuttcap%
\pgfsetmiterjoin%
\definecolor{currentfill}{rgb}{1.000000,0.411765,0.705882}%
\pgfsetfillcolor{currentfill}%
\pgfsetfillopacity{0.439216}%
\pgfsetlinewidth{0.000000pt}%
\definecolor{currentstroke}{rgb}{0.000000,0.000000,0.000000}%
\pgfsetstrokecolor{currentstroke}%
\pgfsetstrokeopacity{0.000000}%
\pgfsetdash{}{0pt}%
\pgfpathmoveto{\pgfqpoint{2.381538in}{0.398095in}}%
\pgfpathlineto{\pgfqpoint{2.381538in}{2.765705in}}%
\pgfpathlineto{\pgfqpoint{3.634979in}{2.765705in}}%
\pgfpathlineto{\pgfqpoint{3.634979in}{0.398095in}}%
\pgfpathlineto{\pgfqpoint{2.381538in}{0.398095in}}%
\pgfpathclose%
\pgfusepath{fill}%
\end{pgfscope}%
\begin{pgfscope}%
\pgfsys@transformshift{0.502000in}{0.942411in}%
\pgftext[left,bottom]{\includegraphics[interpolate=true,width=5.014000in,height=1.824000in]{th_100_200_masked_radial_transform-img3.png}}%
\end{pgfscope}%
\begin{pgfscope}%
\pgfsetbuttcap%
\pgfsetroundjoin%
\definecolor{currentfill}{rgb}{0.000000,0.000000,0.000000}%
\pgfsetfillcolor{currentfill}%
\pgfsetlinewidth{0.803000pt}%
\definecolor{currentstroke}{rgb}{0.000000,0.000000,0.000000}%
\pgfsetstrokecolor{currentstroke}%
\pgfsetdash{}{0pt}%
\pgfsys@defobject{currentmarker}{\pgfqpoint{0.000000in}{-0.048611in}}{\pgfqpoint{0.000000in}{0.000000in}}{%
\pgfpathmoveto{\pgfqpoint{0.000000in}{0.000000in}}%
\pgfpathlineto{\pgfqpoint{0.000000in}{-0.048611in}}%
\pgfusepath{stroke,fill}%
}%
\begin{pgfscope}%
\pgfsys@transformshift{0.501376in}{0.398095in}%
\pgfsys@useobject{currentmarker}{}%
\end{pgfscope}%
\end{pgfscope}%
\begin{pgfscope}%
\definecolor{textcolor}{rgb}{0.000000,0.000000,0.000000}%
\pgfsetstrokecolor{textcolor}%
\pgfsetfillcolor{textcolor}%
\pgftext[x=0.501376in,y=0.300872in,,top]{\color{textcolor}\rmfamily\fontsize{10.000000}{12.000000}\selectfont \SI{0}{\degree}}%
\end{pgfscope}%
\begin{pgfscope}%
\pgfsetbuttcap%
\pgfsetroundjoin%
\definecolor{currentfill}{rgb}{0.000000,0.000000,0.000000}%
\pgfsetfillcolor{currentfill}%
\pgfsetlinewidth{0.803000pt}%
\definecolor{currentstroke}{rgb}{0.000000,0.000000,0.000000}%
\pgfsetstrokecolor{currentstroke}%
\pgfsetdash{}{0pt}%
\pgfsys@defobject{currentmarker}{\pgfqpoint{0.000000in}{-0.048611in}}{\pgfqpoint{0.000000in}{0.000000in}}{%
\pgfpathmoveto{\pgfqpoint{0.000000in}{0.000000in}}%
\pgfpathlineto{\pgfqpoint{0.000000in}{-0.048611in}}%
\pgfusepath{stroke,fill}%
}%
\begin{pgfscope}%
\pgfsys@transformshift{1.128097in}{0.398095in}%
\pgfsys@useobject{currentmarker}{}%
\end{pgfscope}%
\end{pgfscope}%
\begin{pgfscope}%
\definecolor{textcolor}{rgb}{0.000000,0.000000,0.000000}%
\pgfsetstrokecolor{textcolor}%
\pgfsetfillcolor{textcolor}%
\pgftext[x=1.128097in,y=0.300872in,,top]{\color{textcolor}\rmfamily\fontsize{10.000000}{12.000000}\selectfont \SI{45}{\degree}}%
\end{pgfscope}%
\begin{pgfscope}%
\pgfsetbuttcap%
\pgfsetroundjoin%
\definecolor{currentfill}{rgb}{0.000000,0.000000,0.000000}%
\pgfsetfillcolor{currentfill}%
\pgfsetlinewidth{0.803000pt}%
\definecolor{currentstroke}{rgb}{0.000000,0.000000,0.000000}%
\pgfsetstrokecolor{currentstroke}%
\pgfsetdash{}{0pt}%
\pgfsys@defobject{currentmarker}{\pgfqpoint{0.000000in}{-0.048611in}}{\pgfqpoint{0.000000in}{0.000000in}}{%
\pgfpathmoveto{\pgfqpoint{0.000000in}{0.000000in}}%
\pgfpathlineto{\pgfqpoint{0.000000in}{-0.048611in}}%
\pgfusepath{stroke,fill}%
}%
\begin{pgfscope}%
\pgfsys@transformshift{1.754817in}{0.398095in}%
\pgfsys@useobject{currentmarker}{}%
\end{pgfscope}%
\end{pgfscope}%
\begin{pgfscope}%
\definecolor{textcolor}{rgb}{0.000000,0.000000,0.000000}%
\pgfsetstrokecolor{textcolor}%
\pgfsetfillcolor{textcolor}%
\pgftext[x=1.754817in,y=0.300872in,,top]{\color{textcolor}\rmfamily\fontsize{10.000000}{12.000000}\selectfont \SI{90}{\degree}}%
\end{pgfscope}%
\begin{pgfscope}%
\pgfsetbuttcap%
\pgfsetroundjoin%
\definecolor{currentfill}{rgb}{0.000000,0.000000,0.000000}%
\pgfsetfillcolor{currentfill}%
\pgfsetlinewidth{0.803000pt}%
\definecolor{currentstroke}{rgb}{0.000000,0.000000,0.000000}%
\pgfsetstrokecolor{currentstroke}%
\pgfsetdash{}{0pt}%
\pgfsys@defobject{currentmarker}{\pgfqpoint{0.000000in}{-0.048611in}}{\pgfqpoint{0.000000in}{0.000000in}}{%
\pgfpathmoveto{\pgfqpoint{0.000000in}{0.000000in}}%
\pgfpathlineto{\pgfqpoint{0.000000in}{-0.048611in}}%
\pgfusepath{stroke,fill}%
}%
\begin{pgfscope}%
\pgfsys@transformshift{2.381538in}{0.398095in}%
\pgfsys@useobject{currentmarker}{}%
\end{pgfscope}%
\end{pgfscope}%
\begin{pgfscope}%
\definecolor{textcolor}{rgb}{0.000000,0.000000,0.000000}%
\pgfsetstrokecolor{textcolor}%
\pgfsetfillcolor{textcolor}%
\pgftext[x=2.381538in,y=0.300872in,,top]{\color{textcolor}\rmfamily\fontsize{10.000000}{12.000000}\selectfont \SI{135}{\degree}}%
\end{pgfscope}%
\begin{pgfscope}%
\pgfsetbuttcap%
\pgfsetroundjoin%
\definecolor{currentfill}{rgb}{0.000000,0.000000,0.000000}%
\pgfsetfillcolor{currentfill}%
\pgfsetlinewidth{0.803000pt}%
\definecolor{currentstroke}{rgb}{0.000000,0.000000,0.000000}%
\pgfsetstrokecolor{currentstroke}%
\pgfsetdash{}{0pt}%
\pgfsys@defobject{currentmarker}{\pgfqpoint{0.000000in}{-0.048611in}}{\pgfqpoint{0.000000in}{0.000000in}}{%
\pgfpathmoveto{\pgfqpoint{0.000000in}{0.000000in}}%
\pgfpathlineto{\pgfqpoint{0.000000in}{-0.048611in}}%
\pgfusepath{stroke,fill}%
}%
\begin{pgfscope}%
\pgfsys@transformshift{3.008258in}{0.398095in}%
\pgfsys@useobject{currentmarker}{}%
\end{pgfscope}%
\end{pgfscope}%
\begin{pgfscope}%
\definecolor{textcolor}{rgb}{0.000000,0.000000,0.000000}%
\pgfsetstrokecolor{textcolor}%
\pgfsetfillcolor{textcolor}%
\pgftext[x=3.008258in,y=0.300872in,,top]{\color{textcolor}\rmfamily\fontsize{10.000000}{12.000000}\selectfont \SI{180}{\degree}}%
\end{pgfscope}%
\begin{pgfscope}%
\pgfsetbuttcap%
\pgfsetroundjoin%
\definecolor{currentfill}{rgb}{0.000000,0.000000,0.000000}%
\pgfsetfillcolor{currentfill}%
\pgfsetlinewidth{0.803000pt}%
\definecolor{currentstroke}{rgb}{0.000000,0.000000,0.000000}%
\pgfsetstrokecolor{currentstroke}%
\pgfsetdash{}{0pt}%
\pgfsys@defobject{currentmarker}{\pgfqpoint{0.000000in}{-0.048611in}}{\pgfqpoint{0.000000in}{0.000000in}}{%
\pgfpathmoveto{\pgfqpoint{0.000000in}{0.000000in}}%
\pgfpathlineto{\pgfqpoint{0.000000in}{-0.048611in}}%
\pgfusepath{stroke,fill}%
}%
\begin{pgfscope}%
\pgfsys@transformshift{3.634979in}{0.398095in}%
\pgfsys@useobject{currentmarker}{}%
\end{pgfscope}%
\end{pgfscope}%
\begin{pgfscope}%
\definecolor{textcolor}{rgb}{0.000000,0.000000,0.000000}%
\pgfsetstrokecolor{textcolor}%
\pgfsetfillcolor{textcolor}%
\pgftext[x=3.634979in,y=0.300872in,,top]{\color{textcolor}\rmfamily\fontsize{10.000000}{12.000000}\selectfont \SI{225}{\degree}}%
\end{pgfscope}%
\begin{pgfscope}%
\pgfsetbuttcap%
\pgfsetroundjoin%
\definecolor{currentfill}{rgb}{0.000000,0.000000,0.000000}%
\pgfsetfillcolor{currentfill}%
\pgfsetlinewidth{0.803000pt}%
\definecolor{currentstroke}{rgb}{0.000000,0.000000,0.000000}%
\pgfsetstrokecolor{currentstroke}%
\pgfsetdash{}{0pt}%
\pgfsys@defobject{currentmarker}{\pgfqpoint{0.000000in}{-0.048611in}}{\pgfqpoint{0.000000in}{0.000000in}}{%
\pgfpathmoveto{\pgfqpoint{0.000000in}{0.000000in}}%
\pgfpathlineto{\pgfqpoint{0.000000in}{-0.048611in}}%
\pgfusepath{stroke,fill}%
}%
\begin{pgfscope}%
\pgfsys@transformshift{4.261699in}{0.398095in}%
\pgfsys@useobject{currentmarker}{}%
\end{pgfscope}%
\end{pgfscope}%
\begin{pgfscope}%
\definecolor{textcolor}{rgb}{0.000000,0.000000,0.000000}%
\pgfsetstrokecolor{textcolor}%
\pgfsetfillcolor{textcolor}%
\pgftext[x=4.261699in,y=0.300872in,,top]{\color{textcolor}\rmfamily\fontsize{10.000000}{12.000000}\selectfont \SI{270}{\degree}}%
\end{pgfscope}%
\begin{pgfscope}%
\pgfsetbuttcap%
\pgfsetroundjoin%
\definecolor{currentfill}{rgb}{0.000000,0.000000,0.000000}%
\pgfsetfillcolor{currentfill}%
\pgfsetlinewidth{0.803000pt}%
\definecolor{currentstroke}{rgb}{0.000000,0.000000,0.000000}%
\pgfsetstrokecolor{currentstroke}%
\pgfsetdash{}{0pt}%
\pgfsys@defobject{currentmarker}{\pgfqpoint{0.000000in}{-0.048611in}}{\pgfqpoint{0.000000in}{0.000000in}}{%
\pgfpathmoveto{\pgfqpoint{0.000000in}{0.000000in}}%
\pgfpathlineto{\pgfqpoint{0.000000in}{-0.048611in}}%
\pgfusepath{stroke,fill}%
}%
\begin{pgfscope}%
\pgfsys@transformshift{4.888420in}{0.398095in}%
\pgfsys@useobject{currentmarker}{}%
\end{pgfscope}%
\end{pgfscope}%
\begin{pgfscope}%
\definecolor{textcolor}{rgb}{0.000000,0.000000,0.000000}%
\pgfsetstrokecolor{textcolor}%
\pgfsetfillcolor{textcolor}%
\pgftext[x=4.888420in,y=0.300872in,,top]{\color{textcolor}\rmfamily\fontsize{10.000000}{12.000000}\selectfont \SI{315}{\degree}}%
\end{pgfscope}%
\begin{pgfscope}%
\pgfsetbuttcap%
\pgfsetroundjoin%
\definecolor{currentfill}{rgb}{0.000000,0.000000,0.000000}%
\pgfsetfillcolor{currentfill}%
\pgfsetlinewidth{0.803000pt}%
\definecolor{currentstroke}{rgb}{0.000000,0.000000,0.000000}%
\pgfsetstrokecolor{currentstroke}%
\pgfsetdash{}{0pt}%
\pgfsys@defobject{currentmarker}{\pgfqpoint{0.000000in}{-0.048611in}}{\pgfqpoint{0.000000in}{0.000000in}}{%
\pgfpathmoveto{\pgfqpoint{0.000000in}{0.000000in}}%
\pgfpathlineto{\pgfqpoint{0.000000in}{-0.048611in}}%
\pgfusepath{stroke,fill}%
}%
\begin{pgfscope}%
\pgfsys@transformshift{5.515140in}{0.398095in}%
\pgfsys@useobject{currentmarker}{}%
\end{pgfscope}%
\end{pgfscope}%
\begin{pgfscope}%
\definecolor{textcolor}{rgb}{0.000000,0.000000,0.000000}%
\pgfsetstrokecolor{textcolor}%
\pgfsetfillcolor{textcolor}%
\pgftext[x=5.515140in,y=0.300872in,,top]{\color{textcolor}\rmfamily\fontsize{10.000000}{12.000000}\selectfont \SI{360}{\degree}}%
\end{pgfscope}%
\begin{pgfscope}%
\definecolor{textcolor}{rgb}{0.000000,0.000000,0.000000}%
\pgfsetstrokecolor{textcolor}%
\pgfsetfillcolor{textcolor}%
\pgftext[x=3.008258in,y=0.122662in,,top]{\color{textcolor}\rmfamily\fontsize{10.000000}{12.000000}\selectfont Azimutwinkel \(\displaystyle \varphi\)}%
\end{pgfscope}%
\begin{pgfscope}%
\pgfsetbuttcap%
\pgfsetroundjoin%
\definecolor{currentfill}{rgb}{0.000000,0.000000,0.000000}%
\pgfsetfillcolor{currentfill}%
\pgfsetlinewidth{0.803000pt}%
\definecolor{currentstroke}{rgb}{0.000000,0.000000,0.000000}%
\pgfsetstrokecolor{currentstroke}%
\pgfsetdash{}{0pt}%
\pgfsys@defobject{currentmarker}{\pgfqpoint{-0.048611in}{0.000000in}}{\pgfqpoint{-0.000000in}{0.000000in}}{%
\pgfpathmoveto{\pgfqpoint{-0.000000in}{0.000000in}}%
\pgfpathlineto{\pgfqpoint{-0.048611in}{0.000000in}}%
\pgfusepath{stroke,fill}%
}%
\begin{pgfscope}%
\pgfsys@transformshift{0.501376in}{0.398095in}%
\pgfsys@useobject{currentmarker}{}%
\end{pgfscope}%
\end{pgfscope}%
\begin{pgfscope}%
\definecolor{textcolor}{rgb}{0.000000,0.000000,0.000000}%
\pgfsetstrokecolor{textcolor}%
\pgfsetfillcolor{textcolor}%
\pgftext[x=0.334709in, y=0.350267in, left, base]{\color{textcolor}\rmfamily\fontsize{10.000000}{12.000000}\selectfont \(\displaystyle {0}\)}%
\end{pgfscope}%
\begin{pgfscope}%
\pgfsetbuttcap%
\pgfsetroundjoin%
\definecolor{currentfill}{rgb}{0.000000,0.000000,0.000000}%
\pgfsetfillcolor{currentfill}%
\pgfsetlinewidth{0.803000pt}%
\definecolor{currentstroke}{rgb}{0.000000,0.000000,0.000000}%
\pgfsetstrokecolor{currentstroke}%
\pgfsetdash{}{0pt}%
\pgfsys@defobject{currentmarker}{\pgfqpoint{-0.048611in}{0.000000in}}{\pgfqpoint{-0.000000in}{0.000000in}}{%
\pgfpathmoveto{\pgfqpoint{-0.000000in}{0.000000in}}%
\pgfpathlineto{\pgfqpoint{-0.048611in}{0.000000in}}%
\pgfusepath{stroke,fill}%
}%
\begin{pgfscope}%
\pgfsys@transformshift{0.501376in}{0.746273in}%
\pgfsys@useobject{currentmarker}{}%
\end{pgfscope}%
\end{pgfscope}%
\begin{pgfscope}%
\definecolor{textcolor}{rgb}{0.000000,0.000000,0.000000}%
\pgfsetstrokecolor{textcolor}%
\pgfsetfillcolor{textcolor}%
\pgftext[x=0.265265in, y=0.698445in, left, base]{\color{textcolor}\rmfamily\fontsize{10.000000}{12.000000}\selectfont \(\displaystyle {25}\)}%
\end{pgfscope}%
\begin{pgfscope}%
\pgfsetbuttcap%
\pgfsetroundjoin%
\definecolor{currentfill}{rgb}{0.000000,0.000000,0.000000}%
\pgfsetfillcolor{currentfill}%
\pgfsetlinewidth{0.803000pt}%
\definecolor{currentstroke}{rgb}{0.000000,0.000000,0.000000}%
\pgfsetstrokecolor{currentstroke}%
\pgfsetdash{}{0pt}%
\pgfsys@defobject{currentmarker}{\pgfqpoint{-0.048611in}{0.000000in}}{\pgfqpoint{-0.000000in}{0.000000in}}{%
\pgfpathmoveto{\pgfqpoint{-0.000000in}{0.000000in}}%
\pgfpathlineto{\pgfqpoint{-0.048611in}{0.000000in}}%
\pgfusepath{stroke,fill}%
}%
\begin{pgfscope}%
\pgfsys@transformshift{0.501376in}{1.094451in}%
\pgfsys@useobject{currentmarker}{}%
\end{pgfscope}%
\end{pgfscope}%
\begin{pgfscope}%
\definecolor{textcolor}{rgb}{0.000000,0.000000,0.000000}%
\pgfsetstrokecolor{textcolor}%
\pgfsetfillcolor{textcolor}%
\pgftext[x=0.265265in, y=1.046623in, left, base]{\color{textcolor}\rmfamily\fontsize{10.000000}{12.000000}\selectfont \(\displaystyle {50}\)}%
\end{pgfscope}%
\begin{pgfscope}%
\pgfsetbuttcap%
\pgfsetroundjoin%
\definecolor{currentfill}{rgb}{0.000000,0.000000,0.000000}%
\pgfsetfillcolor{currentfill}%
\pgfsetlinewidth{0.803000pt}%
\definecolor{currentstroke}{rgb}{0.000000,0.000000,0.000000}%
\pgfsetstrokecolor{currentstroke}%
\pgfsetdash{}{0pt}%
\pgfsys@defobject{currentmarker}{\pgfqpoint{-0.048611in}{0.000000in}}{\pgfqpoint{-0.000000in}{0.000000in}}{%
\pgfpathmoveto{\pgfqpoint{-0.000000in}{0.000000in}}%
\pgfpathlineto{\pgfqpoint{-0.048611in}{0.000000in}}%
\pgfusepath{stroke,fill}%
}%
\begin{pgfscope}%
\pgfsys@transformshift{0.501376in}{1.442629in}%
\pgfsys@useobject{currentmarker}{}%
\end{pgfscope}%
\end{pgfscope}%
\begin{pgfscope}%
\definecolor{textcolor}{rgb}{0.000000,0.000000,0.000000}%
\pgfsetstrokecolor{textcolor}%
\pgfsetfillcolor{textcolor}%
\pgftext[x=0.265265in, y=1.394801in, left, base]{\color{textcolor}\rmfamily\fontsize{10.000000}{12.000000}\selectfont \(\displaystyle {75}\)}%
\end{pgfscope}%
\begin{pgfscope}%
\pgfsetbuttcap%
\pgfsetroundjoin%
\definecolor{currentfill}{rgb}{0.000000,0.000000,0.000000}%
\pgfsetfillcolor{currentfill}%
\pgfsetlinewidth{0.803000pt}%
\definecolor{currentstroke}{rgb}{0.000000,0.000000,0.000000}%
\pgfsetstrokecolor{currentstroke}%
\pgfsetdash{}{0pt}%
\pgfsys@defobject{currentmarker}{\pgfqpoint{-0.048611in}{0.000000in}}{\pgfqpoint{-0.000000in}{0.000000in}}{%
\pgfpathmoveto{\pgfqpoint{-0.000000in}{0.000000in}}%
\pgfpathlineto{\pgfqpoint{-0.048611in}{0.000000in}}%
\pgfusepath{stroke,fill}%
}%
\begin{pgfscope}%
\pgfsys@transformshift{0.501376in}{1.790807in}%
\pgfsys@useobject{currentmarker}{}%
\end{pgfscope}%
\end{pgfscope}%
\begin{pgfscope}%
\definecolor{textcolor}{rgb}{0.000000,0.000000,0.000000}%
\pgfsetstrokecolor{textcolor}%
\pgfsetfillcolor{textcolor}%
\pgftext[x=0.195820in, y=1.742979in, left, base]{\color{textcolor}\rmfamily\fontsize{10.000000}{12.000000}\selectfont \(\displaystyle {100}\)}%
\end{pgfscope}%
\begin{pgfscope}%
\pgfsetbuttcap%
\pgfsetroundjoin%
\definecolor{currentfill}{rgb}{0.000000,0.000000,0.000000}%
\pgfsetfillcolor{currentfill}%
\pgfsetlinewidth{0.803000pt}%
\definecolor{currentstroke}{rgb}{0.000000,0.000000,0.000000}%
\pgfsetstrokecolor{currentstroke}%
\pgfsetdash{}{0pt}%
\pgfsys@defobject{currentmarker}{\pgfqpoint{-0.048611in}{0.000000in}}{\pgfqpoint{-0.000000in}{0.000000in}}{%
\pgfpathmoveto{\pgfqpoint{-0.000000in}{0.000000in}}%
\pgfpathlineto{\pgfqpoint{-0.048611in}{0.000000in}}%
\pgfusepath{stroke,fill}%
}%
\begin{pgfscope}%
\pgfsys@transformshift{0.501376in}{2.138985in}%
\pgfsys@useobject{currentmarker}{}%
\end{pgfscope}%
\end{pgfscope}%
\begin{pgfscope}%
\definecolor{textcolor}{rgb}{0.000000,0.000000,0.000000}%
\pgfsetstrokecolor{textcolor}%
\pgfsetfillcolor{textcolor}%
\pgftext[x=0.195820in, y=2.091157in, left, base]{\color{textcolor}\rmfamily\fontsize{10.000000}{12.000000}\selectfont \(\displaystyle {125}\)}%
\end{pgfscope}%
\begin{pgfscope}%
\pgfsetbuttcap%
\pgfsetroundjoin%
\definecolor{currentfill}{rgb}{0.000000,0.000000,0.000000}%
\pgfsetfillcolor{currentfill}%
\pgfsetlinewidth{0.803000pt}%
\definecolor{currentstroke}{rgb}{0.000000,0.000000,0.000000}%
\pgfsetstrokecolor{currentstroke}%
\pgfsetdash{}{0pt}%
\pgfsys@defobject{currentmarker}{\pgfqpoint{-0.048611in}{0.000000in}}{\pgfqpoint{-0.000000in}{0.000000in}}{%
\pgfpathmoveto{\pgfqpoint{-0.000000in}{0.000000in}}%
\pgfpathlineto{\pgfqpoint{-0.048611in}{0.000000in}}%
\pgfusepath{stroke,fill}%
}%
\begin{pgfscope}%
\pgfsys@transformshift{0.501376in}{2.487163in}%
\pgfsys@useobject{currentmarker}{}%
\end{pgfscope}%
\end{pgfscope}%
\begin{pgfscope}%
\definecolor{textcolor}{rgb}{0.000000,0.000000,0.000000}%
\pgfsetstrokecolor{textcolor}%
\pgfsetfillcolor{textcolor}%
\pgftext[x=0.195820in, y=2.439335in, left, base]{\color{textcolor}\rmfamily\fontsize{10.000000}{12.000000}\selectfont \(\displaystyle {150}\)}%
\end{pgfscope}%
\begin{pgfscope}%
\definecolor{textcolor}{rgb}{0.000000,0.000000,0.000000}%
\pgfsetstrokecolor{textcolor}%
\pgfsetfillcolor{textcolor}%
\pgftext[x=0.140265in,y=1.581900in,,bottom,rotate=90.000000]{\color{textcolor}\rmfamily\fontsize{10.000000}{12.000000}\selectfont Radius \(\displaystyle r\) in px}%
\end{pgfscope}%
\begin{pgfscope}%
\pgfsetrectcap%
\pgfsetmiterjoin%
\pgfsetlinewidth{0.803000pt}%
\definecolor{currentstroke}{rgb}{0.000000,0.000000,0.000000}%
\pgfsetstrokecolor{currentstroke}%
\pgfsetdash{}{0pt}%
\pgfpathmoveto{\pgfqpoint{0.501376in}{0.398095in}}%
\pgfpathlineto{\pgfqpoint{0.501376in}{2.765705in}}%
\pgfusepath{stroke}%
\end{pgfscope}%
\begin{pgfscope}%
\pgfsetrectcap%
\pgfsetmiterjoin%
\pgfsetlinewidth{0.803000pt}%
\definecolor{currentstroke}{rgb}{0.000000,0.000000,0.000000}%
\pgfsetstrokecolor{currentstroke}%
\pgfsetdash{}{0pt}%
\pgfpathmoveto{\pgfqpoint{5.515140in}{0.398095in}}%
\pgfpathlineto{\pgfqpoint{5.515140in}{2.765705in}}%
\pgfusepath{stroke}%
\end{pgfscope}%
\begin{pgfscope}%
\pgfsetrectcap%
\pgfsetmiterjoin%
\pgfsetlinewidth{0.803000pt}%
\definecolor{currentstroke}{rgb}{0.000000,0.000000,0.000000}%
\pgfsetstrokecolor{currentstroke}%
\pgfsetdash{}{0pt}%
\pgfpathmoveto{\pgfqpoint{0.501376in}{0.398095in}}%
\pgfpathlineto{\pgfqpoint{5.515140in}{0.398095in}}%
\pgfusepath{stroke}%
\end{pgfscope}%
\begin{pgfscope}%
\pgfsetrectcap%
\pgfsetmiterjoin%
\pgfsetlinewidth{0.803000pt}%
\definecolor{currentstroke}{rgb}{0.000000,0.000000,0.000000}%
\pgfsetstrokecolor{currentstroke}%
\pgfsetdash{}{0pt}%
\pgfpathmoveto{\pgfqpoint{0.501376in}{2.765705in}}%
\pgfpathlineto{\pgfqpoint{5.515140in}{2.765705in}}%
\pgfusepath{stroke}%
\end{pgfscope}%
\begin{pgfscope}%
\definecolor{textcolor}{rgb}{0.000000,0.000000,0.000000}%
\pgfsetstrokecolor{textcolor}%
\pgfsetfillcolor{textcolor}%
\pgftext[x=0.000000in,y=3.002466in,left,base]{\color{textcolor}\rmfamily\fontsize{10.000000}{12.000000}\selectfont (c)}%
\end{pgfscope}%
\end{pgfpicture}%
\makeatother%
\endgroup%

    \caption{(a) Die Summe von \num{50000} ausgewerteten Aufnahmen, die sowie mit dem Schwellenwert $s_V = \SI{100}{\adu}$ und der oberen Grenze \SI{600}{\adu} ausgewertet wurden. Davon wird der Offset $\Delta_\text{\gls{fdpa}} = \SI[per-mode = symbol]{1,7}{\photons\per\pixel}$ abgezogen. Der grüne Punkt bezeichnet den Mittelpunkt des elliptischen Umrisses, der für die Transformation von Koordinatensystem benutzt wird. Im Bild (b) ist der Direktstrahl ausmaskiert. Das Pfeilende entspricht dem Winkel $\varphi = \SI{0}{\degree}$, die Pfeilrichtung entspricht der positiven Richtung der Azimutalwinkelkoordinate. In Unterabb. (c) ist die in Polarkoordinaten transformierte Unterabb. (b).}
    \label{fig:th-100-200-maske-radial-transform}
\end{figure}
\noindent
Der Direktstrahl wird vor der Transformation ausmaskiert (Abb. \ref{fig:th-100-200-maske-radial-transform}b), damit er zur Streuintensität des Streurings in dem Überlappbereich nicht beiträgt.

\noindent
Der Betrag des Streuvektors $q$ kann aus dem Ringradius $r$ berechnet werden über
\begin{equation}
    q(r) = \frac{4\pi}{\lambda_\text{Gd, M5}}\sin\left(\arctan\frac{r}{l}\right),
    \label{eq:streuvektor_von_radius}
\end{equation}

\noindent
Die Intensität in Polarkoordinaten (Abb. \ref{fig:th-100-200-maske-radial-transform}c) wird über den Azimutwinkel $\varphi$ aufintegriert. Die radiale Intensität wird zunächst über den Streuvektorbertrag $q$  aufgetragen. Das resultierende Spektrum der Kleinwinkelstreuung wird mit der Gamma-Funktion
\begin{equation}
    g(q, \beta, \alpha, A) = A\frac{q^{\beta-1}\exp\left[-\frac{q}{\alpha}\right]}{\alpha^\beta\Gamma(\beta)}
\end{equation}
angepasst. Diese Verteilungsfunktionen eigenen sich gut für die Beschreibung der Streuung von magnetischen Wurmdomänen \cite{bagschik_employing_2016}. Das Maximum der Funktion $g(q, \beta, \alpha, A)$ liegt an der Stelle $q_\text{max} = \alpha\beta$. Das Spektrum der Kleinwinkelstreuung, sowie die Fit-Funktion werden zusammen mit der Referenz in Abb. \ref{fig:radius_fit} dargestellt.
\begin{figure}[H]
    \centering
    %% Creator: Matplotlib, PGF backend
%%
%% To include the figure in your LaTeX document, write
%%   \input{<filename>.pgf}
%%
%% Make sure the required packages are loaded in your preamble
%%   \usepackage{pgf}
%%
%% Also ensure that all the required font packages are loaded; for instance,
%% the lmodern package is sometimes necessary when using math font.
%%   \usepackage{lmodern}
%%
%% Figures using additional raster images can only be included by \input if
%% they are in the same directory as the main LaTeX file. For loading figures
%% from other directories you can use the `import` package
%%   \usepackage{import}
%%
%% and then include the figures with
%%   \import{<path to file>}{<filename>.pgf}
%%
%% Matplotlib used the following preamble
%%   \usepackage{amsmath} \usepackage[utf8]{inputenc} \usepackage[T1]{fontenc} \usepackage[output-decimal-marker={,},print-unity-mantissa=false]{siunitx} \sisetup{per-mode=fraction, separate-uncertainty = true, locale = DE} \usepackage[acronym, toc, section=section, nonumberlist, nopostdot]{glossaries-extra} \DeclareSIUnit\adu{\text{ADU}} \DeclareSIUnit\px{\text{px}} \DeclareSIUnit\photons{\text{Pho\-to\-nen}} \DeclareSIUnit\photon{\text{Pho\-ton}}
%%
\begingroup%
\makeatletter%
\begin{pgfpicture}%
\pgfpathrectangle{\pgfpointorigin}{\pgfqpoint{6.381121in}{3.407926in}}%
\pgfusepath{use as bounding box, clip}%
\begin{pgfscope}%
\pgfsetbuttcap%
\pgfsetmiterjoin%
\pgfsetlinewidth{0.000000pt}%
\definecolor{currentstroke}{rgb}{1.000000,1.000000,1.000000}%
\pgfsetstrokecolor{currentstroke}%
\pgfsetstrokeopacity{0.000000}%
\pgfsetdash{}{0pt}%
\pgfpathmoveto{\pgfqpoint{0.000000in}{0.000000in}}%
\pgfpathlineto{\pgfqpoint{6.381121in}{0.000000in}}%
\pgfpathlineto{\pgfqpoint{6.381121in}{3.407926in}}%
\pgfpathlineto{\pgfqpoint{0.000000in}{3.407926in}}%
\pgfpathlineto{\pgfqpoint{0.000000in}{0.000000in}}%
\pgfpathclose%
\pgfusepath{}%
\end{pgfscope}%
\begin{pgfscope}%
\pgfsetbuttcap%
\pgfsetmiterjoin%
\definecolor{currentfill}{rgb}{1.000000,1.000000,1.000000}%
\pgfsetfillcolor{currentfill}%
\pgfsetlinewidth{0.000000pt}%
\definecolor{currentstroke}{rgb}{0.000000,0.000000,0.000000}%
\pgfsetstrokecolor{currentstroke}%
\pgfsetstrokeopacity{0.000000}%
\pgfsetdash{}{0pt}%
\pgfpathmoveto{\pgfqpoint{0.552903in}{0.522439in}}%
\pgfpathlineto{\pgfqpoint{6.281121in}{0.522439in}}%
\pgfpathlineto{\pgfqpoint{6.281121in}{3.307926in}}%
\pgfpathlineto{\pgfqpoint{0.552903in}{3.307926in}}%
\pgfpathlineto{\pgfqpoint{0.552903in}{0.522439in}}%
\pgfpathclose%
\pgfusepath{fill}%
\end{pgfscope}%
\begin{pgfscope}%
\pgfsetbuttcap%
\pgfsetroundjoin%
\definecolor{currentfill}{rgb}{0.000000,0.000000,0.000000}%
\pgfsetfillcolor{currentfill}%
\pgfsetlinewidth{0.803000pt}%
\definecolor{currentstroke}{rgb}{0.000000,0.000000,0.000000}%
\pgfsetstrokecolor{currentstroke}%
\pgfsetdash{}{0pt}%
\pgfsys@defobject{currentmarker}{\pgfqpoint{0.000000in}{-0.048611in}}{\pgfqpoint{0.000000in}{0.000000in}}{%
\pgfpathmoveto{\pgfqpoint{0.000000in}{0.000000in}}%
\pgfpathlineto{\pgfqpoint{0.000000in}{-0.048611in}}%
\pgfusepath{stroke,fill}%
}%
\begin{pgfscope}%
\pgfsys@transformshift{0.552903in}{0.522439in}%
\pgfsys@useobject{currentmarker}{}%
\end{pgfscope}%
\end{pgfscope}%
\begin{pgfscope}%
\definecolor{textcolor}{rgb}{0.000000,0.000000,0.000000}%
\pgfsetstrokecolor{textcolor}%
\pgfsetfillcolor{textcolor}%
\pgftext[x=0.552903in,y=0.425217in,,top]{\color{textcolor}\rmfamily\fontsize{10.000000}{12.000000}\selectfont \(\displaystyle {0}\)}%
\end{pgfscope}%
\begin{pgfscope}%
\pgfsetbuttcap%
\pgfsetroundjoin%
\definecolor{currentfill}{rgb}{0.000000,0.000000,0.000000}%
\pgfsetfillcolor{currentfill}%
\pgfsetlinewidth{0.803000pt}%
\definecolor{currentstroke}{rgb}{0.000000,0.000000,0.000000}%
\pgfsetstrokecolor{currentstroke}%
\pgfsetdash{}{0pt}%
\pgfsys@defobject{currentmarker}{\pgfqpoint{0.000000in}{-0.048611in}}{\pgfqpoint{0.000000in}{0.000000in}}{%
\pgfpathmoveto{\pgfqpoint{0.000000in}{0.000000in}}%
\pgfpathlineto{\pgfqpoint{0.000000in}{-0.048611in}}%
\pgfusepath{stroke,fill}%
}%
\begin{pgfscope}%
\pgfsys@transformshift{1.437013in}{0.522439in}%
\pgfsys@useobject{currentmarker}{}%
\end{pgfscope}%
\end{pgfscope}%
\begin{pgfscope}%
\definecolor{textcolor}{rgb}{0.000000,0.000000,0.000000}%
\pgfsetstrokecolor{textcolor}%
\pgfsetfillcolor{textcolor}%
\pgftext[x=1.437013in,y=0.425217in,,top]{\color{textcolor}\rmfamily\fontsize{10.000000}{12.000000}\selectfont \(\displaystyle {10}\)}%
\end{pgfscope}%
\begin{pgfscope}%
\pgfsetbuttcap%
\pgfsetroundjoin%
\definecolor{currentfill}{rgb}{0.000000,0.000000,0.000000}%
\pgfsetfillcolor{currentfill}%
\pgfsetlinewidth{0.803000pt}%
\definecolor{currentstroke}{rgb}{0.000000,0.000000,0.000000}%
\pgfsetstrokecolor{currentstroke}%
\pgfsetdash{}{0pt}%
\pgfsys@defobject{currentmarker}{\pgfqpoint{0.000000in}{-0.048611in}}{\pgfqpoint{0.000000in}{0.000000in}}{%
\pgfpathmoveto{\pgfqpoint{0.000000in}{0.000000in}}%
\pgfpathlineto{\pgfqpoint{0.000000in}{-0.048611in}}%
\pgfusepath{stroke,fill}%
}%
\begin{pgfscope}%
\pgfsys@transformshift{2.321124in}{0.522439in}%
\pgfsys@useobject{currentmarker}{}%
\end{pgfscope}%
\end{pgfscope}%
\begin{pgfscope}%
\definecolor{textcolor}{rgb}{0.000000,0.000000,0.000000}%
\pgfsetstrokecolor{textcolor}%
\pgfsetfillcolor{textcolor}%
\pgftext[x=2.321124in,y=0.425217in,,top]{\color{textcolor}\rmfamily\fontsize{10.000000}{12.000000}\selectfont \(\displaystyle {20}\)}%
\end{pgfscope}%
\begin{pgfscope}%
\pgfsetbuttcap%
\pgfsetroundjoin%
\definecolor{currentfill}{rgb}{0.000000,0.000000,0.000000}%
\pgfsetfillcolor{currentfill}%
\pgfsetlinewidth{0.803000pt}%
\definecolor{currentstroke}{rgb}{0.000000,0.000000,0.000000}%
\pgfsetstrokecolor{currentstroke}%
\pgfsetdash{}{0pt}%
\pgfsys@defobject{currentmarker}{\pgfqpoint{0.000000in}{-0.048611in}}{\pgfqpoint{0.000000in}{0.000000in}}{%
\pgfpathmoveto{\pgfqpoint{0.000000in}{0.000000in}}%
\pgfpathlineto{\pgfqpoint{0.000000in}{-0.048611in}}%
\pgfusepath{stroke,fill}%
}%
\begin{pgfscope}%
\pgfsys@transformshift{3.205234in}{0.522439in}%
\pgfsys@useobject{currentmarker}{}%
\end{pgfscope}%
\end{pgfscope}%
\begin{pgfscope}%
\definecolor{textcolor}{rgb}{0.000000,0.000000,0.000000}%
\pgfsetstrokecolor{textcolor}%
\pgfsetfillcolor{textcolor}%
\pgftext[x=3.205234in,y=0.425217in,,top]{\color{textcolor}\rmfamily\fontsize{10.000000}{12.000000}\selectfont \(\displaystyle {30}\)}%
\end{pgfscope}%
\begin{pgfscope}%
\pgfsetbuttcap%
\pgfsetroundjoin%
\definecolor{currentfill}{rgb}{0.000000,0.000000,0.000000}%
\pgfsetfillcolor{currentfill}%
\pgfsetlinewidth{0.803000pt}%
\definecolor{currentstroke}{rgb}{0.000000,0.000000,0.000000}%
\pgfsetstrokecolor{currentstroke}%
\pgfsetdash{}{0pt}%
\pgfsys@defobject{currentmarker}{\pgfqpoint{0.000000in}{-0.048611in}}{\pgfqpoint{0.000000in}{0.000000in}}{%
\pgfpathmoveto{\pgfqpoint{0.000000in}{0.000000in}}%
\pgfpathlineto{\pgfqpoint{0.000000in}{-0.048611in}}%
\pgfusepath{stroke,fill}%
}%
\begin{pgfscope}%
\pgfsys@transformshift{4.089344in}{0.522439in}%
\pgfsys@useobject{currentmarker}{}%
\end{pgfscope}%
\end{pgfscope}%
\begin{pgfscope}%
\definecolor{textcolor}{rgb}{0.000000,0.000000,0.000000}%
\pgfsetstrokecolor{textcolor}%
\pgfsetfillcolor{textcolor}%
\pgftext[x=4.089344in,y=0.425217in,,top]{\color{textcolor}\rmfamily\fontsize{10.000000}{12.000000}\selectfont \(\displaystyle {40}\)}%
\end{pgfscope}%
\begin{pgfscope}%
\pgfsetbuttcap%
\pgfsetroundjoin%
\definecolor{currentfill}{rgb}{0.000000,0.000000,0.000000}%
\pgfsetfillcolor{currentfill}%
\pgfsetlinewidth{0.803000pt}%
\definecolor{currentstroke}{rgb}{0.000000,0.000000,0.000000}%
\pgfsetstrokecolor{currentstroke}%
\pgfsetdash{}{0pt}%
\pgfsys@defobject{currentmarker}{\pgfqpoint{0.000000in}{-0.048611in}}{\pgfqpoint{0.000000in}{0.000000in}}{%
\pgfpathmoveto{\pgfqpoint{0.000000in}{0.000000in}}%
\pgfpathlineto{\pgfqpoint{0.000000in}{-0.048611in}}%
\pgfusepath{stroke,fill}%
}%
\begin{pgfscope}%
\pgfsys@transformshift{4.973455in}{0.522439in}%
\pgfsys@useobject{currentmarker}{}%
\end{pgfscope}%
\end{pgfscope}%
\begin{pgfscope}%
\definecolor{textcolor}{rgb}{0.000000,0.000000,0.000000}%
\pgfsetstrokecolor{textcolor}%
\pgfsetfillcolor{textcolor}%
\pgftext[x=4.973455in,y=0.425217in,,top]{\color{textcolor}\rmfamily\fontsize{10.000000}{12.000000}\selectfont \(\displaystyle {50}\)}%
\end{pgfscope}%
\begin{pgfscope}%
\pgfsetbuttcap%
\pgfsetroundjoin%
\definecolor{currentfill}{rgb}{0.000000,0.000000,0.000000}%
\pgfsetfillcolor{currentfill}%
\pgfsetlinewidth{0.803000pt}%
\definecolor{currentstroke}{rgb}{0.000000,0.000000,0.000000}%
\pgfsetstrokecolor{currentstroke}%
\pgfsetdash{}{0pt}%
\pgfsys@defobject{currentmarker}{\pgfqpoint{0.000000in}{-0.048611in}}{\pgfqpoint{0.000000in}{0.000000in}}{%
\pgfpathmoveto{\pgfqpoint{0.000000in}{0.000000in}}%
\pgfpathlineto{\pgfqpoint{0.000000in}{-0.048611in}}%
\pgfusepath{stroke,fill}%
}%
\begin{pgfscope}%
\pgfsys@transformshift{5.857565in}{0.522439in}%
\pgfsys@useobject{currentmarker}{}%
\end{pgfscope}%
\end{pgfscope}%
\begin{pgfscope}%
\definecolor{textcolor}{rgb}{0.000000,0.000000,0.000000}%
\pgfsetstrokecolor{textcolor}%
\pgfsetfillcolor{textcolor}%
\pgftext[x=5.857565in,y=0.425217in,,top]{\color{textcolor}\rmfamily\fontsize{10.000000}{12.000000}\selectfont \(\displaystyle {60}\)}%
\end{pgfscope}%
\begin{pgfscope}%
\definecolor{textcolor}{rgb}{0.000000,0.000000,0.000000}%
\pgfsetstrokecolor{textcolor}%
\pgfsetfillcolor{textcolor}%
\pgftext[x=3.417012in,y=0.247007in,,top]{\color{textcolor}\rmfamily\fontsize{10.000000}{12.000000}\selectfont Streuvektorbetrag \(\displaystyle q\) in \(\displaystyle \si{\micro\meter^{-1}}\)}%
\end{pgfscope}%
\begin{pgfscope}%
\pgfsetbuttcap%
\pgfsetroundjoin%
\definecolor{currentfill}{rgb}{0.000000,0.000000,0.000000}%
\pgfsetfillcolor{currentfill}%
\pgfsetlinewidth{0.803000pt}%
\definecolor{currentstroke}{rgb}{0.000000,0.000000,0.000000}%
\pgfsetstrokecolor{currentstroke}%
\pgfsetdash{}{0pt}%
\pgfsys@defobject{currentmarker}{\pgfqpoint{-0.048611in}{0.000000in}}{\pgfqpoint{-0.000000in}{0.000000in}}{%
\pgfpathmoveto{\pgfqpoint{-0.000000in}{0.000000in}}%
\pgfpathlineto{\pgfqpoint{-0.048611in}{0.000000in}}%
\pgfusepath{stroke,fill}%
}%
\begin{pgfscope}%
\pgfsys@transformshift{0.552903in}{0.888104in}%
\pgfsys@useobject{currentmarker}{}%
\end{pgfscope}%
\end{pgfscope}%
\begin{pgfscope}%
\definecolor{textcolor}{rgb}{0.000000,0.000000,0.000000}%
\pgfsetstrokecolor{textcolor}%
\pgfsetfillcolor{textcolor}%
\pgftext[x=0.278211in, y=0.840280in, left, base]{\color{textcolor}\rmfamily\fontsize{10.000000}{12.000000}\selectfont \num{0.0}}%
\end{pgfscope}%
\begin{pgfscope}%
\pgfsetbuttcap%
\pgfsetroundjoin%
\definecolor{currentfill}{rgb}{0.000000,0.000000,0.000000}%
\pgfsetfillcolor{currentfill}%
\pgfsetlinewidth{0.803000pt}%
\definecolor{currentstroke}{rgb}{0.000000,0.000000,0.000000}%
\pgfsetstrokecolor{currentstroke}%
\pgfsetdash{}{0pt}%
\pgfsys@defobject{currentmarker}{\pgfqpoint{-0.048611in}{0.000000in}}{\pgfqpoint{-0.000000in}{0.000000in}}{%
\pgfpathmoveto{\pgfqpoint{-0.000000in}{0.000000in}}%
\pgfpathlineto{\pgfqpoint{-0.048611in}{0.000000in}}%
\pgfusepath{stroke,fill}%
}%
\begin{pgfscope}%
\pgfsys@transformshift{0.552903in}{1.656916in}%
\pgfsys@useobject{currentmarker}{}%
\end{pgfscope}%
\end{pgfscope}%
\begin{pgfscope}%
\definecolor{textcolor}{rgb}{0.000000,0.000000,0.000000}%
\pgfsetstrokecolor{textcolor}%
\pgfsetfillcolor{textcolor}%
\pgftext[x=0.278211in, y=1.609091in, left, base]{\color{textcolor}\rmfamily\fontsize{10.000000}{12.000000}\selectfont \num{1.0}}%
\end{pgfscope}%
\begin{pgfscope}%
\pgfsetbuttcap%
\pgfsetroundjoin%
\definecolor{currentfill}{rgb}{0.000000,0.000000,0.000000}%
\pgfsetfillcolor{currentfill}%
\pgfsetlinewidth{0.803000pt}%
\definecolor{currentstroke}{rgb}{0.000000,0.000000,0.000000}%
\pgfsetstrokecolor{currentstroke}%
\pgfsetdash{}{0pt}%
\pgfsys@defobject{currentmarker}{\pgfqpoint{-0.048611in}{0.000000in}}{\pgfqpoint{-0.000000in}{0.000000in}}{%
\pgfpathmoveto{\pgfqpoint{-0.000000in}{0.000000in}}%
\pgfpathlineto{\pgfqpoint{-0.048611in}{0.000000in}}%
\pgfusepath{stroke,fill}%
}%
\begin{pgfscope}%
\pgfsys@transformshift{0.552903in}{2.425727in}%
\pgfsys@useobject{currentmarker}{}%
\end{pgfscope}%
\end{pgfscope}%
\begin{pgfscope}%
\definecolor{textcolor}{rgb}{0.000000,0.000000,0.000000}%
\pgfsetstrokecolor{textcolor}%
\pgfsetfillcolor{textcolor}%
\pgftext[x=0.278211in, y=2.377902in, left, base]{\color{textcolor}\rmfamily\fontsize{10.000000}{12.000000}\selectfont \num{2.0}}%
\end{pgfscope}%
\begin{pgfscope}%
\pgfsetbuttcap%
\pgfsetroundjoin%
\definecolor{currentfill}{rgb}{0.000000,0.000000,0.000000}%
\pgfsetfillcolor{currentfill}%
\pgfsetlinewidth{0.803000pt}%
\definecolor{currentstroke}{rgb}{0.000000,0.000000,0.000000}%
\pgfsetstrokecolor{currentstroke}%
\pgfsetdash{}{0pt}%
\pgfsys@defobject{currentmarker}{\pgfqpoint{-0.048611in}{0.000000in}}{\pgfqpoint{-0.000000in}{0.000000in}}{%
\pgfpathmoveto{\pgfqpoint{-0.000000in}{0.000000in}}%
\pgfpathlineto{\pgfqpoint{-0.048611in}{0.000000in}}%
\pgfusepath{stroke,fill}%
}%
\begin{pgfscope}%
\pgfsys@transformshift{0.552903in}{3.194538in}%
\pgfsys@useobject{currentmarker}{}%
\end{pgfscope}%
\end{pgfscope}%
\begin{pgfscope}%
\definecolor{textcolor}{rgb}{0.000000,0.000000,0.000000}%
\pgfsetstrokecolor{textcolor}%
\pgfsetfillcolor{textcolor}%
\pgftext[x=0.278211in, y=3.146714in, left, base]{\color{textcolor}\rmfamily\fontsize{10.000000}{12.000000}\selectfont \num{3.0}}%
\end{pgfscope}%
\begin{pgfscope}%
\definecolor{textcolor}{rgb}{0.000000,0.000000,0.000000}%
\pgfsetstrokecolor{textcolor}%
\pgfsetfillcolor{textcolor}%
\pgftext[x=0.222655in,y=1.915183in,,bottom,rotate=90.000000]{\color{textcolor}\rmfamily\fontsize{10.000000}{12.000000}\selectfont radiale Streuintensität in bel. Einheiten}%
\end{pgfscope}%
\begin{pgfscope}%
\pgfpathrectangle{\pgfqpoint{0.552903in}{0.522439in}}{\pgfqpoint{5.728219in}{2.785486in}}%
\pgfusepath{clip}%
\pgfsetrectcap%
\pgfsetroundjoin%
\pgfsetlinewidth{1.505625pt}%
\definecolor{currentstroke}{rgb}{1.000000,0.498039,0.054902}%
\pgfsetstrokecolor{currentstroke}%
\pgfsetdash{}{0pt}%
\pgfpathmoveto{\pgfqpoint{1.276949in}{1.222367in}}%
\pgfpathlineto{\pgfqpoint{1.295280in}{1.238391in}}%
\pgfpathlineto{\pgfqpoint{1.313610in}{1.268595in}}%
\pgfpathlineto{\pgfqpoint{1.331940in}{1.304714in}}%
\pgfpathlineto{\pgfqpoint{1.350271in}{1.334841in}}%
\pgfpathlineto{\pgfqpoint{1.368601in}{1.333933in}}%
\pgfpathlineto{\pgfqpoint{1.386931in}{1.334364in}}%
\pgfpathlineto{\pgfqpoint{1.405262in}{1.303767in}}%
\pgfpathlineto{\pgfqpoint{1.423592in}{1.329195in}}%
\pgfpathlineto{\pgfqpoint{1.441922in}{1.349274in}}%
\pgfpathlineto{\pgfqpoint{1.460252in}{1.359410in}}%
\pgfpathlineto{\pgfqpoint{1.478583in}{1.364356in}}%
\pgfpathlineto{\pgfqpoint{1.496913in}{1.374820in}}%
\pgfpathlineto{\pgfqpoint{1.515243in}{1.423103in}}%
\pgfpathlineto{\pgfqpoint{1.533574in}{1.425599in}}%
\pgfpathlineto{\pgfqpoint{1.551904in}{1.461533in}}%
\pgfpathlineto{\pgfqpoint{1.570234in}{1.506985in}}%
\pgfpathlineto{\pgfqpoint{1.588565in}{1.520465in}}%
\pgfpathlineto{\pgfqpoint{1.606895in}{1.552731in}}%
\pgfpathlineto{\pgfqpoint{1.625225in}{1.562756in}}%
\pgfpathlineto{\pgfqpoint{1.643555in}{1.554645in}}%
\pgfpathlineto{\pgfqpoint{1.661886in}{1.561322in}}%
\pgfpathlineto{\pgfqpoint{1.680216in}{1.555547in}}%
\pgfpathlineto{\pgfqpoint{1.698546in}{1.579216in}}%
\pgfpathlineto{\pgfqpoint{1.716877in}{1.708375in}}%
\pgfpathlineto{\pgfqpoint{1.735207in}{1.773770in}}%
\pgfpathlineto{\pgfqpoint{1.753537in}{1.870226in}}%
\pgfpathlineto{\pgfqpoint{1.771868in}{1.913600in}}%
\pgfpathlineto{\pgfqpoint{1.790198in}{1.992099in}}%
\pgfpathlineto{\pgfqpoint{1.808528in}{2.200571in}}%
\pgfpathlineto{\pgfqpoint{1.826858in}{2.331199in}}%
\pgfpathlineto{\pgfqpoint{1.845189in}{2.354311in}}%
\pgfpathlineto{\pgfqpoint{1.863519in}{2.375043in}}%
\pgfpathlineto{\pgfqpoint{1.881849in}{2.437775in}}%
\pgfpathlineto{\pgfqpoint{1.900180in}{2.544203in}}%
\pgfpathlineto{\pgfqpoint{1.918510in}{2.587929in}}%
\pgfpathlineto{\pgfqpoint{1.936840in}{2.503247in}}%
\pgfpathlineto{\pgfqpoint{1.955171in}{2.559904in}}%
\pgfpathlineto{\pgfqpoint{1.973501in}{2.547973in}}%
\pgfpathlineto{\pgfqpoint{1.991831in}{2.615501in}}%
\pgfpathlineto{\pgfqpoint{2.010161in}{2.751887in}}%
\pgfpathlineto{\pgfqpoint{2.028492in}{2.814614in}}%
\pgfpathlineto{\pgfqpoint{2.046822in}{2.863683in}}%
\pgfpathlineto{\pgfqpoint{2.065152in}{2.952108in}}%
\pgfpathlineto{\pgfqpoint{2.083483in}{3.020847in}}%
\pgfpathlineto{\pgfqpoint{2.101813in}{2.933786in}}%
\pgfpathlineto{\pgfqpoint{2.120143in}{2.832564in}}%
\pgfpathlineto{\pgfqpoint{2.138474in}{2.723731in}}%
\pgfpathlineto{\pgfqpoint{2.156804in}{2.681129in}}%
\pgfpathlineto{\pgfqpoint{2.175134in}{2.616891in}}%
\pgfpathlineto{\pgfqpoint{2.193464in}{2.582685in}}%
\pgfpathlineto{\pgfqpoint{2.211795in}{2.504624in}}%
\pgfpathlineto{\pgfqpoint{2.230125in}{2.427686in}}%
\pgfpathlineto{\pgfqpoint{2.248455in}{2.288815in}}%
\pgfpathlineto{\pgfqpoint{2.266786in}{2.185570in}}%
\pgfpathlineto{\pgfqpoint{2.285116in}{2.040496in}}%
\pgfpathlineto{\pgfqpoint{2.303446in}{1.942840in}}%
\pgfpathlineto{\pgfqpoint{2.321777in}{1.897559in}}%
\pgfpathlineto{\pgfqpoint{2.340107in}{1.850715in}}%
\pgfpathlineto{\pgfqpoint{2.358437in}{1.643127in}}%
\pgfpathlineto{\pgfqpoint{2.376767in}{1.567377in}}%
\pgfpathlineto{\pgfqpoint{2.395098in}{1.508940in}}%
\pgfpathlineto{\pgfqpoint{2.413428in}{1.458675in}}%
\pgfpathlineto{\pgfqpoint{2.431758in}{1.450565in}}%
\pgfpathlineto{\pgfqpoint{2.450089in}{1.346251in}}%
\pgfpathlineto{\pgfqpoint{2.468419in}{1.271344in}}%
\pgfpathlineto{\pgfqpoint{2.486749in}{1.243014in}}%
\pgfpathlineto{\pgfqpoint{2.505080in}{1.203321in}}%
\pgfpathlineto{\pgfqpoint{2.523410in}{1.188704in}}%
\pgfpathlineto{\pgfqpoint{2.541740in}{1.169301in}}%
\pgfpathlineto{\pgfqpoint{2.560070in}{1.128996in}}%
\pgfpathlineto{\pgfqpoint{2.578401in}{1.105615in}}%
\pgfpathlineto{\pgfqpoint{2.596731in}{1.074203in}}%
\pgfpathlineto{\pgfqpoint{2.615061in}{1.053277in}}%
\pgfpathlineto{\pgfqpoint{2.633392in}{1.039171in}}%
\pgfpathlineto{\pgfqpoint{2.651722in}{1.026698in}}%
\pgfpathlineto{\pgfqpoint{2.670052in}{1.009196in}}%
\pgfpathlineto{\pgfqpoint{2.688383in}{1.002542in}}%
\pgfpathlineto{\pgfqpoint{2.706713in}{0.989250in}}%
\pgfpathlineto{\pgfqpoint{2.725043in}{0.982044in}}%
\pgfpathlineto{\pgfqpoint{2.743373in}{0.977785in}}%
\pgfpathlineto{\pgfqpoint{2.780034in}{0.962543in}}%
\pgfpathlineto{\pgfqpoint{2.798364in}{0.954222in}}%
\pgfpathlineto{\pgfqpoint{2.816695in}{0.949674in}}%
\pgfpathlineto{\pgfqpoint{2.835025in}{0.946672in}}%
\pgfpathlineto{\pgfqpoint{2.853355in}{0.941357in}}%
\pgfpathlineto{\pgfqpoint{2.871686in}{0.940230in}}%
\pgfpathlineto{\pgfqpoint{2.890016in}{0.936253in}}%
\pgfpathlineto{\pgfqpoint{2.908346in}{0.931596in}}%
\pgfpathlineto{\pgfqpoint{2.926676in}{0.928109in}}%
\pgfpathlineto{\pgfqpoint{2.945007in}{0.927945in}}%
\pgfpathlineto{\pgfqpoint{2.963337in}{0.925053in}}%
\pgfpathlineto{\pgfqpoint{2.981667in}{0.923391in}}%
\pgfpathlineto{\pgfqpoint{2.999998in}{0.920789in}}%
\pgfpathlineto{\pgfqpoint{3.018328in}{0.920322in}}%
\pgfpathlineto{\pgfqpoint{3.036658in}{0.919079in}}%
\pgfpathlineto{\pgfqpoint{3.073319in}{0.917096in}}%
\pgfpathlineto{\pgfqpoint{3.091649in}{0.914881in}}%
\pgfpathlineto{\pgfqpoint{3.109979in}{0.913589in}}%
\pgfpathlineto{\pgfqpoint{3.128310in}{0.911885in}}%
\pgfpathlineto{\pgfqpoint{3.146640in}{0.910636in}}%
\pgfpathlineto{\pgfqpoint{3.164970in}{0.909076in}}%
\pgfpathlineto{\pgfqpoint{3.183301in}{0.908028in}}%
\pgfpathlineto{\pgfqpoint{3.201631in}{0.906540in}}%
\pgfpathlineto{\pgfqpoint{3.219961in}{0.906422in}}%
\pgfpathlineto{\pgfqpoint{3.256622in}{0.905791in}}%
\pgfpathlineto{\pgfqpoint{3.274952in}{0.905511in}}%
\pgfpathlineto{\pgfqpoint{3.293282in}{0.904871in}}%
\pgfpathlineto{\pgfqpoint{3.329943in}{0.904680in}}%
\pgfpathlineto{\pgfqpoint{3.348273in}{0.905664in}}%
\pgfpathlineto{\pgfqpoint{3.366604in}{0.905608in}}%
\pgfpathlineto{\pgfqpoint{3.384934in}{0.906042in}}%
\pgfpathlineto{\pgfqpoint{3.403264in}{0.905710in}}%
\pgfpathlineto{\pgfqpoint{3.421595in}{0.905022in}}%
\pgfpathlineto{\pgfqpoint{3.439925in}{0.904717in}}%
\pgfpathlineto{\pgfqpoint{3.531576in}{0.901530in}}%
\pgfpathlineto{\pgfqpoint{3.549907in}{0.901184in}}%
\pgfpathlineto{\pgfqpoint{3.586567in}{0.898502in}}%
\pgfpathlineto{\pgfqpoint{3.659888in}{0.897475in}}%
\pgfpathlineto{\pgfqpoint{3.678219in}{0.897718in}}%
\pgfpathlineto{\pgfqpoint{3.733210in}{0.897002in}}%
\pgfpathlineto{\pgfqpoint{3.788200in}{0.896202in}}%
\pgfpathlineto{\pgfqpoint{3.806531in}{0.896047in}}%
\pgfpathlineto{\pgfqpoint{3.824861in}{0.896168in}}%
\pgfpathlineto{\pgfqpoint{3.861522in}{0.895908in}}%
\pgfpathlineto{\pgfqpoint{3.879852in}{0.895980in}}%
\pgfpathlineto{\pgfqpoint{3.916513in}{0.895551in}}%
\pgfpathlineto{\pgfqpoint{3.989834in}{0.894952in}}%
\pgfpathlineto{\pgfqpoint{4.026494in}{0.895266in}}%
\pgfpathlineto{\pgfqpoint{4.044825in}{0.895414in}}%
\pgfpathlineto{\pgfqpoint{4.099816in}{0.894760in}}%
\pgfpathlineto{\pgfqpoint{4.118146in}{0.894675in}}%
\pgfpathlineto{\pgfqpoint{4.173137in}{0.893748in}}%
\pgfpathlineto{\pgfqpoint{4.228128in}{0.894122in}}%
\pgfpathlineto{\pgfqpoint{4.264788in}{0.893551in}}%
\pgfpathlineto{\pgfqpoint{4.301449in}{0.893251in}}%
\pgfpathlineto{\pgfqpoint{4.429761in}{0.892483in}}%
\pgfpathlineto{\pgfqpoint{4.448091in}{0.892252in}}%
\pgfpathlineto{\pgfqpoint{4.484752in}{0.891359in}}%
\pgfpathlineto{\pgfqpoint{4.539743in}{0.891011in}}%
\pgfpathlineto{\pgfqpoint{4.924679in}{0.889670in}}%
\pgfpathlineto{\pgfqpoint{5.034661in}{0.889701in}}%
\pgfpathlineto{\pgfqpoint{5.107982in}{0.889549in}}%
\pgfpathlineto{\pgfqpoint{5.254624in}{0.889629in}}%
\pgfpathlineto{\pgfqpoint{5.291285in}{0.889270in}}%
\pgfpathlineto{\pgfqpoint{5.511249in}{0.889532in}}%
\pgfpathlineto{\pgfqpoint{5.566240in}{0.889633in}}%
\pgfpathlineto{\pgfqpoint{5.639561in}{0.889479in}}%
\pgfpathlineto{\pgfqpoint{5.804533in}{0.889649in}}%
\pgfpathlineto{\pgfqpoint{5.841194in}{0.890017in}}%
\pgfpathlineto{\pgfqpoint{5.896185in}{0.890097in}}%
\pgfpathlineto{\pgfqpoint{5.932846in}{0.890270in}}%
\pgfpathlineto{\pgfqpoint{5.969506in}{0.890302in}}%
\pgfpathlineto{\pgfqpoint{5.987836in}{0.890589in}}%
\pgfpathlineto{\pgfqpoint{6.042827in}{0.890589in}}%
\pgfpathlineto{\pgfqpoint{6.042827in}{0.890589in}}%
\pgfusepath{stroke}%
\end{pgfscope}%
\begin{pgfscope}%
\pgfpathrectangle{\pgfqpoint{0.552903in}{0.522439in}}{\pgfqpoint{5.728219in}{2.785486in}}%
\pgfusepath{clip}%
\pgfsetrectcap%
\pgfsetroundjoin%
\pgfsetlinewidth{1.003750pt}%
\definecolor{currentstroke}{rgb}{0.000000,0.000000,0.000000}%
\pgfsetstrokecolor{currentstroke}%
\pgfsetdash{}{0pt}%
\pgfpathmoveto{\pgfqpoint{0.552903in}{0.888104in}}%
\pgfpathlineto{\pgfqpoint{6.281121in}{0.888104in}}%
\pgfusepath{stroke}%
\end{pgfscope}%
\begin{pgfscope}%
\pgfpathrectangle{\pgfqpoint{0.552903in}{0.522439in}}{\pgfqpoint{5.728219in}{2.785486in}}%
\pgfusepath{clip}%
\pgfsetbuttcap%
\pgfsetroundjoin%
\definecolor{currentfill}{rgb}{0.121569,0.466667,0.705882}%
\pgfsetfillcolor{currentfill}%
\pgfsetfillopacity{0.600000}%
\pgfsetlinewidth{0.000000pt}%
\definecolor{currentstroke}{rgb}{0.121569,0.466667,0.705882}%
\pgfsetstrokecolor{currentstroke}%
\pgfsetstrokeopacity{0.600000}%
\pgfsetdash{}{0pt}%
\pgfsys@defobject{currentmarker}{\pgfqpoint{-0.020833in}{-0.020833in}}{\pgfqpoint{0.020833in}{0.020833in}}{%
\pgfpathmoveto{\pgfqpoint{0.000000in}{-0.020833in}}%
\pgfpathcurveto{\pgfqpoint{0.005525in}{-0.020833in}}{\pgfqpoint{0.010825in}{-0.018638in}}{\pgfqpoint{0.014731in}{-0.014731in}}%
\pgfpathcurveto{\pgfqpoint{0.018638in}{-0.010825in}}{\pgfqpoint{0.020833in}{-0.005525in}}{\pgfqpoint{0.020833in}{0.000000in}}%
\pgfpathcurveto{\pgfqpoint{0.020833in}{0.005525in}}{\pgfqpoint{0.018638in}{0.010825in}}{\pgfqpoint{0.014731in}{0.014731in}}%
\pgfpathcurveto{\pgfqpoint{0.010825in}{0.018638in}}{\pgfqpoint{0.005525in}{0.020833in}}{\pgfqpoint{0.000000in}{0.020833in}}%
\pgfpathcurveto{\pgfqpoint{-0.005525in}{0.020833in}}{\pgfqpoint{-0.010825in}{0.018638in}}{\pgfqpoint{-0.014731in}{0.014731in}}%
\pgfpathcurveto{\pgfqpoint{-0.018638in}{0.010825in}}{\pgfqpoint{-0.020833in}{0.005525in}}{\pgfqpoint{-0.020833in}{0.000000in}}%
\pgfpathcurveto{\pgfqpoint{-0.020833in}{-0.005525in}}{\pgfqpoint{-0.018638in}{-0.010825in}}{\pgfqpoint{-0.014731in}{-0.014731in}}%
\pgfpathcurveto{\pgfqpoint{-0.010825in}{-0.018638in}}{\pgfqpoint{-0.005525in}{-0.020833in}}{\pgfqpoint{0.000000in}{-0.020833in}}%
\pgfpathlineto{\pgfqpoint{0.000000in}{-0.020833in}}%
\pgfpathclose%
\pgfusepath{fill}%
}%
\begin{pgfscope}%
\pgfsys@transformshift{1.529447in}{2.045821in}%
\pgfsys@useobject{currentmarker}{}%
\end{pgfscope}%
\begin{pgfscope}%
\pgfsys@transformshift{1.555174in}{2.252256in}%
\pgfsys@useobject{currentmarker}{}%
\end{pgfscope}%
\begin{pgfscope}%
\pgfsys@transformshift{1.580902in}{2.349282in}%
\pgfsys@useobject{currentmarker}{}%
\end{pgfscope}%
\begin{pgfscope}%
\pgfsys@transformshift{1.606630in}{2.488949in}%
\pgfsys@useobject{currentmarker}{}%
\end{pgfscope}%
\begin{pgfscope}%
\pgfsys@transformshift{1.632357in}{2.499948in}%
\pgfsys@useobject{currentmarker}{}%
\end{pgfscope}%
\begin{pgfscope}%
\pgfsys@transformshift{1.658085in}{2.425526in}%
\pgfsys@useobject{currentmarker}{}%
\end{pgfscope}%
\begin{pgfscope}%
\pgfsys@transformshift{1.683813in}{2.538199in}%
\pgfsys@useobject{currentmarker}{}%
\end{pgfscope}%
\begin{pgfscope}%
\pgfsys@transformshift{1.709540in}{2.912674in}%
\pgfsys@useobject{currentmarker}{}%
\end{pgfscope}%
\begin{pgfscope}%
\pgfsys@transformshift{1.735268in}{2.500038in}%
\pgfsys@useobject{currentmarker}{}%
\end{pgfscope}%
\begin{pgfscope}%
\pgfsys@transformshift{1.760995in}{2.865009in}%
\pgfsys@useobject{currentmarker}{}%
\end{pgfscope}%
\begin{pgfscope}%
\pgfsys@transformshift{1.786723in}{2.572226in}%
\pgfsys@useobject{currentmarker}{}%
\end{pgfscope}%
\begin{pgfscope}%
\pgfsys@transformshift{1.812451in}{2.839372in}%
\pgfsys@useobject{currentmarker}{}%
\end{pgfscope}%
\begin{pgfscope}%
\pgfsys@transformshift{1.838178in}{3.041011in}%
\pgfsys@useobject{currentmarker}{}%
\end{pgfscope}%
\begin{pgfscope}%
\pgfsys@transformshift{1.863906in}{2.905185in}%
\pgfsys@useobject{currentmarker}{}%
\end{pgfscope}%
\begin{pgfscope}%
\pgfsys@transformshift{1.889634in}{2.844486in}%
\pgfsys@useobject{currentmarker}{}%
\end{pgfscope}%
\begin{pgfscope}%
\pgfsys@transformshift{1.915361in}{2.892429in}%
\pgfsys@useobject{currentmarker}{}%
\end{pgfscope}%
\begin{pgfscope}%
\pgfsys@transformshift{1.941089in}{3.059807in}%
\pgfsys@useobject{currentmarker}{}%
\end{pgfscope}%
\begin{pgfscope}%
\pgfsys@transformshift{1.966816in}{3.107434in}%
\pgfsys@useobject{currentmarker}{}%
\end{pgfscope}%
\begin{pgfscope}%
\pgfsys@transformshift{1.992544in}{3.129715in}%
\pgfsys@useobject{currentmarker}{}%
\end{pgfscope}%
\begin{pgfscope}%
\pgfsys@transformshift{2.018272in}{2.919554in}%
\pgfsys@useobject{currentmarker}{}%
\end{pgfscope}%
\begin{pgfscope}%
\pgfsys@transformshift{2.043999in}{2.981874in}%
\pgfsys@useobject{currentmarker}{}%
\end{pgfscope}%
\begin{pgfscope}%
\pgfsys@transformshift{2.069727in}{2.893111in}%
\pgfsys@useobject{currentmarker}{}%
\end{pgfscope}%
\begin{pgfscope}%
\pgfsys@transformshift{2.095454in}{2.754323in}%
\pgfsys@useobject{currentmarker}{}%
\end{pgfscope}%
\begin{pgfscope}%
\pgfsys@transformshift{2.121182in}{2.838068in}%
\pgfsys@useobject{currentmarker}{}%
\end{pgfscope}%
\begin{pgfscope}%
\pgfsys@transformshift{2.146910in}{3.181313in}%
\pgfsys@useobject{currentmarker}{}%
\end{pgfscope}%
\begin{pgfscope}%
\pgfsys@transformshift{2.172637in}{2.911011in}%
\pgfsys@useobject{currentmarker}{}%
\end{pgfscope}%
\begin{pgfscope}%
\pgfsys@transformshift{2.198365in}{2.616940in}%
\pgfsys@useobject{currentmarker}{}%
\end{pgfscope}%
\begin{pgfscope}%
\pgfsys@transformshift{2.224093in}{2.700094in}%
\pgfsys@useobject{currentmarker}{}%
\end{pgfscope}%
\begin{pgfscope}%
\pgfsys@transformshift{2.249820in}{2.656708in}%
\pgfsys@useobject{currentmarker}{}%
\end{pgfscope}%
\begin{pgfscope}%
\pgfsys@transformshift{2.275548in}{2.865694in}%
\pgfsys@useobject{currentmarker}{}%
\end{pgfscope}%
\begin{pgfscope}%
\pgfsys@transformshift{2.301275in}{2.909518in}%
\pgfsys@useobject{currentmarker}{}%
\end{pgfscope}%
\begin{pgfscope}%
\pgfsys@transformshift{2.327003in}{2.700180in}%
\pgfsys@useobject{currentmarker}{}%
\end{pgfscope}%
\begin{pgfscope}%
\pgfsys@transformshift{2.352731in}{2.833419in}%
\pgfsys@useobject{currentmarker}{}%
\end{pgfscope}%
\begin{pgfscope}%
\pgfsys@transformshift{2.378458in}{2.699667in}%
\pgfsys@useobject{currentmarker}{}%
\end{pgfscope}%
\begin{pgfscope}%
\pgfsys@transformshift{2.404186in}{2.487387in}%
\pgfsys@useobject{currentmarker}{}%
\end{pgfscope}%
\begin{pgfscope}%
\pgfsys@transformshift{2.429913in}{2.566527in}%
\pgfsys@useobject{currentmarker}{}%
\end{pgfscope}%
\begin{pgfscope}%
\pgfsys@transformshift{2.455641in}{2.055424in}%
\pgfsys@useobject{currentmarker}{}%
\end{pgfscope}%
\begin{pgfscope}%
\pgfsys@transformshift{2.481369in}{2.150280in}%
\pgfsys@useobject{currentmarker}{}%
\end{pgfscope}%
\begin{pgfscope}%
\pgfsys@transformshift{2.507096in}{2.517765in}%
\pgfsys@useobject{currentmarker}{}%
\end{pgfscope}%
\begin{pgfscope}%
\pgfsys@transformshift{2.532824in}{2.708042in}%
\pgfsys@useobject{currentmarker}{}%
\end{pgfscope}%
\begin{pgfscope}%
\pgfsys@transformshift{2.558552in}{2.410446in}%
\pgfsys@useobject{currentmarker}{}%
\end{pgfscope}%
\begin{pgfscope}%
\pgfsys@transformshift{2.584279in}{2.173230in}%
\pgfsys@useobject{currentmarker}{}%
\end{pgfscope}%
\begin{pgfscope}%
\pgfsys@transformshift{2.610007in}{2.022470in}%
\pgfsys@useobject{currentmarker}{}%
\end{pgfscope}%
\begin{pgfscope}%
\pgfsys@transformshift{2.635734in}{1.741474in}%
\pgfsys@useobject{currentmarker}{}%
\end{pgfscope}%
\begin{pgfscope}%
\pgfsys@transformshift{2.661462in}{1.972332in}%
\pgfsys@useobject{currentmarker}{}%
\end{pgfscope}%
\begin{pgfscope}%
\pgfsys@transformshift{2.687190in}{1.892777in}%
\pgfsys@useobject{currentmarker}{}%
\end{pgfscope}%
\begin{pgfscope}%
\pgfsys@transformshift{2.712917in}{1.686416in}%
\pgfsys@useobject{currentmarker}{}%
\end{pgfscope}%
\begin{pgfscope}%
\pgfsys@transformshift{2.738645in}{1.792405in}%
\pgfsys@useobject{currentmarker}{}%
\end{pgfscope}%
\begin{pgfscope}%
\pgfsys@transformshift{2.764373in}{1.959948in}%
\pgfsys@useobject{currentmarker}{}%
\end{pgfscope}%
\begin{pgfscope}%
\pgfsys@transformshift{2.790100in}{1.827511in}%
\pgfsys@useobject{currentmarker}{}%
\end{pgfscope}%
\begin{pgfscope}%
\pgfsys@transformshift{2.815828in}{1.639653in}%
\pgfsys@useobject{currentmarker}{}%
\end{pgfscope}%
\begin{pgfscope}%
\pgfsys@transformshift{2.841555in}{1.308858in}%
\pgfsys@useobject{currentmarker}{}%
\end{pgfscope}%
\begin{pgfscope}%
\pgfsys@transformshift{2.867283in}{1.667118in}%
\pgfsys@useobject{currentmarker}{}%
\end{pgfscope}%
\begin{pgfscope}%
\pgfsys@transformshift{2.893011in}{1.429541in}%
\pgfsys@useobject{currentmarker}{}%
\end{pgfscope}%
\begin{pgfscope}%
\pgfsys@transformshift{2.918738in}{1.399759in}%
\pgfsys@useobject{currentmarker}{}%
\end{pgfscope}%
\begin{pgfscope}%
\pgfsys@transformshift{2.944466in}{1.126047in}%
\pgfsys@useobject{currentmarker}{}%
\end{pgfscope}%
\begin{pgfscope}%
\pgfsys@transformshift{2.970193in}{1.182455in}%
\pgfsys@useobject{currentmarker}{}%
\end{pgfscope}%
\begin{pgfscope}%
\pgfsys@transformshift{2.995921in}{1.430534in}%
\pgfsys@useobject{currentmarker}{}%
\end{pgfscope}%
\begin{pgfscope}%
\pgfsys@transformshift{3.021649in}{1.305813in}%
\pgfsys@useobject{currentmarker}{}%
\end{pgfscope}%
\begin{pgfscope}%
\pgfsys@transformshift{3.047376in}{1.359893in}%
\pgfsys@useobject{currentmarker}{}%
\end{pgfscope}%
\begin{pgfscope}%
\pgfsys@transformshift{3.073104in}{1.229345in}%
\pgfsys@useobject{currentmarker}{}%
\end{pgfscope}%
\begin{pgfscope}%
\pgfsys@transformshift{3.098832in}{1.117187in}%
\pgfsys@useobject{currentmarker}{}%
\end{pgfscope}%
\begin{pgfscope}%
\pgfsys@transformshift{3.124559in}{1.539769in}%
\pgfsys@useobject{currentmarker}{}%
\end{pgfscope}%
\begin{pgfscope}%
\pgfsys@transformshift{3.150287in}{1.436005in}%
\pgfsys@useobject{currentmarker}{}%
\end{pgfscope}%
\begin{pgfscope}%
\pgfsys@transformshift{3.176014in}{1.191317in}%
\pgfsys@useobject{currentmarker}{}%
\end{pgfscope}%
\begin{pgfscope}%
\pgfsys@transformshift{3.201742in}{1.164530in}%
\pgfsys@useobject{currentmarker}{}%
\end{pgfscope}%
\begin{pgfscope}%
\pgfsys@transformshift{3.227470in}{1.184781in}%
\pgfsys@useobject{currentmarker}{}%
\end{pgfscope}%
\begin{pgfscope}%
\pgfsys@transformshift{3.253197in}{1.351235in}%
\pgfsys@useobject{currentmarker}{}%
\end{pgfscope}%
\begin{pgfscope}%
\pgfsys@transformshift{3.278925in}{1.262235in}%
\pgfsys@useobject{currentmarker}{}%
\end{pgfscope}%
\begin{pgfscope}%
\pgfsys@transformshift{3.304653in}{0.950762in}%
\pgfsys@useobject{currentmarker}{}%
\end{pgfscope}%
\begin{pgfscope}%
\pgfsys@transformshift{3.330380in}{1.204326in}%
\pgfsys@useobject{currentmarker}{}%
\end{pgfscope}%
\begin{pgfscope}%
\pgfsys@transformshift{3.356108in}{1.174978in}%
\pgfsys@useobject{currentmarker}{}%
\end{pgfscope}%
\begin{pgfscope}%
\pgfsys@transformshift{3.381835in}{1.346370in}%
\pgfsys@useobject{currentmarker}{}%
\end{pgfscope}%
\begin{pgfscope}%
\pgfsys@transformshift{3.407563in}{1.136291in}%
\pgfsys@useobject{currentmarker}{}%
\end{pgfscope}%
\begin{pgfscope}%
\pgfsys@transformshift{3.433291in}{0.963504in}%
\pgfsys@useobject{currentmarker}{}%
\end{pgfscope}%
\begin{pgfscope}%
\pgfsys@transformshift{3.459018in}{0.961397in}%
\pgfsys@useobject{currentmarker}{}%
\end{pgfscope}%
\begin{pgfscope}%
\pgfsys@transformshift{3.484746in}{0.970959in}%
\pgfsys@useobject{currentmarker}{}%
\end{pgfscope}%
\begin{pgfscope}%
\pgfsys@transformshift{3.510473in}{1.009575in}%
\pgfsys@useobject{currentmarker}{}%
\end{pgfscope}%
\begin{pgfscope}%
\pgfsys@transformshift{3.536201in}{1.071635in}%
\pgfsys@useobject{currentmarker}{}%
\end{pgfscope}%
\begin{pgfscope}%
\pgfsys@transformshift{3.561929in}{1.039655in}%
\pgfsys@useobject{currentmarker}{}%
\end{pgfscope}%
\begin{pgfscope}%
\pgfsys@transformshift{3.587656in}{0.868036in}%
\pgfsys@useobject{currentmarker}{}%
\end{pgfscope}%
\begin{pgfscope}%
\pgfsys@transformshift{3.613384in}{0.928107in}%
\pgfsys@useobject{currentmarker}{}%
\end{pgfscope}%
\begin{pgfscope}%
\pgfsys@transformshift{3.639112in}{0.769280in}%
\pgfsys@useobject{currentmarker}{}%
\end{pgfscope}%
\begin{pgfscope}%
\pgfsys@transformshift{3.664839in}{0.959837in}%
\pgfsys@useobject{currentmarker}{}%
\end{pgfscope}%
\begin{pgfscope}%
\pgfsys@transformshift{3.690567in}{0.830084in}%
\pgfsys@useobject{currentmarker}{}%
\end{pgfscope}%
\begin{pgfscope}%
\pgfsys@transformshift{3.716294in}{0.890513in}%
\pgfsys@useobject{currentmarker}{}%
\end{pgfscope}%
\begin{pgfscope}%
\pgfsys@transformshift{3.742022in}{0.855083in}%
\pgfsys@useobject{currentmarker}{}%
\end{pgfscope}%
\begin{pgfscope}%
\pgfsys@transformshift{3.767750in}{1.163515in}%
\pgfsys@useobject{currentmarker}{}%
\end{pgfscope}%
\begin{pgfscope}%
\pgfsys@transformshift{3.793477in}{1.145366in}%
\pgfsys@useobject{currentmarker}{}%
\end{pgfscope}%
\begin{pgfscope}%
\pgfsys@transformshift{3.819205in}{1.244720in}%
\pgfsys@useobject{currentmarker}{}%
\end{pgfscope}%
\begin{pgfscope}%
\pgfsys@transformshift{3.844933in}{0.897409in}%
\pgfsys@useobject{currentmarker}{}%
\end{pgfscope}%
\begin{pgfscope}%
\pgfsys@transformshift{3.870660in}{0.976929in}%
\pgfsys@useobject{currentmarker}{}%
\end{pgfscope}%
\begin{pgfscope}%
\pgfsys@transformshift{3.896388in}{0.691537in}%
\pgfsys@useobject{currentmarker}{}%
\end{pgfscope}%
\begin{pgfscope}%
\pgfsys@transformshift{3.922115in}{0.839267in}%
\pgfsys@useobject{currentmarker}{}%
\end{pgfscope}%
\begin{pgfscope}%
\pgfsys@transformshift{3.947843in}{0.923649in}%
\pgfsys@useobject{currentmarker}{}%
\end{pgfscope}%
\begin{pgfscope}%
\pgfsys@transformshift{3.973571in}{0.839738in}%
\pgfsys@useobject{currentmarker}{}%
\end{pgfscope}%
\begin{pgfscope}%
\pgfsys@transformshift{3.999298in}{0.649052in}%
\pgfsys@useobject{currentmarker}{}%
\end{pgfscope}%
\begin{pgfscope}%
\pgfsys@transformshift{4.025026in}{0.821077in}%
\pgfsys@useobject{currentmarker}{}%
\end{pgfscope}%
\begin{pgfscope}%
\pgfsys@transformshift{4.050753in}{0.783748in}%
\pgfsys@useobject{currentmarker}{}%
\end{pgfscope}%
\begin{pgfscope}%
\pgfsys@transformshift{4.076481in}{0.738004in}%
\pgfsys@useobject{currentmarker}{}%
\end{pgfscope}%
\end{pgfscope}%
\begin{pgfscope}%
\pgfpathrectangle{\pgfqpoint{0.552903in}{0.522439in}}{\pgfqpoint{5.728219in}{2.785486in}}%
\pgfusepath{clip}%
\pgfsetrectcap%
\pgfsetroundjoin%
\pgfsetlinewidth{1.505625pt}%
\definecolor{currentstroke}{rgb}{0.121569,0.466667,0.705882}%
\pgfsetstrokecolor{currentstroke}%
\pgfsetdash{}{0pt}%
\pgfpathmoveto{\pgfqpoint{1.529447in}{2.095936in}}%
\pgfpathlineto{\pgfqpoint{1.555174in}{2.182480in}}%
\pgfpathlineto{\pgfqpoint{1.580902in}{2.267505in}}%
\pgfpathlineto{\pgfqpoint{1.606630in}{2.350336in}}%
\pgfpathlineto{\pgfqpoint{1.632357in}{2.430333in}}%
\pgfpathlineto{\pgfqpoint{1.658085in}{2.506899in}}%
\pgfpathlineto{\pgfqpoint{1.683813in}{2.579484in}}%
\pgfpathlineto{\pgfqpoint{1.709540in}{2.647594in}}%
\pgfpathlineto{\pgfqpoint{1.735268in}{2.710791in}}%
\pgfpathlineto{\pgfqpoint{1.760995in}{2.768702in}}%
\pgfpathlineto{\pgfqpoint{1.786723in}{2.821012in}}%
\pgfpathlineto{\pgfqpoint{1.812451in}{2.867475in}}%
\pgfpathlineto{\pgfqpoint{1.838178in}{2.907904in}}%
\pgfpathlineto{\pgfqpoint{1.863906in}{2.942176in}}%
\pgfpathlineto{\pgfqpoint{1.889634in}{2.970226in}}%
\pgfpathlineto{\pgfqpoint{1.915361in}{2.992048in}}%
\pgfpathlineto{\pgfqpoint{1.941089in}{3.007688in}}%
\pgfpathlineto{\pgfqpoint{1.966816in}{3.017240in}}%
\pgfpathlineto{\pgfqpoint{1.992544in}{3.020847in}}%
\pgfpathlineto{\pgfqpoint{2.018272in}{3.018689in}}%
\pgfpathlineto{\pgfqpoint{2.043999in}{3.010983in}}%
\pgfpathlineto{\pgfqpoint{2.069727in}{2.997979in}}%
\pgfpathlineto{\pgfqpoint{2.095454in}{2.979951in}}%
\pgfpathlineto{\pgfqpoint{2.121182in}{2.957197in}}%
\pgfpathlineto{\pgfqpoint{2.146910in}{2.930030in}}%
\pgfpathlineto{\pgfqpoint{2.172637in}{2.898779in}}%
\pgfpathlineto{\pgfqpoint{2.198365in}{2.863781in}}%
\pgfpathlineto{\pgfqpoint{2.224093in}{2.825376in}}%
\pgfpathlineto{\pgfqpoint{2.249820in}{2.783909in}}%
\pgfpathlineto{\pgfqpoint{2.275548in}{2.739721in}}%
\pgfpathlineto{\pgfqpoint{2.301275in}{2.693149in}}%
\pgfpathlineto{\pgfqpoint{2.327003in}{2.644526in}}%
\pgfpathlineto{\pgfqpoint{2.352731in}{2.594170in}}%
\pgfpathlineto{\pgfqpoint{2.378458in}{2.542393in}}%
\pgfpathlineto{\pgfqpoint{2.404186in}{2.489492in}}%
\pgfpathlineto{\pgfqpoint{2.429913in}{2.435748in}}%
\pgfpathlineto{\pgfqpoint{2.455641in}{2.381431in}}%
\pgfpathlineto{\pgfqpoint{2.481369in}{2.326790in}}%
\pgfpathlineto{\pgfqpoint{2.507096in}{2.272061in}}%
\pgfpathlineto{\pgfqpoint{2.532824in}{2.217461in}}%
\pgfpathlineto{\pgfqpoint{2.558552in}{2.163191in}}%
\pgfpathlineto{\pgfqpoint{2.584279in}{2.109433in}}%
\pgfpathlineto{\pgfqpoint{2.610007in}{2.056353in}}%
\pgfpathlineto{\pgfqpoint{2.635734in}{2.004101in}}%
\pgfpathlineto{\pgfqpoint{2.661462in}{1.952809in}}%
\pgfpathlineto{\pgfqpoint{2.687190in}{1.902595in}}%
\pgfpathlineto{\pgfqpoint{2.712917in}{1.853560in}}%
\pgfpathlineto{\pgfqpoint{2.738645in}{1.805792in}}%
\pgfpathlineto{\pgfqpoint{2.764373in}{1.759365in}}%
\pgfpathlineto{\pgfqpoint{2.790100in}{1.714338in}}%
\pgfpathlineto{\pgfqpoint{2.815828in}{1.670762in}}%
\pgfpathlineto{\pgfqpoint{2.841555in}{1.628673in}}%
\pgfpathlineto{\pgfqpoint{2.867283in}{1.588098in}}%
\pgfpathlineto{\pgfqpoint{2.893011in}{1.549053in}}%
\pgfpathlineto{\pgfqpoint{2.918738in}{1.511548in}}%
\pgfpathlineto{\pgfqpoint{2.944466in}{1.475583in}}%
\pgfpathlineto{\pgfqpoint{2.970193in}{1.441150in}}%
\pgfpathlineto{\pgfqpoint{2.995921in}{1.408236in}}%
\pgfpathlineto{\pgfqpoint{3.021649in}{1.376822in}}%
\pgfpathlineto{\pgfqpoint{3.047376in}{1.346882in}}%
\pgfpathlineto{\pgfqpoint{3.073104in}{1.318389in}}%
\pgfpathlineto{\pgfqpoint{3.098832in}{1.291310in}}%
\pgfpathlineto{\pgfqpoint{3.124559in}{1.265608in}}%
\pgfpathlineto{\pgfqpoint{3.150287in}{1.241246in}}%
\pgfpathlineto{\pgfqpoint{3.176014in}{1.218181in}}%
\pgfpathlineto{\pgfqpoint{3.201742in}{1.196371in}}%
\pgfpathlineto{\pgfqpoint{3.227470in}{1.175772in}}%
\pgfpathlineto{\pgfqpoint{3.253197in}{1.156340in}}%
\pgfpathlineto{\pgfqpoint{3.278925in}{1.138028in}}%
\pgfpathlineto{\pgfqpoint{3.304653in}{1.120790in}}%
\pgfpathlineto{\pgfqpoint{3.330380in}{1.104582in}}%
\pgfpathlineto{\pgfqpoint{3.356108in}{1.089356in}}%
\pgfpathlineto{\pgfqpoint{3.381835in}{1.075067in}}%
\pgfpathlineto{\pgfqpoint{3.407563in}{1.061671in}}%
\pgfpathlineto{\pgfqpoint{3.433291in}{1.049125in}}%
\pgfpathlineto{\pgfqpoint{3.459018in}{1.037384in}}%
\pgfpathlineto{\pgfqpoint{3.484746in}{1.026408in}}%
\pgfpathlineto{\pgfqpoint{3.510473in}{1.016155in}}%
\pgfpathlineto{\pgfqpoint{3.536201in}{1.006586in}}%
\pgfpathlineto{\pgfqpoint{3.561929in}{0.997663in}}%
\pgfpathlineto{\pgfqpoint{3.587656in}{0.989350in}}%
\pgfpathlineto{\pgfqpoint{3.613384in}{0.981610in}}%
\pgfpathlineto{\pgfqpoint{3.639112in}{0.974411in}}%
\pgfpathlineto{\pgfqpoint{3.664839in}{0.967719in}}%
\pgfpathlineto{\pgfqpoint{3.690567in}{0.961503in}}%
\pgfpathlineto{\pgfqpoint{3.716294in}{0.955734in}}%
\pgfpathlineto{\pgfqpoint{3.742022in}{0.950384in}}%
\pgfpathlineto{\pgfqpoint{3.767750in}{0.945425in}}%
\pgfpathlineto{\pgfqpoint{3.793477in}{0.940832in}}%
\pgfpathlineto{\pgfqpoint{3.819205in}{0.936582in}}%
\pgfpathlineto{\pgfqpoint{3.844933in}{0.932651in}}%
\pgfpathlineto{\pgfqpoint{3.870660in}{0.929017in}}%
\pgfpathlineto{\pgfqpoint{3.896388in}{0.925661in}}%
\pgfpathlineto{\pgfqpoint{3.922115in}{0.922562in}}%
\pgfpathlineto{\pgfqpoint{3.947843in}{0.919704in}}%
\pgfpathlineto{\pgfqpoint{3.973571in}{0.917069in}}%
\pgfpathlineto{\pgfqpoint{3.999298in}{0.914641in}}%
\pgfpathlineto{\pgfqpoint{4.025026in}{0.912404in}}%
\pgfpathlineto{\pgfqpoint{4.050753in}{0.910346in}}%
\pgfpathlineto{\pgfqpoint{4.076481in}{0.908453in}}%
\pgfusepath{stroke}%
\end{pgfscope}%
\begin{pgfscope}%
\pgfsetrectcap%
\pgfsetmiterjoin%
\pgfsetlinewidth{0.803000pt}%
\definecolor{currentstroke}{rgb}{0.000000,0.000000,0.000000}%
\pgfsetstrokecolor{currentstroke}%
\pgfsetdash{}{0pt}%
\pgfpathmoveto{\pgfqpoint{0.552903in}{0.522439in}}%
\pgfpathlineto{\pgfqpoint{0.552903in}{3.307926in}}%
\pgfusepath{stroke}%
\end{pgfscope}%
\begin{pgfscope}%
\pgfsetrectcap%
\pgfsetmiterjoin%
\pgfsetlinewidth{0.803000pt}%
\definecolor{currentstroke}{rgb}{0.000000,0.000000,0.000000}%
\pgfsetstrokecolor{currentstroke}%
\pgfsetdash{}{0pt}%
\pgfpathmoveto{\pgfqpoint{6.281121in}{0.522439in}}%
\pgfpathlineto{\pgfqpoint{6.281121in}{3.307926in}}%
\pgfusepath{stroke}%
\end{pgfscope}%
\begin{pgfscope}%
\pgfsetrectcap%
\pgfsetmiterjoin%
\pgfsetlinewidth{0.803000pt}%
\definecolor{currentstroke}{rgb}{0.000000,0.000000,0.000000}%
\pgfsetstrokecolor{currentstroke}%
\pgfsetdash{}{0pt}%
\pgfpathmoveto{\pgfqpoint{0.552903in}{0.522439in}}%
\pgfpathlineto{\pgfqpoint{6.281121in}{0.522439in}}%
\pgfusepath{stroke}%
\end{pgfscope}%
\begin{pgfscope}%
\pgfsetrectcap%
\pgfsetmiterjoin%
\pgfsetlinewidth{0.803000pt}%
\definecolor{currentstroke}{rgb}{0.000000,0.000000,0.000000}%
\pgfsetstrokecolor{currentstroke}%
\pgfsetdash{}{0pt}%
\pgfpathmoveto{\pgfqpoint{0.552903in}{3.307926in}}%
\pgfpathlineto{\pgfqpoint{6.281121in}{3.307926in}}%
\pgfusepath{stroke}%
\end{pgfscope}%
\begin{pgfscope}%
\pgfsetbuttcap%
\pgfsetmiterjoin%
\definecolor{currentfill}{rgb}{1.000000,1.000000,1.000000}%
\pgfsetfillcolor{currentfill}%
\pgfsetfillopacity{0.800000}%
\pgfsetlinewidth{1.003750pt}%
\definecolor{currentstroke}{rgb}{0.800000,0.800000,0.800000}%
\pgfsetstrokecolor{currentstroke}%
\pgfsetstrokeopacity{0.800000}%
\pgfsetdash{}{0pt}%
\pgfpathmoveto{\pgfqpoint{3.061868in}{2.077314in}}%
\pgfpathlineto{\pgfqpoint{6.183899in}{2.077314in}}%
\pgfpathquadraticcurveto{\pgfqpoint{6.211677in}{2.077314in}}{\pgfqpoint{6.211677in}{2.105092in}}%
\pgfpathlineto{\pgfqpoint{6.211677in}{3.210704in}}%
\pgfpathquadraticcurveto{\pgfqpoint{6.211677in}{3.238481in}}{\pgfqpoint{6.183899in}{3.238481in}}%
\pgfpathlineto{\pgfqpoint{3.061868in}{3.238481in}}%
\pgfpathquadraticcurveto{\pgfqpoint{3.034090in}{3.238481in}}{\pgfqpoint{3.034090in}{3.210704in}}%
\pgfpathlineto{\pgfqpoint{3.034090in}{2.105092in}}%
\pgfpathquadraticcurveto{\pgfqpoint{3.034090in}{2.077314in}}{\pgfqpoint{3.061868in}{2.077314in}}%
\pgfpathlineto{\pgfqpoint{3.061868in}{2.077314in}}%
\pgfpathclose%
\pgfusepath{stroke,fill}%
\end{pgfscope}%
\begin{pgfscope}%
\pgfsetrectcap%
\pgfsetroundjoin%
\pgfsetlinewidth{1.505625pt}%
\definecolor{currentstroke}{rgb}{1.000000,0.498039,0.054902}%
\pgfsetstrokecolor{currentstroke}%
\pgfsetdash{}{0pt}%
\pgfpathmoveto{\pgfqpoint{3.089646in}{3.042540in}}%
\pgfpathlineto{\pgfqpoint{3.228534in}{3.042540in}}%
\pgfpathlineto{\pgfqpoint{3.367423in}{3.042540in}}%
\pgfusepath{stroke}%
\end{pgfscope}%
\begin{pgfscope}%
\definecolor{textcolor}{rgb}{0.000000,0.000000,0.000000}%
\pgfsetstrokecolor{textcolor}%
\pgfsetfillcolor{textcolor}%
\pgftext[x=3.478534in, y=3.087270in, left, base]{\color{textcolor}\rmfamily\fontsize{10.000000}{12.000000}\selectfont Betragsquadrat der Fourier-Transformierten}%
\end{pgfscope}%
\begin{pgfscope}%
\definecolor{textcolor}{rgb}{0.000000,0.000000,0.000000}%
\pgfsetstrokecolor{textcolor}%
\pgfsetfillcolor{textcolor}%
\pgftext[x=3.478534in, y=2.935301in, left, base]{\color{textcolor}\rmfamily\fontsize{10.000000}{12.000000}\selectfont  vom MFM-Scan der Probe (Abb. \ref{fig:mfm-amplitude-ft}d)}%
\end{pgfscope}%
\begin{pgfscope}%
\pgfsetbuttcap%
\pgfsetroundjoin%
\definecolor{currentfill}{rgb}{0.121569,0.466667,0.705882}%
\pgfsetfillcolor{currentfill}%
\pgfsetfillopacity{0.600000}%
\pgfsetlinewidth{0.000000pt}%
\definecolor{currentstroke}{rgb}{0.121569,0.466667,0.705882}%
\pgfsetstrokecolor{currentstroke}%
\pgfsetstrokeopacity{0.600000}%
\pgfsetdash{}{0pt}%
\pgfsys@defobject{currentmarker}{\pgfqpoint{-0.020833in}{-0.020833in}}{\pgfqpoint{0.020833in}{0.020833in}}{%
\pgfpathmoveto{\pgfqpoint{0.000000in}{-0.020833in}}%
\pgfpathcurveto{\pgfqpoint{0.005525in}{-0.020833in}}{\pgfqpoint{0.010825in}{-0.018638in}}{\pgfqpoint{0.014731in}{-0.014731in}}%
\pgfpathcurveto{\pgfqpoint{0.018638in}{-0.010825in}}{\pgfqpoint{0.020833in}{-0.005525in}}{\pgfqpoint{0.020833in}{0.000000in}}%
\pgfpathcurveto{\pgfqpoint{0.020833in}{0.005525in}}{\pgfqpoint{0.018638in}{0.010825in}}{\pgfqpoint{0.014731in}{0.014731in}}%
\pgfpathcurveto{\pgfqpoint{0.010825in}{0.018638in}}{\pgfqpoint{0.005525in}{0.020833in}}{\pgfqpoint{0.000000in}{0.020833in}}%
\pgfpathcurveto{\pgfqpoint{-0.005525in}{0.020833in}}{\pgfqpoint{-0.010825in}{0.018638in}}{\pgfqpoint{-0.014731in}{0.014731in}}%
\pgfpathcurveto{\pgfqpoint{-0.018638in}{0.010825in}}{\pgfqpoint{-0.020833in}{0.005525in}}{\pgfqpoint{-0.020833in}{0.000000in}}%
\pgfpathcurveto{\pgfqpoint{-0.020833in}{-0.005525in}}{\pgfqpoint{-0.018638in}{-0.010825in}}{\pgfqpoint{-0.014731in}{-0.014731in}}%
\pgfpathcurveto{\pgfqpoint{-0.010825in}{-0.018638in}}{\pgfqpoint{-0.005525in}{-0.020833in}}{\pgfqpoint{0.000000in}{-0.020833in}}%
\pgfpathlineto{\pgfqpoint{0.000000in}{-0.020833in}}%
\pgfpathclose%
\pgfusepath{fill}%
}%
\begin{pgfscope}%
\pgfsys@transformshift{3.228534in}{2.690757in}%
\pgfsys@useobject{currentmarker}{}%
\end{pgfscope}%
\end{pgfscope}%
\begin{pgfscope}%
\definecolor{textcolor}{rgb}{0.000000,0.000000,0.000000}%
\pgfsetstrokecolor{textcolor}%
\pgfsetfillcolor{textcolor}%
\pgftext[x=3.478534in, y=2.735487in, left, base]{\color{textcolor}\rmfamily\fontsize{10.000000}{12.000000}\selectfont ermitteltes Signal}%
\end{pgfscope}%
\begin{pgfscope}%
\definecolor{textcolor}{rgb}{0.000000,0.000000,0.000000}%
\pgfsetstrokecolor{textcolor}%
\pgfsetfillcolor{textcolor}%
\pgftext[x=3.478534in, y=2.583518in, left, base]{\color{textcolor}\rmfamily\fontsize{10.000000}{12.000000}\selectfont  mit maskiertem Direktstrahl (Abb. \ref{fig:th-100-200-maske-radial-transform}b)}%
\end{pgfscope}%
\begin{pgfscope}%
\pgfsetrectcap%
\pgfsetroundjoin%
\pgfsetlinewidth{1.505625pt}%
\definecolor{currentstroke}{rgb}{0.121569,0.466667,0.705882}%
\pgfsetstrokecolor{currentstroke}%
\pgfsetdash{}{0pt}%
\pgfpathmoveto{\pgfqpoint{3.089646in}{2.306898in}}%
\pgfpathlineto{\pgfqpoint{3.367423in}{2.306898in}}%
\pgfusepath{stroke}%
\end{pgfscope}%
\begin{pgfscope}%
\definecolor{textcolor}{rgb}{0.000000,0.000000,0.000000}%
\pgfsetstrokecolor{textcolor}%
\pgfsetfillcolor{textcolor}%
\pgftext[x=3.478534in, y=2.375193in, left, base]{\color{textcolor}\rmfamily\fontsize{10.000000}{12.000000}\selectfont Fit \(\displaystyle g(q)\) vom Signal}%
\end{pgfscope}%
\begin{pgfscope}%
\definecolor{textcolor}{rgb}{0.000000,0.000000,0.000000}%
\pgfsetstrokecolor{textcolor}%
\pgfsetfillcolor{textcolor}%
\pgftext[x=3.478534in, y=2.199664in, left, base]{\color{textcolor}\rmfamily\fontsize{10.000000}{12.000000}\selectfont mit \(\displaystyle q_\text{max}\) = \SI{18(2)}{\per\micro\meter}}%
\end{pgfscope}%
\end{pgfpicture}%
\makeatother%
\endgroup%

    \caption{azimutal (hellblaue Punkte) aufintegrierte Streuintensität und (orange Linie) die Fourier-Transformierte der magnetischen Struktur der Probe (Abb. \ref{fig:mfm-amplitude-ft}d) mit dem Maximum an der $q =\SI{18(1)}{\per\micro\meter}$. Radiale Streuintensität wird mit Fit (blaue Linie) $g(q)$ mit den Parametern $\beta = \SI{9.5(5)}{}$, $\alpha = \SI{1.93(9)}{}$ und $A = \SI{39.4(8)}{}$ angepasst. Das Maximum vom Fit liegt bei $q_\text{max} = \SI{18(2)}{\per\micro\meter}$.}
    \label{fig:radius_fit}
\end{figure}
\noindent
Die gemessene Intensitätsvertilung hat eine größere radiale Verbreitung des Streuringes, als die Fourier-Transformierte von MFM-Scan der Probe. Dieses Phänomen liegt an der endlichen Breite des Strahlprofils. Das Maximum der radialen Streuintensität von Fourier-Transformierten liegen innerhalb des Fit-Vertrauensbereichs.

\noindent
Der mittlere Umfang des Streurings beträgt ca. $2\pi\cdot\SI{75}{\px}$. So wird das \gls{snr} mit der azimutalen Integration von 1 bis ca. 22 nach Gl. (\ref{eq:snr_pixel}) erhöht.

\subsection{Nicht-resonante Messung}
Es wird eine Kontrollmessung an einer nichtresonanten Photonenenergie $h\nu_{\text{Gd, Off-Res}} \approx \SI{1163}{\eV}$ durchgeführt, um zu beweisen, dass ein resonanter magnetischer Effekt der beobachtete Intensitätsverteilung zugrunde liegt.

\noindent
Es werden ebenfalls \SI{50000}{\captures} aufgenommen und mit dem Schwellenwert-Algorithmus ausgewertet, wobei die Scwellenwerte $s_V$ im Intervall von \SIrange{50}{180}{\adu} variiert werden, damit sich die Ergebnisse mit den Eregnissen an der resonanten Photonenenergie (Abb. \ref{fig:th_50_100_125_150_170_180_200_220_260}) vergleichen lassen.
\begin{figure}[H]
    \centering
    \input{images/auswertung/th_50_100_125_150_170_180_off_resonance.pgf}
    \caption{Anzahl von den detektierten Photonen mithilfe des Schwellenwert-Algorithmuses mit dem Schwellenwert $s_V$ (a) \SI{50}{\adu}, (b) \SI{100}{\adu}, (c) \SI{125}{\adu}, (d) \SI{150}{\adu}, (e) \SI{170}{\adu} und (f) \SI{180}{\adu}. Aufsummiert werden jeweils \num{50000} Aufnahmen. Aufgenommen an der Photonenenergie $h\nu_\text{Gd, Off-Res} \approx \SI{1163}{\eV}$.}
    \label{fig:th_50_100_125_150_170_180_off_resonance}
\end{figure}
\noindent
Bis zu einem Schwellenwert von \SI{125}{\adu} wird eine fast homogene Intensitätsverteilung beobachtet, die auf die Probetopographie (Rauligkeit znd Kristalline von Dünnschichten und Substrat) sowie fehldetektierte Photon zurückzuführen ist. Oberhalb dieses Schwellelnwertes wird kaum noch Intensität außerhalb des Direktstrahls detektiert. Ein Streuring ist bei keinem Schwellenwert erkennbar.

\section{Auswertung mit Clustering-Algorithmus}
Die Analyse der Punkspreizfunktion eines isolierten Photons (s. Abschnitt \ref{text:punktspreizfunktion}) demonstriert, dass die bis zu \SI{36}{\percent} des Ein-Photon-Signals außerhalb des zentralen Pixels liegt. In dem Fall kann der Ansatz des Clustering-Algorithmuses, der in Abschnitt \ref{text:clustering_algorithm} beschrieben wurde, vorteilhaft sein, um die Gesamtladung eines Photons zurückzugewinnen und dadurch die Sensitivität der Photonenerkennung zu erhöhen.

\noindent
Unter Berücksichtigung der Ergebnisse der durchgeführten Analyse und zwar, dass der gesamte \gls{adu}-Wert eines Photons grundsätzlich innerhalb zwei benachbarter Pixel verteilt wird, wird der Cluster-Kern
\begin{equation}
    \mathbf{K}_2 = \begin{bmatrix}
1 & 1\\
1 & 1
\end{bmatrix}
\end{equation}
benutzt.

\noindent
Die Einzelschritte der Anwendung des Clustering-Algorithmuses werden exemplarisch in Abb. \ref{fig:capture_ped_diff_clustering} am Beispiel desselben Streubildes dargestellt, das zur Demonstration der Einzelschritte vom Schwellen\-wert-Algorithmus in Abb. \ref{fig:capture_ped_diff} benutzt wurde.
\begin{figure}[H]
    \centering
    %% Creator: Matplotlib, PGF backend
%%
%% To include the figure in your LaTeX document, write
%%   \input{<filename>.pgf}
%%
%% Make sure the required packages are loaded in your preamble
%%   \usepackage{pgf}
%%
%% Also ensure that all the required font packages are loaded; for instance,
%% the lmodern package is sometimes necessary when using math font.
%%   \usepackage{lmodern}
%%
%% Figures using additional raster images can only be included by \input if
%% they are in the same directory as the main LaTeX file. For loading figures
%% from other directories you can use the `import` package
%%   \usepackage{import}
%%
%% and then include the figures with
%%   \import{<path to file>}{<filename>.pgf}
%%
%% Matplotlib used the following preamble
%%   \usepackage{amsmath} \usepackage[utf8]{inputenc} \usepackage[T1]{fontenc} \usepackage[output-decimal-marker={,},print-unity-mantissa=false]{siunitx} \sisetup{per-mode=fraction, separate-uncertainty = true, locale = DE} \usepackage[acronym, toc, section=section, nonumberlist, nopostdot]{glossaries-extra} \DeclareSIUnit\adu{\text{ADU}} \DeclareSIUnit\px{\text{px}} \DeclareSIUnit\photons{\text{Pho\-to\-nen}} \DeclareSIUnit\photon{\text{Pho\-ton}}
%%
\begingroup%
\makeatletter%
\begin{pgfpicture}%
\pgfpathrectangle{\pgfpointorigin}{\pgfqpoint{6.206186in}{2.306328in}}%
\pgfusepath{use as bounding box, clip}%
\begin{pgfscope}%
\pgfsetbuttcap%
\pgfsetmiterjoin%
\pgfsetlinewidth{0.000000pt}%
\definecolor{currentstroke}{rgb}{1.000000,1.000000,1.000000}%
\pgfsetstrokecolor{currentstroke}%
\pgfsetstrokeopacity{0.000000}%
\pgfsetdash{}{0pt}%
\pgfpathmoveto{\pgfqpoint{0.000000in}{0.000000in}}%
\pgfpathlineto{\pgfqpoint{6.206186in}{0.000000in}}%
\pgfpathlineto{\pgfqpoint{6.206186in}{2.306328in}}%
\pgfpathlineto{\pgfqpoint{0.000000in}{2.306328in}}%
\pgfpathlineto{\pgfqpoint{0.000000in}{0.000000in}}%
\pgfpathclose%
\pgfusepath{}%
\end{pgfscope}%
\begin{pgfscope}%
\pgfsetbuttcap%
\pgfsetmiterjoin%
\definecolor{currentfill}{rgb}{1.000000,1.000000,1.000000}%
\pgfsetfillcolor{currentfill}%
\pgfsetlinewidth{0.000000pt}%
\definecolor{currentstroke}{rgb}{0.000000,0.000000,0.000000}%
\pgfsetstrokecolor{currentstroke}%
\pgfsetstrokeopacity{0.000000}%
\pgfsetdash{}{0pt}%
\pgfpathmoveto{\pgfqpoint{0.136736in}{0.698088in}}%
\pgfpathlineto{\pgfqpoint{1.504099in}{0.698088in}}%
\pgfpathlineto{\pgfqpoint{1.504099in}{2.065451in}}%
\pgfpathlineto{\pgfqpoint{0.136736in}{2.065451in}}%
\pgfpathlineto{\pgfqpoint{0.136736in}{0.698088in}}%
\pgfpathclose%
\pgfusepath{fill}%
\end{pgfscope}%
\begin{pgfscope}%
\pgfsys@transformshift{0.136000in}{0.698328in}%
\pgftext[left,bottom]{\includegraphics[interpolate=true,width=1.368000in,height=1.368000in]{capture_ped_diff_clustering-img0.png}}%
\end{pgfscope}%
\begin{pgfscope}%
\pgfsetrectcap%
\pgfsetmiterjoin%
\pgfsetlinewidth{0.803000pt}%
\definecolor{currentstroke}{rgb}{0.000000,0.000000,0.000000}%
\pgfsetstrokecolor{currentstroke}%
\pgfsetdash{}{0pt}%
\pgfpathmoveto{\pgfqpoint{0.136736in}{0.698088in}}%
\pgfpathlineto{\pgfqpoint{0.136736in}{2.065451in}}%
\pgfusepath{stroke}%
\end{pgfscope}%
\begin{pgfscope}%
\pgfsetrectcap%
\pgfsetmiterjoin%
\pgfsetlinewidth{0.803000pt}%
\definecolor{currentstroke}{rgb}{0.000000,0.000000,0.000000}%
\pgfsetstrokecolor{currentstroke}%
\pgfsetdash{}{0pt}%
\pgfpathmoveto{\pgfqpoint{1.504099in}{0.698088in}}%
\pgfpathlineto{\pgfqpoint{1.504099in}{2.065451in}}%
\pgfusepath{stroke}%
\end{pgfscope}%
\begin{pgfscope}%
\pgfsetrectcap%
\pgfsetmiterjoin%
\pgfsetlinewidth{0.803000pt}%
\definecolor{currentstroke}{rgb}{0.000000,0.000000,0.000000}%
\pgfsetstrokecolor{currentstroke}%
\pgfsetdash{}{0pt}%
\pgfpathmoveto{\pgfqpoint{0.136736in}{0.698088in}}%
\pgfpathlineto{\pgfqpoint{1.504099in}{0.698088in}}%
\pgfusepath{stroke}%
\end{pgfscope}%
\begin{pgfscope}%
\pgfsetrectcap%
\pgfsetmiterjoin%
\pgfsetlinewidth{0.803000pt}%
\definecolor{currentstroke}{rgb}{0.000000,0.000000,0.000000}%
\pgfsetstrokecolor{currentstroke}%
\pgfsetdash{}{0pt}%
\pgfpathmoveto{\pgfqpoint{0.136736in}{2.065451in}}%
\pgfpathlineto{\pgfqpoint{1.504099in}{2.065451in}}%
\pgfusepath{stroke}%
\end{pgfscope}%
\begin{pgfscope}%
\definecolor{textcolor}{rgb}{0.000000,0.000000,0.000000}%
\pgfsetstrokecolor{textcolor}%
\pgfsetfillcolor{textcolor}%
\pgftext[x=0.000000in,y=2.202187in,left,base]{\color{textcolor}\rmfamily\fontsize{10.000000}{12.000000}\selectfont (c)}%
\end{pgfscope}%
\begin{pgfscope}%
\pgfsetbuttcap%
\pgfsetmiterjoin%
\definecolor{currentfill}{rgb}{1.000000,1.000000,1.000000}%
\pgfsetfillcolor{currentfill}%
\pgfsetlinewidth{0.000000pt}%
\definecolor{currentstroke}{rgb}{0.000000,0.000000,0.000000}%
\pgfsetstrokecolor{currentstroke}%
\pgfsetstrokeopacity{0.000000}%
\pgfsetdash{}{0pt}%
\pgfpathmoveto{\pgfqpoint{1.704099in}{0.698088in}}%
\pgfpathlineto{\pgfqpoint{3.071461in}{0.698088in}}%
\pgfpathlineto{\pgfqpoint{3.071461in}{2.065451in}}%
\pgfpathlineto{\pgfqpoint{1.704099in}{2.065451in}}%
\pgfpathlineto{\pgfqpoint{1.704099in}{0.698088in}}%
\pgfpathclose%
\pgfusepath{fill}%
\end{pgfscope}%
\begin{pgfscope}%
\pgfsys@transformshift{1.704000in}{0.698328in}%
\pgftext[left,bottom]{\includegraphics[interpolate=true,width=1.368000in,height=1.368000in]{capture_ped_diff_clustering-img1.png}}%
\end{pgfscope}%
\begin{pgfscope}%
\pgfsetrectcap%
\pgfsetmiterjoin%
\pgfsetlinewidth{0.803000pt}%
\definecolor{currentstroke}{rgb}{0.000000,0.000000,0.000000}%
\pgfsetstrokecolor{currentstroke}%
\pgfsetdash{}{0pt}%
\pgfpathmoveto{\pgfqpoint{1.704099in}{0.698088in}}%
\pgfpathlineto{\pgfqpoint{1.704099in}{2.065451in}}%
\pgfusepath{stroke}%
\end{pgfscope}%
\begin{pgfscope}%
\pgfsetrectcap%
\pgfsetmiterjoin%
\pgfsetlinewidth{0.803000pt}%
\definecolor{currentstroke}{rgb}{0.000000,0.000000,0.000000}%
\pgfsetstrokecolor{currentstroke}%
\pgfsetdash{}{0pt}%
\pgfpathmoveto{\pgfqpoint{3.071461in}{0.698088in}}%
\pgfpathlineto{\pgfqpoint{3.071461in}{2.065451in}}%
\pgfusepath{stroke}%
\end{pgfscope}%
\begin{pgfscope}%
\pgfsetrectcap%
\pgfsetmiterjoin%
\pgfsetlinewidth{0.803000pt}%
\definecolor{currentstroke}{rgb}{0.000000,0.000000,0.000000}%
\pgfsetstrokecolor{currentstroke}%
\pgfsetdash{}{0pt}%
\pgfpathmoveto{\pgfqpoint{1.704099in}{0.698088in}}%
\pgfpathlineto{\pgfqpoint{3.071461in}{0.698088in}}%
\pgfusepath{stroke}%
\end{pgfscope}%
\begin{pgfscope}%
\pgfsetrectcap%
\pgfsetmiterjoin%
\pgfsetlinewidth{0.803000pt}%
\definecolor{currentstroke}{rgb}{0.000000,0.000000,0.000000}%
\pgfsetstrokecolor{currentstroke}%
\pgfsetdash{}{0pt}%
\pgfpathmoveto{\pgfqpoint{1.704099in}{2.065451in}}%
\pgfpathlineto{\pgfqpoint{3.071461in}{2.065451in}}%
\pgfusepath{stroke}%
\end{pgfscope}%
\begin{pgfscope}%
\definecolor{textcolor}{rgb}{0.000000,0.000000,0.000000}%
\pgfsetstrokecolor{textcolor}%
\pgfsetfillcolor{textcolor}%
\pgftext[x=1.567363in,y=2.202187in,left,base]{\color{textcolor}\rmfamily\fontsize{10.000000}{12.000000}\selectfont (d)}%
\end{pgfscope}%
\begin{pgfscope}%
\pgfsetbuttcap%
\pgfsetmiterjoin%
\definecolor{currentfill}{rgb}{1.000000,1.000000,1.000000}%
\pgfsetfillcolor{currentfill}%
\pgfsetlinewidth{0.000000pt}%
\definecolor{currentstroke}{rgb}{0.000000,0.000000,0.000000}%
\pgfsetstrokecolor{currentstroke}%
\pgfsetstrokeopacity{0.000000}%
\pgfsetdash{}{0pt}%
\pgfpathmoveto{\pgfqpoint{3.271461in}{0.698088in}}%
\pgfpathlineto{\pgfqpoint{4.638824in}{0.698088in}}%
\pgfpathlineto{\pgfqpoint{4.638824in}{2.065451in}}%
\pgfpathlineto{\pgfqpoint{3.271461in}{2.065451in}}%
\pgfpathlineto{\pgfqpoint{3.271461in}{0.698088in}}%
\pgfpathclose%
\pgfusepath{fill}%
\end{pgfscope}%
\begin{pgfscope}%
\pgfsys@transformshift{3.272000in}{0.698328in}%
\pgftext[left,bottom]{\includegraphics[interpolate=true,width=1.366000in,height=1.368000in]{capture_ped_diff_clustering-img2.png}}%
\end{pgfscope}%
\begin{pgfscope}%
\pgfsetrectcap%
\pgfsetmiterjoin%
\pgfsetlinewidth{0.803000pt}%
\definecolor{currentstroke}{rgb}{0.000000,0.000000,0.000000}%
\pgfsetstrokecolor{currentstroke}%
\pgfsetdash{}{0pt}%
\pgfpathmoveto{\pgfqpoint{3.271461in}{0.698088in}}%
\pgfpathlineto{\pgfqpoint{3.271461in}{2.065451in}}%
\pgfusepath{stroke}%
\end{pgfscope}%
\begin{pgfscope}%
\pgfsetrectcap%
\pgfsetmiterjoin%
\pgfsetlinewidth{0.803000pt}%
\definecolor{currentstroke}{rgb}{0.000000,0.000000,0.000000}%
\pgfsetstrokecolor{currentstroke}%
\pgfsetdash{}{0pt}%
\pgfpathmoveto{\pgfqpoint{4.638824in}{0.698088in}}%
\pgfpathlineto{\pgfqpoint{4.638824in}{2.065451in}}%
\pgfusepath{stroke}%
\end{pgfscope}%
\begin{pgfscope}%
\pgfsetrectcap%
\pgfsetmiterjoin%
\pgfsetlinewidth{0.803000pt}%
\definecolor{currentstroke}{rgb}{0.000000,0.000000,0.000000}%
\pgfsetstrokecolor{currentstroke}%
\pgfsetdash{}{0pt}%
\pgfpathmoveto{\pgfqpoint{3.271461in}{0.698088in}}%
\pgfpathlineto{\pgfqpoint{4.638824in}{0.698088in}}%
\pgfusepath{stroke}%
\end{pgfscope}%
\begin{pgfscope}%
\pgfsetrectcap%
\pgfsetmiterjoin%
\pgfsetlinewidth{0.803000pt}%
\definecolor{currentstroke}{rgb}{0.000000,0.000000,0.000000}%
\pgfsetstrokecolor{currentstroke}%
\pgfsetdash{}{0pt}%
\pgfpathmoveto{\pgfqpoint{3.271461in}{2.065451in}}%
\pgfpathlineto{\pgfqpoint{4.638824in}{2.065451in}}%
\pgfusepath{stroke}%
\end{pgfscope}%
\begin{pgfscope}%
\definecolor{textcolor}{rgb}{0.000000,0.000000,0.000000}%
\pgfsetstrokecolor{textcolor}%
\pgfsetfillcolor{textcolor}%
\pgftext[x=3.134725in,y=2.202187in,left,base]{\color{textcolor}\rmfamily\fontsize{10.000000}{12.000000}\selectfont (f)}%
\end{pgfscope}%
\begin{pgfscope}%
\pgfsetbuttcap%
\pgfsetmiterjoin%
\definecolor{currentfill}{rgb}{1.000000,1.000000,1.000000}%
\pgfsetfillcolor{currentfill}%
\pgfsetlinewidth{0.000000pt}%
\definecolor{currentstroke}{rgb}{0.000000,0.000000,0.000000}%
\pgfsetstrokecolor{currentstroke}%
\pgfsetstrokeopacity{0.000000}%
\pgfsetdash{}{0pt}%
\pgfpathmoveto{\pgfqpoint{4.838824in}{0.698088in}}%
\pgfpathlineto{\pgfqpoint{6.206186in}{0.698088in}}%
\pgfpathlineto{\pgfqpoint{6.206186in}{2.065451in}}%
\pgfpathlineto{\pgfqpoint{4.838824in}{2.065451in}}%
\pgfpathlineto{\pgfqpoint{4.838824in}{0.698088in}}%
\pgfpathclose%
\pgfusepath{fill}%
\end{pgfscope}%
\begin{pgfscope}%
\pgfsys@transformshift{4.896000in}{0.790328in}%
\pgftext[left,bottom]{\includegraphics[interpolate=true,width=1.248000in,height=1.270000in]{capture_ped_diff_clustering-img3.png}}%
\end{pgfscope}%
\begin{pgfscope}%
\pgfsetrectcap%
\pgfsetmiterjoin%
\pgfsetlinewidth{0.803000pt}%
\definecolor{currentstroke}{rgb}{0.000000,0.000000,0.000000}%
\pgfsetstrokecolor{currentstroke}%
\pgfsetdash{}{0pt}%
\pgfpathmoveto{\pgfqpoint{4.838824in}{0.698088in}}%
\pgfpathlineto{\pgfqpoint{4.838824in}{2.065451in}}%
\pgfusepath{stroke}%
\end{pgfscope}%
\begin{pgfscope}%
\pgfsetrectcap%
\pgfsetmiterjoin%
\pgfsetlinewidth{0.803000pt}%
\definecolor{currentstroke}{rgb}{0.000000,0.000000,0.000000}%
\pgfsetstrokecolor{currentstroke}%
\pgfsetdash{}{0pt}%
\pgfpathmoveto{\pgfqpoint{6.206186in}{0.698088in}}%
\pgfpathlineto{\pgfqpoint{6.206186in}{2.065451in}}%
\pgfusepath{stroke}%
\end{pgfscope}%
\begin{pgfscope}%
\pgfsetrectcap%
\pgfsetmiterjoin%
\pgfsetlinewidth{0.803000pt}%
\definecolor{currentstroke}{rgb}{0.000000,0.000000,0.000000}%
\pgfsetstrokecolor{currentstroke}%
\pgfsetdash{}{0pt}%
\pgfpathmoveto{\pgfqpoint{4.838824in}{0.698088in}}%
\pgfpathlineto{\pgfqpoint{6.206186in}{0.698088in}}%
\pgfusepath{stroke}%
\end{pgfscope}%
\begin{pgfscope}%
\pgfsetrectcap%
\pgfsetmiterjoin%
\pgfsetlinewidth{0.803000pt}%
\definecolor{currentstroke}{rgb}{0.000000,0.000000,0.000000}%
\pgfsetstrokecolor{currentstroke}%
\pgfsetdash{}{0pt}%
\pgfpathmoveto{\pgfqpoint{4.838824in}{2.065451in}}%
\pgfpathlineto{\pgfqpoint{6.206186in}{2.065451in}}%
\pgfusepath{stroke}%
\end{pgfscope}%
\begin{pgfscope}%
\definecolor{textcolor}{rgb}{0.000000,0.000000,0.000000}%
\pgfsetstrokecolor{textcolor}%
\pgfsetfillcolor{textcolor}%
\pgftext[x=4.702087in,y=2.202187in,left,base]{\color{textcolor}\rmfamily\fontsize{10.000000}{12.000000}\selectfont (g)}%
\end{pgfscope}%
\begin{pgfscope}%
\pgfsetbuttcap%
\pgfsetmiterjoin%
\definecolor{currentfill}{rgb}{1.000000,1.000000,1.000000}%
\pgfsetfillcolor{currentfill}%
\pgfsetlinewidth{0.000000pt}%
\definecolor{currentstroke}{rgb}{0.000000,0.000000,0.000000}%
\pgfsetstrokecolor{currentstroke}%
\pgfsetstrokeopacity{0.000000}%
\pgfsetdash{}{0pt}%
\pgfpathmoveto{\pgfqpoint{0.136736in}{0.398088in}}%
\pgfpathlineto{\pgfqpoint{1.504099in}{0.398088in}}%
\pgfpathlineto{\pgfqpoint{1.504099in}{0.498088in}}%
\pgfpathlineto{\pgfqpoint{0.136736in}{0.498088in}}%
\pgfpathlineto{\pgfqpoint{0.136736in}{0.398088in}}%
\pgfpathclose%
\pgfusepath{fill}%
\end{pgfscope}%
\begin{pgfscope}%
\pgfpathrectangle{\pgfqpoint{0.136736in}{0.398088in}}{\pgfqpoint{1.367363in}{0.100000in}}%
\pgfusepath{clip}%
\pgfsetbuttcap%
\pgfsetmiterjoin%
\definecolor{currentfill}{rgb}{1.000000,1.000000,1.000000}%
\pgfsetfillcolor{currentfill}%
\pgfsetlinewidth{0.010037pt}%
\definecolor{currentstroke}{rgb}{1.000000,1.000000,1.000000}%
\pgfsetstrokecolor{currentstroke}%
\pgfsetdash{}{0pt}%
\pgfusepath{stroke,fill}%
\end{pgfscope}%
\begin{pgfscope}%
\pgfsys@transformshift{0.136000in}{0.398328in}%
\pgftext[left,bottom]{\includegraphics[interpolate=true,width=1.368000in,height=0.100000in]{capture_ped_diff_clustering-img4.png}}%
\end{pgfscope}%
\begin{pgfscope}%
\pgfsetbuttcap%
\pgfsetroundjoin%
\definecolor{currentfill}{rgb}{0.000000,0.000000,0.000000}%
\pgfsetfillcolor{currentfill}%
\pgfsetlinewidth{0.803000pt}%
\definecolor{currentstroke}{rgb}{0.000000,0.000000,0.000000}%
\pgfsetstrokecolor{currentstroke}%
\pgfsetdash{}{0pt}%
\pgfsys@defobject{currentmarker}{\pgfqpoint{0.000000in}{-0.048611in}}{\pgfqpoint{0.000000in}{0.000000in}}{%
\pgfpathmoveto{\pgfqpoint{0.000000in}{0.000000in}}%
\pgfpathlineto{\pgfqpoint{0.000000in}{-0.048611in}}%
\pgfusepath{stroke,fill}%
}%
\begin{pgfscope}%
\pgfsys@transformshift{0.364630in}{0.398088in}%
\pgfsys@useobject{currentmarker}{}%
\end{pgfscope}%
\end{pgfscope}%
\begin{pgfscope}%
\definecolor{textcolor}{rgb}{0.000000,0.000000,0.000000}%
\pgfsetstrokecolor{textcolor}%
\pgfsetfillcolor{textcolor}%
\pgftext[x=0.364630in,y=0.300866in,,top]{\color{textcolor}\rmfamily\fontsize{10.000000}{12.000000}\selectfont -100}%
\end{pgfscope}%
\begin{pgfscope}%
\pgfsetbuttcap%
\pgfsetroundjoin%
\definecolor{currentfill}{rgb}{0.000000,0.000000,0.000000}%
\pgfsetfillcolor{currentfill}%
\pgfsetlinewidth{0.803000pt}%
\definecolor{currentstroke}{rgb}{0.000000,0.000000,0.000000}%
\pgfsetstrokecolor{currentstroke}%
\pgfsetdash{}{0pt}%
\pgfsys@defobject{currentmarker}{\pgfqpoint{0.000000in}{-0.048611in}}{\pgfqpoint{0.000000in}{0.000000in}}{%
\pgfpathmoveto{\pgfqpoint{0.000000in}{0.000000in}}%
\pgfpathlineto{\pgfqpoint{0.000000in}{-0.048611in}}%
\pgfusepath{stroke,fill}%
}%
\begin{pgfscope}%
\pgfsys@transformshift{0.820418in}{0.398088in}%
\pgfsys@useobject{currentmarker}{}%
\end{pgfscope}%
\end{pgfscope}%
\begin{pgfscope}%
\definecolor{textcolor}{rgb}{0.000000,0.000000,0.000000}%
\pgfsetstrokecolor{textcolor}%
\pgfsetfillcolor{textcolor}%
\pgftext[x=0.820418in,y=0.300866in,,top]{\color{textcolor}\rmfamily\fontsize{10.000000}{12.000000}\selectfont 0}%
\end{pgfscope}%
\begin{pgfscope}%
\pgfsetbuttcap%
\pgfsetroundjoin%
\definecolor{currentfill}{rgb}{0.000000,0.000000,0.000000}%
\pgfsetfillcolor{currentfill}%
\pgfsetlinewidth{0.803000pt}%
\definecolor{currentstroke}{rgb}{0.000000,0.000000,0.000000}%
\pgfsetstrokecolor{currentstroke}%
\pgfsetdash{}{0pt}%
\pgfsys@defobject{currentmarker}{\pgfqpoint{0.000000in}{-0.048611in}}{\pgfqpoint{0.000000in}{0.000000in}}{%
\pgfpathmoveto{\pgfqpoint{0.000000in}{0.000000in}}%
\pgfpathlineto{\pgfqpoint{0.000000in}{-0.048611in}}%
\pgfusepath{stroke,fill}%
}%
\begin{pgfscope}%
\pgfsys@transformshift{1.276205in}{0.398088in}%
\pgfsys@useobject{currentmarker}{}%
\end{pgfscope}%
\end{pgfscope}%
\begin{pgfscope}%
\definecolor{textcolor}{rgb}{0.000000,0.000000,0.000000}%
\pgfsetstrokecolor{textcolor}%
\pgfsetfillcolor{textcolor}%
\pgftext[x=1.276205in,y=0.300866in,,top]{\color{textcolor}\rmfamily\fontsize{10.000000}{12.000000}\selectfont 100}%
\end{pgfscope}%
\begin{pgfscope}%
\definecolor{textcolor}{rgb}{0.000000,0.000000,0.000000}%
\pgfsetstrokecolor{textcolor}%
\pgfsetfillcolor{textcolor}%
\pgftext[x=0.820418in,y=0.122655in,,top]{\color{textcolor}\rmfamily\fontsize{10.000000}{12.000000}\selectfont Auslesesignal in ADU}%
\end{pgfscope}%
\begin{pgfscope}%
\pgfsetrectcap%
\pgfsetmiterjoin%
\pgfsetlinewidth{0.803000pt}%
\definecolor{currentstroke}{rgb}{0.000000,0.000000,0.000000}%
\pgfsetstrokecolor{currentstroke}%
\pgfsetdash{}{0pt}%
\pgfpathmoveto{\pgfqpoint{0.136736in}{0.398088in}}%
\pgfpathlineto{\pgfqpoint{0.136736in}{0.448088in}}%
\pgfpathlineto{\pgfqpoint{0.136736in}{0.498088in}}%
\pgfpathlineto{\pgfqpoint{1.504099in}{0.498088in}}%
\pgfpathlineto{\pgfqpoint{1.504099in}{0.448088in}}%
\pgfpathlineto{\pgfqpoint{1.504099in}{0.398088in}}%
\pgfpathlineto{\pgfqpoint{0.136736in}{0.398088in}}%
\pgfpathclose%
\pgfusepath{stroke}%
\end{pgfscope}%
\begin{pgfscope}%
\pgfsetbuttcap%
\pgfsetmiterjoin%
\definecolor{currentfill}{rgb}{1.000000,1.000000,1.000000}%
\pgfsetfillcolor{currentfill}%
\pgfsetlinewidth{0.000000pt}%
\definecolor{currentstroke}{rgb}{0.000000,0.000000,0.000000}%
\pgfsetstrokecolor{currentstroke}%
\pgfsetstrokeopacity{0.000000}%
\pgfsetdash{}{0pt}%
\pgfpathmoveto{\pgfqpoint{1.704099in}{0.398088in}}%
\pgfpathlineto{\pgfqpoint{3.071461in}{0.398088in}}%
\pgfpathlineto{\pgfqpoint{3.071461in}{0.498088in}}%
\pgfpathlineto{\pgfqpoint{1.704099in}{0.498088in}}%
\pgfpathlineto{\pgfqpoint{1.704099in}{0.398088in}}%
\pgfpathclose%
\pgfusepath{fill}%
\end{pgfscope}%
\begin{pgfscope}%
\pgfpathrectangle{\pgfqpoint{1.704099in}{0.398088in}}{\pgfqpoint{1.367363in}{0.100000in}}%
\pgfusepath{clip}%
\pgfsetbuttcap%
\pgfsetmiterjoin%
\definecolor{currentfill}{rgb}{1.000000,1.000000,1.000000}%
\pgfsetfillcolor{currentfill}%
\pgfsetlinewidth{0.010037pt}%
\definecolor{currentstroke}{rgb}{1.000000,1.000000,1.000000}%
\pgfsetstrokecolor{currentstroke}%
\pgfsetdash{}{0pt}%
\pgfusepath{stroke,fill}%
\end{pgfscope}%
\begin{pgfscope}%
\pgfsys@transformshift{1.704000in}{0.398328in}%
\pgftext[left,bottom]{\includegraphics[interpolate=true,width=1.368000in,height=0.100000in]{capture_ped_diff_clustering-img5.png}}%
\end{pgfscope}%
\begin{pgfscope}%
\pgfsetbuttcap%
\pgfsetroundjoin%
\definecolor{currentfill}{rgb}{0.000000,0.000000,0.000000}%
\pgfsetfillcolor{currentfill}%
\pgfsetlinewidth{0.803000pt}%
\definecolor{currentstroke}{rgb}{0.000000,0.000000,0.000000}%
\pgfsetstrokecolor{currentstroke}%
\pgfsetdash{}{0pt}%
\pgfsys@defobject{currentmarker}{\pgfqpoint{0.000000in}{-0.048611in}}{\pgfqpoint{0.000000in}{0.000000in}}{%
\pgfpathmoveto{\pgfqpoint{0.000000in}{0.000000in}}%
\pgfpathlineto{\pgfqpoint{0.000000in}{-0.048611in}}%
\pgfusepath{stroke,fill}%
}%
\begin{pgfscope}%
\pgfsys@transformshift{1.811734in}{0.398088in}%
\pgfsys@useobject{currentmarker}{}%
\end{pgfscope}%
\end{pgfscope}%
\begin{pgfscope}%
\definecolor{textcolor}{rgb}{0.000000,0.000000,0.000000}%
\pgfsetstrokecolor{textcolor}%
\pgfsetfillcolor{textcolor}%
\pgftext[x=1.811734in,y=0.300866in,,top]{\color{textcolor}\rmfamily\fontsize{10.000000}{12.000000}\selectfont -250}%
\end{pgfscope}%
\begin{pgfscope}%
\pgfsetbuttcap%
\pgfsetroundjoin%
\definecolor{currentfill}{rgb}{0.000000,0.000000,0.000000}%
\pgfsetfillcolor{currentfill}%
\pgfsetlinewidth{0.803000pt}%
\definecolor{currentstroke}{rgb}{0.000000,0.000000,0.000000}%
\pgfsetstrokecolor{currentstroke}%
\pgfsetdash{}{0pt}%
\pgfsys@defobject{currentmarker}{\pgfqpoint{0.000000in}{-0.048611in}}{\pgfqpoint{0.000000in}{0.000000in}}{%
\pgfpathmoveto{\pgfqpoint{0.000000in}{0.000000in}}%
\pgfpathlineto{\pgfqpoint{0.000000in}{-0.048611in}}%
\pgfusepath{stroke,fill}%
}%
\begin{pgfscope}%
\pgfsys@transformshift{2.310044in}{0.398088in}%
\pgfsys@useobject{currentmarker}{}%
\end{pgfscope}%
\end{pgfscope}%
\begin{pgfscope}%
\definecolor{textcolor}{rgb}{0.000000,0.000000,0.000000}%
\pgfsetstrokecolor{textcolor}%
\pgfsetfillcolor{textcolor}%
\pgftext[x=2.310044in,y=0.300866in,,top]{\color{textcolor}\rmfamily\fontsize{10.000000}{12.000000}\selectfont 0}%
\end{pgfscope}%
\begin{pgfscope}%
\pgfsetbuttcap%
\pgfsetroundjoin%
\definecolor{currentfill}{rgb}{0.000000,0.000000,0.000000}%
\pgfsetfillcolor{currentfill}%
\pgfsetlinewidth{0.803000pt}%
\definecolor{currentstroke}{rgb}{0.000000,0.000000,0.000000}%
\pgfsetstrokecolor{currentstroke}%
\pgfsetdash{}{0pt}%
\pgfsys@defobject{currentmarker}{\pgfqpoint{0.000000in}{-0.048611in}}{\pgfqpoint{0.000000in}{0.000000in}}{%
\pgfpathmoveto{\pgfqpoint{0.000000in}{0.000000in}}%
\pgfpathlineto{\pgfqpoint{0.000000in}{-0.048611in}}%
\pgfusepath{stroke,fill}%
}%
\begin{pgfscope}%
\pgfsys@transformshift{2.808354in}{0.398088in}%
\pgfsys@useobject{currentmarker}{}%
\end{pgfscope}%
\end{pgfscope}%
\begin{pgfscope}%
\definecolor{textcolor}{rgb}{0.000000,0.000000,0.000000}%
\pgfsetstrokecolor{textcolor}%
\pgfsetfillcolor{textcolor}%
\pgftext[x=2.808354in,y=0.300866in,,top]{\color{textcolor}\rmfamily\fontsize{10.000000}{12.000000}\selectfont 250}%
\end{pgfscope}%
\begin{pgfscope}%
\definecolor{textcolor}{rgb}{0.000000,0.000000,0.000000}%
\pgfsetstrokecolor{textcolor}%
\pgfsetfillcolor{textcolor}%
\pgftext[x=2.387780in,y=0.122655in,,top]{\color{textcolor}\rmfamily\fontsize{10.000000}{12.000000}\selectfont Auslesesignal in ADU}%
\end{pgfscope}%
\begin{pgfscope}%
\pgfsetrectcap%
\pgfsetmiterjoin%
\pgfsetlinewidth{0.803000pt}%
\definecolor{currentstroke}{rgb}{0.000000,0.000000,0.000000}%
\pgfsetstrokecolor{currentstroke}%
\pgfsetdash{}{0pt}%
\pgfpathmoveto{\pgfqpoint{1.704099in}{0.398088in}}%
\pgfpathlineto{\pgfqpoint{1.704099in}{0.448088in}}%
\pgfpathlineto{\pgfqpoint{1.704099in}{0.498088in}}%
\pgfpathlineto{\pgfqpoint{3.071461in}{0.498088in}}%
\pgfpathlineto{\pgfqpoint{3.071461in}{0.448088in}}%
\pgfpathlineto{\pgfqpoint{3.071461in}{0.398088in}}%
\pgfpathlineto{\pgfqpoint{1.704099in}{0.398088in}}%
\pgfpathclose%
\pgfusepath{stroke}%
\end{pgfscope}%
\begin{pgfscope}%
\pgfsetbuttcap%
\pgfsetmiterjoin%
\definecolor{currentfill}{rgb}{1.000000,1.000000,1.000000}%
\pgfsetfillcolor{currentfill}%
\pgfsetlinewidth{0.000000pt}%
\definecolor{currentstroke}{rgb}{0.000000,0.000000,0.000000}%
\pgfsetstrokecolor{currentstroke}%
\pgfsetstrokeopacity{0.000000}%
\pgfsetdash{}{0pt}%
\pgfpathmoveto{\pgfqpoint{3.271461in}{0.398088in}}%
\pgfpathlineto{\pgfqpoint{4.638824in}{0.398088in}}%
\pgfpathlineto{\pgfqpoint{4.638824in}{0.498088in}}%
\pgfpathlineto{\pgfqpoint{3.271461in}{0.498088in}}%
\pgfpathlineto{\pgfqpoint{3.271461in}{0.398088in}}%
\pgfpathclose%
\pgfusepath{fill}%
\end{pgfscope}%
\begin{pgfscope}%
\pgfpathrectangle{\pgfqpoint{3.271461in}{0.398088in}}{\pgfqpoint{1.367362in}{0.100000in}}%
\pgfusepath{clip}%
\pgfsetbuttcap%
\pgfsetmiterjoin%
\definecolor{currentfill}{rgb}{1.000000,1.000000,1.000000}%
\pgfsetfillcolor{currentfill}%
\pgfsetlinewidth{0.010037pt}%
\definecolor{currentstroke}{rgb}{1.000000,1.000000,1.000000}%
\pgfsetstrokecolor{currentstroke}%
\pgfsetdash{}{0pt}%
\pgfusepath{stroke,fill}%
\end{pgfscope}%
\begin{pgfscope}%
\pgfsys@transformshift{3.272000in}{0.398328in}%
\pgftext[left,bottom]{\includegraphics[interpolate=true,width=1.366000in,height=0.100000in]{capture_ped_diff_clustering-img6.png}}%
\end{pgfscope}%
\begin{pgfscope}%
\pgfsetbuttcap%
\pgfsetroundjoin%
\definecolor{currentfill}{rgb}{0.000000,0.000000,0.000000}%
\pgfsetfillcolor{currentfill}%
\pgfsetlinewidth{0.803000pt}%
\definecolor{currentstroke}{rgb}{0.000000,0.000000,0.000000}%
\pgfsetstrokecolor{currentstroke}%
\pgfsetdash{}{0pt}%
\pgfsys@defobject{currentmarker}{\pgfqpoint{0.000000in}{-0.048611in}}{\pgfqpoint{0.000000in}{0.000000in}}{%
\pgfpathmoveto{\pgfqpoint{0.000000in}{0.000000in}}%
\pgfpathlineto{\pgfqpoint{0.000000in}{-0.048611in}}%
\pgfusepath{stroke,fill}%
}%
\begin{pgfscope}%
\pgfsys@transformshift{3.271461in}{0.398088in}%
\pgfsys@useobject{currentmarker}{}%
\end{pgfscope}%
\end{pgfscope}%
\begin{pgfscope}%
\definecolor{textcolor}{rgb}{0.000000,0.000000,0.000000}%
\pgfsetstrokecolor{textcolor}%
\pgfsetfillcolor{textcolor}%
\pgftext[x=3.271461in,y=0.300866in,,top]{\color{textcolor}\rmfamily\fontsize{10.000000}{12.000000}\selectfont 0}%
\end{pgfscope}%
\begin{pgfscope}%
\pgfsetbuttcap%
\pgfsetroundjoin%
\definecolor{currentfill}{rgb}{0.000000,0.000000,0.000000}%
\pgfsetfillcolor{currentfill}%
\pgfsetlinewidth{0.803000pt}%
\definecolor{currentstroke}{rgb}{0.000000,0.000000,0.000000}%
\pgfsetstrokecolor{currentstroke}%
\pgfsetdash{}{0pt}%
\pgfsys@defobject{currentmarker}{\pgfqpoint{0.000000in}{-0.048611in}}{\pgfqpoint{0.000000in}{0.000000in}}{%
\pgfpathmoveto{\pgfqpoint{0.000000in}{0.000000in}}%
\pgfpathlineto{\pgfqpoint{0.000000in}{-0.048611in}}%
\pgfusepath{stroke,fill}%
}%
\begin{pgfscope}%
\pgfsys@transformshift{3.987358in}{0.398088in}%
\pgfsys@useobject{currentmarker}{}%
\end{pgfscope}%
\end{pgfscope}%
\begin{pgfscope}%
\definecolor{textcolor}{rgb}{0.000000,0.000000,0.000000}%
\pgfsetstrokecolor{textcolor}%
\pgfsetfillcolor{textcolor}%
\pgftext[x=3.987358in,y=0.300866in,,top]{\color{textcolor}\rmfamily\fontsize{10.000000}{12.000000}\selectfont 200}%
\end{pgfscope}%
\begin{pgfscope}%
\definecolor{textcolor}{rgb}{0.000000,0.000000,0.000000}%
\pgfsetstrokecolor{textcolor}%
\pgfsetfillcolor{textcolor}%
\pgftext[x=3.955143in,y=0.122655in,,top]{\color{textcolor}\rmfamily\fontsize{10.000000}{12.000000}\selectfont Auslesesignal in ADU}%
\end{pgfscope}%
\begin{pgfscope}%
\pgfsetrectcap%
\pgfsetmiterjoin%
\pgfsetlinewidth{0.803000pt}%
\definecolor{currentstroke}{rgb}{0.000000,0.000000,0.000000}%
\pgfsetstrokecolor{currentstroke}%
\pgfsetdash{}{0pt}%
\pgfpathmoveto{\pgfqpoint{3.271461in}{0.398088in}}%
\pgfpathlineto{\pgfqpoint{3.271461in}{0.448088in}}%
\pgfpathlineto{\pgfqpoint{3.271461in}{0.498088in}}%
\pgfpathlineto{\pgfqpoint{4.638824in}{0.498088in}}%
\pgfpathlineto{\pgfqpoint{4.638824in}{0.448088in}}%
\pgfpathlineto{\pgfqpoint{4.638824in}{0.398088in}}%
\pgfpathlineto{\pgfqpoint{3.271461in}{0.398088in}}%
\pgfpathclose%
\pgfusepath{stroke}%
\end{pgfscope}%
\begin{pgfscope}%
\pgfsetbuttcap%
\pgfsetmiterjoin%
\definecolor{currentfill}{rgb}{1.000000,1.000000,1.000000}%
\pgfsetfillcolor{currentfill}%
\pgfsetlinewidth{0.000000pt}%
\definecolor{currentstroke}{rgb}{0.000000,0.000000,0.000000}%
\pgfsetstrokecolor{currentstroke}%
\pgfsetstrokeopacity{0.000000}%
\pgfsetdash{}{0pt}%
\pgfpathmoveto{\pgfqpoint{4.838824in}{0.398088in}}%
\pgfpathlineto{\pgfqpoint{6.206186in}{0.398088in}}%
\pgfpathlineto{\pgfqpoint{6.206186in}{0.498088in}}%
\pgfpathlineto{\pgfqpoint{4.838824in}{0.498088in}}%
\pgfpathlineto{\pgfqpoint{4.838824in}{0.398088in}}%
\pgfpathclose%
\pgfusepath{fill}%
\end{pgfscope}%
\begin{pgfscope}%
\pgfpathrectangle{\pgfqpoint{4.838824in}{0.398088in}}{\pgfqpoint{1.367362in}{0.100000in}}%
\pgfusepath{clip}%
\pgfsetbuttcap%
\pgfsetmiterjoin%
\definecolor{currentfill}{rgb}{1.000000,1.000000,1.000000}%
\pgfsetfillcolor{currentfill}%
\pgfsetlinewidth{0.010037pt}%
\definecolor{currentstroke}{rgb}{1.000000,1.000000,1.000000}%
\pgfsetstrokecolor{currentstroke}%
\pgfsetdash{}{0pt}%
\pgfusepath{stroke,fill}%
\end{pgfscope}%
\begin{pgfscope}%
\pgfpathrectangle{\pgfqpoint{4.838824in}{0.398088in}}{\pgfqpoint{1.367362in}{0.100000in}}%
\pgfusepath{clip}%
\pgfsetbuttcap%
\pgfsetroundjoin%
\pgfsetlinewidth{0.000000pt}%
\definecolor{currentstroke}{rgb}{0.000000,0.000000,0.000000}%
\pgfsetstrokecolor{currentstroke}%
\pgfsetdash{}{0pt}%
\pgfpathmoveto{\pgfqpoint{4.838824in}{0.398088in}}%
\pgfpathlineto{\pgfqpoint{4.838824in}{0.498088in}}%
\pgfpathlineto{\pgfqpoint{5.522505in}{0.498088in}}%
\pgfpathlineto{\pgfqpoint{5.522505in}{0.398088in}}%
\pgfpathlineto{\pgfqpoint{4.838824in}{0.398088in}}%
\pgfusepath{}%
\end{pgfscope}%
\begin{pgfscope}%
\pgfpathrectangle{\pgfqpoint{4.838824in}{0.398088in}}{\pgfqpoint{1.367362in}{0.100000in}}%
\pgfusepath{clip}%
\pgfsetbuttcap%
\pgfsetroundjoin%
\definecolor{currentfill}{rgb}{1.000000,0.000000,0.000000}%
\pgfsetfillcolor{currentfill}%
\pgfsetlinewidth{0.000000pt}%
\definecolor{currentstroke}{rgb}{0.000000,0.000000,0.000000}%
\pgfsetstrokecolor{currentstroke}%
\pgfsetdash{}{0pt}%
\pgfpathmoveto{\pgfqpoint{5.522505in}{0.398088in}}%
\pgfpathlineto{\pgfqpoint{5.522505in}{0.498088in}}%
\pgfpathlineto{\pgfqpoint{6.206186in}{0.498088in}}%
\pgfpathlineto{\pgfqpoint{6.206186in}{0.398088in}}%
\pgfpathlineto{\pgfqpoint{5.522505in}{0.398088in}}%
\pgfusepath{fill}%
\end{pgfscope}%
\begin{pgfscope}%
\pgfsetbuttcap%
\pgfsetroundjoin%
\definecolor{currentfill}{rgb}{0.000000,0.000000,0.000000}%
\pgfsetfillcolor{currentfill}%
\pgfsetlinewidth{0.803000pt}%
\definecolor{currentstroke}{rgb}{0.000000,0.000000,0.000000}%
\pgfsetstrokecolor{currentstroke}%
\pgfsetdash{}{0pt}%
\pgfsys@defobject{currentmarker}{\pgfqpoint{0.000000in}{-0.048611in}}{\pgfqpoint{0.000000in}{0.000000in}}{%
\pgfpathmoveto{\pgfqpoint{0.000000in}{0.000000in}}%
\pgfpathlineto{\pgfqpoint{0.000000in}{-0.048611in}}%
\pgfusepath{stroke,fill}%
}%
\begin{pgfscope}%
\pgfsys@transformshift{5.180664in}{0.398088in}%
\pgfsys@useobject{currentmarker}{}%
\end{pgfscope}%
\end{pgfscope}%
\begin{pgfscope}%
\definecolor{textcolor}{rgb}{0.000000,0.000000,0.000000}%
\pgfsetstrokecolor{textcolor}%
\pgfsetfillcolor{textcolor}%
\pgftext[x=5.180664in,y=0.300866in,,top]{\color{textcolor}\rmfamily\fontsize{10.000000}{12.000000}\selectfont 0}%
\end{pgfscope}%
\begin{pgfscope}%
\pgfsetbuttcap%
\pgfsetroundjoin%
\definecolor{currentfill}{rgb}{0.000000,0.000000,0.000000}%
\pgfsetfillcolor{currentfill}%
\pgfsetlinewidth{0.803000pt}%
\definecolor{currentstroke}{rgb}{0.000000,0.000000,0.000000}%
\pgfsetstrokecolor{currentstroke}%
\pgfsetdash{}{0pt}%
\pgfsys@defobject{currentmarker}{\pgfqpoint{0.000000in}{-0.048611in}}{\pgfqpoint{0.000000in}{0.000000in}}{%
\pgfpathmoveto{\pgfqpoint{0.000000in}{0.000000in}}%
\pgfpathlineto{\pgfqpoint{0.000000in}{-0.048611in}}%
\pgfusepath{stroke,fill}%
}%
\begin{pgfscope}%
\pgfsys@transformshift{5.864346in}{0.398088in}%
\pgfsys@useobject{currentmarker}{}%
\end{pgfscope}%
\end{pgfscope}%
\begin{pgfscope}%
\definecolor{textcolor}{rgb}{0.000000,0.000000,0.000000}%
\pgfsetstrokecolor{textcolor}%
\pgfsetfillcolor{textcolor}%
\pgftext[x=5.864346in,y=0.300866in,,top]{\color{textcolor}\rmfamily\fontsize{10.000000}{12.000000}\selectfont 1}%
\end{pgfscope}%
\begin{pgfscope}%
\definecolor{textcolor}{rgb}{0.000000,0.000000,0.000000}%
\pgfsetstrokecolor{textcolor}%
\pgfsetfillcolor{textcolor}%
\pgftext[x=5.522505in,y=0.122655in,,top]{\color{textcolor}\rmfamily\fontsize{10.000000}{12.000000}\selectfont Photonen}%
\end{pgfscope}%
\begin{pgfscope}%
\pgfsetrectcap%
\pgfsetmiterjoin%
\pgfsetlinewidth{0.803000pt}%
\definecolor{currentstroke}{rgb}{0.000000,0.000000,0.000000}%
\pgfsetstrokecolor{currentstroke}%
\pgfsetdash{}{0pt}%
\pgfpathmoveto{\pgfqpoint{4.838824in}{0.398088in}}%
\pgfpathlineto{\pgfqpoint{4.838824in}{0.448088in}}%
\pgfpathlineto{\pgfqpoint{4.838824in}{0.498088in}}%
\pgfpathlineto{\pgfqpoint{6.206186in}{0.498088in}}%
\pgfpathlineto{\pgfqpoint{6.206186in}{0.448088in}}%
\pgfpathlineto{\pgfqpoint{6.206186in}{0.398088in}}%
\pgfpathlineto{\pgfqpoint{4.838824in}{0.398088in}}%
\pgfpathclose%
\pgfusepath{stroke}%
\end{pgfscope}%
\end{pgfpicture}%
\makeatother%
\endgroup%

    \caption{Die Unterabbildungen (a), (b) und (c) stimmen mit Abb. \ref{fig:capture_ped_diff} überein. Die  (c) Differenz vom aufgenommenen Streubild und dem konstanten Offset wird mit dem Cluster-Kern $\mathbf{K}_2$ gefaltet (d). Es werden nur lokale Maxima (f) in \qtyproduct{2 x 2}{\px}-Umgebungen behalten. Zum Schluss (g) wird der Schwellenwert $s_Q = \SI{150}{\adu}$ angewendet.}
    \label{fig:capture_ped_diff_clustering}
\end{figure}
\noindent
Die ersten zwei Schritte (Abb. \ref{fig:capture_ped_diff_clustering} a und b), in denen der konstante Offset vom aufgenommenen Streubild abgezogen wird, ist zum Schwellenwert-Algorithmus identisch (Abb. \ref{fig:capture_ped_diff}a und b) und werden daher weggelassen. Die resultierende Differenz (Abb. \ref{fig:capture_ped_diff_clustering}c) wird mit dem Cluster-Kern  $\mathbf{K}_2$ gefaltet (Abb. \ref{fig:capture_ped_diff_clustering}d). Als nächstes werden die lokalen Maxima in \qtyproduct{2 x 2}{\px}-Nachbarschaft gesucht und behalten (Abb. \ref{fig:capture_ped_diff_clustering}f). Zum Schluss werden diejenige Punkte, die den Wert $s_Q$ überschreiten, als Photonen bezeichnet (Abb. \ref{fig:capture_ped_diff_clustering}g). Mit dem Einsatz des Clustering-Algorithmuses können in demselben Streubild mehr Photonen detektiert werden, als mit dem Schwellenwert-Algorithmus (vgl. Abb. \ref{fig:capture_ped_diff}d).


\noindent
Auf diese Weise werden \SI{50000}{\captures} einzeln ausgewertet und aufsummiert. In Abb. \ref{fig:cl_2_150_170_180_200_220_250_resonance} sind die Summen von ausgewerteten Aufnahmen in log-Skala dargestellt. Für die Auswertung jeder Summen wird ein Schwellenwert $s_Q$ aus dem Intervall von \SIrange{150}{250}{\adu} genommen.
\begin{figure}[H]
    \centering
    \input{images/auswertung/cl_2_150_170_180_200_220_250_resonance.pgf}
    \caption{Anzahl von den detektierten Photonen mithilfe des Clustering-Algorithmuses mit Cluster-Kern $\mathbf{K}_2$ und dem Schwellenwert $s_Q$ (a) \SI{150}{\adu}, (b) \SI{170}{\adu}, (c) \SI{180}{\adu}, (d) \SI{200}{\adu}, (e) \SI{220}{\adu} und (f) \SI{250}{\adu}. Aufsummiert werden jeweils \num{50000} Aufnahmen. Aufgenommen an der Gd M5 Resonanzfrequenz.}
    \label{fig:cl_2_150_170_180_200_220_250_resonance}
\end{figure}
\noindent
Man kann in der Auswertung der Daten aufgenommene an der resonanten Photonenenergie das ringähnliche Muster erkennen, aber es ist nötig, zunächst einen Blick auf die Messdaten an der nichtresonanten Photonenenergie $h\nu_\text{Gd, Off-Res}$ zu werfen.
\begin{figure}[H]
    \centering
    \input{images/auswertung/cl_150_170_180_200_220_250_off_resonance.pgf}
    \caption{Anzahl von den detektierten Photonen mithilfe des Clustering-Algorithmuses mit Cluster-Kern $\mathbf{K}_2$ und dem Schwellenwert $s_Q$ (a) \SI{150}{\adu}, (b) \SI{170}{\adu}, (c) \SI{180}{\adu}, (d) \SI{200}{\adu}, (e) \SI{220}{\adu} und (f) \SI{250}{\adu}. Aufsummiert werden jeweils \num{50000} Aufnahmen. Aufgenommen an der Photonenenergie $h\nu_\text{Gd, Off-Res} \approx \SI{1163}{\eV}$.}
    \label{fig:cl_150_170_180_200_220_250_off_resonance}
\end{figure}
\noindent
Die Anzahl von fehldetektierten Photonen ist so hoch, dass sich die vergleichbaren Photonenzahlen im Bereich des Streuringes an den resonanten und nichtresonanten Photonenergien ergeben. Dazu können am wahrscheinlichsten zwei Aspekte beitragen. 

\noindent
Der erste Faktor ist Detektorrauschen mit der Standardabweichung $\sigma_R = \SI{19.94}{\adu}$, die vergleichbar mit dem Ein-Photon-Signal $W_\text{Gd, M5} = \SI{180}{\adu}$ ist. Wird die Summe eines \qtyproduct{2 x 2}{\px}-Clusters erfasst, ist die Standardabweichung dieser Summe $\sigma_{2\times 2} = \sqrt{4}\sigma_R = 2\sigma_R$. 

\noindent
Als nächstes geht man davon aus, dass das gesamte Ein-Photon-Signal \SI{180}{\adu} eines Photons innerhalb des \qtyproduct{2 x 2}{\px}-Clusters liegt und der Schwellenwert $s_Q$ bis auf \SI{170}{\adu} oder \SI{180}{\adu} erhöht werden kann.

\noindent
Da die Verteilung vom Detektorrauscehn folgt gut einer Gauß-Verteilung $G(W, \mu, \sigma, A)$, kann die Zahl der fehldetektierten Photonen im Bezug auf den eingesetzten Schwellenwert $s_{V/Q}$ und bekannte Standardabweichung vom Detektorrauschen $\sigma$ wie folgt ermittelt werden: 
\begin{equation}
    \text{\gls{fdpa}}_{\text{EW, $s_{V/Q}$}}=\int_{s_{V/Q}}^\infty G(W, 0, \sigma, A) \,dW = \frac{A}{2}\left[1 - \erf\left({\frac{s_{V/Q}}{\sqrt{2}\sigma}}\right)\right]
\end{equation}

\noindent
Im Falle eines \qtyproduct{2 x 2}{\px}-Clusters mit der resultierenden Standardabweichung $\sigma_{2\times 2}$ erhält man bei dem Schwellenwert $s_Q = \SI{170}{\adu}$ der folgende Erwartungswert der fehldetektierten Photonen:
\begin{equation}
     \text{\gls{fdpa}}_{\text{EW, \SI{170}{\adu}}} = 1{,}01A\cdot 10^{-5} %\int_{\SI{170}{\adu}}^\infty G(W, 0, \sigma_{2\times 2}, A) \,dW = \frac{A}{2}\left[1 - \erf\left({\frac{\SI{170}{\adu}}{\sqrt{2}\sigma_{2\times 2}}}\right)\right] 
\end{equation}

\noindent
Wird der Wert $s_Q$ bis auf \SI{180}{\adu} erhöht, dann verringert sich der Erwartungswert der fehldetektierten Photonen:
\begin{equation}
    \text{\gls{fdpa}}_{\text{EW, \SI{180}{\adu}}} = 3{,}19A\cdot 10^{-6} %\int_{\SI{180}{\adu}}^\infty G(W, 0, \sigma_{2\times 2}, A) \,dW = \frac{A}{2}\left[1 - \erf\left({\frac{\SI{180}{\adu}}{\sqrt{2}\sigma_{2\times 2}}}\right)\right]
\end{equation}

\noindent
Wird der Schwellenwert-Algorithmus mit dem Wert $s_V = \SI{100}{\adu}$ eingesetzt, ist die Standardabweichung $\sigma_{1\times 1} = \sigma_{R}$ und der Erwartungswert der fehldetektierten Photonen gleicht:
\begin{equation}
   \text{\gls{fdpa}}_{\text{EW, \SI{110}{\adu}}} = 2{,}65A\cdot 10^{-7} %\int_{\SI{100}{\adu}}^\infty G(W, 0, \sigma_{1\times 1}, A) \,dW = \frac{A}{2}\left[1 - \erf\left({\frac{\SI{100}{\adu}}{\sqrt{2}\sigma_{1\times 1}}}\right)\right] 
\end{equation}

\noindent
Man sieht, dass die erwartete Zahl der fehldetektierten Photonen bei dem Ansatz des Clustering-Al\-go\-rith\-muses im besten Fall ca. 10 höher als ohne Clustering ist. 

\noindent
Es werden Histogramme über die Pixelwerte und Summen von \qtyproduct{2 x 2}{\px}-Clusters von \num{10000} Dunkel- und Streubildern in Abb. \ref{fig:no_pr_cl_2_histograms} aufgetragen. In der Regel können mehrere Peaks in einem Histogramm über die Pixelwerte identifiziert werden. So liegt typischerweise der erste Peak bei \SI{0}{\adu} und entspricht dem Detektorrauschen. Weiter sollen die äquidistanten Peaks folgen, die einem, zwei oder mehr Photonen-Ereignissen entsprechen.
\begin{figure}[H]
    \centering
    %% Creator: Matplotlib, PGF backend
%%
%% To include the figure in your LaTeX document, write
%%   \input{<filename>.pgf}
%%
%% Make sure the required packages are loaded in your preamble
%%   \usepackage{pgf}
%%
%% Also ensure that all the required font packages are loaded; for instance,
%% the lmodern package is sometimes necessary when using math font.
%%   \usepackage{lmodern}
%%
%% Figures using additional raster images can only be included by \input if
%% they are in the same directory as the main LaTeX file. For loading figures
%% from other directories you can use the `import` package
%%   \usepackage{import}
%%
%% and then include the figures with
%%   \import{<path to file>}{<filename>.pgf}
%%
%% Matplotlib used the following preamble
%%   \usepackage{amsmath} \usepackage[utf8]{inputenc} \usepackage[T1]{fontenc} \usepackage[output-decimal-marker={,},print-unity-mantissa=false]{siunitx} \sisetup{per-mode=fraction, separate-uncertainty = true, locale = DE} \usepackage[acronym, toc, section=section, nonumberlist, nopostdot]{glossaries-extra} \DeclareSIUnit\adu{\text{ADU}} \DeclareSIUnit\px{\text{px}} \DeclareSIUnit\photons{\text{Pho\-to\-nen}} \DeclareSIUnit\photon{\text{Pho\-ton}}
%%
\begingroup%
\makeatletter%
\begin{pgfpicture}%
\pgfpathrectangle{\pgfpointorigin}{\pgfqpoint{6.235591in}{4.204223in}}%
\pgfusepath{use as bounding box, clip}%
\begin{pgfscope}%
\pgfsetbuttcap%
\pgfsetmiterjoin%
\pgfsetlinewidth{0.000000pt}%
\definecolor{currentstroke}{rgb}{1.000000,1.000000,1.000000}%
\pgfsetstrokecolor{currentstroke}%
\pgfsetstrokeopacity{0.000000}%
\pgfsetdash{}{0pt}%
\pgfpathmoveto{\pgfqpoint{0.000000in}{0.000000in}}%
\pgfpathlineto{\pgfqpoint{6.235591in}{0.000000in}}%
\pgfpathlineto{\pgfqpoint{6.235591in}{4.204223in}}%
\pgfpathlineto{\pgfqpoint{0.000000in}{4.204223in}}%
\pgfpathlineto{\pgfqpoint{0.000000in}{0.000000in}}%
\pgfpathclose%
\pgfusepath{}%
\end{pgfscope}%
\begin{pgfscope}%
\pgfsetbuttcap%
\pgfsetmiterjoin%
\definecolor{currentfill}{rgb}{1.000000,1.000000,1.000000}%
\pgfsetfillcolor{currentfill}%
\pgfsetlinewidth{0.000000pt}%
\definecolor{currentstroke}{rgb}{0.000000,0.000000,0.000000}%
\pgfsetstrokecolor{currentstroke}%
\pgfsetstrokeopacity{0.000000}%
\pgfsetdash{}{0pt}%
\pgfpathmoveto{\pgfqpoint{0.557402in}{2.573088in}}%
\pgfpathlineto{\pgfqpoint{6.131424in}{2.573088in}}%
\pgfpathlineto{\pgfqpoint{6.131424in}{3.961264in}}%
\pgfpathlineto{\pgfqpoint{0.557402in}{3.961264in}}%
\pgfpathlineto{\pgfqpoint{0.557402in}{2.573088in}}%
\pgfpathclose%
\pgfusepath{fill}%
\end{pgfscope}%
\begin{pgfscope}%
\pgfsetbuttcap%
\pgfsetroundjoin%
\definecolor{currentfill}{rgb}{0.000000,0.000000,0.000000}%
\pgfsetfillcolor{currentfill}%
\pgfsetlinewidth{0.803000pt}%
\definecolor{currentstroke}{rgb}{0.000000,0.000000,0.000000}%
\pgfsetstrokecolor{currentstroke}%
\pgfsetdash{}{0pt}%
\pgfsys@defobject{currentmarker}{\pgfqpoint{0.000000in}{-0.048611in}}{\pgfqpoint{0.000000in}{0.000000in}}{%
\pgfpathmoveto{\pgfqpoint{0.000000in}{0.000000in}}%
\pgfpathlineto{\pgfqpoint{0.000000in}{-0.048611in}}%
\pgfusepath{stroke,fill}%
}%
\begin{pgfscope}%
\pgfsys@transformshift{0.666697in}{2.573088in}%
\pgfsys@useobject{currentmarker}{}%
\end{pgfscope}%
\end{pgfscope}%
\begin{pgfscope}%
\definecolor{textcolor}{rgb}{0.000000,0.000000,0.000000}%
\pgfsetstrokecolor{textcolor}%
\pgfsetfillcolor{textcolor}%
\pgftext[x=0.666697in,y=2.475866in,,top]{\color{textcolor}\rmfamily\fontsize{10.000000}{12.000000}\selectfont \(\displaystyle {\ensuremath{-}200}\)}%
\end{pgfscope}%
\begin{pgfscope}%
\pgfsetbuttcap%
\pgfsetroundjoin%
\definecolor{currentfill}{rgb}{0.000000,0.000000,0.000000}%
\pgfsetfillcolor{currentfill}%
\pgfsetlinewidth{0.803000pt}%
\definecolor{currentstroke}{rgb}{0.000000,0.000000,0.000000}%
\pgfsetstrokecolor{currentstroke}%
\pgfsetdash{}{0pt}%
\pgfsys@defobject{currentmarker}{\pgfqpoint{0.000000in}{-0.048611in}}{\pgfqpoint{0.000000in}{0.000000in}}{%
\pgfpathmoveto{\pgfqpoint{0.000000in}{0.000000in}}%
\pgfpathlineto{\pgfqpoint{0.000000in}{-0.048611in}}%
\pgfusepath{stroke,fill}%
}%
\begin{pgfscope}%
\pgfsys@transformshift{1.759642in}{2.573088in}%
\pgfsys@useobject{currentmarker}{}%
\end{pgfscope}%
\end{pgfscope}%
\begin{pgfscope}%
\definecolor{textcolor}{rgb}{0.000000,0.000000,0.000000}%
\pgfsetstrokecolor{textcolor}%
\pgfsetfillcolor{textcolor}%
\pgftext[x=1.759642in,y=2.475866in,,top]{\color{textcolor}\rmfamily\fontsize{10.000000}{12.000000}\selectfont \(\displaystyle {\ensuremath{-}100}\)}%
\end{pgfscope}%
\begin{pgfscope}%
\pgfsetbuttcap%
\pgfsetroundjoin%
\definecolor{currentfill}{rgb}{0.000000,0.000000,0.000000}%
\pgfsetfillcolor{currentfill}%
\pgfsetlinewidth{0.803000pt}%
\definecolor{currentstroke}{rgb}{0.000000,0.000000,0.000000}%
\pgfsetstrokecolor{currentstroke}%
\pgfsetdash{}{0pt}%
\pgfsys@defobject{currentmarker}{\pgfqpoint{0.000000in}{-0.048611in}}{\pgfqpoint{0.000000in}{0.000000in}}{%
\pgfpathmoveto{\pgfqpoint{0.000000in}{0.000000in}}%
\pgfpathlineto{\pgfqpoint{0.000000in}{-0.048611in}}%
\pgfusepath{stroke,fill}%
}%
\begin{pgfscope}%
\pgfsys@transformshift{2.852588in}{2.573088in}%
\pgfsys@useobject{currentmarker}{}%
\end{pgfscope}%
\end{pgfscope}%
\begin{pgfscope}%
\definecolor{textcolor}{rgb}{0.000000,0.000000,0.000000}%
\pgfsetstrokecolor{textcolor}%
\pgfsetfillcolor{textcolor}%
\pgftext[x=2.852588in,y=2.475866in,,top]{\color{textcolor}\rmfamily\fontsize{10.000000}{12.000000}\selectfont \(\displaystyle {0}\)}%
\end{pgfscope}%
\begin{pgfscope}%
\pgfsetbuttcap%
\pgfsetroundjoin%
\definecolor{currentfill}{rgb}{0.000000,0.000000,0.000000}%
\pgfsetfillcolor{currentfill}%
\pgfsetlinewidth{0.803000pt}%
\definecolor{currentstroke}{rgb}{0.000000,0.000000,0.000000}%
\pgfsetstrokecolor{currentstroke}%
\pgfsetdash{}{0pt}%
\pgfsys@defobject{currentmarker}{\pgfqpoint{0.000000in}{-0.048611in}}{\pgfqpoint{0.000000in}{0.000000in}}{%
\pgfpathmoveto{\pgfqpoint{0.000000in}{0.000000in}}%
\pgfpathlineto{\pgfqpoint{0.000000in}{-0.048611in}}%
\pgfusepath{stroke,fill}%
}%
\begin{pgfscope}%
\pgfsys@transformshift{3.945533in}{2.573088in}%
\pgfsys@useobject{currentmarker}{}%
\end{pgfscope}%
\end{pgfscope}%
\begin{pgfscope}%
\definecolor{textcolor}{rgb}{0.000000,0.000000,0.000000}%
\pgfsetstrokecolor{textcolor}%
\pgfsetfillcolor{textcolor}%
\pgftext[x=3.945533in,y=2.475866in,,top]{\color{textcolor}\rmfamily\fontsize{10.000000}{12.000000}\selectfont \(\displaystyle {100}\)}%
\end{pgfscope}%
\begin{pgfscope}%
\pgfsetbuttcap%
\pgfsetroundjoin%
\definecolor{currentfill}{rgb}{0.000000,0.000000,0.000000}%
\pgfsetfillcolor{currentfill}%
\pgfsetlinewidth{0.803000pt}%
\definecolor{currentstroke}{rgb}{0.000000,0.000000,0.000000}%
\pgfsetstrokecolor{currentstroke}%
\pgfsetdash{}{0pt}%
\pgfsys@defobject{currentmarker}{\pgfqpoint{0.000000in}{-0.048611in}}{\pgfqpoint{0.000000in}{0.000000in}}{%
\pgfpathmoveto{\pgfqpoint{0.000000in}{0.000000in}}%
\pgfpathlineto{\pgfqpoint{0.000000in}{-0.048611in}}%
\pgfusepath{stroke,fill}%
}%
\begin{pgfscope}%
\pgfsys@transformshift{5.038479in}{2.573088in}%
\pgfsys@useobject{currentmarker}{}%
\end{pgfscope}%
\end{pgfscope}%
\begin{pgfscope}%
\definecolor{textcolor}{rgb}{0.000000,0.000000,0.000000}%
\pgfsetstrokecolor{textcolor}%
\pgfsetfillcolor{textcolor}%
\pgftext[x=5.038479in,y=2.475866in,,top]{\color{textcolor}\rmfamily\fontsize{10.000000}{12.000000}\selectfont \(\displaystyle {200}\)}%
\end{pgfscope}%
\begin{pgfscope}%
\pgfsetbuttcap%
\pgfsetroundjoin%
\definecolor{currentfill}{rgb}{0.000000,0.000000,0.000000}%
\pgfsetfillcolor{currentfill}%
\pgfsetlinewidth{0.803000pt}%
\definecolor{currentstroke}{rgb}{0.000000,0.000000,0.000000}%
\pgfsetstrokecolor{currentstroke}%
\pgfsetdash{}{0pt}%
\pgfsys@defobject{currentmarker}{\pgfqpoint{0.000000in}{-0.048611in}}{\pgfqpoint{0.000000in}{0.000000in}}{%
\pgfpathmoveto{\pgfqpoint{0.000000in}{0.000000in}}%
\pgfpathlineto{\pgfqpoint{0.000000in}{-0.048611in}}%
\pgfusepath{stroke,fill}%
}%
\begin{pgfscope}%
\pgfsys@transformshift{6.131424in}{2.573088in}%
\pgfsys@useobject{currentmarker}{}%
\end{pgfscope}%
\end{pgfscope}%
\begin{pgfscope}%
\definecolor{textcolor}{rgb}{0.000000,0.000000,0.000000}%
\pgfsetstrokecolor{textcolor}%
\pgfsetfillcolor{textcolor}%
\pgftext[x=6.131424in,y=2.475866in,,top]{\color{textcolor}\rmfamily\fontsize{10.000000}{12.000000}\selectfont \(\displaystyle {300}\)}%
\end{pgfscope}%
\begin{pgfscope}%
\definecolor{textcolor}{rgb}{0.000000,0.000000,0.000000}%
\pgfsetstrokecolor{textcolor}%
\pgfsetfillcolor{textcolor}%
\pgftext[x=3.344413in,y=2.297655in,,top]{\color{textcolor}\rmfamily\fontsize{10.000000}{12.000000}\selectfont Pixelwert \(\displaystyle W\) in ADU}%
\end{pgfscope}%
\begin{pgfscope}%
\pgfsetbuttcap%
\pgfsetroundjoin%
\definecolor{currentfill}{rgb}{0.000000,0.000000,0.000000}%
\pgfsetfillcolor{currentfill}%
\pgfsetlinewidth{0.803000pt}%
\definecolor{currentstroke}{rgb}{0.000000,0.000000,0.000000}%
\pgfsetstrokecolor{currentstroke}%
\pgfsetdash{}{0pt}%
\pgfsys@defobject{currentmarker}{\pgfqpoint{-0.048611in}{0.000000in}}{\pgfqpoint{-0.000000in}{0.000000in}}{%
\pgfpathmoveto{\pgfqpoint{-0.000000in}{0.000000in}}%
\pgfpathlineto{\pgfqpoint{-0.048611in}{0.000000in}}%
\pgfusepath{stroke,fill}%
}%
\begin{pgfscope}%
\pgfsys@transformshift{0.557402in}{2.636553in}%
\pgfsys@useobject{currentmarker}{}%
\end{pgfscope}%
\end{pgfscope}%
\begin{pgfscope}%
\definecolor{textcolor}{rgb}{0.000000,0.000000,0.000000}%
\pgfsetstrokecolor{textcolor}%
\pgfsetfillcolor{textcolor}%
\pgftext[x=0.282710in, y=2.588728in, left, base]{\color{textcolor}\rmfamily\fontsize{10.000000}{12.000000}\selectfont \num{0.0}}%
\end{pgfscope}%
\begin{pgfscope}%
\pgfsetbuttcap%
\pgfsetroundjoin%
\definecolor{currentfill}{rgb}{0.000000,0.000000,0.000000}%
\pgfsetfillcolor{currentfill}%
\pgfsetlinewidth{0.803000pt}%
\definecolor{currentstroke}{rgb}{0.000000,0.000000,0.000000}%
\pgfsetstrokecolor{currentstroke}%
\pgfsetdash{}{0pt}%
\pgfsys@defobject{currentmarker}{\pgfqpoint{-0.048611in}{0.000000in}}{\pgfqpoint{-0.000000in}{0.000000in}}{%
\pgfpathmoveto{\pgfqpoint{-0.000000in}{0.000000in}}%
\pgfpathlineto{\pgfqpoint{-0.048611in}{0.000000in}}%
\pgfusepath{stroke,fill}%
}%
\begin{pgfscope}%
\pgfsys@transformshift{0.557402in}{3.031530in}%
\pgfsys@useobject{currentmarker}{}%
\end{pgfscope}%
\end{pgfscope}%
\begin{pgfscope}%
\definecolor{textcolor}{rgb}{0.000000,0.000000,0.000000}%
\pgfsetstrokecolor{textcolor}%
\pgfsetfillcolor{textcolor}%
\pgftext[x=0.282710in, y=2.983705in, left, base]{\color{textcolor}\rmfamily\fontsize{10.000000}{12.000000}\selectfont \num{0.1}}%
\end{pgfscope}%
\begin{pgfscope}%
\pgfsetbuttcap%
\pgfsetroundjoin%
\definecolor{currentfill}{rgb}{0.000000,0.000000,0.000000}%
\pgfsetfillcolor{currentfill}%
\pgfsetlinewidth{0.803000pt}%
\definecolor{currentstroke}{rgb}{0.000000,0.000000,0.000000}%
\pgfsetstrokecolor{currentstroke}%
\pgfsetdash{}{0pt}%
\pgfsys@defobject{currentmarker}{\pgfqpoint{-0.048611in}{0.000000in}}{\pgfqpoint{-0.000000in}{0.000000in}}{%
\pgfpathmoveto{\pgfqpoint{-0.000000in}{0.000000in}}%
\pgfpathlineto{\pgfqpoint{-0.048611in}{0.000000in}}%
\pgfusepath{stroke,fill}%
}%
\begin{pgfscope}%
\pgfsys@transformshift{0.557402in}{3.426507in}%
\pgfsys@useobject{currentmarker}{}%
\end{pgfscope}%
\end{pgfscope}%
\begin{pgfscope}%
\definecolor{textcolor}{rgb}{0.000000,0.000000,0.000000}%
\pgfsetstrokecolor{textcolor}%
\pgfsetfillcolor{textcolor}%
\pgftext[x=0.282710in, y=3.378682in, left, base]{\color{textcolor}\rmfamily\fontsize{10.000000}{12.000000}\selectfont \num{0.2}}%
\end{pgfscope}%
\begin{pgfscope}%
\pgfsetbuttcap%
\pgfsetroundjoin%
\definecolor{currentfill}{rgb}{0.000000,0.000000,0.000000}%
\pgfsetfillcolor{currentfill}%
\pgfsetlinewidth{0.803000pt}%
\definecolor{currentstroke}{rgb}{0.000000,0.000000,0.000000}%
\pgfsetstrokecolor{currentstroke}%
\pgfsetdash{}{0pt}%
\pgfsys@defobject{currentmarker}{\pgfqpoint{-0.048611in}{0.000000in}}{\pgfqpoint{-0.000000in}{0.000000in}}{%
\pgfpathmoveto{\pgfqpoint{-0.000000in}{0.000000in}}%
\pgfpathlineto{\pgfqpoint{-0.048611in}{0.000000in}}%
\pgfusepath{stroke,fill}%
}%
\begin{pgfscope}%
\pgfsys@transformshift{0.557402in}{3.821484in}%
\pgfsys@useobject{currentmarker}{}%
\end{pgfscope}%
\end{pgfscope}%
\begin{pgfscope}%
\definecolor{textcolor}{rgb}{0.000000,0.000000,0.000000}%
\pgfsetstrokecolor{textcolor}%
\pgfsetfillcolor{textcolor}%
\pgftext[x=0.282710in, y=3.773659in, left, base]{\color{textcolor}\rmfamily\fontsize{10.000000}{12.000000}\selectfont \num{0.3}}%
\end{pgfscope}%
\begin{pgfscope}%
\definecolor{textcolor}{rgb}{0.000000,0.000000,0.000000}%
\pgfsetstrokecolor{textcolor}%
\pgfsetfillcolor{textcolor}%
\pgftext[x=0.227155in,y=3.267176in,,bottom,rotate=90.000000]{\color{textcolor}\rmfamily\fontsize{10.000000}{12.000000}\selectfont Pixelzahl}%
\end{pgfscope}%
\begin{pgfscope}%
\definecolor{textcolor}{rgb}{0.000000,0.000000,0.000000}%
\pgfsetstrokecolor{textcolor}%
\pgfsetfillcolor{textcolor}%
\pgftext[x=0.557402in,y=4.002931in,left,base]{\color{textcolor}\rmfamily\fontsize{10.000000}{12.000000}\selectfont \(\displaystyle \times{10^{8}}{}\)}%
\end{pgfscope}%
\begin{pgfscope}%
\pgfpathrectangle{\pgfqpoint{0.557402in}{2.573088in}}{\pgfqpoint{5.574022in}{1.388176in}}%
\pgfusepath{clip}%
\pgfsetrectcap%
\pgfsetroundjoin%
\pgfsetlinewidth{1.505625pt}%
\definecolor{currentstroke}{rgb}{0.121569,0.466667,0.705882}%
\pgfsetstrokecolor{currentstroke}%
\pgfsetdash{}{0pt}%
\pgfpathmoveto{\pgfqpoint{0.666697in}{2.636553in}}%
\pgfpathlineto{\pgfqpoint{1.737783in}{2.636761in}}%
\pgfpathlineto{\pgfqpoint{1.836148in}{2.637124in}}%
\pgfpathlineto{\pgfqpoint{1.912655in}{2.637780in}}%
\pgfpathlineto{\pgfqpoint{1.967302in}{2.638651in}}%
\pgfpathlineto{\pgfqpoint{2.021949in}{2.640184in}}%
\pgfpathlineto{\pgfqpoint{2.054738in}{2.641633in}}%
\pgfpathlineto{\pgfqpoint{2.076596in}{2.642820in}}%
\pgfpathlineto{\pgfqpoint{2.098455in}{2.644406in}}%
\pgfpathlineto{\pgfqpoint{2.120314in}{2.646381in}}%
\pgfpathlineto{\pgfqpoint{2.142173in}{2.648836in}}%
\pgfpathlineto{\pgfqpoint{2.164032in}{2.651961in}}%
\pgfpathlineto{\pgfqpoint{2.185891in}{2.655861in}}%
\pgfpathlineto{\pgfqpoint{2.196821in}{2.658123in}}%
\pgfpathlineto{\pgfqpoint{2.218679in}{2.663526in}}%
\pgfpathlineto{\pgfqpoint{2.229609in}{2.666734in}}%
\pgfpathlineto{\pgfqpoint{2.240538in}{2.670343in}}%
\pgfpathlineto{\pgfqpoint{2.251468in}{2.674202in}}%
\pgfpathlineto{\pgfqpoint{2.262397in}{2.678686in}}%
\pgfpathlineto{\pgfqpoint{2.273327in}{2.683543in}}%
\pgfpathlineto{\pgfqpoint{2.284256in}{2.688912in}}%
\pgfpathlineto{\pgfqpoint{2.295186in}{2.694907in}}%
\pgfpathlineto{\pgfqpoint{2.306115in}{2.701428in}}%
\pgfpathlineto{\pgfqpoint{2.317045in}{2.708942in}}%
\pgfpathlineto{\pgfqpoint{2.327974in}{2.716951in}}%
\pgfpathlineto{\pgfqpoint{2.338903in}{2.725771in}}%
\pgfpathlineto{\pgfqpoint{2.349833in}{2.735313in}}%
\pgfpathlineto{\pgfqpoint{2.360762in}{2.745981in}}%
\pgfpathlineto{\pgfqpoint{2.371692in}{2.757583in}}%
\pgfpathlineto{\pgfqpoint{2.382621in}{2.770104in}}%
\pgfpathlineto{\pgfqpoint{2.393551in}{2.783644in}}%
\pgfpathlineto{\pgfqpoint{2.404480in}{2.798529in}}%
\pgfpathlineto{\pgfqpoint{2.415410in}{2.814451in}}%
\pgfpathlineto{\pgfqpoint{2.426339in}{2.831723in}}%
\pgfpathlineto{\pgfqpoint{2.437269in}{2.850091in}}%
\pgfpathlineto{\pgfqpoint{2.448198in}{2.869995in}}%
\pgfpathlineto{\pgfqpoint{2.459127in}{2.891131in}}%
\pgfpathlineto{\pgfqpoint{2.470057in}{2.913760in}}%
\pgfpathlineto{\pgfqpoint{2.480986in}{2.937797in}}%
\pgfpathlineto{\pgfqpoint{2.491916in}{2.963337in}}%
\pgfpathlineto{\pgfqpoint{2.502845in}{2.990020in}}%
\pgfpathlineto{\pgfqpoint{2.513775in}{3.018083in}}%
\pgfpathlineto{\pgfqpoint{2.524704in}{3.047796in}}%
\pgfpathlineto{\pgfqpoint{2.535634in}{3.078541in}}%
\pgfpathlineto{\pgfqpoint{2.557493in}{3.144291in}}%
\pgfpathlineto{\pgfqpoint{2.579351in}{3.213985in}}%
\pgfpathlineto{\pgfqpoint{2.590281in}{3.250467in}}%
\pgfpathlineto{\pgfqpoint{2.612140in}{3.325377in}}%
\pgfpathlineto{\pgfqpoint{2.666787in}{3.516911in}}%
\pgfpathlineto{\pgfqpoint{2.688646in}{3.590523in}}%
\pgfpathlineto{\pgfqpoint{2.699575in}{3.626227in}}%
\pgfpathlineto{\pgfqpoint{2.710505in}{3.660359in}}%
\pgfpathlineto{\pgfqpoint{2.721434in}{3.693245in}}%
\pgfpathlineto{\pgfqpoint{2.732364in}{3.724071in}}%
\pgfpathlineto{\pgfqpoint{2.743293in}{3.752943in}}%
\pgfpathlineto{\pgfqpoint{2.754223in}{3.780184in}}%
\pgfpathlineto{\pgfqpoint{2.765152in}{3.805003in}}%
\pgfpathlineto{\pgfqpoint{2.776082in}{3.826703in}}%
\pgfpathlineto{\pgfqpoint{2.787011in}{3.845467in}}%
\pgfpathlineto{\pgfqpoint{2.797941in}{3.862500in}}%
\pgfpathlineto{\pgfqpoint{2.808870in}{3.876172in}}%
\pgfpathlineto{\pgfqpoint{2.819799in}{3.885971in}}%
\pgfpathlineto{\pgfqpoint{2.830729in}{3.893809in}}%
\pgfpathlineto{\pgfqpoint{2.841658in}{3.896752in}}%
\pgfpathlineto{\pgfqpoint{2.852588in}{3.898165in}}%
\pgfpathlineto{\pgfqpoint{2.863517in}{3.894488in}}%
\pgfpathlineto{\pgfqpoint{2.874447in}{3.888245in}}%
\pgfpathlineto{\pgfqpoint{2.885376in}{3.879277in}}%
\pgfpathlineto{\pgfqpoint{2.896306in}{3.867496in}}%
\pgfpathlineto{\pgfqpoint{2.907235in}{3.851458in}}%
\pgfpathlineto{\pgfqpoint{2.918165in}{3.833139in}}%
\pgfpathlineto{\pgfqpoint{2.929094in}{3.812527in}}%
\pgfpathlineto{\pgfqpoint{2.940023in}{3.788785in}}%
\pgfpathlineto{\pgfqpoint{2.950953in}{3.762767in}}%
\pgfpathlineto{\pgfqpoint{2.961882in}{3.733935in}}%
\pgfpathlineto{\pgfqpoint{2.972812in}{3.703456in}}%
\pgfpathlineto{\pgfqpoint{2.983741in}{3.671545in}}%
\pgfpathlineto{\pgfqpoint{2.994671in}{3.637858in}}%
\pgfpathlineto{\pgfqpoint{3.005600in}{3.602869in}}%
\pgfpathlineto{\pgfqpoint{3.016530in}{3.566777in}}%
\pgfpathlineto{\pgfqpoint{3.027459in}{3.529815in}}%
\pgfpathlineto{\pgfqpoint{3.093036in}{3.300971in}}%
\pgfpathlineto{\pgfqpoint{3.103965in}{3.263490in}}%
\pgfpathlineto{\pgfqpoint{3.125824in}{3.191331in}}%
\pgfpathlineto{\pgfqpoint{3.136754in}{3.156188in}}%
\pgfpathlineto{\pgfqpoint{3.147683in}{3.122836in}}%
\pgfpathlineto{\pgfqpoint{3.158613in}{3.090331in}}%
\pgfpathlineto{\pgfqpoint{3.169542in}{3.059026in}}%
\pgfpathlineto{\pgfqpoint{3.180471in}{3.028846in}}%
\pgfpathlineto{\pgfqpoint{3.191401in}{3.000299in}}%
\pgfpathlineto{\pgfqpoint{3.202330in}{2.973163in}}%
\pgfpathlineto{\pgfqpoint{3.213260in}{2.947175in}}%
\pgfpathlineto{\pgfqpoint{3.224189in}{2.922728in}}%
\pgfpathlineto{\pgfqpoint{3.235119in}{2.899628in}}%
\pgfpathlineto{\pgfqpoint{3.246048in}{2.878026in}}%
\pgfpathlineto{\pgfqpoint{3.256978in}{2.857646in}}%
\pgfpathlineto{\pgfqpoint{3.267907in}{2.838724in}}%
\pgfpathlineto{\pgfqpoint{3.278837in}{2.821136in}}%
\pgfpathlineto{\pgfqpoint{3.289766in}{2.804779in}}%
\pgfpathlineto{\pgfqpoint{3.300695in}{2.789373in}}%
\pgfpathlineto{\pgfqpoint{3.311625in}{2.775306in}}%
\pgfpathlineto{\pgfqpoint{3.322554in}{2.762531in}}%
\pgfpathlineto{\pgfqpoint{3.333484in}{2.750444in}}%
\pgfpathlineto{\pgfqpoint{3.344413in}{2.739490in}}%
\pgfpathlineto{\pgfqpoint{3.355343in}{2.729550in}}%
\pgfpathlineto{\pgfqpoint{3.366272in}{2.720407in}}%
\pgfpathlineto{\pgfqpoint{3.377202in}{2.712190in}}%
\pgfpathlineto{\pgfqpoint{3.388131in}{2.704494in}}%
\pgfpathlineto{\pgfqpoint{3.399061in}{2.697548in}}%
\pgfpathlineto{\pgfqpoint{3.409990in}{2.691402in}}%
\pgfpathlineto{\pgfqpoint{3.420919in}{2.685749in}}%
\pgfpathlineto{\pgfqpoint{3.431849in}{2.680685in}}%
\pgfpathlineto{\pgfqpoint{3.442778in}{2.676035in}}%
\pgfpathlineto{\pgfqpoint{3.453708in}{2.671904in}}%
\pgfpathlineto{\pgfqpoint{3.475567in}{2.664826in}}%
\pgfpathlineto{\pgfqpoint{3.486496in}{2.661884in}}%
\pgfpathlineto{\pgfqpoint{3.508355in}{2.656779in}}%
\pgfpathlineto{\pgfqpoint{3.519285in}{2.654631in}}%
\pgfpathlineto{\pgfqpoint{3.530214in}{2.652726in}}%
\pgfpathlineto{\pgfqpoint{3.552073in}{2.649424in}}%
\pgfpathlineto{\pgfqpoint{3.573932in}{2.646841in}}%
\pgfpathlineto{\pgfqpoint{3.595791in}{2.644773in}}%
\pgfpathlineto{\pgfqpoint{3.617650in}{2.643148in}}%
\pgfpathlineto{\pgfqpoint{3.650438in}{2.641265in}}%
\pgfpathlineto{\pgfqpoint{3.683226in}{2.639959in}}%
\pgfpathlineto{\pgfqpoint{3.716015in}{2.638982in}}%
\pgfpathlineto{\pgfqpoint{3.759733in}{2.638130in}}%
\pgfpathlineto{\pgfqpoint{3.814380in}{2.637453in}}%
\pgfpathlineto{\pgfqpoint{3.890886in}{2.636958in}}%
\pgfpathlineto{\pgfqpoint{4.000181in}{2.636684in}}%
\pgfpathlineto{\pgfqpoint{4.218770in}{2.636564in}}%
\pgfpathlineto{\pgfqpoint{6.133424in}{2.636553in}}%
\pgfpathlineto{\pgfqpoint{6.133424in}{2.636553in}}%
\pgfusepath{stroke}%
\end{pgfscope}%
\begin{pgfscope}%
\pgfpathrectangle{\pgfqpoint{0.557402in}{2.573088in}}{\pgfqpoint{5.574022in}{1.388176in}}%
\pgfusepath{clip}%
\pgfsetrectcap%
\pgfsetroundjoin%
\pgfsetlinewidth{1.505625pt}%
\definecolor{currentstroke}{rgb}{1.000000,0.498039,0.054902}%
\pgfsetstrokecolor{currentstroke}%
\pgfsetdash{}{0pt}%
\pgfpathmoveto{\pgfqpoint{0.666697in}{2.636554in}}%
\pgfpathlineto{\pgfqpoint{1.213170in}{2.636753in}}%
\pgfpathlineto{\pgfqpoint{1.311535in}{2.637245in}}%
\pgfpathlineto{\pgfqpoint{1.377111in}{2.638072in}}%
\pgfpathlineto{\pgfqpoint{1.420829in}{2.639072in}}%
\pgfpathlineto{\pgfqpoint{1.464547in}{2.640643in}}%
\pgfpathlineto{\pgfqpoint{1.497335in}{2.642328in}}%
\pgfpathlineto{\pgfqpoint{1.530124in}{2.644644in}}%
\pgfpathlineto{\pgfqpoint{1.551983in}{2.646574in}}%
\pgfpathlineto{\pgfqpoint{1.573842in}{2.648860in}}%
\pgfpathlineto{\pgfqpoint{1.606630in}{2.653204in}}%
\pgfpathlineto{\pgfqpoint{1.628489in}{2.656770in}}%
\pgfpathlineto{\pgfqpoint{1.639418in}{2.658661in}}%
\pgfpathlineto{\pgfqpoint{1.661277in}{2.663143in}}%
\pgfpathlineto{\pgfqpoint{1.683136in}{2.668179in}}%
\pgfpathlineto{\pgfqpoint{1.704995in}{2.674042in}}%
\pgfpathlineto{\pgfqpoint{1.715924in}{2.677196in}}%
\pgfpathlineto{\pgfqpoint{1.737783in}{2.684215in}}%
\pgfpathlineto{\pgfqpoint{1.759642in}{2.692089in}}%
\pgfpathlineto{\pgfqpoint{1.781501in}{2.700872in}}%
\pgfpathlineto{\pgfqpoint{1.803360in}{2.710593in}}%
\pgfpathlineto{\pgfqpoint{1.825219in}{2.721125in}}%
\pgfpathlineto{\pgfqpoint{1.836148in}{2.726805in}}%
\pgfpathlineto{\pgfqpoint{1.868937in}{2.745236in}}%
\pgfpathlineto{\pgfqpoint{1.879866in}{2.751927in}}%
\pgfpathlineto{\pgfqpoint{1.901725in}{2.765771in}}%
\pgfpathlineto{\pgfqpoint{1.912655in}{2.773338in}}%
\pgfpathlineto{\pgfqpoint{1.923584in}{2.780597in}}%
\pgfpathlineto{\pgfqpoint{1.934514in}{2.788290in}}%
\pgfpathlineto{\pgfqpoint{1.978231in}{2.821208in}}%
\pgfpathlineto{\pgfqpoint{2.011020in}{2.847963in}}%
\pgfpathlineto{\pgfqpoint{2.043808in}{2.875685in}}%
\pgfpathlineto{\pgfqpoint{2.054738in}{2.885567in}}%
\pgfpathlineto{\pgfqpoint{2.076596in}{2.904748in}}%
\pgfpathlineto{\pgfqpoint{2.087526in}{2.914781in}}%
\pgfpathlineto{\pgfqpoint{2.098455in}{2.924486in}}%
\pgfpathlineto{\pgfqpoint{2.109385in}{2.934942in}}%
\pgfpathlineto{\pgfqpoint{2.131244in}{2.955184in}}%
\pgfpathlineto{\pgfqpoint{2.142173in}{2.965445in}}%
\pgfpathlineto{\pgfqpoint{2.153103in}{2.975243in}}%
\pgfpathlineto{\pgfqpoint{2.164032in}{2.985677in}}%
\pgfpathlineto{\pgfqpoint{2.174962in}{2.995697in}}%
\pgfpathlineto{\pgfqpoint{2.185891in}{3.006371in}}%
\pgfpathlineto{\pgfqpoint{2.240538in}{3.057953in}}%
\pgfpathlineto{\pgfqpoint{2.262397in}{3.078162in}}%
\pgfpathlineto{\pgfqpoint{2.273327in}{3.088985in}}%
\pgfpathlineto{\pgfqpoint{2.284256in}{3.099136in}}%
\pgfpathlineto{\pgfqpoint{2.295186in}{3.109745in}}%
\pgfpathlineto{\pgfqpoint{2.306115in}{3.119353in}}%
\pgfpathlineto{\pgfqpoint{2.317045in}{3.129939in}}%
\pgfpathlineto{\pgfqpoint{2.327974in}{3.139864in}}%
\pgfpathlineto{\pgfqpoint{2.338903in}{3.150155in}}%
\pgfpathlineto{\pgfqpoint{2.393551in}{3.199854in}}%
\pgfpathlineto{\pgfqpoint{2.404480in}{3.209973in}}%
\pgfpathlineto{\pgfqpoint{2.459127in}{3.258107in}}%
\pgfpathlineto{\pgfqpoint{2.470057in}{3.267378in}}%
\pgfpathlineto{\pgfqpoint{2.480986in}{3.277060in}}%
\pgfpathlineto{\pgfqpoint{2.491916in}{3.286174in}}%
\pgfpathlineto{\pgfqpoint{2.513775in}{3.303643in}}%
\pgfpathlineto{\pgfqpoint{2.524704in}{3.312786in}}%
\pgfpathlineto{\pgfqpoint{2.535634in}{3.320698in}}%
\pgfpathlineto{\pgfqpoint{2.546563in}{3.329505in}}%
\pgfpathlineto{\pgfqpoint{2.579351in}{3.351935in}}%
\pgfpathlineto{\pgfqpoint{2.601210in}{3.365841in}}%
\pgfpathlineto{\pgfqpoint{2.612140in}{3.372227in}}%
\pgfpathlineto{\pgfqpoint{2.623069in}{3.378133in}}%
\pgfpathlineto{\pgfqpoint{2.633999in}{3.383719in}}%
\pgfpathlineto{\pgfqpoint{2.644928in}{3.388303in}}%
\pgfpathlineto{\pgfqpoint{2.655858in}{3.392639in}}%
\pgfpathlineto{\pgfqpoint{2.677717in}{3.398927in}}%
\pgfpathlineto{\pgfqpoint{2.699575in}{3.403021in}}%
\pgfpathlineto{\pgfqpoint{2.710505in}{3.403951in}}%
\pgfpathlineto{\pgfqpoint{2.721434in}{3.403976in}}%
\pgfpathlineto{\pgfqpoint{2.732364in}{3.403396in}}%
\pgfpathlineto{\pgfqpoint{2.743293in}{3.401783in}}%
\pgfpathlineto{\pgfqpoint{2.754223in}{3.399799in}}%
\pgfpathlineto{\pgfqpoint{2.765152in}{3.396479in}}%
\pgfpathlineto{\pgfqpoint{2.776082in}{3.392340in}}%
\pgfpathlineto{\pgfqpoint{2.787011in}{3.387224in}}%
\pgfpathlineto{\pgfqpoint{2.797941in}{3.381640in}}%
\pgfpathlineto{\pgfqpoint{2.808870in}{3.374634in}}%
\pgfpathlineto{\pgfqpoint{2.819799in}{3.367182in}}%
\pgfpathlineto{\pgfqpoint{2.830729in}{3.358728in}}%
\pgfpathlineto{\pgfqpoint{2.841658in}{3.348528in}}%
\pgfpathlineto{\pgfqpoint{2.852588in}{3.338774in}}%
\pgfpathlineto{\pgfqpoint{2.863517in}{3.327383in}}%
\pgfpathlineto{\pgfqpoint{2.885376in}{3.302031in}}%
\pgfpathlineto{\pgfqpoint{2.896306in}{3.288454in}}%
\pgfpathlineto{\pgfqpoint{2.918165in}{3.259053in}}%
\pgfpathlineto{\pgfqpoint{2.929094in}{3.243563in}}%
\pgfpathlineto{\pgfqpoint{2.950953in}{3.210553in}}%
\pgfpathlineto{\pgfqpoint{2.961882in}{3.193009in}}%
\pgfpathlineto{\pgfqpoint{2.972812in}{3.174868in}}%
\pgfpathlineto{\pgfqpoint{2.983741in}{3.157578in}}%
\pgfpathlineto{\pgfqpoint{2.994671in}{3.139235in}}%
\pgfpathlineto{\pgfqpoint{3.005600in}{3.120456in}}%
\pgfpathlineto{\pgfqpoint{3.016530in}{3.102429in}}%
\pgfpathlineto{\pgfqpoint{3.071177in}{3.009624in}}%
\pgfpathlineto{\pgfqpoint{3.093036in}{2.973786in}}%
\pgfpathlineto{\pgfqpoint{3.114895in}{2.939213in}}%
\pgfpathlineto{\pgfqpoint{3.136754in}{2.906600in}}%
\pgfpathlineto{\pgfqpoint{3.147683in}{2.890727in}}%
\pgfpathlineto{\pgfqpoint{3.158613in}{2.875481in}}%
\pgfpathlineto{\pgfqpoint{3.169542in}{2.860862in}}%
\pgfpathlineto{\pgfqpoint{3.191401in}{2.833083in}}%
\pgfpathlineto{\pgfqpoint{3.202330in}{2.820252in}}%
\pgfpathlineto{\pgfqpoint{3.213260in}{2.807840in}}%
\pgfpathlineto{\pgfqpoint{3.224189in}{2.795884in}}%
\pgfpathlineto{\pgfqpoint{3.235119in}{2.784474in}}%
\pgfpathlineto{\pgfqpoint{3.246048in}{2.773960in}}%
\pgfpathlineto{\pgfqpoint{3.256978in}{2.763969in}}%
\pgfpathlineto{\pgfqpoint{3.267907in}{2.754376in}}%
\pgfpathlineto{\pgfqpoint{3.278837in}{2.745432in}}%
\pgfpathlineto{\pgfqpoint{3.289766in}{2.736950in}}%
\pgfpathlineto{\pgfqpoint{3.300695in}{2.729116in}}%
\pgfpathlineto{\pgfqpoint{3.311625in}{2.721766in}}%
\pgfpathlineto{\pgfqpoint{3.322554in}{2.714765in}}%
\pgfpathlineto{\pgfqpoint{3.333484in}{2.708408in}}%
\pgfpathlineto{\pgfqpoint{3.344413in}{2.702452in}}%
\pgfpathlineto{\pgfqpoint{3.355343in}{2.696873in}}%
\pgfpathlineto{\pgfqpoint{3.366272in}{2.691752in}}%
\pgfpathlineto{\pgfqpoint{3.377202in}{2.687022in}}%
\pgfpathlineto{\pgfqpoint{3.388131in}{2.682635in}}%
\pgfpathlineto{\pgfqpoint{3.399061in}{2.678618in}}%
\pgfpathlineto{\pgfqpoint{3.409990in}{2.674887in}}%
\pgfpathlineto{\pgfqpoint{3.420919in}{2.671478in}}%
\pgfpathlineto{\pgfqpoint{3.442778in}{2.665443in}}%
\pgfpathlineto{\pgfqpoint{3.453708in}{2.662858in}}%
\pgfpathlineto{\pgfqpoint{3.475567in}{2.658359in}}%
\pgfpathlineto{\pgfqpoint{3.486496in}{2.656315in}}%
\pgfpathlineto{\pgfqpoint{3.508355in}{2.652905in}}%
\pgfpathlineto{\pgfqpoint{3.530214in}{2.650001in}}%
\pgfpathlineto{\pgfqpoint{3.563002in}{2.646705in}}%
\pgfpathlineto{\pgfqpoint{3.584861in}{2.645001in}}%
\pgfpathlineto{\pgfqpoint{3.606720in}{2.643529in}}%
\pgfpathlineto{\pgfqpoint{3.650438in}{2.641417in}}%
\pgfpathlineto{\pgfqpoint{3.694156in}{2.640038in}}%
\pgfpathlineto{\pgfqpoint{3.737874in}{2.639083in}}%
\pgfpathlineto{\pgfqpoint{3.792521in}{2.638360in}}%
\pgfpathlineto{\pgfqpoint{3.901816in}{2.637598in}}%
\pgfpathlineto{\pgfqpoint{4.328064in}{2.636929in}}%
\pgfpathlineto{\pgfqpoint{5.399151in}{2.636561in}}%
\pgfpathlineto{\pgfqpoint{6.133424in}{2.636553in}}%
\pgfpathlineto{\pgfqpoint{6.133424in}{2.636553in}}%
\pgfusepath{stroke}%
\end{pgfscope}%
\begin{pgfscope}%
\pgfsetrectcap%
\pgfsetmiterjoin%
\pgfsetlinewidth{0.803000pt}%
\definecolor{currentstroke}{rgb}{0.000000,0.000000,0.000000}%
\pgfsetstrokecolor{currentstroke}%
\pgfsetdash{}{0pt}%
\pgfpathmoveto{\pgfqpoint{0.557402in}{2.573088in}}%
\pgfpathlineto{\pgfqpoint{0.557402in}{3.961264in}}%
\pgfusepath{stroke}%
\end{pgfscope}%
\begin{pgfscope}%
\pgfsetrectcap%
\pgfsetmiterjoin%
\pgfsetlinewidth{0.803000pt}%
\definecolor{currentstroke}{rgb}{0.000000,0.000000,0.000000}%
\pgfsetstrokecolor{currentstroke}%
\pgfsetdash{}{0pt}%
\pgfpathmoveto{\pgfqpoint{6.131424in}{2.573088in}}%
\pgfpathlineto{\pgfqpoint{6.131424in}{3.961264in}}%
\pgfusepath{stroke}%
\end{pgfscope}%
\begin{pgfscope}%
\pgfsetrectcap%
\pgfsetmiterjoin%
\pgfsetlinewidth{0.803000pt}%
\definecolor{currentstroke}{rgb}{0.000000,0.000000,0.000000}%
\pgfsetstrokecolor{currentstroke}%
\pgfsetdash{}{0pt}%
\pgfpathmoveto{\pgfqpoint{0.557402in}{2.573088in}}%
\pgfpathlineto{\pgfqpoint{6.131424in}{2.573088in}}%
\pgfusepath{stroke}%
\end{pgfscope}%
\begin{pgfscope}%
\pgfsetrectcap%
\pgfsetmiterjoin%
\pgfsetlinewidth{0.803000pt}%
\definecolor{currentstroke}{rgb}{0.000000,0.000000,0.000000}%
\pgfsetstrokecolor{currentstroke}%
\pgfsetdash{}{0pt}%
\pgfpathmoveto{\pgfqpoint{0.557402in}{3.961264in}}%
\pgfpathlineto{\pgfqpoint{6.131424in}{3.961264in}}%
\pgfusepath{stroke}%
\end{pgfscope}%
\begin{pgfscope}%
\definecolor{textcolor}{rgb}{0.000000,0.000000,0.000000}%
\pgfsetstrokecolor{textcolor}%
\pgfsetfillcolor{textcolor}%
\pgftext[x=0.000000in,y=4.100082in,left,base]{\color{textcolor}\rmfamily\fontsize{10.000000}{12.000000}\selectfont (a)}%
\end{pgfscope}%
\begin{pgfscope}%
\pgfpathrectangle{\pgfqpoint{0.557402in}{2.573088in}}{\pgfqpoint{5.574022in}{1.388176in}}%
\pgfusepath{clip}%
\pgfsetbuttcap%
\pgfsetmiterjoin%
\pgfsetlinewidth{1.003750pt}%
\definecolor{currentstroke}{rgb}{0.000000,0.000000,0.000000}%
\pgfsetstrokecolor{currentstroke}%
\pgfsetstrokeopacity{0.500000}%
\pgfsetdash{}{0pt}%
\pgfpathmoveto{\pgfqpoint{3.497426in}{2.636187in}}%
\pgfpathlineto{\pgfqpoint{4.699666in}{2.636187in}}%
\pgfpathlineto{\pgfqpoint{4.699666in}{2.644256in}}%
\pgfpathlineto{\pgfqpoint{3.497426in}{2.644256in}}%
\pgfpathlineto{\pgfqpoint{3.497426in}{2.636187in}}%
\pgfpathclose%
\pgfusepath{stroke}%
\end{pgfscope}%
\begin{pgfscope}%
\pgfsetroundcap%
\pgfsetroundjoin%
\pgfsetlinewidth{1.003750pt}%
\definecolor{currentstroke}{rgb}{0.000000,0.000000,0.000000}%
\pgfsetstrokecolor{currentstroke}%
\pgfsetstrokeopacity{0.500000}%
\pgfsetdash{}{0pt}%
\pgfpathmoveto{\pgfqpoint{3.511634in}{3.711392in}}%
\pgfpathquadraticcurveto{\pgfqpoint{3.504530in}{3.177824in}}{\pgfqpoint{3.497426in}{2.644256in}}%
\pgfusepath{stroke}%
\end{pgfscope}%
\begin{pgfscope}%
\pgfsetroundcap%
\pgfsetroundjoin%
\pgfsetlinewidth{1.003750pt}%
\definecolor{currentstroke}{rgb}{0.000000,0.000000,0.000000}%
\pgfsetstrokecolor{currentstroke}%
\pgfsetstrokeopacity{0.500000}%
\pgfsetdash{}{0pt}%
\pgfpathmoveto{\pgfqpoint{5.072360in}{2.920132in}}%
\pgfpathquadraticcurveto{\pgfqpoint{4.886013in}{2.778160in}}{\pgfqpoint{4.699666in}{2.636187in}}%
\pgfusepath{stroke}%
\end{pgfscope}%
\begin{pgfscope}%
\pgfsetbuttcap%
\pgfsetmiterjoin%
\definecolor{currentfill}{rgb}{1.000000,1.000000,1.000000}%
\pgfsetfillcolor{currentfill}%
\pgfsetlinewidth{0.000000pt}%
\definecolor{currentstroke}{rgb}{0.000000,0.000000,0.000000}%
\pgfsetstrokecolor{currentstroke}%
\pgfsetstrokeopacity{0.000000}%
\pgfsetdash{}{0pt}%
\pgfpathmoveto{\pgfqpoint{3.511634in}{2.920132in}}%
\pgfpathlineto{\pgfqpoint{5.072360in}{2.920132in}}%
\pgfpathlineto{\pgfqpoint{5.072360in}{3.711392in}}%
\pgfpathlineto{\pgfqpoint{3.511634in}{3.711392in}}%
\pgfpathlineto{\pgfqpoint{3.511634in}{2.920132in}}%
\pgfpathclose%
\pgfusepath{fill}%
\end{pgfscope}%
\begin{pgfscope}%
\pgfsetbuttcap%
\pgfsetroundjoin%
\definecolor{currentfill}{rgb}{0.000000,0.000000,0.000000}%
\pgfsetfillcolor{currentfill}%
\pgfsetlinewidth{0.803000pt}%
\definecolor{currentstroke}{rgb}{0.000000,0.000000,0.000000}%
\pgfsetstrokecolor{currentstroke}%
\pgfsetdash{}{0pt}%
\pgfsys@defobject{currentmarker}{\pgfqpoint{0.000000in}{0.000000in}}{\pgfqpoint{0.000000in}{0.048611in}}{%
\pgfpathmoveto{\pgfqpoint{0.000000in}{0.000000in}}%
\pgfpathlineto{\pgfqpoint{0.000000in}{0.048611in}}%
\pgfusepath{stroke,fill}%
}%
\begin{pgfscope}%
\pgfsys@transformshift{3.525822in}{3.711392in}%
\pgfsys@useobject{currentmarker}{}%
\end{pgfscope}%
\end{pgfscope}%
\begin{pgfscope}%
\definecolor{textcolor}{rgb}{0.000000,0.000000,0.000000}%
\pgfsetstrokecolor{textcolor}%
\pgfsetfillcolor{textcolor}%
\pgftext[x=3.525822in,y=3.808615in,,bottom]{\color{textcolor}\rmfamily\fontsize{10.000000}{12.000000}\selectfont \(\displaystyle {60}\)}%
\end{pgfscope}%
\begin{pgfscope}%
\pgfsetbuttcap%
\pgfsetroundjoin%
\definecolor{currentfill}{rgb}{0.000000,0.000000,0.000000}%
\pgfsetfillcolor{currentfill}%
\pgfsetlinewidth{0.803000pt}%
\definecolor{currentstroke}{rgb}{0.000000,0.000000,0.000000}%
\pgfsetstrokecolor{currentstroke}%
\pgfsetdash{}{0pt}%
\pgfsys@defobject{currentmarker}{\pgfqpoint{0.000000in}{0.000000in}}{\pgfqpoint{0.000000in}{0.048611in}}{%
\pgfpathmoveto{\pgfqpoint{0.000000in}{0.000000in}}%
\pgfpathlineto{\pgfqpoint{0.000000in}{0.048611in}}%
\pgfusepath{stroke,fill}%
}%
\begin{pgfscope}%
\pgfsys@transformshift{3.951475in}{3.711392in}%
\pgfsys@useobject{currentmarker}{}%
\end{pgfscope}%
\end{pgfscope}%
\begin{pgfscope}%
\definecolor{textcolor}{rgb}{0.000000,0.000000,0.000000}%
\pgfsetstrokecolor{textcolor}%
\pgfsetfillcolor{textcolor}%
\pgftext[x=3.951475in,y=3.808615in,,bottom]{\color{textcolor}\rmfamily\fontsize{10.000000}{12.000000}\selectfont \(\displaystyle {90}\)}%
\end{pgfscope}%
\begin{pgfscope}%
\pgfsetbuttcap%
\pgfsetroundjoin%
\definecolor{currentfill}{rgb}{0.000000,0.000000,0.000000}%
\pgfsetfillcolor{currentfill}%
\pgfsetlinewidth{0.803000pt}%
\definecolor{currentstroke}{rgb}{0.000000,0.000000,0.000000}%
\pgfsetstrokecolor{currentstroke}%
\pgfsetdash{}{0pt}%
\pgfsys@defobject{currentmarker}{\pgfqpoint{0.000000in}{0.000000in}}{\pgfqpoint{0.000000in}{0.048611in}}{%
\pgfpathmoveto{\pgfqpoint{0.000000in}{0.000000in}}%
\pgfpathlineto{\pgfqpoint{0.000000in}{0.048611in}}%
\pgfusepath{stroke,fill}%
}%
\begin{pgfscope}%
\pgfsys@transformshift{4.377128in}{3.711392in}%
\pgfsys@useobject{currentmarker}{}%
\end{pgfscope}%
\end{pgfscope}%
\begin{pgfscope}%
\definecolor{textcolor}{rgb}{0.000000,0.000000,0.000000}%
\pgfsetstrokecolor{textcolor}%
\pgfsetfillcolor{textcolor}%
\pgftext[x=4.377128in,y=3.808615in,,bottom]{\color{textcolor}\rmfamily\fontsize{10.000000}{12.000000}\selectfont \(\displaystyle {120}\)}%
\end{pgfscope}%
\begin{pgfscope}%
\pgfsetbuttcap%
\pgfsetroundjoin%
\definecolor{currentfill}{rgb}{0.000000,0.000000,0.000000}%
\pgfsetfillcolor{currentfill}%
\pgfsetlinewidth{0.803000pt}%
\definecolor{currentstroke}{rgb}{0.000000,0.000000,0.000000}%
\pgfsetstrokecolor{currentstroke}%
\pgfsetdash{}{0pt}%
\pgfsys@defobject{currentmarker}{\pgfqpoint{0.000000in}{0.000000in}}{\pgfqpoint{0.000000in}{0.048611in}}{%
\pgfpathmoveto{\pgfqpoint{0.000000in}{0.000000in}}%
\pgfpathlineto{\pgfqpoint{0.000000in}{0.048611in}}%
\pgfusepath{stroke,fill}%
}%
\begin{pgfscope}%
\pgfsys@transformshift{4.802780in}{3.711392in}%
\pgfsys@useobject{currentmarker}{}%
\end{pgfscope}%
\end{pgfscope}%
\begin{pgfscope}%
\definecolor{textcolor}{rgb}{0.000000,0.000000,0.000000}%
\pgfsetstrokecolor{textcolor}%
\pgfsetfillcolor{textcolor}%
\pgftext[x=4.802780in,y=3.808615in,,bottom]{\color{textcolor}\rmfamily\fontsize{10.000000}{12.000000}\selectfont \(\displaystyle {150}\)}%
\end{pgfscope}%
\begin{pgfscope}%
\pgfsetbuttcap%
\pgfsetroundjoin%
\definecolor{currentfill}{rgb}{0.000000,0.000000,0.000000}%
\pgfsetfillcolor{currentfill}%
\pgfsetlinewidth{0.602250pt}%
\definecolor{currentstroke}{rgb}{0.000000,0.000000,0.000000}%
\pgfsetstrokecolor{currentstroke}%
\pgfsetdash{}{0pt}%
\pgfsys@defobject{currentmarker}{\pgfqpoint{0.000000in}{0.000000in}}{\pgfqpoint{0.000000in}{0.027778in}}{%
\pgfpathmoveto{\pgfqpoint{0.000000in}{0.000000in}}%
\pgfpathlineto{\pgfqpoint{0.000000in}{0.027778in}}%
\pgfusepath{stroke,fill}%
}%
\begin{pgfscope}%
\pgfsys@transformshift{3.632236in}{3.711392in}%
\pgfsys@useobject{currentmarker}{}%
\end{pgfscope}%
\end{pgfscope}%
\begin{pgfscope}%
\pgfsetbuttcap%
\pgfsetroundjoin%
\definecolor{currentfill}{rgb}{0.000000,0.000000,0.000000}%
\pgfsetfillcolor{currentfill}%
\pgfsetlinewidth{0.602250pt}%
\definecolor{currentstroke}{rgb}{0.000000,0.000000,0.000000}%
\pgfsetstrokecolor{currentstroke}%
\pgfsetdash{}{0pt}%
\pgfsys@defobject{currentmarker}{\pgfqpoint{0.000000in}{0.000000in}}{\pgfqpoint{0.000000in}{0.027778in}}{%
\pgfpathmoveto{\pgfqpoint{0.000000in}{0.000000in}}%
\pgfpathlineto{\pgfqpoint{0.000000in}{0.027778in}}%
\pgfusepath{stroke,fill}%
}%
\begin{pgfscope}%
\pgfsys@transformshift{3.738649in}{3.711392in}%
\pgfsys@useobject{currentmarker}{}%
\end{pgfscope}%
\end{pgfscope}%
\begin{pgfscope}%
\pgfsetbuttcap%
\pgfsetroundjoin%
\definecolor{currentfill}{rgb}{0.000000,0.000000,0.000000}%
\pgfsetfillcolor{currentfill}%
\pgfsetlinewidth{0.602250pt}%
\definecolor{currentstroke}{rgb}{0.000000,0.000000,0.000000}%
\pgfsetstrokecolor{currentstroke}%
\pgfsetdash{}{0pt}%
\pgfsys@defobject{currentmarker}{\pgfqpoint{0.000000in}{0.000000in}}{\pgfqpoint{0.000000in}{0.027778in}}{%
\pgfpathmoveto{\pgfqpoint{0.000000in}{0.000000in}}%
\pgfpathlineto{\pgfqpoint{0.000000in}{0.027778in}}%
\pgfusepath{stroke,fill}%
}%
\begin{pgfscope}%
\pgfsys@transformshift{3.845062in}{3.711392in}%
\pgfsys@useobject{currentmarker}{}%
\end{pgfscope}%
\end{pgfscope}%
\begin{pgfscope}%
\pgfsetbuttcap%
\pgfsetroundjoin%
\definecolor{currentfill}{rgb}{0.000000,0.000000,0.000000}%
\pgfsetfillcolor{currentfill}%
\pgfsetlinewidth{0.602250pt}%
\definecolor{currentstroke}{rgb}{0.000000,0.000000,0.000000}%
\pgfsetstrokecolor{currentstroke}%
\pgfsetdash{}{0pt}%
\pgfsys@defobject{currentmarker}{\pgfqpoint{0.000000in}{0.000000in}}{\pgfqpoint{0.000000in}{0.027778in}}{%
\pgfpathmoveto{\pgfqpoint{0.000000in}{0.000000in}}%
\pgfpathlineto{\pgfqpoint{0.000000in}{0.027778in}}%
\pgfusepath{stroke,fill}%
}%
\begin{pgfscope}%
\pgfsys@transformshift{4.057888in}{3.711392in}%
\pgfsys@useobject{currentmarker}{}%
\end{pgfscope}%
\end{pgfscope}%
\begin{pgfscope}%
\pgfsetbuttcap%
\pgfsetroundjoin%
\definecolor{currentfill}{rgb}{0.000000,0.000000,0.000000}%
\pgfsetfillcolor{currentfill}%
\pgfsetlinewidth{0.602250pt}%
\definecolor{currentstroke}{rgb}{0.000000,0.000000,0.000000}%
\pgfsetstrokecolor{currentstroke}%
\pgfsetdash{}{0pt}%
\pgfsys@defobject{currentmarker}{\pgfqpoint{0.000000in}{0.000000in}}{\pgfqpoint{0.000000in}{0.027778in}}{%
\pgfpathmoveto{\pgfqpoint{0.000000in}{0.000000in}}%
\pgfpathlineto{\pgfqpoint{0.000000in}{0.027778in}}%
\pgfusepath{stroke,fill}%
}%
\begin{pgfscope}%
\pgfsys@transformshift{4.164301in}{3.711392in}%
\pgfsys@useobject{currentmarker}{}%
\end{pgfscope}%
\end{pgfscope}%
\begin{pgfscope}%
\pgfsetbuttcap%
\pgfsetroundjoin%
\definecolor{currentfill}{rgb}{0.000000,0.000000,0.000000}%
\pgfsetfillcolor{currentfill}%
\pgfsetlinewidth{0.602250pt}%
\definecolor{currentstroke}{rgb}{0.000000,0.000000,0.000000}%
\pgfsetstrokecolor{currentstroke}%
\pgfsetdash{}{0pt}%
\pgfsys@defobject{currentmarker}{\pgfqpoint{0.000000in}{0.000000in}}{\pgfqpoint{0.000000in}{0.027778in}}{%
\pgfpathmoveto{\pgfqpoint{0.000000in}{0.000000in}}%
\pgfpathlineto{\pgfqpoint{0.000000in}{0.027778in}}%
\pgfusepath{stroke,fill}%
}%
\begin{pgfscope}%
\pgfsys@transformshift{4.270714in}{3.711392in}%
\pgfsys@useobject{currentmarker}{}%
\end{pgfscope}%
\end{pgfscope}%
\begin{pgfscope}%
\pgfsetbuttcap%
\pgfsetroundjoin%
\definecolor{currentfill}{rgb}{0.000000,0.000000,0.000000}%
\pgfsetfillcolor{currentfill}%
\pgfsetlinewidth{0.602250pt}%
\definecolor{currentstroke}{rgb}{0.000000,0.000000,0.000000}%
\pgfsetstrokecolor{currentstroke}%
\pgfsetdash{}{0pt}%
\pgfsys@defobject{currentmarker}{\pgfqpoint{0.000000in}{0.000000in}}{\pgfqpoint{0.000000in}{0.027778in}}{%
\pgfpathmoveto{\pgfqpoint{0.000000in}{0.000000in}}%
\pgfpathlineto{\pgfqpoint{0.000000in}{0.027778in}}%
\pgfusepath{stroke,fill}%
}%
\begin{pgfscope}%
\pgfsys@transformshift{4.483541in}{3.711392in}%
\pgfsys@useobject{currentmarker}{}%
\end{pgfscope}%
\end{pgfscope}%
\begin{pgfscope}%
\pgfsetbuttcap%
\pgfsetroundjoin%
\definecolor{currentfill}{rgb}{0.000000,0.000000,0.000000}%
\pgfsetfillcolor{currentfill}%
\pgfsetlinewidth{0.602250pt}%
\definecolor{currentstroke}{rgb}{0.000000,0.000000,0.000000}%
\pgfsetstrokecolor{currentstroke}%
\pgfsetdash{}{0pt}%
\pgfsys@defobject{currentmarker}{\pgfqpoint{0.000000in}{0.000000in}}{\pgfqpoint{0.000000in}{0.027778in}}{%
\pgfpathmoveto{\pgfqpoint{0.000000in}{0.000000in}}%
\pgfpathlineto{\pgfqpoint{0.000000in}{0.027778in}}%
\pgfusepath{stroke,fill}%
}%
\begin{pgfscope}%
\pgfsys@transformshift{4.589954in}{3.711392in}%
\pgfsys@useobject{currentmarker}{}%
\end{pgfscope}%
\end{pgfscope}%
\begin{pgfscope}%
\pgfsetbuttcap%
\pgfsetroundjoin%
\definecolor{currentfill}{rgb}{0.000000,0.000000,0.000000}%
\pgfsetfillcolor{currentfill}%
\pgfsetlinewidth{0.602250pt}%
\definecolor{currentstroke}{rgb}{0.000000,0.000000,0.000000}%
\pgfsetstrokecolor{currentstroke}%
\pgfsetdash{}{0pt}%
\pgfsys@defobject{currentmarker}{\pgfqpoint{0.000000in}{0.000000in}}{\pgfqpoint{0.000000in}{0.027778in}}{%
\pgfpathmoveto{\pgfqpoint{0.000000in}{0.000000in}}%
\pgfpathlineto{\pgfqpoint{0.000000in}{0.027778in}}%
\pgfusepath{stroke,fill}%
}%
\begin{pgfscope}%
\pgfsys@transformshift{4.696367in}{3.711392in}%
\pgfsys@useobject{currentmarker}{}%
\end{pgfscope}%
\end{pgfscope}%
\begin{pgfscope}%
\pgfsetbuttcap%
\pgfsetroundjoin%
\definecolor{currentfill}{rgb}{0.000000,0.000000,0.000000}%
\pgfsetfillcolor{currentfill}%
\pgfsetlinewidth{0.602250pt}%
\definecolor{currentstroke}{rgb}{0.000000,0.000000,0.000000}%
\pgfsetstrokecolor{currentstroke}%
\pgfsetdash{}{0pt}%
\pgfsys@defobject{currentmarker}{\pgfqpoint{0.000000in}{0.000000in}}{\pgfqpoint{0.000000in}{0.027778in}}{%
\pgfpathmoveto{\pgfqpoint{0.000000in}{0.000000in}}%
\pgfpathlineto{\pgfqpoint{0.000000in}{0.027778in}}%
\pgfusepath{stroke,fill}%
}%
\begin{pgfscope}%
\pgfsys@transformshift{4.909193in}{3.711392in}%
\pgfsys@useobject{currentmarker}{}%
\end{pgfscope}%
\end{pgfscope}%
\begin{pgfscope}%
\pgfsetbuttcap%
\pgfsetroundjoin%
\definecolor{currentfill}{rgb}{0.000000,0.000000,0.000000}%
\pgfsetfillcolor{currentfill}%
\pgfsetlinewidth{0.602250pt}%
\definecolor{currentstroke}{rgb}{0.000000,0.000000,0.000000}%
\pgfsetstrokecolor{currentstroke}%
\pgfsetdash{}{0pt}%
\pgfsys@defobject{currentmarker}{\pgfqpoint{0.000000in}{0.000000in}}{\pgfqpoint{0.000000in}{0.027778in}}{%
\pgfpathmoveto{\pgfqpoint{0.000000in}{0.000000in}}%
\pgfpathlineto{\pgfqpoint{0.000000in}{0.027778in}}%
\pgfusepath{stroke,fill}%
}%
\begin{pgfscope}%
\pgfsys@transformshift{5.015607in}{3.711392in}%
\pgfsys@useobject{currentmarker}{}%
\end{pgfscope}%
\end{pgfscope}%
\begin{pgfscope}%
\pgfsetbuttcap%
\pgfsetroundjoin%
\definecolor{currentfill}{rgb}{0.000000,0.000000,0.000000}%
\pgfsetfillcolor{currentfill}%
\pgfsetlinewidth{0.803000pt}%
\definecolor{currentstroke}{rgb}{0.000000,0.000000,0.000000}%
\pgfsetstrokecolor{currentstroke}%
\pgfsetdash{}{0pt}%
\pgfsys@defobject{currentmarker}{\pgfqpoint{0.000000in}{0.000000in}}{\pgfqpoint{0.048611in}{0.000000in}}{%
\pgfpathmoveto{\pgfqpoint{0.000000in}{0.000000in}}%
\pgfpathlineto{\pgfqpoint{0.048611in}{0.000000in}}%
\pgfusepath{stroke,fill}%
}%
\begin{pgfscope}%
\pgfsys@transformshift{5.072360in}{2.955982in}%
\pgfsys@useobject{currentmarker}{}%
\end{pgfscope}%
\end{pgfscope}%
\begin{pgfscope}%
\definecolor{textcolor}{rgb}{0.000000,0.000000,0.000000}%
\pgfsetstrokecolor{textcolor}%
\pgfsetfillcolor{textcolor}%
\pgftext[x=5.169582in, y=2.908154in, left, base]{\color{textcolor}\rmfamily\fontsize{10.000000}{12.000000}\selectfont \(\displaystyle {0}\)}%
\end{pgfscope}%
\begin{pgfscope}%
\pgfsetbuttcap%
\pgfsetroundjoin%
\definecolor{currentfill}{rgb}{0.000000,0.000000,0.000000}%
\pgfsetfillcolor{currentfill}%
\pgfsetlinewidth{0.803000pt}%
\definecolor{currentstroke}{rgb}{0.000000,0.000000,0.000000}%
\pgfsetstrokecolor{currentstroke}%
\pgfsetdash{}{0pt}%
\pgfsys@defobject{currentmarker}{\pgfqpoint{0.000000in}{0.000000in}}{\pgfqpoint{0.048611in}{0.000000in}}{%
\pgfpathmoveto{\pgfqpoint{0.000000in}{0.000000in}}%
\pgfpathlineto{\pgfqpoint{0.048611in}{0.000000in}}%
\pgfusepath{stroke,fill}%
}%
\begin{pgfscope}%
\pgfsys@transformshift{5.072360in}{3.343286in}%
\pgfsys@useobject{currentmarker}{}%
\end{pgfscope}%
\end{pgfscope}%
\begin{pgfscope}%
\definecolor{textcolor}{rgb}{0.000000,0.000000,0.000000}%
\pgfsetstrokecolor{textcolor}%
\pgfsetfillcolor{textcolor}%
\pgftext[x=5.169582in, y=3.295458in, left, base]{\color{textcolor}\rmfamily\fontsize{10.000000}{12.000000}\selectfont \(\displaystyle {100000}\)}%
\end{pgfscope}%
\begin{pgfscope}%
\pgfsetbuttcap%
\pgfsetroundjoin%
\definecolor{currentfill}{rgb}{0.000000,0.000000,0.000000}%
\pgfsetfillcolor{currentfill}%
\pgfsetlinewidth{0.602250pt}%
\definecolor{currentstroke}{rgb}{0.000000,0.000000,0.000000}%
\pgfsetstrokecolor{currentstroke}%
\pgfsetdash{}{0pt}%
\pgfsys@defobject{currentmarker}{\pgfqpoint{0.000000in}{0.000000in}}{\pgfqpoint{0.027778in}{0.000000in}}{%
\pgfpathmoveto{\pgfqpoint{0.000000in}{0.000000in}}%
\pgfpathlineto{\pgfqpoint{0.027778in}{0.000000in}}%
\pgfusepath{stroke,fill}%
}%
\begin{pgfscope}%
\pgfsys@transformshift{5.072360in}{3.033443in}%
\pgfsys@useobject{currentmarker}{}%
\end{pgfscope}%
\end{pgfscope}%
\begin{pgfscope}%
\pgfsetbuttcap%
\pgfsetroundjoin%
\definecolor{currentfill}{rgb}{0.000000,0.000000,0.000000}%
\pgfsetfillcolor{currentfill}%
\pgfsetlinewidth{0.602250pt}%
\definecolor{currentstroke}{rgb}{0.000000,0.000000,0.000000}%
\pgfsetstrokecolor{currentstroke}%
\pgfsetdash{}{0pt}%
\pgfsys@defobject{currentmarker}{\pgfqpoint{0.000000in}{0.000000in}}{\pgfqpoint{0.027778in}{0.000000in}}{%
\pgfpathmoveto{\pgfqpoint{0.000000in}{0.000000in}}%
\pgfpathlineto{\pgfqpoint{0.027778in}{0.000000in}}%
\pgfusepath{stroke,fill}%
}%
\begin{pgfscope}%
\pgfsys@transformshift{5.072360in}{3.110904in}%
\pgfsys@useobject{currentmarker}{}%
\end{pgfscope}%
\end{pgfscope}%
\begin{pgfscope}%
\pgfsetbuttcap%
\pgfsetroundjoin%
\definecolor{currentfill}{rgb}{0.000000,0.000000,0.000000}%
\pgfsetfillcolor{currentfill}%
\pgfsetlinewidth{0.602250pt}%
\definecolor{currentstroke}{rgb}{0.000000,0.000000,0.000000}%
\pgfsetstrokecolor{currentstroke}%
\pgfsetdash{}{0pt}%
\pgfsys@defobject{currentmarker}{\pgfqpoint{0.000000in}{0.000000in}}{\pgfqpoint{0.027778in}{0.000000in}}{%
\pgfpathmoveto{\pgfqpoint{0.000000in}{0.000000in}}%
\pgfpathlineto{\pgfqpoint{0.027778in}{0.000000in}}%
\pgfusepath{stroke,fill}%
}%
\begin{pgfscope}%
\pgfsys@transformshift{5.072360in}{3.188365in}%
\pgfsys@useobject{currentmarker}{}%
\end{pgfscope}%
\end{pgfscope}%
\begin{pgfscope}%
\pgfsetbuttcap%
\pgfsetroundjoin%
\definecolor{currentfill}{rgb}{0.000000,0.000000,0.000000}%
\pgfsetfillcolor{currentfill}%
\pgfsetlinewidth{0.602250pt}%
\definecolor{currentstroke}{rgb}{0.000000,0.000000,0.000000}%
\pgfsetstrokecolor{currentstroke}%
\pgfsetdash{}{0pt}%
\pgfsys@defobject{currentmarker}{\pgfqpoint{0.000000in}{0.000000in}}{\pgfqpoint{0.027778in}{0.000000in}}{%
\pgfpathmoveto{\pgfqpoint{0.000000in}{0.000000in}}%
\pgfpathlineto{\pgfqpoint{0.027778in}{0.000000in}}%
\pgfusepath{stroke,fill}%
}%
\begin{pgfscope}%
\pgfsys@transformshift{5.072360in}{3.265825in}%
\pgfsys@useobject{currentmarker}{}%
\end{pgfscope}%
\end{pgfscope}%
\begin{pgfscope}%
\pgfsetbuttcap%
\pgfsetroundjoin%
\definecolor{currentfill}{rgb}{0.000000,0.000000,0.000000}%
\pgfsetfillcolor{currentfill}%
\pgfsetlinewidth{0.602250pt}%
\definecolor{currentstroke}{rgb}{0.000000,0.000000,0.000000}%
\pgfsetstrokecolor{currentstroke}%
\pgfsetdash{}{0pt}%
\pgfsys@defobject{currentmarker}{\pgfqpoint{0.000000in}{0.000000in}}{\pgfqpoint{0.027778in}{0.000000in}}{%
\pgfpathmoveto{\pgfqpoint{0.000000in}{0.000000in}}%
\pgfpathlineto{\pgfqpoint{0.027778in}{0.000000in}}%
\pgfusepath{stroke,fill}%
}%
\begin{pgfscope}%
\pgfsys@transformshift{5.072360in}{3.420747in}%
\pgfsys@useobject{currentmarker}{}%
\end{pgfscope}%
\end{pgfscope}%
\begin{pgfscope}%
\pgfsetbuttcap%
\pgfsetroundjoin%
\definecolor{currentfill}{rgb}{0.000000,0.000000,0.000000}%
\pgfsetfillcolor{currentfill}%
\pgfsetlinewidth{0.602250pt}%
\definecolor{currentstroke}{rgb}{0.000000,0.000000,0.000000}%
\pgfsetstrokecolor{currentstroke}%
\pgfsetdash{}{0pt}%
\pgfsys@defobject{currentmarker}{\pgfqpoint{0.000000in}{0.000000in}}{\pgfqpoint{0.027778in}{0.000000in}}{%
\pgfpathmoveto{\pgfqpoint{0.000000in}{0.000000in}}%
\pgfpathlineto{\pgfqpoint{0.027778in}{0.000000in}}%
\pgfusepath{stroke,fill}%
}%
\begin{pgfscope}%
\pgfsys@transformshift{5.072360in}{3.498207in}%
\pgfsys@useobject{currentmarker}{}%
\end{pgfscope}%
\end{pgfscope}%
\begin{pgfscope}%
\pgfsetbuttcap%
\pgfsetroundjoin%
\definecolor{currentfill}{rgb}{0.000000,0.000000,0.000000}%
\pgfsetfillcolor{currentfill}%
\pgfsetlinewidth{0.602250pt}%
\definecolor{currentstroke}{rgb}{0.000000,0.000000,0.000000}%
\pgfsetstrokecolor{currentstroke}%
\pgfsetdash{}{0pt}%
\pgfsys@defobject{currentmarker}{\pgfqpoint{0.000000in}{0.000000in}}{\pgfqpoint{0.027778in}{0.000000in}}{%
\pgfpathmoveto{\pgfqpoint{0.000000in}{0.000000in}}%
\pgfpathlineto{\pgfqpoint{0.027778in}{0.000000in}}%
\pgfusepath{stroke,fill}%
}%
\begin{pgfscope}%
\pgfsys@transformshift{5.072360in}{3.575668in}%
\pgfsys@useobject{currentmarker}{}%
\end{pgfscope}%
\end{pgfscope}%
\begin{pgfscope}%
\pgfsetbuttcap%
\pgfsetroundjoin%
\definecolor{currentfill}{rgb}{0.000000,0.000000,0.000000}%
\pgfsetfillcolor{currentfill}%
\pgfsetlinewidth{0.602250pt}%
\definecolor{currentstroke}{rgb}{0.000000,0.000000,0.000000}%
\pgfsetstrokecolor{currentstroke}%
\pgfsetdash{}{0pt}%
\pgfsys@defobject{currentmarker}{\pgfqpoint{0.000000in}{0.000000in}}{\pgfqpoint{0.027778in}{0.000000in}}{%
\pgfpathmoveto{\pgfqpoint{0.000000in}{0.000000in}}%
\pgfpathlineto{\pgfqpoint{0.027778in}{0.000000in}}%
\pgfusepath{stroke,fill}%
}%
\begin{pgfscope}%
\pgfsys@transformshift{5.072360in}{3.653129in}%
\pgfsys@useobject{currentmarker}{}%
\end{pgfscope}%
\end{pgfscope}%
\begin{pgfscope}%
\pgfpathrectangle{\pgfqpoint{3.511634in}{2.920132in}}{\pgfqpoint{1.560726in}{0.791260in}}%
\pgfusepath{clip}%
\pgfsetrectcap%
\pgfsetroundjoin%
\pgfsetlinewidth{1.505625pt}%
\definecolor{currentstroke}{rgb}{0.121569,0.466667,0.705882}%
\pgfsetstrokecolor{currentstroke}%
\pgfsetdash{}{0pt}%
\pgfpathmoveto{\pgfqpoint{3.653518in}{3.675426in}}%
\pgfpathlineto{\pgfqpoint{3.667707in}{3.602714in}}%
\pgfpathlineto{\pgfqpoint{3.681895in}{3.535183in}}%
\pgfpathlineto{\pgfqpoint{3.696083in}{3.471379in}}%
\pgfpathlineto{\pgfqpoint{3.710272in}{3.418113in}}%
\pgfpathlineto{\pgfqpoint{3.724460in}{3.369673in}}%
\pgfpathlineto{\pgfqpoint{3.738649in}{3.328588in}}%
\pgfpathlineto{\pgfqpoint{3.752837in}{3.289985in}}%
\pgfpathlineto{\pgfqpoint{3.767026in}{3.254102in}}%
\pgfpathlineto{\pgfqpoint{3.781214in}{3.221436in}}%
\pgfpathlineto{\pgfqpoint{3.795402in}{3.194217in}}%
\pgfpathlineto{\pgfqpoint{3.809591in}{3.169615in}}%
\pgfpathlineto{\pgfqpoint{3.823779in}{3.147248in}}%
\pgfpathlineto{\pgfqpoint{3.837968in}{3.125830in}}%
\pgfpathlineto{\pgfqpoint{3.852156in}{3.110671in}}%
\pgfpathlineto{\pgfqpoint{3.866344in}{3.093979in}}%
\pgfpathlineto{\pgfqpoint{3.880533in}{3.078781in}}%
\pgfpathlineto{\pgfqpoint{3.894721in}{3.066155in}}%
\pgfpathlineto{\pgfqpoint{3.908910in}{3.055353in}}%
\pgfpathlineto{\pgfqpoint{3.923098in}{3.044284in}}%
\pgfpathlineto{\pgfqpoint{3.937287in}{3.034888in}}%
\pgfpathlineto{\pgfqpoint{3.951475in}{3.027184in}}%
\pgfpathlineto{\pgfqpoint{3.965663in}{3.019097in}}%
\pgfpathlineto{\pgfqpoint{3.979852in}{3.012928in}}%
\pgfpathlineto{\pgfqpoint{3.994040in}{3.005801in}}%
\pgfpathlineto{\pgfqpoint{4.008229in}{3.001130in}}%
\pgfpathlineto{\pgfqpoint{4.022417in}{2.995723in}}%
\pgfpathlineto{\pgfqpoint{4.036606in}{2.991014in}}%
\pgfpathlineto{\pgfqpoint{4.050794in}{2.988365in}}%
\pgfpathlineto{\pgfqpoint{4.064982in}{2.984116in}}%
\pgfpathlineto{\pgfqpoint{4.079171in}{2.981219in}}%
\pgfpathlineto{\pgfqpoint{4.093359in}{2.977931in}}%
\pgfpathlineto{\pgfqpoint{4.107548in}{2.976134in}}%
\pgfpathlineto{\pgfqpoint{4.121736in}{2.973353in}}%
\pgfpathlineto{\pgfqpoint{4.135924in}{2.971935in}}%
\pgfpathlineto{\pgfqpoint{4.150113in}{2.970138in}}%
\pgfpathlineto{\pgfqpoint{4.164301in}{2.968825in}}%
\pgfpathlineto{\pgfqpoint{4.178490in}{2.967226in}}%
\pgfpathlineto{\pgfqpoint{4.192678in}{2.966195in}}%
\pgfpathlineto{\pgfqpoint{4.206867in}{2.964778in}}%
\pgfpathlineto{\pgfqpoint{4.221055in}{2.963856in}}%
\pgfpathlineto{\pgfqpoint{4.235243in}{2.963198in}}%
\pgfpathlineto{\pgfqpoint{4.249432in}{2.962036in}}%
\pgfpathlineto{\pgfqpoint{4.263620in}{2.961424in}}%
\pgfpathlineto{\pgfqpoint{4.277809in}{2.960692in}}%
\pgfpathlineto{\pgfqpoint{4.291997in}{2.960370in}}%
\pgfpathlineto{\pgfqpoint{4.306186in}{2.960014in}}%
\pgfpathlineto{\pgfqpoint{4.320374in}{2.959425in}}%
\pgfpathlineto{\pgfqpoint{4.334562in}{2.959030in}}%
\pgfpathlineto{\pgfqpoint{4.348751in}{2.958848in}}%
\pgfpathlineto{\pgfqpoint{4.362939in}{2.958395in}}%
\pgfpathlineto{\pgfqpoint{4.377128in}{2.958023in}}%
\pgfpathlineto{\pgfqpoint{4.391316in}{2.957884in}}%
\pgfpathlineto{\pgfqpoint{4.405504in}{2.957756in}}%
\pgfpathlineto{\pgfqpoint{4.419693in}{2.957640in}}%
\pgfpathlineto{\pgfqpoint{4.433881in}{2.957419in}}%
\pgfpathlineto{\pgfqpoint{4.448070in}{2.957067in}}%
\pgfpathlineto{\pgfqpoint{4.462258in}{2.957105in}}%
\pgfpathlineto{\pgfqpoint{4.476447in}{2.957152in}}%
\pgfpathlineto{\pgfqpoint{4.490635in}{2.956900in}}%
\pgfpathlineto{\pgfqpoint{4.504823in}{2.956846in}}%
\pgfpathlineto{\pgfqpoint{4.519012in}{2.956683in}}%
\pgfpathlineto{\pgfqpoint{4.533200in}{2.956776in}}%
\pgfpathlineto{\pgfqpoint{4.547389in}{2.956714in}}%
\pgfpathlineto{\pgfqpoint{4.561577in}{2.956679in}}%
\pgfpathlineto{\pgfqpoint{4.575765in}{2.956544in}}%
\pgfpathlineto{\pgfqpoint{4.589954in}{2.956509in}}%
\pgfpathlineto{\pgfqpoint{4.604142in}{2.956513in}}%
\pgfpathlineto{\pgfqpoint{4.618331in}{2.956459in}}%
\pgfpathlineto{\pgfqpoint{4.632519in}{2.956366in}}%
\pgfpathlineto{\pgfqpoint{4.646708in}{2.956397in}}%
\pgfpathlineto{\pgfqpoint{4.660896in}{2.956377in}}%
\pgfpathlineto{\pgfqpoint{4.675084in}{2.956362in}}%
\pgfpathlineto{\pgfqpoint{4.689273in}{2.956381in}}%
\pgfpathlineto{\pgfqpoint{4.703461in}{2.956362in}}%
\pgfpathlineto{\pgfqpoint{4.717650in}{2.956288in}}%
\pgfpathlineto{\pgfqpoint{4.731838in}{2.956234in}}%
\pgfpathlineto{\pgfqpoint{4.746027in}{2.956199in}}%
\pgfpathlineto{\pgfqpoint{4.760215in}{2.956300in}}%
\pgfpathlineto{\pgfqpoint{4.774403in}{2.956195in}}%
\pgfpathlineto{\pgfqpoint{4.788592in}{2.956215in}}%
\pgfpathlineto{\pgfqpoint{4.802780in}{2.956219in}}%
\pgfpathlineto{\pgfqpoint{4.816969in}{2.956191in}}%
\pgfpathlineto{\pgfqpoint{4.831157in}{2.956207in}}%
\pgfpathlineto{\pgfqpoint{4.845345in}{2.956195in}}%
\pgfpathlineto{\pgfqpoint{4.859534in}{2.956176in}}%
\pgfpathlineto{\pgfqpoint{4.873722in}{2.956168in}}%
\pgfpathlineto{\pgfqpoint{4.887911in}{2.956211in}}%
\pgfpathlineto{\pgfqpoint{4.902099in}{2.956160in}}%
\pgfpathlineto{\pgfqpoint{4.916288in}{2.956164in}}%
\pgfpathlineto{\pgfqpoint{4.930476in}{2.956199in}}%
\pgfpathlineto{\pgfqpoint{4.944664in}{2.956129in}}%
\pgfpathlineto{\pgfqpoint{4.958853in}{2.956172in}}%
\pgfpathlineto{\pgfqpoint{4.973041in}{2.956122in}}%
\pgfpathlineto{\pgfqpoint{4.987230in}{2.956114in}}%
\pgfpathlineto{\pgfqpoint{5.001418in}{2.956110in}}%
\pgfpathlineto{\pgfqpoint{5.015607in}{2.956106in}}%
\pgfpathlineto{\pgfqpoint{5.029795in}{2.956184in}}%
\pgfpathlineto{\pgfqpoint{5.043983in}{2.956141in}}%
\pgfpathlineto{\pgfqpoint{5.058172in}{2.956098in}}%
\pgfusepath{stroke}%
\end{pgfscope}%
\begin{pgfscope}%
\pgfpathrectangle{\pgfqpoint{3.511634in}{2.920132in}}{\pgfqpoint{1.560726in}{0.791260in}}%
\pgfusepath{clip}%
\pgfsetrectcap%
\pgfsetroundjoin%
\pgfsetlinewidth{1.505625pt}%
\definecolor{currentstroke}{rgb}{1.000000,0.498039,0.054902}%
\pgfsetstrokecolor{currentstroke}%
\pgfsetdash{}{0pt}%
\pgfpathmoveto{\pgfqpoint{3.653518in}{3.640096in}}%
\pgfpathlineto{\pgfqpoint{3.667707in}{3.583724in}}%
\pgfpathlineto{\pgfqpoint{3.681895in}{3.525501in}}%
\pgfpathlineto{\pgfqpoint{3.696083in}{3.479233in}}%
\pgfpathlineto{\pgfqpoint{3.710272in}{3.432982in}}%
\pgfpathlineto{\pgfqpoint{3.724460in}{3.395196in}}%
\pgfpathlineto{\pgfqpoint{3.738649in}{3.360598in}}%
\pgfpathlineto{\pgfqpoint{3.752837in}{3.324664in}}%
\pgfpathlineto{\pgfqpoint{3.767026in}{3.297758in}}%
\pgfpathlineto{\pgfqpoint{3.781214in}{3.271096in}}%
\pgfpathlineto{\pgfqpoint{3.795402in}{3.248796in}}%
\pgfpathlineto{\pgfqpoint{3.809591in}{3.225003in}}%
\pgfpathlineto{\pgfqpoint{3.823779in}{3.204097in}}%
\pgfpathlineto{\pgfqpoint{3.837968in}{3.187017in}}%
\pgfpathlineto{\pgfqpoint{3.852156in}{3.171033in}}%
\pgfpathlineto{\pgfqpoint{3.866344in}{3.157361in}}%
\pgfpathlineto{\pgfqpoint{3.880533in}{3.143809in}}%
\pgfpathlineto{\pgfqpoint{3.894721in}{3.133166in}}%
\pgfpathlineto{\pgfqpoint{3.908910in}{3.123863in}}%
\pgfpathlineto{\pgfqpoint{3.923098in}{3.110222in}}%
\pgfpathlineto{\pgfqpoint{3.937287in}{3.102193in}}%
\pgfpathlineto{\pgfqpoint{3.951475in}{3.093587in}}%
\pgfpathlineto{\pgfqpoint{3.965663in}{3.086872in}}%
\pgfpathlineto{\pgfqpoint{3.979852in}{3.080717in}}%
\pgfpathlineto{\pgfqpoint{3.994040in}{3.072890in}}%
\pgfpathlineto{\pgfqpoint{4.008229in}{3.066941in}}%
\pgfpathlineto{\pgfqpoint{4.022417in}{3.062832in}}%
\pgfpathlineto{\pgfqpoint{4.036606in}{3.058494in}}%
\pgfpathlineto{\pgfqpoint{4.050794in}{3.056271in}}%
\pgfpathlineto{\pgfqpoint{4.064982in}{3.051073in}}%
\pgfpathlineto{\pgfqpoint{4.079171in}{3.046309in}}%
\pgfpathlineto{\pgfqpoint{4.093359in}{3.043122in}}%
\pgfpathlineto{\pgfqpoint{4.107548in}{3.039609in}}%
\pgfpathlineto{\pgfqpoint{4.121736in}{3.038745in}}%
\pgfpathlineto{\pgfqpoint{4.135924in}{3.034934in}}%
\pgfpathlineto{\pgfqpoint{4.150113in}{3.032537in}}%
\pgfpathlineto{\pgfqpoint{4.164301in}{3.031173in}}%
\pgfpathlineto{\pgfqpoint{4.178490in}{3.028354in}}%
\pgfpathlineto{\pgfqpoint{4.192678in}{3.027308in}}%
\pgfpathlineto{\pgfqpoint{4.206867in}{3.025527in}}%
\pgfpathlineto{\pgfqpoint{4.221055in}{3.023997in}}%
\pgfpathlineto{\pgfqpoint{4.235243in}{3.021584in}}%
\pgfpathlineto{\pgfqpoint{4.249432in}{3.020186in}}%
\pgfpathlineto{\pgfqpoint{4.263620in}{3.018509in}}%
\pgfpathlineto{\pgfqpoint{4.277809in}{3.017413in}}%
\pgfpathlineto{\pgfqpoint{4.291997in}{3.016398in}}%
\pgfpathlineto{\pgfqpoint{4.306186in}{3.014082in}}%
\pgfpathlineto{\pgfqpoint{4.320374in}{3.013811in}}%
\pgfpathlineto{\pgfqpoint{4.334562in}{3.012126in}}%
\pgfpathlineto{\pgfqpoint{4.348751in}{3.011049in}}%
\pgfpathlineto{\pgfqpoint{4.362939in}{3.009728in}}%
\pgfpathlineto{\pgfqpoint{4.377128in}{3.009794in}}%
\pgfpathlineto{\pgfqpoint{4.391316in}{3.007315in}}%
\pgfpathlineto{\pgfqpoint{4.405504in}{3.006804in}}%
\pgfpathlineto{\pgfqpoint{4.419693in}{3.005356in}}%
\pgfpathlineto{\pgfqpoint{4.433881in}{3.004465in}}%
\pgfpathlineto{\pgfqpoint{4.448070in}{3.003396in}}%
\pgfpathlineto{\pgfqpoint{4.462258in}{3.002033in}}%
\pgfpathlineto{\pgfqpoint{4.476447in}{3.000940in}}%
\pgfpathlineto{\pgfqpoint{4.490635in}{3.000495in}}%
\pgfpathlineto{\pgfqpoint{4.504823in}{2.999399in}}%
\pgfpathlineto{\pgfqpoint{4.519012in}{2.997850in}}%
\pgfpathlineto{\pgfqpoint{4.533200in}{2.996556in}}%
\pgfpathlineto{\pgfqpoint{4.547389in}{2.995967in}}%
\pgfpathlineto{\pgfqpoint{4.561577in}{2.995615in}}%
\pgfpathlineto{\pgfqpoint{4.575765in}{2.993233in}}%
\pgfpathlineto{\pgfqpoint{4.589954in}{2.992935in}}%
\pgfpathlineto{\pgfqpoint{4.604142in}{2.992350in}}%
\pgfpathlineto{\pgfqpoint{4.618331in}{2.990696in}}%
\pgfpathlineto{\pgfqpoint{4.632519in}{2.989937in}}%
\pgfpathlineto{\pgfqpoint{4.646708in}{2.989713in}}%
\pgfpathlineto{\pgfqpoint{4.660896in}{2.988175in}}%
\pgfpathlineto{\pgfqpoint{4.675084in}{2.986572in}}%
\pgfpathlineto{\pgfqpoint{4.689273in}{2.985615in}}%
\pgfpathlineto{\pgfqpoint{4.703461in}{2.984999in}}%
\pgfpathlineto{\pgfqpoint{4.717650in}{2.983717in}}%
\pgfpathlineto{\pgfqpoint{4.731838in}{2.982648in}}%
\pgfpathlineto{\pgfqpoint{4.746027in}{2.981924in}}%
\pgfpathlineto{\pgfqpoint{4.760215in}{2.981362in}}%
\pgfpathlineto{\pgfqpoint{4.774403in}{2.980464in}}%
\pgfpathlineto{\pgfqpoint{4.788592in}{2.979507in}}%
\pgfpathlineto{\pgfqpoint{4.802780in}{2.978976in}}%
\pgfpathlineto{\pgfqpoint{4.816969in}{2.978020in}}%
\pgfpathlineto{\pgfqpoint{4.831157in}{2.976734in}}%
\pgfpathlineto{\pgfqpoint{4.845345in}{2.975835in}}%
\pgfpathlineto{\pgfqpoint{4.859534in}{2.975622in}}%
\pgfpathlineto{\pgfqpoint{4.873722in}{2.974623in}}%
\pgfpathlineto{\pgfqpoint{4.887911in}{2.973806in}}%
\pgfpathlineto{\pgfqpoint{4.902099in}{2.972896in}}%
\pgfpathlineto{\pgfqpoint{4.916288in}{2.971800in}}%
\pgfpathlineto{\pgfqpoint{4.930476in}{2.971219in}}%
\pgfpathlineto{\pgfqpoint{4.944664in}{2.970518in}}%
\pgfpathlineto{\pgfqpoint{4.958853in}{2.970130in}}%
\pgfpathlineto{\pgfqpoint{4.973041in}{2.969224in}}%
\pgfpathlineto{\pgfqpoint{4.987230in}{2.968845in}}%
\pgfpathlineto{\pgfqpoint{5.001418in}{2.968159in}}%
\pgfpathlineto{\pgfqpoint{5.015607in}{2.967338in}}%
\pgfpathlineto{\pgfqpoint{5.029795in}{2.967009in}}%
\pgfpathlineto{\pgfqpoint{5.043983in}{2.966273in}}%
\pgfpathlineto{\pgfqpoint{5.058172in}{2.965529in}}%
\pgfusepath{stroke}%
\end{pgfscope}%
\begin{pgfscope}%
\pgfpathrectangle{\pgfqpoint{3.511634in}{2.920132in}}{\pgfqpoint{1.560726in}{0.791260in}}%
\pgfusepath{clip}%
\pgfsetrectcap%
\pgfsetroundjoin%
\pgfsetlinewidth{1.003750pt}%
\definecolor{currentstroke}{rgb}{0.000000,0.000000,0.000000}%
\pgfsetstrokecolor{currentstroke}%
\pgfsetdash{}{0pt}%
\pgfpathmoveto{\pgfqpoint{3.696083in}{2.920132in}}%
\pgfpathlineto{\pgfqpoint{3.696083in}{3.711392in}}%
\pgfusepath{stroke}%
\end{pgfscope}%
\begin{pgfscope}%
\pgfsetrectcap%
\pgfsetmiterjoin%
\pgfsetlinewidth{0.803000pt}%
\definecolor{currentstroke}{rgb}{0.000000,0.000000,0.000000}%
\pgfsetstrokecolor{currentstroke}%
\pgfsetdash{}{0pt}%
\pgfpathmoveto{\pgfqpoint{3.511634in}{2.920132in}}%
\pgfpathlineto{\pgfqpoint{3.511634in}{3.711392in}}%
\pgfusepath{stroke}%
\end{pgfscope}%
\begin{pgfscope}%
\pgfsetrectcap%
\pgfsetmiterjoin%
\pgfsetlinewidth{0.803000pt}%
\definecolor{currentstroke}{rgb}{0.000000,0.000000,0.000000}%
\pgfsetstrokecolor{currentstroke}%
\pgfsetdash{}{0pt}%
\pgfpathmoveto{\pgfqpoint{5.072360in}{2.920132in}}%
\pgfpathlineto{\pgfqpoint{5.072360in}{3.711392in}}%
\pgfusepath{stroke}%
\end{pgfscope}%
\begin{pgfscope}%
\pgfsetrectcap%
\pgfsetmiterjoin%
\pgfsetlinewidth{0.803000pt}%
\definecolor{currentstroke}{rgb}{0.000000,0.000000,0.000000}%
\pgfsetstrokecolor{currentstroke}%
\pgfsetdash{}{0pt}%
\pgfpathmoveto{\pgfqpoint{3.511634in}{2.920132in}}%
\pgfpathlineto{\pgfqpoint{5.072360in}{2.920132in}}%
\pgfusepath{stroke}%
\end{pgfscope}%
\begin{pgfscope}%
\pgfsetrectcap%
\pgfsetmiterjoin%
\pgfsetlinewidth{0.803000pt}%
\definecolor{currentstroke}{rgb}{0.000000,0.000000,0.000000}%
\pgfsetstrokecolor{currentstroke}%
\pgfsetdash{}{0pt}%
\pgfpathmoveto{\pgfqpoint{3.511634in}{3.711392in}}%
\pgfpathlineto{\pgfqpoint{5.072360in}{3.711392in}}%
\pgfusepath{stroke}%
\end{pgfscope}%
\begin{pgfscope}%
\pgfsetbuttcap%
\pgfsetmiterjoin%
\definecolor{currentfill}{rgb}{1.000000,1.000000,1.000000}%
\pgfsetfillcolor{currentfill}%
\pgfsetfillopacity{0.800000}%
\pgfsetlinewidth{1.003750pt}%
\definecolor{currentstroke}{rgb}{0.800000,0.800000,0.800000}%
\pgfsetstrokecolor{currentstroke}%
\pgfsetstrokeopacity{0.800000}%
\pgfsetdash{}{0pt}%
\pgfpathmoveto{\pgfqpoint{0.654624in}{3.462821in}}%
\pgfpathlineto{\pgfqpoint{1.876657in}{3.462821in}}%
\pgfpathquadraticcurveto{\pgfqpoint{1.904435in}{3.462821in}}{\pgfqpoint{1.904435in}{3.490598in}}%
\pgfpathlineto{\pgfqpoint{1.904435in}{3.864042in}}%
\pgfpathquadraticcurveto{\pgfqpoint{1.904435in}{3.891820in}}{\pgfqpoint{1.876657in}{3.891820in}}%
\pgfpathlineto{\pgfqpoint{0.654624in}{3.891820in}}%
\pgfpathquadraticcurveto{\pgfqpoint{0.626847in}{3.891820in}}{\pgfqpoint{0.626847in}{3.864042in}}%
\pgfpathlineto{\pgfqpoint{0.626847in}{3.490598in}}%
\pgfpathquadraticcurveto{\pgfqpoint{0.626847in}{3.462821in}}{\pgfqpoint{0.654624in}{3.462821in}}%
\pgfpathlineto{\pgfqpoint{0.654624in}{3.462821in}}%
\pgfpathclose%
\pgfusepath{stroke,fill}%
\end{pgfscope}%
\begin{pgfscope}%
\pgfsetrectcap%
\pgfsetroundjoin%
\pgfsetlinewidth{1.505625pt}%
\definecolor{currentstroke}{rgb}{0.121569,0.466667,0.705882}%
\pgfsetstrokecolor{currentstroke}%
\pgfsetdash{}{0pt}%
\pgfpathmoveto{\pgfqpoint{0.682402in}{3.787653in}}%
\pgfpathlineto{\pgfqpoint{0.821291in}{3.787653in}}%
\pgfpathlineto{\pgfqpoint{0.960180in}{3.787653in}}%
\pgfusepath{stroke}%
\end{pgfscope}%
\begin{pgfscope}%
\definecolor{textcolor}{rgb}{0.000000,0.000000,0.000000}%
\pgfsetstrokecolor{textcolor}%
\pgfsetfillcolor{textcolor}%
\pgftext[x=1.071291in,y=3.739042in,left,base]{\color{textcolor}\rmfamily\fontsize{10.000000}{12.000000}\selectfont Dunkelbilder}%
\end{pgfscope}%
\begin{pgfscope}%
\pgfsetrectcap%
\pgfsetroundjoin%
\pgfsetlinewidth{1.505625pt}%
\definecolor{currentstroke}{rgb}{1.000000,0.498039,0.054902}%
\pgfsetstrokecolor{currentstroke}%
\pgfsetdash{}{0pt}%
\pgfpathmoveto{\pgfqpoint{0.682402in}{3.593987in}}%
\pgfpathlineto{\pgfqpoint{0.821291in}{3.593987in}}%
\pgfpathlineto{\pgfqpoint{0.960180in}{3.593987in}}%
\pgfusepath{stroke}%
\end{pgfscope}%
\begin{pgfscope}%
\definecolor{textcolor}{rgb}{0.000000,0.000000,0.000000}%
\pgfsetstrokecolor{textcolor}%
\pgfsetfillcolor{textcolor}%
\pgftext[x=1.071291in,y=3.545376in,left,base]{\color{textcolor}\rmfamily\fontsize{10.000000}{12.000000}\selectfont Streubilder}%
\end{pgfscope}%
\begin{pgfscope}%
\pgfsetbuttcap%
\pgfsetmiterjoin%
\definecolor{currentfill}{rgb}{1.000000,1.000000,1.000000}%
\pgfsetfillcolor{currentfill}%
\pgfsetlinewidth{0.000000pt}%
\definecolor{currentstroke}{rgb}{0.000000,0.000000,0.000000}%
\pgfsetstrokecolor{currentstroke}%
\pgfsetstrokeopacity{0.000000}%
\pgfsetdash{}{0pt}%
\pgfpathmoveto{\pgfqpoint{0.557402in}{0.398088in}}%
\pgfpathlineto{\pgfqpoint{6.131424in}{0.398088in}}%
\pgfpathlineto{\pgfqpoint{6.131424in}{1.786264in}}%
\pgfpathlineto{\pgfqpoint{0.557402in}{1.786264in}}%
\pgfpathlineto{\pgfqpoint{0.557402in}{0.398088in}}%
\pgfpathclose%
\pgfusepath{fill}%
\end{pgfscope}%
\begin{pgfscope}%
\pgfsetbuttcap%
\pgfsetroundjoin%
\definecolor{currentfill}{rgb}{0.000000,0.000000,0.000000}%
\pgfsetfillcolor{currentfill}%
\pgfsetlinewidth{0.803000pt}%
\definecolor{currentstroke}{rgb}{0.000000,0.000000,0.000000}%
\pgfsetstrokecolor{currentstroke}%
\pgfsetdash{}{0pt}%
\pgfsys@defobject{currentmarker}{\pgfqpoint{0.000000in}{-0.048611in}}{\pgfqpoint{0.000000in}{0.000000in}}{%
\pgfpathmoveto{\pgfqpoint{0.000000in}{0.000000in}}%
\pgfpathlineto{\pgfqpoint{0.000000in}{-0.048611in}}%
\pgfusepath{stroke,fill}%
}%
\begin{pgfscope}%
\pgfsys@transformshift{0.900419in}{0.398088in}%
\pgfsys@useobject{currentmarker}{}%
\end{pgfscope}%
\end{pgfscope}%
\begin{pgfscope}%
\definecolor{textcolor}{rgb}{0.000000,0.000000,0.000000}%
\pgfsetstrokecolor{textcolor}%
\pgfsetfillcolor{textcolor}%
\pgftext[x=0.900419in,y=0.300866in,,top]{\color{textcolor}\rmfamily\fontsize{10.000000}{12.000000}\selectfont \(\displaystyle {0}\)}%
\end{pgfscope}%
\begin{pgfscope}%
\pgfsetbuttcap%
\pgfsetroundjoin%
\definecolor{currentfill}{rgb}{0.000000,0.000000,0.000000}%
\pgfsetfillcolor{currentfill}%
\pgfsetlinewidth{0.803000pt}%
\definecolor{currentstroke}{rgb}{0.000000,0.000000,0.000000}%
\pgfsetstrokecolor{currentstroke}%
\pgfsetdash{}{0pt}%
\pgfsys@defobject{currentmarker}{\pgfqpoint{0.000000in}{-0.048611in}}{\pgfqpoint{0.000000in}{0.000000in}}{%
\pgfpathmoveto{\pgfqpoint{0.000000in}{0.000000in}}%
\pgfpathlineto{\pgfqpoint{0.000000in}{-0.048611in}}%
\pgfusepath{stroke,fill}%
}%
\begin{pgfscope}%
\pgfsys@transformshift{1.757961in}{0.398088in}%
\pgfsys@useobject{currentmarker}{}%
\end{pgfscope}%
\end{pgfscope}%
\begin{pgfscope}%
\definecolor{textcolor}{rgb}{0.000000,0.000000,0.000000}%
\pgfsetstrokecolor{textcolor}%
\pgfsetfillcolor{textcolor}%
\pgftext[x=1.757961in,y=0.300866in,,top]{\color{textcolor}\rmfamily\fontsize{10.000000}{12.000000}\selectfont \(\displaystyle {50}\)}%
\end{pgfscope}%
\begin{pgfscope}%
\pgfsetbuttcap%
\pgfsetroundjoin%
\definecolor{currentfill}{rgb}{0.000000,0.000000,0.000000}%
\pgfsetfillcolor{currentfill}%
\pgfsetlinewidth{0.803000pt}%
\definecolor{currentstroke}{rgb}{0.000000,0.000000,0.000000}%
\pgfsetstrokecolor{currentstroke}%
\pgfsetdash{}{0pt}%
\pgfsys@defobject{currentmarker}{\pgfqpoint{0.000000in}{-0.048611in}}{\pgfqpoint{0.000000in}{0.000000in}}{%
\pgfpathmoveto{\pgfqpoint{0.000000in}{0.000000in}}%
\pgfpathlineto{\pgfqpoint{0.000000in}{-0.048611in}}%
\pgfusepath{stroke,fill}%
}%
\begin{pgfscope}%
\pgfsys@transformshift{2.615503in}{0.398088in}%
\pgfsys@useobject{currentmarker}{}%
\end{pgfscope}%
\end{pgfscope}%
\begin{pgfscope}%
\definecolor{textcolor}{rgb}{0.000000,0.000000,0.000000}%
\pgfsetstrokecolor{textcolor}%
\pgfsetfillcolor{textcolor}%
\pgftext[x=2.615503in,y=0.300866in,,top]{\color{textcolor}\rmfamily\fontsize{10.000000}{12.000000}\selectfont \(\displaystyle {100}\)}%
\end{pgfscope}%
\begin{pgfscope}%
\pgfsetbuttcap%
\pgfsetroundjoin%
\definecolor{currentfill}{rgb}{0.000000,0.000000,0.000000}%
\pgfsetfillcolor{currentfill}%
\pgfsetlinewidth{0.803000pt}%
\definecolor{currentstroke}{rgb}{0.000000,0.000000,0.000000}%
\pgfsetstrokecolor{currentstroke}%
\pgfsetdash{}{0pt}%
\pgfsys@defobject{currentmarker}{\pgfqpoint{0.000000in}{-0.048611in}}{\pgfqpoint{0.000000in}{0.000000in}}{%
\pgfpathmoveto{\pgfqpoint{0.000000in}{0.000000in}}%
\pgfpathlineto{\pgfqpoint{0.000000in}{-0.048611in}}%
\pgfusepath{stroke,fill}%
}%
\begin{pgfscope}%
\pgfsys@transformshift{3.473045in}{0.398088in}%
\pgfsys@useobject{currentmarker}{}%
\end{pgfscope}%
\end{pgfscope}%
\begin{pgfscope}%
\definecolor{textcolor}{rgb}{0.000000,0.000000,0.000000}%
\pgfsetstrokecolor{textcolor}%
\pgfsetfillcolor{textcolor}%
\pgftext[x=3.473045in,y=0.300866in,,top]{\color{textcolor}\rmfamily\fontsize{10.000000}{12.000000}\selectfont \(\displaystyle {150}\)}%
\end{pgfscope}%
\begin{pgfscope}%
\pgfsetbuttcap%
\pgfsetroundjoin%
\definecolor{currentfill}{rgb}{0.000000,0.000000,0.000000}%
\pgfsetfillcolor{currentfill}%
\pgfsetlinewidth{0.803000pt}%
\definecolor{currentstroke}{rgb}{0.000000,0.000000,0.000000}%
\pgfsetstrokecolor{currentstroke}%
\pgfsetdash{}{0pt}%
\pgfsys@defobject{currentmarker}{\pgfqpoint{0.000000in}{-0.048611in}}{\pgfqpoint{0.000000in}{0.000000in}}{%
\pgfpathmoveto{\pgfqpoint{0.000000in}{0.000000in}}%
\pgfpathlineto{\pgfqpoint{0.000000in}{-0.048611in}}%
\pgfusepath{stroke,fill}%
}%
\begin{pgfscope}%
\pgfsys@transformshift{4.330586in}{0.398088in}%
\pgfsys@useobject{currentmarker}{}%
\end{pgfscope}%
\end{pgfscope}%
\begin{pgfscope}%
\definecolor{textcolor}{rgb}{0.000000,0.000000,0.000000}%
\pgfsetstrokecolor{textcolor}%
\pgfsetfillcolor{textcolor}%
\pgftext[x=4.330586in,y=0.300866in,,top]{\color{textcolor}\rmfamily\fontsize{10.000000}{12.000000}\selectfont \(\displaystyle {200}\)}%
\end{pgfscope}%
\begin{pgfscope}%
\pgfsetbuttcap%
\pgfsetroundjoin%
\definecolor{currentfill}{rgb}{0.000000,0.000000,0.000000}%
\pgfsetfillcolor{currentfill}%
\pgfsetlinewidth{0.803000pt}%
\definecolor{currentstroke}{rgb}{0.000000,0.000000,0.000000}%
\pgfsetstrokecolor{currentstroke}%
\pgfsetdash{}{0pt}%
\pgfsys@defobject{currentmarker}{\pgfqpoint{0.000000in}{-0.048611in}}{\pgfqpoint{0.000000in}{0.000000in}}{%
\pgfpathmoveto{\pgfqpoint{0.000000in}{0.000000in}}%
\pgfpathlineto{\pgfqpoint{0.000000in}{-0.048611in}}%
\pgfusepath{stroke,fill}%
}%
\begin{pgfscope}%
\pgfsys@transformshift{5.188128in}{0.398088in}%
\pgfsys@useobject{currentmarker}{}%
\end{pgfscope}%
\end{pgfscope}%
\begin{pgfscope}%
\definecolor{textcolor}{rgb}{0.000000,0.000000,0.000000}%
\pgfsetstrokecolor{textcolor}%
\pgfsetfillcolor{textcolor}%
\pgftext[x=5.188128in,y=0.300866in,,top]{\color{textcolor}\rmfamily\fontsize{10.000000}{12.000000}\selectfont \(\displaystyle {250}\)}%
\end{pgfscope}%
\begin{pgfscope}%
\pgfsetbuttcap%
\pgfsetroundjoin%
\definecolor{currentfill}{rgb}{0.000000,0.000000,0.000000}%
\pgfsetfillcolor{currentfill}%
\pgfsetlinewidth{0.803000pt}%
\definecolor{currentstroke}{rgb}{0.000000,0.000000,0.000000}%
\pgfsetstrokecolor{currentstroke}%
\pgfsetdash{}{0pt}%
\pgfsys@defobject{currentmarker}{\pgfqpoint{0.000000in}{-0.048611in}}{\pgfqpoint{0.000000in}{0.000000in}}{%
\pgfpathmoveto{\pgfqpoint{0.000000in}{0.000000in}}%
\pgfpathlineto{\pgfqpoint{0.000000in}{-0.048611in}}%
\pgfusepath{stroke,fill}%
}%
\begin{pgfscope}%
\pgfsys@transformshift{6.045670in}{0.398088in}%
\pgfsys@useobject{currentmarker}{}%
\end{pgfscope}%
\end{pgfscope}%
\begin{pgfscope}%
\definecolor{textcolor}{rgb}{0.000000,0.000000,0.000000}%
\pgfsetstrokecolor{textcolor}%
\pgfsetfillcolor{textcolor}%
\pgftext[x=6.045670in,y=0.300866in,,top]{\color{textcolor}\rmfamily\fontsize{10.000000}{12.000000}\selectfont \(\displaystyle {300}\)}%
\end{pgfscope}%
\begin{pgfscope}%
\definecolor{textcolor}{rgb}{0.000000,0.000000,0.000000}%
\pgfsetstrokecolor{textcolor}%
\pgfsetfillcolor{textcolor}%
\pgftext[x=3.344413in,y=0.122655in,,top]{\color{textcolor}\rmfamily\fontsize{10.000000}{12.000000}\selectfont Clusterwert \(\displaystyle W\) in ADU}%
\end{pgfscope}%
\begin{pgfscope}%
\pgfsetbuttcap%
\pgfsetroundjoin%
\definecolor{currentfill}{rgb}{0.000000,0.000000,0.000000}%
\pgfsetfillcolor{currentfill}%
\pgfsetlinewidth{0.803000pt}%
\definecolor{currentstroke}{rgb}{0.000000,0.000000,0.000000}%
\pgfsetstrokecolor{currentstroke}%
\pgfsetdash{}{0pt}%
\pgfsys@defobject{currentmarker}{\pgfqpoint{-0.048611in}{0.000000in}}{\pgfqpoint{-0.000000in}{0.000000in}}{%
\pgfpathmoveto{\pgfqpoint{-0.000000in}{0.000000in}}%
\pgfpathlineto{\pgfqpoint{-0.048611in}{0.000000in}}%
\pgfusepath{stroke,fill}%
}%
\begin{pgfscope}%
\pgfsys@transformshift{0.557402in}{0.461188in}%
\pgfsys@useobject{currentmarker}{}%
\end{pgfscope}%
\end{pgfscope}%
\begin{pgfscope}%
\definecolor{textcolor}{rgb}{0.000000,0.000000,0.000000}%
\pgfsetstrokecolor{textcolor}%
\pgfsetfillcolor{textcolor}%
\pgftext[x=0.282710in, y=0.413363in, left, base]{\color{textcolor}\rmfamily\fontsize{10.000000}{12.000000}\selectfont \num{0.0}}%
\end{pgfscope}%
\begin{pgfscope}%
\pgfsetbuttcap%
\pgfsetroundjoin%
\definecolor{currentfill}{rgb}{0.000000,0.000000,0.000000}%
\pgfsetfillcolor{currentfill}%
\pgfsetlinewidth{0.803000pt}%
\definecolor{currentstroke}{rgb}{0.000000,0.000000,0.000000}%
\pgfsetstrokecolor{currentstroke}%
\pgfsetdash{}{0pt}%
\pgfsys@defobject{currentmarker}{\pgfqpoint{-0.048611in}{0.000000in}}{\pgfqpoint{-0.000000in}{0.000000in}}{%
\pgfpathmoveto{\pgfqpoint{-0.000000in}{0.000000in}}%
\pgfpathlineto{\pgfqpoint{-0.048611in}{0.000000in}}%
\pgfusepath{stroke,fill}%
}%
\begin{pgfscope}%
\pgfsys@transformshift{0.557402in}{0.882809in}%
\pgfsys@useobject{currentmarker}{}%
\end{pgfscope}%
\end{pgfscope}%
\begin{pgfscope}%
\definecolor{textcolor}{rgb}{0.000000,0.000000,0.000000}%
\pgfsetstrokecolor{textcolor}%
\pgfsetfillcolor{textcolor}%
\pgftext[x=0.282710in, y=0.834984in, left, base]{\color{textcolor}\rmfamily\fontsize{10.000000}{12.000000}\selectfont \num{0.5}}%
\end{pgfscope}%
\begin{pgfscope}%
\pgfsetbuttcap%
\pgfsetroundjoin%
\definecolor{currentfill}{rgb}{0.000000,0.000000,0.000000}%
\pgfsetfillcolor{currentfill}%
\pgfsetlinewidth{0.803000pt}%
\definecolor{currentstroke}{rgb}{0.000000,0.000000,0.000000}%
\pgfsetstrokecolor{currentstroke}%
\pgfsetdash{}{0pt}%
\pgfsys@defobject{currentmarker}{\pgfqpoint{-0.048611in}{0.000000in}}{\pgfqpoint{-0.000000in}{0.000000in}}{%
\pgfpathmoveto{\pgfqpoint{-0.000000in}{0.000000in}}%
\pgfpathlineto{\pgfqpoint{-0.048611in}{0.000000in}}%
\pgfusepath{stroke,fill}%
}%
\begin{pgfscope}%
\pgfsys@transformshift{0.557402in}{1.304429in}%
\pgfsys@useobject{currentmarker}{}%
\end{pgfscope}%
\end{pgfscope}%
\begin{pgfscope}%
\definecolor{textcolor}{rgb}{0.000000,0.000000,0.000000}%
\pgfsetstrokecolor{textcolor}%
\pgfsetfillcolor{textcolor}%
\pgftext[x=0.282710in, y=1.256605in, left, base]{\color{textcolor}\rmfamily\fontsize{10.000000}{12.000000}\selectfont \num{1.0}}%
\end{pgfscope}%
\begin{pgfscope}%
\pgfsetbuttcap%
\pgfsetroundjoin%
\definecolor{currentfill}{rgb}{0.000000,0.000000,0.000000}%
\pgfsetfillcolor{currentfill}%
\pgfsetlinewidth{0.803000pt}%
\definecolor{currentstroke}{rgb}{0.000000,0.000000,0.000000}%
\pgfsetstrokecolor{currentstroke}%
\pgfsetdash{}{0pt}%
\pgfsys@defobject{currentmarker}{\pgfqpoint{-0.048611in}{0.000000in}}{\pgfqpoint{-0.000000in}{0.000000in}}{%
\pgfpathmoveto{\pgfqpoint{-0.000000in}{0.000000in}}%
\pgfpathlineto{\pgfqpoint{-0.048611in}{0.000000in}}%
\pgfusepath{stroke,fill}%
}%
\begin{pgfscope}%
\pgfsys@transformshift{0.557402in}{1.726050in}%
\pgfsys@useobject{currentmarker}{}%
\end{pgfscope}%
\end{pgfscope}%
\begin{pgfscope}%
\definecolor{textcolor}{rgb}{0.000000,0.000000,0.000000}%
\pgfsetstrokecolor{textcolor}%
\pgfsetfillcolor{textcolor}%
\pgftext[x=0.282710in, y=1.678226in, left, base]{\color{textcolor}\rmfamily\fontsize{10.000000}{12.000000}\selectfont \num{1.5}}%
\end{pgfscope}%
\begin{pgfscope}%
\definecolor{textcolor}{rgb}{0.000000,0.000000,0.000000}%
\pgfsetstrokecolor{textcolor}%
\pgfsetfillcolor{textcolor}%
\pgftext[x=0.227155in,y=1.092176in,,bottom,rotate=90.000000]{\color{textcolor}\rmfamily\fontsize{10.000000}{12.000000}\selectfont Clusterzahl}%
\end{pgfscope}%
\begin{pgfscope}%
\definecolor{textcolor}{rgb}{0.000000,0.000000,0.000000}%
\pgfsetstrokecolor{textcolor}%
\pgfsetfillcolor{textcolor}%
\pgftext[x=0.557402in,y=1.827931in,left,base]{\color{textcolor}\rmfamily\fontsize{10.000000}{12.000000}\selectfont \(\displaystyle \times{10^{9}}{}\)}%
\end{pgfscope}%
\begin{pgfscope}%
\pgfpathrectangle{\pgfqpoint{0.557402in}{0.398088in}}{\pgfqpoint{5.574022in}{1.388176in}}%
\pgfusepath{clip}%
\pgfsetrectcap%
\pgfsetroundjoin%
\pgfsetlinewidth{1.505625pt}%
\definecolor{currentstroke}{rgb}{0.121569,0.466667,0.705882}%
\pgfsetstrokecolor{currentstroke}%
\pgfsetdash{}{0pt}%
\pgfpathmoveto{\pgfqpoint{0.555402in}{0.461219in}}%
\pgfpathlineto{\pgfqpoint{0.883268in}{0.461329in}}%
\pgfpathlineto{\pgfqpoint{0.900419in}{1.723165in}}%
\pgfpathlineto{\pgfqpoint{0.917570in}{0.461360in}}%
\pgfpathlineto{\pgfqpoint{2.203883in}{0.462118in}}%
\pgfpathlineto{\pgfqpoint{3.833212in}{0.461202in}}%
\pgfpathlineto{\pgfqpoint{6.133424in}{0.461188in}}%
\pgfpathlineto{\pgfqpoint{6.133424in}{0.461188in}}%
\pgfusepath{stroke}%
\end{pgfscope}%
\begin{pgfscope}%
\pgfpathrectangle{\pgfqpoint{0.557402in}{0.398088in}}{\pgfqpoint{5.574022in}{1.388176in}}%
\pgfusepath{clip}%
\pgfsetrectcap%
\pgfsetroundjoin%
\pgfsetlinewidth{1.505625pt}%
\definecolor{currentstroke}{rgb}{1.000000,0.498039,0.054902}%
\pgfsetstrokecolor{currentstroke}%
\pgfsetdash{}{0pt}%
\pgfpathmoveto{\pgfqpoint{0.555402in}{0.461579in}}%
\pgfpathlineto{\pgfqpoint{0.883268in}{0.461631in}}%
\pgfpathlineto{\pgfqpoint{0.900419in}{1.721983in}}%
\pgfpathlineto{\pgfqpoint{0.917570in}{0.461638in}}%
\pgfpathlineto{\pgfqpoint{2.221033in}{0.461521in}}%
\pgfpathlineto{\pgfqpoint{3.850363in}{0.461201in}}%
\pgfpathlineto{\pgfqpoint{6.133424in}{0.461189in}}%
\pgfpathlineto{\pgfqpoint{6.133424in}{0.461189in}}%
\pgfusepath{stroke}%
\end{pgfscope}%
\begin{pgfscope}%
\pgfsetrectcap%
\pgfsetmiterjoin%
\pgfsetlinewidth{0.803000pt}%
\definecolor{currentstroke}{rgb}{0.000000,0.000000,0.000000}%
\pgfsetstrokecolor{currentstroke}%
\pgfsetdash{}{0pt}%
\pgfpathmoveto{\pgfqpoint{0.557402in}{0.398088in}}%
\pgfpathlineto{\pgfqpoint{0.557402in}{1.786264in}}%
\pgfusepath{stroke}%
\end{pgfscope}%
\begin{pgfscope}%
\pgfsetrectcap%
\pgfsetmiterjoin%
\pgfsetlinewidth{0.803000pt}%
\definecolor{currentstroke}{rgb}{0.000000,0.000000,0.000000}%
\pgfsetstrokecolor{currentstroke}%
\pgfsetdash{}{0pt}%
\pgfpathmoveto{\pgfqpoint{6.131424in}{0.398088in}}%
\pgfpathlineto{\pgfqpoint{6.131424in}{1.786264in}}%
\pgfusepath{stroke}%
\end{pgfscope}%
\begin{pgfscope}%
\pgfsetrectcap%
\pgfsetmiterjoin%
\pgfsetlinewidth{0.803000pt}%
\definecolor{currentstroke}{rgb}{0.000000,0.000000,0.000000}%
\pgfsetstrokecolor{currentstroke}%
\pgfsetdash{}{0pt}%
\pgfpathmoveto{\pgfqpoint{0.557402in}{0.398088in}}%
\pgfpathlineto{\pgfqpoint{6.131424in}{0.398088in}}%
\pgfusepath{stroke}%
\end{pgfscope}%
\begin{pgfscope}%
\pgfsetrectcap%
\pgfsetmiterjoin%
\pgfsetlinewidth{0.803000pt}%
\definecolor{currentstroke}{rgb}{0.000000,0.000000,0.000000}%
\pgfsetstrokecolor{currentstroke}%
\pgfsetdash{}{0pt}%
\pgfpathmoveto{\pgfqpoint{0.557402in}{1.786264in}}%
\pgfpathlineto{\pgfqpoint{6.131424in}{1.786264in}}%
\pgfusepath{stroke}%
\end{pgfscope}%
\begin{pgfscope}%
\definecolor{textcolor}{rgb}{0.000000,0.000000,0.000000}%
\pgfsetstrokecolor{textcolor}%
\pgfsetfillcolor{textcolor}%
\pgftext[x=0.000000in,y=1.925082in,left,base]{\color{textcolor}\rmfamily\fontsize{10.000000}{12.000000}\selectfont (b)}%
\end{pgfscope}%
\begin{pgfscope}%
\pgfpathrectangle{\pgfqpoint{0.557402in}{0.398088in}}{\pgfqpoint{5.574022in}{1.388176in}}%
\pgfusepath{clip}%
\pgfsetbuttcap%
\pgfsetmiterjoin%
\pgfsetlinewidth{1.003750pt}%
\definecolor{currentstroke}{rgb}{0.000000,0.000000,0.000000}%
\pgfsetstrokecolor{currentstroke}%
\pgfsetstrokeopacity{0.500000}%
\pgfsetdash{}{0pt}%
\pgfpathmoveto{\pgfqpoint{3.610251in}{0.461187in}}%
\pgfpathlineto{\pgfqpoint{6.131424in}{0.461187in}}%
\pgfpathlineto{\pgfqpoint{6.131424in}{0.461205in}}%
\pgfpathlineto{\pgfqpoint{3.610251in}{0.461205in}}%
\pgfpathlineto{\pgfqpoint{3.610251in}{0.461187in}}%
\pgfpathclose%
\pgfusepath{stroke}%
\end{pgfscope}%
\begin{pgfscope}%
\pgfsetroundcap%
\pgfsetroundjoin%
\pgfsetlinewidth{1.003750pt}%
\definecolor{currentstroke}{rgb}{0.000000,0.000000,0.000000}%
\pgfsetstrokecolor{currentstroke}%
\pgfsetstrokeopacity{0.500000}%
\pgfsetdash{}{0pt}%
\pgfpathmoveto{\pgfqpoint{1.839427in}{0.745132in}}%
\pgfpathquadraticcurveto{\pgfqpoint{2.724839in}{0.603160in}}{\pgfqpoint{3.610251in}{0.461187in}}%
\pgfusepath{stroke}%
\end{pgfscope}%
\begin{pgfscope}%
\pgfsetroundcap%
\pgfsetroundjoin%
\pgfsetlinewidth{1.003750pt}%
\definecolor{currentstroke}{rgb}{0.000000,0.000000,0.000000}%
\pgfsetstrokecolor{currentstroke}%
\pgfsetstrokeopacity{0.500000}%
\pgfsetdash{}{0pt}%
\pgfpathmoveto{\pgfqpoint{3.790335in}{1.536392in}}%
\pgfpathquadraticcurveto{\pgfqpoint{4.960880in}{0.998799in}}{\pgfqpoint{6.131424in}{0.461205in}}%
\pgfusepath{stroke}%
\end{pgfscope}%
\begin{pgfscope}%
\pgfsetbuttcap%
\pgfsetmiterjoin%
\definecolor{currentfill}{rgb}{1.000000,1.000000,1.000000}%
\pgfsetfillcolor{currentfill}%
\pgfsetlinewidth{0.000000pt}%
\definecolor{currentstroke}{rgb}{0.000000,0.000000,0.000000}%
\pgfsetstrokecolor{currentstroke}%
\pgfsetstrokeopacity{0.000000}%
\pgfsetdash{}{0pt}%
\pgfpathmoveto{\pgfqpoint{1.839427in}{0.745132in}}%
\pgfpathlineto{\pgfqpoint{3.790335in}{0.745132in}}%
\pgfpathlineto{\pgfqpoint{3.790335in}{1.536392in}}%
\pgfpathlineto{\pgfqpoint{1.839427in}{1.536392in}}%
\pgfpathlineto{\pgfqpoint{1.839427in}{0.745132in}}%
\pgfpathclose%
\pgfusepath{fill}%
\end{pgfscope}%
\begin{pgfscope}%
\pgfsetbuttcap%
\pgfsetroundjoin%
\definecolor{currentfill}{rgb}{0.000000,0.000000,0.000000}%
\pgfsetfillcolor{currentfill}%
\pgfsetlinewidth{0.803000pt}%
\definecolor{currentstroke}{rgb}{0.000000,0.000000,0.000000}%
\pgfsetstrokecolor{currentstroke}%
\pgfsetdash{}{0pt}%
\pgfsys@defobject{currentmarker}{\pgfqpoint{0.000000in}{0.000000in}}{\pgfqpoint{0.000000in}{0.048611in}}{%
\pgfpathmoveto{\pgfqpoint{0.000000in}{0.000000in}}%
\pgfpathlineto{\pgfqpoint{0.000000in}{0.048611in}}%
\pgfusepath{stroke,fill}%
}%
\begin{pgfscope}%
\pgfsys@transformshift{2.131400in}{1.536392in}%
\pgfsys@useobject{currentmarker}{}%
\end{pgfscope}%
\end{pgfscope}%
\begin{pgfscope}%
\definecolor{textcolor}{rgb}{0.000000,0.000000,0.000000}%
\pgfsetstrokecolor{textcolor}%
\pgfsetfillcolor{textcolor}%
\pgftext[x=2.131400in,y=1.633615in,,bottom]{\color{textcolor}\rmfamily\fontsize{10.000000}{12.000000}\selectfont \(\displaystyle {180}\)}%
\end{pgfscope}%
\begin{pgfscope}%
\pgfsetbuttcap%
\pgfsetroundjoin%
\definecolor{currentfill}{rgb}{0.000000,0.000000,0.000000}%
\pgfsetfillcolor{currentfill}%
\pgfsetlinewidth{0.803000pt}%
\definecolor{currentstroke}{rgb}{0.000000,0.000000,0.000000}%
\pgfsetstrokecolor{currentstroke}%
\pgfsetdash{}{0pt}%
\pgfsys@defobject{currentmarker}{\pgfqpoint{0.000000in}{0.000000in}}{\pgfqpoint{0.000000in}{0.048611in}}{%
\pgfpathmoveto{\pgfqpoint{0.000000in}{0.000000in}}%
\pgfpathlineto{\pgfqpoint{0.000000in}{0.048611in}}%
\pgfusepath{stroke,fill}%
}%
\begin{pgfscope}%
\pgfsys@transformshift{2.529544in}{1.536392in}%
\pgfsys@useobject{currentmarker}{}%
\end{pgfscope}%
\end{pgfscope}%
\begin{pgfscope}%
\definecolor{textcolor}{rgb}{0.000000,0.000000,0.000000}%
\pgfsetstrokecolor{textcolor}%
\pgfsetfillcolor{textcolor}%
\pgftext[x=2.529544in,y=1.633615in,,bottom]{\color{textcolor}\rmfamily\fontsize{10.000000}{12.000000}\selectfont \(\displaystyle {210}\)}%
\end{pgfscope}%
\begin{pgfscope}%
\pgfsetbuttcap%
\pgfsetroundjoin%
\definecolor{currentfill}{rgb}{0.000000,0.000000,0.000000}%
\pgfsetfillcolor{currentfill}%
\pgfsetlinewidth{0.803000pt}%
\definecolor{currentstroke}{rgb}{0.000000,0.000000,0.000000}%
\pgfsetstrokecolor{currentstroke}%
\pgfsetdash{}{0pt}%
\pgfsys@defobject{currentmarker}{\pgfqpoint{0.000000in}{0.000000in}}{\pgfqpoint{0.000000in}{0.048611in}}{%
\pgfpathmoveto{\pgfqpoint{0.000000in}{0.000000in}}%
\pgfpathlineto{\pgfqpoint{0.000000in}{0.048611in}}%
\pgfusepath{stroke,fill}%
}%
\begin{pgfscope}%
\pgfsys@transformshift{2.927689in}{1.536392in}%
\pgfsys@useobject{currentmarker}{}%
\end{pgfscope}%
\end{pgfscope}%
\begin{pgfscope}%
\definecolor{textcolor}{rgb}{0.000000,0.000000,0.000000}%
\pgfsetstrokecolor{textcolor}%
\pgfsetfillcolor{textcolor}%
\pgftext[x=2.927689in,y=1.633615in,,bottom]{\color{textcolor}\rmfamily\fontsize{10.000000}{12.000000}\selectfont \(\displaystyle {240}\)}%
\end{pgfscope}%
\begin{pgfscope}%
\pgfsetbuttcap%
\pgfsetroundjoin%
\definecolor{currentfill}{rgb}{0.000000,0.000000,0.000000}%
\pgfsetfillcolor{currentfill}%
\pgfsetlinewidth{0.803000pt}%
\definecolor{currentstroke}{rgb}{0.000000,0.000000,0.000000}%
\pgfsetstrokecolor{currentstroke}%
\pgfsetdash{}{0pt}%
\pgfsys@defobject{currentmarker}{\pgfqpoint{0.000000in}{0.000000in}}{\pgfqpoint{0.000000in}{0.048611in}}{%
\pgfpathmoveto{\pgfqpoint{0.000000in}{0.000000in}}%
\pgfpathlineto{\pgfqpoint{0.000000in}{0.048611in}}%
\pgfusepath{stroke,fill}%
}%
\begin{pgfscope}%
\pgfsys@transformshift{3.325833in}{1.536392in}%
\pgfsys@useobject{currentmarker}{}%
\end{pgfscope}%
\end{pgfscope}%
\begin{pgfscope}%
\definecolor{textcolor}{rgb}{0.000000,0.000000,0.000000}%
\pgfsetstrokecolor{textcolor}%
\pgfsetfillcolor{textcolor}%
\pgftext[x=3.325833in,y=1.633615in,,bottom]{\color{textcolor}\rmfamily\fontsize{10.000000}{12.000000}\selectfont \(\displaystyle {270}\)}%
\end{pgfscope}%
\begin{pgfscope}%
\pgfsetbuttcap%
\pgfsetroundjoin%
\definecolor{currentfill}{rgb}{0.000000,0.000000,0.000000}%
\pgfsetfillcolor{currentfill}%
\pgfsetlinewidth{0.803000pt}%
\definecolor{currentstroke}{rgb}{0.000000,0.000000,0.000000}%
\pgfsetstrokecolor{currentstroke}%
\pgfsetdash{}{0pt}%
\pgfsys@defobject{currentmarker}{\pgfqpoint{0.000000in}{0.000000in}}{\pgfqpoint{0.000000in}{0.048611in}}{%
\pgfpathmoveto{\pgfqpoint{0.000000in}{0.000000in}}%
\pgfpathlineto{\pgfqpoint{0.000000in}{0.048611in}}%
\pgfusepath{stroke,fill}%
}%
\begin{pgfscope}%
\pgfsys@transformshift{3.723978in}{1.536392in}%
\pgfsys@useobject{currentmarker}{}%
\end{pgfscope}%
\end{pgfscope}%
\begin{pgfscope}%
\definecolor{textcolor}{rgb}{0.000000,0.000000,0.000000}%
\pgfsetstrokecolor{textcolor}%
\pgfsetfillcolor{textcolor}%
\pgftext[x=3.723978in,y=1.633615in,,bottom]{\color{textcolor}\rmfamily\fontsize{10.000000}{12.000000}\selectfont \(\displaystyle {300}\)}%
\end{pgfscope}%
\begin{pgfscope}%
\pgfsetbuttcap%
\pgfsetroundjoin%
\definecolor{currentfill}{rgb}{0.000000,0.000000,0.000000}%
\pgfsetfillcolor{currentfill}%
\pgfsetlinewidth{0.602250pt}%
\definecolor{currentstroke}{rgb}{0.000000,0.000000,0.000000}%
\pgfsetstrokecolor{currentstroke}%
\pgfsetdash{}{0pt}%
\pgfsys@defobject{currentmarker}{\pgfqpoint{0.000000in}{0.000000in}}{\pgfqpoint{0.000000in}{0.027778in}}{%
\pgfpathmoveto{\pgfqpoint{0.000000in}{0.000000in}}%
\pgfpathlineto{\pgfqpoint{0.000000in}{0.027778in}}%
\pgfusepath{stroke,fill}%
}%
\begin{pgfscope}%
\pgfsys@transformshift{1.932328in}{1.536392in}%
\pgfsys@useobject{currentmarker}{}%
\end{pgfscope}%
\end{pgfscope}%
\begin{pgfscope}%
\pgfsetbuttcap%
\pgfsetroundjoin%
\definecolor{currentfill}{rgb}{0.000000,0.000000,0.000000}%
\pgfsetfillcolor{currentfill}%
\pgfsetlinewidth{0.602250pt}%
\definecolor{currentstroke}{rgb}{0.000000,0.000000,0.000000}%
\pgfsetstrokecolor{currentstroke}%
\pgfsetdash{}{0pt}%
\pgfsys@defobject{currentmarker}{\pgfqpoint{0.000000in}{0.000000in}}{\pgfqpoint{0.000000in}{0.027778in}}{%
\pgfpathmoveto{\pgfqpoint{0.000000in}{0.000000in}}%
\pgfpathlineto{\pgfqpoint{0.000000in}{0.027778in}}%
\pgfusepath{stroke,fill}%
}%
\begin{pgfscope}%
\pgfsys@transformshift{2.031864in}{1.536392in}%
\pgfsys@useobject{currentmarker}{}%
\end{pgfscope}%
\end{pgfscope}%
\begin{pgfscope}%
\pgfsetbuttcap%
\pgfsetroundjoin%
\definecolor{currentfill}{rgb}{0.000000,0.000000,0.000000}%
\pgfsetfillcolor{currentfill}%
\pgfsetlinewidth{0.602250pt}%
\definecolor{currentstroke}{rgb}{0.000000,0.000000,0.000000}%
\pgfsetstrokecolor{currentstroke}%
\pgfsetdash{}{0pt}%
\pgfsys@defobject{currentmarker}{\pgfqpoint{0.000000in}{0.000000in}}{\pgfqpoint{0.000000in}{0.027778in}}{%
\pgfpathmoveto{\pgfqpoint{0.000000in}{0.000000in}}%
\pgfpathlineto{\pgfqpoint{0.000000in}{0.027778in}}%
\pgfusepath{stroke,fill}%
}%
\begin{pgfscope}%
\pgfsys@transformshift{2.230936in}{1.536392in}%
\pgfsys@useobject{currentmarker}{}%
\end{pgfscope}%
\end{pgfscope}%
\begin{pgfscope}%
\pgfsetbuttcap%
\pgfsetroundjoin%
\definecolor{currentfill}{rgb}{0.000000,0.000000,0.000000}%
\pgfsetfillcolor{currentfill}%
\pgfsetlinewidth{0.602250pt}%
\definecolor{currentstroke}{rgb}{0.000000,0.000000,0.000000}%
\pgfsetstrokecolor{currentstroke}%
\pgfsetdash{}{0pt}%
\pgfsys@defobject{currentmarker}{\pgfqpoint{0.000000in}{0.000000in}}{\pgfqpoint{0.000000in}{0.027778in}}{%
\pgfpathmoveto{\pgfqpoint{0.000000in}{0.000000in}}%
\pgfpathlineto{\pgfqpoint{0.000000in}{0.027778in}}%
\pgfusepath{stroke,fill}%
}%
\begin{pgfscope}%
\pgfsys@transformshift{2.330472in}{1.536392in}%
\pgfsys@useobject{currentmarker}{}%
\end{pgfscope}%
\end{pgfscope}%
\begin{pgfscope}%
\pgfsetbuttcap%
\pgfsetroundjoin%
\definecolor{currentfill}{rgb}{0.000000,0.000000,0.000000}%
\pgfsetfillcolor{currentfill}%
\pgfsetlinewidth{0.602250pt}%
\definecolor{currentstroke}{rgb}{0.000000,0.000000,0.000000}%
\pgfsetstrokecolor{currentstroke}%
\pgfsetdash{}{0pt}%
\pgfsys@defobject{currentmarker}{\pgfqpoint{0.000000in}{0.000000in}}{\pgfqpoint{0.000000in}{0.027778in}}{%
\pgfpathmoveto{\pgfqpoint{0.000000in}{0.000000in}}%
\pgfpathlineto{\pgfqpoint{0.000000in}{0.027778in}}%
\pgfusepath{stroke,fill}%
}%
\begin{pgfscope}%
\pgfsys@transformshift{2.430008in}{1.536392in}%
\pgfsys@useobject{currentmarker}{}%
\end{pgfscope}%
\end{pgfscope}%
\begin{pgfscope}%
\pgfsetbuttcap%
\pgfsetroundjoin%
\definecolor{currentfill}{rgb}{0.000000,0.000000,0.000000}%
\pgfsetfillcolor{currentfill}%
\pgfsetlinewidth{0.602250pt}%
\definecolor{currentstroke}{rgb}{0.000000,0.000000,0.000000}%
\pgfsetstrokecolor{currentstroke}%
\pgfsetdash{}{0pt}%
\pgfsys@defobject{currentmarker}{\pgfqpoint{0.000000in}{0.000000in}}{\pgfqpoint{0.000000in}{0.027778in}}{%
\pgfpathmoveto{\pgfqpoint{0.000000in}{0.000000in}}%
\pgfpathlineto{\pgfqpoint{0.000000in}{0.027778in}}%
\pgfusepath{stroke,fill}%
}%
\begin{pgfscope}%
\pgfsys@transformshift{2.629080in}{1.536392in}%
\pgfsys@useobject{currentmarker}{}%
\end{pgfscope}%
\end{pgfscope}%
\begin{pgfscope}%
\pgfsetbuttcap%
\pgfsetroundjoin%
\definecolor{currentfill}{rgb}{0.000000,0.000000,0.000000}%
\pgfsetfillcolor{currentfill}%
\pgfsetlinewidth{0.602250pt}%
\definecolor{currentstroke}{rgb}{0.000000,0.000000,0.000000}%
\pgfsetstrokecolor{currentstroke}%
\pgfsetdash{}{0pt}%
\pgfsys@defobject{currentmarker}{\pgfqpoint{0.000000in}{0.000000in}}{\pgfqpoint{0.000000in}{0.027778in}}{%
\pgfpathmoveto{\pgfqpoint{0.000000in}{0.000000in}}%
\pgfpathlineto{\pgfqpoint{0.000000in}{0.027778in}}%
\pgfusepath{stroke,fill}%
}%
\begin{pgfscope}%
\pgfsys@transformshift{2.728617in}{1.536392in}%
\pgfsys@useobject{currentmarker}{}%
\end{pgfscope}%
\end{pgfscope}%
\begin{pgfscope}%
\pgfsetbuttcap%
\pgfsetroundjoin%
\definecolor{currentfill}{rgb}{0.000000,0.000000,0.000000}%
\pgfsetfillcolor{currentfill}%
\pgfsetlinewidth{0.602250pt}%
\definecolor{currentstroke}{rgb}{0.000000,0.000000,0.000000}%
\pgfsetstrokecolor{currentstroke}%
\pgfsetdash{}{0pt}%
\pgfsys@defobject{currentmarker}{\pgfqpoint{0.000000in}{0.000000in}}{\pgfqpoint{0.000000in}{0.027778in}}{%
\pgfpathmoveto{\pgfqpoint{0.000000in}{0.000000in}}%
\pgfpathlineto{\pgfqpoint{0.000000in}{0.027778in}}%
\pgfusepath{stroke,fill}%
}%
\begin{pgfscope}%
\pgfsys@transformshift{2.828153in}{1.536392in}%
\pgfsys@useobject{currentmarker}{}%
\end{pgfscope}%
\end{pgfscope}%
\begin{pgfscope}%
\pgfsetbuttcap%
\pgfsetroundjoin%
\definecolor{currentfill}{rgb}{0.000000,0.000000,0.000000}%
\pgfsetfillcolor{currentfill}%
\pgfsetlinewidth{0.602250pt}%
\definecolor{currentstroke}{rgb}{0.000000,0.000000,0.000000}%
\pgfsetstrokecolor{currentstroke}%
\pgfsetdash{}{0pt}%
\pgfsys@defobject{currentmarker}{\pgfqpoint{0.000000in}{0.000000in}}{\pgfqpoint{0.000000in}{0.027778in}}{%
\pgfpathmoveto{\pgfqpoint{0.000000in}{0.000000in}}%
\pgfpathlineto{\pgfqpoint{0.000000in}{0.027778in}}%
\pgfusepath{stroke,fill}%
}%
\begin{pgfscope}%
\pgfsys@transformshift{3.027225in}{1.536392in}%
\pgfsys@useobject{currentmarker}{}%
\end{pgfscope}%
\end{pgfscope}%
\begin{pgfscope}%
\pgfsetbuttcap%
\pgfsetroundjoin%
\definecolor{currentfill}{rgb}{0.000000,0.000000,0.000000}%
\pgfsetfillcolor{currentfill}%
\pgfsetlinewidth{0.602250pt}%
\definecolor{currentstroke}{rgb}{0.000000,0.000000,0.000000}%
\pgfsetstrokecolor{currentstroke}%
\pgfsetdash{}{0pt}%
\pgfsys@defobject{currentmarker}{\pgfqpoint{0.000000in}{0.000000in}}{\pgfqpoint{0.000000in}{0.027778in}}{%
\pgfpathmoveto{\pgfqpoint{0.000000in}{0.000000in}}%
\pgfpathlineto{\pgfqpoint{0.000000in}{0.027778in}}%
\pgfusepath{stroke,fill}%
}%
\begin{pgfscope}%
\pgfsys@transformshift{3.126761in}{1.536392in}%
\pgfsys@useobject{currentmarker}{}%
\end{pgfscope}%
\end{pgfscope}%
\begin{pgfscope}%
\pgfsetbuttcap%
\pgfsetroundjoin%
\definecolor{currentfill}{rgb}{0.000000,0.000000,0.000000}%
\pgfsetfillcolor{currentfill}%
\pgfsetlinewidth{0.602250pt}%
\definecolor{currentstroke}{rgb}{0.000000,0.000000,0.000000}%
\pgfsetstrokecolor{currentstroke}%
\pgfsetdash{}{0pt}%
\pgfsys@defobject{currentmarker}{\pgfqpoint{0.000000in}{0.000000in}}{\pgfqpoint{0.000000in}{0.027778in}}{%
\pgfpathmoveto{\pgfqpoint{0.000000in}{0.000000in}}%
\pgfpathlineto{\pgfqpoint{0.000000in}{0.027778in}}%
\pgfusepath{stroke,fill}%
}%
\begin{pgfscope}%
\pgfsys@transformshift{3.226297in}{1.536392in}%
\pgfsys@useobject{currentmarker}{}%
\end{pgfscope}%
\end{pgfscope}%
\begin{pgfscope}%
\pgfsetbuttcap%
\pgfsetroundjoin%
\definecolor{currentfill}{rgb}{0.000000,0.000000,0.000000}%
\pgfsetfillcolor{currentfill}%
\pgfsetlinewidth{0.602250pt}%
\definecolor{currentstroke}{rgb}{0.000000,0.000000,0.000000}%
\pgfsetstrokecolor{currentstroke}%
\pgfsetdash{}{0pt}%
\pgfsys@defobject{currentmarker}{\pgfqpoint{0.000000in}{0.000000in}}{\pgfqpoint{0.000000in}{0.027778in}}{%
\pgfpathmoveto{\pgfqpoint{0.000000in}{0.000000in}}%
\pgfpathlineto{\pgfqpoint{0.000000in}{0.027778in}}%
\pgfusepath{stroke,fill}%
}%
\begin{pgfscope}%
\pgfsys@transformshift{3.425369in}{1.536392in}%
\pgfsys@useobject{currentmarker}{}%
\end{pgfscope}%
\end{pgfscope}%
\begin{pgfscope}%
\pgfsetbuttcap%
\pgfsetroundjoin%
\definecolor{currentfill}{rgb}{0.000000,0.000000,0.000000}%
\pgfsetfillcolor{currentfill}%
\pgfsetlinewidth{0.602250pt}%
\definecolor{currentstroke}{rgb}{0.000000,0.000000,0.000000}%
\pgfsetstrokecolor{currentstroke}%
\pgfsetdash{}{0pt}%
\pgfsys@defobject{currentmarker}{\pgfqpoint{0.000000in}{0.000000in}}{\pgfqpoint{0.000000in}{0.027778in}}{%
\pgfpathmoveto{\pgfqpoint{0.000000in}{0.000000in}}%
\pgfpathlineto{\pgfqpoint{0.000000in}{0.027778in}}%
\pgfusepath{stroke,fill}%
}%
\begin{pgfscope}%
\pgfsys@transformshift{3.524905in}{1.536392in}%
\pgfsys@useobject{currentmarker}{}%
\end{pgfscope}%
\end{pgfscope}%
\begin{pgfscope}%
\pgfsetbuttcap%
\pgfsetroundjoin%
\definecolor{currentfill}{rgb}{0.000000,0.000000,0.000000}%
\pgfsetfillcolor{currentfill}%
\pgfsetlinewidth{0.602250pt}%
\definecolor{currentstroke}{rgb}{0.000000,0.000000,0.000000}%
\pgfsetstrokecolor{currentstroke}%
\pgfsetdash{}{0pt}%
\pgfsys@defobject{currentmarker}{\pgfqpoint{0.000000in}{0.000000in}}{\pgfqpoint{0.000000in}{0.027778in}}{%
\pgfpathmoveto{\pgfqpoint{0.000000in}{0.000000in}}%
\pgfpathlineto{\pgfqpoint{0.000000in}{0.027778in}}%
\pgfusepath{stroke,fill}%
}%
\begin{pgfscope}%
\pgfsys@transformshift{3.624442in}{1.536392in}%
\pgfsys@useobject{currentmarker}{}%
\end{pgfscope}%
\end{pgfscope}%
\begin{pgfscope}%
\pgfsetbuttcap%
\pgfsetroundjoin%
\definecolor{currentfill}{rgb}{0.000000,0.000000,0.000000}%
\pgfsetfillcolor{currentfill}%
\pgfsetlinewidth{0.803000pt}%
\definecolor{currentstroke}{rgb}{0.000000,0.000000,0.000000}%
\pgfsetstrokecolor{currentstroke}%
\pgfsetdash{}{0pt}%
\pgfsys@defobject{currentmarker}{\pgfqpoint{-0.048611in}{0.000000in}}{\pgfqpoint{-0.000000in}{0.000000in}}{%
\pgfpathmoveto{\pgfqpoint{-0.000000in}{0.000000in}}%
\pgfpathlineto{\pgfqpoint{-0.048611in}{0.000000in}}%
\pgfusepath{stroke,fill}%
}%
\begin{pgfscope}%
\pgfsys@transformshift{1.839427in}{0.781098in}%
\pgfsys@useobject{currentmarker}{}%
\end{pgfscope}%
\end{pgfscope}%
\begin{pgfscope}%
\definecolor{textcolor}{rgb}{0.000000,0.000000,0.000000}%
\pgfsetstrokecolor{textcolor}%
\pgfsetfillcolor{textcolor}%
\pgftext[x=1.672760in, y=0.733271in, left, base]{\color{textcolor}\rmfamily\fontsize{10.000000}{12.000000}\selectfont \(\displaystyle {0}\)}%
\end{pgfscope}%
\begin{pgfscope}%
\pgfsetbuttcap%
\pgfsetroundjoin%
\definecolor{currentfill}{rgb}{0.000000,0.000000,0.000000}%
\pgfsetfillcolor{currentfill}%
\pgfsetlinewidth{0.803000pt}%
\definecolor{currentstroke}{rgb}{0.000000,0.000000,0.000000}%
\pgfsetstrokecolor{currentstroke}%
\pgfsetdash{}{0pt}%
\pgfsys@defobject{currentmarker}{\pgfqpoint{-0.048611in}{0.000000in}}{\pgfqpoint{-0.000000in}{0.000000in}}{%
\pgfpathmoveto{\pgfqpoint{-0.000000in}{0.000000in}}%
\pgfpathlineto{\pgfqpoint{-0.048611in}{0.000000in}}%
\pgfusepath{stroke,fill}%
}%
\begin{pgfscope}%
\pgfsys@transformshift{1.839427in}{1.503460in}%
\pgfsys@useobject{currentmarker}{}%
\end{pgfscope}%
\end{pgfscope}%
\begin{pgfscope}%
\definecolor{textcolor}{rgb}{0.000000,0.000000,0.000000}%
\pgfsetstrokecolor{textcolor}%
\pgfsetfillcolor{textcolor}%
\pgftext[x=1.394982in, y=1.455632in, left, base]{\color{textcolor}\rmfamily\fontsize{10.000000}{12.000000}\selectfont \(\displaystyle {20000}\)}%
\end{pgfscope}%
\begin{pgfscope}%
\pgfsetbuttcap%
\pgfsetroundjoin%
\definecolor{currentfill}{rgb}{0.000000,0.000000,0.000000}%
\pgfsetfillcolor{currentfill}%
\pgfsetlinewidth{0.602250pt}%
\definecolor{currentstroke}{rgb}{0.000000,0.000000,0.000000}%
\pgfsetstrokecolor{currentstroke}%
\pgfsetdash{}{0pt}%
\pgfsys@defobject{currentmarker}{\pgfqpoint{-0.027778in}{0.000000in}}{\pgfqpoint{-0.000000in}{0.000000in}}{%
\pgfpathmoveto{\pgfqpoint{-0.000000in}{0.000000in}}%
\pgfpathlineto{\pgfqpoint{-0.027778in}{0.000000in}}%
\pgfusepath{stroke,fill}%
}%
\begin{pgfscope}%
\pgfsys@transformshift{1.839427in}{0.961689in}%
\pgfsys@useobject{currentmarker}{}%
\end{pgfscope}%
\end{pgfscope}%
\begin{pgfscope}%
\pgfsetbuttcap%
\pgfsetroundjoin%
\definecolor{currentfill}{rgb}{0.000000,0.000000,0.000000}%
\pgfsetfillcolor{currentfill}%
\pgfsetlinewidth{0.602250pt}%
\definecolor{currentstroke}{rgb}{0.000000,0.000000,0.000000}%
\pgfsetstrokecolor{currentstroke}%
\pgfsetdash{}{0pt}%
\pgfsys@defobject{currentmarker}{\pgfqpoint{-0.027778in}{0.000000in}}{\pgfqpoint{-0.000000in}{0.000000in}}{%
\pgfpathmoveto{\pgfqpoint{-0.000000in}{0.000000in}}%
\pgfpathlineto{\pgfqpoint{-0.027778in}{0.000000in}}%
\pgfusepath{stroke,fill}%
}%
\begin{pgfscope}%
\pgfsys@transformshift{1.839427in}{1.142279in}%
\pgfsys@useobject{currentmarker}{}%
\end{pgfscope}%
\end{pgfscope}%
\begin{pgfscope}%
\pgfsetbuttcap%
\pgfsetroundjoin%
\definecolor{currentfill}{rgb}{0.000000,0.000000,0.000000}%
\pgfsetfillcolor{currentfill}%
\pgfsetlinewidth{0.602250pt}%
\definecolor{currentstroke}{rgb}{0.000000,0.000000,0.000000}%
\pgfsetstrokecolor{currentstroke}%
\pgfsetdash{}{0pt}%
\pgfsys@defobject{currentmarker}{\pgfqpoint{-0.027778in}{0.000000in}}{\pgfqpoint{-0.000000in}{0.000000in}}{%
\pgfpathmoveto{\pgfqpoint{-0.000000in}{0.000000in}}%
\pgfpathlineto{\pgfqpoint{-0.027778in}{0.000000in}}%
\pgfusepath{stroke,fill}%
}%
\begin{pgfscope}%
\pgfsys@transformshift{1.839427in}{1.322870in}%
\pgfsys@useobject{currentmarker}{}%
\end{pgfscope}%
\end{pgfscope}%
\begin{pgfscope}%
\pgfpathrectangle{\pgfqpoint{1.839427in}{0.745132in}}{\pgfqpoint{1.950908in}{0.791260in}}%
\pgfusepath{clip}%
\pgfsetrectcap%
\pgfsetroundjoin%
\pgfsetlinewidth{1.505625pt}%
\definecolor{currentstroke}{rgb}{0.121569,0.466667,0.705882}%
\pgfsetstrokecolor{currentstroke}%
\pgfsetdash{}{0pt}%
\pgfpathmoveto{\pgfqpoint{1.972142in}{1.500426in}}%
\pgfpathlineto{\pgfqpoint{1.985414in}{1.454014in}}%
\pgfpathlineto{\pgfqpoint{1.998685in}{1.415801in}}%
\pgfpathlineto{\pgfqpoint{2.011957in}{1.368812in}}%
\pgfpathlineto{\pgfqpoint{2.025228in}{1.348224in}}%
\pgfpathlineto{\pgfqpoint{2.038500in}{1.306002in}}%
\pgfpathlineto{\pgfqpoint{2.051771in}{1.267356in}}%
\pgfpathlineto{\pgfqpoint{2.065043in}{1.231202in}}%
\pgfpathlineto{\pgfqpoint{2.078314in}{1.212168in}}%
\pgfpathlineto{\pgfqpoint{2.091585in}{1.179950in}}%
\pgfpathlineto{\pgfqpoint{2.104857in}{1.148744in}}%
\pgfpathlineto{\pgfqpoint{2.118128in}{1.132672in}}%
\pgfpathlineto{\pgfqpoint{2.131400in}{1.108689in}}%
\pgfpathlineto{\pgfqpoint{2.144671in}{1.080517in}}%
\pgfpathlineto{\pgfqpoint{2.157943in}{1.065889in}}%
\pgfpathlineto{\pgfqpoint{2.171214in}{1.041112in}}%
\pgfpathlineto{\pgfqpoint{2.184486in}{1.025148in}}%
\pgfpathlineto{\pgfqpoint{2.197757in}{1.008462in}}%
\pgfpathlineto{\pgfqpoint{2.211029in}{0.996254in}}%
\pgfpathlineto{\pgfqpoint{2.224300in}{0.979314in}}%
\pgfpathlineto{\pgfqpoint{2.237572in}{0.969743in}}%
\pgfpathlineto{\pgfqpoint{2.250843in}{0.959124in}}%
\pgfpathlineto{\pgfqpoint{2.264115in}{0.939729in}}%
\pgfpathlineto{\pgfqpoint{2.277386in}{0.931747in}}%
\pgfpathlineto{\pgfqpoint{2.290658in}{0.918744in}}%
\pgfpathlineto{\pgfqpoint{2.303929in}{0.912604in}}%
\pgfpathlineto{\pgfqpoint{2.317201in}{0.902202in}}%
\pgfpathlineto{\pgfqpoint{2.330472in}{0.893895in}}%
\pgfpathlineto{\pgfqpoint{2.343744in}{0.888477in}}%
\pgfpathlineto{\pgfqpoint{2.357015in}{0.877317in}}%
\pgfpathlineto{\pgfqpoint{2.370287in}{0.873019in}}%
\pgfpathlineto{\pgfqpoint{2.383558in}{0.869190in}}%
\pgfpathlineto{\pgfqpoint{2.396830in}{0.860992in}}%
\pgfpathlineto{\pgfqpoint{2.410101in}{0.853985in}}%
\pgfpathlineto{\pgfqpoint{2.423373in}{0.847375in}}%
\pgfpathlineto{\pgfqpoint{2.436644in}{0.844739in}}%
\pgfpathlineto{\pgfqpoint{2.449915in}{0.839899in}}%
\pgfpathlineto{\pgfqpoint{2.463187in}{0.836323in}}%
\pgfpathlineto{\pgfqpoint{2.489730in}{0.827835in}}%
\pgfpathlineto{\pgfqpoint{2.503001in}{0.823357in}}%
\pgfpathlineto{\pgfqpoint{2.516273in}{0.821298in}}%
\pgfpathlineto{\pgfqpoint{2.529544in}{0.819817in}}%
\pgfpathlineto{\pgfqpoint{2.542816in}{0.815627in}}%
\pgfpathlineto{\pgfqpoint{2.556087in}{0.812377in}}%
\pgfpathlineto{\pgfqpoint{2.582630in}{0.808765in}}%
\pgfpathlineto{\pgfqpoint{2.595902in}{0.805695in}}%
\pgfpathlineto{\pgfqpoint{2.609173in}{0.805550in}}%
\pgfpathlineto{\pgfqpoint{2.622445in}{0.803853in}}%
\pgfpathlineto{\pgfqpoint{2.635716in}{0.801614in}}%
\pgfpathlineto{\pgfqpoint{2.648988in}{0.801686in}}%
\pgfpathlineto{\pgfqpoint{2.662259in}{0.798796in}}%
\pgfpathlineto{\pgfqpoint{2.675531in}{0.796701in}}%
\pgfpathlineto{\pgfqpoint{2.688802in}{0.796701in}}%
\pgfpathlineto{\pgfqpoint{2.702074in}{0.796954in}}%
\pgfpathlineto{\pgfqpoint{2.715345in}{0.794896in}}%
\pgfpathlineto{\pgfqpoint{2.755160in}{0.790886in}}%
\pgfpathlineto{\pgfqpoint{2.781702in}{0.790056in}}%
\pgfpathlineto{\pgfqpoint{2.821517in}{0.787889in}}%
\pgfpathlineto{\pgfqpoint{2.848060in}{0.786913in}}%
\pgfpathlineto{\pgfqpoint{2.861331in}{0.787275in}}%
\pgfpathlineto{\pgfqpoint{2.874603in}{0.785830in}}%
\pgfpathlineto{\pgfqpoint{2.887874in}{0.785180in}}%
\pgfpathlineto{\pgfqpoint{2.914417in}{0.784602in}}%
\pgfpathlineto{\pgfqpoint{2.927689in}{0.784999in}}%
\pgfpathlineto{\pgfqpoint{2.954232in}{0.783952in}}%
\pgfpathlineto{\pgfqpoint{2.967503in}{0.785035in}}%
\pgfpathlineto{\pgfqpoint{2.980775in}{0.783771in}}%
\pgfpathlineto{\pgfqpoint{2.994046in}{0.783771in}}%
\pgfpathlineto{\pgfqpoint{3.007318in}{0.783085in}}%
\pgfpathlineto{\pgfqpoint{3.020589in}{0.782760in}}%
\pgfpathlineto{\pgfqpoint{3.033861in}{0.783013in}}%
\pgfpathlineto{\pgfqpoint{3.047132in}{0.782868in}}%
\pgfpathlineto{\pgfqpoint{3.060404in}{0.783518in}}%
\pgfpathlineto{\pgfqpoint{3.073675in}{0.782904in}}%
\pgfpathlineto{\pgfqpoint{3.086947in}{0.782579in}}%
\pgfpathlineto{\pgfqpoint{3.100218in}{0.782688in}}%
\pgfpathlineto{\pgfqpoint{3.113490in}{0.782074in}}%
\pgfpathlineto{\pgfqpoint{3.126761in}{0.782363in}}%
\pgfpathlineto{\pgfqpoint{3.140032in}{0.781965in}}%
\pgfpathlineto{\pgfqpoint{3.166575in}{0.782001in}}%
\pgfpathlineto{\pgfqpoint{3.179847in}{0.781821in}}%
\pgfpathlineto{\pgfqpoint{3.193118in}{0.782146in}}%
\pgfpathlineto{\pgfqpoint{3.232933in}{0.781676in}}%
\pgfpathlineto{\pgfqpoint{3.246204in}{0.782001in}}%
\pgfpathlineto{\pgfqpoint{3.272747in}{0.781640in}}%
\pgfpathlineto{\pgfqpoint{3.312562in}{0.781424in}}%
\pgfpathlineto{\pgfqpoint{3.339105in}{0.781460in}}%
\pgfpathlineto{\pgfqpoint{3.352376in}{0.781568in}}%
\pgfpathlineto{\pgfqpoint{3.378919in}{0.781424in}}%
\pgfpathlineto{\pgfqpoint{3.485091in}{0.781207in}}%
\pgfpathlineto{\pgfqpoint{3.511634in}{0.781315in}}%
\pgfpathlineto{\pgfqpoint{3.551448in}{0.781135in}}%
\pgfpathlineto{\pgfqpoint{3.591263in}{0.781135in}}%
\pgfpathlineto{\pgfqpoint{3.657620in}{0.781135in}}%
\pgfpathlineto{\pgfqpoint{3.684163in}{0.781207in}}%
\pgfpathlineto{\pgfqpoint{3.723978in}{0.781171in}}%
\pgfpathlineto{\pgfqpoint{3.750521in}{0.781171in}}%
\pgfpathlineto{\pgfqpoint{3.792335in}{0.781135in}}%
\pgfpathlineto{\pgfqpoint{3.792335in}{0.781135in}}%
\pgfusepath{stroke}%
\end{pgfscope}%
\begin{pgfscope}%
\pgfpathrectangle{\pgfqpoint{1.839427in}{0.745132in}}{\pgfqpoint{1.950908in}{0.791260in}}%
\pgfusepath{clip}%
\pgfsetrectcap%
\pgfsetroundjoin%
\pgfsetlinewidth{1.505625pt}%
\definecolor{currentstroke}{rgb}{1.000000,0.498039,0.054902}%
\pgfsetstrokecolor{currentstroke}%
\pgfsetdash{}{0pt}%
\pgfpathmoveto{\pgfqpoint{1.972142in}{1.446646in}}%
\pgfpathlineto{\pgfqpoint{1.985414in}{1.431729in}}%
\pgfpathlineto{\pgfqpoint{1.998685in}{1.411576in}}%
\pgfpathlineto{\pgfqpoint{2.011957in}{1.388677in}}%
\pgfpathlineto{\pgfqpoint{2.025228in}{1.360432in}}%
\pgfpathlineto{\pgfqpoint{2.038500in}{1.347610in}}%
\pgfpathlineto{\pgfqpoint{2.051771in}{1.328901in}}%
\pgfpathlineto{\pgfqpoint{2.065043in}{1.311962in}}%
\pgfpathlineto{\pgfqpoint{2.078314in}{1.298598in}}%
\pgfpathlineto{\pgfqpoint{2.091585in}{1.285704in}}%
\pgfpathlineto{\pgfqpoint{2.104857in}{1.262950in}}%
\pgfpathlineto{\pgfqpoint{2.118128in}{1.252945in}}%
\pgfpathlineto{\pgfqpoint{2.131400in}{1.240665in}}%
\pgfpathlineto{\pgfqpoint{2.144671in}{1.229035in}}%
\pgfpathlineto{\pgfqpoint{2.157943in}{1.218019in}}%
\pgfpathlineto{\pgfqpoint{2.171214in}{1.207834in}}%
\pgfpathlineto{\pgfqpoint{2.184486in}{1.189341in}}%
\pgfpathlineto{\pgfqpoint{2.197757in}{1.185079in}}%
\pgfpathlineto{\pgfqpoint{2.211029in}{1.171860in}}%
\pgfpathlineto{\pgfqpoint{2.224300in}{1.163192in}}%
\pgfpathlineto{\pgfqpoint{2.237572in}{1.163372in}}%
\pgfpathlineto{\pgfqpoint{2.250843in}{1.155101in}}%
\pgfpathlineto{\pgfqpoint{2.264115in}{1.144013in}}%
\pgfpathlineto{\pgfqpoint{2.277386in}{1.137873in}}%
\pgfpathlineto{\pgfqpoint{2.290658in}{1.128663in}}%
\pgfpathlineto{\pgfqpoint{2.303929in}{1.127326in}}%
\pgfpathlineto{\pgfqpoint{2.317201in}{1.118514in}}%
\pgfpathlineto{\pgfqpoint{2.330472in}{1.110893in}}%
\pgfpathlineto{\pgfqpoint{2.343744in}{1.110387in}}%
\pgfpathlineto{\pgfqpoint{2.357015in}{1.097132in}}%
\pgfpathlineto{\pgfqpoint{2.370287in}{1.100924in}}%
\pgfpathlineto{\pgfqpoint{2.383558in}{1.093159in}}%
\pgfpathlineto{\pgfqpoint{2.396830in}{1.088644in}}%
\pgfpathlineto{\pgfqpoint{2.410101in}{1.086332in}}%
\pgfpathlineto{\pgfqpoint{2.423373in}{1.077014in}}%
\pgfpathlineto{\pgfqpoint{2.436644in}{1.069285in}}%
\pgfpathlineto{\pgfqpoint{2.449915in}{1.067623in}}%
\pgfpathlineto{\pgfqpoint{2.463187in}{1.062639in}}%
\pgfpathlineto{\pgfqpoint{2.476458in}{1.064373in}}%
\pgfpathlineto{\pgfqpoint{2.489730in}{1.053465in}}%
\pgfpathlineto{\pgfqpoint{2.503001in}{1.053934in}}%
\pgfpathlineto{\pgfqpoint{2.516273in}{1.053357in}}%
\pgfpathlineto{\pgfqpoint{2.529544in}{1.048192in}}%
\pgfpathlineto{\pgfqpoint{2.542816in}{1.047108in}}%
\pgfpathlineto{\pgfqpoint{2.569359in}{1.028074in}}%
\pgfpathlineto{\pgfqpoint{2.595902in}{1.029446in}}%
\pgfpathlineto{\pgfqpoint{2.609173in}{1.021789in}}%
\pgfpathlineto{\pgfqpoint{2.622445in}{1.022078in}}%
\pgfpathlineto{\pgfqpoint{2.635716in}{1.021031in}}%
\pgfpathlineto{\pgfqpoint{2.648988in}{1.016444in}}%
\pgfpathlineto{\pgfqpoint{2.662259in}{1.009220in}}%
\pgfpathlineto{\pgfqpoint{2.675531in}{1.010484in}}%
\pgfpathlineto{\pgfqpoint{2.688802in}{1.008137in}}%
\pgfpathlineto{\pgfqpoint{2.702074in}{1.003261in}}%
\pgfpathlineto{\pgfqpoint{2.715345in}{0.996615in}}%
\pgfpathlineto{\pgfqpoint{2.728617in}{0.991956in}}%
\pgfpathlineto{\pgfqpoint{2.741888in}{0.987874in}}%
\pgfpathlineto{\pgfqpoint{2.755160in}{0.993545in}}%
\pgfpathlineto{\pgfqpoint{2.768431in}{0.989716in}}%
\pgfpathlineto{\pgfqpoint{2.781702in}{0.983938in}}%
\pgfpathlineto{\pgfqpoint{2.794974in}{0.981120in}}%
\pgfpathlineto{\pgfqpoint{2.808245in}{0.979278in}}%
\pgfpathlineto{\pgfqpoint{2.821517in}{0.976642in}}%
\pgfpathlineto{\pgfqpoint{2.834788in}{0.969490in}}%
\pgfpathlineto{\pgfqpoint{2.848060in}{0.965951in}}%
\pgfpathlineto{\pgfqpoint{2.861331in}{0.964398in}}%
\pgfpathlineto{\pgfqpoint{2.874603in}{0.962375in}}%
\pgfpathlineto{\pgfqpoint{2.887874in}{0.960750in}}%
\pgfpathlineto{\pgfqpoint{2.901146in}{0.958799in}}%
\pgfpathlineto{\pgfqpoint{2.914417in}{0.956343in}}%
\pgfpathlineto{\pgfqpoint{2.927689in}{0.955404in}}%
\pgfpathlineto{\pgfqpoint{2.940960in}{0.951251in}}%
\pgfpathlineto{\pgfqpoint{2.954232in}{0.946230in}}%
\pgfpathlineto{\pgfqpoint{2.967503in}{0.941860in}}%
\pgfpathlineto{\pgfqpoint{2.980775in}{0.942871in}}%
\pgfpathlineto{\pgfqpoint{2.994046in}{0.936876in}}%
\pgfpathlineto{\pgfqpoint{3.007318in}{0.935684in}}%
\pgfpathlineto{\pgfqpoint{3.020589in}{0.934022in}}%
\pgfpathlineto{\pgfqpoint{3.033861in}{0.930808in}}%
\pgfpathlineto{\pgfqpoint{3.047132in}{0.925824in}}%
\pgfpathlineto{\pgfqpoint{3.060404in}{0.924776in}}%
\pgfpathlineto{\pgfqpoint{3.073675in}{0.924596in}}%
\pgfpathlineto{\pgfqpoint{3.086947in}{0.917155in}}%
\pgfpathlineto{\pgfqpoint{3.100218in}{0.916975in}}%
\pgfpathlineto{\pgfqpoint{3.126761in}{0.907620in}}%
\pgfpathlineto{\pgfqpoint{3.140032in}{0.908306in}}%
\pgfpathlineto{\pgfqpoint{3.153304in}{0.905742in}}%
\pgfpathlineto{\pgfqpoint{3.166575in}{0.901769in}}%
\pgfpathlineto{\pgfqpoint{3.179847in}{0.900649in}}%
\pgfpathlineto{\pgfqpoint{3.193118in}{0.900541in}}%
\pgfpathlineto{\pgfqpoint{3.206390in}{0.898952in}}%
\pgfpathlineto{\pgfqpoint{3.219661in}{0.896207in}}%
\pgfpathlineto{\pgfqpoint{3.232933in}{0.894581in}}%
\pgfpathlineto{\pgfqpoint{3.246204in}{0.886888in}}%
\pgfpathlineto{\pgfqpoint{3.259476in}{0.889633in}}%
\pgfpathlineto{\pgfqpoint{3.272747in}{0.880929in}}%
\pgfpathlineto{\pgfqpoint{3.286019in}{0.884071in}}%
\pgfpathlineto{\pgfqpoint{3.299290in}{0.880387in}}%
\pgfpathlineto{\pgfqpoint{3.312562in}{0.878762in}}%
\pgfpathlineto{\pgfqpoint{3.325833in}{0.875764in}}%
\pgfpathlineto{\pgfqpoint{3.339105in}{0.879520in}}%
\pgfpathlineto{\pgfqpoint{3.352376in}{0.872297in}}%
\pgfpathlineto{\pgfqpoint{3.365648in}{0.871574in}}%
\pgfpathlineto{\pgfqpoint{3.378919in}{0.869190in}}%
\pgfpathlineto{\pgfqpoint{3.392191in}{0.869010in}}%
\pgfpathlineto{\pgfqpoint{3.418734in}{0.863592in}}%
\pgfpathlineto{\pgfqpoint{3.432005in}{0.861642in}}%
\pgfpathlineto{\pgfqpoint{3.445277in}{0.863267in}}%
\pgfpathlineto{\pgfqpoint{3.458548in}{0.862292in}}%
\pgfpathlineto{\pgfqpoint{3.471820in}{0.855429in}}%
\pgfpathlineto{\pgfqpoint{3.485091in}{0.858102in}}%
\pgfpathlineto{\pgfqpoint{3.498362in}{0.856188in}}%
\pgfpathlineto{\pgfqpoint{3.511634in}{0.855032in}}%
\pgfpathlineto{\pgfqpoint{3.538177in}{0.853154in}}%
\pgfpathlineto{\pgfqpoint{3.551448in}{0.850409in}}%
\pgfpathlineto{\pgfqpoint{3.564720in}{0.847231in}}%
\pgfpathlineto{\pgfqpoint{3.577991in}{0.850156in}}%
\pgfpathlineto{\pgfqpoint{3.591263in}{0.847375in}}%
\pgfpathlineto{\pgfqpoint{3.604534in}{0.845750in}}%
\pgfpathlineto{\pgfqpoint{3.617806in}{0.848025in}}%
\pgfpathlineto{\pgfqpoint{3.631077in}{0.843763in}}%
\pgfpathlineto{\pgfqpoint{3.644349in}{0.844377in}}%
\pgfpathlineto{\pgfqpoint{3.657620in}{0.843691in}}%
\pgfpathlineto{\pgfqpoint{3.670892in}{0.840404in}}%
\pgfpathlineto{\pgfqpoint{3.684163in}{0.837768in}}%
\pgfpathlineto{\pgfqpoint{3.697435in}{0.838815in}}%
\pgfpathlineto{\pgfqpoint{3.710706in}{0.835131in}}%
\pgfpathlineto{\pgfqpoint{3.723978in}{0.834156in}}%
\pgfpathlineto{\pgfqpoint{3.737249in}{0.834914in}}%
\pgfpathlineto{\pgfqpoint{3.750521in}{0.834734in}}%
\pgfpathlineto{\pgfqpoint{3.763792in}{0.833614in}}%
\pgfpathlineto{\pgfqpoint{3.777064in}{0.833000in}}%
\pgfpathlineto{\pgfqpoint{3.792335in}{0.831287in}}%
\pgfpathlineto{\pgfqpoint{3.792335in}{0.831287in}}%
\pgfusepath{stroke}%
\end{pgfscope}%
\begin{pgfscope}%
\pgfpathrectangle{\pgfqpoint{1.839427in}{0.745132in}}{\pgfqpoint{1.950908in}{0.791260in}}%
\pgfusepath{clip}%
\pgfsetrectcap%
\pgfsetroundjoin%
\pgfsetlinewidth{1.003750pt}%
\definecolor{currentstroke}{rgb}{0.000000,0.000000,0.000000}%
\pgfsetstrokecolor{currentstroke}%
\pgfsetdash{}{0pt}%
\pgfpathmoveto{\pgfqpoint{2.011957in}{0.745132in}}%
\pgfpathlineto{\pgfqpoint{2.011957in}{1.536392in}}%
\pgfusepath{stroke}%
\end{pgfscope}%
\begin{pgfscope}%
\pgfsetrectcap%
\pgfsetmiterjoin%
\pgfsetlinewidth{0.803000pt}%
\definecolor{currentstroke}{rgb}{0.000000,0.000000,0.000000}%
\pgfsetstrokecolor{currentstroke}%
\pgfsetdash{}{0pt}%
\pgfpathmoveto{\pgfqpoint{1.839427in}{0.745132in}}%
\pgfpathlineto{\pgfqpoint{1.839427in}{1.536392in}}%
\pgfusepath{stroke}%
\end{pgfscope}%
\begin{pgfscope}%
\pgfsetrectcap%
\pgfsetmiterjoin%
\pgfsetlinewidth{0.803000pt}%
\definecolor{currentstroke}{rgb}{0.000000,0.000000,0.000000}%
\pgfsetstrokecolor{currentstroke}%
\pgfsetdash{}{0pt}%
\pgfpathmoveto{\pgfqpoint{3.790335in}{0.745132in}}%
\pgfpathlineto{\pgfqpoint{3.790335in}{1.536392in}}%
\pgfusepath{stroke}%
\end{pgfscope}%
\begin{pgfscope}%
\pgfsetrectcap%
\pgfsetmiterjoin%
\pgfsetlinewidth{0.803000pt}%
\definecolor{currentstroke}{rgb}{0.000000,0.000000,0.000000}%
\pgfsetstrokecolor{currentstroke}%
\pgfsetdash{}{0pt}%
\pgfpathmoveto{\pgfqpoint{1.839427in}{0.745132in}}%
\pgfpathlineto{\pgfqpoint{3.790335in}{0.745132in}}%
\pgfusepath{stroke}%
\end{pgfscope}%
\begin{pgfscope}%
\pgfsetrectcap%
\pgfsetmiterjoin%
\pgfsetlinewidth{0.803000pt}%
\definecolor{currentstroke}{rgb}{0.000000,0.000000,0.000000}%
\pgfsetstrokecolor{currentstroke}%
\pgfsetdash{}{0pt}%
\pgfpathmoveto{\pgfqpoint{1.839427in}{1.536392in}}%
\pgfpathlineto{\pgfqpoint{3.790335in}{1.536392in}}%
\pgfusepath{stroke}%
\end{pgfscope}%
\end{pgfpicture}%
\makeatother%
\endgroup%

    %%% Creator: Matplotlib, PGF backend
%%
%% To include the figure in your LaTeX document, write
%%   \input{<filename>.pgf}
%%
%% Make sure the required packages are loaded in your preamble
%%   \usepackage{pgf}
%%
%% Also ensure that all the required font packages are loaded; for instance,
%% the lmodern package is sometimes necessary when using math font.
%%   \usepackage{lmodern}
%%
%% Figures using additional raster images can only be included by \input if
%% they are in the same directory as the main LaTeX file. For loading figures
%% from other directories you can use the `import` package
%%   \usepackage{import}
%%
%% and then include the figures with
%%   \import{<path to file>}{<filename>.pgf}
%%
%% Matplotlib used the following preamble
%%   \usepackage{amsmath} \usepackage[utf8]{inputenc} \usepackage[T1]{fontenc} \usepackage[output-decimal-marker={,},print-unity-mantissa=false]{siunitx} \sisetup{per-mode=fraction, separate-uncertainty = true, locale = DE} \usepackage[acronym, toc, section=section, nonumberlist, nopostdot]{glossaries-extra} \DeclareSIUnit\adu{\text{ADU}} \DeclareSIUnit\px{\text{px}} \DeclareSIUnit\photons{\text{Pho\-to\-nen}} \DeclareSIUnit\photon{\text{Pho\-ton}}
%%
\begingroup%
\makeatletter%
\begin{pgfpicture}%
\pgfpathrectangle{\pgfpointorigin}{\pgfqpoint{6.235591in}{7.225494in}}%
\pgfusepath{use as bounding box, clip}%
\begin{pgfscope}%
\pgfsetbuttcap%
\pgfsetmiterjoin%
\pgfsetlinewidth{0.000000pt}%
\definecolor{currentstroke}{rgb}{1.000000,1.000000,1.000000}%
\pgfsetstrokecolor{currentstroke}%
\pgfsetstrokeopacity{0.000000}%
\pgfsetdash{}{0pt}%
\pgfpathmoveto{\pgfqpoint{0.000000in}{0.000000in}}%
\pgfpathlineto{\pgfqpoint{6.235591in}{0.000000in}}%
\pgfpathlineto{\pgfqpoint{6.235591in}{7.225494in}}%
\pgfpathlineto{\pgfqpoint{0.000000in}{7.225494in}}%
\pgfpathlineto{\pgfqpoint{0.000000in}{0.000000in}}%
\pgfpathclose%
\pgfusepath{}%
\end{pgfscope}%
\begin{pgfscope}%
\pgfsetbuttcap%
\pgfsetmiterjoin%
\definecolor{currentfill}{rgb}{1.000000,1.000000,1.000000}%
\pgfsetfillcolor{currentfill}%
\pgfsetlinewidth{0.000000pt}%
\definecolor{currentstroke}{rgb}{0.000000,0.000000,0.000000}%
\pgfsetstrokecolor{currentstroke}%
\pgfsetstrokeopacity{0.000000}%
\pgfsetdash{}{0pt}%
\pgfpathmoveto{\pgfqpoint{0.557402in}{5.298088in}}%
\pgfpathlineto{\pgfqpoint{6.131424in}{5.298088in}}%
\pgfpathlineto{\pgfqpoint{6.131424in}{6.955602in}}%
\pgfpathlineto{\pgfqpoint{0.557402in}{6.955602in}}%
\pgfpathlineto{\pgfqpoint{0.557402in}{5.298088in}}%
\pgfpathclose%
\pgfusepath{fill}%
\end{pgfscope}%
\begin{pgfscope}%
\pgfsetbuttcap%
\pgfsetroundjoin%
\definecolor{currentfill}{rgb}{0.000000,0.000000,0.000000}%
\pgfsetfillcolor{currentfill}%
\pgfsetlinewidth{0.803000pt}%
\definecolor{currentstroke}{rgb}{0.000000,0.000000,0.000000}%
\pgfsetstrokecolor{currentstroke}%
\pgfsetdash{}{0pt}%
\pgfsys@defobject{currentmarker}{\pgfqpoint{0.000000in}{-0.048611in}}{\pgfqpoint{0.000000in}{0.000000in}}{%
\pgfpathmoveto{\pgfqpoint{0.000000in}{0.000000in}}%
\pgfpathlineto{\pgfqpoint{0.000000in}{-0.048611in}}%
\pgfusepath{stroke,fill}%
}%
\begin{pgfscope}%
\pgfsys@transformshift{0.635910in}{5.298088in}%
\pgfsys@useobject{currentmarker}{}%
\end{pgfscope}%
\end{pgfscope}%
\begin{pgfscope}%
\definecolor{textcolor}{rgb}{0.000000,0.000000,0.000000}%
\pgfsetstrokecolor{textcolor}%
\pgfsetfillcolor{textcolor}%
\pgftext[x=0.635910in,y=5.200866in,,top]{\color{textcolor}\rmfamily\fontsize{10.000000}{12.000000}\selectfont \(\displaystyle {\ensuremath{-}200}\)}%
\end{pgfscope}%
\begin{pgfscope}%
\pgfsetbuttcap%
\pgfsetroundjoin%
\definecolor{currentfill}{rgb}{0.000000,0.000000,0.000000}%
\pgfsetfillcolor{currentfill}%
\pgfsetlinewidth{0.803000pt}%
\definecolor{currentstroke}{rgb}{0.000000,0.000000,0.000000}%
\pgfsetstrokecolor{currentstroke}%
\pgfsetdash{}{0pt}%
\pgfsys@defobject{currentmarker}{\pgfqpoint{0.000000in}{-0.048611in}}{\pgfqpoint{0.000000in}{0.000000in}}{%
\pgfpathmoveto{\pgfqpoint{0.000000in}{0.000000in}}%
\pgfpathlineto{\pgfqpoint{0.000000in}{-0.048611in}}%
\pgfusepath{stroke,fill}%
}%
\begin{pgfscope}%
\pgfsys@transformshift{1.420983in}{5.298088in}%
\pgfsys@useobject{currentmarker}{}%
\end{pgfscope}%
\end{pgfscope}%
\begin{pgfscope}%
\definecolor{textcolor}{rgb}{0.000000,0.000000,0.000000}%
\pgfsetstrokecolor{textcolor}%
\pgfsetfillcolor{textcolor}%
\pgftext[x=1.420983in,y=5.200866in,,top]{\color{textcolor}\rmfamily\fontsize{10.000000}{12.000000}\selectfont \(\displaystyle {\ensuremath{-}100}\)}%
\end{pgfscope}%
\begin{pgfscope}%
\pgfsetbuttcap%
\pgfsetroundjoin%
\definecolor{currentfill}{rgb}{0.000000,0.000000,0.000000}%
\pgfsetfillcolor{currentfill}%
\pgfsetlinewidth{0.803000pt}%
\definecolor{currentstroke}{rgb}{0.000000,0.000000,0.000000}%
\pgfsetstrokecolor{currentstroke}%
\pgfsetdash{}{0pt}%
\pgfsys@defobject{currentmarker}{\pgfqpoint{0.000000in}{-0.048611in}}{\pgfqpoint{0.000000in}{0.000000in}}{%
\pgfpathmoveto{\pgfqpoint{0.000000in}{0.000000in}}%
\pgfpathlineto{\pgfqpoint{0.000000in}{-0.048611in}}%
\pgfusepath{stroke,fill}%
}%
\begin{pgfscope}%
\pgfsys@transformshift{2.206057in}{5.298088in}%
\pgfsys@useobject{currentmarker}{}%
\end{pgfscope}%
\end{pgfscope}%
\begin{pgfscope}%
\definecolor{textcolor}{rgb}{0.000000,0.000000,0.000000}%
\pgfsetstrokecolor{textcolor}%
\pgfsetfillcolor{textcolor}%
\pgftext[x=2.206057in,y=5.200866in,,top]{\color{textcolor}\rmfamily\fontsize{10.000000}{12.000000}\selectfont \(\displaystyle {0}\)}%
\end{pgfscope}%
\begin{pgfscope}%
\pgfsetbuttcap%
\pgfsetroundjoin%
\definecolor{currentfill}{rgb}{0.000000,0.000000,0.000000}%
\pgfsetfillcolor{currentfill}%
\pgfsetlinewidth{0.803000pt}%
\definecolor{currentstroke}{rgb}{0.000000,0.000000,0.000000}%
\pgfsetstrokecolor{currentstroke}%
\pgfsetdash{}{0pt}%
\pgfsys@defobject{currentmarker}{\pgfqpoint{0.000000in}{-0.048611in}}{\pgfqpoint{0.000000in}{0.000000in}}{%
\pgfpathmoveto{\pgfqpoint{0.000000in}{0.000000in}}%
\pgfpathlineto{\pgfqpoint{0.000000in}{-0.048611in}}%
\pgfusepath{stroke,fill}%
}%
\begin{pgfscope}%
\pgfsys@transformshift{2.991130in}{5.298088in}%
\pgfsys@useobject{currentmarker}{}%
\end{pgfscope}%
\end{pgfscope}%
\begin{pgfscope}%
\definecolor{textcolor}{rgb}{0.000000,0.000000,0.000000}%
\pgfsetstrokecolor{textcolor}%
\pgfsetfillcolor{textcolor}%
\pgftext[x=2.991130in,y=5.200866in,,top]{\color{textcolor}\rmfamily\fontsize{10.000000}{12.000000}\selectfont \(\displaystyle {100}\)}%
\end{pgfscope}%
\begin{pgfscope}%
\pgfsetbuttcap%
\pgfsetroundjoin%
\definecolor{currentfill}{rgb}{0.000000,0.000000,0.000000}%
\pgfsetfillcolor{currentfill}%
\pgfsetlinewidth{0.803000pt}%
\definecolor{currentstroke}{rgb}{0.000000,0.000000,0.000000}%
\pgfsetstrokecolor{currentstroke}%
\pgfsetdash{}{0pt}%
\pgfsys@defobject{currentmarker}{\pgfqpoint{0.000000in}{-0.048611in}}{\pgfqpoint{0.000000in}{0.000000in}}{%
\pgfpathmoveto{\pgfqpoint{0.000000in}{0.000000in}}%
\pgfpathlineto{\pgfqpoint{0.000000in}{-0.048611in}}%
\pgfusepath{stroke,fill}%
}%
\begin{pgfscope}%
\pgfsys@transformshift{3.776204in}{5.298088in}%
\pgfsys@useobject{currentmarker}{}%
\end{pgfscope}%
\end{pgfscope}%
\begin{pgfscope}%
\definecolor{textcolor}{rgb}{0.000000,0.000000,0.000000}%
\pgfsetstrokecolor{textcolor}%
\pgfsetfillcolor{textcolor}%
\pgftext[x=3.776204in,y=5.200866in,,top]{\color{textcolor}\rmfamily\fontsize{10.000000}{12.000000}\selectfont \(\displaystyle {200}\)}%
\end{pgfscope}%
\begin{pgfscope}%
\pgfsetbuttcap%
\pgfsetroundjoin%
\definecolor{currentfill}{rgb}{0.000000,0.000000,0.000000}%
\pgfsetfillcolor{currentfill}%
\pgfsetlinewidth{0.803000pt}%
\definecolor{currentstroke}{rgb}{0.000000,0.000000,0.000000}%
\pgfsetstrokecolor{currentstroke}%
\pgfsetdash{}{0pt}%
\pgfsys@defobject{currentmarker}{\pgfqpoint{0.000000in}{-0.048611in}}{\pgfqpoint{0.000000in}{0.000000in}}{%
\pgfpathmoveto{\pgfqpoint{0.000000in}{0.000000in}}%
\pgfpathlineto{\pgfqpoint{0.000000in}{-0.048611in}}%
\pgfusepath{stroke,fill}%
}%
\begin{pgfscope}%
\pgfsys@transformshift{4.561277in}{5.298088in}%
\pgfsys@useobject{currentmarker}{}%
\end{pgfscope}%
\end{pgfscope}%
\begin{pgfscope}%
\definecolor{textcolor}{rgb}{0.000000,0.000000,0.000000}%
\pgfsetstrokecolor{textcolor}%
\pgfsetfillcolor{textcolor}%
\pgftext[x=4.561277in,y=5.200866in,,top]{\color{textcolor}\rmfamily\fontsize{10.000000}{12.000000}\selectfont \(\displaystyle {300}\)}%
\end{pgfscope}%
\begin{pgfscope}%
\pgfsetbuttcap%
\pgfsetroundjoin%
\definecolor{currentfill}{rgb}{0.000000,0.000000,0.000000}%
\pgfsetfillcolor{currentfill}%
\pgfsetlinewidth{0.803000pt}%
\definecolor{currentstroke}{rgb}{0.000000,0.000000,0.000000}%
\pgfsetstrokecolor{currentstroke}%
\pgfsetdash{}{0pt}%
\pgfsys@defobject{currentmarker}{\pgfqpoint{0.000000in}{-0.048611in}}{\pgfqpoint{0.000000in}{0.000000in}}{%
\pgfpathmoveto{\pgfqpoint{0.000000in}{0.000000in}}%
\pgfpathlineto{\pgfqpoint{0.000000in}{-0.048611in}}%
\pgfusepath{stroke,fill}%
}%
\begin{pgfscope}%
\pgfsys@transformshift{5.346351in}{5.298088in}%
\pgfsys@useobject{currentmarker}{}%
\end{pgfscope}%
\end{pgfscope}%
\begin{pgfscope}%
\definecolor{textcolor}{rgb}{0.000000,0.000000,0.000000}%
\pgfsetstrokecolor{textcolor}%
\pgfsetfillcolor{textcolor}%
\pgftext[x=5.346351in,y=5.200866in,,top]{\color{textcolor}\rmfamily\fontsize{10.000000}{12.000000}\selectfont \(\displaystyle {400}\)}%
\end{pgfscope}%
\begin{pgfscope}%
\pgfsetbuttcap%
\pgfsetroundjoin%
\definecolor{currentfill}{rgb}{0.000000,0.000000,0.000000}%
\pgfsetfillcolor{currentfill}%
\pgfsetlinewidth{0.803000pt}%
\definecolor{currentstroke}{rgb}{0.000000,0.000000,0.000000}%
\pgfsetstrokecolor{currentstroke}%
\pgfsetdash{}{0pt}%
\pgfsys@defobject{currentmarker}{\pgfqpoint{0.000000in}{-0.048611in}}{\pgfqpoint{0.000000in}{0.000000in}}{%
\pgfpathmoveto{\pgfqpoint{0.000000in}{0.000000in}}%
\pgfpathlineto{\pgfqpoint{0.000000in}{-0.048611in}}%
\pgfusepath{stroke,fill}%
}%
\begin{pgfscope}%
\pgfsys@transformshift{6.131424in}{5.298088in}%
\pgfsys@useobject{currentmarker}{}%
\end{pgfscope}%
\end{pgfscope}%
\begin{pgfscope}%
\definecolor{textcolor}{rgb}{0.000000,0.000000,0.000000}%
\pgfsetstrokecolor{textcolor}%
\pgfsetfillcolor{textcolor}%
\pgftext[x=6.131424in,y=5.200866in,,top]{\color{textcolor}\rmfamily\fontsize{10.000000}{12.000000}\selectfont \(\displaystyle {500}\)}%
\end{pgfscope}%
\begin{pgfscope}%
\definecolor{textcolor}{rgb}{0.000000,0.000000,0.000000}%
\pgfsetstrokecolor{textcolor}%
\pgfsetfillcolor{textcolor}%
\pgftext[x=3.344413in,y=5.022655in,,top]{\color{textcolor}\rmfamily\fontsize{10.000000}{12.000000}\selectfont Pixelwert \(\displaystyle W\) in \si{\adu}}%
\end{pgfscope}%
\begin{pgfscope}%
\pgfsetbuttcap%
\pgfsetroundjoin%
\definecolor{currentfill}{rgb}{0.000000,0.000000,0.000000}%
\pgfsetfillcolor{currentfill}%
\pgfsetlinewidth{0.803000pt}%
\definecolor{currentstroke}{rgb}{0.000000,0.000000,0.000000}%
\pgfsetstrokecolor{currentstroke}%
\pgfsetdash{}{0pt}%
\pgfsys@defobject{currentmarker}{\pgfqpoint{-0.048611in}{0.000000in}}{\pgfqpoint{-0.000000in}{0.000000in}}{%
\pgfpathmoveto{\pgfqpoint{-0.000000in}{0.000000in}}%
\pgfpathlineto{\pgfqpoint{-0.048611in}{0.000000in}}%
\pgfusepath{stroke,fill}%
}%
\begin{pgfscope}%
\pgfsys@transformshift{0.557402in}{5.373866in}%
\pgfsys@useobject{currentmarker}{}%
\end{pgfscope}%
\end{pgfscope}%
\begin{pgfscope}%
\definecolor{textcolor}{rgb}{0.000000,0.000000,0.000000}%
\pgfsetstrokecolor{textcolor}%
\pgfsetfillcolor{textcolor}%
\pgftext[x=0.282710in, y=5.326042in, left, base]{\color{textcolor}\rmfamily\fontsize{10.000000}{12.000000}\selectfont \num{0.0}}%
\end{pgfscope}%
\begin{pgfscope}%
\pgfsetbuttcap%
\pgfsetroundjoin%
\definecolor{currentfill}{rgb}{0.000000,0.000000,0.000000}%
\pgfsetfillcolor{currentfill}%
\pgfsetlinewidth{0.803000pt}%
\definecolor{currentstroke}{rgb}{0.000000,0.000000,0.000000}%
\pgfsetstrokecolor{currentstroke}%
\pgfsetdash{}{0pt}%
\pgfsys@defobject{currentmarker}{\pgfqpoint{-0.048611in}{0.000000in}}{\pgfqpoint{-0.000000in}{0.000000in}}{%
\pgfpathmoveto{\pgfqpoint{-0.000000in}{0.000000in}}%
\pgfpathlineto{\pgfqpoint{-0.048611in}{0.000000in}}%
\pgfusepath{stroke,fill}%
}%
\begin{pgfscope}%
\pgfsys@transformshift{0.557402in}{5.845478in}%
\pgfsys@useobject{currentmarker}{}%
\end{pgfscope}%
\end{pgfscope}%
\begin{pgfscope}%
\definecolor{textcolor}{rgb}{0.000000,0.000000,0.000000}%
\pgfsetstrokecolor{textcolor}%
\pgfsetfillcolor{textcolor}%
\pgftext[x=0.282710in, y=5.797653in, left, base]{\color{textcolor}\rmfamily\fontsize{10.000000}{12.000000}\selectfont \num{0.1}}%
\end{pgfscope}%
\begin{pgfscope}%
\pgfsetbuttcap%
\pgfsetroundjoin%
\definecolor{currentfill}{rgb}{0.000000,0.000000,0.000000}%
\pgfsetfillcolor{currentfill}%
\pgfsetlinewidth{0.803000pt}%
\definecolor{currentstroke}{rgb}{0.000000,0.000000,0.000000}%
\pgfsetstrokecolor{currentstroke}%
\pgfsetdash{}{0pt}%
\pgfsys@defobject{currentmarker}{\pgfqpoint{-0.048611in}{0.000000in}}{\pgfqpoint{-0.000000in}{0.000000in}}{%
\pgfpathmoveto{\pgfqpoint{-0.000000in}{0.000000in}}%
\pgfpathlineto{\pgfqpoint{-0.048611in}{0.000000in}}%
\pgfusepath{stroke,fill}%
}%
\begin{pgfscope}%
\pgfsys@transformshift{0.557402in}{6.317089in}%
\pgfsys@useobject{currentmarker}{}%
\end{pgfscope}%
\end{pgfscope}%
\begin{pgfscope}%
\definecolor{textcolor}{rgb}{0.000000,0.000000,0.000000}%
\pgfsetstrokecolor{textcolor}%
\pgfsetfillcolor{textcolor}%
\pgftext[x=0.282710in, y=6.269265in, left, base]{\color{textcolor}\rmfamily\fontsize{10.000000}{12.000000}\selectfont \num{0.2}}%
\end{pgfscope}%
\begin{pgfscope}%
\pgfsetbuttcap%
\pgfsetroundjoin%
\definecolor{currentfill}{rgb}{0.000000,0.000000,0.000000}%
\pgfsetfillcolor{currentfill}%
\pgfsetlinewidth{0.803000pt}%
\definecolor{currentstroke}{rgb}{0.000000,0.000000,0.000000}%
\pgfsetstrokecolor{currentstroke}%
\pgfsetdash{}{0pt}%
\pgfsys@defobject{currentmarker}{\pgfqpoint{-0.048611in}{0.000000in}}{\pgfqpoint{-0.000000in}{0.000000in}}{%
\pgfpathmoveto{\pgfqpoint{-0.000000in}{0.000000in}}%
\pgfpathlineto{\pgfqpoint{-0.048611in}{0.000000in}}%
\pgfusepath{stroke,fill}%
}%
\begin{pgfscope}%
\pgfsys@transformshift{0.557402in}{6.788700in}%
\pgfsys@useobject{currentmarker}{}%
\end{pgfscope}%
\end{pgfscope}%
\begin{pgfscope}%
\definecolor{textcolor}{rgb}{0.000000,0.000000,0.000000}%
\pgfsetstrokecolor{textcolor}%
\pgfsetfillcolor{textcolor}%
\pgftext[x=0.282710in, y=6.740876in, left, base]{\color{textcolor}\rmfamily\fontsize{10.000000}{12.000000}\selectfont \num{0.3}}%
\end{pgfscope}%
\begin{pgfscope}%
\definecolor{textcolor}{rgb}{0.000000,0.000000,0.000000}%
\pgfsetstrokecolor{textcolor}%
\pgfsetfillcolor{textcolor}%
\pgftext[x=0.227155in,y=6.126845in,,bottom,rotate=90.000000]{\color{textcolor}\rmfamily\fontsize{10.000000}{12.000000}\selectfont Pixelzahl}%
\end{pgfscope}%
\begin{pgfscope}%
\definecolor{textcolor}{rgb}{0.000000,0.000000,0.000000}%
\pgfsetstrokecolor{textcolor}%
\pgfsetfillcolor{textcolor}%
\pgftext[x=0.557402in,y=6.997268in,left,base]{\color{textcolor}\rmfamily\fontsize{10.000000}{12.000000}\selectfont \(\displaystyle \times{10^{8}}{}\)}%
\end{pgfscope}%
\begin{pgfscope}%
\pgfpathrectangle{\pgfqpoint{0.557402in}{5.298088in}}{\pgfqpoint{5.574022in}{1.657514in}}%
\pgfusepath{clip}%
\pgfsetrectcap%
\pgfsetroundjoin%
\pgfsetlinewidth{1.505625pt}%
\definecolor{currentstroke}{rgb}{0.121569,0.466667,0.705882}%
\pgfsetstrokecolor{currentstroke}%
\pgfsetdash{}{0pt}%
\pgfpathmoveto{\pgfqpoint{0.635910in}{5.373867in}}%
\pgfpathlineto{\pgfqpoint{1.397431in}{5.374085in}}%
\pgfpathlineto{\pgfqpoint{1.483789in}{5.374626in}}%
\pgfpathlineto{\pgfqpoint{1.523043in}{5.375179in}}%
\pgfpathlineto{\pgfqpoint{1.562296in}{5.376137in}}%
\pgfpathlineto{\pgfqpoint{1.585849in}{5.377004in}}%
\pgfpathlineto{\pgfqpoint{1.609401in}{5.378202in}}%
\pgfpathlineto{\pgfqpoint{1.632953in}{5.379932in}}%
\pgfpathlineto{\pgfqpoint{1.648654in}{5.381350in}}%
\pgfpathlineto{\pgfqpoint{1.664356in}{5.383243in}}%
\pgfpathlineto{\pgfqpoint{1.680057in}{5.385601in}}%
\pgfpathlineto{\pgfqpoint{1.695759in}{5.388533in}}%
\pgfpathlineto{\pgfqpoint{1.703610in}{5.390277in}}%
\pgfpathlineto{\pgfqpoint{1.711460in}{5.392264in}}%
\pgfpathlineto{\pgfqpoint{1.727162in}{5.396921in}}%
\pgfpathlineto{\pgfqpoint{1.735013in}{5.399621in}}%
\pgfpathlineto{\pgfqpoint{1.750714in}{5.406073in}}%
\pgfpathlineto{\pgfqpoint{1.758565in}{5.409903in}}%
\pgfpathlineto{\pgfqpoint{1.766415in}{5.414213in}}%
\pgfpathlineto{\pgfqpoint{1.774266in}{5.418820in}}%
\pgfpathlineto{\pgfqpoint{1.782117in}{5.424174in}}%
\pgfpathlineto{\pgfqpoint{1.789968in}{5.429974in}}%
\pgfpathlineto{\pgfqpoint{1.797818in}{5.436384in}}%
\pgfpathlineto{\pgfqpoint{1.805669in}{5.443542in}}%
\pgfpathlineto{\pgfqpoint{1.813520in}{5.451329in}}%
\pgfpathlineto{\pgfqpoint{1.821371in}{5.460301in}}%
\pgfpathlineto{\pgfqpoint{1.829221in}{5.469864in}}%
\pgfpathlineto{\pgfqpoint{1.837072in}{5.480395in}}%
\pgfpathlineto{\pgfqpoint{1.844923in}{5.491789in}}%
\pgfpathlineto{\pgfqpoint{1.852774in}{5.504526in}}%
\pgfpathlineto{\pgfqpoint{1.860624in}{5.518379in}}%
\pgfpathlineto{\pgfqpoint{1.868475in}{5.533329in}}%
\pgfpathlineto{\pgfqpoint{1.876326in}{5.549497in}}%
\pgfpathlineto{\pgfqpoint{1.884177in}{5.567269in}}%
\pgfpathlineto{\pgfqpoint{1.892027in}{5.586281in}}%
\pgfpathlineto{\pgfqpoint{1.899878in}{5.606904in}}%
\pgfpathlineto{\pgfqpoint{1.907729in}{5.628836in}}%
\pgfpathlineto{\pgfqpoint{1.915579in}{5.652601in}}%
\pgfpathlineto{\pgfqpoint{1.923430in}{5.677838in}}%
\pgfpathlineto{\pgfqpoint{1.931281in}{5.704858in}}%
\pgfpathlineto{\pgfqpoint{1.939132in}{5.733559in}}%
\pgfpathlineto{\pgfqpoint{1.946982in}{5.764054in}}%
\pgfpathlineto{\pgfqpoint{1.954833in}{5.795914in}}%
\pgfpathlineto{\pgfqpoint{1.962684in}{5.829422in}}%
\pgfpathlineto{\pgfqpoint{1.970535in}{5.864900in}}%
\pgfpathlineto{\pgfqpoint{1.978385in}{5.901610in}}%
\pgfpathlineto{\pgfqpoint{1.994087in}{5.980117in}}%
\pgfpathlineto{\pgfqpoint{2.009788in}{6.063333in}}%
\pgfpathlineto{\pgfqpoint{2.025490in}{6.151498in}}%
\pgfpathlineto{\pgfqpoint{2.041191in}{6.242091in}}%
\pgfpathlineto{\pgfqpoint{2.072594in}{6.425034in}}%
\pgfpathlineto{\pgfqpoint{2.088296in}{6.512929in}}%
\pgfpathlineto{\pgfqpoint{2.096146in}{6.555560in}}%
\pgfpathlineto{\pgfqpoint{2.103997in}{6.596314in}}%
\pgfpathlineto{\pgfqpoint{2.111848in}{6.635581in}}%
\pgfpathlineto{\pgfqpoint{2.119699in}{6.672388in}}%
\pgfpathlineto{\pgfqpoint{2.127549in}{6.706861in}}%
\pgfpathlineto{\pgfqpoint{2.135400in}{6.739388in}}%
\pgfpathlineto{\pgfqpoint{2.143251in}{6.769022in}}%
\pgfpathlineto{\pgfqpoint{2.151102in}{6.794933in}}%
\pgfpathlineto{\pgfqpoint{2.158952in}{6.817337in}}%
\pgfpathlineto{\pgfqpoint{2.166803in}{6.837674in}}%
\pgfpathlineto{\pgfqpoint{2.174654in}{6.853999in}}%
\pgfpathlineto{\pgfqpoint{2.182504in}{6.865700in}}%
\pgfpathlineto{\pgfqpoint{2.190355in}{6.875059in}}%
\pgfpathlineto{\pgfqpoint{2.198206in}{6.878573in}}%
\pgfpathlineto{\pgfqpoint{2.206057in}{6.880260in}}%
\pgfpathlineto{\pgfqpoint{2.213907in}{6.875870in}}%
\pgfpathlineto{\pgfqpoint{2.221758in}{6.868415in}}%
\pgfpathlineto{\pgfqpoint{2.229609in}{6.857707in}}%
\pgfpathlineto{\pgfqpoint{2.237460in}{6.843641in}}%
\pgfpathlineto{\pgfqpoint{2.245310in}{6.824490in}}%
\pgfpathlineto{\pgfqpoint{2.253161in}{6.802617in}}%
\pgfpathlineto{\pgfqpoint{2.261012in}{6.778006in}}%
\pgfpathlineto{\pgfqpoint{2.268863in}{6.749658in}}%
\pgfpathlineto{\pgfqpoint{2.276713in}{6.718591in}}%
\pgfpathlineto{\pgfqpoint{2.284564in}{6.684165in}}%
\pgfpathlineto{\pgfqpoint{2.292415in}{6.647773in}}%
\pgfpathlineto{\pgfqpoint{2.300265in}{6.609671in}}%
\pgfpathlineto{\pgfqpoint{2.308116in}{6.569447in}}%
\pgfpathlineto{\pgfqpoint{2.315967in}{6.527670in}}%
\pgfpathlineto{\pgfqpoint{2.331668in}{6.440442in}}%
\pgfpathlineto{\pgfqpoint{2.378773in}{6.167197in}}%
\pgfpathlineto{\pgfqpoint{2.394474in}{6.078951in}}%
\pgfpathlineto{\pgfqpoint{2.410176in}{5.994323in}}%
\pgfpathlineto{\pgfqpoint{2.425877in}{5.915688in}}%
\pgfpathlineto{\pgfqpoint{2.433728in}{5.878309in}}%
\pgfpathlineto{\pgfqpoint{2.441579in}{5.842273in}}%
\pgfpathlineto{\pgfqpoint{2.449429in}{5.808187in}}%
\pgfpathlineto{\pgfqpoint{2.457280in}{5.775786in}}%
\pgfpathlineto{\pgfqpoint{2.465131in}{5.744756in}}%
\pgfpathlineto{\pgfqpoint{2.472982in}{5.715566in}}%
\pgfpathlineto{\pgfqpoint{2.480832in}{5.687984in}}%
\pgfpathlineto{\pgfqpoint{2.488683in}{5.662191in}}%
\pgfpathlineto{\pgfqpoint{2.496534in}{5.637857in}}%
\pgfpathlineto{\pgfqpoint{2.504385in}{5.615263in}}%
\pgfpathlineto{\pgfqpoint{2.512235in}{5.594263in}}%
\pgfpathlineto{\pgfqpoint{2.520086in}{5.574732in}}%
\pgfpathlineto{\pgfqpoint{2.527937in}{5.556337in}}%
\pgfpathlineto{\pgfqpoint{2.535788in}{5.539540in}}%
\pgfpathlineto{\pgfqpoint{2.543638in}{5.524287in}}%
\pgfpathlineto{\pgfqpoint{2.551489in}{5.509855in}}%
\pgfpathlineto{\pgfqpoint{2.559340in}{5.496775in}}%
\pgfpathlineto{\pgfqpoint{2.567190in}{5.484908in}}%
\pgfpathlineto{\pgfqpoint{2.575041in}{5.473990in}}%
\pgfpathlineto{\pgfqpoint{2.582892in}{5.464179in}}%
\pgfpathlineto{\pgfqpoint{2.590743in}{5.454990in}}%
\pgfpathlineto{\pgfqpoint{2.598593in}{5.446696in}}%
\pgfpathlineto{\pgfqpoint{2.606444in}{5.439358in}}%
\pgfpathlineto{\pgfqpoint{2.614295in}{5.432607in}}%
\pgfpathlineto{\pgfqpoint{2.622146in}{5.426561in}}%
\pgfpathlineto{\pgfqpoint{2.629996in}{5.421009in}}%
\pgfpathlineto{\pgfqpoint{2.637847in}{5.416077in}}%
\pgfpathlineto{\pgfqpoint{2.645698in}{5.411717in}}%
\pgfpathlineto{\pgfqpoint{2.653549in}{5.407625in}}%
\pgfpathlineto{\pgfqpoint{2.661399in}{5.404113in}}%
\pgfpathlineto{\pgfqpoint{2.677101in}{5.398017in}}%
\pgfpathlineto{\pgfqpoint{2.684952in}{5.395453in}}%
\pgfpathlineto{\pgfqpoint{2.692802in}{5.393177in}}%
\pgfpathlineto{\pgfqpoint{2.708504in}{5.389235in}}%
\pgfpathlineto{\pgfqpoint{2.724205in}{5.386151in}}%
\pgfpathlineto{\pgfqpoint{2.739907in}{5.383681in}}%
\pgfpathlineto{\pgfqpoint{2.755608in}{5.381741in}}%
\pgfpathlineto{\pgfqpoint{2.779160in}{5.379493in}}%
\pgfpathlineto{\pgfqpoint{2.802713in}{5.377933in}}%
\pgfpathlineto{\pgfqpoint{2.826265in}{5.376767in}}%
\pgfpathlineto{\pgfqpoint{2.857668in}{5.375750in}}%
\pgfpathlineto{\pgfqpoint{2.896921in}{5.374941in}}%
\pgfpathlineto{\pgfqpoint{2.951877in}{5.374350in}}%
\pgfpathlineto{\pgfqpoint{3.022533in}{5.374038in}}%
\pgfpathlineto{\pgfqpoint{3.203100in}{5.373880in}}%
\pgfpathlineto{\pgfqpoint{3.799756in}{5.373866in}}%
\pgfpathlineto{\pgfqpoint{6.133424in}{5.373866in}}%
\pgfpathlineto{\pgfqpoint{6.133424in}{5.373866in}}%
\pgfusepath{stroke}%
\end{pgfscope}%
\begin{pgfscope}%
\pgfpathrectangle{\pgfqpoint{0.557402in}{5.298088in}}{\pgfqpoint{5.574022in}{1.657514in}}%
\pgfusepath{clip}%
\pgfsetrectcap%
\pgfsetroundjoin%
\pgfsetlinewidth{1.505625pt}%
\definecolor{currentstroke}{rgb}{1.000000,0.498039,0.054902}%
\pgfsetstrokecolor{currentstroke}%
\pgfsetdash{}{0pt}%
\pgfpathmoveto{\pgfqpoint{0.635910in}{5.373867in}}%
\pgfpathlineto{\pgfqpoint{1.020596in}{5.374079in}}%
\pgfpathlineto{\pgfqpoint{1.083401in}{5.374502in}}%
\pgfpathlineto{\pgfqpoint{1.130506in}{5.375288in}}%
\pgfpathlineto{\pgfqpoint{1.161909in}{5.376211in}}%
\pgfpathlineto{\pgfqpoint{1.193312in}{5.377721in}}%
\pgfpathlineto{\pgfqpoint{1.216864in}{5.379375in}}%
\pgfpathlineto{\pgfqpoint{1.232565in}{5.380762in}}%
\pgfpathlineto{\pgfqpoint{1.256118in}{5.383527in}}%
\pgfpathlineto{\pgfqpoint{1.271819in}{5.385832in}}%
\pgfpathlineto{\pgfqpoint{1.287521in}{5.388561in}}%
\pgfpathlineto{\pgfqpoint{1.303222in}{5.391925in}}%
\pgfpathlineto{\pgfqpoint{1.318924in}{5.395789in}}%
\pgfpathlineto{\pgfqpoint{1.334625in}{5.400264in}}%
\pgfpathlineto{\pgfqpoint{1.350327in}{5.405616in}}%
\pgfpathlineto{\pgfqpoint{1.358177in}{5.408503in}}%
\pgfpathlineto{\pgfqpoint{1.366028in}{5.411629in}}%
\pgfpathlineto{\pgfqpoint{1.381729in}{5.418629in}}%
\pgfpathlineto{\pgfqpoint{1.389580in}{5.422395in}}%
\pgfpathlineto{\pgfqpoint{1.405282in}{5.430777in}}%
\pgfpathlineto{\pgfqpoint{1.413132in}{5.435340in}}%
\pgfpathlineto{\pgfqpoint{1.420983in}{5.440177in}}%
\pgfpathlineto{\pgfqpoint{1.428834in}{5.445287in}}%
\pgfpathlineto{\pgfqpoint{1.436685in}{5.450665in}}%
\pgfpathlineto{\pgfqpoint{1.452386in}{5.462273in}}%
\pgfpathlineto{\pgfqpoint{1.468088in}{5.474848in}}%
\pgfpathlineto{\pgfqpoint{1.475938in}{5.481629in}}%
\pgfpathlineto{\pgfqpoint{1.499490in}{5.503636in}}%
\pgfpathlineto{\pgfqpoint{1.507341in}{5.511626in}}%
\pgfpathlineto{\pgfqpoint{1.523043in}{5.528156in}}%
\pgfpathlineto{\pgfqpoint{1.530893in}{5.537191in}}%
\pgfpathlineto{\pgfqpoint{1.538744in}{5.545859in}}%
\pgfpathlineto{\pgfqpoint{1.546595in}{5.555044in}}%
\pgfpathlineto{\pgfqpoint{1.577998in}{5.594348in}}%
\pgfpathlineto{\pgfqpoint{1.601550in}{5.626295in}}%
\pgfpathlineto{\pgfqpoint{1.625102in}{5.659396in}}%
\pgfpathlineto{\pgfqpoint{1.632953in}{5.671195in}}%
\pgfpathlineto{\pgfqpoint{1.648654in}{5.694097in}}%
\pgfpathlineto{\pgfqpoint{1.687908in}{5.754320in}}%
\pgfpathlineto{\pgfqpoint{1.695759in}{5.766571in}}%
\pgfpathlineto{\pgfqpoint{1.703610in}{5.778270in}}%
\pgfpathlineto{\pgfqpoint{1.711460in}{5.790728in}}%
\pgfpathlineto{\pgfqpoint{1.719311in}{5.802692in}}%
\pgfpathlineto{\pgfqpoint{1.735013in}{5.827806in}}%
\pgfpathlineto{\pgfqpoint{1.742863in}{5.839812in}}%
\pgfpathlineto{\pgfqpoint{1.766415in}{5.877028in}}%
\pgfpathlineto{\pgfqpoint{1.782117in}{5.901158in}}%
\pgfpathlineto{\pgfqpoint{1.789968in}{5.914080in}}%
\pgfpathlineto{\pgfqpoint{1.797818in}{5.926202in}}%
\pgfpathlineto{\pgfqpoint{1.805669in}{5.938868in}}%
\pgfpathlineto{\pgfqpoint{1.813520in}{5.950341in}}%
\pgfpathlineto{\pgfqpoint{1.821371in}{5.962981in}}%
\pgfpathlineto{\pgfqpoint{1.829221in}{5.974832in}}%
\pgfpathlineto{\pgfqpoint{1.837072in}{5.987119in}}%
\pgfpathlineto{\pgfqpoint{1.876326in}{6.046461in}}%
\pgfpathlineto{\pgfqpoint{1.884177in}{6.058543in}}%
\pgfpathlineto{\pgfqpoint{1.923430in}{6.116016in}}%
\pgfpathlineto{\pgfqpoint{1.931281in}{6.127086in}}%
\pgfpathlineto{\pgfqpoint{1.939132in}{6.138646in}}%
\pgfpathlineto{\pgfqpoint{1.946982in}{6.149529in}}%
\pgfpathlineto{\pgfqpoint{1.962684in}{6.170387in}}%
\pgfpathlineto{\pgfqpoint{1.970535in}{6.181304in}}%
\pgfpathlineto{\pgfqpoint{1.978385in}{6.190751in}}%
\pgfpathlineto{\pgfqpoint{1.986236in}{6.201267in}}%
\pgfpathlineto{\pgfqpoint{2.009788in}{6.228049in}}%
\pgfpathlineto{\pgfqpoint{2.025490in}{6.244653in}}%
\pgfpathlineto{\pgfqpoint{2.033340in}{6.252278in}}%
\pgfpathlineto{\pgfqpoint{2.041191in}{6.259330in}}%
\pgfpathlineto{\pgfqpoint{2.049042in}{6.266000in}}%
\pgfpathlineto{\pgfqpoint{2.056893in}{6.271473in}}%
\pgfpathlineto{\pgfqpoint{2.064743in}{6.276650in}}%
\pgfpathlineto{\pgfqpoint{2.080445in}{6.284158in}}%
\pgfpathlineto{\pgfqpoint{2.096146in}{6.289047in}}%
\pgfpathlineto{\pgfqpoint{2.103997in}{6.290157in}}%
\pgfpathlineto{\pgfqpoint{2.111848in}{6.290187in}}%
\pgfpathlineto{\pgfqpoint{2.119699in}{6.289495in}}%
\pgfpathlineto{\pgfqpoint{2.127549in}{6.287568in}}%
\pgfpathlineto{\pgfqpoint{2.135400in}{6.285200in}}%
\pgfpathlineto{\pgfqpoint{2.143251in}{6.281236in}}%
\pgfpathlineto{\pgfqpoint{2.151102in}{6.276293in}}%
\pgfpathlineto{\pgfqpoint{2.158952in}{6.270185in}}%
\pgfpathlineto{\pgfqpoint{2.166803in}{6.263517in}}%
\pgfpathlineto{\pgfqpoint{2.174654in}{6.255151in}}%
\pgfpathlineto{\pgfqpoint{2.182504in}{6.246254in}}%
\pgfpathlineto{\pgfqpoint{2.190355in}{6.236160in}}%
\pgfpathlineto{\pgfqpoint{2.198206in}{6.223981in}}%
\pgfpathlineto{\pgfqpoint{2.206057in}{6.212335in}}%
\pgfpathlineto{\pgfqpoint{2.213907in}{6.198734in}}%
\pgfpathlineto{\pgfqpoint{2.229609in}{6.168462in}}%
\pgfpathlineto{\pgfqpoint{2.237460in}{6.152252in}}%
\pgfpathlineto{\pgfqpoint{2.253161in}{6.117146in}}%
\pgfpathlineto{\pgfqpoint{2.261012in}{6.098650in}}%
\pgfpathlineto{\pgfqpoint{2.276713in}{6.059235in}}%
\pgfpathlineto{\pgfqpoint{2.284564in}{6.038288in}}%
\pgfpathlineto{\pgfqpoint{2.292415in}{6.016627in}}%
\pgfpathlineto{\pgfqpoint{2.300265in}{5.995982in}}%
\pgfpathlineto{\pgfqpoint{2.331668in}{5.907994in}}%
\pgfpathlineto{\pgfqpoint{2.339519in}{5.885431in}}%
\pgfpathlineto{\pgfqpoint{2.378773in}{5.776530in}}%
\pgfpathlineto{\pgfqpoint{2.394474in}{5.735250in}}%
\pgfpathlineto{\pgfqpoint{2.410176in}{5.696309in}}%
\pgfpathlineto{\pgfqpoint{2.418027in}{5.677356in}}%
\pgfpathlineto{\pgfqpoint{2.425877in}{5.659152in}}%
\pgfpathlineto{\pgfqpoint{2.433728in}{5.641697in}}%
\pgfpathlineto{\pgfqpoint{2.449429in}{5.608528in}}%
\pgfpathlineto{\pgfqpoint{2.457280in}{5.593208in}}%
\pgfpathlineto{\pgfqpoint{2.465131in}{5.578387in}}%
\pgfpathlineto{\pgfqpoint{2.472982in}{5.564111in}}%
\pgfpathlineto{\pgfqpoint{2.480832in}{5.550488in}}%
\pgfpathlineto{\pgfqpoint{2.488683in}{5.537934in}}%
\pgfpathlineto{\pgfqpoint{2.496534in}{5.526004in}}%
\pgfpathlineto{\pgfqpoint{2.504385in}{5.514551in}}%
\pgfpathlineto{\pgfqpoint{2.512235in}{5.503871in}}%
\pgfpathlineto{\pgfqpoint{2.520086in}{5.493743in}}%
\pgfpathlineto{\pgfqpoint{2.527937in}{5.484389in}}%
\pgfpathlineto{\pgfqpoint{2.535788in}{5.475613in}}%
\pgfpathlineto{\pgfqpoint{2.543638in}{5.467253in}}%
\pgfpathlineto{\pgfqpoint{2.551489in}{5.459663in}}%
\pgfpathlineto{\pgfqpoint{2.559340in}{5.452552in}}%
\pgfpathlineto{\pgfqpoint{2.567190in}{5.445889in}}%
\pgfpathlineto{\pgfqpoint{2.575041in}{5.439775in}}%
\pgfpathlineto{\pgfqpoint{2.582892in}{5.434128in}}%
\pgfpathlineto{\pgfqpoint{2.590743in}{5.428890in}}%
\pgfpathlineto{\pgfqpoint{2.598593in}{5.424093in}}%
\pgfpathlineto{\pgfqpoint{2.606444in}{5.419638in}}%
\pgfpathlineto{\pgfqpoint{2.614295in}{5.415568in}}%
\pgfpathlineto{\pgfqpoint{2.629996in}{5.408362in}}%
\pgfpathlineto{\pgfqpoint{2.637847in}{5.405275in}}%
\pgfpathlineto{\pgfqpoint{2.653549in}{5.399903in}}%
\pgfpathlineto{\pgfqpoint{2.661399in}{5.397462in}}%
\pgfpathlineto{\pgfqpoint{2.677101in}{5.393391in}}%
\pgfpathlineto{\pgfqpoint{2.692802in}{5.389923in}}%
\pgfpathlineto{\pgfqpoint{2.716354in}{5.385988in}}%
\pgfpathlineto{\pgfqpoint{2.732056in}{5.383954in}}%
\pgfpathlineto{\pgfqpoint{2.747757in}{5.382196in}}%
\pgfpathlineto{\pgfqpoint{2.771310in}{5.380238in}}%
\pgfpathlineto{\pgfqpoint{2.787011in}{5.379214in}}%
\pgfpathlineto{\pgfqpoint{2.818414in}{5.377703in}}%
\pgfpathlineto{\pgfqpoint{2.857668in}{5.376485in}}%
\pgfpathlineto{\pgfqpoint{2.904772in}{5.375647in}}%
\pgfpathlineto{\pgfqpoint{2.959727in}{5.375114in}}%
\pgfpathlineto{\pgfqpoint{3.242354in}{5.374353in}}%
\pgfpathlineto{\pgfqpoint{4.529874in}{5.373867in}}%
\pgfpathlineto{\pgfqpoint{6.133424in}{5.373866in}}%
\pgfpathlineto{\pgfqpoint{6.133424in}{5.373866in}}%
\pgfusepath{stroke}%
\end{pgfscope}%
\begin{pgfscope}%
\pgfsetrectcap%
\pgfsetmiterjoin%
\pgfsetlinewidth{0.803000pt}%
\definecolor{currentstroke}{rgb}{0.000000,0.000000,0.000000}%
\pgfsetstrokecolor{currentstroke}%
\pgfsetdash{}{0pt}%
\pgfpathmoveto{\pgfqpoint{0.557402in}{5.298088in}}%
\pgfpathlineto{\pgfqpoint{0.557402in}{6.955602in}}%
\pgfusepath{stroke}%
\end{pgfscope}%
\begin{pgfscope}%
\pgfsetrectcap%
\pgfsetmiterjoin%
\pgfsetlinewidth{0.803000pt}%
\definecolor{currentstroke}{rgb}{0.000000,0.000000,0.000000}%
\pgfsetstrokecolor{currentstroke}%
\pgfsetdash{}{0pt}%
\pgfpathmoveto{\pgfqpoint{6.131424in}{5.298088in}}%
\pgfpathlineto{\pgfqpoint{6.131424in}{6.955602in}}%
\pgfusepath{stroke}%
\end{pgfscope}%
\begin{pgfscope}%
\pgfsetrectcap%
\pgfsetmiterjoin%
\pgfsetlinewidth{0.803000pt}%
\definecolor{currentstroke}{rgb}{0.000000,0.000000,0.000000}%
\pgfsetstrokecolor{currentstroke}%
\pgfsetdash{}{0pt}%
\pgfpathmoveto{\pgfqpoint{0.557402in}{5.298088in}}%
\pgfpathlineto{\pgfqpoint{6.131424in}{5.298088in}}%
\pgfusepath{stroke}%
\end{pgfscope}%
\begin{pgfscope}%
\pgfsetrectcap%
\pgfsetmiterjoin%
\pgfsetlinewidth{0.803000pt}%
\definecolor{currentstroke}{rgb}{0.000000,0.000000,0.000000}%
\pgfsetstrokecolor{currentstroke}%
\pgfsetdash{}{0pt}%
\pgfpathmoveto{\pgfqpoint{0.557402in}{6.955602in}}%
\pgfpathlineto{\pgfqpoint{6.131424in}{6.955602in}}%
\pgfusepath{stroke}%
\end{pgfscope}%
\begin{pgfscope}%
\definecolor{textcolor}{rgb}{0.000000,0.000000,0.000000}%
\pgfsetstrokecolor{textcolor}%
\pgfsetfillcolor{textcolor}%
\pgftext[x=0.000000in,y=7.121353in,left,base]{\color{textcolor}\rmfamily\fontsize{10.000000}{12.000000}\selectfont (a)}%
\end{pgfscope}%
\begin{pgfscope}%
\pgfpathrectangle{\pgfqpoint{0.557402in}{5.298088in}}{\pgfqpoint{5.574022in}{1.657514in}}%
\pgfusepath{clip}%
\pgfsetbuttcap%
\pgfsetmiterjoin%
\pgfsetlinewidth{1.003750pt}%
\definecolor{currentstroke}{rgb}{0.000000,0.000000,0.000000}%
\pgfsetstrokecolor{currentstroke}%
\pgfsetstrokeopacity{0.500000}%
\pgfsetdash{}{0pt}%
\pgfpathmoveto{\pgfqpoint{2.669250in}{5.373430in}}%
\pgfpathlineto{\pgfqpoint{3.532831in}{5.373430in}}%
\pgfpathlineto{\pgfqpoint{3.532831in}{5.383065in}}%
\pgfpathlineto{\pgfqpoint{2.669250in}{5.383065in}}%
\pgfpathlineto{\pgfqpoint{2.669250in}{5.373430in}}%
\pgfpathclose%
\pgfusepath{stroke}%
\end{pgfscope}%
\begin{pgfscope}%
\pgfsetroundcap%
\pgfsetroundjoin%
\pgfsetlinewidth{1.003750pt}%
\definecolor{currentstroke}{rgb}{0.000000,0.000000,0.000000}%
\pgfsetstrokecolor{currentstroke}%
\pgfsetstrokeopacity{0.500000}%
\pgfsetdash{}{0pt}%
\pgfpathmoveto{\pgfqpoint{2.842751in}{6.657249in}}%
\pgfpathquadraticcurveto{\pgfqpoint{2.756001in}{6.020157in}}{\pgfqpoint{2.669250in}{5.383065in}}%
\pgfusepath{stroke}%
\end{pgfscope}%
\begin{pgfscope}%
\pgfsetroundcap%
\pgfsetroundjoin%
\pgfsetlinewidth{1.003750pt}%
\definecolor{currentstroke}{rgb}{0.000000,0.000000,0.000000}%
\pgfsetstrokecolor{currentstroke}%
\pgfsetstrokeopacity{0.500000}%
\pgfsetdash{}{0pt}%
\pgfpathmoveto{\pgfqpoint{5.462542in}{5.712466in}}%
\pgfpathquadraticcurveto{\pgfqpoint{4.497686in}{5.542948in}}{\pgfqpoint{3.532831in}{5.373430in}}%
\pgfusepath{stroke}%
\end{pgfscope}%
\begin{pgfscope}%
\pgfsetbuttcap%
\pgfsetmiterjoin%
\definecolor{currentfill}{rgb}{1.000000,1.000000,1.000000}%
\pgfsetfillcolor{currentfill}%
\pgfsetlinewidth{0.000000pt}%
\definecolor{currentstroke}{rgb}{0.000000,0.000000,0.000000}%
\pgfsetstrokecolor{currentstroke}%
\pgfsetstrokeopacity{0.000000}%
\pgfsetdash{}{0pt}%
\pgfpathmoveto{\pgfqpoint{2.842751in}{5.712466in}}%
\pgfpathlineto{\pgfqpoint{5.462542in}{5.712466in}}%
\pgfpathlineto{\pgfqpoint{5.462542in}{6.657249in}}%
\pgfpathlineto{\pgfqpoint{2.842751in}{6.657249in}}%
\pgfpathlineto{\pgfqpoint{2.842751in}{5.712466in}}%
\pgfpathclose%
\pgfusepath{fill}%
\end{pgfscope}%
\begin{pgfscope}%
\pgfsetbuttcap%
\pgfsetroundjoin%
\definecolor{currentfill}{rgb}{0.000000,0.000000,0.000000}%
\pgfsetfillcolor{currentfill}%
\pgfsetlinewidth{0.803000pt}%
\definecolor{currentstroke}{rgb}{0.000000,0.000000,0.000000}%
\pgfsetstrokecolor{currentstroke}%
\pgfsetdash{}{0pt}%
\pgfsys@defobject{currentmarker}{\pgfqpoint{0.000000in}{0.000000in}}{\pgfqpoint{0.000000in}{0.048611in}}{%
\pgfpathmoveto{\pgfqpoint{0.000000in}{0.000000in}}%
\pgfpathlineto{\pgfqpoint{0.000000in}{0.048611in}}%
\pgfusepath{stroke,fill}%
}%
\begin{pgfscope}%
\pgfsys@transformshift{3.152363in}{6.657249in}%
\pgfsys@useobject{currentmarker}{}%
\end{pgfscope}%
\end{pgfscope}%
\begin{pgfscope}%
\definecolor{textcolor}{rgb}{0.000000,0.000000,0.000000}%
\pgfsetstrokecolor{textcolor}%
\pgfsetfillcolor{textcolor}%
\pgftext[x=3.152363in,y=6.754471in,,bottom]{\color{textcolor}\rmfamily\fontsize{10.000000}{12.000000}\selectfont \(\displaystyle {72}\)}%
\end{pgfscope}%
\begin{pgfscope}%
\pgfsetbuttcap%
\pgfsetroundjoin%
\definecolor{currentfill}{rgb}{0.000000,0.000000,0.000000}%
\pgfsetfillcolor{currentfill}%
\pgfsetlinewidth{0.803000pt}%
\definecolor{currentstroke}{rgb}{0.000000,0.000000,0.000000}%
\pgfsetstrokecolor{currentstroke}%
\pgfsetdash{}{0pt}%
\pgfsys@defobject{currentmarker}{\pgfqpoint{0.000000in}{0.000000in}}{\pgfqpoint{0.000000in}{0.048611in}}{%
\pgfpathmoveto{\pgfqpoint{0.000000in}{0.000000in}}%
\pgfpathlineto{\pgfqpoint{0.000000in}{0.048611in}}%
\pgfusepath{stroke,fill}%
}%
\begin{pgfscope}%
\pgfsys@transformshift{3.819219in}{6.657249in}%
\pgfsys@useobject{currentmarker}{}%
\end{pgfscope}%
\end{pgfscope}%
\begin{pgfscope}%
\definecolor{textcolor}{rgb}{0.000000,0.000000,0.000000}%
\pgfsetstrokecolor{textcolor}%
\pgfsetfillcolor{textcolor}%
\pgftext[x=3.819219in,y=6.754471in,,bottom]{\color{textcolor}\rmfamily\fontsize{10.000000}{12.000000}\selectfont \(\displaystyle {100}\)}%
\end{pgfscope}%
\begin{pgfscope}%
\pgfsetbuttcap%
\pgfsetroundjoin%
\definecolor{currentfill}{rgb}{0.000000,0.000000,0.000000}%
\pgfsetfillcolor{currentfill}%
\pgfsetlinewidth{0.803000pt}%
\definecolor{currentstroke}{rgb}{0.000000,0.000000,0.000000}%
\pgfsetstrokecolor{currentstroke}%
\pgfsetdash{}{0pt}%
\pgfsys@defobject{currentmarker}{\pgfqpoint{0.000000in}{0.000000in}}{\pgfqpoint{0.000000in}{0.048611in}}{%
\pgfpathmoveto{\pgfqpoint{0.000000in}{0.000000in}}%
\pgfpathlineto{\pgfqpoint{0.000000in}{0.048611in}}%
\pgfusepath{stroke,fill}%
}%
\begin{pgfscope}%
\pgfsys@transformshift{4.414626in}{6.657249in}%
\pgfsys@useobject{currentmarker}{}%
\end{pgfscope}%
\end{pgfscope}%
\begin{pgfscope}%
\definecolor{textcolor}{rgb}{0.000000,0.000000,0.000000}%
\pgfsetstrokecolor{textcolor}%
\pgfsetfillcolor{textcolor}%
\pgftext[x=4.414626in,y=6.754471in,,bottom]{\color{textcolor}\rmfamily\fontsize{10.000000}{12.000000}\selectfont \(\displaystyle {125}\)}%
\end{pgfscope}%
\begin{pgfscope}%
\pgfsetbuttcap%
\pgfsetroundjoin%
\definecolor{currentfill}{rgb}{0.000000,0.000000,0.000000}%
\pgfsetfillcolor{currentfill}%
\pgfsetlinewidth{0.803000pt}%
\definecolor{currentstroke}{rgb}{0.000000,0.000000,0.000000}%
\pgfsetstrokecolor{currentstroke}%
\pgfsetdash{}{0pt}%
\pgfsys@defobject{currentmarker}{\pgfqpoint{0.000000in}{0.000000in}}{\pgfqpoint{0.000000in}{0.048611in}}{%
\pgfpathmoveto{\pgfqpoint{0.000000in}{0.000000in}}%
\pgfpathlineto{\pgfqpoint{0.000000in}{0.048611in}}%
\pgfusepath{stroke,fill}%
}%
\begin{pgfscope}%
\pgfsys@transformshift{5.010032in}{6.657249in}%
\pgfsys@useobject{currentmarker}{}%
\end{pgfscope}%
\end{pgfscope}%
\begin{pgfscope}%
\definecolor{textcolor}{rgb}{0.000000,0.000000,0.000000}%
\pgfsetstrokecolor{textcolor}%
\pgfsetfillcolor{textcolor}%
\pgftext[x=5.010032in,y=6.754471in,,bottom]{\color{textcolor}\rmfamily\fontsize{10.000000}{12.000000}\selectfont \(\displaystyle {150}\)}%
\end{pgfscope}%
\begin{pgfscope}%
\pgfsetbuttcap%
\pgfsetroundjoin%
\definecolor{currentfill}{rgb}{0.000000,0.000000,0.000000}%
\pgfsetfillcolor{currentfill}%
\pgfsetlinewidth{0.803000pt}%
\definecolor{currentstroke}{rgb}{0.000000,0.000000,0.000000}%
\pgfsetstrokecolor{currentstroke}%
\pgfsetdash{}{0pt}%
\pgfsys@defobject{currentmarker}{\pgfqpoint{0.000000in}{0.000000in}}{\pgfqpoint{0.048611in}{0.000000in}}{%
\pgfpathmoveto{\pgfqpoint{0.000000in}{0.000000in}}%
\pgfpathlineto{\pgfqpoint{0.048611in}{0.000000in}}%
\pgfusepath{stroke,fill}%
}%
\begin{pgfscope}%
\pgfsys@transformshift{5.462542in}{5.755272in}%
\pgfsys@useobject{currentmarker}{}%
\end{pgfscope}%
\end{pgfscope}%
\begin{pgfscope}%
\definecolor{textcolor}{rgb}{0.000000,0.000000,0.000000}%
\pgfsetstrokecolor{textcolor}%
\pgfsetfillcolor{textcolor}%
\pgftext[x=5.559764in, y=5.707445in, left, base]{\color{textcolor}\rmfamily\fontsize{10.000000}{12.000000}\selectfont \(\displaystyle {0}\)}%
\end{pgfscope}%
\begin{pgfscope}%
\pgfsetbuttcap%
\pgfsetroundjoin%
\definecolor{currentfill}{rgb}{0.000000,0.000000,0.000000}%
\pgfsetfillcolor{currentfill}%
\pgfsetlinewidth{0.803000pt}%
\definecolor{currentstroke}{rgb}{0.000000,0.000000,0.000000}%
\pgfsetstrokecolor{currentstroke}%
\pgfsetdash{}{0pt}%
\pgfsys@defobject{currentmarker}{\pgfqpoint{0.000000in}{0.000000in}}{\pgfqpoint{0.048611in}{0.000000in}}{%
\pgfpathmoveto{\pgfqpoint{0.000000in}{0.000000in}}%
\pgfpathlineto{\pgfqpoint{0.048611in}{0.000000in}}%
\pgfusepath{stroke,fill}%
}%
\begin{pgfscope}%
\pgfsys@transformshift{5.462542in}{6.217722in}%
\pgfsys@useobject{currentmarker}{}%
\end{pgfscope}%
\end{pgfscope}%
\begin{pgfscope}%
\definecolor{textcolor}{rgb}{0.000000,0.000000,0.000000}%
\pgfsetstrokecolor{textcolor}%
\pgfsetfillcolor{textcolor}%
\pgftext[x=5.559764in, y=6.169894in, left, base]{\color{textcolor}\rmfamily\fontsize{10.000000}{12.000000}\selectfont \(\displaystyle {100000}\)}%
\end{pgfscope}%
\begin{pgfscope}%
\pgfpathrectangle{\pgfqpoint{2.842751in}{5.712466in}}{\pgfqpoint{2.619790in}{0.944783in}}%
\pgfusepath{clip}%
\pgfsetrectcap%
\pgfsetroundjoin%
\pgfsetlinewidth{1.505625pt}%
\definecolor{currentstroke}{rgb}{0.121569,0.466667,0.705882}%
\pgfsetstrokecolor{currentstroke}%
\pgfsetdash{}{0pt}%
\pgfpathmoveto{\pgfqpoint{3.080914in}{6.614305in}}%
\pgfpathlineto{\pgfqpoint{3.104730in}{6.527484in}}%
\pgfpathlineto{\pgfqpoint{3.128547in}{6.446852in}}%
\pgfpathlineto{\pgfqpoint{3.152363in}{6.370668in}}%
\pgfpathlineto{\pgfqpoint{3.176179in}{6.307067in}}%
\pgfpathlineto{\pgfqpoint{3.199995in}{6.249228in}}%
\pgfpathlineto{\pgfqpoint{3.223812in}{6.200172in}}%
\pgfpathlineto{\pgfqpoint{3.247628in}{6.154079in}}%
\pgfpathlineto{\pgfqpoint{3.271444in}{6.111234in}}%
\pgfpathlineto{\pgfqpoint{3.295261in}{6.072231in}}%
\pgfpathlineto{\pgfqpoint{3.319077in}{6.039730in}}%
\pgfpathlineto{\pgfqpoint{3.342893in}{6.010355in}}%
\pgfpathlineto{\pgfqpoint{3.366709in}{5.983648in}}%
\pgfpathlineto{\pgfqpoint{3.390526in}{5.958075in}}%
\pgfpathlineto{\pgfqpoint{3.414342in}{5.939975in}}%
\pgfpathlineto{\pgfqpoint{3.438158in}{5.920043in}}%
\pgfpathlineto{\pgfqpoint{3.461975in}{5.901897in}}%
\pgfpathlineto{\pgfqpoint{3.485791in}{5.886821in}}%
\pgfpathlineto{\pgfqpoint{3.509607in}{5.873923in}}%
\pgfpathlineto{\pgfqpoint{3.533423in}{5.860706in}}%
\pgfpathlineto{\pgfqpoint{3.557240in}{5.849487in}}%
\pgfpathlineto{\pgfqpoint{3.581056in}{5.840289in}}%
\pgfpathlineto{\pgfqpoint{3.604872in}{5.830633in}}%
\pgfpathlineto{\pgfqpoint{3.628688in}{5.823266in}}%
\pgfpathlineto{\pgfqpoint{3.652505in}{5.814757in}}%
\pgfpathlineto{\pgfqpoint{3.676321in}{5.809180in}}%
\pgfpathlineto{\pgfqpoint{3.700137in}{5.802724in}}%
\pgfpathlineto{\pgfqpoint{3.723954in}{5.797101in}}%
\pgfpathlineto{\pgfqpoint{3.747770in}{5.793938in}}%
\pgfpathlineto{\pgfqpoint{3.771586in}{5.788865in}}%
\pgfpathlineto{\pgfqpoint{3.795402in}{5.785406in}}%
\pgfpathlineto{\pgfqpoint{3.819219in}{5.781479in}}%
\pgfpathlineto{\pgfqpoint{3.843035in}{5.779334in}}%
\pgfpathlineto{\pgfqpoint{3.866851in}{5.776013in}}%
\pgfpathlineto{\pgfqpoint{3.890667in}{5.774321in}}%
\pgfpathlineto{\pgfqpoint{3.914484in}{5.772175in}}%
\pgfpathlineto{\pgfqpoint{3.938300in}{5.770607in}}%
\pgfpathlineto{\pgfqpoint{3.962116in}{5.768697in}}%
\pgfpathlineto{\pgfqpoint{3.985933in}{5.767467in}}%
\pgfpathlineto{\pgfqpoint{4.009749in}{5.765775in}}%
\pgfpathlineto{\pgfqpoint{4.033565in}{5.764674in}}%
\pgfpathlineto{\pgfqpoint{4.057381in}{5.763888in}}%
\pgfpathlineto{\pgfqpoint{4.081198in}{5.762500in}}%
\pgfpathlineto{\pgfqpoint{4.105014in}{5.761770in}}%
\pgfpathlineto{\pgfqpoint{4.128830in}{5.760896in}}%
\pgfpathlineto{\pgfqpoint{4.152647in}{5.760512in}}%
\pgfpathlineto{\pgfqpoint{4.176463in}{5.760086in}}%
\pgfpathlineto{\pgfqpoint{4.200279in}{5.759384in}}%
\pgfpathlineto{\pgfqpoint{4.224095in}{5.758912in}}%
\pgfpathlineto{\pgfqpoint{4.247912in}{5.758695in}}%
\pgfpathlineto{\pgfqpoint{4.271728in}{5.758153in}}%
\pgfpathlineto{\pgfqpoint{4.295544in}{5.757709in}}%
\pgfpathlineto{\pgfqpoint{4.319360in}{5.757543in}}%
\pgfpathlineto{\pgfqpoint{4.343177in}{5.757390in}}%
\pgfpathlineto{\pgfqpoint{4.366993in}{5.757252in}}%
\pgfpathlineto{\pgfqpoint{4.390809in}{5.756988in}}%
\pgfpathlineto{\pgfqpoint{4.414626in}{5.756567in}}%
\pgfpathlineto{\pgfqpoint{4.438442in}{5.756613in}}%
\pgfpathlineto{\pgfqpoint{4.462258in}{5.756669in}}%
\pgfpathlineto{\pgfqpoint{4.486074in}{5.756368in}}%
\pgfpathlineto{\pgfqpoint{4.509891in}{5.756304in}}%
\pgfpathlineto{\pgfqpoint{4.533707in}{5.756109in}}%
\pgfpathlineto{\pgfqpoint{4.557523in}{5.756220in}}%
\pgfpathlineto{\pgfqpoint{4.581340in}{5.756146in}}%
\pgfpathlineto{\pgfqpoint{4.605156in}{5.756105in}}%
\pgfpathlineto{\pgfqpoint{4.628972in}{5.755943in}}%
\pgfpathlineto{\pgfqpoint{4.652788in}{5.755901in}}%
\pgfpathlineto{\pgfqpoint{4.676605in}{5.755906in}}%
\pgfpathlineto{\pgfqpoint{4.700421in}{5.755841in}}%
\pgfpathlineto{\pgfqpoint{4.724237in}{5.755730in}}%
\pgfpathlineto{\pgfqpoint{4.748053in}{5.755767in}}%
\pgfpathlineto{\pgfqpoint{4.771870in}{5.755744in}}%
\pgfpathlineto{\pgfqpoint{4.795686in}{5.755726in}}%
\pgfpathlineto{\pgfqpoint{4.819502in}{5.755749in}}%
\pgfpathlineto{\pgfqpoint{4.843319in}{5.755726in}}%
\pgfpathlineto{\pgfqpoint{4.867135in}{5.755638in}}%
\pgfpathlineto{\pgfqpoint{4.890951in}{5.755573in}}%
\pgfpathlineto{\pgfqpoint{4.914767in}{5.755531in}}%
\pgfpathlineto{\pgfqpoint{4.938584in}{5.755652in}}%
\pgfpathlineto{\pgfqpoint{4.962400in}{5.755527in}}%
\pgfpathlineto{\pgfqpoint{4.986216in}{5.755550in}}%
\pgfpathlineto{\pgfqpoint{5.010032in}{5.755554in}}%
\pgfpathlineto{\pgfqpoint{5.033849in}{5.755522in}}%
\pgfpathlineto{\pgfqpoint{5.057665in}{5.755541in}}%
\pgfpathlineto{\pgfqpoint{5.081481in}{5.755527in}}%
\pgfpathlineto{\pgfqpoint{5.105298in}{5.755504in}}%
\pgfpathlineto{\pgfqpoint{5.129114in}{5.755494in}}%
\pgfpathlineto{\pgfqpoint{5.152930in}{5.755545in}}%
\pgfpathlineto{\pgfqpoint{5.176746in}{5.755485in}}%
\pgfpathlineto{\pgfqpoint{5.200563in}{5.755490in}}%
\pgfpathlineto{\pgfqpoint{5.224379in}{5.755531in}}%
\pgfpathlineto{\pgfqpoint{5.248195in}{5.755448in}}%
\pgfpathlineto{\pgfqpoint{5.272012in}{5.755499in}}%
\pgfpathlineto{\pgfqpoint{5.295828in}{5.755439in}}%
\pgfpathlineto{\pgfqpoint{5.319644in}{5.755430in}}%
\pgfpathlineto{\pgfqpoint{5.343460in}{5.755425in}}%
\pgfpathlineto{\pgfqpoint{5.367277in}{5.755420in}}%
\pgfpathlineto{\pgfqpoint{5.391093in}{5.755513in}}%
\pgfpathlineto{\pgfqpoint{5.414909in}{5.755462in}}%
\pgfpathlineto{\pgfqpoint{5.438725in}{5.755411in}}%
\pgfusepath{stroke}%
\end{pgfscope}%
\begin{pgfscope}%
\pgfpathrectangle{\pgfqpoint{2.842751in}{5.712466in}}{\pgfqpoint{2.619790in}{0.944783in}}%
\pgfusepath{clip}%
\pgfsetrectcap%
\pgfsetroundjoin%
\pgfsetlinewidth{1.505625pt}%
\definecolor{currentstroke}{rgb}{1.000000,0.498039,0.054902}%
\pgfsetstrokecolor{currentstroke}%
\pgfsetdash{}{0pt}%
\pgfpathmoveto{\pgfqpoint{3.080914in}{6.572120in}}%
\pgfpathlineto{\pgfqpoint{3.104730in}{6.504810in}}%
\pgfpathlineto{\pgfqpoint{3.128547in}{6.435290in}}%
\pgfpathlineto{\pgfqpoint{3.152363in}{6.380046in}}%
\pgfpathlineto{\pgfqpoint{3.176179in}{6.324820in}}%
\pgfpathlineto{\pgfqpoint{3.199995in}{6.279704in}}%
\pgfpathlineto{\pgfqpoint{3.223812in}{6.238393in}}%
\pgfpathlineto{\pgfqpoint{3.247628in}{6.195487in}}%
\pgfpathlineto{\pgfqpoint{3.271444in}{6.163361in}}%
\pgfpathlineto{\pgfqpoint{3.295261in}{6.131526in}}%
\pgfpathlineto{\pgfqpoint{3.319077in}{6.104898in}}%
\pgfpathlineto{\pgfqpoint{3.342893in}{6.076490in}}%
\pgfpathlineto{\pgfqpoint{3.366709in}{6.051527in}}%
\pgfpathlineto{\pgfqpoint{3.390526in}{6.031133in}}%
\pgfpathlineto{\pgfqpoint{3.414342in}{6.012047in}}%
\pgfpathlineto{\pgfqpoint{3.438158in}{5.995723in}}%
\pgfpathlineto{\pgfqpoint{3.461975in}{5.979542in}}%
\pgfpathlineto{\pgfqpoint{3.485791in}{5.966834in}}%
\pgfpathlineto{\pgfqpoint{3.509607in}{5.955726in}}%
\pgfpathlineto{\pgfqpoint{3.533423in}{5.939438in}}%
\pgfpathlineto{\pgfqpoint{3.557240in}{5.929852in}}%
\pgfpathlineto{\pgfqpoint{3.581056in}{5.919576in}}%
\pgfpathlineto{\pgfqpoint{3.604872in}{5.911557in}}%
\pgfpathlineto{\pgfqpoint{3.628688in}{5.904209in}}%
\pgfpathlineto{\pgfqpoint{3.652505in}{5.894863in}}%
\pgfpathlineto{\pgfqpoint{3.676321in}{5.887760in}}%
\pgfpathlineto{\pgfqpoint{3.700137in}{5.882853in}}%
\pgfpathlineto{\pgfqpoint{3.723954in}{5.877673in}}%
\pgfpathlineto{\pgfqpoint{3.747770in}{5.875019in}}%
\pgfpathlineto{\pgfqpoint{3.771586in}{5.868813in}}%
\pgfpathlineto{\pgfqpoint{3.795402in}{5.863125in}}%
\pgfpathlineto{\pgfqpoint{3.819219in}{5.859319in}}%
\pgfpathlineto{\pgfqpoint{3.843035in}{5.855124in}}%
\pgfpathlineto{\pgfqpoint{3.866851in}{5.854093in}}%
\pgfpathlineto{\pgfqpoint{3.890667in}{5.849543in}}%
\pgfpathlineto{\pgfqpoint{3.914484in}{5.846680in}}%
\pgfpathlineto{\pgfqpoint{3.938300in}{5.845052in}}%
\pgfpathlineto{\pgfqpoint{3.962116in}{5.841686in}}%
\pgfpathlineto{\pgfqpoint{3.985933in}{5.840437in}}%
\pgfpathlineto{\pgfqpoint{4.009749in}{5.838310in}}%
\pgfpathlineto{\pgfqpoint{4.033565in}{5.836483in}}%
\pgfpathlineto{\pgfqpoint{4.057381in}{5.833602in}}%
\pgfpathlineto{\pgfqpoint{4.081198in}{5.831933in}}%
\pgfpathlineto{\pgfqpoint{4.105014in}{5.829930in}}%
\pgfpathlineto{\pgfqpoint{4.128830in}{5.828621in}}%
\pgfpathlineto{\pgfqpoint{4.152647in}{5.827410in}}%
\pgfpathlineto{\pgfqpoint{4.176463in}{5.824644in}}%
\pgfpathlineto{\pgfqpoint{4.200279in}{5.824321in}}%
\pgfpathlineto{\pgfqpoint{4.224095in}{5.822309in}}%
\pgfpathlineto{\pgfqpoint{4.247912in}{5.821023in}}%
\pgfpathlineto{\pgfqpoint{4.271728in}{5.819446in}}%
\pgfpathlineto{\pgfqpoint{4.295544in}{5.819525in}}%
\pgfpathlineto{\pgfqpoint{4.319360in}{5.816565in}}%
\pgfpathlineto{\pgfqpoint{4.343177in}{5.815955in}}%
\pgfpathlineto{\pgfqpoint{4.366993in}{5.814225in}}%
\pgfpathlineto{\pgfqpoint{4.390809in}{5.813162in}}%
\pgfpathlineto{\pgfqpoint{4.414626in}{5.811885in}}%
\pgfpathlineto{\pgfqpoint{4.438442in}{5.810258in}}%
\pgfpathlineto{\pgfqpoint{4.462258in}{5.808954in}}%
\pgfpathlineto{\pgfqpoint{4.486074in}{5.808422in}}%
\pgfpathlineto{\pgfqpoint{4.509891in}{5.807113in}}%
\pgfpathlineto{\pgfqpoint{4.533707in}{5.805263in}}%
\pgfpathlineto{\pgfqpoint{4.557523in}{5.803719in}}%
\pgfpathlineto{\pgfqpoint{4.581340in}{5.803016in}}%
\pgfpathlineto{\pgfqpoint{4.605156in}{5.802595in}}%
\pgfpathlineto{\pgfqpoint{4.628972in}{5.799751in}}%
\pgfpathlineto{\pgfqpoint{4.652788in}{5.799395in}}%
\pgfpathlineto{\pgfqpoint{4.676605in}{5.798696in}}%
\pgfpathlineto{\pgfqpoint{4.700421in}{5.796722in}}%
\pgfpathlineto{\pgfqpoint{4.724237in}{5.795815in}}%
\pgfpathlineto{\pgfqpoint{4.748053in}{5.795547in}}%
\pgfpathlineto{\pgfqpoint{4.771870in}{5.793711in}}%
\pgfpathlineto{\pgfqpoint{4.795686in}{5.791797in}}%
\pgfpathlineto{\pgfqpoint{4.819502in}{5.790654in}}%
\pgfpathlineto{\pgfqpoint{4.843319in}{5.789919in}}%
\pgfpathlineto{\pgfqpoint{4.867135in}{5.788388in}}%
\pgfpathlineto{\pgfqpoint{4.890951in}{5.787112in}}%
\pgfpathlineto{\pgfqpoint{4.914767in}{5.786247in}}%
\pgfpathlineto{\pgfqpoint{4.938584in}{5.785577in}}%
\pgfpathlineto{\pgfqpoint{4.962400in}{5.784504in}}%
\pgfpathlineto{\pgfqpoint{4.986216in}{5.783362in}}%
\pgfpathlineto{\pgfqpoint{5.010032in}{5.782728in}}%
\pgfpathlineto{\pgfqpoint{5.033849in}{5.781586in}}%
\pgfpathlineto{\pgfqpoint{5.057665in}{5.780050in}}%
\pgfpathlineto{\pgfqpoint{5.081481in}{5.778978in}}%
\pgfpathlineto{\pgfqpoint{5.105298in}{5.778723in}}%
\pgfpathlineto{\pgfqpoint{5.129114in}{5.777530in}}%
\pgfpathlineto{\pgfqpoint{5.152930in}{5.776554in}}%
\pgfpathlineto{\pgfqpoint{5.176746in}{5.775468in}}%
\pgfpathlineto{\pgfqpoint{5.200563in}{5.774159in}}%
\pgfpathlineto{\pgfqpoint{5.224379in}{5.773465in}}%
\pgfpathlineto{\pgfqpoint{5.248195in}{5.772628in}}%
\pgfpathlineto{\pgfqpoint{5.272012in}{5.772166in}}%
\pgfpathlineto{\pgfqpoint{5.295828in}{5.771084in}}%
\pgfpathlineto{\pgfqpoint{5.319644in}{5.770630in}}%
\pgfpathlineto{\pgfqpoint{5.343460in}{5.769812in}}%
\pgfpathlineto{\pgfqpoint{5.367277in}{5.768831in}}%
\pgfpathlineto{\pgfqpoint{5.391093in}{5.768438in}}%
\pgfpathlineto{\pgfqpoint{5.414909in}{5.767560in}}%
\pgfpathlineto{\pgfqpoint{5.438725in}{5.766672in}}%
\pgfusepath{stroke}%
\end{pgfscope}%
\begin{pgfscope}%
\pgfpathrectangle{\pgfqpoint{2.842751in}{5.712466in}}{\pgfqpoint{2.619790in}{0.944783in}}%
\pgfusepath{clip}%
\pgfsetrectcap%
\pgfsetroundjoin%
\pgfsetlinewidth{1.003750pt}%
\definecolor{currentstroke}{rgb}{0.000000,0.000000,0.000000}%
\pgfsetstrokecolor{currentstroke}%
\pgfsetdash{}{0pt}%
\pgfpathmoveto{\pgfqpoint{3.152363in}{5.712466in}}%
\pgfpathlineto{\pgfqpoint{3.152363in}{6.657249in}}%
\pgfusepath{stroke}%
\end{pgfscope}%
\begin{pgfscope}%
\pgfsetrectcap%
\pgfsetmiterjoin%
\pgfsetlinewidth{0.803000pt}%
\definecolor{currentstroke}{rgb}{0.000000,0.000000,0.000000}%
\pgfsetstrokecolor{currentstroke}%
\pgfsetdash{}{0pt}%
\pgfpathmoveto{\pgfqpoint{2.842751in}{5.712466in}}%
\pgfpathlineto{\pgfqpoint{2.842751in}{6.657249in}}%
\pgfusepath{stroke}%
\end{pgfscope}%
\begin{pgfscope}%
\pgfsetrectcap%
\pgfsetmiterjoin%
\pgfsetlinewidth{0.803000pt}%
\definecolor{currentstroke}{rgb}{0.000000,0.000000,0.000000}%
\pgfsetstrokecolor{currentstroke}%
\pgfsetdash{}{0pt}%
\pgfpathmoveto{\pgfqpoint{5.462542in}{5.712466in}}%
\pgfpathlineto{\pgfqpoint{5.462542in}{6.657249in}}%
\pgfusepath{stroke}%
\end{pgfscope}%
\begin{pgfscope}%
\pgfsetrectcap%
\pgfsetmiterjoin%
\pgfsetlinewidth{0.803000pt}%
\definecolor{currentstroke}{rgb}{0.000000,0.000000,0.000000}%
\pgfsetstrokecolor{currentstroke}%
\pgfsetdash{}{0pt}%
\pgfpathmoveto{\pgfqpoint{2.842751in}{5.712466in}}%
\pgfpathlineto{\pgfqpoint{5.462542in}{5.712466in}}%
\pgfusepath{stroke}%
\end{pgfscope}%
\begin{pgfscope}%
\pgfsetrectcap%
\pgfsetmiterjoin%
\pgfsetlinewidth{0.803000pt}%
\definecolor{currentstroke}{rgb}{0.000000,0.000000,0.000000}%
\pgfsetstrokecolor{currentstroke}%
\pgfsetdash{}{0pt}%
\pgfpathmoveto{\pgfqpoint{2.842751in}{6.657249in}}%
\pgfpathlineto{\pgfqpoint{5.462542in}{6.657249in}}%
\pgfusepath{stroke}%
\end{pgfscope}%
\begin{pgfscope}%
\pgfsetbuttcap%
\pgfsetmiterjoin%
\definecolor{currentfill}{rgb}{1.000000,1.000000,1.000000}%
\pgfsetfillcolor{currentfill}%
\pgfsetfillopacity{0.800000}%
\pgfsetlinewidth{1.003750pt}%
\definecolor{currentstroke}{rgb}{0.800000,0.800000,0.800000}%
\pgfsetstrokecolor{currentstroke}%
\pgfsetstrokeopacity{0.800000}%
\pgfsetdash{}{0pt}%
\pgfpathmoveto{\pgfqpoint{0.654624in}{6.457158in}}%
\pgfpathlineto{\pgfqpoint{1.876657in}{6.457158in}}%
\pgfpathquadraticcurveto{\pgfqpoint{1.904435in}{6.457158in}}{\pgfqpoint{1.904435in}{6.484936in}}%
\pgfpathlineto{\pgfqpoint{1.904435in}{6.858379in}}%
\pgfpathquadraticcurveto{\pgfqpoint{1.904435in}{6.886157in}}{\pgfqpoint{1.876657in}{6.886157in}}%
\pgfpathlineto{\pgfqpoint{0.654624in}{6.886157in}}%
\pgfpathquadraticcurveto{\pgfqpoint{0.626847in}{6.886157in}}{\pgfqpoint{0.626847in}{6.858379in}}%
\pgfpathlineto{\pgfqpoint{0.626847in}{6.484936in}}%
\pgfpathquadraticcurveto{\pgfqpoint{0.626847in}{6.457158in}}{\pgfqpoint{0.654624in}{6.457158in}}%
\pgfpathlineto{\pgfqpoint{0.654624in}{6.457158in}}%
\pgfpathclose%
\pgfusepath{stroke,fill}%
\end{pgfscope}%
\begin{pgfscope}%
\pgfsetrectcap%
\pgfsetroundjoin%
\pgfsetlinewidth{1.505625pt}%
\definecolor{currentstroke}{rgb}{0.121569,0.466667,0.705882}%
\pgfsetstrokecolor{currentstroke}%
\pgfsetdash{}{0pt}%
\pgfpathmoveto{\pgfqpoint{0.682402in}{6.781991in}}%
\pgfpathlineto{\pgfqpoint{0.821291in}{6.781991in}}%
\pgfpathlineto{\pgfqpoint{0.960180in}{6.781991in}}%
\pgfusepath{stroke}%
\end{pgfscope}%
\begin{pgfscope}%
\definecolor{textcolor}{rgb}{0.000000,0.000000,0.000000}%
\pgfsetstrokecolor{textcolor}%
\pgfsetfillcolor{textcolor}%
\pgftext[x=1.071291in,y=6.733379in,left,base]{\color{textcolor}\rmfamily\fontsize{10.000000}{12.000000}\selectfont Dunkelbilder}%
\end{pgfscope}%
\begin{pgfscope}%
\pgfsetrectcap%
\pgfsetroundjoin%
\pgfsetlinewidth{1.505625pt}%
\definecolor{currentstroke}{rgb}{1.000000,0.498039,0.054902}%
\pgfsetstrokecolor{currentstroke}%
\pgfsetdash{}{0pt}%
\pgfpathmoveto{\pgfqpoint{0.682402in}{6.588324in}}%
\pgfpathlineto{\pgfqpoint{0.821291in}{6.588324in}}%
\pgfpathlineto{\pgfqpoint{0.960180in}{6.588324in}}%
\pgfusepath{stroke}%
\end{pgfscope}%
\begin{pgfscope}%
\definecolor{textcolor}{rgb}{0.000000,0.000000,0.000000}%
\pgfsetstrokecolor{textcolor}%
\pgfsetfillcolor{textcolor}%
\pgftext[x=1.071291in,y=6.539713in,left,base]{\color{textcolor}\rmfamily\fontsize{10.000000}{12.000000}\selectfont Streubilder}%
\end{pgfscope}%
\begin{pgfscope}%
\pgfsetbuttcap%
\pgfsetmiterjoin%
\definecolor{currentfill}{rgb}{1.000000,1.000000,1.000000}%
\pgfsetfillcolor{currentfill}%
\pgfsetlinewidth{0.000000pt}%
\definecolor{currentstroke}{rgb}{0.000000,0.000000,0.000000}%
\pgfsetstrokecolor{currentstroke}%
\pgfsetstrokeopacity{0.000000}%
\pgfsetdash{}{0pt}%
\pgfpathmoveto{\pgfqpoint{0.557402in}{2.848088in}}%
\pgfpathlineto{\pgfqpoint{6.131424in}{2.848088in}}%
\pgfpathlineto{\pgfqpoint{6.131424in}{4.505602in}}%
\pgfpathlineto{\pgfqpoint{0.557402in}{4.505602in}}%
\pgfpathlineto{\pgfqpoint{0.557402in}{2.848088in}}%
\pgfpathclose%
\pgfusepath{fill}%
\end{pgfscope}%
\begin{pgfscope}%
\pgfsetbuttcap%
\pgfsetroundjoin%
\definecolor{currentfill}{rgb}{0.000000,0.000000,0.000000}%
\pgfsetfillcolor{currentfill}%
\pgfsetlinewidth{0.803000pt}%
\definecolor{currentstroke}{rgb}{0.000000,0.000000,0.000000}%
\pgfsetstrokecolor{currentstroke}%
\pgfsetdash{}{0pt}%
\pgfsys@defobject{currentmarker}{\pgfqpoint{0.000000in}{-0.048611in}}{\pgfqpoint{0.000000in}{0.000000in}}{%
\pgfpathmoveto{\pgfqpoint{0.000000in}{0.000000in}}%
\pgfpathlineto{\pgfqpoint{0.000000in}{-0.048611in}}%
\pgfusepath{stroke,fill}%
}%
\begin{pgfscope}%
\pgfsys@transformshift{0.929004in}{2.848088in}%
\pgfsys@useobject{currentmarker}{}%
\end{pgfscope}%
\end{pgfscope}%
\begin{pgfscope}%
\definecolor{textcolor}{rgb}{0.000000,0.000000,0.000000}%
\pgfsetstrokecolor{textcolor}%
\pgfsetfillcolor{textcolor}%
\pgftext[x=0.929004in,y=2.750866in,,top]{\color{textcolor}\rmfamily\fontsize{10.000000}{12.000000}\selectfont \(\displaystyle {0}\)}%
\end{pgfscope}%
\begin{pgfscope}%
\pgfsetbuttcap%
\pgfsetroundjoin%
\definecolor{currentfill}{rgb}{0.000000,0.000000,0.000000}%
\pgfsetfillcolor{currentfill}%
\pgfsetlinewidth{0.803000pt}%
\definecolor{currentstroke}{rgb}{0.000000,0.000000,0.000000}%
\pgfsetstrokecolor{currentstroke}%
\pgfsetdash{}{0pt}%
\pgfsys@defobject{currentmarker}{\pgfqpoint{0.000000in}{-0.048611in}}{\pgfqpoint{0.000000in}{0.000000in}}{%
\pgfpathmoveto{\pgfqpoint{0.000000in}{0.000000in}}%
\pgfpathlineto{\pgfqpoint{0.000000in}{-0.048611in}}%
\pgfusepath{stroke,fill}%
}%
\begin{pgfscope}%
\pgfsys@transformshift{1.672207in}{2.848088in}%
\pgfsys@useobject{currentmarker}{}%
\end{pgfscope}%
\end{pgfscope}%
\begin{pgfscope}%
\definecolor{textcolor}{rgb}{0.000000,0.000000,0.000000}%
\pgfsetstrokecolor{textcolor}%
\pgfsetfillcolor{textcolor}%
\pgftext[x=1.672207in,y=2.750866in,,top]{\color{textcolor}\rmfamily\fontsize{10.000000}{12.000000}\selectfont \(\displaystyle {100}\)}%
\end{pgfscope}%
\begin{pgfscope}%
\pgfsetbuttcap%
\pgfsetroundjoin%
\definecolor{currentfill}{rgb}{0.000000,0.000000,0.000000}%
\pgfsetfillcolor{currentfill}%
\pgfsetlinewidth{0.803000pt}%
\definecolor{currentstroke}{rgb}{0.000000,0.000000,0.000000}%
\pgfsetstrokecolor{currentstroke}%
\pgfsetdash{}{0pt}%
\pgfsys@defobject{currentmarker}{\pgfqpoint{0.000000in}{-0.048611in}}{\pgfqpoint{0.000000in}{0.000000in}}{%
\pgfpathmoveto{\pgfqpoint{0.000000in}{0.000000in}}%
\pgfpathlineto{\pgfqpoint{0.000000in}{-0.048611in}}%
\pgfusepath{stroke,fill}%
}%
\begin{pgfscope}%
\pgfsys@transformshift{2.415410in}{2.848088in}%
\pgfsys@useobject{currentmarker}{}%
\end{pgfscope}%
\end{pgfscope}%
\begin{pgfscope}%
\definecolor{textcolor}{rgb}{0.000000,0.000000,0.000000}%
\pgfsetstrokecolor{textcolor}%
\pgfsetfillcolor{textcolor}%
\pgftext[x=2.415410in,y=2.750866in,,top]{\color{textcolor}\rmfamily\fontsize{10.000000}{12.000000}\selectfont \(\displaystyle {200}\)}%
\end{pgfscope}%
\begin{pgfscope}%
\pgfsetbuttcap%
\pgfsetroundjoin%
\definecolor{currentfill}{rgb}{0.000000,0.000000,0.000000}%
\pgfsetfillcolor{currentfill}%
\pgfsetlinewidth{0.803000pt}%
\definecolor{currentstroke}{rgb}{0.000000,0.000000,0.000000}%
\pgfsetstrokecolor{currentstroke}%
\pgfsetdash{}{0pt}%
\pgfsys@defobject{currentmarker}{\pgfqpoint{0.000000in}{-0.048611in}}{\pgfqpoint{0.000000in}{0.000000in}}{%
\pgfpathmoveto{\pgfqpoint{0.000000in}{0.000000in}}%
\pgfpathlineto{\pgfqpoint{0.000000in}{-0.048611in}}%
\pgfusepath{stroke,fill}%
}%
\begin{pgfscope}%
\pgfsys@transformshift{3.158613in}{2.848088in}%
\pgfsys@useobject{currentmarker}{}%
\end{pgfscope}%
\end{pgfscope}%
\begin{pgfscope}%
\definecolor{textcolor}{rgb}{0.000000,0.000000,0.000000}%
\pgfsetstrokecolor{textcolor}%
\pgfsetfillcolor{textcolor}%
\pgftext[x=3.158613in,y=2.750866in,,top]{\color{textcolor}\rmfamily\fontsize{10.000000}{12.000000}\selectfont \(\displaystyle {300}\)}%
\end{pgfscope}%
\begin{pgfscope}%
\pgfsetbuttcap%
\pgfsetroundjoin%
\definecolor{currentfill}{rgb}{0.000000,0.000000,0.000000}%
\pgfsetfillcolor{currentfill}%
\pgfsetlinewidth{0.803000pt}%
\definecolor{currentstroke}{rgb}{0.000000,0.000000,0.000000}%
\pgfsetstrokecolor{currentstroke}%
\pgfsetdash{}{0pt}%
\pgfsys@defobject{currentmarker}{\pgfqpoint{0.000000in}{-0.048611in}}{\pgfqpoint{0.000000in}{0.000000in}}{%
\pgfpathmoveto{\pgfqpoint{0.000000in}{0.000000in}}%
\pgfpathlineto{\pgfqpoint{0.000000in}{-0.048611in}}%
\pgfusepath{stroke,fill}%
}%
\begin{pgfscope}%
\pgfsys@transformshift{3.901816in}{2.848088in}%
\pgfsys@useobject{currentmarker}{}%
\end{pgfscope}%
\end{pgfscope}%
\begin{pgfscope}%
\definecolor{textcolor}{rgb}{0.000000,0.000000,0.000000}%
\pgfsetstrokecolor{textcolor}%
\pgfsetfillcolor{textcolor}%
\pgftext[x=3.901816in,y=2.750866in,,top]{\color{textcolor}\rmfamily\fontsize{10.000000}{12.000000}\selectfont \(\displaystyle {400}\)}%
\end{pgfscope}%
\begin{pgfscope}%
\pgfsetbuttcap%
\pgfsetroundjoin%
\definecolor{currentfill}{rgb}{0.000000,0.000000,0.000000}%
\pgfsetfillcolor{currentfill}%
\pgfsetlinewidth{0.803000pt}%
\definecolor{currentstroke}{rgb}{0.000000,0.000000,0.000000}%
\pgfsetstrokecolor{currentstroke}%
\pgfsetdash{}{0pt}%
\pgfsys@defobject{currentmarker}{\pgfqpoint{0.000000in}{-0.048611in}}{\pgfqpoint{0.000000in}{0.000000in}}{%
\pgfpathmoveto{\pgfqpoint{0.000000in}{0.000000in}}%
\pgfpathlineto{\pgfqpoint{0.000000in}{-0.048611in}}%
\pgfusepath{stroke,fill}%
}%
\begin{pgfscope}%
\pgfsys@transformshift{4.645018in}{2.848088in}%
\pgfsys@useobject{currentmarker}{}%
\end{pgfscope}%
\end{pgfscope}%
\begin{pgfscope}%
\definecolor{textcolor}{rgb}{0.000000,0.000000,0.000000}%
\pgfsetstrokecolor{textcolor}%
\pgfsetfillcolor{textcolor}%
\pgftext[x=4.645018in,y=2.750866in,,top]{\color{textcolor}\rmfamily\fontsize{10.000000}{12.000000}\selectfont \(\displaystyle {500}\)}%
\end{pgfscope}%
\begin{pgfscope}%
\pgfsetbuttcap%
\pgfsetroundjoin%
\definecolor{currentfill}{rgb}{0.000000,0.000000,0.000000}%
\pgfsetfillcolor{currentfill}%
\pgfsetlinewidth{0.803000pt}%
\definecolor{currentstroke}{rgb}{0.000000,0.000000,0.000000}%
\pgfsetstrokecolor{currentstroke}%
\pgfsetdash{}{0pt}%
\pgfsys@defobject{currentmarker}{\pgfqpoint{0.000000in}{-0.048611in}}{\pgfqpoint{0.000000in}{0.000000in}}{%
\pgfpathmoveto{\pgfqpoint{0.000000in}{0.000000in}}%
\pgfpathlineto{\pgfqpoint{0.000000in}{-0.048611in}}%
\pgfusepath{stroke,fill}%
}%
\begin{pgfscope}%
\pgfsys@transformshift{5.388221in}{2.848088in}%
\pgfsys@useobject{currentmarker}{}%
\end{pgfscope}%
\end{pgfscope}%
\begin{pgfscope}%
\definecolor{textcolor}{rgb}{0.000000,0.000000,0.000000}%
\pgfsetstrokecolor{textcolor}%
\pgfsetfillcolor{textcolor}%
\pgftext[x=5.388221in,y=2.750866in,,top]{\color{textcolor}\rmfamily\fontsize{10.000000}{12.000000}\selectfont \(\displaystyle {600}\)}%
\end{pgfscope}%
\begin{pgfscope}%
\pgfsetbuttcap%
\pgfsetroundjoin%
\definecolor{currentfill}{rgb}{0.000000,0.000000,0.000000}%
\pgfsetfillcolor{currentfill}%
\pgfsetlinewidth{0.803000pt}%
\definecolor{currentstroke}{rgb}{0.000000,0.000000,0.000000}%
\pgfsetstrokecolor{currentstroke}%
\pgfsetdash{}{0pt}%
\pgfsys@defobject{currentmarker}{\pgfqpoint{0.000000in}{-0.048611in}}{\pgfqpoint{0.000000in}{0.000000in}}{%
\pgfpathmoveto{\pgfqpoint{0.000000in}{0.000000in}}%
\pgfpathlineto{\pgfqpoint{0.000000in}{-0.048611in}}%
\pgfusepath{stroke,fill}%
}%
\begin{pgfscope}%
\pgfsys@transformshift{6.131424in}{2.848088in}%
\pgfsys@useobject{currentmarker}{}%
\end{pgfscope}%
\end{pgfscope}%
\begin{pgfscope}%
\definecolor{textcolor}{rgb}{0.000000,0.000000,0.000000}%
\pgfsetstrokecolor{textcolor}%
\pgfsetfillcolor{textcolor}%
\pgftext[x=6.131424in,y=2.750866in,,top]{\color{textcolor}\rmfamily\fontsize{10.000000}{12.000000}\selectfont \(\displaystyle {700}\)}%
\end{pgfscope}%
\begin{pgfscope}%
\definecolor{textcolor}{rgb}{0.000000,0.000000,0.000000}%
\pgfsetstrokecolor{textcolor}%
\pgfsetfillcolor{textcolor}%
\pgftext[x=3.344413in,y=2.572655in,,top]{\color{textcolor}\rmfamily\fontsize{10.000000}{12.000000}\selectfont Gesamtwert eines Clusters \(\displaystyle W\) in \si{\adu}}%
\end{pgfscope}%
\begin{pgfscope}%
\pgfsetbuttcap%
\pgfsetroundjoin%
\definecolor{currentfill}{rgb}{0.000000,0.000000,0.000000}%
\pgfsetfillcolor{currentfill}%
\pgfsetlinewidth{0.803000pt}%
\definecolor{currentstroke}{rgb}{0.000000,0.000000,0.000000}%
\pgfsetstrokecolor{currentstroke}%
\pgfsetdash{}{0pt}%
\pgfsys@defobject{currentmarker}{\pgfqpoint{-0.048611in}{0.000000in}}{\pgfqpoint{-0.000000in}{0.000000in}}{%
\pgfpathmoveto{\pgfqpoint{-0.000000in}{0.000000in}}%
\pgfpathlineto{\pgfqpoint{-0.048611in}{0.000000in}}%
\pgfusepath{stroke,fill}%
}%
\begin{pgfscope}%
\pgfsys@transformshift{0.557402in}{2.923431in}%
\pgfsys@useobject{currentmarker}{}%
\end{pgfscope}%
\end{pgfscope}%
\begin{pgfscope}%
\definecolor{textcolor}{rgb}{0.000000,0.000000,0.000000}%
\pgfsetstrokecolor{textcolor}%
\pgfsetfillcolor{textcolor}%
\pgftext[x=0.282710in, y=2.875606in, left, base]{\color{textcolor}\rmfamily\fontsize{10.000000}{12.000000}\selectfont \num{0.0}}%
\end{pgfscope}%
\begin{pgfscope}%
\pgfsetbuttcap%
\pgfsetroundjoin%
\definecolor{currentfill}{rgb}{0.000000,0.000000,0.000000}%
\pgfsetfillcolor{currentfill}%
\pgfsetlinewidth{0.803000pt}%
\definecolor{currentstroke}{rgb}{0.000000,0.000000,0.000000}%
\pgfsetstrokecolor{currentstroke}%
\pgfsetdash{}{0pt}%
\pgfsys@defobject{currentmarker}{\pgfqpoint{-0.048611in}{0.000000in}}{\pgfqpoint{-0.000000in}{0.000000in}}{%
\pgfpathmoveto{\pgfqpoint{-0.000000in}{0.000000in}}%
\pgfpathlineto{\pgfqpoint{-0.048611in}{0.000000in}}%
\pgfusepath{stroke,fill}%
}%
\begin{pgfscope}%
\pgfsys@transformshift{0.557402in}{3.426855in}%
\pgfsys@useobject{currentmarker}{}%
\end{pgfscope}%
\end{pgfscope}%
\begin{pgfscope}%
\definecolor{textcolor}{rgb}{0.000000,0.000000,0.000000}%
\pgfsetstrokecolor{textcolor}%
\pgfsetfillcolor{textcolor}%
\pgftext[x=0.282710in, y=3.379031in, left, base]{\color{textcolor}\rmfamily\fontsize{10.000000}{12.000000}\selectfont \num{0.5}}%
\end{pgfscope}%
\begin{pgfscope}%
\pgfsetbuttcap%
\pgfsetroundjoin%
\definecolor{currentfill}{rgb}{0.000000,0.000000,0.000000}%
\pgfsetfillcolor{currentfill}%
\pgfsetlinewidth{0.803000pt}%
\definecolor{currentstroke}{rgb}{0.000000,0.000000,0.000000}%
\pgfsetstrokecolor{currentstroke}%
\pgfsetdash{}{0pt}%
\pgfsys@defobject{currentmarker}{\pgfqpoint{-0.048611in}{0.000000in}}{\pgfqpoint{-0.000000in}{0.000000in}}{%
\pgfpathmoveto{\pgfqpoint{-0.000000in}{0.000000in}}%
\pgfpathlineto{\pgfqpoint{-0.048611in}{0.000000in}}%
\pgfusepath{stroke,fill}%
}%
\begin{pgfscope}%
\pgfsys@transformshift{0.557402in}{3.930280in}%
\pgfsys@useobject{currentmarker}{}%
\end{pgfscope}%
\end{pgfscope}%
\begin{pgfscope}%
\definecolor{textcolor}{rgb}{0.000000,0.000000,0.000000}%
\pgfsetstrokecolor{textcolor}%
\pgfsetfillcolor{textcolor}%
\pgftext[x=0.282710in, y=3.882456in, left, base]{\color{textcolor}\rmfamily\fontsize{10.000000}{12.000000}\selectfont \num{1.0}}%
\end{pgfscope}%
\begin{pgfscope}%
\pgfsetbuttcap%
\pgfsetroundjoin%
\definecolor{currentfill}{rgb}{0.000000,0.000000,0.000000}%
\pgfsetfillcolor{currentfill}%
\pgfsetlinewidth{0.803000pt}%
\definecolor{currentstroke}{rgb}{0.000000,0.000000,0.000000}%
\pgfsetstrokecolor{currentstroke}%
\pgfsetdash{}{0pt}%
\pgfsys@defobject{currentmarker}{\pgfqpoint{-0.048611in}{0.000000in}}{\pgfqpoint{-0.000000in}{0.000000in}}{%
\pgfpathmoveto{\pgfqpoint{-0.000000in}{0.000000in}}%
\pgfpathlineto{\pgfqpoint{-0.048611in}{0.000000in}}%
\pgfusepath{stroke,fill}%
}%
\begin{pgfscope}%
\pgfsys@transformshift{0.557402in}{4.433705in}%
\pgfsys@useobject{currentmarker}{}%
\end{pgfscope}%
\end{pgfscope}%
\begin{pgfscope}%
\definecolor{textcolor}{rgb}{0.000000,0.000000,0.000000}%
\pgfsetstrokecolor{textcolor}%
\pgfsetfillcolor{textcolor}%
\pgftext[x=0.282710in, y=4.385880in, left, base]{\color{textcolor}\rmfamily\fontsize{10.000000}{12.000000}\selectfont \num{1.5}}%
\end{pgfscope}%
\begin{pgfscope}%
\definecolor{textcolor}{rgb}{0.000000,0.000000,0.000000}%
\pgfsetstrokecolor{textcolor}%
\pgfsetfillcolor{textcolor}%
\pgftext[x=0.227155in,y=3.676845in,,bottom,rotate=90.000000]{\color{textcolor}\rmfamily\fontsize{10.000000}{12.000000}\selectfont Clusterzahl}%
\end{pgfscope}%
\begin{pgfscope}%
\definecolor{textcolor}{rgb}{0.000000,0.000000,0.000000}%
\pgfsetstrokecolor{textcolor}%
\pgfsetfillcolor{textcolor}%
\pgftext[x=0.557402in,y=4.547268in,left,base]{\color{textcolor}\rmfamily\fontsize{10.000000}{12.000000}\selectfont \(\displaystyle \times{10^{9}}{}\)}%
\end{pgfscope}%
\begin{pgfscope}%
\pgfpathrectangle{\pgfqpoint{0.557402in}{2.848088in}}{\pgfqpoint{5.574022in}{1.657514in}}%
\pgfusepath{clip}%
\pgfsetrectcap%
\pgfsetroundjoin%
\pgfsetlinewidth{1.505625pt}%
\definecolor{currentstroke}{rgb}{0.121569,0.466667,0.705882}%
\pgfsetstrokecolor{currentstroke}%
\pgfsetdash{}{0pt}%
\pgfpathmoveto{\pgfqpoint{0.555402in}{2.923432in}}%
\pgfpathlineto{\pgfqpoint{0.921572in}{2.923599in}}%
\pgfpathlineto{\pgfqpoint{0.929004in}{4.430260in}}%
\pgfpathlineto{\pgfqpoint{0.936436in}{2.923636in}}%
\pgfpathlineto{\pgfqpoint{1.203989in}{2.924413in}}%
\pgfpathlineto{\pgfqpoint{1.367493in}{2.924689in}}%
\pgfpathlineto{\pgfqpoint{1.508702in}{2.924509in}}%
\pgfpathlineto{\pgfqpoint{2.014080in}{2.923508in}}%
\pgfpathlineto{\pgfqpoint{2.526890in}{2.923431in}}%
\pgfpathlineto{\pgfqpoint{6.133424in}{2.923431in}}%
\pgfpathlineto{\pgfqpoint{6.133424in}{2.923431in}}%
\pgfusepath{stroke}%
\end{pgfscope}%
\begin{pgfscope}%
\pgfpathrectangle{\pgfqpoint{0.557402in}{2.848088in}}{\pgfqpoint{5.574022in}{1.657514in}}%
\pgfusepath{clip}%
\pgfsetrectcap%
\pgfsetroundjoin%
\pgfsetlinewidth{1.505625pt}%
\definecolor{currentstroke}{rgb}{1.000000,0.498039,0.054902}%
\pgfsetstrokecolor{currentstroke}%
\pgfsetdash{}{0pt}%
\pgfpathmoveto{\pgfqpoint{0.555402in}{2.923799in}}%
\pgfpathlineto{\pgfqpoint{0.921572in}{2.923960in}}%
\pgfpathlineto{\pgfqpoint{0.929004in}{4.428849in}}%
\pgfpathlineto{\pgfqpoint{0.936436in}{2.923968in}}%
\pgfpathlineto{\pgfqpoint{1.464110in}{2.923862in}}%
\pgfpathlineto{\pgfqpoint{2.147857in}{2.923452in}}%
\pgfpathlineto{\pgfqpoint{4.741635in}{2.923431in}}%
\pgfpathlineto{\pgfqpoint{6.133424in}{2.923431in}}%
\pgfpathlineto{\pgfqpoint{6.133424in}{2.923431in}}%
\pgfusepath{stroke}%
\end{pgfscope}%
\begin{pgfscope}%
\pgfsetrectcap%
\pgfsetmiterjoin%
\pgfsetlinewidth{0.803000pt}%
\definecolor{currentstroke}{rgb}{0.000000,0.000000,0.000000}%
\pgfsetstrokecolor{currentstroke}%
\pgfsetdash{}{0pt}%
\pgfpathmoveto{\pgfqpoint{0.557402in}{2.848088in}}%
\pgfpathlineto{\pgfqpoint{0.557402in}{4.505602in}}%
\pgfusepath{stroke}%
\end{pgfscope}%
\begin{pgfscope}%
\pgfsetrectcap%
\pgfsetmiterjoin%
\pgfsetlinewidth{0.803000pt}%
\definecolor{currentstroke}{rgb}{0.000000,0.000000,0.000000}%
\pgfsetstrokecolor{currentstroke}%
\pgfsetdash{}{0pt}%
\pgfpathmoveto{\pgfqpoint{6.131424in}{2.848088in}}%
\pgfpathlineto{\pgfqpoint{6.131424in}{4.505602in}}%
\pgfusepath{stroke}%
\end{pgfscope}%
\begin{pgfscope}%
\pgfsetrectcap%
\pgfsetmiterjoin%
\pgfsetlinewidth{0.803000pt}%
\definecolor{currentstroke}{rgb}{0.000000,0.000000,0.000000}%
\pgfsetstrokecolor{currentstroke}%
\pgfsetdash{}{0pt}%
\pgfpathmoveto{\pgfqpoint{0.557402in}{2.848088in}}%
\pgfpathlineto{\pgfqpoint{6.131424in}{2.848088in}}%
\pgfusepath{stroke}%
\end{pgfscope}%
\begin{pgfscope}%
\pgfsetrectcap%
\pgfsetmiterjoin%
\pgfsetlinewidth{0.803000pt}%
\definecolor{currentstroke}{rgb}{0.000000,0.000000,0.000000}%
\pgfsetstrokecolor{currentstroke}%
\pgfsetdash{}{0pt}%
\pgfpathmoveto{\pgfqpoint{0.557402in}{4.505602in}}%
\pgfpathlineto{\pgfqpoint{6.131424in}{4.505602in}}%
\pgfusepath{stroke}%
\end{pgfscope}%
\begin{pgfscope}%
\definecolor{textcolor}{rgb}{0.000000,0.000000,0.000000}%
\pgfsetstrokecolor{textcolor}%
\pgfsetfillcolor{textcolor}%
\pgftext[x=0.000000in,y=4.671353in,left,base]{\color{textcolor}\rmfamily\fontsize{10.000000}{12.000000}\selectfont (b)}%
\end{pgfscope}%
\begin{pgfscope}%
\pgfpathrectangle{\pgfqpoint{0.557402in}{2.848088in}}{\pgfqpoint{5.574022in}{1.657514in}}%
\pgfusepath{clip}%
\pgfsetbuttcap%
\pgfsetmiterjoin%
\pgfsetlinewidth{1.003750pt}%
\definecolor{currentstroke}{rgb}{0.000000,0.000000,0.000000}%
\pgfsetstrokecolor{currentstroke}%
\pgfsetstrokeopacity{0.500000}%
\pgfsetdash{}{0pt}%
\pgfpathmoveto{\pgfqpoint{2.103264in}{2.923430in}}%
\pgfpathlineto{\pgfqpoint{4.407194in}{2.923430in}}%
\pgfpathlineto{\pgfqpoint{4.407194in}{2.923452in}}%
\pgfpathlineto{\pgfqpoint{2.103264in}{2.923452in}}%
\pgfpathlineto{\pgfqpoint{2.103264in}{2.923430in}}%
\pgfpathclose%
\pgfusepath{stroke}%
\end{pgfscope}%
\begin{pgfscope}%
\pgfsetroundcap%
\pgfsetroundjoin%
\pgfsetlinewidth{1.003750pt}%
\definecolor{currentstroke}{rgb}{0.000000,0.000000,0.000000}%
\pgfsetstrokecolor{currentstroke}%
\pgfsetstrokeopacity{0.500000}%
\pgfsetdash{}{0pt}%
\pgfpathmoveto{\pgfqpoint{2.842751in}{4.207249in}}%
\pgfpathquadraticcurveto{\pgfqpoint{2.473008in}{3.565350in}}{\pgfqpoint{2.103264in}{2.923452in}}%
\pgfusepath{stroke}%
\end{pgfscope}%
\begin{pgfscope}%
\pgfsetroundcap%
\pgfsetroundjoin%
\pgfsetlinewidth{1.003750pt}%
\definecolor{currentstroke}{rgb}{0.000000,0.000000,0.000000}%
\pgfsetstrokecolor{currentstroke}%
\pgfsetstrokeopacity{0.500000}%
\pgfsetdash{}{0pt}%
\pgfpathmoveto{\pgfqpoint{5.462542in}{3.262466in}}%
\pgfpathquadraticcurveto{\pgfqpoint{4.934868in}{3.092948in}}{\pgfqpoint{4.407194in}{2.923430in}}%
\pgfusepath{stroke}%
\end{pgfscope}%
\begin{pgfscope}%
\pgfsetbuttcap%
\pgfsetmiterjoin%
\definecolor{currentfill}{rgb}{1.000000,1.000000,1.000000}%
\pgfsetfillcolor{currentfill}%
\pgfsetlinewidth{0.000000pt}%
\definecolor{currentstroke}{rgb}{0.000000,0.000000,0.000000}%
\pgfsetstrokecolor{currentstroke}%
\pgfsetstrokeopacity{0.000000}%
\pgfsetdash{}{0pt}%
\pgfpathmoveto{\pgfqpoint{2.842751in}{3.262466in}}%
\pgfpathlineto{\pgfqpoint{5.462542in}{3.262466in}}%
\pgfpathlineto{\pgfqpoint{5.462542in}{4.207249in}}%
\pgfpathlineto{\pgfqpoint{2.842751in}{4.207249in}}%
\pgfpathlineto{\pgfqpoint{2.842751in}{3.262466in}}%
\pgfpathclose%
\pgfusepath{fill}%
\end{pgfscope}%
\begin{pgfscope}%
\pgfsetbuttcap%
\pgfsetroundjoin%
\definecolor{currentfill}{rgb}{0.000000,0.000000,0.000000}%
\pgfsetfillcolor{currentfill}%
\pgfsetlinewidth{0.803000pt}%
\definecolor{currentstroke}{rgb}{0.000000,0.000000,0.000000}%
\pgfsetstrokecolor{currentstroke}%
\pgfsetdash{}{0pt}%
\pgfsys@defobject{currentmarker}{\pgfqpoint{0.000000in}{0.000000in}}{\pgfqpoint{0.000000in}{0.048611in}}{%
\pgfpathmoveto{\pgfqpoint{0.000000in}{0.000000in}}%
\pgfpathlineto{\pgfqpoint{0.000000in}{0.048611in}}%
\pgfusepath{stroke,fill}%
}%
\begin{pgfscope}%
\pgfsys@transformshift{2.952613in}{4.207249in}%
\pgfsys@useobject{currentmarker}{}%
\end{pgfscope}%
\end{pgfscope}%
\begin{pgfscope}%
\definecolor{textcolor}{rgb}{0.000000,0.000000,0.000000}%
\pgfsetstrokecolor{textcolor}%
\pgfsetfillcolor{textcolor}%
\pgftext[x=2.952613in,y=4.304471in,,bottom]{\color{textcolor}\rmfamily\fontsize{10.000000}{12.000000}\selectfont \(\displaystyle {171}\)}%
\end{pgfscope}%
\begin{pgfscope}%
\pgfsetbuttcap%
\pgfsetroundjoin%
\definecolor{currentfill}{rgb}{0.000000,0.000000,0.000000}%
\pgfsetfillcolor{currentfill}%
\pgfsetlinewidth{0.803000pt}%
\definecolor{currentstroke}{rgb}{0.000000,0.000000,0.000000}%
\pgfsetstrokecolor{currentstroke}%
\pgfsetdash{}{0pt}%
\pgfsys@defobject{currentmarker}{\pgfqpoint{0.000000in}{0.000000in}}{\pgfqpoint{0.000000in}{0.048611in}}{%
\pgfpathmoveto{\pgfqpoint{0.000000in}{0.000000in}}%
\pgfpathlineto{\pgfqpoint{0.000000in}{0.048611in}}%
\pgfusepath{stroke,fill}%
}%
\begin{pgfscope}%
\pgfsys@transformshift{4.042784in}{4.207249in}%
\pgfsys@useobject{currentmarker}{}%
\end{pgfscope}%
\end{pgfscope}%
\begin{pgfscope}%
\definecolor{textcolor}{rgb}{0.000000,0.000000,0.000000}%
\pgfsetstrokecolor{textcolor}%
\pgfsetfillcolor{textcolor}%
\pgftext[x=4.042784in,y=4.304471in,,bottom]{\color{textcolor}\rmfamily\fontsize{10.000000}{12.000000}\selectfont \(\displaystyle {300}\)}%
\end{pgfscope}%
\begin{pgfscope}%
\pgfsetbuttcap%
\pgfsetroundjoin%
\definecolor{currentfill}{rgb}{0.000000,0.000000,0.000000}%
\pgfsetfillcolor{currentfill}%
\pgfsetlinewidth{0.803000pt}%
\definecolor{currentstroke}{rgb}{0.000000,0.000000,0.000000}%
\pgfsetstrokecolor{currentstroke}%
\pgfsetdash{}{0pt}%
\pgfsys@defobject{currentmarker}{\pgfqpoint{0.000000in}{0.000000in}}{\pgfqpoint{0.000000in}{0.048611in}}{%
\pgfpathmoveto{\pgfqpoint{0.000000in}{0.000000in}}%
\pgfpathlineto{\pgfqpoint{0.000000in}{0.048611in}}%
\pgfusepath{stroke,fill}%
}%
\begin{pgfscope}%
\pgfsys@transformshift{4.887878in}{4.207249in}%
\pgfsys@useobject{currentmarker}{}%
\end{pgfscope}%
\end{pgfscope}%
\begin{pgfscope}%
\definecolor{textcolor}{rgb}{0.000000,0.000000,0.000000}%
\pgfsetstrokecolor{textcolor}%
\pgfsetfillcolor{textcolor}%
\pgftext[x=4.887878in,y=4.304471in,,bottom]{\color{textcolor}\rmfamily\fontsize{10.000000}{12.000000}\selectfont \(\displaystyle {400}\)}%
\end{pgfscope}%
\begin{pgfscope}%
\pgfsetbuttcap%
\pgfsetroundjoin%
\definecolor{currentfill}{rgb}{0.000000,0.000000,0.000000}%
\pgfsetfillcolor{currentfill}%
\pgfsetlinewidth{0.803000pt}%
\definecolor{currentstroke}{rgb}{0.000000,0.000000,0.000000}%
\pgfsetstrokecolor{currentstroke}%
\pgfsetdash{}{0pt}%
\pgfsys@defobject{currentmarker}{\pgfqpoint{0.000000in}{0.000000in}}{\pgfqpoint{0.048611in}{0.000000in}}{%
\pgfpathmoveto{\pgfqpoint{0.000000in}{0.000000in}}%
\pgfpathlineto{\pgfqpoint{0.048611in}{0.000000in}}%
\pgfusepath{stroke,fill}%
}%
\begin{pgfscope}%
\pgfsys@transformshift{5.462542in}{3.305411in}%
\pgfsys@useobject{currentmarker}{}%
\end{pgfscope}%
\end{pgfscope}%
\begin{pgfscope}%
\definecolor{textcolor}{rgb}{0.000000,0.000000,0.000000}%
\pgfsetstrokecolor{textcolor}%
\pgfsetfillcolor{textcolor}%
\pgftext[x=5.559764in, y=3.257583in, left, base]{\color{textcolor}\rmfamily\fontsize{10.000000}{12.000000}\selectfont \(\displaystyle {0}\)}%
\end{pgfscope}%
\begin{pgfscope}%
\pgfsetbuttcap%
\pgfsetroundjoin%
\definecolor{currentfill}{rgb}{0.000000,0.000000,0.000000}%
\pgfsetfillcolor{currentfill}%
\pgfsetlinewidth{0.803000pt}%
\definecolor{currentstroke}{rgb}{0.000000,0.000000,0.000000}%
\pgfsetstrokecolor{currentstroke}%
\pgfsetdash{}{0pt}%
\pgfsys@defobject{currentmarker}{\pgfqpoint{0.000000in}{0.000000in}}{\pgfqpoint{0.048611in}{0.000000in}}{%
\pgfpathmoveto{\pgfqpoint{0.000000in}{0.000000in}}%
\pgfpathlineto{\pgfqpoint{0.048611in}{0.000000in}}%
\pgfusepath{stroke,fill}%
}%
\begin{pgfscope}%
\pgfsys@transformshift{5.462542in}{3.736669in}%
\pgfsys@useobject{currentmarker}{}%
\end{pgfscope}%
\end{pgfscope}%
\begin{pgfscope}%
\definecolor{textcolor}{rgb}{0.000000,0.000000,0.000000}%
\pgfsetstrokecolor{textcolor}%
\pgfsetfillcolor{textcolor}%
\pgftext[x=5.559764in, y=3.688841in, left, base]{\color{textcolor}\rmfamily\fontsize{10.000000}{12.000000}\selectfont \(\displaystyle {10000}\)}%
\end{pgfscope}%
\begin{pgfscope}%
\pgfsetbuttcap%
\pgfsetroundjoin%
\definecolor{currentfill}{rgb}{0.000000,0.000000,0.000000}%
\pgfsetfillcolor{currentfill}%
\pgfsetlinewidth{0.803000pt}%
\definecolor{currentstroke}{rgb}{0.000000,0.000000,0.000000}%
\pgfsetstrokecolor{currentstroke}%
\pgfsetdash{}{0pt}%
\pgfsys@defobject{currentmarker}{\pgfqpoint{0.000000in}{0.000000in}}{\pgfqpoint{0.048611in}{0.000000in}}{%
\pgfpathmoveto{\pgfqpoint{0.000000in}{0.000000in}}%
\pgfpathlineto{\pgfqpoint{0.048611in}{0.000000in}}%
\pgfusepath{stroke,fill}%
}%
\begin{pgfscope}%
\pgfsys@transformshift{5.462542in}{4.167927in}%
\pgfsys@useobject{currentmarker}{}%
\end{pgfscope}%
\end{pgfscope}%
\begin{pgfscope}%
\definecolor{textcolor}{rgb}{0.000000,0.000000,0.000000}%
\pgfsetstrokecolor{textcolor}%
\pgfsetfillcolor{textcolor}%
\pgftext[x=5.559764in, y=4.120099in, left, base]{\color{textcolor}\rmfamily\fontsize{10.000000}{12.000000}\selectfont \(\displaystyle {20000}\)}%
\end{pgfscope}%
\begin{pgfscope}%
\pgfpathrectangle{\pgfqpoint{2.842751in}{3.262466in}}{\pgfqpoint{2.619790in}{0.944783in}}%
\pgfusepath{clip}%
\pgfsetrectcap%
\pgfsetroundjoin%
\pgfsetlinewidth{1.505625pt}%
\definecolor{currentstroke}{rgb}{0.121569,0.466667,0.705882}%
\pgfsetstrokecolor{currentstroke}%
\pgfsetdash{}{0pt}%
\pgfpathmoveto{\pgfqpoint{2.927261in}{4.164305in}}%
\pgfpathlineto{\pgfqpoint{2.935712in}{4.108888in}}%
\pgfpathlineto{\pgfqpoint{2.944163in}{4.063261in}}%
\pgfpathlineto{\pgfqpoint{2.952613in}{4.007154in}}%
\pgfpathlineto{\pgfqpoint{2.961064in}{3.982572in}}%
\pgfpathlineto{\pgfqpoint{2.969515in}{3.932158in}}%
\pgfpathlineto{\pgfqpoint{2.977966in}{3.886014in}}%
\pgfpathlineto{\pgfqpoint{2.986417in}{3.842845in}}%
\pgfpathlineto{\pgfqpoint{2.994868in}{3.820118in}}%
\pgfpathlineto{\pgfqpoint{3.003319in}{3.781649in}}%
\pgfpathlineto{\pgfqpoint{3.011770in}{3.744389in}}%
\pgfpathlineto{\pgfqpoint{3.020221in}{3.725198in}}%
\pgfpathlineto{\pgfqpoint{3.028672in}{3.696562in}}%
\pgfpathlineto{\pgfqpoint{3.037123in}{3.662924in}}%
\pgfpathlineto{\pgfqpoint{3.045574in}{3.645458in}}%
\pgfpathlineto{\pgfqpoint{3.054025in}{3.615874in}}%
\pgfpathlineto{\pgfqpoint{3.062476in}{3.596812in}}%
\pgfpathlineto{\pgfqpoint{3.070927in}{3.576888in}}%
\pgfpathlineto{\pgfqpoint{3.079378in}{3.562311in}}%
\pgfpathlineto{\pgfqpoint{3.087828in}{3.542085in}}%
\pgfpathlineto{\pgfqpoint{3.096279in}{3.530657in}}%
\pgfpathlineto{\pgfqpoint{3.104730in}{3.517978in}}%
\pgfpathlineto{\pgfqpoint{3.113181in}{3.494820in}}%
\pgfpathlineto{\pgfqpoint{3.121632in}{3.485289in}}%
\pgfpathlineto{\pgfqpoint{3.130083in}{3.469764in}}%
\pgfpathlineto{\pgfqpoint{3.138534in}{3.462432in}}%
\pgfpathlineto{\pgfqpoint{3.146985in}{3.450012in}}%
\pgfpathlineto{\pgfqpoint{3.155436in}{3.440093in}}%
\pgfpathlineto{\pgfqpoint{3.163887in}{3.433624in}}%
\pgfpathlineto{\pgfqpoint{3.172338in}{3.420298in}}%
\pgfpathlineto{\pgfqpoint{3.180789in}{3.415166in}}%
\pgfpathlineto{\pgfqpoint{3.189240in}{3.410595in}}%
\pgfpathlineto{\pgfqpoint{3.197691in}{3.400805in}}%
\pgfpathlineto{\pgfqpoint{3.206142in}{3.392439in}}%
\pgfpathlineto{\pgfqpoint{3.214593in}{3.384547in}}%
\pgfpathlineto{\pgfqpoint{3.223043in}{3.381399in}}%
\pgfpathlineto{\pgfqpoint{3.231494in}{3.375620in}}%
\pgfpathlineto{\pgfqpoint{3.239945in}{3.371350in}}%
\pgfpathlineto{\pgfqpoint{3.256847in}{3.361216in}}%
\pgfpathlineto{\pgfqpoint{3.265298in}{3.355868in}}%
\pgfpathlineto{\pgfqpoint{3.273749in}{3.353410in}}%
\pgfpathlineto{\pgfqpoint{3.282200in}{3.351642in}}%
\pgfpathlineto{\pgfqpoint{3.290651in}{3.346639in}}%
\pgfpathlineto{\pgfqpoint{3.299102in}{3.342758in}}%
\pgfpathlineto{\pgfqpoint{3.316004in}{3.338445in}}%
\pgfpathlineto{\pgfqpoint{3.324455in}{3.334780in}}%
\pgfpathlineto{\pgfqpoint{3.332906in}{3.334607in}}%
\pgfpathlineto{\pgfqpoint{3.341357in}{3.332580in}}%
\pgfpathlineto{\pgfqpoint{3.349808in}{3.329907in}}%
\pgfpathlineto{\pgfqpoint{3.358258in}{3.329993in}}%
\pgfpathlineto{\pgfqpoint{3.366709in}{3.326543in}}%
\pgfpathlineto{\pgfqpoint{3.375160in}{3.324041in}}%
\pgfpathlineto{\pgfqpoint{3.383611in}{3.324041in}}%
\pgfpathlineto{\pgfqpoint{3.392062in}{3.324343in}}%
\pgfpathlineto{\pgfqpoint{3.400513in}{3.321885in}}%
\pgfpathlineto{\pgfqpoint{3.425866in}{3.317098in}}%
\pgfpathlineto{\pgfqpoint{3.442768in}{3.316106in}}%
\pgfpathlineto{\pgfqpoint{3.459670in}{3.314079in}}%
\pgfpathlineto{\pgfqpoint{3.485023in}{3.312354in}}%
\pgfpathlineto{\pgfqpoint{3.493473in}{3.312786in}}%
\pgfpathlineto{\pgfqpoint{3.501924in}{3.311061in}}%
\pgfpathlineto{\pgfqpoint{3.510375in}{3.310284in}}%
\pgfpathlineto{\pgfqpoint{3.527277in}{3.309594in}}%
\pgfpathlineto{\pgfqpoint{3.535728in}{3.310069in}}%
\pgfpathlineto{\pgfqpoint{3.552630in}{3.308818in}}%
\pgfpathlineto{\pgfqpoint{3.561081in}{3.310112in}}%
\pgfpathlineto{\pgfqpoint{3.569532in}{3.308602in}}%
\pgfpathlineto{\pgfqpoint{3.577983in}{3.308602in}}%
\pgfpathlineto{\pgfqpoint{3.586434in}{3.307783in}}%
\pgfpathlineto{\pgfqpoint{3.594885in}{3.307395in}}%
\pgfpathlineto{\pgfqpoint{3.603336in}{3.307697in}}%
\pgfpathlineto{\pgfqpoint{3.611787in}{3.307524in}}%
\pgfpathlineto{\pgfqpoint{3.620238in}{3.308301in}}%
\pgfpathlineto{\pgfqpoint{3.628688in}{3.307567in}}%
\pgfpathlineto{\pgfqpoint{3.637139in}{3.307179in}}%
\pgfpathlineto{\pgfqpoint{3.645590in}{3.307309in}}%
\pgfpathlineto{\pgfqpoint{3.654041in}{3.306576in}}%
\pgfpathlineto{\pgfqpoint{3.662492in}{3.306921in}}%
\pgfpathlineto{\pgfqpoint{3.670943in}{3.306446in}}%
\pgfpathlineto{\pgfqpoint{3.687845in}{3.306489in}}%
\pgfpathlineto{\pgfqpoint{3.696296in}{3.306274in}}%
\pgfpathlineto{\pgfqpoint{3.704747in}{3.306662in}}%
\pgfpathlineto{\pgfqpoint{3.730100in}{3.306101in}}%
\pgfpathlineto{\pgfqpoint{3.738551in}{3.306489in}}%
\pgfpathlineto{\pgfqpoint{3.747002in}{3.306144in}}%
\pgfpathlineto{\pgfqpoint{3.763903in}{3.306058in}}%
\pgfpathlineto{\pgfqpoint{3.772354in}{3.306231in}}%
\pgfpathlineto{\pgfqpoint{3.780805in}{3.305799in}}%
\pgfpathlineto{\pgfqpoint{3.797707in}{3.305842in}}%
\pgfpathlineto{\pgfqpoint{3.806158in}{3.305972in}}%
\pgfpathlineto{\pgfqpoint{3.823060in}{3.305799in}}%
\pgfpathlineto{\pgfqpoint{3.890667in}{3.305540in}}%
\pgfpathlineto{\pgfqpoint{3.907569in}{3.305670in}}%
\pgfpathlineto{\pgfqpoint{3.932922in}{3.305454in}}%
\pgfpathlineto{\pgfqpoint{3.949824in}{3.305454in}}%
\pgfpathlineto{\pgfqpoint{5.454091in}{3.305411in}}%
\pgfpathlineto{\pgfqpoint{5.454091in}{3.305411in}}%
\pgfusepath{stroke}%
\end{pgfscope}%
\begin{pgfscope}%
\pgfpathrectangle{\pgfqpoint{2.842751in}{3.262466in}}{\pgfqpoint{2.619790in}{0.944783in}}%
\pgfusepath{clip}%
\pgfsetrectcap%
\pgfsetroundjoin%
\pgfsetlinewidth{1.505625pt}%
\definecolor{currentstroke}{rgb}{1.000000,0.498039,0.054902}%
\pgfsetstrokecolor{currentstroke}%
\pgfsetdash{}{0pt}%
\pgfpathmoveto{\pgfqpoint{2.927261in}{4.100090in}}%
\pgfpathlineto{\pgfqpoint{2.935712in}{4.082279in}}%
\pgfpathlineto{\pgfqpoint{2.944163in}{4.058215in}}%
\pgfpathlineto{\pgfqpoint{2.952613in}{4.030873in}}%
\pgfpathlineto{\pgfqpoint{2.961064in}{3.997149in}}%
\pgfpathlineto{\pgfqpoint{2.969515in}{3.981839in}}%
\pgfpathlineto{\pgfqpoint{2.977966in}{3.959500in}}%
\pgfpathlineto{\pgfqpoint{2.986417in}{3.939274in}}%
\pgfpathlineto{\pgfqpoint{2.994868in}{3.923318in}}%
\pgfpathlineto{\pgfqpoint{3.003319in}{3.907922in}}%
\pgfpathlineto{\pgfqpoint{3.011770in}{3.880752in}}%
\pgfpathlineto{\pgfqpoint{3.020221in}{3.868807in}}%
\pgfpathlineto{\pgfqpoint{3.028672in}{3.854144in}}%
\pgfpathlineto{\pgfqpoint{3.037123in}{3.840257in}}%
\pgfpathlineto{\pgfqpoint{3.045574in}{3.827104in}}%
\pgfpathlineto{\pgfqpoint{3.054025in}{3.814942in}}%
\pgfpathlineto{\pgfqpoint{3.062476in}{3.792862in}}%
\pgfpathlineto{\pgfqpoint{3.070927in}{3.787773in}}%
\pgfpathlineto{\pgfqpoint{3.079378in}{3.771989in}}%
\pgfpathlineto{\pgfqpoint{3.087828in}{3.761639in}}%
\pgfpathlineto{\pgfqpoint{3.096279in}{3.761855in}}%
\pgfpathlineto{\pgfqpoint{3.104730in}{3.751979in}}%
\pgfpathlineto{\pgfqpoint{3.113181in}{3.738739in}}%
\pgfpathlineto{\pgfqpoint{3.121632in}{3.731408in}}%
\pgfpathlineto{\pgfqpoint{3.130083in}{3.720411in}}%
\pgfpathlineto{\pgfqpoint{3.138534in}{3.718815in}}%
\pgfpathlineto{\pgfqpoint{3.146985in}{3.708292in}}%
\pgfpathlineto{\pgfqpoint{3.155436in}{3.699193in}}%
\pgfpathlineto{\pgfqpoint{3.163887in}{3.698589in}}%
\pgfpathlineto{\pgfqpoint{3.172338in}{3.682762in}}%
\pgfpathlineto{\pgfqpoint{3.180789in}{3.687290in}}%
\pgfpathlineto{\pgfqpoint{3.189240in}{3.678018in}}%
\pgfpathlineto{\pgfqpoint{3.197691in}{3.672627in}}%
\pgfpathlineto{\pgfqpoint{3.206142in}{3.669867in}}%
\pgfpathlineto{\pgfqpoint{3.214593in}{3.658741in}}%
\pgfpathlineto{\pgfqpoint{3.223043in}{3.649512in}}%
\pgfpathlineto{\pgfqpoint{3.231494in}{3.647528in}}%
\pgfpathlineto{\pgfqpoint{3.239945in}{3.641577in}}%
\pgfpathlineto{\pgfqpoint{3.248396in}{3.643647in}}%
\pgfpathlineto{\pgfqpoint{3.256847in}{3.630623in}}%
\pgfpathlineto{\pgfqpoint{3.265298in}{3.631183in}}%
\pgfpathlineto{\pgfqpoint{3.273749in}{3.630493in}}%
\pgfpathlineto{\pgfqpoint{3.282200in}{3.624326in}}%
\pgfpathlineto{\pgfqpoint{3.290651in}{3.623033in}}%
\pgfpathlineto{\pgfqpoint{3.307553in}{3.600305in}}%
\pgfpathlineto{\pgfqpoint{3.324455in}{3.601944in}}%
\pgfpathlineto{\pgfqpoint{3.332906in}{3.592801in}}%
\pgfpathlineto{\pgfqpoint{3.341357in}{3.593146in}}%
\pgfpathlineto{\pgfqpoint{3.349808in}{3.591896in}}%
\pgfpathlineto{\pgfqpoint{3.358258in}{3.586419in}}%
\pgfpathlineto{\pgfqpoint{3.366709in}{3.577794in}}%
\pgfpathlineto{\pgfqpoint{3.375160in}{3.579303in}}%
\pgfpathlineto{\pgfqpoint{3.383611in}{3.576500in}}%
\pgfpathlineto{\pgfqpoint{3.392062in}{3.570678in}}%
\pgfpathlineto{\pgfqpoint{3.400513in}{3.562743in}}%
\pgfpathlineto{\pgfqpoint{3.408964in}{3.557180in}}%
\pgfpathlineto{\pgfqpoint{3.417415in}{3.552306in}}%
\pgfpathlineto{\pgfqpoint{3.425866in}{3.559077in}}%
\pgfpathlineto{\pgfqpoint{3.434317in}{3.554506in}}%
\pgfpathlineto{\pgfqpoint{3.442768in}{3.547606in}}%
\pgfpathlineto{\pgfqpoint{3.451219in}{3.544242in}}%
\pgfpathlineto{\pgfqpoint{3.459670in}{3.542042in}}%
\pgfpathlineto{\pgfqpoint{3.468121in}{3.538894in}}%
\pgfpathlineto{\pgfqpoint{3.476572in}{3.530355in}}%
\pgfpathlineto{\pgfqpoint{3.485023in}{3.526129in}}%
\pgfpathlineto{\pgfqpoint{3.493473in}{3.524275in}}%
\pgfpathlineto{\pgfqpoint{3.501924in}{3.521859in}}%
\pgfpathlineto{\pgfqpoint{3.510375in}{3.519919in}}%
\pgfpathlineto{\pgfqpoint{3.518826in}{3.517590in}}%
\pgfpathlineto{\pgfqpoint{3.527277in}{3.514657in}}%
\pgfpathlineto{\pgfqpoint{3.535728in}{3.513536in}}%
\pgfpathlineto{\pgfqpoint{3.544179in}{3.508577in}}%
\pgfpathlineto{\pgfqpoint{3.552630in}{3.502582in}}%
\pgfpathlineto{\pgfqpoint{3.561081in}{3.497364in}}%
\pgfpathlineto{\pgfqpoint{3.569532in}{3.498572in}}%
\pgfpathlineto{\pgfqpoint{3.577983in}{3.491413in}}%
\pgfpathlineto{\pgfqpoint{3.586434in}{3.489990in}}%
\pgfpathlineto{\pgfqpoint{3.594885in}{3.488006in}}%
\pgfpathlineto{\pgfqpoint{3.603336in}{3.484168in}}%
\pgfpathlineto{\pgfqpoint{3.611787in}{3.478216in}}%
\pgfpathlineto{\pgfqpoint{3.620238in}{3.476966in}}%
\pgfpathlineto{\pgfqpoint{3.628688in}{3.476750in}}%
\pgfpathlineto{\pgfqpoint{3.637139in}{3.467866in}}%
\pgfpathlineto{\pgfqpoint{3.645590in}{3.467650in}}%
\pgfpathlineto{\pgfqpoint{3.662492in}{3.456481in}}%
\pgfpathlineto{\pgfqpoint{3.670943in}{3.457300in}}%
\pgfpathlineto{\pgfqpoint{3.679394in}{3.454238in}}%
\pgfpathlineto{\pgfqpoint{3.687845in}{3.449494in}}%
\pgfpathlineto{\pgfqpoint{3.696296in}{3.448158in}}%
\pgfpathlineto{\pgfqpoint{3.704747in}{3.448028in}}%
\pgfpathlineto{\pgfqpoint{3.713198in}{3.446131in}}%
\pgfpathlineto{\pgfqpoint{3.721649in}{3.442853in}}%
\pgfpathlineto{\pgfqpoint{3.730100in}{3.440912in}}%
\pgfpathlineto{\pgfqpoint{3.738551in}{3.431727in}}%
\pgfpathlineto{\pgfqpoint{3.747002in}{3.435004in}}%
\pgfpathlineto{\pgfqpoint{3.755452in}{3.424611in}}%
\pgfpathlineto{\pgfqpoint{3.763903in}{3.428363in}}%
\pgfpathlineto{\pgfqpoint{3.772354in}{3.423964in}}%
\pgfpathlineto{\pgfqpoint{3.780805in}{3.422023in}}%
\pgfpathlineto{\pgfqpoint{3.789256in}{3.418444in}}%
\pgfpathlineto{\pgfqpoint{3.797707in}{3.422929in}}%
\pgfpathlineto{\pgfqpoint{3.806158in}{3.414304in}}%
\pgfpathlineto{\pgfqpoint{3.814609in}{3.413441in}}%
\pgfpathlineto{\pgfqpoint{3.823060in}{3.410595in}}%
\pgfpathlineto{\pgfqpoint{3.831511in}{3.410379in}}%
\pgfpathlineto{\pgfqpoint{3.848413in}{3.403910in}}%
\pgfpathlineto{\pgfqpoint{3.856864in}{3.401582in}}%
\pgfpathlineto{\pgfqpoint{3.865315in}{3.403522in}}%
\pgfpathlineto{\pgfqpoint{3.873766in}{3.402358in}}%
\pgfpathlineto{\pgfqpoint{3.882217in}{3.394164in}}%
\pgfpathlineto{\pgfqpoint{3.890667in}{3.397355in}}%
\pgfpathlineto{\pgfqpoint{3.899118in}{3.395070in}}%
\pgfpathlineto{\pgfqpoint{3.907569in}{3.393690in}}%
\pgfpathlineto{\pgfqpoint{3.916020in}{3.392698in}}%
\pgfpathlineto{\pgfqpoint{3.924471in}{3.391447in}}%
\pgfpathlineto{\pgfqpoint{3.932922in}{3.388170in}}%
\pgfpathlineto{\pgfqpoint{3.941373in}{3.384374in}}%
\pgfpathlineto{\pgfqpoint{3.949824in}{3.387868in}}%
\pgfpathlineto{\pgfqpoint{3.958275in}{3.384547in}}%
\pgfpathlineto{\pgfqpoint{3.966726in}{3.382606in}}%
\pgfpathlineto{\pgfqpoint{3.975177in}{3.385323in}}%
\pgfpathlineto{\pgfqpoint{3.983628in}{3.380234in}}%
\pgfpathlineto{\pgfqpoint{3.992079in}{3.380968in}}%
\pgfpathlineto{\pgfqpoint{4.000530in}{3.380148in}}%
\pgfpathlineto{\pgfqpoint{4.008981in}{3.376224in}}%
\pgfpathlineto{\pgfqpoint{4.017432in}{3.373075in}}%
\pgfpathlineto{\pgfqpoint{4.025882in}{3.374326in}}%
\pgfpathlineto{\pgfqpoint{4.034333in}{3.369927in}}%
\pgfpathlineto{\pgfqpoint{4.042784in}{3.368763in}}%
\pgfpathlineto{\pgfqpoint{4.051235in}{3.369669in}}%
\pgfpathlineto{\pgfqpoint{4.059686in}{3.369453in}}%
\pgfpathlineto{\pgfqpoint{4.068137in}{3.368116in}}%
\pgfpathlineto{\pgfqpoint{4.076588in}{3.367383in}}%
\pgfpathlineto{\pgfqpoint{4.085039in}{3.365701in}}%
\pgfpathlineto{\pgfqpoint{4.093490in}{3.363286in}}%
\pgfpathlineto{\pgfqpoint{4.101941in}{3.366520in}}%
\pgfpathlineto{\pgfqpoint{4.110392in}{3.368979in}}%
\pgfpathlineto{\pgfqpoint{4.118843in}{3.363113in}}%
\pgfpathlineto{\pgfqpoint{4.127294in}{3.364493in}}%
\pgfpathlineto{\pgfqpoint{4.135745in}{3.359620in}}%
\pgfpathlineto{\pgfqpoint{4.144196in}{3.360828in}}%
\pgfpathlineto{\pgfqpoint{4.152647in}{3.358499in}}%
\pgfpathlineto{\pgfqpoint{4.161097in}{3.361173in}}%
\pgfpathlineto{\pgfqpoint{4.169548in}{3.360742in}}%
\pgfpathlineto{\pgfqpoint{4.177999in}{3.358758in}}%
\pgfpathlineto{\pgfqpoint{4.186450in}{3.358283in}}%
\pgfpathlineto{\pgfqpoint{4.194901in}{3.356429in}}%
\pgfpathlineto{\pgfqpoint{4.203352in}{3.355480in}}%
\pgfpathlineto{\pgfqpoint{4.211803in}{3.360052in}}%
\pgfpathlineto{\pgfqpoint{4.220254in}{3.354100in}}%
\pgfpathlineto{\pgfqpoint{4.228705in}{3.357378in}}%
\pgfpathlineto{\pgfqpoint{4.237156in}{3.351771in}}%
\pgfpathlineto{\pgfqpoint{4.245607in}{3.352461in}}%
\pgfpathlineto{\pgfqpoint{4.254058in}{3.353798in}}%
\pgfpathlineto{\pgfqpoint{4.262509in}{3.350823in}}%
\pgfpathlineto{\pgfqpoint{4.270960in}{3.350952in}}%
\pgfpathlineto{\pgfqpoint{4.279411in}{3.352548in}}%
\pgfpathlineto{\pgfqpoint{4.287862in}{3.351513in}}%
\pgfpathlineto{\pgfqpoint{4.296312in}{3.348623in}}%
\pgfpathlineto{\pgfqpoint{4.304763in}{3.348149in}}%
\pgfpathlineto{\pgfqpoint{4.313214in}{3.350262in}}%
\pgfpathlineto{\pgfqpoint{4.321665in}{3.352936in}}%
\pgfpathlineto{\pgfqpoint{4.330116in}{3.348623in}}%
\pgfpathlineto{\pgfqpoint{4.338567in}{3.347028in}}%
\pgfpathlineto{\pgfqpoint{4.355469in}{3.347286in}}%
\pgfpathlineto{\pgfqpoint{4.363920in}{3.346596in}}%
\pgfpathlineto{\pgfqpoint{4.372371in}{3.344224in}}%
\pgfpathlineto{\pgfqpoint{4.380822in}{3.348364in}}%
\pgfpathlineto{\pgfqpoint{4.397724in}{3.344095in}}%
\pgfpathlineto{\pgfqpoint{4.406175in}{3.343232in}}%
\pgfpathlineto{\pgfqpoint{4.414626in}{3.344785in}}%
\pgfpathlineto{\pgfqpoint{4.423077in}{3.342715in}}%
\pgfpathlineto{\pgfqpoint{4.431527in}{3.342197in}}%
\pgfpathlineto{\pgfqpoint{4.439978in}{3.342370in}}%
\pgfpathlineto{\pgfqpoint{4.448429in}{3.341896in}}%
\pgfpathlineto{\pgfqpoint{4.456880in}{3.341939in}}%
\pgfpathlineto{\pgfqpoint{4.465331in}{3.342499in}}%
\pgfpathlineto{\pgfqpoint{4.473782in}{3.342456in}}%
\pgfpathlineto{\pgfqpoint{4.482233in}{3.340602in}}%
\pgfpathlineto{\pgfqpoint{4.490684in}{3.339394in}}%
\pgfpathlineto{\pgfqpoint{4.499135in}{3.340343in}}%
\pgfpathlineto{\pgfqpoint{4.507586in}{3.339696in}}%
\pgfpathlineto{\pgfqpoint{4.516037in}{3.339782in}}%
\pgfpathlineto{\pgfqpoint{4.524488in}{3.336979in}}%
\pgfpathlineto{\pgfqpoint{4.532939in}{3.339869in}}%
\pgfpathlineto{\pgfqpoint{4.541390in}{3.337583in}}%
\pgfpathlineto{\pgfqpoint{4.549841in}{3.338877in}}%
\pgfpathlineto{\pgfqpoint{4.558291in}{3.336203in}}%
\pgfpathlineto{\pgfqpoint{4.566742in}{3.337109in}}%
\pgfpathlineto{\pgfqpoint{4.575193in}{3.336246in}}%
\pgfpathlineto{\pgfqpoint{4.600546in}{3.335513in}}%
\pgfpathlineto{\pgfqpoint{4.608997in}{3.336591in}}%
\pgfpathlineto{\pgfqpoint{4.617448in}{3.336634in}}%
\pgfpathlineto{\pgfqpoint{4.634350in}{3.335642in}}%
\pgfpathlineto{\pgfqpoint{4.642801in}{3.333270in}}%
\pgfpathlineto{\pgfqpoint{4.651252in}{3.333874in}}%
\pgfpathlineto{\pgfqpoint{4.659703in}{3.332235in}}%
\pgfpathlineto{\pgfqpoint{4.668154in}{3.332580in}}%
\pgfpathlineto{\pgfqpoint{4.676605in}{3.331761in}}%
\pgfpathlineto{\pgfqpoint{4.685056in}{3.332451in}}%
\pgfpathlineto{\pgfqpoint{4.693506in}{3.331890in}}%
\pgfpathlineto{\pgfqpoint{4.701957in}{3.332278in}}%
\pgfpathlineto{\pgfqpoint{4.710408in}{3.329001in}}%
\pgfpathlineto{\pgfqpoint{4.718859in}{3.329907in}}%
\pgfpathlineto{\pgfqpoint{4.727310in}{3.329734in}}%
\pgfpathlineto{\pgfqpoint{4.735761in}{3.330122in}}%
\pgfpathlineto{\pgfqpoint{4.752663in}{3.329303in}}%
\pgfpathlineto{\pgfqpoint{4.761114in}{3.326974in}}%
\pgfpathlineto{\pgfqpoint{4.769565in}{3.328828in}}%
\pgfpathlineto{\pgfqpoint{4.778016in}{3.329087in}}%
\pgfpathlineto{\pgfqpoint{4.786467in}{3.326155in}}%
\pgfpathlineto{\pgfqpoint{4.794918in}{3.327707in}}%
\pgfpathlineto{\pgfqpoint{4.803369in}{3.326543in}}%
\pgfpathlineto{\pgfqpoint{4.811820in}{3.326715in}}%
\pgfpathlineto{\pgfqpoint{4.820271in}{3.328052in}}%
\pgfpathlineto{\pgfqpoint{4.828721in}{3.325982in}}%
\pgfpathlineto{\pgfqpoint{4.837172in}{3.325551in}}%
\pgfpathlineto{\pgfqpoint{4.845623in}{3.324386in}}%
\pgfpathlineto{\pgfqpoint{4.854074in}{3.322963in}}%
\pgfpathlineto{\pgfqpoint{4.862525in}{3.324602in}}%
\pgfpathlineto{\pgfqpoint{4.879427in}{3.323653in}}%
\pgfpathlineto{\pgfqpoint{4.887878in}{3.321928in}}%
\pgfpathlineto{\pgfqpoint{4.896329in}{3.323308in}}%
\pgfpathlineto{\pgfqpoint{4.904780in}{3.322403in}}%
\pgfpathlineto{\pgfqpoint{4.913231in}{3.322748in}}%
\pgfpathlineto{\pgfqpoint{4.921682in}{3.322273in}}%
\pgfpathlineto{\pgfqpoint{4.930133in}{3.321497in}}%
\pgfpathlineto{\pgfqpoint{4.938584in}{3.322101in}}%
\pgfpathlineto{\pgfqpoint{4.947035in}{3.323955in}}%
\pgfpathlineto{\pgfqpoint{4.955486in}{3.321885in}}%
\pgfpathlineto{\pgfqpoint{4.963936in}{3.320462in}}%
\pgfpathlineto{\pgfqpoint{4.980838in}{3.321583in}}%
\pgfpathlineto{\pgfqpoint{4.989289in}{3.319384in}}%
\pgfpathlineto{\pgfqpoint{4.997740in}{3.320850in}}%
\pgfpathlineto{\pgfqpoint{5.006191in}{3.320548in}}%
\pgfpathlineto{\pgfqpoint{5.014642in}{3.321109in}}%
\pgfpathlineto{\pgfqpoint{5.023093in}{3.319125in}}%
\pgfpathlineto{\pgfqpoint{5.031544in}{3.320333in}}%
\pgfpathlineto{\pgfqpoint{5.039995in}{3.319082in}}%
\pgfpathlineto{\pgfqpoint{5.048446in}{3.318564in}}%
\pgfpathlineto{\pgfqpoint{5.056897in}{3.318435in}}%
\pgfpathlineto{\pgfqpoint{5.065348in}{3.320074in}}%
\pgfpathlineto{\pgfqpoint{5.073799in}{3.316494in}}%
\pgfpathlineto{\pgfqpoint{5.082250in}{3.319298in}}%
\pgfpathlineto{\pgfqpoint{5.090701in}{3.316667in}}%
\pgfpathlineto{\pgfqpoint{5.099151in}{3.315934in}}%
\pgfpathlineto{\pgfqpoint{5.107602in}{3.317918in}}%
\pgfpathlineto{\pgfqpoint{5.116053in}{3.317141in}}%
\pgfpathlineto{\pgfqpoint{5.124504in}{3.317314in}}%
\pgfpathlineto{\pgfqpoint{5.132955in}{3.316796in}}%
\pgfpathlineto{\pgfqpoint{5.141406in}{3.318004in}}%
\pgfpathlineto{\pgfqpoint{5.149857in}{3.316063in}}%
\pgfpathlineto{\pgfqpoint{5.158308in}{3.316020in}}%
\pgfpathlineto{\pgfqpoint{5.166759in}{3.314985in}}%
\pgfpathlineto{\pgfqpoint{5.183661in}{3.314726in}}%
\pgfpathlineto{\pgfqpoint{5.192112in}{3.316106in}}%
\pgfpathlineto{\pgfqpoint{5.200563in}{3.314554in}}%
\pgfpathlineto{\pgfqpoint{5.209014in}{3.315503in}}%
\pgfpathlineto{\pgfqpoint{5.217465in}{3.315201in}}%
\pgfpathlineto{\pgfqpoint{5.234366in}{3.313734in}}%
\pgfpathlineto{\pgfqpoint{5.251268in}{3.313950in}}%
\pgfpathlineto{\pgfqpoint{5.259719in}{3.313605in}}%
\pgfpathlineto{\pgfqpoint{5.268170in}{3.314036in}}%
\pgfpathlineto{\pgfqpoint{5.276621in}{3.313001in}}%
\pgfpathlineto{\pgfqpoint{5.285072in}{3.313562in}}%
\pgfpathlineto{\pgfqpoint{5.293523in}{3.312441in}}%
\pgfpathlineto{\pgfqpoint{5.301974in}{3.313821in}}%
\pgfpathlineto{\pgfqpoint{5.310425in}{3.313907in}}%
\pgfpathlineto{\pgfqpoint{5.318876in}{3.312225in}}%
\pgfpathlineto{\pgfqpoint{5.327327in}{3.313088in}}%
\pgfpathlineto{\pgfqpoint{5.335778in}{3.312311in}}%
\pgfpathlineto{\pgfqpoint{5.344229in}{3.311966in}}%
\pgfpathlineto{\pgfqpoint{5.352680in}{3.312225in}}%
\pgfpathlineto{\pgfqpoint{5.369581in}{3.311923in}}%
\pgfpathlineto{\pgfqpoint{5.378032in}{3.311147in}}%
\pgfpathlineto{\pgfqpoint{5.386483in}{3.312872in}}%
\pgfpathlineto{\pgfqpoint{5.394934in}{3.311837in}}%
\pgfpathlineto{\pgfqpoint{5.403385in}{3.311492in}}%
\pgfpathlineto{\pgfqpoint{5.411836in}{3.312096in}}%
\pgfpathlineto{\pgfqpoint{5.420287in}{3.311017in}}%
\pgfpathlineto{\pgfqpoint{5.428738in}{3.311535in}}%
\pgfpathlineto{\pgfqpoint{5.437189in}{3.310586in}}%
\pgfpathlineto{\pgfqpoint{5.445640in}{3.311449in}}%
\pgfpathlineto{\pgfqpoint{5.454091in}{3.310500in}}%
\pgfpathlineto{\pgfqpoint{5.454091in}{3.310500in}}%
\pgfusepath{stroke}%
\end{pgfscope}%
\begin{pgfscope}%
\pgfpathrectangle{\pgfqpoint{2.842751in}{3.262466in}}{\pgfqpoint{2.619790in}{0.944783in}}%
\pgfusepath{clip}%
\pgfsetrectcap%
\pgfsetroundjoin%
\pgfsetlinewidth{1.003750pt}%
\definecolor{currentstroke}{rgb}{0.000000,0.000000,0.000000}%
\pgfsetstrokecolor{currentstroke}%
\pgfsetdash{}{0pt}%
\pgfpathmoveto{\pgfqpoint{2.952613in}{3.262466in}}%
\pgfpathlineto{\pgfqpoint{2.952613in}{4.207249in}}%
\pgfusepath{stroke}%
\end{pgfscope}%
\begin{pgfscope}%
\pgfsetrectcap%
\pgfsetmiterjoin%
\pgfsetlinewidth{0.803000pt}%
\definecolor{currentstroke}{rgb}{0.000000,0.000000,0.000000}%
\pgfsetstrokecolor{currentstroke}%
\pgfsetdash{}{0pt}%
\pgfpathmoveto{\pgfqpoint{2.842751in}{3.262466in}}%
\pgfpathlineto{\pgfqpoint{2.842751in}{4.207249in}}%
\pgfusepath{stroke}%
\end{pgfscope}%
\begin{pgfscope}%
\pgfsetrectcap%
\pgfsetmiterjoin%
\pgfsetlinewidth{0.803000pt}%
\definecolor{currentstroke}{rgb}{0.000000,0.000000,0.000000}%
\pgfsetstrokecolor{currentstroke}%
\pgfsetdash{}{0pt}%
\pgfpathmoveto{\pgfqpoint{5.462542in}{3.262466in}}%
\pgfpathlineto{\pgfqpoint{5.462542in}{4.207249in}}%
\pgfusepath{stroke}%
\end{pgfscope}%
\begin{pgfscope}%
\pgfsetrectcap%
\pgfsetmiterjoin%
\pgfsetlinewidth{0.803000pt}%
\definecolor{currentstroke}{rgb}{0.000000,0.000000,0.000000}%
\pgfsetstrokecolor{currentstroke}%
\pgfsetdash{}{0pt}%
\pgfpathmoveto{\pgfqpoint{2.842751in}{3.262466in}}%
\pgfpathlineto{\pgfqpoint{5.462542in}{3.262466in}}%
\pgfusepath{stroke}%
\end{pgfscope}%
\begin{pgfscope}%
\pgfsetrectcap%
\pgfsetmiterjoin%
\pgfsetlinewidth{0.803000pt}%
\definecolor{currentstroke}{rgb}{0.000000,0.000000,0.000000}%
\pgfsetstrokecolor{currentstroke}%
\pgfsetdash{}{0pt}%
\pgfpathmoveto{\pgfqpoint{2.842751in}{4.207249in}}%
\pgfpathlineto{\pgfqpoint{5.462542in}{4.207249in}}%
\pgfusepath{stroke}%
\end{pgfscope}%
\begin{pgfscope}%
\pgfsetbuttcap%
\pgfsetmiterjoin%
\definecolor{currentfill}{rgb}{1.000000,1.000000,1.000000}%
\pgfsetfillcolor{currentfill}%
\pgfsetlinewidth{0.000000pt}%
\definecolor{currentstroke}{rgb}{0.000000,0.000000,0.000000}%
\pgfsetstrokecolor{currentstroke}%
\pgfsetstrokeopacity{0.000000}%
\pgfsetdash{}{0pt}%
\pgfpathmoveto{\pgfqpoint{0.557402in}{0.398088in}}%
\pgfpathlineto{\pgfqpoint{6.131424in}{0.398088in}}%
\pgfpathlineto{\pgfqpoint{6.131424in}{2.055602in}}%
\pgfpathlineto{\pgfqpoint{0.557402in}{2.055602in}}%
\pgfpathlineto{\pgfqpoint{0.557402in}{0.398088in}}%
\pgfpathclose%
\pgfusepath{fill}%
\end{pgfscope}%
\begin{pgfscope}%
\pgfsetbuttcap%
\pgfsetroundjoin%
\definecolor{currentfill}{rgb}{0.000000,0.000000,0.000000}%
\pgfsetfillcolor{currentfill}%
\pgfsetlinewidth{0.803000pt}%
\definecolor{currentstroke}{rgb}{0.000000,0.000000,0.000000}%
\pgfsetstrokecolor{currentstroke}%
\pgfsetdash{}{0pt}%
\pgfsys@defobject{currentmarker}{\pgfqpoint{0.000000in}{-0.048611in}}{\pgfqpoint{0.000000in}{0.000000in}}{%
\pgfpathmoveto{\pgfqpoint{0.000000in}{0.000000in}}%
\pgfpathlineto{\pgfqpoint{0.000000in}{-0.048611in}}%
\pgfusepath{stroke,fill}%
}%
\begin{pgfscope}%
\pgfsys@transformshift{0.929004in}{0.398088in}%
\pgfsys@useobject{currentmarker}{}%
\end{pgfscope}%
\end{pgfscope}%
\begin{pgfscope}%
\definecolor{textcolor}{rgb}{0.000000,0.000000,0.000000}%
\pgfsetstrokecolor{textcolor}%
\pgfsetfillcolor{textcolor}%
\pgftext[x=0.929004in,y=0.300866in,,top]{\color{textcolor}\rmfamily\fontsize{10.000000}{12.000000}\selectfont \(\displaystyle {0}\)}%
\end{pgfscope}%
\begin{pgfscope}%
\pgfsetbuttcap%
\pgfsetroundjoin%
\definecolor{currentfill}{rgb}{0.000000,0.000000,0.000000}%
\pgfsetfillcolor{currentfill}%
\pgfsetlinewidth{0.803000pt}%
\definecolor{currentstroke}{rgb}{0.000000,0.000000,0.000000}%
\pgfsetstrokecolor{currentstroke}%
\pgfsetdash{}{0pt}%
\pgfsys@defobject{currentmarker}{\pgfqpoint{0.000000in}{-0.048611in}}{\pgfqpoint{0.000000in}{0.000000in}}{%
\pgfpathmoveto{\pgfqpoint{0.000000in}{0.000000in}}%
\pgfpathlineto{\pgfqpoint{0.000000in}{-0.048611in}}%
\pgfusepath{stroke,fill}%
}%
\begin{pgfscope}%
\pgfsys@transformshift{1.672207in}{0.398088in}%
\pgfsys@useobject{currentmarker}{}%
\end{pgfscope}%
\end{pgfscope}%
\begin{pgfscope}%
\definecolor{textcolor}{rgb}{0.000000,0.000000,0.000000}%
\pgfsetstrokecolor{textcolor}%
\pgfsetfillcolor{textcolor}%
\pgftext[x=1.672207in,y=0.300866in,,top]{\color{textcolor}\rmfamily\fontsize{10.000000}{12.000000}\selectfont \(\displaystyle {100}\)}%
\end{pgfscope}%
\begin{pgfscope}%
\pgfsetbuttcap%
\pgfsetroundjoin%
\definecolor{currentfill}{rgb}{0.000000,0.000000,0.000000}%
\pgfsetfillcolor{currentfill}%
\pgfsetlinewidth{0.803000pt}%
\definecolor{currentstroke}{rgb}{0.000000,0.000000,0.000000}%
\pgfsetstrokecolor{currentstroke}%
\pgfsetdash{}{0pt}%
\pgfsys@defobject{currentmarker}{\pgfqpoint{0.000000in}{-0.048611in}}{\pgfqpoint{0.000000in}{0.000000in}}{%
\pgfpathmoveto{\pgfqpoint{0.000000in}{0.000000in}}%
\pgfpathlineto{\pgfqpoint{0.000000in}{-0.048611in}}%
\pgfusepath{stroke,fill}%
}%
\begin{pgfscope}%
\pgfsys@transformshift{2.415410in}{0.398088in}%
\pgfsys@useobject{currentmarker}{}%
\end{pgfscope}%
\end{pgfscope}%
\begin{pgfscope}%
\definecolor{textcolor}{rgb}{0.000000,0.000000,0.000000}%
\pgfsetstrokecolor{textcolor}%
\pgfsetfillcolor{textcolor}%
\pgftext[x=2.415410in,y=0.300866in,,top]{\color{textcolor}\rmfamily\fontsize{10.000000}{12.000000}\selectfont \(\displaystyle {200}\)}%
\end{pgfscope}%
\begin{pgfscope}%
\pgfsetbuttcap%
\pgfsetroundjoin%
\definecolor{currentfill}{rgb}{0.000000,0.000000,0.000000}%
\pgfsetfillcolor{currentfill}%
\pgfsetlinewidth{0.803000pt}%
\definecolor{currentstroke}{rgb}{0.000000,0.000000,0.000000}%
\pgfsetstrokecolor{currentstroke}%
\pgfsetdash{}{0pt}%
\pgfsys@defobject{currentmarker}{\pgfqpoint{0.000000in}{-0.048611in}}{\pgfqpoint{0.000000in}{0.000000in}}{%
\pgfpathmoveto{\pgfqpoint{0.000000in}{0.000000in}}%
\pgfpathlineto{\pgfqpoint{0.000000in}{-0.048611in}}%
\pgfusepath{stroke,fill}%
}%
\begin{pgfscope}%
\pgfsys@transformshift{3.158613in}{0.398088in}%
\pgfsys@useobject{currentmarker}{}%
\end{pgfscope}%
\end{pgfscope}%
\begin{pgfscope}%
\definecolor{textcolor}{rgb}{0.000000,0.000000,0.000000}%
\pgfsetstrokecolor{textcolor}%
\pgfsetfillcolor{textcolor}%
\pgftext[x=3.158613in,y=0.300866in,,top]{\color{textcolor}\rmfamily\fontsize{10.000000}{12.000000}\selectfont \(\displaystyle {300}\)}%
\end{pgfscope}%
\begin{pgfscope}%
\pgfsetbuttcap%
\pgfsetroundjoin%
\definecolor{currentfill}{rgb}{0.000000,0.000000,0.000000}%
\pgfsetfillcolor{currentfill}%
\pgfsetlinewidth{0.803000pt}%
\definecolor{currentstroke}{rgb}{0.000000,0.000000,0.000000}%
\pgfsetstrokecolor{currentstroke}%
\pgfsetdash{}{0pt}%
\pgfsys@defobject{currentmarker}{\pgfqpoint{0.000000in}{-0.048611in}}{\pgfqpoint{0.000000in}{0.000000in}}{%
\pgfpathmoveto{\pgfqpoint{0.000000in}{0.000000in}}%
\pgfpathlineto{\pgfqpoint{0.000000in}{-0.048611in}}%
\pgfusepath{stroke,fill}%
}%
\begin{pgfscope}%
\pgfsys@transformshift{3.901816in}{0.398088in}%
\pgfsys@useobject{currentmarker}{}%
\end{pgfscope}%
\end{pgfscope}%
\begin{pgfscope}%
\definecolor{textcolor}{rgb}{0.000000,0.000000,0.000000}%
\pgfsetstrokecolor{textcolor}%
\pgfsetfillcolor{textcolor}%
\pgftext[x=3.901816in,y=0.300866in,,top]{\color{textcolor}\rmfamily\fontsize{10.000000}{12.000000}\selectfont \(\displaystyle {400}\)}%
\end{pgfscope}%
\begin{pgfscope}%
\pgfsetbuttcap%
\pgfsetroundjoin%
\definecolor{currentfill}{rgb}{0.000000,0.000000,0.000000}%
\pgfsetfillcolor{currentfill}%
\pgfsetlinewidth{0.803000pt}%
\definecolor{currentstroke}{rgb}{0.000000,0.000000,0.000000}%
\pgfsetstrokecolor{currentstroke}%
\pgfsetdash{}{0pt}%
\pgfsys@defobject{currentmarker}{\pgfqpoint{0.000000in}{-0.048611in}}{\pgfqpoint{0.000000in}{0.000000in}}{%
\pgfpathmoveto{\pgfqpoint{0.000000in}{0.000000in}}%
\pgfpathlineto{\pgfqpoint{0.000000in}{-0.048611in}}%
\pgfusepath{stroke,fill}%
}%
\begin{pgfscope}%
\pgfsys@transformshift{4.645018in}{0.398088in}%
\pgfsys@useobject{currentmarker}{}%
\end{pgfscope}%
\end{pgfscope}%
\begin{pgfscope}%
\definecolor{textcolor}{rgb}{0.000000,0.000000,0.000000}%
\pgfsetstrokecolor{textcolor}%
\pgfsetfillcolor{textcolor}%
\pgftext[x=4.645018in,y=0.300866in,,top]{\color{textcolor}\rmfamily\fontsize{10.000000}{12.000000}\selectfont \(\displaystyle {500}\)}%
\end{pgfscope}%
\begin{pgfscope}%
\pgfsetbuttcap%
\pgfsetroundjoin%
\definecolor{currentfill}{rgb}{0.000000,0.000000,0.000000}%
\pgfsetfillcolor{currentfill}%
\pgfsetlinewidth{0.803000pt}%
\definecolor{currentstroke}{rgb}{0.000000,0.000000,0.000000}%
\pgfsetstrokecolor{currentstroke}%
\pgfsetdash{}{0pt}%
\pgfsys@defobject{currentmarker}{\pgfqpoint{0.000000in}{-0.048611in}}{\pgfqpoint{0.000000in}{0.000000in}}{%
\pgfpathmoveto{\pgfqpoint{0.000000in}{0.000000in}}%
\pgfpathlineto{\pgfqpoint{0.000000in}{-0.048611in}}%
\pgfusepath{stroke,fill}%
}%
\begin{pgfscope}%
\pgfsys@transformshift{5.388221in}{0.398088in}%
\pgfsys@useobject{currentmarker}{}%
\end{pgfscope}%
\end{pgfscope}%
\begin{pgfscope}%
\definecolor{textcolor}{rgb}{0.000000,0.000000,0.000000}%
\pgfsetstrokecolor{textcolor}%
\pgfsetfillcolor{textcolor}%
\pgftext[x=5.388221in,y=0.300866in,,top]{\color{textcolor}\rmfamily\fontsize{10.000000}{12.000000}\selectfont \(\displaystyle {600}\)}%
\end{pgfscope}%
\begin{pgfscope}%
\pgfsetbuttcap%
\pgfsetroundjoin%
\definecolor{currentfill}{rgb}{0.000000,0.000000,0.000000}%
\pgfsetfillcolor{currentfill}%
\pgfsetlinewidth{0.803000pt}%
\definecolor{currentstroke}{rgb}{0.000000,0.000000,0.000000}%
\pgfsetstrokecolor{currentstroke}%
\pgfsetdash{}{0pt}%
\pgfsys@defobject{currentmarker}{\pgfqpoint{0.000000in}{-0.048611in}}{\pgfqpoint{0.000000in}{0.000000in}}{%
\pgfpathmoveto{\pgfqpoint{0.000000in}{0.000000in}}%
\pgfpathlineto{\pgfqpoint{0.000000in}{-0.048611in}}%
\pgfusepath{stroke,fill}%
}%
\begin{pgfscope}%
\pgfsys@transformshift{6.131424in}{0.398088in}%
\pgfsys@useobject{currentmarker}{}%
\end{pgfscope}%
\end{pgfscope}%
\begin{pgfscope}%
\definecolor{textcolor}{rgb}{0.000000,0.000000,0.000000}%
\pgfsetstrokecolor{textcolor}%
\pgfsetfillcolor{textcolor}%
\pgftext[x=6.131424in,y=0.300866in,,top]{\color{textcolor}\rmfamily\fontsize{10.000000}{12.000000}\selectfont \(\displaystyle {700}\)}%
\end{pgfscope}%
\begin{pgfscope}%
\definecolor{textcolor}{rgb}{0.000000,0.000000,0.000000}%
\pgfsetstrokecolor{textcolor}%
\pgfsetfillcolor{textcolor}%
\pgftext[x=3.344413in,y=0.122655in,,top]{\color{textcolor}\rmfamily\fontsize{10.000000}{12.000000}\selectfont Gesamtwert eines Clusters \(\displaystyle W\) in \si{\adu}}%
\end{pgfscope}%
\begin{pgfscope}%
\pgfsetbuttcap%
\pgfsetroundjoin%
\definecolor{currentfill}{rgb}{0.000000,0.000000,0.000000}%
\pgfsetfillcolor{currentfill}%
\pgfsetlinewidth{0.803000pt}%
\definecolor{currentstroke}{rgb}{0.000000,0.000000,0.000000}%
\pgfsetstrokecolor{currentstroke}%
\pgfsetdash{}{0pt}%
\pgfsys@defobject{currentmarker}{\pgfqpoint{-0.048611in}{0.000000in}}{\pgfqpoint{-0.000000in}{0.000000in}}{%
\pgfpathmoveto{\pgfqpoint{-0.000000in}{0.000000in}}%
\pgfpathlineto{\pgfqpoint{-0.048611in}{0.000000in}}%
\pgfusepath{stroke,fill}%
}%
\begin{pgfscope}%
\pgfsys@transformshift{0.557402in}{0.473430in}%
\pgfsys@useobject{currentmarker}{}%
\end{pgfscope}%
\end{pgfscope}%
\begin{pgfscope}%
\definecolor{textcolor}{rgb}{0.000000,0.000000,0.000000}%
\pgfsetstrokecolor{textcolor}%
\pgfsetfillcolor{textcolor}%
\pgftext[x=0.282710in, y=0.425605in, left, base]{\color{textcolor}\rmfamily\fontsize{10.000000}{12.000000}\selectfont \num{0.0}}%
\end{pgfscope}%
\begin{pgfscope}%
\pgfsetbuttcap%
\pgfsetroundjoin%
\definecolor{currentfill}{rgb}{0.000000,0.000000,0.000000}%
\pgfsetfillcolor{currentfill}%
\pgfsetlinewidth{0.803000pt}%
\definecolor{currentstroke}{rgb}{0.000000,0.000000,0.000000}%
\pgfsetstrokecolor{currentstroke}%
\pgfsetdash{}{0pt}%
\pgfsys@defobject{currentmarker}{\pgfqpoint{-0.048611in}{0.000000in}}{\pgfqpoint{-0.000000in}{0.000000in}}{%
\pgfpathmoveto{\pgfqpoint{-0.000000in}{0.000000in}}%
\pgfpathlineto{\pgfqpoint{-0.048611in}{0.000000in}}%
\pgfusepath{stroke,fill}%
}%
\begin{pgfscope}%
\pgfsys@transformshift{0.557402in}{0.954730in}%
\pgfsys@useobject{currentmarker}{}%
\end{pgfscope}%
\end{pgfscope}%
\begin{pgfscope}%
\definecolor{textcolor}{rgb}{0.000000,0.000000,0.000000}%
\pgfsetstrokecolor{textcolor}%
\pgfsetfillcolor{textcolor}%
\pgftext[x=0.282710in, y=0.906906in, left, base]{\color{textcolor}\rmfamily\fontsize{10.000000}{12.000000}\selectfont \num{0.5}}%
\end{pgfscope}%
\begin{pgfscope}%
\pgfsetbuttcap%
\pgfsetroundjoin%
\definecolor{currentfill}{rgb}{0.000000,0.000000,0.000000}%
\pgfsetfillcolor{currentfill}%
\pgfsetlinewidth{0.803000pt}%
\definecolor{currentstroke}{rgb}{0.000000,0.000000,0.000000}%
\pgfsetstrokecolor{currentstroke}%
\pgfsetdash{}{0pt}%
\pgfsys@defobject{currentmarker}{\pgfqpoint{-0.048611in}{0.000000in}}{\pgfqpoint{-0.000000in}{0.000000in}}{%
\pgfpathmoveto{\pgfqpoint{-0.000000in}{0.000000in}}%
\pgfpathlineto{\pgfqpoint{-0.048611in}{0.000000in}}%
\pgfusepath{stroke,fill}%
}%
\begin{pgfscope}%
\pgfsys@transformshift{0.557402in}{1.436031in}%
\pgfsys@useobject{currentmarker}{}%
\end{pgfscope}%
\end{pgfscope}%
\begin{pgfscope}%
\definecolor{textcolor}{rgb}{0.000000,0.000000,0.000000}%
\pgfsetstrokecolor{textcolor}%
\pgfsetfillcolor{textcolor}%
\pgftext[x=0.282710in, y=1.388206in, left, base]{\color{textcolor}\rmfamily\fontsize{10.000000}{12.000000}\selectfont \num{1.0}}%
\end{pgfscope}%
\begin{pgfscope}%
\pgfsetbuttcap%
\pgfsetroundjoin%
\definecolor{currentfill}{rgb}{0.000000,0.000000,0.000000}%
\pgfsetfillcolor{currentfill}%
\pgfsetlinewidth{0.803000pt}%
\definecolor{currentstroke}{rgb}{0.000000,0.000000,0.000000}%
\pgfsetstrokecolor{currentstroke}%
\pgfsetdash{}{0pt}%
\pgfsys@defobject{currentmarker}{\pgfqpoint{-0.048611in}{0.000000in}}{\pgfqpoint{-0.000000in}{0.000000in}}{%
\pgfpathmoveto{\pgfqpoint{-0.000000in}{0.000000in}}%
\pgfpathlineto{\pgfqpoint{-0.048611in}{0.000000in}}%
\pgfusepath{stroke,fill}%
}%
\begin{pgfscope}%
\pgfsys@transformshift{0.557402in}{1.917331in}%
\pgfsys@useobject{currentmarker}{}%
\end{pgfscope}%
\end{pgfscope}%
\begin{pgfscope}%
\definecolor{textcolor}{rgb}{0.000000,0.000000,0.000000}%
\pgfsetstrokecolor{textcolor}%
\pgfsetfillcolor{textcolor}%
\pgftext[x=0.282710in, y=1.869507in, left, base]{\color{textcolor}\rmfamily\fontsize{10.000000}{12.000000}\selectfont \num{1.5}}%
\end{pgfscope}%
\begin{pgfscope}%
\definecolor{textcolor}{rgb}{0.000000,0.000000,0.000000}%
\pgfsetstrokecolor{textcolor}%
\pgfsetfillcolor{textcolor}%
\pgftext[x=0.227155in,y=1.226845in,,bottom,rotate=90.000000]{\color{textcolor}\rmfamily\fontsize{10.000000}{12.000000}\selectfont Clusterzahl}%
\end{pgfscope}%
\begin{pgfscope}%
\definecolor{textcolor}{rgb}{0.000000,0.000000,0.000000}%
\pgfsetstrokecolor{textcolor}%
\pgfsetfillcolor{textcolor}%
\pgftext[x=0.557402in,y=2.097268in,left,base]{\color{textcolor}\rmfamily\fontsize{10.000000}{12.000000}\selectfont \(\displaystyle \times{10^{9}}{}\)}%
\end{pgfscope}%
\begin{pgfscope}%
\pgfpathrectangle{\pgfqpoint{0.557402in}{0.398088in}}{\pgfqpoint{5.574022in}{1.657514in}}%
\pgfusepath{clip}%
\pgfsetrectcap%
\pgfsetroundjoin%
\pgfsetlinewidth{1.505625pt}%
\definecolor{currentstroke}{rgb}{0.121569,0.466667,0.705882}%
\pgfsetstrokecolor{currentstroke}%
\pgfsetdash{}{0pt}%
\pgfpathmoveto{\pgfqpoint{0.555402in}{0.473430in}}%
\pgfpathlineto{\pgfqpoint{0.921572in}{0.473440in}}%
\pgfpathlineto{\pgfqpoint{0.929004in}{1.976176in}}%
\pgfpathlineto{\pgfqpoint{0.936436in}{0.473442in}}%
\pgfpathlineto{\pgfqpoint{2.883627in}{0.473440in}}%
\pgfpathlineto{\pgfqpoint{6.133424in}{0.473430in}}%
\pgfpathlineto{\pgfqpoint{6.133424in}{0.473430in}}%
\pgfusepath{stroke}%
\end{pgfscope}%
\begin{pgfscope}%
\pgfpathrectangle{\pgfqpoint{0.557402in}{0.398088in}}{\pgfqpoint{5.574022in}{1.657514in}}%
\pgfusepath{clip}%
\pgfsetrectcap%
\pgfsetroundjoin%
\pgfsetlinewidth{1.505625pt}%
\definecolor{currentstroke}{rgb}{1.000000,0.498039,0.054902}%
\pgfsetstrokecolor{currentstroke}%
\pgfsetdash{}{0pt}%
\pgfpathmoveto{\pgfqpoint{0.555402in}{0.473510in}}%
\pgfpathlineto{\pgfqpoint{0.921572in}{0.473522in}}%
\pgfpathlineto{\pgfqpoint{0.929004in}{1.980260in}}%
\pgfpathlineto{\pgfqpoint{0.936436in}{0.473523in}}%
\pgfpathlineto{\pgfqpoint{4.251121in}{0.473431in}}%
\pgfpathlineto{\pgfqpoint{6.133424in}{0.473430in}}%
\pgfpathlineto{\pgfqpoint{6.133424in}{0.473430in}}%
\pgfusepath{stroke}%
\end{pgfscope}%
\begin{pgfscope}%
\pgfsetrectcap%
\pgfsetmiterjoin%
\pgfsetlinewidth{0.803000pt}%
\definecolor{currentstroke}{rgb}{0.000000,0.000000,0.000000}%
\pgfsetstrokecolor{currentstroke}%
\pgfsetdash{}{0pt}%
\pgfpathmoveto{\pgfqpoint{0.557402in}{0.398088in}}%
\pgfpathlineto{\pgfqpoint{0.557402in}{2.055602in}}%
\pgfusepath{stroke}%
\end{pgfscope}%
\begin{pgfscope}%
\pgfsetrectcap%
\pgfsetmiterjoin%
\pgfsetlinewidth{0.803000pt}%
\definecolor{currentstroke}{rgb}{0.000000,0.000000,0.000000}%
\pgfsetstrokecolor{currentstroke}%
\pgfsetdash{}{0pt}%
\pgfpathmoveto{\pgfqpoint{6.131424in}{0.398088in}}%
\pgfpathlineto{\pgfqpoint{6.131424in}{2.055602in}}%
\pgfusepath{stroke}%
\end{pgfscope}%
\begin{pgfscope}%
\pgfsetrectcap%
\pgfsetmiterjoin%
\pgfsetlinewidth{0.803000pt}%
\definecolor{currentstroke}{rgb}{0.000000,0.000000,0.000000}%
\pgfsetstrokecolor{currentstroke}%
\pgfsetdash{}{0pt}%
\pgfpathmoveto{\pgfqpoint{0.557402in}{0.398088in}}%
\pgfpathlineto{\pgfqpoint{6.131424in}{0.398088in}}%
\pgfusepath{stroke}%
\end{pgfscope}%
\begin{pgfscope}%
\pgfsetrectcap%
\pgfsetmiterjoin%
\pgfsetlinewidth{0.803000pt}%
\definecolor{currentstroke}{rgb}{0.000000,0.000000,0.000000}%
\pgfsetstrokecolor{currentstroke}%
\pgfsetdash{}{0pt}%
\pgfpathmoveto{\pgfqpoint{0.557402in}{2.055602in}}%
\pgfpathlineto{\pgfqpoint{6.131424in}{2.055602in}}%
\pgfusepath{stroke}%
\end{pgfscope}%
\begin{pgfscope}%
\definecolor{textcolor}{rgb}{0.000000,0.000000,0.000000}%
\pgfsetstrokecolor{textcolor}%
\pgfsetfillcolor{textcolor}%
\pgftext[x=0.000000in,y=2.221353in,left,base]{\color{textcolor}\rmfamily\fontsize{10.000000}{12.000000}\selectfont (c)}%
\end{pgfscope}%
\begin{pgfscope}%
\pgfpathrectangle{\pgfqpoint{0.557402in}{0.398088in}}{\pgfqpoint{5.574022in}{1.657514in}}%
\pgfusepath{clip}%
\pgfsetbuttcap%
\pgfsetmiterjoin%
\pgfsetlinewidth{1.003750pt}%
\definecolor{currentstroke}{rgb}{0.000000,0.000000,0.000000}%
\pgfsetstrokecolor{currentstroke}%
\pgfsetstrokeopacity{0.500000}%
\pgfsetdash{}{0pt}%
\pgfpathmoveto{\pgfqpoint{2.891060in}{0.473430in}}%
\pgfpathlineto{\pgfqpoint{5.194989in}{0.473430in}}%
\pgfpathlineto{\pgfqpoint{5.194989in}{0.473437in}}%
\pgfpathlineto{\pgfqpoint{2.891060in}{0.473437in}}%
\pgfpathlineto{\pgfqpoint{2.891060in}{0.473430in}}%
\pgfpathclose%
\pgfusepath{stroke}%
\end{pgfscope}%
\begin{pgfscope}%
\pgfsetroundcap%
\pgfsetroundjoin%
\pgfsetlinewidth{1.003750pt}%
\definecolor{currentstroke}{rgb}{0.000000,0.000000,0.000000}%
\pgfsetstrokecolor{currentstroke}%
\pgfsetstrokeopacity{0.500000}%
\pgfsetdash{}{0pt}%
\pgfpathmoveto{\pgfqpoint{2.842751in}{0.812466in}}%
\pgfpathquadraticcurveto{\pgfqpoint{2.866905in}{0.642948in}}{\pgfqpoint{2.891060in}{0.473430in}}%
\pgfusepath{stroke}%
\end{pgfscope}%
\begin{pgfscope}%
\pgfsetroundcap%
\pgfsetroundjoin%
\pgfsetlinewidth{1.003750pt}%
\definecolor{currentstroke}{rgb}{0.000000,0.000000,0.000000}%
\pgfsetstrokecolor{currentstroke}%
\pgfsetstrokeopacity{0.500000}%
\pgfsetdash{}{0pt}%
\pgfpathmoveto{\pgfqpoint{5.462542in}{0.812466in}}%
\pgfpathquadraticcurveto{\pgfqpoint{5.328765in}{0.642948in}}{\pgfqpoint{5.194989in}{0.473430in}}%
\pgfusepath{stroke}%
\end{pgfscope}%
\begin{pgfscope}%
\pgfsetbuttcap%
\pgfsetmiterjoin%
\definecolor{currentfill}{rgb}{1.000000,1.000000,1.000000}%
\pgfsetfillcolor{currentfill}%
\pgfsetlinewidth{0.000000pt}%
\definecolor{currentstroke}{rgb}{0.000000,0.000000,0.000000}%
\pgfsetstrokecolor{currentstroke}%
\pgfsetstrokeopacity{0.000000}%
\pgfsetdash{}{0pt}%
\pgfpathmoveto{\pgfqpoint{2.842751in}{0.812466in}}%
\pgfpathlineto{\pgfqpoint{5.462542in}{0.812466in}}%
\pgfpathlineto{\pgfqpoint{5.462542in}{1.757249in}}%
\pgfpathlineto{\pgfqpoint{2.842751in}{1.757249in}}%
\pgfpathlineto{\pgfqpoint{2.842751in}{0.812466in}}%
\pgfpathclose%
\pgfusepath{fill}%
\end{pgfscope}%
\begin{pgfscope}%
\pgfsetbuttcap%
\pgfsetroundjoin%
\definecolor{currentfill}{rgb}{0.000000,0.000000,0.000000}%
\pgfsetfillcolor{currentfill}%
\pgfsetlinewidth{0.803000pt}%
\definecolor{currentstroke}{rgb}{0.000000,0.000000,0.000000}%
\pgfsetstrokecolor{currentstroke}%
\pgfsetdash{}{0pt}%
\pgfsys@defobject{currentmarker}{\pgfqpoint{0.000000in}{0.000000in}}{\pgfqpoint{0.000000in}{0.048611in}}{%
\pgfpathmoveto{\pgfqpoint{0.000000in}{0.000000in}}%
\pgfpathlineto{\pgfqpoint{0.000000in}{0.048611in}}%
\pgfusepath{stroke,fill}%
}%
\begin{pgfscope}%
\pgfsys@transformshift{2.952613in}{1.757249in}%
\pgfsys@useobject{currentmarker}{}%
\end{pgfscope}%
\end{pgfscope}%
\begin{pgfscope}%
\definecolor{textcolor}{rgb}{0.000000,0.000000,0.000000}%
\pgfsetstrokecolor{textcolor}%
\pgfsetfillcolor{textcolor}%
\pgftext[x=2.952613in,y=1.854471in,,bottom]{\color{textcolor}\rmfamily\fontsize{10.000000}{12.000000}\selectfont \(\displaystyle {277}\)}%
\end{pgfscope}%
\begin{pgfscope}%
\pgfsetbuttcap%
\pgfsetroundjoin%
\definecolor{currentfill}{rgb}{0.000000,0.000000,0.000000}%
\pgfsetfillcolor{currentfill}%
\pgfsetlinewidth{0.803000pt}%
\definecolor{currentstroke}{rgb}{0.000000,0.000000,0.000000}%
\pgfsetstrokecolor{currentstroke}%
\pgfsetdash{}{0pt}%
\pgfsys@defobject{currentmarker}{\pgfqpoint{0.000000in}{0.000000in}}{\pgfqpoint{0.000000in}{0.048611in}}{%
\pgfpathmoveto{\pgfqpoint{0.000000in}{0.000000in}}%
\pgfpathlineto{\pgfqpoint{0.000000in}{0.048611in}}%
\pgfusepath{stroke,fill}%
}%
\begin{pgfscope}%
\pgfsys@transformshift{3.992079in}{1.757249in}%
\pgfsys@useobject{currentmarker}{}%
\end{pgfscope}%
\end{pgfscope}%
\begin{pgfscope}%
\definecolor{textcolor}{rgb}{0.000000,0.000000,0.000000}%
\pgfsetstrokecolor{textcolor}%
\pgfsetfillcolor{textcolor}%
\pgftext[x=3.992079in,y=1.854471in,,bottom]{\color{textcolor}\rmfamily\fontsize{10.000000}{12.000000}\selectfont \(\displaystyle {400}\)}%
\end{pgfscope}%
\begin{pgfscope}%
\pgfsetbuttcap%
\pgfsetroundjoin%
\definecolor{currentfill}{rgb}{0.000000,0.000000,0.000000}%
\pgfsetfillcolor{currentfill}%
\pgfsetlinewidth{0.803000pt}%
\definecolor{currentstroke}{rgb}{0.000000,0.000000,0.000000}%
\pgfsetstrokecolor{currentstroke}%
\pgfsetdash{}{0pt}%
\pgfsys@defobject{currentmarker}{\pgfqpoint{0.000000in}{0.000000in}}{\pgfqpoint{0.000000in}{0.048611in}}{%
\pgfpathmoveto{\pgfqpoint{0.000000in}{0.000000in}}%
\pgfpathlineto{\pgfqpoint{0.000000in}{0.048611in}}%
\pgfusepath{stroke,fill}%
}%
\begin{pgfscope}%
\pgfsys@transformshift{4.837172in}{1.757249in}%
\pgfsys@useobject{currentmarker}{}%
\end{pgfscope}%
\end{pgfscope}%
\begin{pgfscope}%
\definecolor{textcolor}{rgb}{0.000000,0.000000,0.000000}%
\pgfsetstrokecolor{textcolor}%
\pgfsetfillcolor{textcolor}%
\pgftext[x=4.837172in,y=1.854471in,,bottom]{\color{textcolor}\rmfamily\fontsize{10.000000}{12.000000}\selectfont \(\displaystyle {500}\)}%
\end{pgfscope}%
\begin{pgfscope}%
\pgfsetbuttcap%
\pgfsetroundjoin%
\definecolor{currentfill}{rgb}{0.000000,0.000000,0.000000}%
\pgfsetfillcolor{currentfill}%
\pgfsetlinewidth{0.803000pt}%
\definecolor{currentstroke}{rgb}{0.000000,0.000000,0.000000}%
\pgfsetstrokecolor{currentstroke}%
\pgfsetdash{}{0pt}%
\pgfsys@defobject{currentmarker}{\pgfqpoint{0.000000in}{0.000000in}}{\pgfqpoint{0.048611in}{0.000000in}}{%
\pgfpathmoveto{\pgfqpoint{0.000000in}{0.000000in}}%
\pgfpathlineto{\pgfqpoint{0.048611in}{0.000000in}}%
\pgfusepath{stroke,fill}%
}%
\begin{pgfscope}%
\pgfsys@transformshift{5.462542in}{0.855411in}%
\pgfsys@useobject{currentmarker}{}%
\end{pgfscope}%
\end{pgfscope}%
\begin{pgfscope}%
\definecolor{textcolor}{rgb}{0.000000,0.000000,0.000000}%
\pgfsetstrokecolor{textcolor}%
\pgfsetfillcolor{textcolor}%
\pgftext[x=5.559764in, y=0.807583in, left, base]{\color{textcolor}\rmfamily\fontsize{10.000000}{12.000000}\selectfont \(\displaystyle {0}\)}%
\end{pgfscope}%
\begin{pgfscope}%
\pgfsetbuttcap%
\pgfsetroundjoin%
\definecolor{currentfill}{rgb}{0.000000,0.000000,0.000000}%
\pgfsetfillcolor{currentfill}%
\pgfsetlinewidth{0.803000pt}%
\definecolor{currentstroke}{rgb}{0.000000,0.000000,0.000000}%
\pgfsetstrokecolor{currentstroke}%
\pgfsetdash{}{0pt}%
\pgfsys@defobject{currentmarker}{\pgfqpoint{0.000000in}{0.000000in}}{\pgfqpoint{0.048611in}{0.000000in}}{%
\pgfpathmoveto{\pgfqpoint{0.000000in}{0.000000in}}%
\pgfpathlineto{\pgfqpoint{0.048611in}{0.000000in}}%
\pgfusepath{stroke,fill}%
}%
\begin{pgfscope}%
\pgfsys@transformshift{5.462542in}{1.176709in}%
\pgfsys@useobject{currentmarker}{}%
\end{pgfscope}%
\end{pgfscope}%
\begin{pgfscope}%
\definecolor{textcolor}{rgb}{0.000000,0.000000,0.000000}%
\pgfsetstrokecolor{textcolor}%
\pgfsetfillcolor{textcolor}%
\pgftext[x=5.559764in, y=1.128881in, left, base]{\color{textcolor}\rmfamily\fontsize{10.000000}{12.000000}\selectfont \(\displaystyle {2500}\)}%
\end{pgfscope}%
\begin{pgfscope}%
\pgfsetbuttcap%
\pgfsetroundjoin%
\definecolor{currentfill}{rgb}{0.000000,0.000000,0.000000}%
\pgfsetfillcolor{currentfill}%
\pgfsetlinewidth{0.803000pt}%
\definecolor{currentstroke}{rgb}{0.000000,0.000000,0.000000}%
\pgfsetstrokecolor{currentstroke}%
\pgfsetdash{}{0pt}%
\pgfsys@defobject{currentmarker}{\pgfqpoint{0.000000in}{0.000000in}}{\pgfqpoint{0.048611in}{0.000000in}}{%
\pgfpathmoveto{\pgfqpoint{0.000000in}{0.000000in}}%
\pgfpathlineto{\pgfqpoint{0.048611in}{0.000000in}}%
\pgfusepath{stroke,fill}%
}%
\begin{pgfscope}%
\pgfsys@transformshift{5.462542in}{1.498007in}%
\pgfsys@useobject{currentmarker}{}%
\end{pgfscope}%
\end{pgfscope}%
\begin{pgfscope}%
\definecolor{textcolor}{rgb}{0.000000,0.000000,0.000000}%
\pgfsetstrokecolor{textcolor}%
\pgfsetfillcolor{textcolor}%
\pgftext[x=5.559764in, y=1.450179in, left, base]{\color{textcolor}\rmfamily\fontsize{10.000000}{12.000000}\selectfont \(\displaystyle {5000}\)}%
\end{pgfscope}%
\begin{pgfscope}%
\pgfpathrectangle{\pgfqpoint{2.842751in}{0.812466in}}{\pgfqpoint{2.619790in}{0.944783in}}%
\pgfusepath{clip}%
\pgfsetrectcap%
\pgfsetroundjoin%
\pgfsetlinewidth{1.505625pt}%
\definecolor{currentstroke}{rgb}{0.121569,0.466667,0.705882}%
\pgfsetstrokecolor{currentstroke}%
\pgfsetdash{}{0pt}%
\pgfpathmoveto{\pgfqpoint{2.927261in}{1.714305in}}%
\pgfpathlineto{\pgfqpoint{2.935712in}{1.688986in}}%
\pgfpathlineto{\pgfqpoint{2.944163in}{1.641691in}}%
\pgfpathlineto{\pgfqpoint{2.952613in}{1.595681in}}%
\pgfpathlineto{\pgfqpoint{2.961064in}{1.584629in}}%
\pgfpathlineto{\pgfqpoint{2.969515in}{1.549543in}}%
\pgfpathlineto{\pgfqpoint{2.977966in}{1.523197in}}%
\pgfpathlineto{\pgfqpoint{2.986417in}{1.480400in}}%
\pgfpathlineto{\pgfqpoint{2.994868in}{1.465877in}}%
\pgfpathlineto{\pgfqpoint{3.003319in}{1.442358in}}%
\pgfpathlineto{\pgfqpoint{3.011770in}{1.418068in}}%
\pgfpathlineto{\pgfqpoint{3.020221in}{1.408429in}}%
\pgfpathlineto{\pgfqpoint{3.028672in}{1.363447in}}%
\pgfpathlineto{\pgfqpoint{3.037123in}{1.351238in}}%
\pgfpathlineto{\pgfqpoint{3.045574in}{1.326177in}}%
\pgfpathlineto{\pgfqpoint{3.054025in}{1.294947in}}%
\pgfpathlineto{\pgfqpoint{3.062476in}{1.292633in}}%
\pgfpathlineto{\pgfqpoint{3.070927in}{1.268215in}}%
\pgfpathlineto{\pgfqpoint{3.079378in}{1.246495in}}%
\pgfpathlineto{\pgfqpoint{3.087828in}{1.234928in}}%
\pgfpathlineto{\pgfqpoint{3.096279in}{1.203184in}}%
\pgfpathlineto{\pgfqpoint{3.104730in}{1.203569in}}%
\pgfpathlineto{\pgfqpoint{3.113181in}{1.185448in}}%
\pgfpathlineto{\pgfqpoint{3.121632in}{1.160516in}}%
\pgfpathlineto{\pgfqpoint{3.130083in}{1.153190in}}%
\pgfpathlineto{\pgfqpoint{3.138534in}{1.148435in}}%
\pgfpathlineto{\pgfqpoint{3.146985in}{1.118875in}}%
\pgfpathlineto{\pgfqpoint{3.155436in}{1.106152in}}%
\pgfpathlineto{\pgfqpoint{3.163887in}{1.102168in}}%
\pgfpathlineto{\pgfqpoint{3.172338in}{1.078906in}}%
\pgfpathlineto{\pgfqpoint{3.180789in}{1.085974in}}%
\pgfpathlineto{\pgfqpoint{3.189240in}{1.069010in}}%
\pgfpathlineto{\pgfqpoint{3.197691in}{1.062584in}}%
\pgfpathlineto{\pgfqpoint{3.206142in}{1.043949in}}%
\pgfpathlineto{\pgfqpoint{3.214593in}{1.043820in}}%
\pgfpathlineto{\pgfqpoint{3.223043in}{1.029169in}}%
\pgfpathlineto{\pgfqpoint{3.231494in}{1.021586in}}%
\pgfpathlineto{\pgfqpoint{3.239945in}{1.020301in}}%
\pgfpathlineto{\pgfqpoint{3.248396in}{1.007321in}}%
\pgfpathlineto{\pgfqpoint{3.256847in}{1.000381in}}%
\pgfpathlineto{\pgfqpoint{3.265298in}{0.993055in}}%
\pgfpathlineto{\pgfqpoint{3.273749in}{0.989328in}}%
\pgfpathlineto{\pgfqpoint{3.282200in}{0.990999in}}%
\pgfpathlineto{\pgfqpoint{3.290651in}{0.981745in}}%
\pgfpathlineto{\pgfqpoint{3.299102in}{0.966580in}}%
\pgfpathlineto{\pgfqpoint{3.307553in}{0.963239in}}%
\pgfpathlineto{\pgfqpoint{3.316004in}{0.957841in}}%
\pgfpathlineto{\pgfqpoint{3.324455in}{0.962468in}}%
\pgfpathlineto{\pgfqpoint{3.332906in}{0.955656in}}%
\pgfpathlineto{\pgfqpoint{3.341357in}{0.939848in}}%
\pgfpathlineto{\pgfqpoint{3.349808in}{0.943447in}}%
\pgfpathlineto{\pgfqpoint{3.358258in}{0.939720in}}%
\pgfpathlineto{\pgfqpoint{3.366709in}{0.934707in}}%
\pgfpathlineto{\pgfqpoint{3.375160in}{0.930338in}}%
\pgfpathlineto{\pgfqpoint{3.383611in}{0.929567in}}%
\pgfpathlineto{\pgfqpoint{3.392062in}{0.929567in}}%
\pgfpathlineto{\pgfqpoint{3.400513in}{0.920056in}}%
\pgfpathlineto{\pgfqpoint{3.408964in}{0.915558in}}%
\pgfpathlineto{\pgfqpoint{3.417415in}{0.912345in}}%
\pgfpathlineto{\pgfqpoint{3.425866in}{0.912474in}}%
\pgfpathlineto{\pgfqpoint{3.434317in}{0.907718in}}%
\pgfpathlineto{\pgfqpoint{3.442768in}{0.910417in}}%
\pgfpathlineto{\pgfqpoint{3.451219in}{0.904891in}}%
\pgfpathlineto{\pgfqpoint{3.459670in}{0.902449in}}%
\pgfpathlineto{\pgfqpoint{3.468121in}{0.902706in}}%
\pgfpathlineto{\pgfqpoint{3.476572in}{0.896409in}}%
\pgfpathlineto{\pgfqpoint{3.485023in}{0.891011in}}%
\pgfpathlineto{\pgfqpoint{3.493473in}{0.894866in}}%
\pgfpathlineto{\pgfqpoint{3.501924in}{0.888312in}}%
\pgfpathlineto{\pgfqpoint{3.510375in}{0.885870in}}%
\pgfpathlineto{\pgfqpoint{3.518826in}{0.886256in}}%
\pgfpathlineto{\pgfqpoint{3.527277in}{0.883685in}}%
\pgfpathlineto{\pgfqpoint{3.535728in}{0.885356in}}%
\pgfpathlineto{\pgfqpoint{3.544179in}{0.883814in}}%
\pgfpathlineto{\pgfqpoint{3.552630in}{0.882015in}}%
\pgfpathlineto{\pgfqpoint{3.561081in}{0.878030in}}%
\pgfpathlineto{\pgfqpoint{3.569532in}{0.879187in}}%
\pgfpathlineto{\pgfqpoint{3.577983in}{0.876488in}}%
\pgfpathlineto{\pgfqpoint{3.586434in}{0.875589in}}%
\pgfpathlineto{\pgfqpoint{3.594885in}{0.872633in}}%
\pgfpathlineto{\pgfqpoint{3.603336in}{0.874689in}}%
\pgfpathlineto{\pgfqpoint{3.611787in}{0.873918in}}%
\pgfpathlineto{\pgfqpoint{3.620238in}{0.874818in}}%
\pgfpathlineto{\pgfqpoint{3.628688in}{0.873661in}}%
\pgfpathlineto{\pgfqpoint{3.637139in}{0.871090in}}%
\pgfpathlineto{\pgfqpoint{3.645590in}{0.868906in}}%
\pgfpathlineto{\pgfqpoint{3.654041in}{0.868520in}}%
\pgfpathlineto{\pgfqpoint{3.662492in}{0.865436in}}%
\pgfpathlineto{\pgfqpoint{3.670943in}{0.867877in}}%
\pgfpathlineto{\pgfqpoint{3.679394in}{0.865179in}}%
\pgfpathlineto{\pgfqpoint{3.687845in}{0.865436in}}%
\pgfpathlineto{\pgfqpoint{3.696296in}{0.866464in}}%
\pgfpathlineto{\pgfqpoint{3.704747in}{0.866592in}}%
\pgfpathlineto{\pgfqpoint{3.721649in}{0.864279in}}%
\pgfpathlineto{\pgfqpoint{3.730100in}{0.863508in}}%
\pgfpathlineto{\pgfqpoint{3.738551in}{0.862223in}}%
\pgfpathlineto{\pgfqpoint{3.747002in}{0.863251in}}%
\pgfpathlineto{\pgfqpoint{3.755452in}{0.861194in}}%
\pgfpathlineto{\pgfqpoint{3.763903in}{0.861966in}}%
\pgfpathlineto{\pgfqpoint{3.772354in}{0.862351in}}%
\pgfpathlineto{\pgfqpoint{3.780805in}{0.861066in}}%
\pgfpathlineto{\pgfqpoint{3.797707in}{0.861066in}}%
\pgfpathlineto{\pgfqpoint{3.806158in}{0.859781in}}%
\pgfpathlineto{\pgfqpoint{3.814609in}{0.860166in}}%
\pgfpathlineto{\pgfqpoint{3.823060in}{0.859909in}}%
\pgfpathlineto{\pgfqpoint{3.831511in}{0.858753in}}%
\pgfpathlineto{\pgfqpoint{3.839962in}{0.860295in}}%
\pgfpathlineto{\pgfqpoint{3.848413in}{0.859010in}}%
\pgfpathlineto{\pgfqpoint{3.856864in}{0.857210in}}%
\pgfpathlineto{\pgfqpoint{3.865315in}{0.859267in}}%
\pgfpathlineto{\pgfqpoint{3.873766in}{0.859652in}}%
\pgfpathlineto{\pgfqpoint{3.882217in}{0.858239in}}%
\pgfpathlineto{\pgfqpoint{3.890667in}{0.857853in}}%
\pgfpathlineto{\pgfqpoint{3.899118in}{0.857724in}}%
\pgfpathlineto{\pgfqpoint{3.907569in}{0.857982in}}%
\pgfpathlineto{\pgfqpoint{3.916020in}{0.857082in}}%
\pgfpathlineto{\pgfqpoint{3.924471in}{0.857596in}}%
\pgfpathlineto{\pgfqpoint{3.932922in}{0.857853in}}%
\pgfpathlineto{\pgfqpoint{3.941373in}{0.857082in}}%
\pgfpathlineto{\pgfqpoint{3.949824in}{0.856825in}}%
\pgfpathlineto{\pgfqpoint{3.958275in}{0.857339in}}%
\pgfpathlineto{\pgfqpoint{3.966726in}{0.856953in}}%
\pgfpathlineto{\pgfqpoint{3.975177in}{0.857210in}}%
\pgfpathlineto{\pgfqpoint{4.000530in}{0.856311in}}%
\pgfpathlineto{\pgfqpoint{4.025882in}{0.856696in}}%
\pgfpathlineto{\pgfqpoint{4.034333in}{0.856054in}}%
\pgfpathlineto{\pgfqpoint{4.042784in}{0.856054in}}%
\pgfpathlineto{\pgfqpoint{4.051235in}{0.856953in}}%
\pgfpathlineto{\pgfqpoint{4.093490in}{0.855668in}}%
\pgfpathlineto{\pgfqpoint{4.101941in}{0.856311in}}%
\pgfpathlineto{\pgfqpoint{4.110392in}{0.855668in}}%
\pgfpathlineto{\pgfqpoint{4.118843in}{0.856054in}}%
\pgfpathlineto{\pgfqpoint{4.135745in}{0.856182in}}%
\pgfpathlineto{\pgfqpoint{4.144196in}{0.855540in}}%
\pgfpathlineto{\pgfqpoint{4.152647in}{0.855925in}}%
\pgfpathlineto{\pgfqpoint{4.169548in}{0.855540in}}%
\pgfpathlineto{\pgfqpoint{4.177999in}{0.855797in}}%
\pgfpathlineto{\pgfqpoint{4.194901in}{0.855411in}}%
\pgfpathlineto{\pgfqpoint{4.203352in}{0.855925in}}%
\pgfpathlineto{\pgfqpoint{4.211803in}{0.855540in}}%
\pgfpathlineto{\pgfqpoint{4.220254in}{0.855797in}}%
\pgfpathlineto{\pgfqpoint{4.228705in}{0.855540in}}%
\pgfpathlineto{\pgfqpoint{4.237156in}{0.855668in}}%
\pgfpathlineto{\pgfqpoint{4.245607in}{0.855540in}}%
\pgfpathlineto{\pgfqpoint{4.254058in}{0.855925in}}%
\pgfpathlineto{\pgfqpoint{4.287862in}{0.855540in}}%
\pgfpathlineto{\pgfqpoint{4.313214in}{0.855668in}}%
\pgfpathlineto{\pgfqpoint{4.321665in}{0.855797in}}%
\pgfpathlineto{\pgfqpoint{4.330116in}{0.855411in}}%
\pgfpathlineto{\pgfqpoint{4.338567in}{0.855540in}}%
\pgfpathlineto{\pgfqpoint{4.355469in}{0.855411in}}%
\pgfpathlineto{\pgfqpoint{4.389273in}{0.855411in}}%
\pgfpathlineto{\pgfqpoint{4.397724in}{0.855668in}}%
\pgfpathlineto{\pgfqpoint{4.406175in}{0.855411in}}%
\pgfpathlineto{\pgfqpoint{4.685056in}{0.855411in}}%
\pgfpathlineto{\pgfqpoint{4.693506in}{0.855668in}}%
\pgfpathlineto{\pgfqpoint{4.701957in}{0.855411in}}%
\pgfpathlineto{\pgfqpoint{5.454091in}{0.855411in}}%
\pgfpathlineto{\pgfqpoint{5.454091in}{0.855411in}}%
\pgfusepath{stroke}%
\end{pgfscope}%
\begin{pgfscope}%
\pgfpathrectangle{\pgfqpoint{2.842751in}{0.812466in}}{\pgfqpoint{2.619790in}{0.944783in}}%
\pgfusepath{clip}%
\pgfsetrectcap%
\pgfsetroundjoin%
\pgfsetlinewidth{1.505625pt}%
\definecolor{currentstroke}{rgb}{1.000000,0.498039,0.054902}%
\pgfsetstrokecolor{currentstroke}%
\pgfsetdash{}{0pt}%
\pgfpathmoveto{\pgfqpoint{2.927261in}{1.647346in}}%
\pgfpathlineto{\pgfqpoint{2.935712in}{1.654158in}}%
\pgfpathlineto{\pgfqpoint{2.944163in}{1.624598in}}%
\pgfpathlineto{\pgfqpoint{2.952613in}{1.606220in}}%
\pgfpathlineto{\pgfqpoint{2.961064in}{1.577560in}}%
\pgfpathlineto{\pgfqpoint{2.969515in}{1.577560in}}%
\pgfpathlineto{\pgfqpoint{2.977966in}{1.566893in}}%
\pgfpathlineto{\pgfqpoint{2.986417in}{1.558925in}}%
\pgfpathlineto{\pgfqpoint{2.994868in}{1.528851in}}%
\pgfpathlineto{\pgfqpoint{3.003319in}{1.511116in}}%
\pgfpathlineto{\pgfqpoint{3.011770in}{1.504304in}}%
\pgfpathlineto{\pgfqpoint{3.020221in}{1.511373in}}%
\pgfpathlineto{\pgfqpoint{3.028672in}{1.483741in}}%
\pgfpathlineto{\pgfqpoint{3.037123in}{1.483613in}}%
\pgfpathlineto{\pgfqpoint{3.045574in}{1.481813in}}%
\pgfpathlineto{\pgfqpoint{3.054025in}{1.459708in}}%
\pgfpathlineto{\pgfqpoint{3.062476in}{1.438759in}}%
\pgfpathlineto{\pgfqpoint{3.070927in}{1.436189in}}%
\pgfpathlineto{\pgfqpoint{3.079378in}{1.432205in}}%
\pgfpathlineto{\pgfqpoint{3.087828in}{1.400589in}}%
\pgfpathlineto{\pgfqpoint{3.096279in}{1.398019in}}%
\pgfpathlineto{\pgfqpoint{3.104730in}{1.384396in}}%
\pgfpathlineto{\pgfqpoint{3.113181in}{1.395706in}}%
\pgfpathlineto{\pgfqpoint{3.121632in}{1.372958in}}%
\pgfpathlineto{\pgfqpoint{3.130083in}{1.365889in}}%
\pgfpathlineto{\pgfqpoint{3.138534in}{1.353680in}}%
\pgfpathlineto{\pgfqpoint{3.146985in}{1.352395in}}%
\pgfpathlineto{\pgfqpoint{3.155436in}{1.343270in}}%
\pgfpathlineto{\pgfqpoint{3.163887in}{1.339157in}}%
\pgfpathlineto{\pgfqpoint{3.172338in}{1.322064in}}%
\pgfpathlineto{\pgfqpoint{3.180789in}{1.317951in}}%
\pgfpathlineto{\pgfqpoint{3.189240in}{1.322450in}}%
\pgfpathlineto{\pgfqpoint{3.197691in}{1.303429in}}%
\pgfpathlineto{\pgfqpoint{3.206142in}{1.300987in}}%
\pgfpathlineto{\pgfqpoint{3.214593in}{1.279524in}}%
\pgfpathlineto{\pgfqpoint{3.223043in}{1.287621in}}%
\pgfpathlineto{\pgfqpoint{3.231494in}{1.262560in}}%
\pgfpathlineto{\pgfqpoint{3.239945in}{1.261532in}}%
\pgfpathlineto{\pgfqpoint{3.248396in}{1.264359in}}%
\pgfpathlineto{\pgfqpoint{3.256847in}{1.251764in}}%
\pgfpathlineto{\pgfqpoint{3.265298in}{1.240454in}}%
\pgfpathlineto{\pgfqpoint{3.273749in}{1.243667in}}%
\pgfpathlineto{\pgfqpoint{3.282200in}{1.235828in}}%
\pgfpathlineto{\pgfqpoint{3.290651in}{1.242896in}}%
\pgfpathlineto{\pgfqpoint{3.299102in}{1.241097in}}%
\pgfpathlineto{\pgfqpoint{3.307553in}{1.210895in}}%
\pgfpathlineto{\pgfqpoint{3.316004in}{1.206397in}}%
\pgfpathlineto{\pgfqpoint{3.324455in}{1.205240in}}%
\pgfpathlineto{\pgfqpoint{3.332906in}{1.217964in}}%
\pgfpathlineto{\pgfqpoint{3.341357in}{1.200228in}}%
\pgfpathlineto{\pgfqpoint{3.349808in}{1.199071in}}%
\pgfpathlineto{\pgfqpoint{3.358258in}{1.190718in}}%
\pgfpathlineto{\pgfqpoint{3.366709in}{1.189432in}}%
\pgfpathlineto{\pgfqpoint{3.375160in}{1.188661in}}%
\pgfpathlineto{\pgfqpoint{3.383611in}{1.173625in}}%
\pgfpathlineto{\pgfqpoint{3.392062in}{1.162829in}}%
\pgfpathlineto{\pgfqpoint{3.400513in}{1.184677in}}%
\pgfpathlineto{\pgfqpoint{3.408964in}{1.154732in}}%
\pgfpathlineto{\pgfqpoint{3.417415in}{1.161672in}}%
\pgfpathlineto{\pgfqpoint{3.425866in}{1.164628in}}%
\pgfpathlineto{\pgfqpoint{3.434317in}{1.159873in}}%
\pgfpathlineto{\pgfqpoint{3.442768in}{1.142394in}}%
\pgfpathlineto{\pgfqpoint{3.451219in}{1.151648in}}%
\pgfpathlineto{\pgfqpoint{3.459670in}{1.147021in}}%
\pgfpathlineto{\pgfqpoint{3.468121in}{1.139310in}}%
\pgfpathlineto{\pgfqpoint{3.476572in}{1.137125in}}%
\pgfpathlineto{\pgfqpoint{3.485023in}{1.125173in}}%
\pgfpathlineto{\pgfqpoint{3.493473in}{1.131984in}}%
\pgfpathlineto{\pgfqpoint{3.501924in}{1.123374in}}%
\pgfpathlineto{\pgfqpoint{3.510375in}{1.119646in}}%
\pgfpathlineto{\pgfqpoint{3.518826in}{1.121831in}}%
\pgfpathlineto{\pgfqpoint{3.527277in}{1.113735in}}%
\pgfpathlineto{\pgfqpoint{3.535728in}{1.107823in}}%
\pgfpathlineto{\pgfqpoint{3.544179in}{1.103967in}}%
\pgfpathlineto{\pgfqpoint{3.552630in}{1.114506in}}%
\pgfpathlineto{\pgfqpoint{3.561081in}{1.096256in}}%
\pgfpathlineto{\pgfqpoint{3.569532in}{1.103710in}}%
\pgfpathlineto{\pgfqpoint{3.577983in}{1.110779in}}%
\pgfpathlineto{\pgfqpoint{3.586434in}{1.083019in}}%
\pgfpathlineto{\pgfqpoint{3.594885in}{1.097027in}}%
\pgfpathlineto{\pgfqpoint{3.603336in}{1.090730in}}%
\pgfpathlineto{\pgfqpoint{3.611787in}{1.082376in}}%
\pgfpathlineto{\pgfqpoint{3.620238in}{1.084561in}}%
\pgfpathlineto{\pgfqpoint{3.628688in}{1.085846in}}%
\pgfpathlineto{\pgfqpoint{3.637139in}{1.081862in}}%
\pgfpathlineto{\pgfqpoint{3.645590in}{1.080191in}}%
\pgfpathlineto{\pgfqpoint{3.654041in}{1.075307in}}%
\pgfpathlineto{\pgfqpoint{3.662492in}{1.076850in}}%
\pgfpathlineto{\pgfqpoint{3.670943in}{1.072608in}}%
\pgfpathlineto{\pgfqpoint{3.679394in}{1.063869in}}%
\pgfpathlineto{\pgfqpoint{3.687845in}{1.070938in}}%
\pgfpathlineto{\pgfqpoint{3.696296in}{1.068496in}}%
\pgfpathlineto{\pgfqpoint{3.704747in}{1.063741in}}%
\pgfpathlineto{\pgfqpoint{3.713198in}{1.059885in}}%
\pgfpathlineto{\pgfqpoint{3.721649in}{1.058600in}}%
\pgfpathlineto{\pgfqpoint{3.730100in}{1.058857in}}%
\pgfpathlineto{\pgfqpoint{3.738551in}{1.053588in}}%
\pgfpathlineto{\pgfqpoint{3.747002in}{1.059885in}}%
\pgfpathlineto{\pgfqpoint{3.755452in}{1.049218in}}%
\pgfpathlineto{\pgfqpoint{3.763903in}{1.048318in}}%
\pgfpathlineto{\pgfqpoint{3.772354in}{1.046905in}}%
\pgfpathlineto{\pgfqpoint{3.780805in}{1.045748in}}%
\pgfpathlineto{\pgfqpoint{3.789256in}{1.047547in}}%
\pgfpathlineto{\pgfqpoint{3.797707in}{1.035723in}}%
\pgfpathlineto{\pgfqpoint{3.806158in}{1.047676in}}%
\pgfpathlineto{\pgfqpoint{3.814609in}{1.043178in}}%
\pgfpathlineto{\pgfqpoint{3.823060in}{1.049089in}}%
\pgfpathlineto{\pgfqpoint{3.831511in}{1.032639in}}%
\pgfpathlineto{\pgfqpoint{3.839962in}{1.031482in}}%
\pgfpathlineto{\pgfqpoint{3.848413in}{1.037780in}}%
\pgfpathlineto{\pgfqpoint{3.856864in}{1.037266in}}%
\pgfpathlineto{\pgfqpoint{3.865315in}{1.023000in}}%
\pgfpathlineto{\pgfqpoint{3.873766in}{1.023129in}}%
\pgfpathlineto{\pgfqpoint{3.882217in}{1.022872in}}%
\pgfpathlineto{\pgfqpoint{3.890667in}{1.030968in}}%
\pgfpathlineto{\pgfqpoint{3.899118in}{1.024157in}}%
\pgfpathlineto{\pgfqpoint{3.907569in}{1.023900in}}%
\pgfpathlineto{\pgfqpoint{3.916020in}{1.029683in}}%
\pgfpathlineto{\pgfqpoint{3.924471in}{1.015932in}}%
\pgfpathlineto{\pgfqpoint{3.932922in}{1.016317in}}%
\pgfpathlineto{\pgfqpoint{3.941373in}{1.017088in}}%
\pgfpathlineto{\pgfqpoint{3.949824in}{1.008220in}}%
\pgfpathlineto{\pgfqpoint{3.958275in}{1.007192in}}%
\pgfpathlineto{\pgfqpoint{3.966726in}{1.008477in}}%
\pgfpathlineto{\pgfqpoint{3.975177in}{1.012333in}}%
\pgfpathlineto{\pgfqpoint{3.983628in}{1.017345in}}%
\pgfpathlineto{\pgfqpoint{3.992079in}{1.007835in}}%
\pgfpathlineto{\pgfqpoint{4.000530in}{1.003979in}}%
\pgfpathlineto{\pgfqpoint{4.008981in}{1.008220in}}%
\pgfpathlineto{\pgfqpoint{4.017432in}{1.011305in}}%
\pgfpathlineto{\pgfqpoint{4.025882in}{1.011690in}}%
\pgfpathlineto{\pgfqpoint{4.034333in}{1.004108in}}%
\pgfpathlineto{\pgfqpoint{4.042784in}{1.000381in}}%
\pgfpathlineto{\pgfqpoint{4.051235in}{0.996011in}}%
\pgfpathlineto{\pgfqpoint{4.059686in}{1.003722in}}%
\pgfpathlineto{\pgfqpoint{4.068137in}{1.003979in}}%
\pgfpathlineto{\pgfqpoint{4.076588in}{0.994212in}}%
\pgfpathlineto{\pgfqpoint{4.085039in}{1.003337in}}%
\pgfpathlineto{\pgfqpoint{4.093490in}{0.996525in}}%
\pgfpathlineto{\pgfqpoint{4.101941in}{0.999096in}}%
\pgfpathlineto{\pgfqpoint{4.110392in}{0.990613in}}%
\pgfpathlineto{\pgfqpoint{4.118843in}{0.991770in}}%
\pgfpathlineto{\pgfqpoint{4.127294in}{0.985215in}}%
\pgfpathlineto{\pgfqpoint{4.135745in}{0.989071in}}%
\pgfpathlineto{\pgfqpoint{4.144196in}{0.990485in}}%
\pgfpathlineto{\pgfqpoint{4.152647in}{0.998710in}}%
\pgfpathlineto{\pgfqpoint{4.161097in}{0.993698in}}%
\pgfpathlineto{\pgfqpoint{4.169548in}{0.984059in}}%
\pgfpathlineto{\pgfqpoint{4.177999in}{0.984701in}}%
\pgfpathlineto{\pgfqpoint{4.186450in}{0.984187in}}%
\pgfpathlineto{\pgfqpoint{4.194901in}{0.986629in}}%
\pgfpathlineto{\pgfqpoint{4.203352in}{0.982002in}}%
\pgfpathlineto{\pgfqpoint{4.211803in}{0.983930in}}%
\pgfpathlineto{\pgfqpoint{4.220254in}{0.981360in}}%
\pgfpathlineto{\pgfqpoint{4.228705in}{0.978532in}}%
\pgfpathlineto{\pgfqpoint{4.237156in}{0.972106in}}%
\pgfpathlineto{\pgfqpoint{4.245607in}{0.973263in}}%
\pgfpathlineto{\pgfqpoint{4.254058in}{0.975834in}}%
\pgfpathlineto{\pgfqpoint{4.262509in}{0.969793in}}%
\pgfpathlineto{\pgfqpoint{4.270960in}{0.977247in}}%
\pgfpathlineto{\pgfqpoint{4.279411in}{0.976476in}}%
\pgfpathlineto{\pgfqpoint{4.287862in}{0.967094in}}%
\pgfpathlineto{\pgfqpoint{4.296312in}{0.976733in}}%
\pgfpathlineto{\pgfqpoint{4.304763in}{0.972492in}}%
\pgfpathlineto{\pgfqpoint{4.313214in}{0.965938in}}%
\pgfpathlineto{\pgfqpoint{4.321665in}{0.964267in}}%
\pgfpathlineto{\pgfqpoint{4.330116in}{0.969665in}}%
\pgfpathlineto{\pgfqpoint{4.338567in}{0.961825in}}%
\pgfpathlineto{\pgfqpoint{4.347018in}{0.961825in}}%
\pgfpathlineto{\pgfqpoint{4.355469in}{0.972749in}}%
\pgfpathlineto{\pgfqpoint{4.363920in}{0.967094in}}%
\pgfpathlineto{\pgfqpoint{4.372371in}{0.969279in}}%
\pgfpathlineto{\pgfqpoint{4.380822in}{0.959126in}}%
\pgfpathlineto{\pgfqpoint{4.389273in}{0.965552in}}%
\pgfpathlineto{\pgfqpoint{4.397724in}{0.959640in}}%
\pgfpathlineto{\pgfqpoint{4.406175in}{0.960283in}}%
\pgfpathlineto{\pgfqpoint{4.414626in}{0.958483in}}%
\pgfpathlineto{\pgfqpoint{4.423077in}{0.955270in}}%
\pgfpathlineto{\pgfqpoint{4.431527in}{0.958612in}}%
\pgfpathlineto{\pgfqpoint{4.439978in}{0.952700in}}%
\pgfpathlineto{\pgfqpoint{4.448429in}{0.966195in}}%
\pgfpathlineto{\pgfqpoint{4.456880in}{0.952700in}}%
\pgfpathlineto{\pgfqpoint{4.465331in}{0.954242in}}%
\pgfpathlineto{\pgfqpoint{4.473782in}{0.953214in}}%
\pgfpathlineto{\pgfqpoint{4.482233in}{0.955270in}}%
\pgfpathlineto{\pgfqpoint{4.490684in}{0.956042in}}%
\pgfpathlineto{\pgfqpoint{4.499135in}{0.946146in}}%
\pgfpathlineto{\pgfqpoint{4.507586in}{0.947816in}}%
\pgfpathlineto{\pgfqpoint{4.516037in}{0.949873in}}%
\pgfpathlineto{\pgfqpoint{4.524488in}{0.946917in}}%
\pgfpathlineto{\pgfqpoint{4.541390in}{0.943704in}}%
\pgfpathlineto{\pgfqpoint{4.549841in}{0.943961in}}%
\pgfpathlineto{\pgfqpoint{4.558291in}{0.938820in}}%
\pgfpathlineto{\pgfqpoint{4.566742in}{0.948459in}}%
\pgfpathlineto{\pgfqpoint{4.575193in}{0.943318in}}%
\pgfpathlineto{\pgfqpoint{4.583644in}{0.942933in}}%
\pgfpathlineto{\pgfqpoint{4.592095in}{0.937149in}}%
\pgfpathlineto{\pgfqpoint{4.600546in}{0.942933in}}%
\pgfpathlineto{\pgfqpoint{4.608997in}{0.942290in}}%
\pgfpathlineto{\pgfqpoint{4.617448in}{0.933808in}}%
\pgfpathlineto{\pgfqpoint{4.625899in}{0.938692in}}%
\pgfpathlineto{\pgfqpoint{4.634350in}{0.932780in}}%
\pgfpathlineto{\pgfqpoint{4.642801in}{0.943832in}}%
\pgfpathlineto{\pgfqpoint{4.651252in}{0.932523in}}%
\pgfpathlineto{\pgfqpoint{4.659703in}{0.933679in}}%
\pgfpathlineto{\pgfqpoint{4.668154in}{0.937021in}}%
\pgfpathlineto{\pgfqpoint{4.676605in}{0.933808in}}%
\pgfpathlineto{\pgfqpoint{4.685056in}{0.935736in}}%
\pgfpathlineto{\pgfqpoint{4.693506in}{0.933808in}}%
\pgfpathlineto{\pgfqpoint{4.701957in}{0.933294in}}%
\pgfpathlineto{\pgfqpoint{4.710408in}{0.931366in}}%
\pgfpathlineto{\pgfqpoint{4.718859in}{0.930338in}}%
\pgfpathlineto{\pgfqpoint{4.727310in}{0.929567in}}%
\pgfpathlineto{\pgfqpoint{4.735761in}{0.929824in}}%
\pgfpathlineto{\pgfqpoint{4.744212in}{0.933679in}}%
\pgfpathlineto{\pgfqpoint{4.752663in}{0.929952in}}%
\pgfpathlineto{\pgfqpoint{4.761114in}{0.932651in}}%
\pgfpathlineto{\pgfqpoint{4.769565in}{0.926097in}}%
\pgfpathlineto{\pgfqpoint{4.778016in}{0.928667in}}%
\pgfpathlineto{\pgfqpoint{4.786467in}{0.926739in}}%
\pgfpathlineto{\pgfqpoint{4.794918in}{0.926996in}}%
\pgfpathlineto{\pgfqpoint{4.803369in}{0.924297in}}%
\pgfpathlineto{\pgfqpoint{4.811820in}{0.925197in}}%
\pgfpathlineto{\pgfqpoint{4.820271in}{0.923783in}}%
\pgfpathlineto{\pgfqpoint{4.828721in}{0.920827in}}%
\pgfpathlineto{\pgfqpoint{4.837172in}{0.928539in}}%
\pgfpathlineto{\pgfqpoint{4.845623in}{0.924554in}}%
\pgfpathlineto{\pgfqpoint{4.854074in}{0.918000in}}%
\pgfpathlineto{\pgfqpoint{4.862525in}{0.915815in}}%
\pgfpathlineto{\pgfqpoint{4.870976in}{0.921984in}}%
\pgfpathlineto{\pgfqpoint{4.879427in}{0.922627in}}%
\pgfpathlineto{\pgfqpoint{4.887878in}{0.924297in}}%
\pgfpathlineto{\pgfqpoint{4.896329in}{0.921598in}}%
\pgfpathlineto{\pgfqpoint{4.904780in}{0.916201in}}%
\pgfpathlineto{\pgfqpoint{4.913231in}{0.920056in}}%
\pgfpathlineto{\pgfqpoint{4.921682in}{0.918257in}}%
\pgfpathlineto{\pgfqpoint{4.930133in}{0.913116in}}%
\pgfpathlineto{\pgfqpoint{4.938584in}{0.912859in}}%
\pgfpathlineto{\pgfqpoint{4.947035in}{0.916072in}}%
\pgfpathlineto{\pgfqpoint{4.955486in}{0.913245in}}%
\pgfpathlineto{\pgfqpoint{4.963936in}{0.914273in}}%
\pgfpathlineto{\pgfqpoint{4.972387in}{0.918000in}}%
\pgfpathlineto{\pgfqpoint{4.980838in}{0.915558in}}%
\pgfpathlineto{\pgfqpoint{4.989289in}{0.916972in}}%
\pgfpathlineto{\pgfqpoint{4.997740in}{0.911060in}}%
\pgfpathlineto{\pgfqpoint{5.006191in}{0.910417in}}%
\pgfpathlineto{\pgfqpoint{5.014642in}{0.911702in}}%
\pgfpathlineto{\pgfqpoint{5.023093in}{0.912731in}}%
\pgfpathlineto{\pgfqpoint{5.031544in}{0.910289in}}%
\pgfpathlineto{\pgfqpoint{5.039995in}{0.912217in}}%
\pgfpathlineto{\pgfqpoint{5.048446in}{0.902963in}}%
\pgfpathlineto{\pgfqpoint{5.056897in}{0.909775in}}%
\pgfpathlineto{\pgfqpoint{5.065348in}{0.902963in}}%
\pgfpathlineto{\pgfqpoint{5.073799in}{0.908232in}}%
\pgfpathlineto{\pgfqpoint{5.082250in}{0.904505in}}%
\pgfpathlineto{\pgfqpoint{5.090701in}{0.904762in}}%
\pgfpathlineto{\pgfqpoint{5.099151in}{0.901549in}}%
\pgfpathlineto{\pgfqpoint{5.107602in}{0.905534in}}%
\pgfpathlineto{\pgfqpoint{5.116053in}{0.908361in}}%
\pgfpathlineto{\pgfqpoint{5.124504in}{0.903991in}}%
\pgfpathlineto{\pgfqpoint{5.132955in}{0.905148in}}%
\pgfpathlineto{\pgfqpoint{5.141406in}{0.902321in}}%
\pgfpathlineto{\pgfqpoint{5.149857in}{0.901678in}}%
\pgfpathlineto{\pgfqpoint{5.158308in}{0.900521in}}%
\pgfpathlineto{\pgfqpoint{5.166759in}{0.900264in}}%
\pgfpathlineto{\pgfqpoint{5.175210in}{0.901807in}}%
\pgfpathlineto{\pgfqpoint{5.183661in}{0.899365in}}%
\pgfpathlineto{\pgfqpoint{5.192112in}{0.894481in}}%
\pgfpathlineto{\pgfqpoint{5.200563in}{0.901421in}}%
\pgfpathlineto{\pgfqpoint{5.209014in}{0.899236in}}%
\pgfpathlineto{\pgfqpoint{5.217465in}{0.896280in}}%
\pgfpathlineto{\pgfqpoint{5.225916in}{0.898079in}}%
\pgfpathlineto{\pgfqpoint{5.234366in}{0.899493in}}%
\pgfpathlineto{\pgfqpoint{5.242817in}{0.896023in}}%
\pgfpathlineto{\pgfqpoint{5.251268in}{0.892810in}}%
\pgfpathlineto{\pgfqpoint{5.259719in}{0.895895in}}%
\pgfpathlineto{\pgfqpoint{5.268170in}{0.901035in}}%
\pgfpathlineto{\pgfqpoint{5.276621in}{0.895509in}}%
\pgfpathlineto{\pgfqpoint{5.285072in}{0.890625in}}%
\pgfpathlineto{\pgfqpoint{5.293523in}{0.896023in}}%
\pgfpathlineto{\pgfqpoint{5.301974in}{0.894609in}}%
\pgfpathlineto{\pgfqpoint{5.310425in}{0.896409in}}%
\pgfpathlineto{\pgfqpoint{5.318876in}{0.893453in}}%
\pgfpathlineto{\pgfqpoint{5.327327in}{0.891911in}}%
\pgfpathlineto{\pgfqpoint{5.335778in}{0.894224in}}%
\pgfpathlineto{\pgfqpoint{5.344229in}{0.891782in}}%
\pgfpathlineto{\pgfqpoint{5.352680in}{0.890111in}}%
\pgfpathlineto{\pgfqpoint{5.361130in}{0.894352in}}%
\pgfpathlineto{\pgfqpoint{5.369581in}{0.893453in}}%
\pgfpathlineto{\pgfqpoint{5.378032in}{0.890625in}}%
\pgfpathlineto{\pgfqpoint{5.386483in}{0.894738in}}%
\pgfpathlineto{\pgfqpoint{5.394934in}{0.894095in}}%
\pgfpathlineto{\pgfqpoint{5.403385in}{0.889597in}}%
\pgfpathlineto{\pgfqpoint{5.411836in}{0.892939in}}%
\pgfpathlineto{\pgfqpoint{5.420287in}{0.891525in}}%
\pgfpathlineto{\pgfqpoint{5.428738in}{0.887027in}}%
\pgfpathlineto{\pgfqpoint{5.437189in}{0.890754in}}%
\pgfpathlineto{\pgfqpoint{5.445640in}{0.887412in}}%
\pgfpathlineto{\pgfqpoint{5.454091in}{0.887798in}}%
\pgfpathlineto{\pgfqpoint{5.454091in}{0.887798in}}%
\pgfusepath{stroke}%
\end{pgfscope}%
\begin{pgfscope}%
\pgfpathrectangle{\pgfqpoint{2.842751in}{0.812466in}}{\pgfqpoint{2.619790in}{0.944783in}}%
\pgfusepath{clip}%
\pgfsetrectcap%
\pgfsetroundjoin%
\pgfsetlinewidth{1.003750pt}%
\definecolor{currentstroke}{rgb}{0.000000,0.000000,0.000000}%
\pgfsetstrokecolor{currentstroke}%
\pgfsetdash{}{0pt}%
\pgfpathmoveto{\pgfqpoint{2.952613in}{0.812466in}}%
\pgfpathlineto{\pgfqpoint{2.952613in}{1.757249in}}%
\pgfusepath{stroke}%
\end{pgfscope}%
\begin{pgfscope}%
\pgfsetrectcap%
\pgfsetmiterjoin%
\pgfsetlinewidth{0.803000pt}%
\definecolor{currentstroke}{rgb}{0.000000,0.000000,0.000000}%
\pgfsetstrokecolor{currentstroke}%
\pgfsetdash{}{0pt}%
\pgfpathmoveto{\pgfqpoint{2.842751in}{0.812466in}}%
\pgfpathlineto{\pgfqpoint{2.842751in}{1.757249in}}%
\pgfusepath{stroke}%
\end{pgfscope}%
\begin{pgfscope}%
\pgfsetrectcap%
\pgfsetmiterjoin%
\pgfsetlinewidth{0.803000pt}%
\definecolor{currentstroke}{rgb}{0.000000,0.000000,0.000000}%
\pgfsetstrokecolor{currentstroke}%
\pgfsetdash{}{0pt}%
\pgfpathmoveto{\pgfqpoint{5.462542in}{0.812466in}}%
\pgfpathlineto{\pgfqpoint{5.462542in}{1.757249in}}%
\pgfusepath{stroke}%
\end{pgfscope}%
\begin{pgfscope}%
\pgfsetrectcap%
\pgfsetmiterjoin%
\pgfsetlinewidth{0.803000pt}%
\definecolor{currentstroke}{rgb}{0.000000,0.000000,0.000000}%
\pgfsetstrokecolor{currentstroke}%
\pgfsetdash{}{0pt}%
\pgfpathmoveto{\pgfqpoint{2.842751in}{0.812466in}}%
\pgfpathlineto{\pgfqpoint{5.462542in}{0.812466in}}%
\pgfusepath{stroke}%
\end{pgfscope}%
\begin{pgfscope}%
\pgfsetrectcap%
\pgfsetmiterjoin%
\pgfsetlinewidth{0.803000pt}%
\definecolor{currentstroke}{rgb}{0.000000,0.000000,0.000000}%
\pgfsetstrokecolor{currentstroke}%
\pgfsetdash{}{0pt}%
\pgfpathmoveto{\pgfqpoint{2.842751in}{1.757249in}}%
\pgfpathlineto{\pgfqpoint{5.462542in}{1.757249in}}%
\pgfusepath{stroke}%
\end{pgfscope}%
\end{pgfpicture}%
\makeatother%
\endgroup%
  und (c) $\mathbf{K}_3$ 
    \caption{Histogramme der (a) direkten Pixelwerte und der mit (b) $\mathbf{K}_2$-Clustering-Kerne bearbeiteten Pixelwerte. In den eingefügten Graphen sind die Bereiche detailliert dargestellt, wo sich die Schnittpunkte (vertikale Linien) von Dunkelbilder- und Streubilder-Histogramme befinden. Für jedes Histogramm werden \num{10000} Dunkelbilder sowie Streubilder verwendet.}
    \label{fig:no_pr_cl_2_histograms}
\end{figure}
\noindent
Allerdings können keine Peaks, die Photonen-Ereignissen entsprechen sollen, sowie in Histogramm von Pixelwerten in Abb. \ref{fig:no_pr_cl_2_histograms}a, als aus von den \qtyproduct{2 x 2}{\px}-geclusterten Pixelwerten in Abb. \ref{fig:no_pr_cl_2_histograms}b, erkannt werden. Allerdings können Intervalle beobachtet werden, wo die Zahl von Pixeln bzw. Clustern der Streubilder größer ist, als in Dunkelbildern. Von besonderem Interesse ist die Asymmetrie des ersten Peaks im Falle der Streubilder in Abb. \ref{fig:no_pr_cl_2_histograms}a.

\noindent
Der zweitwichtigste Faktor ist der konstante Offset, der sich aus Mittelung der Dunkelbilder ergibt und von jedem Streubild subtrahiert wird. Es scheint so zu sein, dass der statische Hintergrund sich im Laufe der Zeit verändert. Darüber hinaus hängt die Veränderung vom Photonenfluss ab. So wird ein zusätzlicher Offset $W_\Delta$ zu jedem Pixel addiert.

\noindent
Dieser Offset taucht sich wohl bei der Anwendung des Schwellenwert-Algorithmuses auf, scheint aber vernachlässigbar klein gegenüber dem einzelnen Pixelwert zu sein. Bei der Anwendung des Clustering-Algorithmuses mit dem \qtyproduct{2 x 2}{\px} vervierfacht sich der Offset $W_\Delta$ und erhöht die Zahl der fehldetektierten Photonen.

\noindent
Der Cluster-Kern kann in Bezug auf die vorliegende Gesamtladungsverteilung, die grundlegend zwischen zwei benachbarten Pixeln stattfindet, angepasst werden. Zum Beispiel können jeweils die Cluster-Kerne
\begin{equation}
    \mathbf{K}_{2\times1} = \begin{bmatrix}
1\\
1
\end{bmatrix}
\text{ bzw. }
    \mathbf{K}_{1\times2} = \begin{bmatrix}
1 & 1
\end{bmatrix}
\end{equation}
benutzt werden. So wird die Standardbweichung eines Clusters von $2\sigma_R$ bis $\sqrt{2}\sigma_R$ verringert. Ebenso wird der gesamte Offset nur $2W_\Delta$ statt $4W_\Delta$ sein. Der Offset $W_\Delta$ kann durch die häufigere Aufnahme von Dunkelbildern gesenkt werden.