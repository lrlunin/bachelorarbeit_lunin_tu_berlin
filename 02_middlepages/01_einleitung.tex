\chapter{Einleitung}
\label{text:einleitung}
Magnetische Effekte sind seit langem der Menschheit bekannt und spielten eine große Rolle bei der Entwicklung der menschlichen Zivilisation. So fanden diese Effekte ihre Anwendungen bereits im 11. Jahrhundert in Schiffsnavigation und machten bisher unvorstellbare Fernreisen möglich.

\noindent
Eine sprunghafte/großartige Entwicklung im Magnetismus Forschungsfeld fand im 19. Jahrhundert mit den Werken von Maxwell, Lorentz und Lenz statt. Somit wurde die Verwandtschaft der elektrischen und magnetischen Felder geklärt, die sich zunächst im 20. Jahrhundert auf quantenmechanischer Ebene in den Werken von Dirac, Bohr und Pauli aufgreifen ließ.

\noindent
Die digitale Revolution, allein die originale Erfindung eines Festplattenlaufwerkes, sei ohne Vorwissen zum Thema Magnetismus durchaus nicht möglich. Die Information dargestellt als binäre Folge wird in Form der alternierenden lokalen Magnetisierungen (\emph{magnetischen Domänen}) auf den drehenden Scheiben gespeichert. Der Wendepunkt bei den Datenspeicherungslösungen für Verbraucher und Unternehmen, und zwar die drastische Erhöhung der Datendichte, kam mit der Entdeckung des Effektes von Riesenmagnetowiderstand in den Dünnschichtsystemen. Für diese Entdeckung wurde man im Jahr 2007 mit dem Nobelpreis für Physik ausgezeichnet.

\noindent
Magnetische Domänen in Dünnschichtsystemen sind auch heute in der Festkörperphysik von großem Interesse. Erforscht werden die statischen und dynamischen Eigenschaften der Domänenbildung in Bezug auf diverse Bedingungen, wie Probentemperatur, angelegte Magnetfeldstärke und Energiezufuhr in Form von Laserpulsen. So sind die modernen Messmethoden und Messanlagen erforderlich.

\noindent
Es gibt mehrere Messverfahren, die den Zugang zur Forschung der magnetischen Effekten ermöglichen. Ich möchte allerdings auf diejenige fokussieren, deren die magneto-optischen Wechselwirkungen zugrunde liegen. Dieser Forschungsbereich ist jedoch sehr umfangreich und komplex. So beschränke ich mich erstmal auf Streuexperimente, um den roten Faden nicht zu verlieren.

\noindent
Zum Durchführen der Streuexperimentean an den mikroskopischen magnetischen Strukturen wird die Röntgenstrahlung verwendet. Die vorausgesetzten Anforderungen an die Kohärenz und Brillanz der Quelle können hauptsächlich nur mithilfe von speziellen Anlagen wie Synchrotron oder Freie-Elektronen-Laser erfüllt werden. Limitierte Kapazität, technische Komplexität und Einzigartigkeit solcher Anlagen limitiert die Zeit, während deren ein Experiment durchgeführt werden kann. Unter anderem tauchen die zusätzlichen Anforderungen in Form von zeitaufwendiger Einarbeitung und pausenlosem mehrtägigem Einsatz der Mitarbeiter auf.

\noindent
Darüber hinaus sind die Charakteristiken der genannten Anlagen nicht für alle Arten von Experimenten geeignet. Ein Röntgenpulsdauer, also die Zeitauflösung, am Synchrotron PETRA III ist ca. \SI{40}{\pico\second} und wird mit der Frequenz ca. \SIrange[range-units = single]{1}{500}{\mega\hertz} emittiert. So eine hohe Pulsfrequenz ist für die manchen Prozesse zu hoch und kann zur unerwünschten Veränderung oder eben zur Zerstörung der Probe führen.

\noindent
Eine Alternative wäre, diese Art von Experimenten im Labor durchzuführen. Eine laser-getriebene Röntgenquelle, die im Rahmen dieser Arbeit im Labor angewendet wurde, bietet jedoch eine signifikant höhere Zeitauflösung in Höhe von \SI{10}{\pico\second} und viel niedrigere Pulsfrequenz von \SI{100}{\hertz}, wodurch die dynamischen Messungen auf dem längeren Zeitintervall realisierbar sind. Nichtsdestotrotz hat so eine Röntgenquelle deutlich kleineren Photonenfluss und ist einem Synchrotron in Bezug auf die Kohärenz deutlich unterlegen.  

%weiche röntgenstrahlung, kleiner Flux -> kleiner Kontrast
\noindent
Das Ziel dieser Arbeit ist experimentell nachzuprüfen, ob die gegebene Röntgenquelle die Mindestvoraussetzungen an Kohärenz zum Beobachten der Kleinwinkelstreuung erfüllt, und so ein Mess- und Auswertungsverfahren anzubieten, in dem man einzelne Photonen in einem Streubild detektieren und die von Hintergrundrauschen des Bildes trennen kann, womit der Kontrast der Streubilder erhöht wird.

\noindent
Damit die geleistete Arbeit in der Zukunft auch als ein Lernmaterial dienen könnte und das Auswertungsverfahren reproduzierbar und nachvollziehbar wäre, werden alle benutzten \href{https://github.com/lrlunin/bachelorarbeit_python_notebooks_and_data/}{\color{blue}\texttt{.ipynb} Auswertungsscripte} sowie der \href{https://github.com/lrlunin/Bachelorarbeit/}{\color{blue}\LaTeX-Quellcode} dieses Dokuments in den entsprechenden GitHub-Repositories hochgeladen.   
%Eine besondere Herausforderung
% MOTIVATION SINGLE PHOTON COUNTING

% Idee: für die höhere x-ray Energien ist es möglich die Photonen von dem Rauschen mit einem Komporator (x > threshold) zu trennen. In dem Weichröntgenbereich ist es nicht möglich. Die Innovation liegt daran:  einzelne Photonen in den Aufnahmen mit ultrakurzen Belichtungszeiten zu erkennen, was in Verbindung mit einer hervorragend kurzen Auslesezeit des Detektors ermöglicht, die rauschenlosen Aufnahmen zu bekommen.

% Ich habe noch ganz vergessen, was zu deinem Inhaltsverzeichnis des Theorieteils deiner Arbeit zu sagen. Im Prinzip hast du die richtigen Themen erkannt. Ich würde etwas weiter einengen:
% 1. Resonante Röntgenstreuung von mesoskopischen magnetischen Texturen – bitte lass uns rechtzeitig darüber reden, was da genau rein soll.
% 2. Laser-getriebenes Instrument für resonante Streuung mit weichen Röntgenstrahlen – hier benötigst du fast ausschließlich das Paper von Daniel (https://www.osapublishing.org/optica/abstract.cfm?uri=optica-8-9-1237). Einfach die technischen Details zusammenfassen.
% 3. Der MÖNCH-Detektor – Hier bitte die Paper von den PSI-Leuten nutzen. Nicht zu viel allgemeines, lieber genau den MÖNCH-Detektor erklären.
% 4. Droplet Algorithmen – kurze Übersicht und dann genauer ein Algorithmus, den du ausprobieren möchtest.
% e Qualität bietet sich das Signal-Rausch-Verhältnis (engl. signal-to-noise-ratio, kurz: SNR) 


