\chapter{Einleitung}
\newacronym{fel}{FEL}{Freie-Elektronen-Lasern}
\label{text:einleitung}
% Magnetische Effekte sind seit langem der Menschheit bekannt und spielten eine große Rolle bei der Entwicklung der menschlichen Zivilisation. So fanden diese Effekte ihre Anwendungen bereits im 11. Jahrhundert in Schiffsnavigation und machten bisher unvorstellbare Fernreisen möglich.

% \noindent
% Eine sprunghafte/großartige Entwicklung im Magnetismus Forschungsfeld fand im 19. Jahrhundert mit den Werken von Maxwell, Lorentz und Lenz statt. Somit wurde die Verwandtschaft der elektrischen und magnetischen Felder geklärt, die sich zunächst im 20. Jahrhundert auf quantenmechanischer Ebene in den Werken von Dirac, Bohr und Pauli aufgreifen ließ.

% \noindent
% Die digitale Revolution, allein die originale Erfindung eines Festplattenlaufwerkes, sei ohne Vorwissen zum Thema Magnetismus durchaus nicht möglich. Die Information dargestellt als binäre Folge wird in Form der alternierenden lokalen Magnetisierungen (\emph{magnetische Domänen}) auf den drehenden Scheiben gespeichert. Der Wendepunkt bei den Datenspeicherungslösungen für Verbraucher und Unternehmen, und zwar die drastische Erhöhung der Datendichte, kam mit der Entdeckung des Effektes von Riesenmagnetowiderstand in den Dünnschichtsystemen. Für diese Entdeckung wurde man im Jahr 2007 mit dem Nobelpreis für Physik ausgezeichnet.
% \noindent
Die Weiterentwicklung der Theorie des Magnetismus im 20. Jahrhundert führte zur theoretischen Begründung und Entdeckung der Magnetisierungsverteilung in Festkörpern auf mikroskopischem Niveau. So können zum Beispiel alternierende lokale Magnetisierungen (\emph{magnetische Domänen}) entstehen. Zur Bildung der magnetischen Domänen und deren Charakteristiken tragen zahlreiche Faktoren bei, wie die Zusammensetzung und innere Struktur des Materials sowie Probentemperatur und äußeres angelegtes Magnetfeld.

\noindent
Heutzutage werden statische und dynamische Eigenschaften der Domänenbildung in Festkörpern und insbesondere in Dünnschichtsystemen in Bezug auf die Variation der äußeren Bedingungen erforscht. Solche Prozesse können innerhalb von einigen Pikosekunden ablaufen. Daher sind Messmethoden mit einer zeitlichen Auflösung im Bereich von einigen Femtosekunden gefragt \cite{pfau_ultrafast_2012}, um die Dynamik des Prozesses zu untersuchen.

\noindent
Unter anderem betragen die charakteristischen Domänengrößen je nach Dünnschichtsystem \SIrange[range-units = single]{50}{300}{\nano\meter}. So ist die räumliche Auflösung in Höhe von einigen Nanometern gefragt. Es gibt mehrere Messverfahren, die den Zugang zur Erforschung der magnetischen Effekte auf solchen Zeit- und Raumskalen ermöglichen. Ich möchte allerdings auf diejenigen fokussieren, denen die magneto-optischen Wechselwirkungen zugrunde liegen, und zwar auf die Beugungsmethoden. Solche Untersuchungsmethoden sind sehr praktisch, da sie den direkten physischen Kontakt mit der Probe vermeiden und dadurch eine bessere Anpassung der Experimentalbedingungen ermöglichen.

\noindent
Zum Durchführen der Beugungsexperimente an Dünnschichtsystemen wird weiche Röntgenstrahlung verwendet, weil die Resonanzenergien der oft benutzten Übergangs- und Seltenerdmetallatome wie Fe, Co, Pt und Gd in diesem Teil des Spektrums liegen. Außerdem ist eine Wellenlänge der Strahlung dieser Größenordnung für die gewünschte räumliche Auflösung erforderlich. Die vorausgesetzten Anforderungen an die Kohärenz und Brillanz der Quelle konnten bisher nur mithilfe von speziellen Anlagen wie Synchrotronstrahlungsquellen oder \gls{fel} erfüllt werden.  % kann ebenso für solche Messmethoden eingesetzt \cite{pfau_ultrafast_2012}

\noindent
Im Gegensatz zu der guten räumlichen Auflösung, welche die Synchrotronstrahlungsquellen anbieten, ist die zeitliche Auflösung für die Picosekunden-Prozesse nicht hoch genug. So beträgt die Röntgenpulsdauer, also die Zeitauflösung, am Synchrotron PETRA III \SI{44}{\pico\second} (RMS) bzw. \SI{100}{\pico\second} (FWHM) und wird mit einer Frequenz von ca. \SIrange[range-units = single]{5}{125}{\mega\hertz} emittiert \cite{noauthor_machine_nodate}.

\noindent
Die wesentlich kürzeren Pulsdauern sind mit einem \gls{fel} erreichbar. Als Beispiel einer der mo\-dern\-sten \gls{fel}s wird der European XFEL betrachtet. Die Pulsdauern können in unterschiedlichen Betriebsmodi von \SI{25}{\femto\second} bis zu \SI{10}{\femto\second} (FWHM) eingestellt werden \cite{tschentscher_photon_2017}. Die Pulsfrequenz beträgt \SI{4,5}{\mega\hertz}. So eine hohe Frequenz ist für manche Prozesse nicht erforderlich und kann zur unerwünschten Veränderung durch zu hohe Energiezufuhr oder eben zur Zerstörung der zu untersuchenden Probe führen. 

\noindent
Ein großer Nachteil liegt an den stark limitierten Messzeiten an \gls{fel}s, was die Anpassung der experimentellen Umgebung erschwert und die Möglichkeit der häufigen, regelmäßigen Messungen ausschließt.

\noindent
Eine bevorzugte Alternative wäre, diese Art von Experimenten im Labor durchzuführen. Im Rahmen dieser Arbeit wird eine Röntgenquelle benutzt, in der Röntgenstrahlung aus dem mit einem Picosekundenlaser angeregten Plasma erzeugt wird. Sie bietet eine mit \gls{fel} vergleichbare Zeitauflösung mit der Pulsdauer in Höhe von \SI{10}{\pico\second} (FWHM) und viel niedrigerer Pulsfrequenz von \SI{100}{\hertz} \cite{schick_laser-driven_2021}, wodurch die dynamischen Messungen auf dem längeren Zeitintervall ohne Probenzerstörung möglich sind. Nichtsdestotrotz hat so eine Röntgenquelle deutlich geringeren Photonenfluss, was viele technische Herausforderungen beim Mess- und Auswertungsverfahren mit sich bringt.

%weiche röntgenstrahlung, kleiner Flux -> kleiner Kontrast
\noindent
Das Ziel dieser Arbeit ist es, experimentell nachzuprüfen, ob ein Beugungsexperiment an einer Dünnschichtsystemprobe mit der gegebenen Röntgenquelle realisierbar ist. In Hinblick auf den extrem geringeren emittierten Photonenfluss der Röntgenquelle wird ein kommerzieller MÖNCH-Detektor eingesetzt \cite{ramilli-measurements-2017}, dessen hohe Bildrate synchronisierte Aufnahmen der gestreuten Photonen jedes einzelnen Röntgenpulses ermöglicht. So ist ein Mess- und Auswertungsverfahren zu entwickeln, in dem die einzelnen Photonen in der Einzelpulsaufnahme mit bestmöglicher Selektivität in Bezug auf das Detektorrauschen erkannt werden können.

\noindent
Alle benutzten \texttt{.ipynb} Auswertungsscripte sowie der \LaTeX-Quellcode dieses Dokuments werden in die entsprechenden GitHub-Repositories\footnote{\url{https://github.com/lrlunin/bachelorarbeit_python_notebooks_and_data}\newline\url{https://github.com/lrlunin/bachelorarbeit_lunin_tu_berlin}} hochgeladen.   
% 12, den allerersten Satz kann ich nicht so richtig deuten: "Die Entwicklung und Verständnis von Theorie des Magnetismus" ergibt in meinen Augen keinen rechten Sinn, den musst du mir mal in anderen Worten erklären
% 15, "Solche Prozesse können innerhalb von einigen Pikosekunden ablaufen" leicht verändert; vor "ablaufen" könnte man noch "Länge" oder "Dauer" einfügen
% 18 ganz unten, Anpassung an was?
% 21, letzter Satz: "hauptsächlich nur" sind widersprüchliche Worte. "hauptsächlich" ersetzen mit "bisher" (?)
% 21, FEL braucht ein n an den Lasern (ist ja im Plural)
% 27 ich weiß ja, dass LaTeX die Worttrennung automatisch macht, aber modern wird vor dem d getrennt
% Ich glaube, Lieteraturverweise mitten im (Teil-)Satz beeinträchtigen den Lesefluss - ich habe sie mal vor das nächste Satzzeichen gezogen.
