\chapter{Experimenteller Aufbau}
Die Streuung der Röntgenstrahlung an der Probe wird bei den kleinen Winkeln beobachtet. Die Probe im Strahlengang wird \SI{160}{\milli\meter} nach hinten von dem Fokuspunkt der \gls{rzp} gestellt/verschoben. Für die schärfere Abbildung des Streubildes soll der Detektor möglichst nah an der Probe gebracht werden. In der aktuellen Aufbaukonfiguration (s. Abb. \ref{fig:anlage}) heißt das, dass der Detektor an der gegenüberliegenden Wand ca. \SI{620}{\milli\meter} von der Probe entfernt befestigt wird.
\begin{figure}[H]
    \centering
    \includegraphics{images/aufbau/aufbau_empty.pdf}
    \caption{Die Skizze des Anlageaufbaus. Die Einzelbauteile sind farbig kodiert. In der Vakuumkammer (blau) wird der Druck in Höhe von ca. \SI{2.4e6}{\milli\bar} aufrechterhaltet. Der Probehalter (pink) ist aus Kupfer gemacht und kann die Rolle eines Wärmeleiters für die Probenkühlung spielen. Dazu ist der Probehalter mit der Kryoanlage angeschlossen. Der Probehalter lässt sich in. Der horizontale Spalt (fuchsia) dient zum Abschneiden des gewünschten Stahlsanduhrbereichs. Die Spaltlage sowie die Spaltbreite kann beliebig verstellt werden. Der MÖNCH-Detektor (rot) ist an der zum Strahlgang (gelb) gegenüberliegenden Wand der Vakuumkammer befestigt.}
    \label{fig:anlage}
\end{figure}
\noindent
Aufgrund des fast doppelt so hohen \gls{adu}-Wertes (siehe Gl. (\ref{eq:auselesewerte_fe_gd})) pro ein Photon an der $h\nu_\text{Gd, M5}$ Energie wurden die Messungen an dieser Photonenenrgie durchgeführt. 
!!! STREURING RADIUS !!!

\noindent
Um das ursprüngliche \gls{snr} der Aufnahmen zu erhöhen, sollen die Dunkelbilder während einer Messung (s. Abschnitt \ref{text:moench_theorie}) aufgenommen werden, die nachher aus den Aufnahmen subtrahiert werden. Es werden \num{10000} Dunkelbilder jede 30 Minuten aufgenommen, um die zeitliche Entwicklung des Hintergrundrauschens mitzunehmen. Ein kontinuierlicher Aufnahmevorgang der \gls{pxs} kann höchstens bis \SI{20}{\second} dauern. Die Pulsfrequenz $f_\text{\gls{pxs}}$ ist konstant und beträgt \SI{100}{\hertz}. So werden es \num{2000} Pulsen innerhalb eines \SI{20}{\second} Aufnahmevorgangs emittiert. Jedem Puls geht ein Triggersignal voraus, das \SI{1}{\milli\second} vor dem Puls generiert wird.

\noindent
Der \gls{moench03} hat eine einstellbare Verzögerung zwischen dem Aufnahmestart und dem Trigersignaleingang. Diese wird so konfiguriert, dass der \SI{10}{\femto\second} Röntgen-Puls der \gls{pxs} innerhalb des Belichtungszeitfensters $\tau$ vollständig/komplett hineinpasst. So könne die Belichtungszeit theoretisch weit bis dieselbe Großenordnung wie Pulslänge gesenkt. Das konnte aber praktisch nicht realisiert werden, weil die resultierende zeitliche Schwankung des Triggersignal-Generators, inneren Taktieren von \gls{moench03} Dutzende von \si{\nano\second} beträgt. So konnte die Belichtungszeit $\tau$ unter der Bedingung, dass jeder Puls komplett in jeder Aufnahme aufgenommen wird, höchstens bis auf \SI{1}{\micro\second} heruntergesetzt werden.

\noindent
Neben der Strahlung im Röntgenspektrum tritt bei der \gls{pxs} auch Strahlung im sichtbaren Spektrum, Streuung des einfallenden Laserlichts und Elektronenemission auf. Obwohl der Sensor von \gls{moench03} \SI{500}{\nano\meter} Al-Beschichtung habe, scheint sie am Rande dünner zu sein. So wurden die Pixel am Rande des Detektors ständig belichtet, selbst wenn der Röntgenstrahl in großer Entfernung von dem Sensorrand projiziert wurde. Es wurde daher zusätzlich im Strahlgang eine Mylar Folie mit \SI{200}{\nano\meter} Al-Beschichtung installiert, die die Intensität des sichtbaren und infraroten Lichtes deutlich abschwächte und dadurch den beobachteten Halo-Effekt beseitigte. Obwohl die Elektronen mit einem starken Magnet von dem Strahlgang großenteils abgelenkt werden, konnten einigen Elektronen auf dem \gls{moench03} detektiert werden. Der installierte Filter reduzierte ihre Zahl auch.
%
% \noindent
% Könnte die Belichtungszeit $\tau$ auf \SI{100}{\nano\second} gesetzt werden, ließ sich die Standardabweichung $\sigma$ von ca. \SI{21}{\adu} bis \SI{19}{\adu} senken. So würde das ursprüngliche \gls{snr} immer noch im Bereich von 7-\SI{10}{\decibel} liegen.

\noindent
Der Streuungseffekt der Proben ist energieselektiv und tritt am stärksten nur im schmalen Bereich um der Resonanzenergie $h\nu_{\text{Fe, L3}}$ bzw. $h\nu_{\text{Gd, M5}}$ auf. Diese Energien sind die weder Zielenergien von \gls{rzp} für Fe, noch für Gd. Aus diesem Grund muss der Strahl mit dem Fokuspunkt horizontal verschoben werden, damit die Resonanzenergie mittig an dem Detektor liegt und auf die Probe projiziert werden kann. Die Effizienz und die Abbildungsschärfe der \gls{rzp}, also die Energieauflösung pro Längeeinheit entlang der Energieachse, sind jedoch maximal an der Zielenergie und nehmen ständig mit der Entfernung von der Zielenergie ab.

\noindent
Im Abschnitt \ref{text:quelle_roentgen} wurde der detektierte Photonenfluss ohne Probe von \gls{pxs} angegeben. In Hinblick auf die niedrige erwartete Transsmissionsrate der Probe \textbf{DS220126} (Abb. \ref{fig:proben_vergleich_centered}) wurde der Photonenfluss, der schließlich nach der Streuung an der Probe detektiert wird, erneut experimentell bestimmt. Dieses Verfahren ist in Unterabschnitt \ref{text:streuung_counting} detalliert präsentiert.

\noindent
Der detektierte Photonenfluss mit der Probe konnte mit \SI{35(5)}{\photons} pro Puls abgeschätzt werden, wobei ca. \SI{30(5)}{\photons} in dem Direktstrahl liegen. Es werden insgesamt \numrange{75000}{100000} Pulsen aufgenommen, damit Kontrast und Schärfe im Maximabereich 1. Ordnung vom Streumuster ausreichend sind. 
% Die Streuung an der gegebenen Probe kann in erster Näherung als die Streuung an einem Gitter betrachtet werden. So ist die Intensitätsverteilung
% \begin{equation}
%     I(r, \varphi) \propto \sinc^2(R),
% \end{equation}
% wobei $r$ - Entfernung (Radius) vom Zentrum des Streubildes.

% \noindent
% So ist die gesamte Intensität innerhalb der 0. Ordnung
% \begin{equation}
%     I_{0. \text{Ordnung}} = \int_{0}^{2\pi}\int_{0}^{\pi}r\sinc^2(r) dr d\varphi \approx \num{7.66} 
% \end{equation}
% und 1. Ordnung
% \begin{equation}
%     I_{1. \text{Ordnung}} = \int_{0}^{2\pi}\int_{\pi}^{2\pi}r\sinc^2(r) dr d\varphi \approx \num{2.13} 
% \end{equation}
% Die Fläche sind jeweils
% \begin{equation}
%     A_{0. \text{Ordnung}} = \int_{0}^{2\pi}\int_{0}^{\pi}rdr d\varphi = \pi^3
% \end{equation}
% und
% \begin{equation}
%     A_{1. \text{Ordnung}} = \int_{0}^{2\pi}\int_{\pi}^{2\pi}rdr d\varphi = 3\pi^3
% \end{equation}
% D.h., dass die Intensitätdichten $\rho$
% sind
% \begin{equation}
%     \rho_{N. \text{Ordnung}} = \frac{I_{N. \text{Ordnung}}}{A_{N.}}
% \end{equation}
% \noindent
% Also
% \begin{equation}
%     \rho_{0. \text{Ordnung}}:\rho_{1. \text{Ordnung}} =  \frac{I_{0. \text{Ordnung}}}{A_{0. \text{Ordnung}}} : \frac{I_{1. \text{Ordnung}}}{A_{1. \text{Ordnung}}}   \approx 10:1
% \end{equation}
% Der erwartete Radius des Streuringes (1. Ordnung) kann mit der Formel für ein Gitter mit Kleinwinkelnäherung
% \begin{equation}
%     a_k = \frac{kl\lambda}{d},
% \end{equation}
% wobei $k$ - Ordnung des Maximums, $l$ - Abstand vom Gitter bis zum Detektor, $\lambda$ - Photonenwellenlänge und $d$ - Domänengröße. Die Photonenenergie $h\nu_{\text{Gd, M5}} = \SI{1184,79}{\eV}$ entspricht der Wellenlänge $\lambda_{\text{Gd, M5}} = \SI{1,05}{\nano\meter}$. So ist der erwartete Radius für die $h\nu_{\text{Gd, M5}}$ beträgt
% \begin{equation}
%     r_{\text{Gd, M5}} = \frac{kl\lambda_{\text{Gd, M5}}}{d} = \frac{1\cdot\SI{620}{\milli\meter}\cdot\SI{1,05}{\nano\meter}}{\SI{300}{\nano\meter}} = \SI{2.17}{\milli\meter}
% \end{equation}
% Mit der bekannten Auflösung von \gls{moench03}, die \SI{400}{\px}x\SI{400}{\px} ist, und der bekannten Sensorgröße \SI{10}{\milli\meter}x\SI{10}{\milli\meter} kann dieser Wert kann in die Pixel umgerechnet werden
% \begin{equation}
%         r_{\text{Gd, M5}} = \SI{2.17}{\milli\meter} = \frac{\SI{400}{\px}}{\SI{10}{\milli\meter}}\SI{2.1}{\milli\meter}=\SI{87(1)}{\px}
% \end{equation}
% \noindent
% Das entspricht dem 1. Maximum von $\sinc(r)^2$ an der Stelle $r=\SI{4.49}{}$. So ist die Proporitonalität zwischen der ursprünglichen $\sinc(r)^2$ und den tatsächlichen gefunden und beträgt 
% \begin{equation}
%     c_0 = \frac{\SI{87}{\px}}{\num{4.49}} = \SI{19.37}{\px}
% \end{equation}

% \noindent
% So entsprechen den Werten $r=\pi;2\pi$ vom Beugungsminima die Werte
% \begin{equation}
%     r_\text{0. px} = \pi\cdot \SI{19.37}{\px} = \SI{61}{\px}
% \end{equation}
% \begin{equation}
%     r_\text{0. px} = 2\pi\cdot \SI{19.37}{\px} = \SI{122}{\px}
% \end{equation}
% Die Flächen werden dann analog dazu umgerechnet

% \begin{equation}
%     A_{0.} = \pi\cdot (\SI{61(1)}{\px})^2 = \SI{1.2e4}{\px\squared}
% \end{equation}
% \begin{equation}
%     A_{1.} = 3\pi\cdot (\SI{122(1)}{px})^2 = \SI{1.4e5}{\px\squared}
% \end{equation}
% Wir haben tatsächlich \SI{30}{\photons} in $A_{0.}$ pro ein Schuss. So ist die
% \begin{equation}
%     \rho_0 = \frac{\SI{30}{\photons}}{\SI{1.2e4}{\px\squared}} = \SI{2.5e-3}{\photons\per\px\squared}
% \end{equation}
% Für die 1. Ordnung ist das eben dann 10 mal kleiner
% \begin{equation}
%     \rho_1 = \frac{\rho_0}{10} =  \SI{2.5e-4}{\photons\per\px\squared}
% \end{equation}
% Also wollen wir die dichte \SI{1}{\photon\per\px\squared} in dem Ring haben, dann muss man
% \begin{equation}
%     N = \frac{1}{\SI{2.5e-4}{\photons\per\px\squared pro Schuss}} = \SI{4000}{Sch"usse}
% \end{equation}
% machen.
%\cite[„Supplementary information“]{pfau_ultrafast_2012}