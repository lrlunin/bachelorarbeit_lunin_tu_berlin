\begin{titlepage}	
\noindent
\begin{minipage}{0.5\textwidth}
{\small
Technische Universit\"at Berlin\\
Fakult\"at II - Mathematik und Naturwissenschaften\\
Institut für Optik und Atomare Physik\\
}%
\end{minipage}%
\hfill
\begin{minipage}{0.18\textwidth}
\begin{figure}[H]%
\includegraphics[width=\textwidth]{images/logos/mbi/MBILogo_german_short_blue.png}
\end{figure}
\end{minipage}%
\hspace{5mm}
\begin{minipage}{0.18\textwidth}
\begin{figure}[H]%
\includegraphics[width=\textwidth]{images/logos/tu/TU_Logo_kurz_RGB_rot.png}
\end{figure}
\end{minipage}%

\vspace{2cm}

\thispagestyle{empty}
\begin{center}

{\large Bachelorarbeit zur Erlangung des akademischen Grades
Bachelor of Science
im Studiengang Physik}\\
\vspace{1cm}
{\huge\textbf
Pulsgetriggerte Detektion resonanter magnetischer Kleinwinkelstreuung an einer Laser-getriebenen Röntgenquelle}
\vspace{1.2cm}

{\large vorgelegt von\\
	 Leonid Lunin \\
	 geboren am 16.01.1999\\
	 Matrikelnummer: 402920\\[2cm]
	 Berlin, 30. Juli 2022\\[2cm]
}

\begin{minipage}{\linewidth} 
\begin{tabbing}
  		Erstgutachter:\quad \= Prof. Dr. Stefan Eisebitt\\[0.3cm]
  		Zweitgutachter: \> Prof. Dr. Michael Lehmann\\[0.3cm]
    	Betreuer:             \> Dr. Bastian Pfau \\
    						         
\end{tabbing}
\end{minipage}

\vspace{1.5cm}

\end{center}
\thispagestyle{empty}
\newpage
\setcounter{page}{1}
\end{titlepage}

%Clear two pages after the title
\shipout\null
